% qrcode.tex
%%%%%%%%%%%%%%%%%%%%%%%%%%%%%%%%%%%%%%%%%%%%%%%%%%%%%%%%
% Petr Olsak <petr@olsak.net>                  Jul. 2015

% This macro qrcode.tex is (roughly speaking) a copy of qrcode.sty
% macro by Anders Hendrickson <ahendric@cord.edu>, see 
% http://www.ctan.org/tex-archive/macros/latex/contrib/qrcode

% The main difference between qrcode.sty and qrcode.tex is, that
% the LaTeX ballast was removed from qrcode.sty by Petr Olsak. The result:
% The qrcode.tex macro can be used in plain TeX format too.

% Usage: after \input qrcode
% you can type \qrcode{encoded text}.

% More information about options can be found at the end of this file.

\edef\tmp{\catcode`@=\the\catcode`@\relax}\catcode`\@=11 % LaTeX special character :(

\newcount\qr@i
\newcount\qr@j
\newcount\qr@a
\newcount\qr@b
\newcount\qr@c

\def\theqr@i{\the\qr@i}
\def\theqr@j{\the\qr@j}

\def\@relax{\relax}%

\def\preface@macro#1#2{%
  % #1 = macro name
  % #2 = text to add to front of macro
  \def\tempb{#2}%
  \xa\xa\xa\def\xa\xa\xa#1\xa\xa\xa{\xa\tempb #1}%
}%

\def\g@preface@macro#1#2{%
  % #1 = macro to be appended to
  % #2 = code to add
  \edef\codeA{#2}%
  \expandafter\expandafter\expandafter
  \gdef\expandafter\expandafter\expandafter#1\expandafter\expandafter\expandafter
  {\expandafter\codeA#1}%
}

\def\qr@getstringlength#1{%
  \bgroup
    \qr@a=0%
    \xdef\thestring{#1}%
    \expandafter\qr@stringlength@recursive\expandafter(\thestring\relax\relax)%
    \xdef\qr@stringlength{\the\qr@a}%
  \egroup
}%

\def\qr@stringlength@recursive(#1#2){%
  \def\testi{#1}%
  \ifx\testi\@relax
    %we are done.
    \let\qr@next=\relax%
  \else
    \advance\qr@a by 1%
    \def\qr@next{\qr@stringlength@recursive(#2)}%
  \fi
  \qr@next
}%

\def\qr@for#1=#2to#3by#4#{\forA{#1}{#2}{#3}{#4}}
\long\def\forA#1#2#3#4#5{\begingroup
   {\escapechar=`\\ % allocation of #1 as counter:
      \expandafter \ifx\csname for:\string#1\endcsname \relax
         \csname newcount\expandafter\endcsname \csname for:\string#1\endcsname\fi
    \expandafter}\expandafter\let\expandafter#1\csname for:\string#1\endcsname
   #1=#2%
   \def\forB{#5\advance#1by#4\relax \expandafter\forC}%
   \ifnum#4>0 \def\forC{\ifnum#1>#3\relax\else\forB\fi}%
   \else      \def\forC{\ifnum#1<#3\relax\else\forB\fi}%
   \fi
   \ifnum#4=0 \let\forC=\relax \fi
   \forC \endgroup
}

\def\qr@padatfront#1#2{%
  % #1 = macro containing text to pad
  % #2 = desired number of characters
  % Pads a number with initial zeros.
  \qr@getstringlength{#1}%
  \qr@a=\qr@stringlength\relax%
  \advance\qr@a by 1\relax%
  \qr@for \i = \qr@a to #2 by 1
    {\g@preface@macro{#1}{0}}%
}

\qr@a=-1\relax%
\def\qr@savehexsymbols(#1#2){%
  \advance\qr@a by 1\relax%
  \expandafter\def\csname qr@hexchar@\the\qr@a\endcsname{#1}%
  \expandafter\edef\csname qr@hextodecimal@#1\endcsname{\the\qr@a}%
  \ifnum\qr@a=15\relax
    %Done.
    \let\qr@next=\relax%
  \else
    \def\qr@next{\qr@savehexsymbols(#2)}%
  \fi%
  \qr@next%
}%
\qr@savehexsymbols(0123456789abcdef\relax\relax)%

\def\qr@decimaltobase#1#2#3{%
  % #1 = macro to store result
  % #2 = decimal representation of a positive integer
  % #3 = new base
  \bgroup
    \edef\qr@newbase{#3}%
    \gdef\qr@base@result{}%
    \qr@a=#2\relax%
    \qr@decimaltobase@recursive%
    \xdef#1{\qr@base@result}%
  \egroup
}
\def\qr@decimaltobase@recursive{%
  \qr@b=\qr@a%
  \divide\qr@b by \qr@newbase\relax
  \multiply\qr@b by -\qr@newbase\relax
  \advance\qr@b by \qr@a\relax%
  \divide\qr@a by \qr@newbase\relax%
  \ifnum\qr@b<10\relax
    \edef\newdigit{\the\qr@b}%
  \else
    \edef\newdigit{\csname qr@hexchar@\the\qr@b\endcsname}%
  \fi
  \edef\qr@argument{{\noexpand\qr@base@result}{\newdigit}}%
  \expandafter\g@preface@macro\qr@argument%
  \ifnum\qr@a=0\relax
    \relax
  \else
    \expandafter\qr@decimaltobase@recursive
  \fi
}

\long\def\isnextchar#1#2#3{\begingroup\toks0={\endgroup#2}\toks1={\endgroup#3}%
   \let\tmp=#1\futurelet\next\isnextcharA
}
\def\isnextcharA{\the\toks\ifx\tmp\next0\else1\fi\space}

\long\def\xaddto#1#2{\expandafter\xdef\expandafter#1\expandafter{#1#2}}
\let\g@addto@macro=\xaddto

\def\qr@decimaltohex[#1]#2#3{%
  % #1 (opt.) = number of hex digits to create
  % #2 = macro to store result
  % #3 = decimal digits to convert
  \qr@decimaltobase{#2}{#3}{16}%
  \qr@padatfront{#2}{#1}%
}

\def\qr@decimaltobinary[#1]#2#3{%
  % #1 (opt.) = number of bits to create
  % #2 = macro to store result
  % #3 = decimal digits to convert
  \qr@decimaltobase{#2}{#3}{2}%
  \qr@padatfront{#2}{#1}%
}

\qr@for \i = 0 to 15 by 1%
  {%
   \qr@decimaltohex[1]{\qr@hexchar}{\the\i}%
   \qr@decimaltobinary[4]{\qr@bits}{\the\i}%
   \expandafter\xdef\csname qr@b2h@\qr@bits\endcsname{\qr@hexchar}%
   \expandafter\xdef\csname qr@h2b@\qr@hexchar\endcsname{\qr@bits}%
  }%

\def\qr@binarytohex[#1]#2#3{%
  % #1 (optional) = # digits desired
  % #2 = macro to save to
  % #3 = binary string (must be multiple of 4 bits)
  \def\test@i{#1}%
  \ifx\test@i\@relax%
    %No argument specified
    \def\qr@desireddigits{0}%
  \else
    \def\qr@desireddigits{#1}%
  \fi
  \gdef\qr@base@result{}%
  \edef\qr@argument{(#3\relax\relax\relax\relax\relax)}%
  \xa\qr@binarytohex@int\qr@argument%
  \qr@padatfront{\qr@base@result}{\qr@desireddigits}%
  \xdef#2{\qr@base@result}%
}
\def\qr@binarytohex@int(#1#2#3#4#5){%
  % #1#2#3#4 = 4 bits
  % #5 = remainder, including \relax\relax\relax\relax\relax terminator
  \def\test@i{#1}%
  \ifx\test@i\@relax%
    %Done.
    \def\qr@next{\relax}%
  \else%
    \xdef\qr@base@result{\qr@base@result\csname qr@b2h@#1#2#3#4\endcsname}%
    \def\qr@next{\qr@binarytohex@int(#5)}%
  \fi%
  \qr@next%
}

\def\qr@hextobinary[#1]#2#3{%
  % #1 (optional) = # bits desired
  % #2 = macro to save to
  % #3 = hexadecimal string
  \bgroup
  \def\test@i{#1}%
  \ifx\test@i\@relax%
    %No argument specified
    \def\qr@desireddigits{0}%
  \else
    \def\qr@desireddigits{#1}%
  \fi
  \gdef\qr@base@result{}%
  \edef\qr@argument{(#3\relax\relax)}%
  \xa\qr@hextobinary@int\qr@argument%
  \qr@padatfront{\qr@base@result}{\qr@desireddigits}%
  \xdef#2{\qr@base@result}%
  \egroup
}
\def\qr@hextobinary@int(#1#2){%
  % #1 = hexadecimal character
  % #2 = remainder, including \relax\relax terminator
  \def\test@@i{#1}%
  \ifx\test@@i\@relax%
    %Done.
    \def\qr@next{\relax}%
  \else%
    \xdef\qr@base@result{\qr@base@result\csname qr@h2b@#1\endcsname}%
    \def\qr@next{\qr@hextobinary@int(#2)}%
  \fi%
  \qr@next%
}

\def\qr@hextodecimal#1#2{%
  \edef\qr@argument{#2}%
  \xa\qr@a\xa=\xa\number\xa"\qr@argument\relax%
  \edef#1{\the\qr@a}%
}

\def\qr@hextodecimal#1#2{%
  % #1 = macro to store result
  % #2 = hexadecimal representation of a positive integer
  \bgroup
    \qr@a=0\relax%
    \edef\qr@argument{(#2\relax)}%
    \xa\qr@hextodecimal@recursive\qr@argument%
    \xdef#1{\the\qr@a}%
  \egroup
}
\def\qr@hextodecimal@recursive(#1#2){%
  % #1 = first hex char
  % #2 = remainder
  \advance \qr@a by \csname qr@hextodecimal@#1\endcsname\relax%
  \edef\testii{#2}%
  \ifx\testii\@relax%
    %Done.
    \let\qr@next=\relax%
  \else
    %There's at least one more digit.
    \multiply\qr@a by 16\relax
    \edef\qr@next{\noexpand\qr@hextodecimal@recursive(#2)}%
  \fi%
  \qr@next%
}

\def\qrverbatim{\def\do##1{\catcode`##1=12}\dospecials
  \catcode`\\=0 \catcode`\{=1 \catcode`\}=2
  \escapechar=-1 \def\do##1{\edef##1{\string##1}}\dospecials
  \def\?{^^J}\let\ =\qr@letterspace
  \catcode`\^^M=13 \qr@setMtoJ
  \ifx\mubytein\undefined \else \mubytein=0 \fi
}
{\lccode`\?=`\ \lowercase{\gdef\qr@letterspace{?}}}
{\catcode`\^^M=13 \gdef\qr@setMtoJ{\def^^M{^^J}}}

\def\qr@creatematrix#1{%
  \expandafter\gdef\csname #1\endcsname##1##2{%
    \csname #1@##1@##2\endcsname
  }%
}%

\def\qr@storetomatrix#1#2#3#4{%
  % #1 = matrix name
  % #2 = row number
  % #3 = column number
  % #4 = value of matrix entry
  \xa\gdef\csname #1@#2@#3\endcsname{#4}%
}%

\def\qr@estoretomatrix#1#2#3#4{%
  % This version performs exactly one expansion on #4.
  % #1 = matrix name
  % #2 = row number
  % #3 = column number
  % #4 = value of matrix
  \expandafter\gdef\csname #1@#2@#3\expandafter\endcsname\expandafter{#4}%
}%

\def\qr@matrixentry#1#2#3{%
  % #1 = matrix name
  % #2 = row number
  % #3 = column number
  \csname #1@#2@#3\endcsname%
}%

\def\qr@createsquareblankmatrix#1#2{%
  \qr@creatematrix{#1}%
  \xa\gdef\csname #1@numrows\endcsname{#2}%
  \xa\gdef\csname #1@numcols\endcsname{#2}%
  \qr@for \i = 1 to #2 by 1%
    {\qr@for \j = 1 to #2 by 1%
      {\qr@storetomatrix{#1}{\the\i}{\the\j}{\@blank}}}%
}%

\def\qr@numberofrowsinmatrix#1{%
  \csname #1@numrows\endcsname%
}%

\def\qr@numberofcolsinmatrix#1{%
  \csname #1@numcols\endcsname%
}%

\def\qr@setnumberofrows#1#2{%
  \xa\xdef\csname #1@numrows\endcsname{#2}%
}%

\def\qr@setnumberofcols#1#2{%
  \xa\xdef\csname #1@numcols\endcsname{#2}%
}%

\newdimen\qrdesiredheight
\newdimen\qrmodulesize

\def\qr@link#1#2{\hbox{\pdfstartlink height\ht0 depth0pt \qr@border
   user{/Subtype/Link/A <</Type/Action/S/URI/URI(#1)>>}\relax #2\pdfendlink}%
}
\def\qr@border{\expandafter\ifx \csname kv:qrborder\endcsname\relax \else
   attr{/C[\kv{qrborder}] /Border[0 0 .6]}\fi
}

\def\qr@createliteralmatrix#1#2#3{%
  % #1 = matrix name
  % #2 = m, the number of rows and columns in the square matrix
  % #3 = a string of m^2 tokens to be written into the matrix
  \qr@creatematrix{#1}%
  \expandafter\xdef\csname #1@numrows\endcsname{#2}%
  \expandafter\xdef\csname #1@numcols\endcsname{#2}%
  \gdef\qr@literalmatrix@tokens{#3}%
  \qr@for \i = 1 to #2 by 1%
    {\qr@for \j = 1 to #2 by 1%
      {\expandafter\qr@createliteralmatrix@int\expandafter(\qr@literalmatrix@tokens)%
       \qr@estoretomatrix{#1}{\the\i}{\the\j}{\qr@entrytext}%
      }%
    }%
}
\def\qr@createliteralmatrix@int(#1#2){%
  \def\qr@entrytext{#1}%
  \gdef\qr@literalmatrix@tokens{#2}%
}

\qr@createliteralmatrix{finderpattern}{8}{%
  \qr@black@fixed\qr@black@fixed\qr@black@fixed\qr@black@fixed\qr@black@fixed\qr@black@fixed\qr@black@fixed\qr@white@fixed%
  \qr@black@fixed\qr@white@fixed\qr@white@fixed\qr@white@fixed\qr@white@fixed\qr@white@fixed\qr@black@fixed\qr@white@fixed%
  \qr@black@fixed\qr@white@fixed\qr@black@fixed\qr@black@fixed\qr@black@fixed\qr@white@fixed\qr@black@fixed\qr@white@fixed%
  \qr@black@fixed\qr@white@fixed\qr@black@fixed\qr@black@fixed\qr@black@fixed\qr@white@fixed\qr@black@fixed\qr@white@fixed%
  \qr@black@fixed\qr@white@fixed\qr@black@fixed\qr@black@fixed\qr@black@fixed\qr@white@fixed\qr@black@fixed\qr@white@fixed%
  \qr@black@fixed\qr@white@fixed\qr@white@fixed\qr@white@fixed\qr@white@fixed\qr@white@fixed\qr@black@fixed\qr@white@fixed%
  \qr@black@fixed\qr@black@fixed\qr@black@fixed\qr@black@fixed\qr@black@fixed\qr@black@fixed\qr@black@fixed\qr@white@fixed%
  \qr@white@fixed\qr@white@fixed\qr@white@fixed\qr@white@fixed\qr@white@fixed\qr@white@fixed\qr@white@fixed\qr@white@fixed%
}%

\qr@createliteralmatrix{alignmentpattern}{5}{%
  \qr@black@fixed\qr@black@fixed\qr@black@fixed\qr@black@fixed\qr@black@fixed%
  \qr@black@fixed\qr@white@fixed\qr@white@fixed\qr@white@fixed\qr@black@fixed%
  \qr@black@fixed\qr@white@fixed\qr@black@fixed\qr@white@fixed\qr@black@fixed%
  \qr@black@fixed\qr@white@fixed\qr@white@fixed\qr@white@fixed\qr@black@fixed%
  \qr@black@fixed\qr@black@fixed\qr@black@fixed\qr@black@fixed\qr@black@fixed%
}%

\def\qr@copymatrixentry#1#2#3#4#5#6{%
  % Copy the (#2,#3) entry of matrix #1
  % to the (#5,#6) position of matrix #4.
  \xa\xa\xa\global%
  \xa\xa\xa\let\xa\xa\csname #4@#5@#6\endcsname%
                     \csname #1@#2@#3\endcsname%
}%

\def\qr@createduplicatematrix#1#2{%
  % #1 = name of copy
  % #2 = original matrix to be copied
  \qr@creatematrix{#1}%
  \qr@for \i = 1 to \qr@numberofrowsinmatrix{#2} by 1%
    {\qr@for \j = 1 to \qr@numberofcolsinmatrix{#2} by 1%
      {\qr@copymatrixentry{#2}{\the\i}{\the\j}{#1}{\the\i}{\the\j}%
      }%
    }%
  \qr@setnumberofrows{#1}{\qr@numberofrowsinmatrix{#2}}%
  \qr@setnumberofcols{#1}{\qr@numberofcolsinmatrix{#2}}%
}%

\def\qr@placefinderpattern@int#1#2#3#4#5{%
  % Work on matrix #1.
  % Start in position (#2, #3) -- should be a corner
  % #4 indicates horizontal direction (1=right, -1=left)
  % #5 indicates vertical direction (1=down, -1=up)
  %
  % In this code, \sourcei and \sourcej are TeX counts working through the finderpattern matrix,
  % and i and j are LaTeX counters indicating positions in the big matrix.
  \setcounter{qr@i}{#2}%
  \qr@for \sourcei=1 to 8 by 1%
    {\setcounter{qr@j}{#3}%
     \qr@for \sourcej=1 to 8 by 1%
       {\qr@copymatrixentry{finderpattern}{\the\sourcei}{\the\sourcej}%
                        {#1}{\theqr@i}{\theqr@j}%
        \addtocounter{qr@j}{#5}%
       }%
     \addtocounter{qr@i}{#4}%
    }%
}%

\def\qr@placefinderpatterns#1{%
  % #1=matrix name
  \qr@placefinderpattern@int{#1}{1}{1}{1}{1}%
  \qr@placefinderpattern@int{#1}{\qr@numberofrowsinmatrix{#1}}{1}{-1}{1}%
  \qr@placefinderpattern@int{#1}{1}{\qr@numberofcolsinmatrix{#1}}{1}{-1}%
}%

\def\qr@placetimingpatterns#1{%
  %Set \endingcol to n-8.
  \qr@a=\qr@size\relax%
  \advance\qr@a by -8\relax%
  \edef\endingcol{\the\qr@a}%
  \qr@for \j = 9 to \endingcol by 1%
    {\ifodd\j\relax%
       \qr@storetomatrix{#1}{7}{\the\j}{\qr@black@fixed}%
       \qr@storetomatrix{#1}{\the\j}{7}{\qr@black@fixed}%
     \else%
       \qr@storetomatrix{#1}{7}{\the\j}{\qr@white@fixed}%
       \qr@storetomatrix{#1}{\the\j}{7}{\qr@white@fixed}%
     \fi%
    }%
}%

\def\qr@placealignmentpattern@int#1#2#3{%
  % Work on matrix #1.
  % Write an alignment pattern into the matrix, centered on (#2,#3).
  \qr@a=#2\relax%
  \advance\qr@a by -2\relax%
  \qr@b=#3\relax%
  \advance\qr@b by -2\relax%
  \setcounter{qr@i}{\the\qr@a}%
  \qr@for \i=1 to 5 by 1%
    {\setcounter{qr@j}{\the\qr@b}%
     \qr@for \j=1 to 5 by 1%
      {\qr@copymatrixentry{alignmentpattern}{\the\i}{\the\j}%
                       {#1}{\theqr@i}{\theqr@j}%
       \stepcounter{qr@j}%
      }%
     \stepcounter{qr@i}%
    }%
}%

\newif\ifqr@incorner%
\def\qr@placealignmentpatterns#1{%
  %There are k^2-3 alignment patterns,
  %arranged in a (k x k) grid within the matrix.
  %They begin in row 7, column 7,
  %except that the ones in the NW, NE, and SW corners
  %are omitted because of the finder patterns.
  %Recall that
  %  * \qr@k stores k,
  %  * \qr@alignment@firstskip stores how far between the 1st and 2nd row/col, &
  %  * \qr@alignment@generalskip stores how far between each subsequent row/col.
  \xa\ifnum\qr@k>0\relax
    %There will be at least one alignment pattern.
    %N.B. k cannot equal 1.
    \xa\ifnum\qr@k=2\relax
      % 2*2-3 = exactly 1 alignment pattern.
      \qr@a=7\relax
      \advance\qr@a by \qr@alignment@firstskip\relax
      \xdef\qr@target@ii{\the\qr@a}%
      \qr@placealignmentpattern@int{#1}{\qr@target@ii}{\qr@target@ii}%
    \else
      % k is at least 3, so the following loops should be safe.
      \xdef\qr@target@ii{7}%
      \qr@for \ii = 1 to \qr@k by 1%
        {\ifcase\ii\relax%
           \relax% \ii should never equal 0.
         \or
           \xdef\qr@target@ii{7}% If \ii = 1, we start in row 7.
         \or
           %If \ii = 2, we add the firstskip.
           \qr@a=\qr@target@ii\relax%
           \advance\qr@a by \qr@alignment@firstskip\relax%
           \xdef\qr@target@ii{\the\qr@a}%
         \else
           %If \ii>2, we add the generalskip.
           \qr@a=\qr@target@ii\relax%
           \advance\qr@a by \qr@alignment@generalskip\relax%
           \xdef\qr@target@ii{\the\qr@a}%
         \fi
         \qr@for \jj = 1 to \qr@k by 1%
           {\ifcase\jj\relax%
              \relax% \jj should never equal 0.
            \or
              \xdef\qr@target@jj{7}% If \jj=1, we start in row 7.
            \or
              %If \jj=2, we add the firstskip.
              \qr@a=\qr@target@jj\relax%
              \advance\qr@a by \qr@alignment@firstskip%
              \xdef\qr@target@jj{\the\qr@a}%
            \else
              %If \jj>2, we add the generalskip.
              \qr@a=\qr@target@jj\relax%
              \advance\qr@a by \qr@alignment@generalskip%
              \xdef\qr@target@jj{\the\qr@a}%
            \fi
            \qr@incornerfalse%
            \ifnum\ii=1\relax
              \ifnum\jj=1\relax
                \qr@incornertrue
              \else
                \ifnum\qr@k=\jj\relax
                  \qr@incornertrue
                \fi
              \fi
            \else
              \xa\ifnum\qr@k=\ii\relax
                \ifnum\jj=1\relax
                  \qr@incornertrue
                \fi
              \fi
            \fi
            \ifqr@incorner
              \relax
            \else
              \qr@placealignmentpattern@int{#1}{\qr@target@ii}{\qr@target@jj}%
            \fi
           }% ends \qr@for \jj
        }% ends \qr@for \ii
    \fi
  \fi
}%

\def\qr@placedummyformatpatterns#1{%
  \qr@for \j = 1 to 9 by 1%
    {\ifnum\j=7\relax%
     \else%
       \qr@storetomatrix{#1}{9}{\the\j}{\qr@format@square}%
       \qr@storetomatrix{#1}{\the\j}{9}{\qr@format@square}%
     \fi%
    }%
  \setcounter{qr@j}{\qr@size}%
  \qr@for \j = 1 to 8 by 1%
    {\qr@storetomatrix{#1}{9}{\theqr@j}{\qr@format@square}%
     \qr@storetomatrix{#1}{\theqr@j}{9}{\qr@format@square}%
     \addtocounter{qr@j}{-1}%
    }%
  %Now go back and change the \qr@format@square in (n-8,9) to \qr@black@fixed.
  \addtocounter{qr@j}{1}%
  \qr@storetomatrix{#1}{\theqr@j}{9}{\qr@black@fixed}%
}%

\def\qr@placedummyversionpatterns#1{%
  \xa\ifnum\qr@version>6\relax
    %Must include version information.
    \global\qr@i=\qr@size%
    \global\advance\qr@i by -10\relax%
    \qr@for \i = 1 to 3 by 1%
      {\qr@for \j = 1 to 6 by 1%
        {\qr@storetomatrix{#1}{\theqr@i}{\the\j}{\qr@format@square}%
         \qr@storetomatrix{#1}{\the\j}{\theqr@i}{\qr@format@square}%
        }%
       \stepcounter{qr@i}%
      }%
  \fi
}%

\def\qr@writebit(#1#2)#3{%
  % #3 = matrix name
  % (qr@i,qr@j) = position to write in (LaTeX counters)
  % #1 = bit to be written
  % #2 = remaining bits plus '\relax' as an end-of-file marker
  \edef\qr@datatowrite{#2}%
  \ifnum#1=1
    \qr@storetomatrix{#3}{\theqr@i}{\theqr@j}{\qr@black}%
  \else
    \qr@storetomatrix{#3}{\theqr@i}{\theqr@j}{\@white}%
  \fi
}%

\newif\ifqr@rightcol
\newif\ifqr@goingup

\def\qr@writedata@hex#1#2{%
  % #1 = name of a matrix that has been prepared with finder patterns, timing patterns, etc.
  % #2 = a string consisting of bytes to write into the matrix, in two-char hex format.
  \setcounter{qr@i}{\qr@numberofrowsinmatrix{#1}}%
  \setcounter{qr@j}{\qr@numberofcolsinmatrix{#1}}%
  \qr@rightcoltrue%
  \qr@goinguptrue%
  \edef\qr@argument{{#1}(#2\relax\relax\relax)}%
  \xa\qr@writedata@hex@recursive\qr@argument%
}%

\def\qr@writedata@hex@recursive#1(#2#3#4){%
  % #1 = name of a matrix that has been prepared with finder patterns, timing patterns, etc.
  % (qr@i,qr@j) = position to write in LaTeX counters
  % #2#3#4 contains the hex codes of the bytes to be written, plus \relax\relax\relax
  % as an end-of-file marker
  \edef\testii{#2}%
  \ifx\testii\@relax%
    % #2 is \relax, so there is nothing more to write.
    \relax
    \let\go=\relax
  \else
    % #2 is not \relax, so there is another byte to write.
    \qr@hextobinary[8]{\bytetowrite}{#2#3}%
    \xdef\qr@datatowrite{\bytetowrite\relax}% %Add terminating "\relax"
    \qr@writedata@recursive{#1}% %This function actually writes the 8 bits.
    \edef\qr@argument{{#1}(#4)}%
    \xa\def\xa\go\xa{\xa\qr@writedata@hex@recursive\qr@argument}% %Call self to write the next bit.
  \fi
  \go
}%

\def\qr@writedata#1#2{%
  % #1 = name of a matrix that has been prepared with finder patterns, timing patterns, etc.
  % #2 = a string consisting of 0's and 1's to write into the matrix.
  \setcounter{qr@i}{\qr@numberofrowsinmatrix{#1}}%
  \setcounter{qr@j}{\qr@numberofcolsinmatrix{#1}}%
  \qr@rightcoltrue
  \qr@goinguptrue
  \edef\qr@datatowrite{#2\relax}%
  \qr@writedata@recursive{#1}%
}%

\def\@@blank{\@blank}%

\def\qr@writedata@recursive#1{%
  % #1 = matrix name
  % (qr@i,qr@j) = position to write in (LaTeX counters)
  % \qr@datatowrite contains the bits to be written, plus '\relax' as an end-of-file marker
  \xa\let\xa\squarevalue\csname #1@\theqr@i @\theqr@j\endcsname%
  \ifx\squarevalue\@@blank
    %Square is blank, so write data in it.
    \xa\qr@writebit\xa(\qr@datatowrite){#1}%
    %The \qr@writebit macro not only writes the first bit of \qr@datatowrite into the matrix,
    %but also removes the bit from the 'bitstream' of \qr@datatowrite.
  \fi
  %Now adjust our position in the matrix.
  \ifqr@rightcol
    %From the right-hand half of the two-bit column, we always move left.  Easy peasy.
    \addtocounter{qr@j}{-1}%
    \qr@rightcolfalse
  \else
    %If we're in the left-hand column, things are harder.
    \ifqr@goingup
      %First, suppose we're going upwards.
      \ifnum\qr@i>1\relax%
        %If we're not in the first row, things are easy.
        %We move one to the right and one up.
        \addtocounter{qr@j}{1}%
        \addtocounter{qr@i}{-1}%
        \qr@rightcoltrue
      \else
        %If we are in the first row, then we move to the left,
        %and we are now in the right-hand column on a downward pass.
        \addtocounter{qr@j}{-1}%
        \qr@goingupfalse
        \qr@rightcoltrue
      \fi
    \else
      %Now, suppose we're going downwards.
      \xa\ifnum\qr@size>\qr@i\relax%
        %If we're not yet in the bottom row, things are easy.
        %We move one to the right and one down.
        \addtocounter{qr@j}{1}%
        \addtocounter{qr@i}{1}%
        \qr@rightcoltrue
      \else
        %If we are in the bottom row, then we move to the left,
        %and we are now in the right-hand column on an upward pass.
        \addtocounter{qr@j}{-1}%
        \qr@rightcoltrue
        \qr@goinguptrue
      \fi
    \fi
    %One problem: what if we just moved into the 7th column?
    %Das ist verboten.
    %If we just moved (left) into the 7th column, we should move on into the 6th column.
    \ifnum\qr@j=7\relax%
      \setcounter{qr@j}{6}%
    \fi
  \fi
  %Now check whether there are any more bits to write.
  \ifx\qr@datatowrite\@relax
    % \qr@datatowrite is just `\relax', so we're done.
    \let\nexttoken=\relax
    \relax
  \else
    % Write some more!
    \def\nexttoken{\qr@writedata@recursive{#1}}%
  \fi
  \nexttoken
}%

\def\qr@writeremainderbits#1{%
  % #1 = name of a matrix that has been prepared and partly filled.
  % (qr@i,qr@j) = position to write in LaTeX counters
  \xa\ifnum\qr@numremainderbits>0\relax
    \def\qr@datatowrite{}%
    \qr@for \i = 1 to \qr@numremainderbits by 1%
      {\g@addto@macro{\qr@datatowrite}{0}}%
    \g@addto@macro{\qr@datatowrite}{\relax}% terminator
    \qr@writedata@recursive{#1}%
  \fi
}%

\newif\ifqr@cellinmask

\def\qr@setmaskingfunction#1{%
  % #1 = 1 decimal digit for the mask. (I see no reason to use the 3-bit binary code.)
  % The current position is (\themaski,\themaskj), with indexing starting at 0.
  \edef\maskselection{#1}%
  \xa\ifcase\maskselection\relax
    %Case 0: checkerboard
    \def\qr@parsemaskingfunction{%
      % Compute mod(\themaski+\themaskj,2)%
      \qr@a=\maski%
      \advance\qr@a by \maskj%
      \qr@b=\qr@a%
      \divide\qr@b by 2%
      \multiply\qr@b by 2%
      \advance\qr@a by -\qr@b%
      \edef\qr@maskfunctionresult{\the\qr@a}%
    }%
  \or
    %Case 1: horizontal stripes
    \def\qr@parsemaskingfunction{%
      % Compute mod(\themaski,2)%
      \ifodd\maski\relax%
        \def\qr@maskfunctionresult{1}%
      \else%
        \def\qr@maskfunctionresult{0}%
      \fi%
    }%
  \or
    %Case 2: vertical stripes
    \def\qr@parsemaskingfunction{%
      % Compute mod(\themaskj,3)%
      \qr@a=\maskj%
      \divide\qr@a by 3%
      \multiply\qr@a by 3%
      \advance\qr@a by -\maskj%
      \edef\qr@maskfunctionresult{\the\qr@a}%
    }%
  \or
    %Case 3: diagonal stripes
    \def\qr@parsemaskingfunction{%
      % Compute mod(\themaski+\themaskj,3)%
      \qr@a=\maski%
      \advance\qr@a by \maskj%
      \qr@b=\qr@a%
      \divide\qr@b by 3%
      \multiply\qr@b by 3%
      \advance\qr@b by -\qr@a%
      \edef\qr@maskfunctionresult{\the\qr@b}%
    }%
  \or
    %Case 4: wide checkerboard
    \def\qr@parsemaskingfunction{%
      % Compute mod(floor(\themaski/2) + floor(\themaskj/3),2) %
      \qr@a=\maski%
      \divide\qr@a by 2%
      \qr@b=\maskj%
      \divide\qr@b by 3%
      \advance\qr@a by \qr@b%
      \qr@b=\qr@a%
      \divide\qr@a by 2%
      \multiply\qr@a by 2%
      \advance\qr@a by -\qr@b%
      \edef\qr@maskfunctionresult{\the\qr@a}%
    }%
  \or
    %Case 5: quilt
    \def\qr@parsemaskingfunction{%
      % Compute mod(\themaski*\themaskj,2) + mod(\themaski*\themaskj,3) %
      \qr@a=\maski%
      \multiply\qr@a by \maskj%
      \qr@b=\qr@a%
      \qr@c=\qr@a%
      \divide\qr@a by 2%
      \multiply\qr@a by 2%
      \advance\qr@a by -\qr@c% (result will be -mod(i*j,2), which is negative.)
      \divide\qr@b by 3%
      \multiply\qr@b by 3%
      \advance\qr@b by -\qr@c% (result will be -mod(i*j,3), which is negative.)
      \advance\qr@a by \qr@b% (result is negative of what's in the spec.)
      \edef\qr@maskfunctionresult{\the\qr@a}%
    }%
  \or
    %Case 6: arrows
    \def\qr@parsemaskingfunction{%
      % Compute mod( mod(\themaski*\themaskj,2) + mod(\themaski*\themaskj,3) , 2 ) %
      \qr@a=\maski%
      \multiply\qr@a by \maskj%
      \qr@b=\qr@a%
      \qr@c=\qr@a%
      \multiply\qr@c by 2% % \qr@c equals 2*i*j.
      \divide\qr@a by 2%
      \multiply\qr@a by 2%
      \advance\qr@c by -\qr@a% Now \qr@c equals i*j + mod(i*j,2).
      \divide\qr@b by 3%
      \multiply\qr@b by 3%
      \advance\qr@c by -\qr@b% (Now \qr@c equals mod(i*j,2) + mod(i*j,3).
      \qr@a=\qr@c%
      \divide\qr@a by 2%
      \multiply\qr@a by 2%
      \advance\qr@c by-\qr@a%
      \edef\qr@maskfunctionresult{\the\qr@c}%
    }%
  \or
    %Case 7: shotgun
    \def\qr@parsemaskingfunction{%
      % Compute mod( mod(\themaski+\themaskj,2) + mod(\themaski*\themaskj,3) , 2 ) %
      \qr@a=\maski%
      \advance\qr@a by \maskj% %So \qr@a = i+j
      \qr@b=\maski%
      \multiply\qr@b by \maskj% %So \qr@b = i*j
      \qr@c=\qr@a%
      \advance\qr@c by \qr@b% So \qr@c = i+j+i*j
      \divide\qr@a by 2%
      \multiply\qr@a by 2%
      \advance\qr@c by -\qr@a% So \qr@c = mod(i+j,2) + i*j
      \divide\qr@b by 3%
      \multiply\qr@b by 3%
      \advance\qr@c by -\qr@b% So \qr@c = mod(i+j,2) + mod(i*j,3)
      \qr@a=\qr@c%
      \divide\qr@c by 2%
      \multiply\qr@c by 2%
      \advance\qr@a by -\qr@c%
      \edef\qr@maskfunctionresult{\the\qr@a}%
    }%
  \fi
}%

\def\qr@checkifcellisinmask{%
  % The current position is (\i,\j), in TeX counts,
  % but the LaTeX counters (maski,maskj) should contain
  % the current position with indexing starting at 0.
  % That is, maski = \i-1 and maskj = \j-1.
  %
  % \qr@parsemaskingfunction must have been set by a call to \qr@setmaskingfunction
  \qr@parsemaskingfunction
  \xa\ifnum\qr@maskfunctionresult=0\relax
    \qr@cellinmasktrue
  \else
    \qr@cellinmaskfalse
  \fi
}%

\newcount\maski
\newcount\maskj

\def\qr@applymask#1#2#3{%
  % #1 = name of a matrix that should be filled out completely
  %      except for the format and/or version information.
  % #2 = name of a new matrix to contain the masked version
  % #3 = 1 decimal digit naming the mask
  \qr@createduplicatematrix{#2}{#1}%
  \qr@setmaskingfunction{#3}%
  \setcounter{maski}{-1}%
  \qr@for \i = 1 to \qr@size by 1%
    {\stepcounter{maski}%
     \setcounter{maskj}{-1}%
     \qr@for \j = 1 to \qr@size by 1%
     {\stepcounter{maskj}%
      \qr@checkifcellisinmask
      \ifqr@cellinmask
        \qr@checkifcurrentcellcontainsdata{#2}%
        \ifqr@currentcellcontainsdata
          \qr@flipcurrentcell{#2}%
        \fi
      \fi
      }%
    }%
}%

\newif\ifqr@currentcellcontainsdata
\qr@currentcellcontainsdatafalse

\def\@@white{\@white}%
\def\@@black{\qr@black}%

\def\qr@checkifcurrentcellcontainsdata#1{%
  % #1 = name of matrix
  \qr@currentcellcontainsdatafalse
  \xa\ifx\csname #1@\the\i @\the\j\endcsname\@@white
    \qr@currentcellcontainsdatatrue
  \fi
  \xa\ifx\csname #1@\the\i @\the\j\endcsname\@@black
    \qr@currentcellcontainsdatatrue
  \fi
}%

\def\qr@flipped@black{\qr@black}%
\def\qr@flipped@white{\@white}%

\def\qr@flipcurrentcell#1{%
  % #1 = name of matrix
  % (\i, \j) = current position, in TeX counts.
  % This assumes the cell contains data, either black or white!
  \xa\ifx\csname #1@\the\i @\the\j\endcsname\@@white
    \qr@storetomatrix{#1}{\the\i}{\the\j}{\qr@flipped@black}%
  \else
    \qr@storetomatrix{#1}{\the\i}{\the\j}{\qr@flipped@white}%
  \fi
}%

\def\qr@chooseandapplybestmask#1{%
  % #1 = name of a matrix that should be filled out completely
  %      except for the format and/or version information.
  % This function applies all eight masks in succession,
  % calculates their penalties, and remembers the best.
  % The number indicating which mask was used is saved in \qr@mask@selected.
  \qr@createduplicatematrix{originalmatrix}{#1}%
  \qrmessage{<Applying Mask 0...}%
  \qr@applymask{originalmatrix}{#1}{0}%
  \qrmessage{done. Calculating penalty...}%
  \qr@evaluatemaskpenalty{#1}%
  \xdef\currentbestpenalty{\qr@penalty}%
  \qrmessage{penalty is \qr@penalty>^^J}%
  \gdef\currentbestmask{0}%
  \qr@for \i = 1 to 7 by 1%
    {\qrmessage{<Applying Mask \the\i...}%
     \qr@applymask{originalmatrix}{currentmasked}{\the\i}%
     \qrmessage{done. Calculating penalty...}%
     \qr@evaluatemaskpenalty{currentmasked}%
     \qrmessage{penalty is \qr@penalty>^^J}%
     \xa\xa\xa\ifnum\xa\qr@penalty\xa<\currentbestpenalty\relax
       %We found a better mask.
       \xdef\currentbestmask{\the\i}%
       \qr@createduplicatematrix{#1}{currentmasked}%
       \xdef\currentbestpenalty{\qr@penalty}%
     \fi
    }%
  \xdef\qr@mask@selected{\currentbestmask}%
  \qrmessage{<Selected Mask \qr@mask@selected>^^J}%
}%

\def\qr@Ni{3}%
\def\qr@Nii{3}%
\def\qr@Niii{40}%
\def\qr@Niv{10}%
\def\@fiveones{11111}%
\def\@fivezeros{11111}%
\def\@twoones{11}%
\def\@twozeros{00}%
\def\@finderA{00001011101}%
\def\@finderB{10111010000}%
\def\@finderB@three{1011101000}%
\def\@finderB@two{101110100}%
\def\@finderB@one{10111010}%
\def\@finderB@zero{1011101}%
\newif\ifstringoffive
\def\addpenaltyiii{%
  \addtocounter{penaltyiii}{\qr@Niii}%
}%
\newcount\totalones
\newcount\penaltyi
\newcount\penaltyii
\newcount\penaltyiii
\newcount\penaltyiv
\def\qr@evaluatemaskpenalty#1{%
  % #1 = name of a matrix that we will test for the penalty
  % according to the specs.
  \setcounter{penaltyi}{0}%
  \setcounter{penaltyii}{0}%
  \setcounter{penaltyiii}{0}%
  \setcounter{penaltyiv}{0}%
  \bgroup%localize the meanings we give to the symbols
    \def\qr@black{1}\def\@white{0}%
    \def\qr@black@fixed{1}\def\qr@white@fixed{0}%
    \def\qr@format@square{0}% This is not stated in the specs, but seems
                            % to be the standard implementation.
    \def\@blank{0}% These would be any bits at the end.
    %
    \setcounter{totalones}{0}%
    \qr@for \i=1 to \qr@size by 1%
      {\def\lastfive{z}% %The z is a dummy, that will be removed before any testing.
       \stringoffivefalse
       \def\lasttwo@thisrow{z}% %The z is a dummy.
       \def\lasttwo@nextrow{z}% %The z is a dummy.
       \def\lastnine{z0000}% %The 0000 stands for the white space to the left. The z is a dummy.
       \def\ignore@finderB@at{0}%
       \qr@for \j=1 to \qr@size by 1%
         {\edef\newbit{\qr@matrixentry{#1}{\the\i}{\the\j}}%
          %
          % LASTFIVE CODE FOR PENALTY 1
          % First, add the new bit to the end.
          \xa\g@addto@macro\xa\lastfive\xa{\newbit}%
          \ifnum\j<5\relax%
            %Not yet on the 5th entry.
            %Don't do any testing.
          \else
            % 5th entry or later.
            % Remove the old one, and then test.
            \removefirsttoken\lastfive%
            \ifx\lastfive\@fiveones%
              \ifstringoffive%
                %This is a continuation of a previous block of five or more 1's.
                \stepcounter{penaltyi}%
              \else
                %This is a new string of five 1's.
                \addtocounter{penaltyi}{\qr@Ni}%
                \global\stringoffivetrue
              \fi
            \else
              \ifx\lastfive\@fivezeros%
                \ifstringoffive
                  %This is a continuation of a previous block of five or more 0's.
                  \stepcounter{penaltyi}%
                \else
                  %This is a new string of five 0's.
                  \addtocounter{penaltyi}{\qr@Ni}%
                  \global\stringoffivetrue
                \fi
              \else
                %This is not a string of five 1's or five 0's.
                \global\stringoffivefalse
              \fi
            \fi
          \fi
          %
          % 2x2 BLOCKS FOR PENALTY 2
          % Every 2x2 block of all 1's counts for \qr@Nii penalty points.
          % We do not need to run this test in the last row.
          \xa\ifnum\xa\i\xa<\qr@size\relax
            \xa\g@addto@macro\xa\lasttwo@thisrow\xa{\newbit}%
            %Compute \iplusone
            \qr@a=\i\relax%
            \advance\qr@a by 1%
            \edef\iplusone{\the\qr@a}%
            %
            \edef\nextrowbit{\qr@matrixentry{#1}{\iplusone}{\the\j}}%
            \xa\g@addto@macro\xa\lasttwo@nextrow\xa{\nextrowbit}%
            \ifnum\j<2\relax%
              %Still in the first column; no check.
            \else
              %Second column or later.  Remove the old bits, and then test.
              \removefirsttoken\lasttwo@thisrow
              \removefirsttoken\lasttwo@nextrow
              \ifx\lasttwo@thisrow\@twoones
                \ifx\lasttwo@nextrow\@twoones
                  \addtocounter{penaltyii}{\qr@Nii}%
                \fi
              \else
                \ifx\lasttwo@thisrow\@twozeros
                  \ifx\lasttwo@nextrow\@twozeros
                    \addtocounter{penaltyii}{\qr@Nii}%
                  \fi
                \fi
              \fi
            \fi
          \fi
          %
          % LASTNINE CODE FOR PENALTY 3
          % First, add the new bit to the end.
          \xa\g@addto@macro\xa\lastnine\xa{\newbit}%
          \ifnum\j<7\relax%
            %Not yet on the 7th entry.
            %Don't do any testing.
          \else
            % 7th entry or later.
            % Remove the old one, and then test.
            \removefirsttoken\lastnine
            \xa\ifnum\qr@size=\j\relax%
              % Last column.  Any of the following should count:
              %     1011101 (\@finderB@zero)
              %    10111010 (\@finderB@one)
              %   101110100 (\@finderB@two)
              %  1011101000 (\@finderB@three)
              % 10111010000 (\@finderB)
              \ifx\lastnine\@finderB
                \addpenaltyiii
              \else
                \removefirsttoken\lastnine
                \ifx\lastnine\@finderB@three
                  \addpenaltyiii
                \else
                  \removefirsttoken\lastnine
                  \ifx\lastnine\@finderB@two
                    \addpenaltyiii
                  \else
                    \removefirsttoken\lastnine
                    \ifx\lastnine\@finderB@one
                      \addpenaltyiii
                    \else
                      \removefirsttoken\lastnine
                      \ifx\lastnine\@finderB@zero
                        \addpenaltyiii
                      \fi
                    \fi
                  \fi
                \fi
              \fi
            \else
              \ifx\lastnine\@finderA% %Matches 0000 1011101
                \addpenaltyiii
                %Also, we record our discovery, so that we can't count this pattern again
                %if it shows up four columns later as 1011101 0000.
                %
                %Set \ignore@finderB@at to \j+4.
                \qr@a=\j\relax%
                \advance\qr@a by 4%
                \xdef\ignore@finderB@at{\the\qr@a}%
              \else
                \ifx\lastfive\@finderB% %Matches 1011101 0000.
                  \xa\ifnum\ignore@finderB@at=\j\relax
                    %This pattern was *not* counted already earlier.
                    \addpenaltyiii
                  \fi
                \fi
              \fi
            \fi
          \fi
          %
          %COUNT 1's FOR PENALTY 4
          \xa\ifnum\newbit=1\relax%
            \stepcounter{totalones}%
          \fi
         }% end of j-loop
      }% end of i-loop
    %
    %NOW WE ALSO NEED TO RUN DOWN THE COLUMNS TO FINISH CALCULATING PENALTIES 1 AND 3.
    \qr@for \j=1 to \qr@size by 1%
      {\def\lastfive{z}% %The z is a dummy, that will be removed before any testing.
       \stringoffivefalse
       \def\lastnine{z0000}% %The 0000 stands for the white space to the left. The z is a dummy.
       \def\ignore@finderB@at{0}%
       \qr@for \i=1 to \qr@size by 1%
         {\edef\newbit{\qr@matrixentry{#1}{\the\i}{\the\j}}%
          %
          % LASTFIVE CODE FOR PENALTY 1
          % First, add the new bit to the end.
          \xa\g@addto@macro\xa\lastfive\xa{\newbit}%
          \ifnum\i<5\relax%
            %Not yet on the 5th entry.
            %Don't do any testing.
          \else
            % 5th entry or later.
            % Remove the old one, and then test.
            \removefirsttoken\lastfive%
            \ifx\lastfive\@fiveones%
              \ifstringoffive%
                %This is a continuation of a previous block of five or more 1's.
                \stepcounter{penaltyi}%
              \else
                %This is a new string of five 1's.
                \addtocounter{penaltyi}{\qr@Ni}%
                \global\stringoffivetrue
              \fi
            \else
              \ifx\lastfive\@fivezeros%
                \ifstringoffive
                  %This is a continuation of a previous block of five or more 0's.
                  \stepcounter{penaltyi}%
                \else
                  %This is a new string of five 0's.
                  \addtocounter{penaltyi}{\qr@Ni}%
                  \global\stringoffivetrue
                \fi
              \else
                %This is not a string of five 1's or five 0's.
                \global\stringoffivefalse
              \fi
            \fi
          \fi
          %
          % HAPPILY, WE DON'T NEED TO CALCULATE PENALTY 2 AGAIN.
          %
          % LASTNINE CODE FOR PENALTY 3
          % First, add the new bit to the end.
          \xa\g@addto@macro\xa\lastnine\xa{\newbit}%
          \ifnum\i<7\relax%
            %Not yet on the 7th entry.
            %Don't do any testing.
          \else
            % 7th entry or later.
            % Remove the old one, and then test.
            \removefirsttoken\lastnine
            \xa\ifnum\qr@size=\i\relax%
              % Last column.  Any of the following should count:
              %     1011101 (\@finderB@zero)
              %    10111010 (\@finderB@one)
              %   101110100 (\@finderB@two)
              %  1011101000 (\@finderB@three)
              % 10111010000 (\@finderB)
              \ifx\lastnine\@finderB
                \addpenaltyiii
              \else
                \removefirsttoken\lastnine
                \ifx\lastnine\@finderB@three
                  \addpenaltyiii
                \else
                  \removefirsttoken\lastnine
                  \ifx\lastnine\@finderB@two
                    \addpenaltyiii
                  \else
                    \removefirsttoken\lastnine
                    \ifx\lastnine\@finderB@one
                      \addpenaltyiii
                    \else
                      \removefirsttoken\lastnine
                      \ifx\lastnine\@finderB@zero
                        \addpenaltyiii
                      \fi
                    \fi
                  \fi
                \fi
              \fi
            \else
              \ifx\lastnine\@finderA% %Matches 0000 1011101
                \addpenaltyiii
                %Also, we record our discovery, so that we can't count this pattern again
                %if it shows up four columns later as 1011101 0000.
                %
                %Set \ignore@finderB@at to \i+4.
                \qr@a=\i\relax%
                \advance\qr@a by 4%
                \xdef\ignore@finderB@at{\the\qr@a}%
              \else
                \ifx\lastfive\@finderB% %Matches 1011101 0000.
                  \xa\ifnum\ignore@finderB@at=\i\relax
                    %This pattern was *not* counted already earlier.
                    \addpenaltyiii
                  \fi
                \fi
              \fi
            \fi
          \fi
          %
         }% end of i-loop
      }% end of j-loop
  \egroup%
  %
  %CALCULATE PENALTY 4
  %According to the spec, penalty #4 is computed as
  % floor( |(i/n^2)-0.5|/0.05 )
  % where i is the total number of 1's in the matrix.
  % This is equal to abs(20*i-10n^2) div n^2.
  %
  \qr@a=\totalones\relax
  \multiply\qr@a by 20\relax
  \qr@b=\qr@size\relax
  \multiply\qr@b by \qr@size\relax
  \qr@c=10\relax
  \multiply\qr@c by \qr@b\relax
  \advance\qr@a by -\qr@c\relax
  \ifnum\qr@a<0\relax
    \multiply\qr@a by -1\relax
  \fi
  \divide\qr@a by \qr@b\relax
  \setcounter{penaltyiv}{\the\qr@a}%
  %
  %CALCULATE TOTAL PENALTY
  \qr@a=\the\penaltyi\relax%
  \advance\qr@a by \the\penaltyii\relax%
  \advance\qr@a by \the\penaltyiii\relax%
  \advance\qr@a by \the\penaltyiv\relax%
  \edef\qr@penalty{\the\qr@a}%
}%

\def\removefirsttoken#1{%
  %Removes the first token from the macro named in #1.
  \edef\qr@argument{(#1)}%
  \xa\removefirsttoken@int\qr@argument%
  \xdef#1{\removefirsttoken@result}%
}%
\def\removefirsttoken@int(#1#2){%
  \def\removefirsttoken@result{#2}%
}%

\def\qr@writeformatstring#1#2{%
  % #1 = matrix name
  % #2 = binary string representing the encoded and masked format information
  \setcounter{qr@i}{9}%
  \setcounter{qr@j}{1}%
  \edef\qr@argument{{#1}(#2\relax)}%
  \xa\qr@writeformatA@recursive\qr@argument
  %
  \setcounter{qr@i}{\qr@numberofrowsinmatrix{#1}}%
  \setcounter{qr@j}{9}%
  \xa\qr@writeformatB@recursive\qr@argument
}%

\def\qr@writeformatA@recursive#1(#2#3){%
  % #1 = matrix name
  % #2 = first bit of string
  % #3 = rest of bitstream
  % (qr@i,qr@j) = current (valid) position to write (in LaTeX counters)
  \def\formattowrite{#3}%
  \ifnum#2=1\relax
    \qr@storetomatrix{#1}{\theqr@i}{\theqr@j}{\qr@black@format}%
  \else
    \qr@storetomatrix{#1}{\theqr@i}{\theqr@j}{\qr@white@format}%
  \fi
  % Now the tricky part--moving \i and \j to their next positions.
  \ifnum\qr@j<9\relax
    %If we're not yet in column 9, move right.
    \stepcounter{qr@j}%
    \ifnum\qr@j=7\relax
      %But we skip column 7!
      \stepcounter{qr@j}%
    \fi
  \else
    %If we're in column 9, we move up.
    \addtocounter{qr@i}{-1}%
    \ifnum\qr@i=7\relax
      %But we skip row 7!
      \addtocounter{qr@i}{-1}%
    \fi
  \fi
  %N.B. that at the end of time, this will leave us at invalid position (0,9).
  %That makes for an easy test to know when we are done.
  \ifnum\qr@i<1
    \let\nexttoken=\relax
  \else
    \def\nexttoken{\qr@writeformatA@recursive{#1}(#3)}%
  \fi
  \nexttoken
}%

\def\qr@writeformatB@recursive#1(#2#3){%
  % #1 = matrix name
  % #2 = first bit of string
  % #3 = rest of bitstream
  % (qr@i,qr@j) = current (valid) position to write (in LaTeX counters)
  \def\formattowrite{#3}%
  \ifnum#2=1\relax
    \qr@storetomatrix{#1}{\theqr@i}{\theqr@j}{\qr@black@format}%
  \else
    \qr@storetomatrix{#1}{\theqr@i}{\theqr@j}{\qr@white@format}%
  \fi
  % Now the tricky part--moving counters i and j to their next positions.
  \qr@a=\qr@size%
  \advance\qr@a by -6\relax%
  \ifnum\qr@a<\qr@i\relax
    %If we're not yet in row n-6, move up.
    \addtocounter{qr@i}{-1}%
  \else
    \ifnum\qr@a=\qr@i\relax
      %If we're actually in row n-6, we jump to position (9,n-7).
      \setcounter{qr@i}{9}%
      %Set counter j equal to \qr@size-7.
      \global\qr@j=\qr@size\relax%
      \global\advance\qr@j by -7\relax%
    \else
      %Otherwise, we must be in row 9.
      %In this case, we move right.
      \stepcounter{qr@j}%
    \fi
  \fi
  %N.B. that at the end of time, this will leave us at invalid position (9,n+1).
  %That makes for an easy test to know when we are done.
  \xa\ifnum\qr@size<\qr@j\relax
    \let\nexttoken=\relax
  \else
    \def\nexttoken{\qr@writeformatB@recursive{#1}(#3)}%
  \fi
  \nexttoken
}%

\def\qr@writeversionstring#1#2{%
  % #1 = matrix name
  % #2 = binary string representing the encoded version information
  %
  % Plot the encoded version string into the matrix.
  % This is only done for versions 7 and higher.
  \xa\ifnum\qr@version>6\relax
    %Move to position (n-8,6).
    \setcounter{qr@i}{\qr@size}\relax%
    \addtocounter{qr@i}{-8}\relax%
    \setcounter{qr@j}{6}%
    \edef\qr@argument{{#1}(#2\relax)}%
    \xa\qr@writeversion@recursive\qr@argument
  \fi
}%

\def\qr@writeversion@recursive#1(#2#3){%
  % #1 = matrix name
  % #2 = first bit of string
  % #3 = rest of bitstream
  % (qr@i,qr@j) = current (valid) position to write (in LaTeX counters)
  %
  % The version information is stored symmetrically in the matrix
  % In two transposed regions, so we can write both at the same time.
  % In the comments, we describe what happens in the lower-left region,
  % not the upper-right.
  %
  \def\versiontowrite{#3}%
  %
  %Set \topline equal to n-10.
  \qr@a=\qr@size\relax%
  \advance\qr@a by -10\relax%
  \edef\topline{\the\qr@a}%
  %
  \ifnum#2=1\relax
    \qr@storetomatrix{#1}{\theqr@i}{\theqr@j}{\qr@black@format}%
    \qr@storetomatrix{#1}{\theqr@j}{\theqr@i}{\qr@black@format}%
  \else
    \qr@storetomatrix{#1}{\theqr@i}{\theqr@j}{\qr@white@format}%
    \qr@storetomatrix{#1}{\theqr@j}{\theqr@i}{\qr@white@format}%
  \fi
  % Now the tricky part--moving counters i and j to their next positions.
  \addtocounter{qr@i}{-1}%
  \xa\ifnum\topline>\qr@i\relax
    %We've overshot the top of the region.
    %We need to move left one column and down three.
    \addtocounter{qr@j}{-1}%
    \addtocounter{qr@i}{3}%
  \fi
  %N.B. that at the end of time, this will leave us at invalid position (n-8,0).
  %That makes for an easy test to know when we are done.
  \ifnum\qr@j<1\relax
    \let\nexttoken=\relax
  \else
    \def\nexttoken{\qr@writeversion@recursive{#1}(#3)}%
  \fi
  \nexttoken
}%
\newcount\qr@hexchars

\def\qr@string@binarytohex#1{%
  \qr@binarytohex{\qr@hex@result}{#1}%
}%

\def\qr@encode@binary#1{%
  % #1 = string of ascii characters, to be converted into bitstream
  %
  % We do this one entirely in hex, rather than binary, because we can.
  \edef\plaintext{#1}%
  %
  %First, the mode indicator.
  \def\qr@codetext{4}% %This means `binary'
  %
  %Next, the character count.
  \qr@getstringlength{\plaintext}%
  %Set \charactercountlengthinhex to \qr@charactercountbits@byte/4%
  \qr@a=\qr@charactercountbits@byte\relax%
  \divide \qr@a by 4\relax%
  \edef\charactercountlengthinhex{\the\qr@a}%
  \qr@decimaltohex[\charactercountlengthinhex]{\charactercount}{\qr@stringlength}%
  \xa\g@addto@macro\xa\qr@codetext\xa{\charactercount}%
  %
  %Now comes the actual data.
  \edef\qr@argument{(,\plaintext\relax\relax\relax)}%
  \xa\qr@encode@ascii@recursive\qr@argument%
  %
  %Now the terminator.
  \g@addto@macro\qr@codetext{0}% %This is '0000' in binary.
  %
  %There is no need to pad bits to make a multiple of 8,
  %because the data length is already 4 + 8 + 8n + 4.
  %
  %Now add padding codewords if needed.
  \setcounter{qr@hexchars}{0}%
  \qr@getstringlength{\qr@codetext}%
  \setcounter{qr@hexchars}{\qr@stringlength}%
  %Set \qr@numpaddingcodewords equal to \qr@totaldatacodewords - hexchars/2.
  \qr@a=-\qr@hexchars\relax
  \divide\qr@a by 2\relax
  \advance\qr@a by \qr@totaldatacodewords\relax
  \edef\qr@numpaddingcodewords{\the\qr@a}%
  %
  \xa\ifnum\qr@numpaddingcodewords<0%
    \edef\ds{ERROR: Too much data!  Over by \qr@numpaddingcodewords bytes.}\show\ds%
  \fi%
  \xa\ifnum\qr@numpaddingcodewords>0%
    \qr@for \i = 2 to \qr@numpaddingcodewords by 2%
      {\g@addto@macro{\qr@codetext}{ec11}}%
    \xa\ifodd\qr@numpaddingcodewords\relax%
      \g@addto@macro{\qr@codetext}{ec}%
    \fi%
  \fi%
}%

\def\qr@encode@ascii@recursive(#1,#2#3){%
  % #1 = hex codes translated so far
  % #2 = next plaintext character to translate
  % #3 = remainder of plaintext
  \edef\testii{#2}%
  \ifx\testii\@relax%
    % All done!
    \g@addto@macro\qr@codetext{#1}%
  \else%
    % Another character to translate.
    \edef\asciicode{\number`#2}%
    \qr@decimaltohex[2]{\newhexcodes}{\asciicode}%
    \edef\qr@argument{(#1\newhexcodes,#3)}%
    %\show\qr@argument
    \xa\qr@encode@ascii@recursive\qr@argument%
  \fi%
}%

\def\qr@splitcodetextintoblocks{%
  \setcounter{qr@i}{0}%
  \qr@for \j = 1 to \qr@numshortblocks by 1%
    {\stepcounter{qr@i}%
     \qr@splitoffblock{\qr@codetext}{\theqr@i}{\qr@shortblock@size}%
    }%
  \xa\ifnum\qr@numlongblocks>0\relax%
    \qr@for \j = 1 to \qr@numlongblocks by 1%
      {\stepcounter{qr@i}%
       \qr@splitoffblock{\qr@codetext}{\theqr@i}{\qr@longblock@size}%
      }%
  \fi%
}%

\def\qr@splitoffblock#1#2#3{%
  % #1 = current codetext in hexadecimal
  % #2 = number to use in csname "\datablock@#2".
  % #3 = number of bytes to split off
  \qrmessage{<Splitting off block #2>}%
  \xa\gdef\csname datablock@#2\endcsname{}% %This line is important!
  \qr@for \i = 1 to #3 by 1%
    {\edef\qr@argument{{#2}(#1)}%
     \xa\qr@splitoffblock@int\qr@argument%
    }%
}%

\def\qr@splitoffblock@int#1(#2#3#4){%
  % #1 = number to use in csname "\datablock@#1".
  % #2#3 = next byte to split off
  % #4 = remaining text
  %
  % We add the next byte to "\datablock@#1",
  % and we remove it from the codetext.
  \xa\xdef\csname datablock@#1\endcsname{\csname datablock@#1\endcsname#2#3}%
  \xdef\qr@codetext{#4}%
}%

\def\qr@createerrorblocks{%
  \qr@for \ii = 1 to \qr@numblocks by 1%
    {\qrmessage{<Making error block \the\ii>}%
     \FX@generate@errorbytes{\csname datablock@\the\ii\endcsname}{\qr@num@eccodewords}%
     \xa\xdef\csname errorblock@\the\ii\endcsname{\FX@errorbytes}%
    }%
}%

\def\qr@interleave{%
  \setcounter{qr@i}{0}%
  \def\qr@interleaved@text{}%
  \qrmessage{<Interleaving datablocks of length \qr@shortblock@size\space and \qr@longblock@size: }%
  \qr@for \ii = 1 to \qr@shortblock@size by 1%
    {\qr@for \jj = 1 to \qr@numblocks by 1%
      {\qr@writefromblock{datablock}{\the\jj}%
      }%
     \qrmessage{\the\ii,}%
    }%
  %The long blocks are numbered \qr@numshortblocks+1, \qr@numshortblocks+2, ..., \qr@numblocks.
  \qr@a=\qr@numshortblocks\relax%
  \advance\qr@a by 1\relax%
  \qr@for \jj = \qr@a to \qr@numblocks by 1%
      {\qr@writefromblock{datablock}{\the\jj}}%
  \xa\ifnum\qr@numlongblocks>0\relax%
    \qrmessage{\qr@longblock@size.>}%
  \else
    \qrmessage{.>}%
  \fi
  \qrmessage{<Interleaving errorblocks of length \qr@num@eccodewords: }%
  \qr@for \ii = 1 to \qr@num@eccodewords by 1%
    {\qrmessage{\the\ii,}%
     \qr@for \jj = 1 to \qr@numblocks by 1%
      {\qr@writefromblock{errorblock}{\the\jj}%
      }%
    }%
  \qrmessage{.><Interleaving complete.>}%
}%

\def\qr@writefromblock#1#2{%
  % #1 = either 'datablock' or 'errorblock'
  % #2 = block number, in {1,...,\qr@numblocks}%
  \edef\qr@argument{(\csname #1@#2\endcsname\relax\relax\relax)}%
  \xa\qr@writefromblock@int\qr@argument
  \xa\xdef\csname #1@#2\endcsname{\qr@writefromblock@remainder}%
}%

\def\qr@writefromblock@int(#1#2#3){%
  % #1#2 = first byte (in hex) of text, which will be written to \qr@interleaved@text
  % #3 = remainder, including \relax\relax\relax terminator.
  \g@addto@macro{\qr@interleaved@text}{#1#2}%
  \qr@writefromblock@intint(#3)%
}%

\def\qr@writefromblock@intint(#1\relax\relax\relax){%
  \xdef\qr@writefromblock@remainder{#1}%
}%
\let\xa=\expandafter

\def\preface@macro#1#2{%
  % #1 = macro name
  % #2 = text to add to front of macro
  \def\tempb{#2}%
  \xa\xa\xa\gdef\xa\xa\xa#1\xa\xa\xa{\xa\tempb #1}%
}%

\newif\ifqr@leadingcoeff
\def\qr@testleadingcoeff(#1#2){%
  % Tests whether the leading digit of #1#2 is 1.
  \ifnum#1=1\relax
    \qr@leadingcoefftrue
  \else
    \qr@leadingcoefffalse
  \fi
}%

\def\qr@polynomialdivide#1#2{%
  \edef\qr@numerator{#1}%
  \edef\qr@denominator{#2}%
  \qr@divisiondonefalse%
  \xa\xa\xa\qr@oneroundofdivision\xa\xa\xa{\xa\qr@numerator\xa}\xa{\qr@denominator}%
}%

\def\@qr@empty{}%
\def\qr@oneroundofdivision#1#2{%
  % #1 = f(x), of degree n
  % #2 = g(x), of degree m
  % Obtains a new polynomial h(x), congruent to f(x) modulo g(x),
  % but of degree at most n-1.
  %
  % If leading coefficient of f(x) is 1, subtracts off g(x) * x^(n-m).
  % If leading coefficient of f(x) is 0, strips off that leading zero.
  %
  \qr@testleadingcoeff(#1)%
  \ifqr@leadingcoeff
    \qr@xorbitstrings{#1}{#2}%
    \ifqr@xorfailed
      %If xor failed, that means our #1 was already the remainder!
      \qr@divisiondonetrue
      \edef\theremainder{#1}%
    \else
      %xor succeeded. We need to recurse.
      \xa\xa\xa\edef\xa\xa\xa\qr@numerator\xa\xa\xa{\xa\qr@stripleadingzero\xa(\xorresult)}%
    \fi
  \else
    \xa\def\xa\qr@numerator\xa{\qr@stripleadingzero(#1)}%
    \ifx\qr@numerator\@qr@empty
      \qr@divisiondonetrue
      \def\theremainder{0}%
    \fi
  \fi
  \ifqr@divisiondone
    \relax
  \else
    \xa\qr@oneroundofdivision\xa{\qr@numerator}{#2}%
  \fi
}%

\def\qr@stripleadingzero(0#1){#1}%Strips off a leading zero.

\newif\ifqr@xorfailed% This flag will trigger when #2 is longer than #1.

\def\qr@xorbitstrings#1#2{%
 % #1 = bitstring
 % #2 = bitstring no longer than #1
 \qr@xorfailedfalse
 \edef\qr@argument{(,#1\relax\relax)(#2\relax\relax)}%
 \xa\qr@xorbitstrings@recursive\qr@argument
 %\qr@xorbitstrings@recursive(,#1\relax\relax)(#2\relax\relax)%
}%

\def\qr@xorbitstrings@recursive(#1,#2#3)(#4#5){%
 % #1#2#3 is the first bitstring, xor'ed up through #1.
 % #4#5 is the remaining portion of the second bitstring.
 \def\testii{#2}%
 \def\testiv{#4}%
 \ifx\testii\@relax
   % #1 contains the whole string.
   % Now if #4 is also \relax, that means the two strings started off with equal lengths.
   % If, however, #4 is not \relax, that means the second string was longer than the first, a problem.
   \ifx\testiv\@relax
     %No problem.  We are done.
     \qr@xorbit@saveresult(#1#2#3)%
   \else
     %Problem!  The second string was longer than the first.
     \qr@xorfailedtrue
     \def\xorresult{}%
   \fi
 \else
   % There is still a bit to manipulate in #2.
   % Check whether #4 contains anything.
   \ifx\testiv\@relax
     % No, #4 is empty.  We are done. "#2#3" contains the remainder of the first string,
     % which we append untouched and then strip off the two \relax-es.
     \qr@xorbit@saveresult(#1#2#3)%
   \else
     % Yes, #4 still has something to XOR. Do the task.
     \ifnum#2=#4\relax
       \qr@xorbitstrings@recursive(#1%
                                 0,#3)(#5)%
     \else
       \qr@xorbitstrings@recursive(#1%
                                 1,#3)(#5)%
     \fi
   \fi
 \fi
}%

\def\qr@xorbit@saveresult(#1\relax\relax){%
  %Strips off the extra '\relax'es at the end.
  \def\xorresult{#1}%
}%

\newif\ifqr@divisiondone
\def\dodivision#1#2{%
  \qr@divisiondonefalse
  \dodivision@recursive{#1}{#2}%
}%

\def\BCHcode#1{%
  \edef\formatinfo{#1}%
  \def\formatinfopadded{\formatinfo 0000000000}%
  \def\qr@divisor{10100110111}%
  \qr@divisiondonefalse
  \qr@polynomialdivide{\formatinfopadded}{\qr@divisor}%
  %
  \qr@getstringlength{\theremainder}%
  %Run loop from stringlength+1 to 10.
  \qr@a=\qr@stringlength\relax%
  \advance\qr@a by 1\relax%
  \qr@for \i = \qr@a to 10 by 1%
    {\preface@macro{\theremainder}{0}%
     \xdef\theremainder{\theremainder}%
    }%
  \edef\BCHresult{\formatinfo\theremainder}%
}%

\def\qr@formatmask{101010000010010}%

\def\qr@encodeandmaskformat#1{%
  \BCHcode{#1}%
  \qr@xorbitstrings{\BCHresult}{\qr@formatmask}%
  \edef\qr@format@bitstring{\xorresult}%
}%

\def\qr@Golaycode#1{%
  % #1 = 6-bit version number
  \edef\qr@versioninfo{#1}%
  \def\qr@versioninfopadded{\qr@versioninfo 000000000000}% %Append 12 zeros.
  \def\qr@divisor{1111100100101}%
  \qr@divisiondonefalse
  \qr@polynomialdivide{\qr@versioninfopadded}{\qr@divisor}%
  %
  \qr@getstringlength{\theremainder}%
  %Run loop from stringlength+1 to 12.
  \qr@a=\qr@stringlength\relax%
  \advance\qr@a by 1\relax%
  \qr@for \i = \qr@a to 12 by 1%
    {\preface@macro{\theremainder}{0}%
     \xdef\theremainder{\theremainder}%
    }%
  \edef\Golayresult{\qr@versioninfo\theremainder}%
}%
\def\F@result{}%

\def\qr@xorbitstring#1#2#3{%
  % #1 = new macro to receive result
  % #2, #3 = bitstrings to xor.  The second can be shorter than the first.
  \def\qr@xor@result{}%
  \edef\qr@argument{(#2\relax\relax)(#3\relax\relax)}%
  \xa\qr@xorbitstring@recursive\qr@argument%
  \edef#1{\qr@xor@result}%
}%
\def\qr@xorbitstring@recursive(#1#2)(#3#4){%
  \edef\testi{#1}%
  \ifx\testi\@relax%
    %Done.
    \let\qr@next=\relax%
  \else
    \if#1#3\relax
      \g@addto@macro{\qr@xor@result}{0}%
    \else
      \g@addto@macro{\qr@xor@result}{1}%
    \fi
    \edef\qr@next{\noexpand\qr@xorbitstring@recursive(#2)(#4)}%
  \fi
  \qr@next
}

\def\F@addchar@raw#1#2{%
  %Add two hexadecimal digits using bitwise xor
  \qr@hextobinary[4]{\summandA}{#1}%
  \qr@hextobinary[4]{\summandB}{#2}%
  \qr@xorbitstring{\F@result}{\summandA}{\summandB}%
  \qr@binarytohex[1]{\F@result}{\F@result}%
}%

\def\canceltwos#1{%
  \edef\qr@argument{(#1\relax\relax)}%
  \xa\canceltwos@int\qr@argument%
}%

\def\canceltwos@int(#1#2){%
  \xa\canceltwos@recursion(,#1#2)%
}%

\def\canceltwos@recursion(#1,#2#3){%
  \def\testii{#2}%
  \ifx\testii\@relax
    %Cancelling complete.
    \striptworelaxes(#1#2#3)%
    %Now \F@result contains the answer.
  \else
    \relax
    \ifnum#2=2\relax
      \canceltwos@recursion(#10,#3)%
    \else
      \canceltwos@recursion(#1#2,#3)%
    \fi
  \fi
}%

\def\striptworelaxes(#1\relax\relax){%
  \gdef\F@result{#1}%
}%

\qr@for \i = 0 to 15 by 1%
  {\qr@decimaltohex[1]{\qr@tempa}{\the\i}%
   \qr@for \j = 0 to 15 by 1%
    {\qr@decimaltohex[1]{\qr@tempb}{\the\j}%
     \F@addchar@raw\qr@tempa\qr@tempb
     \xa\xdef\csname F@addchar@\qr@tempa\qr@tempb\endcsname{\F@result}%
    }%
  }%

\def\F@addchar#1#2{%
  \xa\def\xa\F@result\xa{\csname F@addchar@#1#2\endcsname}%
}%

\def\F@addstrings#1#2{%
  \edef\qr@argument{(,#1\relax\relax)(#2\relax\relax)}%
  \xa\F@addstrings@recursion\qr@argument%
}%

\def\F@addstrings@recursion(#1,#2#3)(#4#5){%
  %Adds two hexadecimal strings, bitwise, from left to right.
  %The second string is allowed to be shorter than the first.
  \def\testii{#2}%
  \def\testiv{#4}%
  \ifx\testii\@relax
    %The entire string has been processed.
    \gdef\F@result{#1}%
  \else
    \ifx\testiv\@relax
      %The second string is over.
      \striptworelaxes(#1#2#3)%
      %Now \F@result contains the answer.
    \else
      %We continue to add.
      \F@addchar{#2}{#4}%
      \edef\qr@argument{(#1\F@result,#3)(#5)}%
      \xa\F@addstrings@recursion\qr@argument%
    \fi
  \fi
}%
\gdef\F@stripleadingzero(0#1){\edef\F@result{#1}}%

\qr@i=0%
\def\poweroftwo{1}%
\qr@for \i = 1 to 254 by 1%
  {\global\advance\qr@i by1%
   \qr@a=\poweroftwo\relax
   \multiply\qr@a by 2\relax
   \edef\poweroftwo{\the\qr@a}%
   %\show\poweroftwo
   \qr@decimaltohex[2]{\poweroftwo@hex}{\poweroftwo}%
   \xa\ifnum\poweroftwo>255\relax
     %We need to bitwise add the polynomial represented by 100011101, i.e. 0x11d.
     \F@addstrings{\poweroftwo@hex}{11d}%               %Now it should start with 0.
     \xa\F@stripleadingzero\xa(\F@result)%              %Now it should be two hex digits.
     \edef\poweroftwo@hex{\F@result}%                   %Save the hex version.
     \qr@hextodecimal{\poweroftwo}{\F@result}%
   \fi
   \xdef\poweroftwo{\poweroftwo}%
   \xa\xdef\csname F@twotothe@\theqr@i\endcsname{\poweroftwo@hex}%
   \xa\xdef\csname F@logtwo@\poweroftwo@hex\endcsname{\theqr@i}%
  }%
\xa\xdef\csname F@twotothe@0\endcsname{01}%
\xa\xdef\csname F@logtwo@01\endcsname{0}%

\def\F@twotothe#1{%
  \xa\xdef\xa\F@result\xa{\csname F@twotothe@#1\endcsname}%
}%
\def\F@logtwo#1{%
  \xa\xdef\xa\F@result\xa{\csname F@logtwo@#1\endcsname}%
}%

\def\@zerozero{00}%

\def\F@multiply#1#2{%
  % #1 and #2 are two elements of F_256,
  % given as two-character hexadecimal strings.
  % Multiply them within F_256, and place the answer in \F@result
  \edef\argA{#1}%
  \edef\argB{#2}%
  \ifx\argA\@zerozero
    \def\F@result{00}%
  \else
    \ifx\argB\@zerozero
      \def\F@result{00}%
    \else
      \xa\F@logtwo\xa{\argA}%
        \edef\logA{\F@result}%
      \xa\F@logtwo\xa{\argB}%
        \edef\logB{\F@result}%
      \xa\qr@a\xa=\logA\relax%  \qr@a = \logA
      \xa\advance\xa\qr@a\logB\relax% \advance \qr@a by \logB
      \ifnum\qr@a>254\relax%
        \advance\qr@a by -255\relax%
      \fi%
      \xa\F@twotothe\xa{\the\qr@a}%
      % Now \F@result contains the product, as desired.
    \fi
  \fi
}%

\def\F@multiply#1#2{%
  % #1 and #2 are two elements of F_256,
  % given as two-character hexadecimal strings.
  % Multiply them within F_256, and place the answer in \F@result
  \edef\argA{#1}%
  \edef\argB{#2}%
  \ifx\argA\@zerozero
    \def\F@result{00}%
  \else
    \ifx\argB\@zerozero
      \def\F@result{00}%
    \else
      \xa\F@logtwo\xa{\argA}%
        \edef\logA{\F@result}%
      \xa\F@logtwo\xa{\argB}%
        \edef\logB{\F@result}%
      \xa\qr@a\xa=\logA\relax%  \qr@a = \logA
      \xa\advance\xa\qr@a\logB\relax% \advance \qr@a by \logB
      \ifnum\qr@a>254\relax%
        \advance\qr@a by -255\relax%
      \fi%
      \xa\F@twotothe\xa{\the\qr@a}%
      % Now \F@result contains the product, as desired.
    \fi
  \fi
}%

\def\FX@getstringlength#1{%
  %Count number of two-character coefficients
  \setcounter{qr@i}{0}%
  \xdef\qr@argument{(#1\relax\relax\relax)}%
  \xa\FX@stringlength@recursive\qr@argument%
  \xdef\stringresult{\arabic{qr@i}}%
}%

\def\FX@stringlength@recursive(#1#2#3){%
  \def\testi{#1}%
  \ifx\testi\@relax
    %we are done.
  \else
    \stepcounter{qr@i}%
    %\showthe\c@qr@i
    \qr@stringlength@recursive(#3)%
  \fi
}%

\newif\ifFX@leadingcoeff@zero
\def\FX@testleadingcoeff(#1#2#3){%
  % Tests whether the leading coefficient of the hex-string #1#2#3 is '00'.
  \edef\FX@leadingcoefficient{#1#2}%
  \FX@leadingcoeff@zerofalse
  \ifx\FX@leadingcoefficient\@zerozero
    \FX@leadingcoeff@zerotrue
  \fi
}%

\newif\ifFX@divisiondone

\newcount\qr@divisionsremaining %Keep track of how many divisions to go!
\def\FX@polynomialdivide#1#2{%
  \edef\FX@numerator{#1}%
  \edef\denominator{#2}%
  \qr@getstringlength\FX@numerator%
  \setcounter{qr@divisionsremaining}{\qr@stringlength}%
  \qr@getstringlength\denominator%
  \addtocounter{qr@divisionsremaining}{-\qr@stringlength}%
  \addtocounter{qr@divisionsremaining}{2}%
  \divide\qr@divisionsremaining by 2\relax% %2 hex chars per number
  \FX@divisiondonefalse%
  \xa\xa\xa\FX@polynomialdivide@recursive\xa\xa\xa{\xa\FX@numerator\xa}\xa{\denominator}%
}%

\def\FX@polynomialdivide@recursive#1#2{%
  % #1 = f(x), of degree n
  % #2 = g(x), of degree m
  % Obtains a new polynomial h(x), congruent to f(x) modulo g(x),
  % but of degree at most n-1.
  %
  % If leading coefficient of f(x) is 0, strips off that leading zero.
  % If leading coefficient of f(x) is a, subtracts off a * g(x) * x^(n-m).
  % N.B. we assume g is monic.
  %
  \FX@testleadingcoeff(#1)%
  \ifFX@leadingcoeff@zero%
    %Leading coefficient is zero, so remove it.
    \xa\def\xa\FX@numerator\xa{\FX@stripleadingzero(#1)}%
  \else%
    %Leading coefficient is nonzero, and contained in \FX@leadingcoefficient
    \FX@subtractphase{#1}{#2}{\FX@leadingcoefficient}%
    \ifFX@subtract@failed%
      %If subtraction failed, that means our #1 was already the remainder!
      \FX@divisiondonetrue%
      \edef\theremainder{#1}%
    \else%
      %xor succeeded. We need to recurse.
      \xa\xa\xa\edef\xa\xa\xa\FX@numerator\xa\xa\xa{\xa\FX@stripleadingzero\xa(\FX@subtraction@result)}%
    \fi%
  \fi%
  \addtocounter{qr@divisionsremaining}{-1}%
  \ifnum\qr@divisionsremaining=0\relax
    %Division is done!
    \FX@divisiondonetrue%
    \edef\theremainder{\FX@numerator}%
    \relax%
  \else%
    \xa\FX@polynomialdivide@recursive\xa{\FX@numerator}{#2}%
  \fi%
}%

\def\FX@stripleadingzero(00#1){#1}%Strips off a single leading zero of F_256.

\newif\ifFX@subtract@failed% This flag will trigger when #2 is longer than #1.

\def\FX@subtractphase#1#2#3{%
 % #1 = bitstring
 % #2 = bitstring no longer than #1
 % #3 = leading coefficient
 \FX@subtract@failedfalse%
 \edef\qr@argument{(,#1\relax\relax\relax)(#2\relax\relax\relax)(#3)}%
 \xa\FX@subtract@recursive\qr@argument%
}%

\def\FX@subtract@recursive(#1,#2#3#4)(#5#6#7)(#8){%
 % This is a recursive way to compute f(x) - a*g(x)*x^k.
 % #1#2#3#4 is the first bitstring, subtracted up through #1.
 %          Thus #2#3 constitutes the next two-character coefficient.
 % #5#6#7 is the remaining portion of the second bitstring.
 %          Thus #5#6 constitutes the next two-character coefficient
 % #8 is the element a of F_256.  It should contain two characters.
 \def\testii{#2}%
 \def\testv{#5}%
 \ifx\testii\@relax
   % #1 contains the whole string.
   % Now if #5 is also \relax, that means the two strings started off with equal lengths.
   % If, however, #5 is not \relax, that means the second string was longer than the first, a problem.
   \ifx\testv\@relax
     %No problem.  We are done.
     \FX@subtract@saveresult(#1#2#3#4)% %We keep the #2#3#4 to be sure we have all three relax-es to strip off.
   \else
     %Problem!  The second string was longer than the first.
     %This usually indicates the end of the long division process.
     \FX@subtract@failedtrue
     \def\FX@subtraction@result{}%
   \fi
 \else
   % There is still a coefficient to manipulate in #2#3.
   % Check whether #5 contains anything.
   \ifx\testv\@relax
     % No, #5 is empty.  We are done. "#2#3#4" contains the remainder of the first string,
     % which we append untouched and then strip off the three \relax-es.
     \FX@subtract@saveresult(#1#2#3#4)%
   \else
     % Yes, #5#6 still has something to XOR. Do the task.
     \F@multiply{#5#6}{#8}% Multiply by the factor 'a'.
     \F@addstrings{#2#3}{\F@result}% Subtract.  (We're in characteristic two, so adding works.)
     \edef\qr@argument{(#1\F@result,#4)(#7)(#8)}%
     \xa\FX@subtract@recursive\qr@argument%
   \fi
 \fi
}%

\def\FX@subtract@saveresult(#1\relax\relax\relax){%
  %Strips off the three extra '\relax'es at the end.
  \def\FX@subtraction@result{#1}%
}%

\def\FX@creategeneratorpolynomial#1{%
  % #1 = n, the number of error codewords desired.
  % We need to create \prod_{j=0}^{n-1} (x-2^j).
  \edef\FX@generator@degree{#1}%
  \def\FX@generatorpolynomial{01}% Initially, set it equal to 1.
  \setcounter{qr@i}{0}%
  \FX@creategenerator@recursive%
  %The result is now stored in \FX@generatorpolynomial
}%

\def\FX@creategenerator@recursive{%
  % \c@qr@i contains the current value of i.
  % \FX@generatorpolynomial contains the current polynomial f(x),
  %   which should be a degree-i polynomial
  %   equal to \prod_{j=0}^{i-1} (x-2^j).
  %   (If i=0, then \FX@generatorpolynomial should be 01.)
  % This recursion step should multiply the existing polynomial by (x-2^i),
  % increment i by 1, and check whether we're done or not.
  \edef\summandA{\FX@generatorpolynomial 00}% This is f(x) * x
  \edef\summandB{00\FX@generatorpolynomial}% This is f(x), with a 0x^{i+1} in front.
  \F@twotothe{\theqr@i}%
  \edef\theconstant{\F@result}%
  \FX@subtractphase{\summandA}{\summandB}{\theconstant}%
     %This calculates \summandA + \theconstant * \summandB
     %and stores the result in \FX@subtraction@result
  \edef\FX@generatorpolynomial{\FX@subtraction@result}%
  \stepcounter{qr@i}%
  \xa\ifnum\FX@generator@degree=\qr@i\relax%
    %We just multiplied by (x-2^{n-1}), so we're done.
    \relax%
  \else%
    %We need to do this again!
    \xa%
    \FX@creategenerator@recursive%
  \fi%
}%

\def\FX@generate@errorbytes#1#2{%
  % #1 = datastream in hex
  % #2 = number of error correction bytes requested
  \edef\numerrorbytes{#2}%
  \xa\FX@creategeneratorpolynomial\xa{\numerrorbytes}%
  \edef\FX@numerator{#1}%
  \qr@for \i = 1 to \numerrorbytes by 1%
    {\g@addto@macro\FX@numerator{00}}% %One error byte means two hex codes.
  \FX@polynomialdivide{\FX@numerator}{\FX@generatorpolynomial}%
  \edef\FX@errorbytes{\theremainder}%
}%
\newif\ifqr@versionmodules

\def\qr@level@char#1{%
    \xa\ifcase#1
      M\or L\or H\or Q\fi}%

\newif\ifqr@versiongoodenough
\def\qr@choose@best@version#1{%
  % \qr@desiredversion = user-requested version
  % \qr@desiredlevel = user-requested error-correction level
  \edef\qr@plaintext{#1}%
  \qr@getstringlength{\qr@plaintext}%
  %
  %Run double loop over levels and versions, looking for
  %the smallest version that can contain our data,
  %and then choosing the best error-correcting level at that version,
  %subject to the level being at least as good as the user desires.
  \global\qr@versiongoodenoughfalse%
  \gdef\qr@bestversion{0}%
  \gdef\qr@bestlevel{0}%
  \ifnum\qr@desiredversion=0\relax
    \qr@a=1\relax
  \else
    \qr@a=\qr@desiredversion\relax
  \fi
  \qr@for \i=\qr@a to 40 by 1
    {\edef\qr@version{\the\i}%
     \global\qr@versiongoodenoughfalse
     \qr@for \j=0 to 3 by 1%
      {%First, we map {0,1,2,3} to {1,0,4,3}, so that we loop through {M,L,H,Q}
       %in order of increasing error-correction capabilities.
       \qr@a = \j\relax
       \divide \qr@a by 2\relax
       \multiply \qr@a by 4\relax
       \advance \qr@a by 1\relax
       \advance \qr@a by -\j\relax
       \edef\qr@level{\the\qr@a}%
       \ifnum\qr@desiredlevel=\qr@a\relax
         \global\qr@versiongoodenoughtrue
       \fi
       \ifqr@versiongoodenough
         \qr@calculate@capacity{\qr@version}{\qr@level}%
         \xa\xa\xa\ifnum\xa\qr@truecapacity\xa<\qr@stringlength\relax
           %Too short
           \relax
         \else
           %Long enough!
           \xdef\qr@bestversion{\qr@version}%
           \xdef\qr@bestlevel{\qr@level}%
           \global\i=40%
         \fi
       \fi
      }%
     }%
  \edef\qr@version{\qr@bestversion}%
  \edef\qr@level{\qr@bestlevel}%
  \xa\ifnum\qr@desiredversion>0\relax
    \ifx\qr@bestversion\qr@desiredversion\relax
      %No change from desired version.
    \else
      %Version was increased
      \qrmessage{<Requested QR version '\qr@desiredversion' is too small for desired text.}%
      \qrmessage{Version increased to '\qr@bestversion' to fit text.>^^J}%
    \fi
  \fi
  \ifx\qr@bestlevel\qr@desiredlevel\relax
    %No change in level.
  \else
    \qrmessage{<Error-correction level increased from \qr@level@char{\qr@desiredlevel}}%
    \qrmessage{to \qr@level@char{\qr@bestlevel} at no cost.>^^J}%
  \fi
}%

\def\qr@calculate@capacity#1#2{%
  \edef\qr@version{#1}%
  \edef\qr@level{#2}%
  %Calculate \qr@size, the number of modules per side.
  % The formula is 4\qr@version+17.
  \qr@a=\qr@version\relax%
  \multiply\qr@a by 4\relax%
  \advance\qr@a by 17\relax%
  \xdef\qr@size{\the\qr@a}%
  %
  % Calculate \qr@k, which governs the number of alignment patterns.
  % The alignment patterns lie in a kxk square, except for 3 that are replaced by finding patterns.
  % The formula is 2 + floor( \qr@version / 7 ), except that k=0 for version 1.
  \xa\ifnum\qr@version=1\relax%
    \def\qr@k{0}%
  \else%
    \qr@a=\qr@version\relax
    \divide \qr@a by 7\relax
    \advance\qr@a by 2\relax
    \edef\qr@k{\the\qr@a}%
  \fi%
  %
  %Calculate number of function pattern modules.
  %This consists of the three 8x8 finder patterns, the two timing strips, and the (k^2-3) 5x5 alignment patterns.
  %The formula is 160+2n+25(k^2-3)-10(k-2), unless k=0 in which case we just have 160+2n.
  \qr@a=\qr@size\relax
  \multiply\qr@a by 2\relax
  \advance\qr@a by 160\relax
  \xa\ifnum\qr@k=0\relax\else
    %\qr@k is nonzero, hence at least 2, so we continue to add 25(k^2-3)-10(k-2).
    \qr@b=\qr@k\relax
    \multiply\qr@b by \qr@k\relax
    \advance\qr@b by -3\relax
    \multiply\qr@b by 25\relax
    \advance\qr@a by \qr@b\relax
    \qr@b=\qr@k\relax
    \advance\qr@b by -2\relax
    \multiply\qr@b by 10\relax
    \advance\qr@a by -\qr@b\relax
  \fi
  \edef\qr@numfunctionpatternmodules{\the\qr@a}%
  %
  %Calculate the number of version modules, either 36 or 0.
  \xa\ifnum\qr@version>6\relax
    \qr@versionmodulestrue
    \def\qr@numversionmodules{36}%
  \else
    \qr@versionmodulesfalse
    \def\qr@numversionmodules{0}%
  \fi
  %
  %Now calculate the codeword capacity and remainder bits.
  %Take n^2 modules, subtract all those dedicated to finder patterns etc., format information, and version information,
  %and what's left is the number of bits we can play with.
  %The number of complete bytes is \qr@numdatacodewords;
  %the leftover bits are \qr@numremainderbits.
  \qr@a=\qr@size\relax
  \multiply \qr@a by \qr@size\relax
  \advance \qr@a by -\qr@numfunctionpatternmodules\relax
  \advance \qr@a by -31\relax% % There are 31 format modules.
  \advance \qr@a by -\qr@numversionmodules\relax
  \qr@b=\qr@a\relax
  \divide \qr@a by 8\relax
  \edef\qr@numdatacodewords{\the\qr@a}%
  \multiply\qr@a by 8\relax
  \advance \qr@b by -\qr@a\relax
  \edef\qr@numremainderbits{\the\qr@b}%
  %
  %The size of the character count indicator also varies by version.
  %There are only two options, so hardcoding seems easier than expressing these functionally.
  \xa\ifnum\qr@version<10\relax
    \def\qr@charactercountbytes@byte{1}%
    \def\qr@charactercountbits@byte{8}%
  \else
    \def\qr@charactercountbytes@byte{2}%
    \def\qr@charactercountbits@byte{16}%
  \fi
  %
  %Now we call on the table, from the QR specification,
  %of how many blocks to divide the message into, and how many error bytes each block gets.
  %This affects the true capacity for data, which we store into \qr@totaldatacodewords.
  % The following macro sets \qr@numblocks and \qr@num@eccodewords
  % based on Table 9 of the QR specification.
  \qr@settableix
  \qr@a = -\qr@numblocks\relax
  \multiply \qr@a by \qr@num@eccodewords\relax
  \advance\qr@a by \qr@numdatacodewords\relax
  \edef\qr@totaldatacodewords{\the\qr@a}%
  \advance\qr@a by -\qr@charactercountbytes@byte\relax%Subtract character count
  \advance\qr@a by -1\relax% Subtract 1 byte for the 4-bit mode indicator and the 4-bit terminator at the end.
  \edef\qr@truecapacity{\the\qr@a}%
}

\def\qr@setversion#1#2{%
  % #1 = version number, an integer between 1 and 40 inclusive.
  % #2 = error-correction level, as an integer between 0 and 3 inclusive.
  %      0 = 00 = M
  %      1 = 01 = L
  %      2 = 10 = H
  %      3 = 11 = Q
  % This macro calculates and sets a variety of global macros and/or counters
  % storing version information that is used later in construction the QR code.
  % Thus \setversion should be called every time!
  %
  \edef\qr@version{#1}%
  \edef\qr@level{#2}%
  %
  \qr@calculate@capacity{\qr@version}{\qr@level}%
  %The capacity-check code sets the following:
  % * \qr@size
  % * \qr@k
  % * \ifqr@versionmodules
  % * \qr@numversionmodules
  % * \qr@numdatacodewords
  % * \qr@numremainderbits
  % * \qr@charactercountbits@byte
  % * \qr@charactercountbytes@byte
  % * \qr@numblocks (via \qr@settableix)
  % * \qr@num@eccodewords (via \qr@settableix)
  % * \qr@totaldatacodewords
  %
  % The alignment patterns' square is 7 modules in from each edge.
  % They are spaced "as evenly as possible" with an even number of modules between each row/column,
  % unevenness in division being accommodated by making the first such gap smaller.
  % The formula seems to be
  %    general distance = 2*round((n-13)/(k-1)/2+0.25)
  %                     = 2*floor((n-13)/(k-1)/2+0.75)
  %                     = 2*floor( (2*(n-13)/(k-1)+3) / 4 )
  %                     = (((2*(n-13)) div (k-1) + 3 ) div 4 ) * 2
  %    first distance = leftovers
  % The 0.25 is to accommodate version 32, which is the only time we round down.
  % Otherwise a simple 2*ceiling((n-13)/(k-1)/2) would have sufficed.
  %
  \qr@a = \qr@size\relax
  \advance\qr@a by -13\relax
  \multiply\qr@a by 2\relax
  \qr@b = \qr@k\relax
  \advance \qr@b by -1\relax
  \divide\qr@a by \qr@b\relax
  \advance\qr@a by 3\relax
  \divide\qr@a by 4\relax
  \multiply\qr@a by 2\relax
  \edef\qr@alignment@generalskip{\the\qr@a}%
  %
  %Now set \qr@alignment@firstskip to (\qr@size-13)-(\qr@k-2)*\qr@alignment@generalskip %
  \qr@a = \qr@k\relax
  \advance\qr@a by -2\relax
  \multiply\qr@a by -\qr@alignment@generalskip\relax
  \advance\qr@a by \qr@size\relax
  \advance\qr@a by -13\relax
  \edef\qr@alignment@firstskip{\the\qr@a}%
  %
  %
  %
  % Our \qr@totaldatacodewords bytes of data are broken up as evenly as possible
  % into \qr@numblocks datablocks; some may be one byte longer than others.
  % We set \qr@shortblock@size to floor(\qr@totaldatacodewords / \qr@numblocks)
  % and \qr@numlongblocks to mod(\qr@totaldatacodewords , \qr@numblocks).
  \qr@a=\qr@totaldatacodewords\relax
  \divide\qr@a by \qr@numblocks\relax
  \edef\qr@shortblock@size{\the\qr@a}%
  \multiply\qr@a by -\qr@numblocks\relax
  \advance\qr@a by \qr@totaldatacodewords\relax
  \edef\qr@numlongblocks{\the\qr@a}%
  %
  %Set \qr@longblock@size to \qr@shortblock@size+1.
  \qr@a=\qr@shortblock@size\relax
  \advance\qr@a by 1\relax
  \edef\qr@longblock@size{\the\qr@a}%
  %
  %Set \qr@numshortblocks to \qr@numblocks - \qr@numlongblocks
  \qr@b=\qr@numblocks\relax
  \advance\qr@b by -\qr@numlongblocks\relax
  \edef\qr@numshortblocks{\the\qr@b}%
}%

\def\qr@settableix@int(#1,#2){%
  \edef\qr@numblocks{#1}%
  \edef\qr@num@eccodewords{#2}%
}%

\def\qr@settableix{%
\xa\ifcase\qr@level\relax
  %00: Level 'M', medium error correction
  \edef\tempdata{(%
    \ifcase\qr@version\relax
      \relax %There is no version 0.
    \or1,10%
    \or1,16%
    \or1,26%
    \or2,18%
    \or2,24%
    \or4,16%
    \or4,18%
    \or4,22%
    \or5,22%
    \or5,26%
    \or5,30%
    \or8,22%
    \or9,22%
    \or9,24%
    \or10,24%
    \or10,28%
    \or11,28%
    \or13,26%
    \or14,26%
    \or16,26%
    \or17,26%
    \or17,28%
    \or18,28%
    \or20,28%
    \or21,28%
    \or23,28%
    \or25,28%
    \or26,28%
    \or28,28%
    \or29,28%
    \or31,28%
    \or33,28%
    \or35,28%
    \or37,28%
    \or38,28%
    \or40,28%
    \or43,28%
    \or45,28%
    \or47,28%
    \or49,28%
  \fi)}%
\or
  %01: Level 'L', low error correction
  \edef\tempdata{%
  (\ifcase\qr@version\relax
    \relax %There is no version 0.
  \or 1,7%
  \or 1,10%
  \or 1,15%
  \or 1,20%
  \or 1,26%
  \or 2,18%
  \or 2,20%
  \or 2,24%
  \or 2,30%
  \or 4,18%
  \or 4,20%
  \or 4,24%
  \or 4,26%
  \or 4,30%
  \or 6,22%
  \or 6,24%
  \or 6,28%
  \or 6,30%
  \or 7,28%
  \or 8,28%
  \or 8,28%
  \or 9,28%
  \or 9,30%
  \or 10,30%
  \or 12,26%
  \or 12,28%
  \or 12,30%
  \or 13,30%
  \or 14,30%
  \or 15,30%
  \or 16,30%
  \or 17,30%
  \or 18,30%
  \or 19,30%
  \or 19,30%
  \or 20,30%
  \or 21,30%
  \or 22,30%
  \or 24,30%
  \or 25,30%
  \fi)}%
\or
  %10: Level 'H', high error correction
  \edef\tempdata{(%
    \ifcase\qr@version\relax
      \relax %There is no version 0.
    \or1,17%
    \or1,28%
    \or2,22%
    \or4,16%
    \or4,22%
    \or4,28%
    \or5,26%
    \or6,26%
    \or8,24%
    \or8,28%
    \or11,24%
    \or11,28%
    \or16,22%
    \or16,24%
    \or18,24%
    \or16,30%
    \or19,28%
    \or21,28%
    \or25,26%
    \or25,28%
    \or25,30%
    \or34,24%
    \or30,30%
    \or32,30%
    \or35,30%
    \or37,30%
    \or40,30%
    \or42,30%
    \or45,30%
    \or48,30%
    \or51,30%
    \or54,30%
    \or57,30%
    \or60,30%
    \or63,30%
    \or66,30%
    \or70,30%
    \or74,30%
    \or77,30%
    \or81,30%
  \fi)}%
\or
  %11: Level 'Q', quality error correction
  \edef\tempdata{(%
    \ifcase\qr@version\relax
      \relax %There is no version 0.
    \or1,13%
    \or1,22%
    \or2,18%
    \or2,26%
    \or4,18%
    \or4,24%
    \or6,18%
    \or6,22%
    \or8,20%
    \or8,24%
    \or8,28%
    \or10,26%
    \or12,24%
    \or16,20%
    \or12,30%
    \or17,24%
    \or16,28%
    \or18,28%
    \or21,26%
    \or20,30%
    \or23,28%
    \or23,30%
    \or25,30%
    \or27,30%
    \or29,30%
    \or34,28%
    \or34,30%
    \or35,30%
    \or38,30%
    \or40,30%
    \or43,30%
    \or45,30%
    \or48,30%
    \or51,30%
    \or53,30%
    \or56,30%
    \or59,30%
    \or62,30%
    \or65,30%
    \or68,30%
    \fi)}%
\fi
\xa\qr@settableix@int\tempdata
}%

\def\@qr@M{M}\def\@qr@z{0}%
\def\@qr@L{L}\def\@qr@i{1}%
\def\@qr@H{H}\def\@qr@ii{2}%
\def\@qr@Q{Q}\def\@qr@iii{3}%
\def\qr@setlevel#1{%
  \edef\qr@level@selected{#1}%
  \ifx\qr@level@selected\@qr@M
    \edef\qr@desiredlevel{0}%
  \fi
  \ifx\qr@level@selected\@qr@L
    \edef\qr@desiredlevel{1}%
  \fi
  \ifx\qr@level@selected\@qr@H
    \edef\qr@desiredlevel{2}%
  \fi
  \ifx\qr@level@selected\@qr@Q
    \edef\qr@desiredlevel{3}%
  \fi
  \ifx\qr@level@selected\@qr@z
    \edef\qr@desiredlevel{0}%
  \fi
  \ifx\qr@level@selected\@qr@i
    \edef\qr@desiredlevel{1}%
  \fi
  \ifx\qr@level@selected\@qr@ii
    \edef\qr@desiredlevel{2}%
  \fi
  \ifx\qr@level@selected\@qr@iii
    \edef\qr@desiredlevel{3}%
  \fi
}%

% key-value pairs (OPmac trick 0069)
\def\kv#1{\expandafter\ifx\csname kv:#1\endcsname \relax \expandafter\kvunknown
   \else \csname kv:#1\expandafter\endcsname\fi 
}
\def\kvunknown{???}
\def\kvscan #1#2=#3,{\ifx#1,\else \sdef{kv:#1#2}{#3}\expandafter\kvscan\fi}

\ifx\replacestrings\undefined
\bgroup \catcode`!=3 \catcode`?=3
\gdef\replacestrings#1#2{\long\def\replacestringsA##1#1##2!{%
   \ifx!##2!\addto\tmpb{##1}\else\addto\tmpb{##1#2}\replacestringsA##2!\fi}%
   \edef\tmpb{\expandafter}\expandafter\replacestringsA\tmpb?#1!%
   \long\def\replacestringsA##1?{\def\tmpb{##1}}\expandafter\replacestringsA\tmpb
}
\egroup
\long\def\addto#1#2{\expandafter\def\expandafter#1\expandafter{#1#2}}
\def\sdef#1{\expandafter\def\csname#1\endcsname}
\fi

\def\qrset#1{\def\tmpb{#1,}%
   \replacestrings{ =}{=}\replacestrings{= }{=}%
   \replacestrings{tight,}{qr-border=0,}%
   \replacestrings{padding,}{qr-border=1,}%
   \replacestrings{verbose,}{qr-message=1,}%
   \replacestrings{silent,}{qr-message=0,}%
   \replacestrings{draft,}{qr-final=0,}%
   \replacestrings{final,}{qr-final=1,}%
   \replacestrings{nolink,}{qr-link=0,}%
   \replacestrings{link,}{qr-link=1,}%
   \expandafter\kvscan\tmpb,=,%
   \qrdesiredheight=\kv{height}\relax
   \qr@setlevel{\kv{level}}%
   \edef\qr@desiredversion{\kv{version}}%
}
\qrset{height=2cm, version=0, level=M, tight, verbose, final, nolink}

\def\qrcode{\begingroup
   % LaTeX ballast:
   \def\setcounter##1##2{\global\csname##1\endcsname=##2\relax}%
   \def\stepcounter##1{\global\advance\csname##1\endcsname by1\relax}%
   \def\addtocounter##1##2{\global\advance\csname##1\endcsname by##2\relax}%
   \let\xa=\expandafter \newlinechar=`\^^J
   \isnextchar[{\qrcodeA}{\qrcodeB}%
}
\def\qrcodeA[#1]{\qrset{#1}\expandafter\qrcodeB\romannumeral-`\.}
\def\qrcodeB{%
   \ifx\mubyteout\undefined \else \mubyteout=0 \mubytelog=0 \fi
   \def\xprncodesave{}%
   \ifx\xprncodes\undefined \else 
      \ifnum\xprncode255=0 \def\xprncodesave{\xprncodes=0 }\xprncodes=1 \fi\fi
   \if1\kv{qr-message}\let\qrmessage=\message \else \def\qrmessage##1{}\fi
   \if1\kv{qr-border}\def\padd{\kern4\qrmodulesize}\else\def\padd{}\fi
   \bgroup \qrverbatim \qrcode@i
}
\def\qrcode@i#1{\xdef\qretext{#1}\gdef\qrtext{#1}\egroup
   \qrcode@int
   \xprncodesave
   \endgroup
}

\def\qrcode@int{%
  \qrmessage{<QR code requested for "\qretext" in version
           \qr@desiredversion-\qr@level@char{\qr@desiredlevel}.>^^J}%
  %First, choose the version and level.
  %Recall that \qr@choose@best@version sets \qr@version and \qr@level.
  \xa\qr@choose@best@version\xa{\qretext}%
  \if1\kv{qr-final}%
     \qr@setversion{\qr@version}{\qr@level}%
     \qrcode@int@new
  \else
     \qrmodulesize=\qrdesiredheight%
     \divide\qrmodulesize by \qr@size\relax%
     \let\d=\qrdesiredheight
     \vbox{\padd\hbox{\padd\vbox to\d{\hrule\vss
        \hbox to\d{\vrule height.7\d depth.3\d \hss ...QR...\hss\vrule}%
        \vss\hrule}\padd}\padd}%
  \fi
}%

\def\qrcode@int@new{%
  \qrbeginhook
  \qr@createsquareblankmatrix{newqr}{\qr@size}%
  \qr@placefinderpatterns{newqr}%
  \qr@placetimingpatterns{newqr}%
  \qr@placealignmentpatterns{newqr}%
  \qr@placedummyformatpatterns{newqr}%
  \qr@placedummyversionpatterns{newqr}%
  \qrmessage{<Calculating QR code for "\qretext" in
            version \qr@version-\qr@level@char{\qr@level}.>^^J}%
  \xa\qr@encode@binary\xa{\qretext}%
  \qr@splitcodetextintoblocks
  \qr@createerrorblocks
  \qr@interleave
  \qrmessage{<Writing data...}%
     \qr@writedata@hex{newqr}{\qr@interleaved@text}%
  \qrmessage{done.>^^J}%
  \qr@writeremainderbits{newqr}%
  \qr@chooseandapplybestmask{newqr}%
  \qr@decimaltobinary[2]{\level@binary}{\qr@level}%
  \qr@decimaltobinary[3]{\mask@binary}{\qr@mask@selected}%
  \edef\formatstring{\level@binary\mask@binary}%
  \qrmessage{<Encoding and writing format string...}%
     \xa\qr@encodeandmaskformat\xa{\formatstring}%
     \qr@writeformatstring{newqr}{\qr@format@bitstring}%
  \qrmessage{done.>^^J}%
  \qrmessage{<Encoding and writing version information...}%
     \qr@decimaltobinary[6]{\version@binary}{\qr@version}%
     \qr@Golaycode{\version@binary}%
     \qr@writeversionstring{newqr}{\Golayresult}%
  \qrmessage{done.>^^J}%
  \qrmessage{<Printing QR code...}%
     \qrmatrixtobinary{newqr}%
     \qrrestore\qrdata
  \qrmessage{done.>^^J}%
  \qrendhook
}%

\def\qrmatrixtobinary#1{%
   \bgroup
   \gdef\qrdata{}%
   \def\qr@black{1}\let\qr@black@fixed=\qr@black \let\qr@black@format=\qr@black
   \def\@white{0}\let\qr@white@fixed=\@white \let\qr@white@format=\@white
   \qr@for \i = 1 to \qr@size by 1
      {\qr@for \j = 1 to \qr@size by 1
          {\xdef\qrdata{\qrdata\qr@matrixentry{#1}{\the\i}{\the\j}}}}%
   \xdef\qrdata{{\qr@size}{\qrdata}}%
   \egroup
}

\def\qrrestore#1{\expandafter\qrrestoreA#1}
\def\qrrestoreA#1#2{%
   \qrmodulesize=\qrdesiredheight \divide\qrmodulesize by#1
   \if1\kv{qr-link}\setbox0=\fi
   \vbox\bgroup\padd \offinterlineskip \baselineskip=\qrmodulesize 
      \qr@i=0 \qr@j=0 \let\next=\qrrestoreB 
   \hbox\bgroup\padd \qrrestoreB #2%
   \if1\kv{qr-link}\qr@link{\qretext}{\box0}\fi
}
\def\qrrestoreB#1{\advance \qr@j by1
   \ifx1#1\vrule height\qrmodulesize width\qrmodulesize\else \kern\qrmodulesize\fi
   \ifnum\qr@size=\qr@j \vrule height\qrmodulesize width 0pt \padd\egroup \advance\qr@i by1
      \ifnum\qr@size=\qr@i \padd\egroup \let\next=\relax \else \hbox\bgroup\padd \fi
   \fi \next
}

\def\qrbeginhook{}
\def\qrendhook{}

\tmp % \catcode of @ is returned back.

\endinput


Options
-------

You can use \qrset{options} for global-like options and
\qrcode[options]{encoded text} for local options for one QR code.
The \qrset{options} is valid within a group (if exists) or in whole
document. 

Options are separated by comma and they are in two types: single
word or key=value format. Default options are: 

\qrset{height=2cm, version=0, level=M, tight, verbose, final, nolink}

The options are the same as described in qrcode.pdf at 
http://www.ctan.org/tex-archive/macros/latex/contrib/qrcode.
In short:

height=dimen   ... The height of the QRcode without padding.

version=number ... Number 0 to 40 linearly depends on the density of QRcode.
                   The 0 means that the density is automatically selected.

level=letter   ... L, M, Q o H (low, medium, quality, hight) sets the amount 
                   of redundancy in the code in order of error recovering.

tight   ... Code without margins.
padding ... 4module blank margins around the code.

verbose ... Information about calculating in terminal and in the log.
silent  ... No information about calculating.

final   ... The QR code is calculated and printed.
draft   ... Only empty rectangle in the same size as QR code is printed.

nolink  ... The QR code is not active hyperlink.
link    ... The QR code is active hyperlink to "encoded text".
            Note that link option works in pdfTeX (luaTeX) only.

qrborder={R G B}  ... The color of the frame around active hypertext space
                      if link option is set. R G B (red green blue) are decimal 
                      numbers from 0 to 1. The frame is visible only in
                      pdf viewers. Default: invisible frame.

Example:

\qrset{silent}     % ... all codes will be silent in the log and terminal.
\qrcode [height=3cm, link, padding, qrborder={1 0 0}] {http://petr.olsak.net}  
                   % ... 3cm QRcode as hyperlink

Note:

The saving/restoring pre-calculated QRcodes isn't supported by default.
If you are printing the same QR codes repeatedly, use \setbox/\copy
technique. For example:

\newbox\mybox
\setbox\mybox=\hbox{\qrcode{encoded text}}
\copy\mybox \copy\mybox \copy\mybox etc.

If you have a huge amount of different QR codes, you can use draft/final options
or you can use REF file from OPmac. See the OPmac trick 

   http://petr.olsak.net/opmac-tricks-e.html#qrcode

The \qrdata macro is saved after each \qrcode calculation in the format
{size}{111101011...001} where size is the number of columns or rows in QR
square and second parameter includes size^2 ones or zeros which means black
or white modules (scanned left to right, top to bottom). Another information
can be retrieved from \qrtext macro (encoded text before expanding) and
\qretext macro (encoded text where \{, \\ etc. are expanded to {, \ etc.).
The macros \qrdata, \qrtext and \qretext are saved globally.


Non-ASCII characters
--------------------

If you are using csplain with pdfTeX (no XeTeX, no LuaTeX) then UTF-8 input
is correctly interpreted from \qrcode parameter. 

The technical background: the encTeX's \mubyte is set to zero during
scanning the \qrcode parameter, so the parameter is rawly UTF-8 encoded and
this is correct for QR codes.

Problems: 
1. You cannot use \qrcode{parameter} inside another macro, bacause UTF-8
   encoded parameter is reencoded already.
2. You cannot use XeTeX or LuaTeX because UTF-8 encoded parameter is
   reencoded to Unicode already. And the backward conversion from Unicode 
   to UTF-8 isn't implemented here at macro level.


History
-------

Jun. 2015  released
Jul. 2015 \xprncodes=0space (bug fixed)
Sep. 2018 \isnextchar processed in group
May  2019 strut included for case of empty line (bug fixed)
