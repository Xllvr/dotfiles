\documentclass{ltxdoc}
\usepackage{holtxdoc}
\begin{document}
\title{The \xpackage{bidi-atbegshi} Package}
\author{Vafa Khalighi\\\xemail{tex.ctan@gmail.com}}
\maketitle
The \xpackage{bidi-atbegshi} adds bidi support to package \xpackage{atbegshi}; it modifies the \cs{AtBeginShipoutUpperLeft}, and \cs{AtBeginShipoutUpperLeftForeground} so that they properly both in LTR and RTL modes. In addition, it also defines the following commands:

\begin{declcs}{AtBeginShipoutUpperRight} \M{background material}
\end{declcs}
 This is a macro that puts material in the background of
 box \cs{AtBeginShipoutBox}.
 The \meta{background material} is set in an \cs{hbox}, the
 reference point is the upper right corner of the output page.


 The macro \cs{AtBeginShipoutUpperRight} is intended to be used
 in one of the hook setting macros, such as
 \cs{AtBeginShipout}, \cs{AtBeginShipoutFirst}, or
 \cs{AtBeginShipoutNext}.

 The \meta{background material} is set
 inside a \texttt{picture} environment:
 \begin{quote}
   |\begin{picture}(0,0)|\\
   \mbox{}\quad |\setlength{\unitlength}{1pt}%|\\
   \mbox{}\quad \meta{background material}\\
   |\end{picture}|
 \end{quote}
 
\begin{declcs}{AtBeginShipoutLowerLeft} \M{background material}
\end{declcs}
Similar to the \cs{AtBeginShipoutUpperRight} command but
 the reference point is the lower left corner of the output page.

\begin{declcs}{AtBeginShipoutLowerRight} \M{background material}
\end{declcs}
Similar to the \cs{AtBeginShipoutUpperRight} command but
 the reference point is the lower right corner of the output page.

 \begin{declcs}{AtBeginShipoutUpperRightForeground} \M{foreground material}
 \end{declcs}
 See \cs{AtBeginShipoutUpperRight}. The difference is that the material
 is put in the foreground.
 
  \begin{declcs}{AtBeginShipoutLowerLeftForeground} \M{foreground material}
 \end{declcs}
 See \cs{AtBeginShipoutLowerLeft}. The difference is that the material
 is put in the foreground.
 
  \begin{declcs}{AtBeginShipoutLowerRightForeground} \M{foreground material}
 \end{declcs}
 See \cs{AtBeginShipoutLowerRight}. The difference is that the material
 is put in the foreground.
 
  \begin{declcs}{LengthToUnit} \M{length with units}
 \end{declcs}
 For instance, \cs{LengthToUnit}\texttt{\{12pt\}} converts 12pt to 12 (stripping the unit).
\end{document}