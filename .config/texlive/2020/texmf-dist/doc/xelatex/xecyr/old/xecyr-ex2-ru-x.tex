\documentclass{article}
\usepackage[russian]{babel}
\usepackage[mis]{xecyr}

\setmainfont[Mapping=tex-text]{CMU Serif}
\setmonofont{CMU Typewriter Text}

\begin{document}

\section{Как работать в системе \XeLaTeX{}\\ под Windows}

\begin{itemize}

\item[1.] С \TeX{}ом нельзя работать интерактивно, как, скажем с
Word'ом, поэтому запуск любого exe--файла из директории
\verb|TeXLive| не приведет ни к какому результату.

\item[2.] Необходимая последовательность действий примерно такая:

\begin{itemize}

\item Сначала нужно набрать текст в каком-нибудь текстовом редакторе, который умеет сохранять текст в кодировке Unicode (UTF-8), например, Notepad++, Vim, AkelPad (но не Word). Следите за тем, чтобы текст {\bf не содержал} переносов, номеров страниц и шрифтовых выделений.

\item Затем следует разделить все абзацы дополнительными пустыми
строками (абзацные отступы для \TeX{}а ничего не значат, как и
вообще все лишние пробелы)

\item До начала текста нужно написать следующее:

\begin{verbatim}
\documentclass{article}
\usepackage{xecyr}
\usepackage[russian]{babel}
\setmainfont[Mapping=tex-text]{CMU Serif}
\setmonofont[Mapping=tex-text]{CMU Typewriter Text}
\begin{document}
\end{verbatim}

При этом важно, чтобы у Вас были установлены шрифты CM-Unicode.

\item В конце текста нужно написать следующее:

\verb|\end{document}|

\item Проверьте, нет ли в тексте следующих специальных символов:

\begin{verbatim}
{ } $  & # % _ ^ ~ \
\end{verbatim}
%$

Их надо либо убрать, либо заменить (например, вместо <<\verb|%|>>
поставить \verb|\%|, а вместо <<\verb|&|>> "--- \verb|\&|).

\item Сохраните текст с расширением \verb|tex| или \verb|ltx|,
например, \verb|myfile.tex|. Теперь Ваш файл готов для обработки
\TeX{}ом.

\end{itemize}

\item[3.] Вызовите командную строку так, чтобы она открылась в текущей папке. Напишите в командной строке следующее:

\verb|xelatex myfile.tex|

В ответ запустится программа--транслятор. Она начнет выдавать на
экран сообщения о своей работе, о встретившихся трудностях и
ошибках. В последнем случае трансляция остановится, не дойдя до
конца. Тогда попробуйте нажать клавишу \verb|<ENTER>| или \verb|s| (это
помогает в большинстве случаев, а если не помогло "---
обратитесь к специалисту-\TeX{}нику).

В некоторых случаях выводимую информацию прочитать с экрана нельзя. Это происходит потому, что командная строка Windows не поддерживает кодировку UTF-8. Но, поскольку эти сообщения записываются еще и в специальный файл (в нашем случае он будет называться \verb|myfile.log|, то к ним всегда можно вернуться по окончании трансляции.

\item[4.] Как только трансляция завершится, в текущей директории
появится среди прочих файл с расширением \verb|pdf|. Именно этот
файл можно просмотреть, а если результат Вас удовлетворит, то
и распечатать. Просматривать PDF-файлы можно программами Adobe Reader, Foxit Reader, или SumatraPDF.

Если же то, что получилось, Вам не подходит "--- нужно вернуться
к первоначальному файлу (в нашем примере это \verb|myfile.tex|),
исправить ошибки, ввести новые команды форматирования и т. п. и
опять попытаться его оттранслировать.

\item[5.] Вышеприведенные рекомендации годятся, конечно, лишь
для очень простых случаев (хотя, следуя им, можно порой достичь
прекрасных результатов). Если Вы хотите готовить тексты
приемлемой сложности, стоит обратиться либо к электронному
справочнику по \TeX{}у, либо к литературе.  Особенно хороша в
этом плане книга С.~М.~Львовского <<Набор и верстка в пакете
\LaTeX>>. Единственный ее недостаток "--- то, что она уже несколько устарела, \XeTeX\ там не описан. С другой стороны, почти все, что написано там про \TeX\ и \LaTeX, применимо к \XeTeX\ и \XeLaTeX.

\end{itemize}

\section{Особенности подготовки текста для последующей верстки в
\TeX}

Когда текст готовится для последующей верстки, всегда выгоднее
набирать его как <<чистый текст>> (без шрифтовых выделений,
переносов, специально оформленных заголовков и т.п.), потому что
любая электронная типографская система (будь то Adobe InDesign,
или Scribus, или QuarkXPress) все равно будет форматировать
текст самостоятельно. Однако автору часто хочется, чтобы его
текст был форматирован так, как надо ему, а не редакции. Есть
несколько способов этого добиться: можно самому научиться
верстать свои тексты (это лучший способ), можно делать пометки
на распечатке, представляемой в редакцию (так принято во многих
журналах), можно, наконец, не вникая в верстку досконально,
расставлять некие пометки прямо в файле, надеясь, что система
верстки их поймет. Последний вариант проходит только с \TeX{}ом.
Вот несколько рекомендаций для тех, кто набирает <<чистый
текст для последующей верстки в \TeX>>:

\begin{enumerate}

\item В тексте можно делать пространные комментарии о том, как
надо верстать те или иные куски текста, просто помечая начало
каждой строчки символом \verb|%|, например:

\verb|% Отсюда и до следующей пометки "--- жирный курсив!|

\item Другой способ "--- просто усвоить некоторые \TeX{}овские
макрокоманды для переключения шрифтов (например, на курсивный
--- \verb|\itshape|, а обратно "--- \verb|\normalfont|; прочие
команды см. в книге Львовского).

\item В \TeX{}е несколько видов тире. Обычное (длинное) тире
(em-dash) задается тремя знаками <<\verb|-|>>: \verb|---|;
среднее тире (en-dash) "--- двумя знаками (\verb|--|, это тире
используется в основном между цифрами, например, 22--34);
маленькое тире, или дефис "--- одним знаком \verb|-|. Эти тире
лучше всего набирать сразу (к тому же и текст выглядит <<читабельнее>>).

\item Есть возможность сразу проставить ударения в словах (это
важно, например, для различения слов б\'ольшая и больш\'ая). Для
этого используется команда \verb|\'|, например, тв\'ор\'ог. Так
же легко можно проставить и другие диакритические (надбуквенные)
знаки (перечень "--- в книге Львовского).

\item Часто в тексте сразу нужно отметить сноски. Для этого
служит команда \verb|\footnote{текст сноски}|.

\item Если нужны подстрочные или надстрочные символы, набирайте
их так: \verb|$^{надстрочный текст}$|  и
\verb|$_{подстрочный текст}$|,
например \verb|что-то$^{где-то}_{как-то}$| дает
что-то$^\textnormal{\scriptsize где-то}_\textnormal{\scriptsize
как-то}$.

\item Два значка часто используются в биологических текстах. Это
градус$^\circ$ (набирается \verb|$^\circ$|) и знак умножения
$\times$ (набирается \verb|$\times$|. Звездочка (*) набирается
непосредственно.

\item Значки параграфа (\S), кавычек (<<окавыченный текст>>) и н\'омера (\No)
ст\'оит набирать соотвественно как \verb|\S|, \verb|<<окавыченный текст>>| и \verb|\номер|.

\item \underline{Подчеркнуть} текст можно командой
\verb|\underline{|\texttt{текст, который надо подчеркнуть}\verb|}|.

\item Старайтесь не оставлять лишних пробелов и двойных пустых
строк в набираемом файле "--- хотя \TeX{} их не замечает, они
сильно ухудшают внешний вид и <<читабельность>>~файла.

\end{enumerate}

\end{document}
