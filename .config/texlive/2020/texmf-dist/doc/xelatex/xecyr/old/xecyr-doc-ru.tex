\documentclass{article}
\usepackage[mis]{xecyr}
\usepackage{url}
\usepackage[russian]{babel}

\setmainfont[Mapping=tex-text]{CMU Serif}
\setmonofont{CMU Typewriter Text}

\author{А.\,Б.\,Шипунов\footnote{e-mail: \texttt{plantago at herba.msu.ru}}}

\title{\texttt{xecyr.sty},\\
русский стиль для \XeLaTeX}

\date{}

\begin{document}
\maketitle

Стиль основан на файле А.\,Крюкова <<\texttt{xecyr.tex}>>.

Доступны две пары опций: \texttt{ext/noext} и \texttt{nomis/mis} (умалчиваемые опции первые). Первая грузит пакет \texttt{xltxtra} (и заодно \texttt{fontspec}, без которого жизнь в \XeLaTeX\ осложняется), вторая грузит (если ее задать явно) пакет \texttt{misccorr}, для загрузки которого нужен некий трюк с кодировками.

Кавычки набираются "<по-бабелевски">, при помощи \verb|"< "> "` "'|, лигатуры \verb|<< >>| будут работать только если в поставке имеются последние версии файлов \texttt{tex-text.map} и \texttt{tex-text.tec}, скачанные с \url{http://scripts.sil.org/svn-view/xetex/TRUNK}. В будущем эта ситуация, скорее всего, изменится.

Если кто-то желает использовать комбинированные русско-английские паттерны переносов, то нужно заменить в файле \texttt{language.dat} определение \texttt{russian} на \texttt{xu-ruenhyph.tex}. Соответствующий файл прилагается.

Работа пакета проверялась на дистрибутиве \TeX Live2007 под \textsf{Windows XP SP2} и под \textsf{Linux}. В системе должны быть установлены шрифты Computer Modern Unicode (\url{http://sourceforge.net/projects/cm-unicode}), желательно (для первого примера) также DejaVu (\url{http://sourceforge.net/projects/dejavu}).

\end{document}
