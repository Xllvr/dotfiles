
%%%%%%%%%%%%%%%%%%%%%%%%%%%%%
%% Facture & note de frais %%
%%%%%%%%%%%%%%%%%%%%%%%%%%%%%%%
%% Chargement des extensions %%
%%%%%%%%%%%%%%%%%%%%%%%%%%%%%%%

\RequirePackage{fontspec} 

%\RequirePackage{fontenc}

\RequirePackage[top=2 cm, bottom=2 cm, left=2.5 cm, right=2.5 cm]{geometry}
\RequirePackage{soul}
\RequirePackage{ulem}
\RequirePackage{eurosym}
\RequirePackage{lmodern}
\RequirePackage{color}
\RequirePackage{colortbl}
\RequirePackage[colorlinks=true, urlcolor=magenta]{hyperref}
\RequirePackage{mathtools}
\RequirePackage{amssymb}
\RequirePackage{mathrsfs}
\RequirePackage{multirow}
\RequirePackage{fancyhdr}
\RequirePackage{array}
\RequirePackage{ifthen}

\RequirePackage{calctab}

%%%%%%%%%%%%%%%%%%%%%%%%%%%%%
%% Facture & note de frais %%
%%%%%%%%%%%%%%%%%%%%%%%%%%%%%
%% Déclaration des options %%
%%%%%%%%%%%%%%%%%%%%%%%%%%%%%

\newif\if@latinUn \@latinUnfalse
\DeclareOption{latin1}{\@latinUntrue}

%% \DeclareOption{option}{...}
\DeclareOption*{\PassOptionsToPackage{\CurrentOption}{babel}}
\ProcessOptions

\RequirePackage{babel} 

\if@latinUn
    \RequirePackage{inputenc}
    \newcommand{\rsEncodage}{latin1}
\else
%    \RequirePackage[utf8]{inputenc}
    \newcommand{\rsEncodage}{UTF8}
\fi

%%%%%%%%%%%%%%%%%%%%%%%%%%%%%
%% Facture & note de frais %%
%%%%%%%%%%%%%%%%%%%%%%%%%%%%%%%
%% Définition des commandes  %%
%%%%%%%%%%%%%%%%%%%%%%%%%%%%%%%%%%%%%%%%%%%%%%%%%%%%%%%%%%%%%%%%%%%%%%%%%%%%%
%% Toutes les commandes sont obligatoirement sous la forme \rsQuelqueChose %%
%%%%%%%%%%%%%%%%%%%%%%%%%%%%%%%%%%%%%%%%%%%%%%%%%%%%%%%%%%%%%%%%%%%%%%%%%%%%%

%%%%%%%%%%%%%%%%%%%%%%%%%%%%%
%% Facture & note de frais %%
%% Couleurs de mise en évidence
\definecolor{grisfonce}{gray}{0.3}
\definecolor{grisclair}{gray}{0.7}



%%%%%%%%%%%%%%%%%%%%%%%%%%%%%%%%%%%%%
%% Facture & note de frais communs %%
%%%%%%%%%%%%%
%% diverses

%%%%%%%%%%%%%%%%%%%%%%%%%%%%%%%%%%%%%
%% Facture & note de frais communs %%
%% \rsCredit permet d'ajouter les crédits dans le footer.
\pagestyle{fancy} 
\fancyhf{}
\renewcommand{\headrulewidth}{0pt}
\newcommand{\rsCredit}{
\fancyfoot[EOC]{\footnotesize Extension \og \rsvarname \fg, V \rsvarversion, \rsvardate, \copyright\ \rsvarauthor.}
}

%%%%%%%%%%%%%%%%%%%%%%%%%%%%%%%%%%%%%
%% Facture & note de frais communs %%
%% permet de choisir l'unité monétaire ailleurs que dans le tableau 
%% des produits pour la facture  (package cactab).
%% \euro (defaut), \pounds (livre anglaise) \$ (dollar américain) \textyen (yen japonais)
%% Decomenter ci-dessous pour voir les unités
%%\$ \pounds \euro \textyen
\newcommand{\rsChoisirUniteMonetaire}[1]{\newcommand{\rsuniteMonetaire}{#1}}


%%%%%%%%%%%%%%%%%%%%%%%%%%%%%%%%%%%%%
%% Facture & note de frais communs %%
%% \rsAerationVerticale fixe la longueur de l'espacement vertical, par defaut 1.5 cm
%% et permet d'aérer la page verticalement; le paramètre est une mesure LaTeX
%% TeX comprend six unités de mesure :
%% + pt point = 0,35 mm
%% + mm millimètre
%% + ex correspond à la hauteur d'un x dans la fonte courante
%% + em correspond à la largeur d'un m dans la fonte courante
%% + cm centimètre
%% + in pouce = 2,54 cm
\newlength{\rsespaceVertical}
\newcommand{\rsAerationVerticale}[1]{\setlength{\rsespaceVertical}{#1}}
