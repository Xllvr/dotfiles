%% THIS IS A SAMPLE LATEX FILE DEMONSTRATING HOW TO TYPESET 
%% A CRITICAL EDITION WITH XeTeX, ednotes, and ArabXeTeX.
%% THE TEMPLATE CAN ALSO BE APPLIED, WITH MINOR MODIFICATIONS, 
%% TO CRITICAL EDITIONS INVOLVING RIGHT-TO-LEFT SCRIPTS IN GENERAL 
%% (EITHER IN THE MAIN TEXT OR IN THE CRITICAL APPARATUS). 
%% THUS THE USE OF ArabXeTeX IS NONMANDATORY, BUT THE PACKAGE bidi
%% IS REQUIRED FOR BIDIRECTIONALITY.
%% 
%% François Charette, 13 December 2006
%%
%%
\documentclass[12pt,a4paper,oneside]{memoir}
\usepackage{fontspec}
\setmainfont[Ligatures=TeX]{Junicode}
\usepackage[novoc]{arabxetex}
\newfontfamily\arabicfont[Script=Arabic,WordSpace=2]{Amiri}
%% OK, now we load ednotes with appropriate options. 
%% For each extra level of notes in the critical apparatus we need to 
%% initialize it with the appropriate option, in this case, "Bpara". 
%% Options "modulo", "perpage" and "right" are for line numbers: 
%% they will be printed at the right margin, at a certain interval 
%% to be set by the command \modulolinenumber below, 
%% and with the "perpage" option to start numeration anew on each page.
%% The option "para*" is for having the notes in paragraph format without indentation.
\usepackage[Bpara,modulo,right,perpage,para*]{ednotes}% relies on manyfoot and lineno

%% This makes the footnoterule as wide as the \textwidth
\makeatletter%
\renewcommand{\footnoterule}{\kern-3\p@
  \hrule width \textwidth \kern 2.6\p@}
\makeatother

%%EDNOTES SETUP
%%
%% Line numeration is printed every 5 lines:
\modulolinenumbers[5]

%% This is a little hack to set linenumbering in Arabic 
%% (the mapping is provided with ArabXeTeX):
\renewcommand{\linenumberfont}{\arabicfont\addfontfeature{Mapping=arabicdigits}\tiny}
%%
%% The following macros redefine the default formatting 
%% of various aspects of the critical apparatus.
%% Note that since the edition is contained in a Right-to-Left environment, 
%% the notes should be set RL by default, but to be certain we declare
\PrecedeLevelWith{A}{\setRL}
%% which puts the appropriate "hook" at the beginning of the block A of notes.
%% Here are the customized macros for the default apparatus:
%% The uncommented lines below are changed for typesetting the apparatus in Arabic.
% \renewcommand*{\sameline}[1]{\linesfmt{##1}}%
\renewcommand*{\differentlines}[2]{\linesfmt{\RL{#1\textendash#2}}}% 
\renewcommand*{\linesfmt}[1]{\raisebox{1ex}{\linenumberfont #1}~}% 
\renewcommand{\lemmafmt}[1]{#1~[ }% 
\renewcommand*{\pageandline}[2]{#2.\textLR{#1}}% #1 page, #2 line.
% \renewcommand*{\repeatref}[1]{\raisebox{-.5ex}{$\Vert$}}% << not advisable 
% \renewcommand{\lemmaellipsis}{\textsymmdots}%
% \renewcommand{\notefmt}[1]{##1}%

%% We do not want extra footnoterules between the different levels (see manyfoot doc):
\renewcommand{\extrafootnoterule}{}
\SelectFootnoteRule[0]{extra}

%% We can also customize the critical apparatus for a *specific* level of notes. 
%% In the following case we customize a level B of Left-to-Right notes.
%% To achieve this with ednotes, with need to feed-in the appropriate command
%% as a "hook" to manyfoot by means of:
\PrecedeLevelWith{B}{\unsetRL}
% Now we change the formatting of level B to take into account that it is set left-to-right:
\newcommand{\Bnotefmt}{%
  \renewcommand*{\differentlines}[2]{\linesfmt{##1\textendash##2}}%
  \renewcommand{\lemmafmt}[1]{##1~]\enskip}%
}

%% In our example we define another level of notes, but this time as regular footnotes
%% here keyed by fnsymbols. This customization is independant of ednotes, 
%% and relates to manyfoot directly:
\SetFootnoteHook{\unsetRL}%--> must appear immediately before \DeclareNewFootnote 
\DeclareNewFootnote[para]{C}[fnsymbol]

%% Finally, we define macros that make typing the critical edition as abstract as possible.
%% First the apparatus for VARIANTS in the Arabic text (level A):
\newcommand{\VAR}[2]{\Anote{\textarab{#1}}{\textarab{#2}}}
%% This is for lemmas of the form \VAR{LEMMA1\<LEMMA2\>LEMMA3}{NOTE} 
%% where LEMMA2 is the portion omitted in the note: this macro is necessary 
%% because \< and \> cannot appear inside the argument of \textarab
\newcommand{\VARX}[4]{\Anote{\textarab{#1}\ \<\textarab{#2}\>\ \textarab{#3}}{\textarab{#4}}}
%% Second the comparisions to the Latin translation (level B):
\newcommand{\VARB}[2]{\Bnote{\textarab{#1}}{#2}}
%% Third, the level for plain footnotes (level C):
\newcommand{\NOTE}[1]{\footnoteC{#1}}
%% The typographical "object" that separates successive variants in one note:
\newcommand{\SEP}{\enskip;\enskip} 
%% The typographical "object" that indicates additions in a MS:
\newcommand{\ADD}{\textLR{\textbf{+}}\,}%
%% The typographical "object" that indicates omissions in a MS:
\newcommand{\OM}{\textLR{\textbf{–}}\,}%
%% The typographical "object" that indicates a correction not reflected in any MS:
\newcommand{\CORR}{\textarab{.s.h.h-}}%
%% The typographical "object" that indicates illegible passages in a MS:
\newcommand{\ILLEG}{\textarab{.gayr maqrU'}}%
%% The typographical "object" that indicates a blank in a MS:
\newcommand{\BLANK}{\textarab{bayA.d}}%
%% The typographical "object" that indicates a marginal passage in a MS:
\newcommand{\MARG}{\textarab{bi-al-hAmi^s}}%
%% The typographical "object" that indicates a supralinear passage in a MS:
\newcommand{\SUPERLIN}{\textarab{ta.ht al-sa.tr}}%
%% The typographical "object" in the main text that indicates a lacuna in all MSS:
\newcommand{\LACUNA}{\textLR{\textlangle\,\dots\textrangle}}
%% The typographical "object" in the main text that indicates a restoration:
\newcommand{\RESTOR}[1]{\}~#1~\{}
%%
%% Convenient abbreviations for referring to individual manuscripts
\newcommand{\msE}{\textbf{أ}}
\newcommand{\msB}{\textbf{ب}}
\newcommand{\msN}{\textbf{ن}}
\newcommand{\msL}{\textbf{ل}}
\newcommand{\msT}{\textbf{ط}}
\newcommand{\msM}{\textbf{م}}

\begin{document}
\thispagestyle{empty}
\begin{center}\Large
	Extract from Abū Maʿšar, \textit{Kitāb al-Milal wa-l-Duwal},
	Part 1, Chapter 1, §\,26\footnoteC{% 
		Text, apparatus and references to the Latin translation taken
		from: \textit{Abū Maʿšar on Historical Astrology. The Book of
		Religions and Dynasties (On the Great Conjuntions)}.  Edited
		and Translated by Keiji Yamamoto and Charles Burnett. 2 vols.
		Leiden: Brill, 2000 (ISBN 90 04 11733 4).
		Vol I pp.~22 and 24 (Arabic text) and Vol II pp.~15–16 (Latin
		translation).  NB: This is for illustrative purposes only!}
\end{center}
\begin{arab}
\resetlinenumber\pagewiselinenumbers
wa-mi_tAl _dalika 'anna \VAR{al-qirAn}{ala_dI kAna \ADD \msN} al-dAll `alY
al-.tUfAn \VAR{kAna}{\msE,\msN \SEP \OM \msB} qabl al-qirAn al-dAll `alY millaT
al-`arab bi-_talA_taT 'AlAf sanaT wa-tis`ami'aT sanaT wa-_tamAn wa-_hamsIn
\VAR{sanaT}{wa-ha_dihi .sUratuhA \ADD \msN}
\VARB{\VAR{3950}{3958: \msE,\msN}}{3958}. 
wa-kAna \VARB{wAlI al-dawr}{prefuit scilicet illi orbi}
\VARB{fI _dalika \VAR{al-waqt}{\msE,\msN \SEP \OM \msB}}{om.} 
zu.hal ma`a burj al-sara.tAn. wa-kAna al-.tUfAn ba`d _dalika 
\VAR{\VAR{bi-mi'atayn}{li-mi'atayn 287: \msE \SEP _tamAnIn sanaT: \msN} 
wa-sab`a \VAR{wa-_tamAnIn}{_tamAnIn: \msN}}{296: \msM}%
\NOTE{T lacks a folio} 
sanaT \VAR{287}{\OM \msE}.  
fa-yakUn \VAR{bayna}{yawm al-jum`aT \ADD \msN} 
'awwal yawm min sanaT al-.tUfAn wa-bayn 'awwal yawm min al-sanaT alatI kAna
fIhA al-qirAn al-dAll `alY millaT al-`arab _talA_taT 'AlAf sanaT
\VAR{wa-sittami'aT sanaT}{\OM \msE} wa-'i.hdY
\VAR{wa-sab`In}{wa-sab`Un: \msE,\msN} 
\VAR{sanaT}{`alY ha_dihi al-.sUraT \ADD \msN} 3671.  
wa-qad _dakara \VARB{\VAR{Ab_tnUs}%
{\msE,\msN \SEP mlbws: \msB \SEP \OM \msT \SEP Asws: \msM}}{Bentemiz} 
wa-.gayruhu 'anna \VAR{bayna}{\OM \msN} 
\VARB{ibtidA'}{om.} _half 'Adam \VAR{.salwAt al-ll_ah `alayhi}{\OM \msN}
wa-bayna laylaT al-jum`aT alatI kAna fIhA al-.tUfAn 
\VAR{'alfayn}{\msE \SEP alfAn: \msB,\msN} \VAR{wa-mi'atayn}{wa-mi'atA sanaT: \msN}
\VAR{wa-sitta}{wa-sittaT: \msE,\msN} \VAR{wa-`i^srIn}{\CORR \SEP wa-`i^srUn:
\msB,\msE,\msN \SEP \OM \msT \SEP 2226: \msM} sanaT 3671 \VAR{wa-^sahr"aN
wA.hid"aN}{\msE \SEP wa-^sahr wA.hid: \msB,\msN} wa-\aemph{kj-} yawm"aN
\VAR{wa-'arba`a}{j-: \msN} sA`At \VAR{2226}{\OM \msE,\msN}. fa-yakUn `alY
ha_dihi al-jihaT mA bayna _halaf 'Adam \VAR{`alayhi al-salAm}{\OM \msN}
wa-bayna \VAR{'awwal}{\OM \msE,\msN} yawm min al-sanaT alatI kAna fIhA al-qirAn
al-dAll `alY millaT al-`arab \VAR{_hamsaT}{\msL,\msN \SEP _hamsa: \msB,\msE}
\VAR{'AlAf}{'alf sanaT: \msN} wa-_tamAnami'aT 
\VAR{\VAR{wa-sab`"aN}{\CORR \SEP wa-sab`a: \msB,\msE \SEP wa-sab`aT: \msN \SEP 
\OM \msT \SEP 5778: \msM} \VAR{wa-tis`In}{\msE \SEP wa-tis`Un: \msB} 
sanaT}{wa-tis`aT wa-sab`Un sanaT 5789: \msN} 
\VAR{wa-^sahr"aN wA.hid"aN}{\msE \SEP wa-^sahr wA.hid: \msB,\msN}
wa-\aemph{kj-} yawm"aN  \VAR{wa-'arba`a}{j-: \msN} sA`At 
\VAR{5897}{\OM \msE,\msN}. 
fa-'i_dA qasamnA al-sinIn alatI \VAR{bayna}{\msE,\msN \SEP Awl \ADD \msB} 
qirAn al-.tUfAn wa-al-qirAn \VAR{al-dAll}{ala_dI dal: \msN} `alY
\VAR{millaT}{\msL,\msE,\msN \SEP al-mlh: \msB} al-`arab `alY 
\VAR{_tala_tami'aT wa-sittIn wa-'a_ha_dnA}{s.s: \msN} li-kull burjiN sanaT
\VAR{wa-ibtada'nA}{wa-ibtadY: \msE,\msN} 
\VAR{bi-al-.tar.h}{al-.tar.h li-isti_hrAj burj al-muntahY: \msN} 
min al-.hamal intahat \VAR{al-sanaT}{\OM \msN} 'ilY \VAR{al-.hUt}{al-jawzA': \msN}.%
\NOTE{Because 3950 = 360 × 10 + 350.}
\VAR{fa-'in}{knA \ADD \msN} qasamnA tilka al-sinIn%
\NOTE{That is, 3950.} 
`alY _tala_tami'aT wa-sittIn wa-'a_ha_dnA li-kulli dawriN burj"aN
\VARB{\VAR{wa-ibtada'nA}{wa-ibtadY: \msE,\msN} 
      \VAR{bi-al-'ilqA'}{bi-al-.tar.h: \msN}}{proiecerimus} 
min al-burj al-mudabbir \VAR{ala_dI}{\OM \msN} \VARB{kAna
\VAR{li-l-dawr}{al-dawr: \msN}}{fuit orbis} fI _dalika 
\VAR{al-waqt}{al-burj: \msN} 
\VARB{\VAR{al-dAll `alY al-.tUfAn}{\OM \msE,\msN}}{om.} ala_dI huwa
\VAR{al-sara.tAn \VAR{intahY}{wa-intihA': \msN}}{al-sara.tAnI intihA' intihA': \msE} 
al-dawr \VAR{fI al-qirAn}{min al-burj: \msN} al-dAll `alY
\VAR{al-millaT}{millaT al-`arab: \msN} 'ilY al-jawzA'. wa-'in 'a`.taynA li-kull
kawkab dawr"aN min al-'adwAr wa-ibtada'nA \VAR{bi-al-'ilqA'}{bi-al-.tar.h: \msN} 
min \VAR{al-kawkab}{al-kawAkib: \msN} 
\VARB{ala_dI li-l-dawr}{cuius est orbis} 
fI al-qirAn al-dAll `alY al-millaT intahY al-`adad 'ilY al-zuhraT.
wa-'in 'alqaynA min .tAli` al-qirAn ala_dI kAna fI ra's al-.hamal li-kull
darajaT sanaT intahY fI al-qirAn al-dAll `alY al-millaT 'ilY \aemph{k} darajaT
min al-.hUt. \VAR{wa-kAna}{fa-kAna: \msE} al-mubtazz `alY al-dawr wa-`alY
\VARX{.sA.hib al-.tAli` wa-.sA.hib}%
{burj al-qirAn al-mirrI_h, wa-kAna al-tadbIr li-l-rub` al-'awwal 
 \VAR{ala_dI huwa}{\msE \SEP \OM \msB,\msN} \aemph{.s} sanaT
  min ibtidA' al-qirAn al-dAll}%
{`alY}{\OM \msN} dawlaT al-`arab. wa-al-rub` al-_tAnI 
\VAR{li-l-^sams wa-al-rub`}{w: \msN} al-_tAli_t li-`u.tArid
\VAR{wa-al-rub`}{w: \msN} al-rAbi` li-zu.hal, wa-_dalika `alY qadr
\VARB{ibtizAzAt}{dominium} al-kawAkib `alY .sA.hib al-dawr wa-`alY al-.tAli`
wa-burj al-qirAn \VAR{wa-mu^sArikatihimA}{\msE \SEP wa-mu^sArikatihA: \msB,\msN} 
li-.sA.hib al-dawr al-'awwal ala_dI huwa al-zuhraT.
\end{arab}
\end{document}
