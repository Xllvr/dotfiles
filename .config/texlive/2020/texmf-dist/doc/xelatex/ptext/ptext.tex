\documentclass[a4paper,12pt]{article}      
\usepackage{fancyvrb}
\usepackage{hyperref}  
\usepackage{xepersian}
\settextfont{XB Niloofar}
\setdigitfont{Yas}
\title{بسته
\textsf{ptext}\RTLthanks{نسخه $1.1$}
}\author{وحید دامن‌افشان%
\LTRthanks{\href{mailto:vdamanafshan@gmail.com}{\texttt{vdamanafshan@kut.ac.ir}}}}
\begin{document}
\baselineskip=.7cm
\maketitle
\begin{abstract}
بسته \textsf{ptext} با الهام از بسته \textsf{lipsum} نوشته شده است و کار آن، تولید $100$ پاراگراف
متن از شاهنامه فردوسی، برای پر کردن صفحات یک نوشتار است.
\end{abstract}
\section{روش به کارگیری}
این بسته را  طبق معمول می‌توان با استفاده از دستور
\verb!\usepackage{...}!
فراخوانی کرد. همانند سایر بسته‌های سازگار شده با زی‌پرشین، بهتر است که این بسته نیز،  قبل از بسته زی‌پرشین، فراخوانی شود.

این بسته، چهار دستور مهم ارایه می‌کند. مهم‌ترین آن‌ها، 
\verb!\ptext!
است. این دستور، به طور پیش‌فرض، پاراگراف $1$ تا $7$ تعریف شده در بسته را در خروجی چاپ 
می‌کند که مقدار آن، در حالت استاندارد، بیشتر از یک 
\verb!a4paper!
است. برای تغییر این مقدار به پاراگراف‌های \texttt{n} تا \texttt{m}   می‌توان از دستور
\LRE{\verb!\setptextdefault{n-m}!}
استفاده کرد. دستور 
\verb!\ptext!
یک آرگومان اختیاری هم دارد که از آن برای تعیین محدوده پاراگراف‌ها استفاده می‌شود. به عنوان مثال، دستور 
\LRE{\verb+\ptext[4-57]+}،
پاراگراف‌های $4$ تا $57$ را در خروجی چاپ می‌کند و یا دستور 
\verb!\ptext[23]!،
پاراگراف $23$ام را چاپ می‌کند.

پاراگراف‌های تولید شده به وسیله
\verb!\ptext!
توسط دستور
\verb!\par!
از هم جدا می‌شوند. به عبارت دیگر، هر پاراگراف با دستور 
\verb!\par!
به پایان می‌رسد. این امر، ممکن است گاهی باعث اثرات تصادفی شود. لذا این بسته، گزینه 
\textsf{nopar}
را ارایه می‌کند. برای این منظور، بسته را باید به صورت 
\begin{Verbatim}
\usepackage[nopar]{ptext}
\end{Verbatim}
فراخوانی کرد.

علاوه بر دستورات گفته شده، نسخه ستاره‌دار دستور
\verb!\ptext!،
یعنی
\verb!\ptext*!
نیز تعریف شده است که همان کار
گزینه
\textsf{nopar}
را انجام می‌دهد. به عبارت دیگر، اگر از گزینه 
\textsf{nopar}
استفاده نشود، دستور
\verb!\ptext*!،
پاراگراف‌ها را بدون دستور
\verb!\par!
چاپ می‌کند.

در صورت نیاز به راهنمایی در مورد   این بسته، می‌توان به 
\href{http://forum.parsilatex.com}{تالار گفتگوی پارسی‌لاتک}%
\LTRfootnote{\href{http://forum.parsilatex.com}{\texttt{http://forum.parsilatex.com}}}
مراجعه کرد.
\section{سپاس‌گزاری}
از آقای وفا خلیقی، نویسنده بسته‌های زی‌پرشین و بی‌دی به خاطر پیشنهادهای ارزنده‌شان
و نیز از سید رضی علوی‌زاده  به خاطر در اختیار گذاشتن 
\href{http://saaghar.pozh.org/}{متن شاهنامه}%
\LTRfootnote{\href{http://saaghar.pozh.org}{\texttt{http://saaghar.pozh.org}}}، 
 سپاس‌گزاری می‌شود. 
\section*{تغییرات بسته}
\begin{enumerate}
\item
$8$ آبان $1391$، انتشار نسخه $1$؛
\item
$9$ اردیبهشت $1392$، رفع باگ مربوط به استفاده از دستور \verb!\ptext! در دستور \verb!\caption! و انتشار نسخه $1.1$.
\end{enumerate}
\end{document}