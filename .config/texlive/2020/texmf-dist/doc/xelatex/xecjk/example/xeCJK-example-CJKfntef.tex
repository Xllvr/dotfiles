%%
%% This is file `xeCJK-example-CJKfntef.tex',
%% generated with the docstrip utility.
%%
%% The original source files were:
%%
%% xeCJK.dtx  (with options: `ex-fntef')
%% 
\documentclass{ctexart}
\usepackage{xcolor}
\usepackage{xeCJKfntef}
\xeCJKDeclareSubCJKBlock{test}{ `殆 , `已 }
\setCJKmainfont[test=SimHei]{SimSun}

\def\showtext{%
  庄子曰:“吾生也有涯,而知也无涯。以有涯随无涯,殆已!已而为知者,殆而已矣!%
  为善无近名,为恶无近刑,缘督以为经,可以保身,可以全生,可以养亲,可以尽年。”}
\newcommand\CJKfntef[1]{%
  \csname#1\endcsname{\showtext}\par
  \csname#1\endcsname*{\showtext}\par
  \CJKunderdot{\showtext\csname#1\endcsname{\showtext}}\par
  \CJKunderdot{\showtext\csname#1\endcsname*{\showtext}}\par
  \csname#1\endcsname{\showtext\CJKunderdot{\showtext}}\par
  \csname#1\endcsname*{\showtext\CJKunderdot{\showtext}}\par\bigskip}

\begin{document}

\section{\textsf{xeCJKfntef} 的\CJKunderdot{简单测试文件}}

\CJKunderline{汉 字}\CJKunderline{加下划线}
\varCJKunderline{汉字}\varCJKunderline{加下划线}
\CJKunderanyline{0.5em}{\sixly \kern-.021em\char58 \kern-.021em}{自定义下划线}
\CJKunderanyline{0.2em}{\kern-.03em\vtop{\kern.2ex\hrule width.2em\kern 0.11em
  \hrule height 0.1em}\kern-.03em}{自定义下划线}

\CJKunderanyline{0.5em}{-}{\showtext}

\CJKunderanysymbol{0.2em}{-}{\showtext}

\CJKunderdot{\showtext}

\hrule depth \dp\strutbox width 0 pt \relax

\begin{CJKfilltwosides}{40mm}
两端分散对齐\\
分散对齐\\
\CJKunderdot{汉字可加点}\\
\CJKunderline{汉字可加下划线}
\end{CJKfilltwosides}

\hrule depth \dp\strutbox width 0 pt \relax

\CJKfntef{CJKunderline}
\CJKfntef{CJKunderwave}
\CJKfntef{CJKunderdblline}
\CJKfntef{CJKsout}
\CJKfntef{CJKxout}

\def\Book{\CJKunderwave-}
\def\Noun{\CJKunderline-}

如上所述,\Book{中國文學史}原是\Noun{復旦大學出版社}出版的,\Book{中國文學史新著}則由\Noun{上海文藝出版社}出版。現承\Noun{復旦大學出版社}\Noun{賀聖遂}先生的盛情和\Noun{上海文藝出版總社}\Noun{楊益萍}、\Noun{何承偉}先生以及有關負責人\Noun{陳鳴華}先生的高誼,此書增訂本改由兩社聯合出版。\Noun{賀聖遂}先生並與\Noun{杜榮根}副社長、\Noun{韓結根}編審共同擔任此書的責任編輯。對上述諸位先生謹在這裡表示衷心的感謝。對\Book{中國文學史新著}原版的責任編輯\Noun{戴俊}先生(現為\Noun{上海三聯書店}副總經理)也在此一併致謝。

\Noun{與文學史的}\Noun{應有任務}\Noun{對文學發展過程}\Noun{的內在聯繫的描述}\Noun{還有很大的距離}\Noun{至於距離的所在}\Noun{本書原已有說明}\Book{此不贅述}\Book{為了對學術負責}\Book{我們決定重寫一部並迅即出版}

\Noun{南朝}\Noun{梁}\Noun{劉勰}\Book{文心雕龍}\Book{養氣}

\Noun{鎌池和馬}\Book{魔法禁書目錄}\Book{某科學的超電磁砲}

\Noun{西尾維新}\Book{化物語}\Book{偽物語}\Book{貓物語}%
\Book{傾物语}\Book{囤物語}\Book{鬼物語}\Book{戀物語}\Book{花物語}

\Noun{岡田麿里}\Book{我們仍未知道那天所看見的花的名字。}\Book{絕園的暴風雨}

\Noun{虚淵玄}\Book{Fate/Zero}\Book{PSYCHO-PASS}\Book{ALDNOAH.ZERO}

\Noun{劉海洋}\Book{\LaTeX 入門}

\def\NAME{%
  \x{御坂美琴}\x{白井黑子}\x{戰場原緋多木}\x{阿良良木歷}%
  \x{宿海仁太}\x{本間芽衣子}\x{瀧川吉野}\x{鎖部葉風}%
  \x{衛宮切嗣}\x{言峰綺禮}\x{常守朱}\x{狡嚙慎也}%
  \x{界塚伊奈帆}\x*{艾瑟依拉姆·沃斯·艾莉歐斯亞}}

\let\x\Noun\NAME

\let\x\Book\NAME

\end{document}
%% 
%%
%% End of file `xeCJK-example-CJKfntef.tex'.
