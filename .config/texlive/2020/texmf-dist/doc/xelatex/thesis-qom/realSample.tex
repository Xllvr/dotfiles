\chapter{یک مثال واقعی }

\section{تانسورها}
بسیاری از تکنیک‌های هندسه به وسیله تانسور بیان می‌شوند به‌عنوان مثال متر ریمانی خودش یک تانسور است و البته تانسورها سراسر هندسه فینسلری را نیز پر کرده‌اند. به همین دلیل در اینجا قسمتی را به آن‌ها  اختصاص داده‌ایم.
\subsection{تانسورها روی یک فضای برداری}
فرض کنیم $V$ یک فضای برداری روی میدان اعداد حقیقی با بعد متناهی باشد. دقت کنید که در سراسر این پایان نامه همه‌ی فضاهای برداری با بعد متناهی هستند مگر اینکه خلافش صریحا ذکر شود و البته همه فضاهای برداری بدون استثناء روی میدان اعداد حقیقی در نظر گرفته می‌شوند. می‌دانیم که  $ V^* $، دوگان فضای $ V $، مجموعه همه یک فرمی‌ها و یا همان تابعک‌های خطی روی $V$  می‌باشد. نگاشت طبیعی 
$ V^* \times V \longrightarrow \mathbb{R} $ را اینگونه نمایش می‌دهیم
\begin{align*}
\left\lbrace
\begin{array}{lr}
V^* \times V \longrightarrow \mathbb{R} & \quad\\
\left( \omega,X\right)  \longmapsto \left \langle \omega,X \right \rangle = \omega(X) & \omega \in V^*,  \quad X \in V
\end{array}
 \right. 
\end{align*}
یک تانسور $ k $ - هموردا روی $ V $ نگاشت چند خطی 
\begin{align*}
F: \underbrace{V\times\cdots\times V}_{\text{بار}\ k} \longrightarrow \mathbb{R}
\end{align*}
 می‌باشد. به طور مشابه یک تانسور $ l $- پادوردا یک نگاشت چند خطی به صورت زیر است: 
\begin{align*}
F:\overbrace{V^*\times\cdots\times V^*}^{\text{بار}\ l}\longrightarrow \mathbb{R}
\end{align*}
اغلب اوقات ما تانسورهایی داریم که ترکیبی از این دو حالت فوق‌اند. یک تانسور $\binom{k}{l} $ که به آن $k$-هموردا و $l$-پادوردا می‌گوییم یک نگاشت چند خطی به فرم 
\begin{align*}
F:\overbrace{V^*\times\cdots\times V^*}^{\text{بار}\ l}%
\underbrace{\times V\times\cdots\times V}_{\text{بار}\ k}\longrightarrow \mathbb{R}
\end{align*}
می‌باشد. در واقع در بسیاری از حالات با نگاشت‌هایی چند خطی سروکار داریم که $l$ تا پارامتر 1-فرمی و $k$ پارامتر برداری دارند ولی ترتیب پارامترهایشان لزوما مثل فوق نیست با این حال به آن‌ها نیز تانسور نوع $\binom{k}{l}$ می‌گوییم.\\
مجموعه همه تانسورهای $k$-هموردا را با $T^k(V)$، فضای همه تانسورهای $ l $ - پادوردا را با $T_l(V)$ و همچنین مجموعه‌ی همه تانسورهای $ \binom{k}{l} $ را  با  $ T^k_l(V) $ نمایش می‌دهیم. تساوی‌های $T^k_\circ(V) = T^k(V)$، $T^\circ_l(V) = T_l(V)$، $T^1(V)=V^*$ و $T_1(V) = V^{**}$ بدیهی هستند. همین طور
 \mbox{قرار‌داد} 
  می‌کنیم که $T^\circ(V) = \mathbb{R}$. 
\subsection{ضرب تانسورها}
ضرب روی تانسورها به روش بسیار طبیعی تعریف می‌شود. اگر $F$ یک تانسور $\binom{k}{l} $ و $G$ یک تانسور $\binom{p}{q} $ باشد $F \otimes G$ به روش زیر تعریف می‌شود
\begin{align*}
&F\otimes G(\omega^1,\cdots,\omega^{l+q},X_1,\cdots,X_{k+p}) =\\ 
&\quad F(\omega^1,\cdots,\omega^l,X_1,\cdots,X_k)G(\omega^1,\cdots,\omega^q,X_1,\cdots,X_p)  
\end{align*}
و روشن است که تانسور بدست آمده یک تانسور $\binom{p+k}{q+l} $ خواهد بود.

\section{منیفلد و کلاف برداری}

\subsection{منیفلد}

فرض کنید $ M $ یک فضای توپولوژیک باشد.  $ M $ را یک منیفلد توپولوژیک $ n $ - بعدی گویند هرگاه در شرایط زیر صدق کند: \\
1. $ M $ یک فضای هاسدورف باشد؛\\
2. $ M $ شمارای نوع دوم باشد؛\\
3. $ M $  موضعا اقلیدسی $ -n $  بعدی باشد.\\
در طول این پایان نامه، منیفلدها هموار، هاوسدورف و شمارای نوع دوم در نظر گرفته می‌شوند.

\begin{definition}
زوج مرتب $(U,\phi)$ را  که $ U $ زیر  مجموعه‌ی بازی در $ M $ و 
$ \phi :U \longrightarrow \tilde{U }$
همئومورفیسمی  از  $ U $  به  زیر مجموعه‌ی  باز  
$  \tilde{U}  $  از  $ \mathbb{R}^n  $ 
باشد یک  کارت  مختصات  روی  منیفلد  $ M $  گویند.
\end{definition}

مختصات موضعی برای  یک کارت $(x,U)$ را به صورت $(x^1,\cdots,x^n)$ و یا $(x^i)$ می‌نویسیم
و هرگاه کارت متناظر $(x,U)$ برای $ TM $ مدنظر باشد $ (x^1,\cdots,x^n,y^1,\cdots,y^n) $ نشان دهنده‌ی 
بردار
 $y^{1}\frac{\partial}{\partial x^{1}}+\cdots+y^{n}\frac{\partial}{\partial x^{n}}$ در نقطه $p$ به مختصات 
موضعی $(x^1,\cdots,x^n)$ خواهد بود. والبته همین‌طور که می‌دانیم $\frac{\partial}{\partial x^i}$ یک بردار در نقطه ی $p$ است که معادل با مشتق‌گیری
\[ X.f =\frac{\partial}{\partial x^i}\bigg|_{x(p)}fox^{-1}\qquad f\in C^\infty(M)\]
در نقطه‌ی $x(p) = (x^1,\cdots,x^n)$ می‌باشد. اغلب (ولی نه همیشه) از $\partial_i$ به جای $ \frac{\partial}{\partial x^i}$ استفاده خواهیم کرد. حال با توجه به آنچه گفتیم و نمادگذاری انیشتینی  می‌توان نوشت 
\[ (x^1,x^2,\cdots,x^n,y^1,y^2,\cdots,y^n) = y^i\partial_i\]
\subsection{کلاف‌های برداری}
هنگامی که فضای مماس تمام نقاط منیفلد را به نقاط متناظرشان ملحق کنیم مجموعه‌ای بدست می‌آوریم که می‌توان هم به‌صورت مجموعه‌ای از فضاهای برداری و هم به‌عنوان یک منیفلد به آن نگاه کرد. چنین ساختارهایی در هندسه (دیفرانسیل) چنان رایجند که نام خاصی به آن‌ها اختصاص یافته است.
یک کلاف برداری (هموار) $k$-بعدی  عبارت است از دو منیفلد $E$ (منیفلد مماس) و $M$ (منیفلد پایه) به‌همراه نگاشت $\pi:E\longrightarrow M$ (نگاشت تصویر) که در شرایط زیر صدق کند\\
1) هر  
$ E_p= \pi^{-1}(p ) $ 
(به آن تار $E$ روی $p$  می‌گوییم) دارای ساختار فضای برداری باشد.\\
2) برای هر $p\in M$ همسایگی $U$ از $p$، دیفئومورفیسم 
$ \varphi:\pi^{-1}(U)\longrightarrow U\times \mathbb{R}^k $
 که به آن ساده‌ساز موضعی می‌گوییم چنان موجود باشد که نمودار زیر جابجایی شود

 $$ \begin{CD}
 \pi^{-1}U@>\varphi>>U\times\mathbf{R}^k\\
 @V\pi VV @VV{\pi}_1 V\\
 U @= U
 \end{CD} $$
 
که در آن $\pi_1$ تصویر روی مولفه‌ی اول است.\\
3) تحدید $\varphi$ به هر تار ( یعنی تابع $\varphi:E_p\longrightarrow \left\lbrace p \right \rbrace\times \mathbb{R}^k$ ) یک 
ایزومتری خطی باشد.  \\                                          
\begin{definition}
فرض  کنیم 
$ (E^*,\pi,M,F) $  یک   کلاف  برداری  باشد و  قرار دهیم:  
$$ E ^*= \cup _{p \in M} E_p ^* $$ 
    در این  صورت $  E^* $   دارای  خاصیت  کلاف  برداری است  که  نگاشت  تصویر  آن 
$$ \pi : u \in  E_p ^* \longrightarrow p \in M $$
  می‌باشد $ E ^* $ را کلاف  دوگان  $ E $   می‌نامیم.
\end{definition}

ما دو کلاف برداری بسیار آشنا را می‌شناسیم: $TM$ و $T^*M$ که تار روی $p$ در آن‌ها به 
ترتیب $ T_pM $ و 
$ T^*_p M = (T_p M)^* $
 می‌باشد.
\begin {definition}(برش)
فرض کنید $ E $ یک  کلاف برداری  روی $ M $   با  نگاشت  $ \pi $  و  $ U \subset M $ 
یک  مجموعه‌ی  باز  در $ M $ باشد.  نگاشت  هموار $ S : U \longrightarrow E $
را  یک  برش  $ E $ روی $ U $ می‌نامیم هرگاه  به ازای  هر  $ p $ عضو  
$ U $، $ S(p) \in {E_p} $ 
. $ S $ را  یک  برش  سراسری  گوییم  هرگاه $ U = M $.
 \\ اگر  $ U \subset M $  باشد، $ S $  را  یک  برش  موضعی  روی  $ U $   می‌نامیم. مجموعه  تمام  برش‌های $ M $ را  با  نماد  $ \Gamma (M , E) $ 
نشان  می‌دهیم. لذا اگر $E = TM$ آنگاه برش مذکور همان میدان برداری خواهد بود.
\end{definition}

\begin{definition}(یکریختی  کلاف  برداری)
فرض کنید 
$ (E,\pi, M) $ و $ ( E^ \prime , \pi ^ \prime , M^ \prime) $
دو کلاف  برداری  باشند.  دو تایی  $ (F ,f ) $ از نگاشت‌های  $ C ^r $، 
$ F :E \longrightarrow E^ \prime $ و  $ f : M \longrightarrow M^ \prime $
 را  یک  یکریختی  کلاف  برداری  می‌نامیم اگر  برای  هر $ m \in M $ نگاشت  خطی 
 $$ F |_{E_m}: E_m \longrightarrow E _{f(m)}^ \prime $$
 یک  یکریختی  و  $ f $  دیفئومورفیسم  باشد. آنگاه  می‌نویسیم
 $ E \cong E^ \prime  $. 
\end{definition}
\begin {definition}
فرض کنیم $ M $ یک منیفلد $ C^{\infty} $  و $ T_{x}M $ فضای مماس در نقطه $ x\in M $  باشد. کلاف مماس بر $ M $ را با عبارت $ TM:=\bigcup_{x\in M} T_{x}M $  نمایش می‌دهیم. نگاشت تصویر طبیعی عبارت است از:
\begin{equation*}
\pi:TM \rightarrow M
\end{equation*}
که در آن $ \pi(x,y)=x $. فضای دوگان $ T_{x}M $ توسط $ T^*_{x}M $ نمایش داده شده و فضای کتانژانت در نقطه $ x \in M $ نامیده می‌شود. کلاف $ T^*M:=\bigcup_{ x\in M}T^*_{x}M $ کلاف کتانژانت نام دارد.\\
\end{definition}

\subsection{کلاف برگشت }
فرض  کنیم 
$ (E,\pi,N ) $
یک  کلاف  برداری و 
$ f: M \longrightarrow N $
نگاشتی  هموار  بین دو منیفلد
$ M $ 
و 
$ N $
باشد. با استفاده از 
$ f $
 می‌توان یک  کلاف  برداری روی 
$ M $
با همان تار کلاف  برداری 
$ (E,\pi, N) $
تعریف نمود. این  ساختار جدید را کلاف برگشت  می‌نامیم و به صورت زیر تعریف  می‌کنیم.
\begin{definition}[کلاف فیبره برگشت]
فرض  می‌کنیم $ (E,\pi,N,F) $ یک کلاف تاری و $$ f:M \rightarrow N $$  یک نگاشت هموار باشد. کلاف برگشت $E $ را با نماد $ f^*E $ نشان داده و آن را به‌صورت زیر تعریف  می‌کنیم:

$$ f^*E=\{(x,v) \in M \times E \rvert f(x)=\pi (v)\} $$
که در آن نمودار زیر جابجایی است:
	$$\begin{CD}
	f^*E@>>>E\\
	@VVV @VV\pi V\\
	M @>>f>N
	\end{CD}$$
\end{definition}

\subsection{زیر کلاف}
اگر 
$ (E_1, \pi_1, M_1, F_1) $
 و 
 $ (E_2, \pi_2, M_2, F_2) $
 دو  کلاف  برداری  باشند  به‌طوری   که  $ E_2$ و $ M_2 $ و  $ F_2 $
  به‌ترتیب  زیر منیفلدهای  $ E_1 $ و  $ M_1 $ و  $ F_1 $  باشند آنگاه  $ E_2 $ را  زیر  کلاف 
   \LTRfootnote{subbundel}
   $ E _1 $
    می‌گویند،  اگر  برای  هر  کارت  کلاف  تاری   $ ( \pi_2, \psi_ 2) $
   روی  $ U_2 \subset M_2 $  در کلاف 
  $ E_2 $، 
   یک  همسایگی  $ U _1 \subset M_1 $ و  یک  کارت  کلاف  تاری   $ (\pi_1, \psi _1) $ 
   روی  $ U _1 $  در  کلاف  
  $  E _ 1 $
     وجود  داشته  باشد  به‌طوری  که  
  $$ \Big( \pi_1, \psi _1|_{\pi _1^{-1}(U_1 \cap U_2)} \Big) = \Big( \pi_2, \psi _2|_{\pi _2^{-1}(U_1 \cap U_2)} \Big) $$
\subsection{زیر کلاف‌های افقی و قائم}
%---------------------------------------------------------------------------
\begin{definition}[زیر کلاف قائم]
فرض  کنیم  $  M $ یک  منیفلد،  $ (E,p,M) $ یک  کلاف  برداری  از  رتبه  $ r $  روی  $ M $  و $ p:E \rightarrow M $  نگاشت  تصویر  کلاف  برداری  باشند.  اگر  $ e \in E $  برداری  دلخواه  باشد، در  این  صورت  قرار   می‌دهیم: 
\begin{equation*}
v_eE := ker(p_*)_e, \qquad vE= \bigcup_{e \in E} v_eE
\end{equation*}
تعریف  می‌کنیم  $ p:w \in v_eE \rightarrow e \in TM $.  در  این  صورت  $ (vTM,p,TM) $  یک  کلاف برداری از  رتبه  $ r $ است، این  کلاف  برداری  را  کلاف  قائم  می‌گوییم.
\end{definition}
\begin{definition}(زیر کلاف افقی)
فرض  کنیم $ (E,\pi,M) $  یک  کلاف  برداری  روی  $ M $  باشد. هرگاه  بتوان  $ TE $ را  به  صورت  جمع مستقیم زیر نوشت:
\begin{equation*}
TE=VE \oplus HE
\end{equation*}
که  در  آن  $ VE $ زیر کلاف عمودی است،  در  این  صورت  زیر کلاف  $ HE $  را  زیر کلاف  افقی  می‌نامیم.  با  توجه  به  جمع  مستقیم  فوق،  زیر کلاف  افقی  لزوما  یکتا  نیست.
\end{definition}

