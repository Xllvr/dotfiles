\chapter{نگارش صحیح}
%\thispagestyle{empty}
\section{مقدمه}

فصل مقدمه یک پایان‌نامه/رساله، با بیان نیاز موضوع، تعریف مسئله و اهمیت آن در یک یا چند بند (پاراگراف) آغاز می‌شود\footnote{شروع مقدمه نباید 
چنان طولانی باشد كه هدف اصلی را تحت‌ تاثیر قرار دهد.}  و با مرور پیشینه موضوع (سابقه کارهای انجام‌شده پیشین که ارتباط مستقیمی با مسئله مورد بررسی دارند) 
ادامه می‌یابد. سپس در یک یا دو بند توضیح داده می‌شود كه در این پایان‌نامه/رساله، چه دیدگاه یا راهكار جدیدی نسبت به مسئله (موضوع) مورد بررسی وجود دارد. 
به‌عبارت دیگر نوآوری‌ها به‌صورت کاملاً شفاف و صریح بیان می‌شود. در ادامه ممکن است به نتایج بدست‌آمده نیز به‌طور مختصر و کلی اشاره ‌شود. 
در آخرین بند از مقدمه به محتوای فصل‌های بعدی پایان‌نامه/رساله به‌اختصار اشاره می‌شود.

نگارش صحیح یک پایان‌نامه/رساله در فهم آسان آن بسیار موثر است. در این فصل مهمترین قواعد نگارشی که باید مورد توجه جدی نگارنده قرار گیرد، 
به اختصار بیان می‌شود. این قواعد را می‌توان در محورهای اصلی زیر دسته‌بندی کرد:

\begin{enumerate*}[label=\textbf{\alph*)}]
\item
فارسی نویسی
\item
رعایت املای صحیح 
\item
رعایت قواعد نشانه‌گذاری
\end{enumerate*}
\section{فارسی نویسی}
در حد امکان سعی کنید به جای کلمات غیر‌فارسی از معادل فارسی آن‌ها استفاده کنید، به‌ویژه در مواردی که معادل فارسی مصطلح و رایج است‌.‌ به‌طور مثال استفاده از کلمه «لذا» به‌جای «برای همین» یا «به‌همین دلیل» توجیهی ندارد‌. همچنین کلمه «پردازش» زیباتر از «پروسس» و معادل فارسی «ریز‌پردازنده» مناسب‌تر از «میکروپروسسور» است‌.‌ در این‌گونه موارد چنانچه احتمال عدم آشنایی خواننده با معادل فارسی وجود دارد، یا اصطلاح غیر‌فارسی معمول‌تر است، در اولین ظهور کلمه فارسی، اصل غیر‌فارسی آن به‌صورت پاورقی آورده شود‌.‌ اگر به‌ناچار باید کلمات انگلیسی در لابه‌لای جملات گنجانده شوند، از هر طرف یک فاصله بین آن‌ها و کلمات فارسی پیش و پس از آن‌ها در‌نظر گرفته شود‌.‌ چنانچه در پایان‌نامه/رساله از مختصر‌نویسی استفاده شود، لازم است در اولین استفاده، تفصیل آن در پاورقی آورده شود‌.‌ 

\section{رعایت املای صحیح }
رعایت املای صحیح فارسی به مطالعه و درک راحت‌تر کمک می‌کند. همچنین در نوشته‌های فارسی باید در حد امکان از همزه « ء، أ، ؤ، ة، إ، ئ» استفاده نشود‌.‌ به‌عنوان مثال «اجزاء هواپیما» و «آئین نگارش» ناصحیح، اما «اجزای هواپیما» و «آیین نگارش» صحیح هستند.‌
\section{رعایت قواعد نشانه‌گذاری}
منظور از نشانه‌گذاری به‌کار‌بردن علامت‌ها و نشانه‌هایی است که خواندن و فهم درست یک جمله را ممکن و آسان می‌کند. در ادامه نشانه‌های معمول و متداول در زبان فارسی و موارد کاربرد آن‌ها به اختصار معرفی می‌شوند.
\subsection{ویرگول}
ویرگول نشانه ضرورت یک مکث کوتاه است و در موارد زیر به‌کار می‌رود:
\begin{itemize}
\item
در میان دو کلمه که احتمال داده شود خواننده آن‌ها را با کسره اضافه بخواند، یا نبودن ویرگول موجب بروز اشتباه در خواندن جمله شود.
\item
در موردی که کلمه یا عبارتی به‌‌‌‌عنوان توضیح، در ضمن یک جمله آورده شود. مثلاً برای کنترل وضعیت فضاپیماها، به‌دلیل آن‌که در خارج از جو هستند، نمی‌توان از بالک‌های آیرودینامیکی استفاده کرد.
\item
جدا‌کردن بخش‌های مختلف یک نشانی یا یک مرجع
\item
موارد دیگر از این قبیل
\end{itemize}
پیش از ویرگول نباید فاصله گذاشته شود و پس از آن یک فاصله لازم است و بیشتر از آن صحیح نیست.
\subsection{نقطه}
نقطه نشانه پایان یک جمله است. پیش از نقطه نباید فاصله گذاشته شود و پس از آن یک فاصله لازم است و بیشتر از آن صحیح نیست.
\subsection{دونقطه}
موارد کاربرد دونقطه عبارتند از:
\begin{itemize}
\item
پیش از نقل قول مستقیم
\item
پیش از بیان تفصیل مطلبی که به اجمال به آن اشاره شده‌است.
\item
پس از واژه‌ای که معنی آن در برابرش آورده و نوشته می‌شود.
\item
پس از کلمات تفسیر‌کننده از قبیل «یعنی» و ...
\end{itemize}
پیش از دونقطه نباید فاصله گذاشته شود و پس از آن یک فاصله لازم است و بیشتر از آن صحیح نیست.
\subsection{گیومه}
موارد کاربرد گیومه عبارتند از:
\begin{itemize}
\item
وقتی که عین گفته یا نوشته کسی را در ضمن نوشته و مطلب خود می‌آوریم. 
\item
در آغاز و پایان کلمات و اصطلاحات علمی و یا هر کلمه و عبارتی که باید به‌صورت ممتاز از قسمت‌های دیگر نشان داده شود.
\item
در ذکر عنوان مقاله‌ها، رساله‌ها، اشعار، روزنامه‌ها و ...
\end{itemize}
\subsection{نشانه پرسشی}
پیش از «؟» نباید فاصله گذاشته شود و پس از آن یک فاصله لازم است و بیشتر از آن صحیح نیست.
\subsection{خط تیره}
موارد کاربرد خط تیره عبارتند از:
\begin{itemize}
\item
جدا‌کردن عبارت‌های توضیحی، بدل، عطف بیان و ...
\item
به‌جای حرف اضافه «تا» و «به» بین تاریخ‌ها، اعداد و کلمات
\end{itemize}
\subsection{پرانتز}
موارد کاربرد پرانتز عبارتند از:
\begin{itemize}
\item
به‌معنی «یا» و «یعنی» و وقتی که یک کلمه یا عبارت را برای توضیح بیشتر کلام بیاورند.
\item
وقتی که نویسنده بخواهد آگاهی‌های بیشتر (اطلاعات تکمیلی) به خواننده عرضه کند.
\item
برای ذکر مرجع در پایان مثال‌ها و شواهد.
\end{itemize}
نکته: بین کلمه یا عبارت داخل پرانتز و پرانتز باز و بسته نباید فاصله وجود داشته باشد.
\section{جدا یا سرهم نوشتن برخی کلمات}
تقریباً تمامی کلمات مرکب در زبان فارسی باید از هم جدا نوشته شوند؛ به استثنای صفات فاعلی مانند «عملگر»، «باغبان» و یا «دانشمند» و کلماتی نظیر «اینکه»، \ldots. 
در ادامه به نمونه‌هایی از مواردی که باید اجزای یک کلمه جدا، اما بدون فاصله نوشته شوند، اشاره می‌شود‌:
\begin{enumerate}
\item
در افعال مضارع و ماضی استمراری که با «می» شروع می‌شوند، لازم است که در عین جدا نوشتن، «می» از بخش بعدی فعل جدا نیافتد‌.‌ برای این منظور باید از «فاصله متصل» استفاده و «می» در اول فعل با \lr{SS}%
\footnote{\lr{Shift+Ctrl+@}؛ البته اگر از سیستم عامل لینوکس/مک و یا صفحه کلید استاندارد پارسی در ویندوز استفاده می‌کنید 
    می‌توانید از  \lr{Shift+Space} استفاده نمایید.} 
 از آن جدا شود.‌ به‌طور مثال «می‌شود» به‌جای «می شود». 
\item
	«ها»ی جمع باید از کلمه جمع بسته‌شده جدا نوشته شود. این امر در مورد کلمات غیر‌فارسی که وارد زبان فارسی شده‌اند و با حرف «ها» جمع بسته می‌شوند، مانند «کانال‌ها» یا «فرمول‌ها» مورد تاکید است.
\item
	حروف اضافه مانند «به» وقتی به‌صورت ترکیب ثابت همراه کلمه پس از خود آورده می‌شوند، بهتر است با \lr{SS} از آن جدا شوند‌.‌ مانند «به‌صورت»، «به‌عنوان» و «به‌‌‌لحاظ»‌.‌ لازم به ذکر است هنگامی که حرف اضافه «به» با کلمه پس از خود معنای قیدی داشته باشد، مثل «بشدت» یا «بسادگی»، بهتر است که به‌صورت چسبیده نوشته شود‌.
\item
	کلمات فارسی نباید با قواعد عربی جمع بسته شوند؛ پس «پیشنهادها» صحیح و «پیشنهادات» اشتباه است‌.‌
\item
	اسم‌ها و صفت‌های دو‌قسمتی مثل «خط‌چین» و «نوشته‌شده» با \lr{SS} از هم جدا می‌شود‌.‌
\item
	شناسه‌ها با \lr{SS} از کلمه اصلی جدا می‌شود‌.‌ مثل «شده‌اند»‌ و «شده‌است». 
\item
	‌ «است» هنگامی که نقش شناسه را داشته باشد توسط \lr{SS} از قسمت اصلی جدا می‌شود‌.‌ مانند «گفته‌است»‌.
\item
	بند پیشین نباید باعث افراط در استفاده از فاصله متصل شود. مثلاً عبارت «نوشته~می‌شود‌» صحیح و عبارت «نوشته‌می‌شود» ناصحیح است. 
\item
	فعل‌های دو‌کلمه‌ای که معنای اجزای آن‌ها کاملاً با معنای کل متفاوت است، بهتر است که با \lr{SS} از هم جدا ‌شوند‌.‌
\item
	کلمات مرکب مثل کلمه «دوکلمه‌ای» در عبارت «فعل‌های دوکلمه‌ای» و \linebreak  «یادداشت‌برداری».
\item
	مصدرهای دو قسمتی با \lr{SS} از هم جدا می‌شوند‌.‌ مثل «ذوب‌کردن» و «واردکردن»‌.
\item
	 صفات تفضیلی مثل « آسان‌تر».
\end{enumerate}
