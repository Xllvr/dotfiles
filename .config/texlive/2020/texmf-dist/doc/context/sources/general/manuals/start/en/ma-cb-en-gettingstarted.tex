\startcomponent ma-cb-en-gettingstarted

\enablemode[**en-us]

\project ma-cb

\startchapter[title=How to process a file]

\index{input file+processing}
\index[pdffile]{\type{pdf}--file}

In this chapter we assume that you have installed and initiated \CONTEXT\ \MKIV\
correctly so that you can run it from the commandline in your working directory.
You can find the \CONTEXT\ installation procedure on the \goto {\CONTEXTWIKI}
[ url (http://wiki.contextgarden.net/ConTeXt_Standalone#Windows) ].

If you want to process a \CONTEXT\ input file, you should type at the command line
prompt:

\starttyping
context myfile.tex
\stoptyping

the extension \type{.tex} is not needed. See \in{appendices}[contextcommand] and
\in[runtimefiles] for more information on the \type{context} command.

After pressing \Enter\ processing will be started. \CONTEXT\ will show processing
information on your screen. During the processing of your input file \CONTEXT\
will also inform you of what it is doing with your document. For example it will
show page numbers and information about processing steps. Further more it gives
warnings. These are of a typographical order and tells you when line breaking is
not successful. All information on processing is stored in a \type{log} file that
can be used for reviewing warnings and errors and the respective line numbers
where they occur in your file.

If processing is succesful the command line prompt will return and \CONTEXT\ will
produce the file \type{myfile.pdf}. The abbreviation \PDF\ stands for Portable
Document Format. This is a platform independent format for printing and viewing
with \READER.

When you use a configurable text editor you can also run \CONTEXT\ from that editor.
More information on that topic can be found \in{appendix}[texteditor].

\stopchapter

\stopcomponent
