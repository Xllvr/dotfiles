% language=uk

% naming-nature.jog

\startcomponent musings-children

\environment musings-style

\definedescription
  [presomething]
  [headstyle=\bold,
   alternative=hanging,
   width=fit,
   hang=1]

\startchapter[title={Children of \TEX}]

\startsection[title={The theme}]

Nearly always \TEX\ conferences carry a theme. As there have been many
conferences the organizers have run out of themes involving fonts, macros and
typesetting and are now cooking up more fuzzy ones. Take the Bacho\TUG\ 2017
theme:

\startnarrower[left,8*right] \startpacked
\startpresomething {Premises}
    The starting point, what we have, what do we use, what has been achieved?
\stoppresomething
\startpresomething {Predilections}
    How do we act now, how do we want to act, what is important to us and what do
    we miss?
\stoppresomething
\startpresomething {Predictions}
    What is the future of \TEX, what we'd like to achieve and can we influence
    it?
\stoppresomething
\stoppacked \stopnarrower

My first impression with these three P words was: what do they mean? Followed by
the thought: this is no longer a place to take kids to. But the Internet gives
access to the Cambridge Dictionary, so instead of running to the dusty meter of
dictionaries somewhere else in my place, I made sure that I googled the most
recent definitions:

\startnarrower[left] \startpacked
\startpresomething {premise}
    an idea or theory on which a statement or action is based
\stoppresomething
\startpresomething {predilection}
    if someone has a predilection for something, they like it a lot
\stoppresomething
\startpresomething {prediction}
    a statement about what you think will happen in the future
\stoppresomething
\stoppacked \stopnarrower

I won't try to relate these two sets of definitions but several words stand out
in the second set: idea, theory, action, like, statement and future. Now, as a
preparation for the usual sobering thoughts that Jerzy, Volker and I have when we
staring into a Bacho\TEX\ campfire I decided to wrap up some ideas around these
themes and words. The books that I will mention are just a selection of what you
can find distributed around my place. This is not some systematic research but
just the result of a few weeks making a couple of notes while pondering about
this conference.

\stopsection

\startsection[title=Introduction]

One cannot write the amount of \TEX\ macros that I've written without also liking
books. If you look at my bookshelves the topics are somewhat spread over the
possible spectrum of topics: history, biology, astronomy, paleontology, general
science but surprisingly little math. There are a bunch of typography|-|related
books but only some have been read: it's the visuals that matter most and as
there are no real developments I haven't bought new ones in over a decade,
although I do buy books that look nice for our office display but the content
should be interesting too. Of course I do have a couple of books about computer
(related) science and technology but only a few are worth a second look.
Sometimes I bought computer books expecting to use them (in some project) but I
must admit that most have not been read and many will soon end up in the paper
bin (some already went that way). I'll make an exception for Knuth, Wirth and a
few other fundamental ones that I (want to) read. And, I need to catch up on deep
learning, so that might need a book.

My colleagues and I have many discussions, especially about what we read, and
after a few decades one starts seeing patterns. Therefore the last few years it
was a pleasant surprise for me to run into books and lectures that nicely
summarize what one has noticed and discussed in a consistent way. My memory is
not that good, but good enough to let some bells ring.

\startplacefigure[location=top]
    \startcombination[nx=4,ny=1,width=\textwidth,distance=0pt]
        {\externalfigure[covers/sapiens.jpg]          [height=5cm]} {history}
        {\externalfigure[covers/homo-deus.jpg]        [height=5cm]} {futurology}
        {\externalfigure[covers/children-of-time.jpg] [height=5cm]} {science fiction}
        {\externalfigure[covers/superintelligence.jpg][height=5cm]} {informatics}
    \stopcombination
\stopplacefigure

The first book that gave me this \quotation {finally a perfect summary of
historic developments} feeling is \quotation{Sapiens} by Yuval Noah Harari. The
author summarizes human history from a broad perspective where modern views on
psychology, anthropology and technical developments are integrated. It's a follow
up on a history writing trend started by Jared Diamond. The follow up \quotation
{Homo Deus} looks ahead and is just as well written. It also integrates ideas
from other fields, for instance those related to development of artificial
intelligence (Dennett, Bostrom, etc.).

Another inspiration for this talk and article is the 50 hour lecture series on
behavioral biology by Robert Sapolsky of Stanford University, brought to my
attention by my nephew Bram who visited a few \TEX\ conferences with me and who
is now also forced to use \TEX\ for assignments and reports. (How come
self|-|published books used at universities often look so bad?)

The title of this talk is inspired by the book \quotation {Children of Time} by
Adrian Tchaikovsky that I read recently. There are science fiction writers who
focus on long term science and technology, such as some of Alastair Reynolds,
while others follow up on recent development in all kind of sciences. One can
recognize aspects of \quotation {Superintelligence} by Bostrom in Neal Asher's
books, insights in psychology in the older Greg Bear books, while in the
mentioned \quotation {Children of Time} (socio)biological insights dominate. The
main thread in that book is the development of intelligence, social behaviour,
language, script and cooperation in a species quite different from us: spiders.
It definitely avoids the anthropocentric focus that we normally have.

So how does this relate to the themes of the Bacho\TEX\ conference? I will pick
out some ways to approach them using ideas from the kind of resources mentioned
above. I could probably go on and on for pages because once you start relating
what you read and hear to this \TEX\ ecosystem and community, there is no end.
So, consider this a snapshot, that somehow relates to the themes:

\startnarrower[left,8*right] \startpacked
\startpresomething {premise}
    Let's look at what the live sciences have to say about \TEX\ and friends and
    let's hope that I don't offend the reader and the field.
\stoppresomething
\startpresomething {predilection}
    Let's figure out what brings us here to this place deeply hidden in the woods,
    a secret gathering of the \TEX\ sect.
\stoppresomething
\startpresomething {prediction}
    Let's see if the brains present here can predict the future because after
    all, according to Dennett, that is what brains are for.
\stoppresomething
\stoppacked \stopnarrower

At school I was already intrigued by patterns in history: a cyclic, spiral and
sinusoid social evolution instead of a pure linear sequence of events. It became
my first typeset|-|by|-|typewriter document: Is history an exact science? Next I
will use and abuse patterns and ideas to describe the \TEX\ world, not wearing a
layman's mathematical glasses, but more from the perspective of live sciences,
where chaos dominates.

\stopsection

\startsection[title={The larger picture}]

History of mankind can be roughly summarized as follows. For a really long time
we were hunters but at some point (10K years ago) became farmers. As a result we
could live in larger groups and still feed them. The growing complexity of
society triggered rules and religion as instruments for stability and
organization (I use the term religion in its broadest sense here). For quite a
while cultures came and went, and climate changes are among the reasons.

After the industrial revolution new religions were invented (social, economic and
national liberalism) and we're now getting dataism (search for Harari on youtube
for a better summary). Some pretty great minds seem to agree that we're heading
to a time when humans as we are will be outdated. Massive automation, interaction
between the self and computer driven ecosystems, lack of jobs and purpose,
messing around with our genome. Some countries and cultures still have to catch
up on the industrial revolution, if they manage at all, and maybe we ourselves
will be just as behind reality soon. Just ask yourself: did you manage to catch
up? Is \TEX\ a stone age tool or a revolutionary turning point?

A few decades ago a trip to Bacho\TEX\ took more than a day. Now you drive there
in just over half a day. There was a time that it took weeks: preparation,
changing horses, avoiding bad roads. Not only your own man|-|hours were involved.
It became easier later (my first trip took only 24 hours) and recently it turned
into a piece of cake: you don't pick up maps but start your device; you don't
need a travel agent but use the Internet; there are no border patrols, you can
just drive on. (Okay, maybe some day soon border patrols at the Polish border
show up again, just like road tax police in Germany, but that might be a
temporary glitch.)

Life gets easier and jobs get lost. Taxi and truck drivers, travel agents, and
cashiers become as obsolete as agricultural workers before. Next in line are
doctors, lawyers, typesetters, printers, and all those who think they're safe.
Well, how many people were needed 400 years ago to produce the proceedings of a
conference like this in a few days' time span? Why read the introduction of a
book or a review when you can just listen to the author's summary on the web? How
many conferences still make proceedings (or go for videos instead), will we
actually need editors and typesetters in the future? How much easier has it
become to design a font, including variants? What stories can designers tell in
the future when programs do the lot? The narrower your speciality is, the worse
are your changes; hopefully the people present at this conference operate on a
broader spectrum. It's a snapshot. I will show some book covers as reference but
am aware that years ago or ahead the selection could have been different.

\stopsection

\startsection[title=Words]

Words (whatever they represent) found a perfect spot to survive: our minds. Then
they made it from speech (and imagination) into writing: carved in stone, wood,
lead. At some point they managed to travel over wires but no matter what
happened, they are still around. Typesetting as visualization is also still
surrounding us so that might give us a starting point for ensuring a future for
\TEX\ to work on, because \TEX\ is all about words. There is a lot we don't see;
imagine if our eyes had microscopic qualities. What if we could hear beyond
20KHz. Imagine we could see infrared. How is that with words. What tools, similar
in impact as \TEX, can evolve once we figure that out. What if we get access to
the areas of our brain that hold information? We went from print to screen and
\TEX\ could cope with that. Can it cope with what comes next?

The first printing press replaced literal copying by hand. Later we got these
linotype|-|like machines but apart from a few left, these are already thrown out
of windows (as we saw in a movie a few Bacho\TeX's ago). Photo|-|typesetting has
been replaced too and because a traditional centuries old printing press is a
nice to see item, these probably ring more bells than that gray metal closed box
typesetters. Organizers of \TEX\ conferences love to bring the audience to old
printing workshops and museums. At some point computers got used for typesetting
and in that arena \TEX\ found its place. These gray closed boxes are way less
interesting than something mechanical that at least invites us to touch it. How
excited can one be about a stack of \TEX\,Live \DVD{}s?

\stopsection

\startsection[title=Remembering]

Two times I visited the part of the science museum in London with young family
members: distracted by constantly swiping their small powerful devices, they
didn't have the least interest in the exhibited computer related items, let alone
the fact that the couch they were sitting on was a Cray mainframe. Later on,
climbing on some old monument or an old cannon seemed more fun. So, in a few
decades folks will still look at wooden printing presses but quickly walk through
the part of an exhibition where the tools that we use are shown. We need to find
ways to look interesting. But don't think we're unique: how many kids find
graphical trend|-|setting games like Myst and Riven still interesting? On the
other hand a couple of month ago a bunch of nieces and nephews had a lot of fun
with an old Atari console running low|-|res bitmap games. Maybe there is hope for
good old \TEX.

If indeed we're heading to a radically different society one can argue if this
whole discussion makes sense. When the steam engine showed up, the metaphor for
what went on in our heads was that technology, It's a popular example of speakers
on this topic: \quotation {venting off steam}. When electricity and radio came
around metaphors like \quotation {being on the same wavelength} showed up. A few
decades ago the computer replaced that model although in the meantime the model
is more neurobiological: we're a hormone and neurotransmitter driven computer. We
don't have memory the way computers do.

How relevant will page breaks, paragraph and line breaks be in the future? Just
like \quotation {venting off steam} may make no sense to the youth, asking a
typesetter to \quotation {give me a break} might not make much sense soon.
However, when discussing automated typesetting the question \quotation {are we on
the same page} still has relevance.

Typesetting with a computer might seem like the ultimate solution but it's
actually rather dumb when we consider truly intelligent systems. On the large
scale of history and developments what we do might get quite unnoticed. Say that
mankind survives the next few hundred years one way or the other. Science fiction
novels by Jack McDevitt have an interesting perspective of rather normal humans
millennia ahead of us who look back on these times in the same way as we look
back now. Nothing fundamental changed in the way we run society. Nearly nothing
from the past is left over and apart from being ruled by \AI{}s people still do
sort of what they do now. \TEX ? What is that? Well, there once was this great
computer scientist Knuth (in the remembered row of names like Aristotle |<|I just
started reading \quotation {The Lagoon} by Armand Leroi|>| Newton, Einstein, his
will show up) who had a group of followers that used a program that he seems to
have written. And even that is unlikely to be remembered, unless maybe user
groups manage to organize an archive and pass that on. Maybe the fact that \TEX\
was one of the first large scale open source programs, of which someone can study
the history, makes it a survivor. The first program that was properly documented
in detail! But then we need to make sure that it gets known and persists.

\startsection[title=Automation]

In a recent interview Daniel Dennett explains that his view of the mind as a big
neural network, one that can be simulated in software on silicon, is a bit too
simplistic. He wonders if we shouldn't more tend to think of a network of
(selfish) neurons that group together in tasks and then compete with each other,
if only because they want to have something to do.

Maybe attempts to catch the creative mindset and working of a typesetter in
algorithms is futile. What actually is great typography or good typesetting?
Recently I took a look at my bookshelf wondering what to get rid of \emdash\
better do that now than when I'm too old to carry the crap down (crap being
defined as uninteresting content or bad looking). I was surprised about the
on|-|the|-|average bad quality of the typesetting and print. It's also not really
getting better. One just gets accustomed to what is the norm at a certain point.
Whenever they change the layout and look and feel of the newspaper I read the
arguments are readability and ease of access. Well, I never had such a hard time
reading my paper as today (with my old eyes).

Are we, like Dennett, willing to discard old views on our tools and models? When
my first computer was a \RCA\ 1802 based kit, that had 256 bytes of memory. My
current laptop (from 2013) is a Dell Precision workstation with an extreme quad
core processor and 16 GB of memory and ssd storage. Before I arrived there I
worked with \DECTEN, \VAX\ and the whole range of Intel \CPU{}s. So if you really
want to compare a brain with a computer, take your choice.

I started with \TEX\ on a 4 MHz desk top with 640 MB memory and a 10 MB hard
disk. Running \CONTEXT\ \MKIV\ with \LUATEX\ on such a machine is no option at
all, but I still carry the burden of trying to write efficient code (which is
still somewhat reflected in the code that makes up \CONTEXT). In the decades that
we have been using \TEX\ we had to adapt! Demands changed, possibilities changed,
technologies changed. And they keep changing. How many successive changes can a
\TEX\ user handle? Sometimes, when I look and listen I wonder.

\startplacefigure[location=top]
    \startcombination[nx=4,ny=1,width=\textwidth,distance=0pt]
        {\externalfigure[covers/the-mind-in-the-cave.jpg]         [height=5cm]} {paleontology}
        {\externalfigure[covers/the-ancestors-tale.jpg]           [height=5cm]} {evolutionary biology}
        {\externalfigure[covers/the-good-book-of-human-nature.jpg][height=5cm]} {anthropology}
        {\externalfigure[covers/chaos-and-harmony.jpg]            [height=5cm]} {physics}
    \stopcombination
\stopplacefigure

If you look back, that is, if you read about the tens of thousands of years that
it took humans to evolve (\quotation {The mind in the cave} by Lewis|-|Williams
is a good exercise) you realize even more in what a fast|-|paced time we live and
that we're witnessing transitions of another magnitude.

In the evolution of species some tools were invented multiple times, like eyes.
You see the same in our \TEX\ world: multiple (sub)macro packages, different font
technologies, the same solutions but with an alternative approach. Some
disappear, some stay around. Just like different circumstances demand different
solutions in nature, so do different situations in typesetting, for instance
different table rendering solutions. Sometime I get the feeling that we focus too
much on getting rid of all but one solution while more natural would be to accept
diversity, like bio|-|diversity is accepted. Transitions nowadays happen faster
but the question is if, like aeons before, we (have to) let them fade away. When
evolution is discussed the terms \quote {random}, \quote {selection}, \quote
{fit}, and so on are used. This probably also applies to typography: at some
point a font can be used a lot, but in the end the best readable and most
attractive one will survive. Newspapers are printed in many copies, but rare
beautiful books hold value. Of course, just like in nature some developments
force the further path of development, we don't suddenly grow more legs or digits
on our hands. The same happens with \TEX\ on a smaller timescale: successors
still have the same core technology, also because if we'd drop it, it would be
something different and then give a reason to reconsider using such technology
(which likely would result in going by another path).

\stopsection

\startsection[title=Quality]

Richard Dawkins \quotation {The Ancestor's Tale} is a non|-|stop read. In a
discussion with Jared Diamond about religion and evolution they ponder this
thread: you holding the hand of your mother who is handing her mother's hand and
so on till at some point fish get into the picture. The question then is, when do
we start calling something human? And a related question is, when does what we
call morality creeps in? Is 50\% neanderthaler human or not?

So, in the history of putting thoughts on paper: where does \TEX\ fit in? When do
we start calling something automated typesetting? When do we decide that we have
quality? Is \TEX\ so much different from its predecessors? And when we see
aspects of \TEX\ (or related font technology) in more modern programs, do we see
points where we cross qualitative or other boundaries? Is a program doing a
better job than a human? Where do we stand? There are fields where there is no
doubt that machines outperform humans. It's probably a bit more difficult in
aesthetic fields except perhaps when we lower the conditions and expectations
(something that happens a lot).

For sure \TEX\ will become obsolete, maybe even faster that we think, but so will
other typesetting technologies. Just look back and have no illusions. Till then
we can have our fun and eventually, when we have more free time than we need, we
might use it out of hobbyism. Maybe \TEX\ will be remembered by probably its most
important side effect: the first large scale open source, the time when users met
over programs, Knuth's disciples gathered in user groups, etc. The tools that we
use are just a step in an evolution. And, as with evolution, most branches are
pruned. So, when in the far future one looks back, will they even notice \TEX ?
The ancestor's tail turns the tree upside down: at the end of the successful
branch one doesn't see the dead ends.

Just a thought: \CD{}s and media servers are recently being replaced (or at least
accompanied) by Long Play records. In the shop where I buy my \CD{}s the space
allocated to records grows at the cost of more modern media. So, maybe at some
point retro|-|typesetting will pop up. Of course it might skip \TEX\ and end up
at woodcutting or printing with lead.

\stopsection

\startsection[title=What mission]

We rely on search engines instead of asking around or browsing libraries. Do
students really still read books and manuals or do they just search and listen to
lectures. Harari claims that instead of teaching kids facts in school we should
just take for granted that they can get all the data they want and that we should
learn them how to deal with data and adapt to what is coming. We take for granted
that small devices with human voices show us the route to drive to Bacho\TEX, for
instance, although by now I can drive it without help. In fact, kids can surprise
you by asking if we're driving in Germany when we are already in Poland.

We accept that computer programs help physicians in analyzing pictures. Some wear
watches that warn them about health issues, and I know a few people who monitor
their sugar levels electronically instead of relying on their own measurements.
We seem to believe and trust the programs. And indeed, we also believe that \TEX\
does the job in the best way possible. How many people really understand the way
\TEX\ works?

We still have mailing lists where we help each other. There are also wikis and
forums like stack exchange. But who says that even a moderate bit of artificial
intelligence doesn't answer questions better. Of course there needs to be input
(manuals, previous answers, etc.) but just like we need fewer people as workforce
soon, the number of experts needed also can be smaller. And we're still talking
about a traditional system like \TEX. Maybe the social experience that we have on
these media will survive somehow, although: how many people are members of
societies, participate in demonstrations, meet weekly in places where ideas get
exchanged, compared to a few decades ago? That being said, I love to watch posts
with beautiful \CONTEXT\ solutions or listen to talks by enthusiastic users who
do things I hadn't expected. I really hope that this property survives, just like
I hope that we will be able to see the difference between a real user's response
and one from an intelligent machine (an unrealistic hope I fear). Satisfaction
wins and just like our neurological subsystems at some point permanently adapt to
thresholds (given that you trigger things often enough), we get accustomed to
what \TEX\ provides and so we stick to it.

\stopsection

\startsection[title={Intelligence versus consciousness}]

Much of what we do is automated. You don't need to think of which leg to move and
what foot to put down when you walk. Reacting to danger also to a large extent is
automated. It doesn't help much to start thinking about how dangerous a lion can
be when it's coming after you, you'd better move fast. Our limbic system is
responsible for such automated behaviour, for instance driven by emotions. The
more difficult tasks and thoughts about them happen in the frontal cortex (sort
of).

\startplacefigure[location=top]
    \startcombination[nx=4,ny=1,width=\textwidth,distance=0pt]
        {\externalfigure[covers/death-by-black-hole.jpg] [height=5cm]} {astronomy}
        {\externalfigure[covers/the-formula.jpg]         [height=5cm]} {informatics}
        {\externalfigure[covers/hals-legacy.jpg]         [height=5cm]} {future science}
        {\externalfigure[covers/lucky-planet.jpg]        [height=5cm]} {earth science}
    \stopcombination
\stopplacefigure

For most users \TEX\ is like the limbic system: there is not much thinking
involved, and the easy solutions are the ones used. Just like hitting a nerve
triggers a chain of reactions, hitting a key eventually produces a typeset
document. Often this is best because the job needs to get done and no one really
cares how it looks; just copy a preamble, key in the text and assume that it
works out well (enough). It is tempting to compare \TEX's penalties, badness and
other parameters with levels of hormones and neurotransmitters. Their function
depends on where they get used and the impact can be accumulated, blocked or
absent. It's all magic, especially when things interact.

Existing \TEX\ users, developers and user groups of course prefer to think
otherwise, that it is a positive choice by free will. That new users have looked
around and arrived at \TEX\ for good reason: their frontal cortex steering a
deliberate choice. Well, it might have played a role but the decision to use
\TEX\ might in the end be due to survival skills: I want to pass this exam and
therefore I will use that weird system called \TEX.

All animals, us included, have some level of intelligence but also have this hard
to describe property that we think makes us what we are. Intelligence and
consciousness are not the same (at least we know a bit about the first but nearly
nothing about the second). We can argue about how well composed some music is but
why we like it is a different matter.

We can make a well thought out choice for using \TEX\ for certain tasks but can
we say why we started liking it (or not)? Why it gives us pleasure or maybe
grief? Has it become a drug that we got addicted to? So, one can make an
intelligent decision about using \TEX\ but getting a grip on why we like it can
be hard. Do we enjoy the first time struggle? Probably not. Do we like the folks
involved? Yes, Don Knuth is a special and very nice person. Can we find help and
run into a friendly community? Yes, and a unique one too, annoying at times,
often stimulating and on the average friendly for all the odd cases running
around.

Artificial intelligence is pretty ambitious, so speaking of machine intelligence
is probably better. Is \TEX\ an intelligent program? There is definitely some
intelligence built in and the designer of that program is for sure very
intelligent. The designer is also a conscious entity: he likes what he did and
finds pleasure in using it. The program on the other hand is just doing its job:
it doesn't care how it's done and how long it takes: a mindless entity. So here
is a question: do we really want a more intelligent program doing the job for us,
or do those who attend conferences like Bacho\TEX\ enjoy \TEX ing so much that
they happily stay with what they have now? Compared to rockets tumbling down
and|/|or exploding or Mars landers thrashing themselves due to programming errors
of interactions, \TEX\ is surprisingly stable and bug free.

\stopsection

\startsection[title={Individual versus group evolution}]

After listening for hours to Sapolsky you start getting accustomed to remarks
about (unconscious) behaviour driven by genes, expression and environment, aimed
at \quotation {spreading many copies of your genes}. In most cases that is an
individual's driving force. However, cooperation between individuals plays a role
in this. A possible view is that we have now reached a state where survival is
more dependent on a group than on an individual. This makes sense when we
consider that developments (around us) can go way faster than regular evolution
(adaptation) can handle. We take control over evolution, a mechanism that needs
time to adapt and time is something we don't give it anymore.

Why does \TEX\ stay around? It started with an individual but eventually it's the
groups that keeps it going. A too|-|small group won't work but too|-|large groups
won't work either. It's a known fact that one can only handle some 150 social
contacts: we evolved in small bands that split when they became too large. Larger
groups demanded abstract beliefs and systems to deal with the numbers: housing,
food production, protection. The \TEX\ user groups also provide some
organization: they organize meetings, somehow keep development going and provide
infrastructure and distributions. They are organized around languages. According
to Diamond new languages are still discovered but many go extinct too. So the
potential for language related user groups is not really growing.

Some of the problems that we face in this world have become too large to be dealt
with by individuals and nations. In spite of what anti|-|globalists want we
cannot deal with our energy hunger, environmental issues, lack of natural
resources, upcoming technologies without global cooperation. We currently see a
regression in cooperation by nationalistic movements, protectionism and the usual
going back to presumed better times, but that won't work.

Local user groups are important but the number of members is not growing. There
is some cooperation between groups but eventually we might need to combine the
groups into one which might succeed unless one wants to come first. Of course we
will get the same sentiments and arguments as in regular politics but on the
other hand, we already have the advantage of \TEX\ systems being multi|-|lingual
and users sharing interest in the diversity of usage and users. The biggest
challenge is to pass on what we have achieved. We're just a momentary highlight
and let's not try to embrace some \quotation {\TEX\ first} madness.

\stopsection

\startplacefigure[location=top]
    \startcombination[nx=4,ny=1,width=\textwidth,distance=0pt]
        {\externalfigure[covers/3-16.jpg]                [height=5cm]} {art}
      % {\externalfigure[covers/dirt.jpg]                [height=5cm]} {history}
        {\externalfigure[covers/the-winds-of-change.jpg] [height=5cm]} {history}
        {\externalfigure[covers/pale-blue-dot.jpg]       [height=5cm]} {astronomy}
        {\externalfigure[covers/the-third-chimpanzee.jpg][height=5cm]} {history}
    \stopcombination
\stopplacefigure

\startsection[title=Sexes]

Most species have two sexes but it is actually a continuum controlled by hormones
and genetic expression: we just have to accept it. Although the situation has
improved there are plenty of places where some gender relationships are
considered bad even to the extent that one's life can be in danger. Actually
having strong ideas about these issues is typically human. But in the end one has
to accept the continuum.

In a similar way we just have to accept that \TEX\ usage, application of \TEX\
engines, etc.\ is a continuum and not a batch versus \WYSIWYG\ battle any more.
It's disturbing to read strong recommendations not to use this or that. Of the
many macro packages that showed up only a few were able to survive. How do users
of outlines look at bitmaps, how do \DVI\ lovers look at \PDF. But, as
typesetting relates to esthetics, strong opinions come with the game.

Sapolsky reports about a group of baboons where due to the fact that they get the
first choice of food the alpha males of pack got poisoned, so that the remaining
suppressed males who treated the females well became dominant. In fact they can
then make sure that no new alpha male from outside joins the pack without
behaving like they do. A sort of social selection. In a similar fashion, until
now the gatherings of \TEX ies managed to keep its social properties and has not
been dominated by for instance commerce.

% So, maybe should focus on acceptance and tolerance and then make sure that that
% we keep what we have and let it not be influenced too much by sectarianism. It
% makes a nice topic for a meeting of the context (sub)group, that actually has a
% women as driving force. How can we preserve what we have but still proceed is a
% legitimate question. Where do we stand in the landscape.

In the animal world often sexes relate to appearance. The word sexy made it to
other domains as well. Is \TEX\ sexy? For some it is. We often don't see the real
colors of birds. What looks gray to us looks vivid to a bird which sees in a
different spectrum. The same is true for \TEX. Some users see a command line
(shell) and think: this is great! Others just see characters and keystrokes and
are more attracted to an interactive program. When I see a graphic made by
\METAPOST, I always note how exact it is. Others don't care if their interactive
effort doesn't connect the dots well. Some people (also present here) think that
we should make \TEX\ attractive but keep in mind that like and dislike are not
fixed human properties. Some mindsets might as well be the result from our
makeup, others can be driven by culture.

\stopsection

\startsection[title=Religion]

One of Sapolsky's lectures is about religion and it comes in the sequence of
mental variations including depression and schizophrenia, because all these
relate to mental states, emotions, thresholds and such (all things human). That
makes it a tricky topic which is why it has not been taped. As I was raised in a
moderate Protestant tradition I can imagine that it's an uncomfortable topic
instead. But there are actually a few years older videos around and they are
interesting to watch and not as threatening as some might expect. Here I just
stick to some common characteristics.

If you separate the functions that religions play into for instance explanation
of the yet unknown, social interactions, control of power and regulation of
morals, then it's clear why at \TEX\ user group meetings the religious aspect of
\TEX\ has been discussed in talks. Those who see programs as infallible and
always right and don't understand the inner working can see it as an almighty
entity. In the Netherlands church-going diminishes but it looks like alternative
meetings are replacing it (and I'm not talking of football matches). So what are
our \TEX\ meetings? What do we believe in? The reason that I bring up this aspect
is that in the \TEX\ community we can find aspects of the more extremist aspects
of religions: if you don't use the macro package that I use, you're wrong. If you
don't use the same operating system as I do, you're evil. You will be punished if
you use the wrong editor for \TEX ? Why don't you use this library (which, by the
way, just replaced that other one)? We create angels and daemons. Even for quite
convinced atheists (it's not hard to run into them on youtube) a religion only
survives when it has benefits, something that puzzles them. So when we're
religious about \TEX\ and friends we have to make sure that it's at least
beneficial. Also, maybe we fall in Dennett's category of \quotation {believers
who want to believe}: it helps us to do our job if we just believe that we have
the perfect tool. Religion has inspired visual and aural art and keeps doing
that. (Don Knuth's current musical composition project is a good example of
this.)

Scientists can be religious, in flexible ways too, which is demonstrated by Don
Knuth. In fact, I'm pretty sure \TEX\ would not be in the position it is in now
if it weren't for his knowledgeable, inspirational, humorous, humble, and always
positive presence. And for sure he's not at all religious about the open source
software that he sent viral.

I'm halfway through reading \quotation {The Good Book of Human Nature} (An
Evolutionary Reading of the Bible) a book about the evolution of the bible and
monotheism which is quite interesting. It discusses for instance how transitions
from a hunter to a farmer society demanded a change of rules and introduced
stories that made sense in that changing paradigm. Staying in one place means
that possessions became more important and therefore inheritance. Often when
religion is discussed by behavioral biologists, historians and anthropologists
they stress this cultural narrative aspect. Also mentioned is that such societies
were willing to support (in food and shelter) the ones that didn't normally fit
it but added to the spiritual character of religions. The social and welcoming
aspect is definitely present in for instance Bacho\TEX\ conferences although a
bystander can wonder what these folks are doing in the middle of the night around
a campfire, singing, drinking, frying sausages, spitting fire, and discussing the
meaning of life.

Those who wrap up the state of religious affairs, do predictions and advocate the
message, are sometimes called evangelists. I remember a \TEX\ conference in the
\USA\ where the gospel of \XML\ was preached (by someone from outside the \TEX\
community). We were all invited to believe it. I was sitting in the back of the
crowded (!)\ room and that speaker was not at all interested in who spoke before
and after. Well, I do my share of \XML\ processing with \CONTEXT, but believe me:
much of the \XML\ that we see is not according to any gospel. It's probably
blessed the same way as those state officials get blessed when they ask and pray
for it in public.

It can get worse at \TEX\ conferences. Some present here at Bacho\TEX\ might
remember the \PDF\ evangelists that we had show up at \TEX\ conferences. You see
this qualification occasionally and I have become quite allergic to
qualifications like architect, innovator, visionary, inspirator and evangelist,
even worse when they look young but qualify as senior. I have no problem with
religion at all but let's stay away from becoming one. And yes, typography also
falls into that trap, so we have to be doubly careful.

\stopsection

\startplacefigure[location=top]
    \startcombination[nx=4,ny=1,width=\textwidth,distance=0pt]
        {\externalfigure[covers/from-bacteria-to-bach-and-back.jpg][height=5cm]} {philosophy}
        {\externalfigure[covers/the-lagoon.jpg]                    [height=5cm]} {science history}
        {\externalfigure[covers/chaos.jpg]                         [height=5cm]} {science}
        {\externalfigure[covers/why-zebras-dont-get-ulcers.jpg]    [height=5cm]} {behavioral biology}
    \stopcombination
\stopplacefigure

\startsection[title=Chaotic solutions]

The lectures on \quotation {chaos and reductionism} and \quotation {emergence and
complexity} were the highlights in Sapolsky's lectures. I'm not a good narrator
so I will not summarize them but it sort of boils down to the fact that certain
classes of problems cannot be split up in smaller tasks that we understand well,
after which we can reassemble the solutions to deal with the complex task.
Emerging systems can however cook up working solutions from random events.
Examples are colonies of ants and bees.

The \TEX\ community is like a colony: we cook up solutions, often by trial and
error. We dream of the perfect solutions but deep down know that esthetics cannot
be programmed in detail. This is a good thing because it doesn't render us
obsolete. At last year's Bacho\TEX, my nephew Teun and I challenged the anthill
outside the canteen to typeset the \TEX\ logo with sticks but it didn't persist.
So we don't need to worry about competition from that end. How do you program a
hive mind anyway?

When chaos theory evolved in the second half of the previous century not every
scientist felt happy about it. Instead of converging to more perfect predictions
and control in some fields a persistent uncertainty became reality.

After about a decade of using \TEX\ and writing macros to solve recurring
situations I came to the conclusion that striving for a perfect \TEX\ (the
engine) that can do everything and anything makes no sense. Don Knuth not only
stopped adding code when he could do what he needed for his books, he also stuck
to what to me seems reasonable endpoints. Every hard|-|coded solution beyond that
is just that: a hard|-|coded solution that is not able to deal with the
exceptions that make up most of the more complex documents. Of course we can
theorize and discuss at length the perfect never|-|reachable solutions but
sometimes it makes more sense to admit that an able user of a desktop publishing
system can do that job in minutes, just by looking at the result and moving
around an image or piece of text a bit.

There are some hard|-|coded solutions and presets in the programs but with
\LUATEX\ and \MPLIB\ we try to open those up. And that's about it. Thinking that
for instance adding features like protrusion or expansion (or whatever else)
always lead to better results is just a dream. Just as a butterfly flapping its
wings on one side of the world can have an effect on the other side, so can
adding a single syllable to your source completely confuse an otherwise clever
column or page break algorithm. So, we settle for not adding more to the engine,
and provide just a flexible framework.

A curious observation is that when Edward Lorenz ran into chaotic models it was
partially due to a restart of a simulation midway, using printed floating point
numbers that then in the computer were represented with a different accuracy than
printed. Aware of floating point numbers being represented differently across
architectures, Don Knuth made sure that \TEX\ was insensitive to this so that its
outcome was predictable, if you knew how it worked internally. Maybe \LUATEX\
introduces a bit of chaos because the \LUA\ we use has only floats. In fact, a
few months ago we did uncover a bug in the backend where the same phenomena gave
a chaotic crash.

In chaos theory there is the concept of an attractor. When visualized this can be
the area (seemingly random) covered by a trajectory. Or it can be a single point
where for instance a pendulum comes to rest. So what is our attractor? We have a
few actually. First there is the engine, the stable core of primitives always
present. You often see programs grow more complex every update and for sure that
happened with \ETEX, \PDFTEX, \XETEX\ and \LUATEX. However there is always the
core that is supposed to be stable. After some time the new kid arrives at a
stable state not much different from the parent. The same is true for \METAPOST.
Fonts are somewhat different because the technology changes but in the end the
shapes and their interactions become stable as well. Yet another example is \TEX\
Live: during a year it might diverge from its route but eventually it settles
down and enters the area where we expect it to end up. The \TEX\ world is at
times chaotic, but stable in the long run.

So, how about the existence, the reason for it still being around? One can
speculate about its future trajectory but one thing is sure: as long as we break
a text into paragraphs and pages \TEX\ is hard to beat. But what if we don't need
that any more? What if the concept of a page is no longer relevant? What if
justified texts no longer matter (often designers don't care anyway)? What if
students are no longer challenged to come up with a nice looking thesis? Do these
collaborative tools with remote \TEX\ processing really bring new long term users
or is \TEX\ then just one of the come|-|and|-|go tools?

\stopsection

\startsection[title=Looking ahead]

In an interview (\quotation {World of ideas}) Asimov explains that science
fiction evolved rapidly when people lived long enough to see that there was a
future (even for their offspring) that is different from today. It is (at least
for me) mind boggling to think of an evolution of hundreds of thousands of years
to achieve something like language. Waiting for the physical being to arrive at a
spot where you can make sounds, where the brain is suitable for linguistic
patterns, etc. A few hundred years ago speed of any developments (and science)
stepped up.

\TEX\ is getting near 40 years old. Now, for software that {\bf is} old! In that
period we have seen computers evolve: thousands of times faster processing, even
more increase in memory and storage. If we read about spaceships that travel at a
reasonable fraction of the speed of light, and think that will not happen soon,
just think back to the terminals that were sitting in computer labs when \TEX\
was developed: 300 baud was normal. I actually spent quite some time on
optimizing time|-|critical components of \CONTEXT\ but on this timescale that is
really a waste of time. But even temporary bottlenecks can be annoying (and
costly) enough to trigger such an effort. (Okay, I admit that it can be a
challenge, a kind of game, too.)

Neil Tyson, in the video \quotation {Storytelling of science} says that when
science made it possible to make photos it also made possible a transition in
painting to impressionism. Other technology could make the exact snapshot so
there was new room for inner feelings and impressions. When the Internet showed
up we went through a similar transition, but \TEX\ actually dates from before the
Internet. Did we also have a shift in typesetting? To some extent yes, browsers
and real time rendering is different from rendering pages on paper. In what space
and time are \TEX ies rooted?

We get older than previous generations. Quoting Sapolsky \quotation{\unknown\ we
are now living well enough and long enough to slowly fall apart.} The opposite is
happening with our tools, especially software: it's useful lifetime becomes
shorter and changes faster each year. Just look at the version numbers of
operating systems. Don Knuth expected \TEX\ to last for a long time and compared
to other software its core concept and implementation is doing surprisingly well.
We use a tool that suits our lifespan! Let's not stress ourselves out too much
with complex themes. (It helps to read \quotation {Why zebras don't get ulcers}.)

\stopsection

\startsection[title=Memes]

If you repeat a message often enough, even if it's something not true, it can
become a meme that gets itself transferred across generations. Conferences like
this is where they can evolve. We tell ourselves and the audience how good \TEX\
is and because we spend so many hours, days, weeks, months using it, it actually
must be good, or otherwise we would not come here and talk about it. We're not so
stupid as to spend time on something not good, are we? We're always surprised
when we run into a (potential) customer who seems to know \TEX. It rings a bell,
and it being around must mean something. Somehow the \TEX\ meme has anchored
itself when someone attended university. Even if experiences might have been bad
or usage was minimal. The meme that \TEX\ is the best in math typesetting is a
strong survivor.

There's a certain kind of person who tries to get away with their own deeds and
decisions by pointing to \quotation {fake news} and accusations of \quotation
{mainstream media} cheating on them. But to what extent are our stories true
about how easy \TEX\ macro packages are to use and how good their result? We have
to make sure we spread the right memes. And the user groups are the guardians.

Maybe macro packages are like memes too. In the beginning there was a bunch but
only some survived. It's about adaptation and evolution. Maybe competition was
too fierce in the beginning. Like ecosystems, organisms and cellular processes in
biology we can see the \TEX\ ecosystem, users and usage, as a chaotic system.
Solutions pop up, succeed, survive, lead to new ones. Some look similar and
slightly different input can give hugely different outcomes. You cannot really
look too far ahead and you cannot deduce the past from the present. Whenever
something kicks it off its stable course, like the arrival of color, graphics,
font technologies, \PDF, \XML, ebooks, the \TEX\ ecosystem has to adapt and find
its stable state again. The core technology has proven to be quite fit for the
kind of adaptation needed. But still, do it wrong and you get amplified out of
existence, don't do anything and the external factors also make you extinct.
There is no denial that (in the computer domain) \TEX\ is surprisingly stable and
adaptive. It's also hard not to see how conservatism can lead to extinction.

\startplacefigure[location=top]
    \startcombination[nx=4,ny=1,width=\textwidth,distance=0pt]
        {\externalfigure[covers/the-epigenetics-revolution.jpg]   [height=5cm]} {genetics}
        {\externalfigure[covers/dark-matter-and-the-dinosaurs.jpg][height=5cm]} {physics}
        {\externalfigure[covers/the-world-without-us.jpg]         [height=5cm]} {history}
        {\externalfigure[covers/what-we-cannot-know.jpg]          [height=5cm]} {science}
    \stopcombination
\stopplacefigure

\stopsection

\startsection[title=Inspiration]

I just took some ideas from different fields. I could have mentioned quantum
biology, which tries to explain some unexplainable phenomena in living creatures.
For instance how do birds navigate without visible and measurable clues. How do
people arrive at \TEX\ while we don't really advertise? Or I could mention
epigenetics and explorations in junk \DNA. It's not the bit of the genome that we
thought that matters, but also the expression of the genes driven by other
factors. Offspring not only gets genetic material passed but it can get presets.
How can the \TEX\ community pass on Knuth's legacy? Do we need to hide the
message in subtle ways? Or how about the quest for dark matter? Does it really
exist or do we want (need) it to exist? Does \TEX\ really have that many users,
or do we cheat by adding the users that are enforced during college but don't
like it at all? There's enough inspiration for topics at \TEX\ conferences, we
just have to look around us.

\stopsection

\startsection[title=Stability]

I didn't go into technical aspects of \TEX\ yet. I must admit that after decades
of writing macros I've reached a point where I can safely say that there will
never be perfect automated solutions for really complex documents. When books
about neural networks show up I wondered if it could be applied (but I couldn't).
When I ran into genetic algorithms I tried to understand its possible impact (but
I never did). So I stuck to writing solutions for problems using visualization:
the trial and error way. Of course, speaking of \CONTEXT, I will adapt what is
needed, and others can do that as well. Is there a new font technology? Fine,
let's support it as it's no big deal, just a boring programming task. Does a user
want a new mechanism? No problem, as solving a reduced subset of problems can be
fun. But to think of \TEX\ in a reductionist way, i.e.\ solving the small
puzzles, and to expect the whole to work in tandem to solve a complex task is not
trivial and maybe even impossible. It's a good thing actually, as it keeps us on
edge. Also, \CONTEXT\ was designed to help you with your own solutions: be
creative.

I mentioned my nephew Bram. He has seen part of this crowd a few times, just like
his brother and sister do now. He's into artificial intelligence now. In a few
years I'll ask him how he sees the current state of \TEX\ affairs. I might learn
a few tricks in the process.

In \quotation {The world without us} Weisman explores how fast the world would be
void of traces of humankind. A mere 10.000 years can be more than enough. Looking
back, that's about the time hunters became farmers. So here's a challenge: say
that we want an ant culture that evolves to the level of having archaeologists to
know that we were here at Bacho\TEX\ \unknown\ what would we leave behind?

Sapolsky ends his series by stressing that we should accept and embrace
individual differences. The person sitting next to you can have the same makeup
but be just a bit more sensitive to depression or be the few percent with genes
controlling schizophrenic behaviour. He stresses that knowing how things work or
where things go wrong doesn't mean that we should fix everything. So look at this
room full of \TEX ies: we don't need to be all the same, use all the same, we
don't need some dominance, we just need to accept and especially we need to
understand that we can never fully understand (and solve) everything forever.

Predictions, one of the themes, can be hard. It's not true that science has the
answer to everything. There will always be room for speculation and maybe we will
always need metaphysics too. I just started to read \quotation {What we cannot
know} by Sautoy. For sure those present here can not predict how \TEX\ will go on
and|/|or be remembered.

\stopsection

\startsection[title=Children of \TEX]

I mentioned \quotation {Children of time}. The author lets you see their spidery
world through spider eyes and physiology. They have different possibilities
(eyesight, smell) than we do and also different mental capabilities. They evolve
rapidly and have to cope conceptually with signals from a human surveillance
satellite up in the sky. Eventually they need to deal with a bunch of (of course)
quarrelling humans who want their place on the planet. We humans have some
pre|-|occupation with spiders and other creatures. In a competitive world it is
sometimes better to be suspicious (and avoid and flee) that to take a risk of
being eaten. A frequently used example is that a rustle in a bush can be the wind
or a lion, so best is to run.

We are not that well adapted to our current environment. We evolved at a very
slow pace so there was no need to look ahead more than a year. And so we still
don't look too far ahead (and choose politicians accordingly). We can also not
deal that well with statistics (Dawkins's \quotation {Climbing Mount Probability}
is a good read) so we make false assumptions, or just forget.

Does our typeset text really look that good on the long run, or do we cheat with
statistics? It's not too hard to find a bad example of something not made by
\TEX\ and extrapolate that to the whole body of typeset documents. Just like we
can take a nice example of something done by \TEX\ and assume that what we do
ourselves is equally okay. I still remember the tests we did with \PDFTEX\ and
hz. When \THANH\ and I discussed that with Hermann Zapf he was not surprised at
all that no one saw a difference between the samples and instead was focusing on
aspects that \TEX ies are told to look at, like two hyphens in a row.

A tool like \TEX\ has a learning curve. If you don't like that just don't use it.
If you think that someone doesn't like that, don't enforce this tool on that
someone. And don't use (or lie with) statistics. Much better arguments are that
it's a long|-|lived stable tool with a large user base and support. That it's not
a waste of time. Watching a designer like Hermann Zapf draw shapes is more fun
than watching click and point in heavily automated tools. It's probably also less
fun to watch a \TEX ie converge towards a solution.

Spiders are resilient. Ants maybe even more. Ants will survive a nuclear blast
(mutations might even bring them benefits), they can handle the impact of a
meteorite, a change in climate won't harm them much. Their biggest enemy is
probably us, when we try to wipe them out with poison. But, as long as they keep
a low profile they're okay. \TEX\ doesn't fit into the economic model as there is
no turnaround involved, no paid development, it is often not seen at all, it's
just a hit in a search engine and even then you might miss it (if only because no
one pays for it being shown at the top).

We can learn from that. Keeping a low profile doesn't trigger the competition to
wipe you out. Many (open source) software projects fade away: some big company
buys out the developer and stalls the project or wraps what they bought in their
own stuff, other projects go professional and enterprise and alienate the
original users. Yet others abort because the authors lose interest. Just like the
ideals of socialism don't automatically mean that every attempt to implement it
is a success, so not all open source and free software is good (natured) by
principle either. The fact that communism failed doesn't mean that capitalism is
better and a long term winner. The same applies to programs, whether successful
or not.

Maybe we should be like the sheep. Dennett uses these animals as a clever
species. They found a way to survive by letting themselves (unconsciously) be
domesticated. The shepherd guarantees food, shelter and protection. He makes sure
they don't get ill. Speaking biologically: they definitely made sure that many
copies of their genes survived. Cows did the same and surprisingly many of them
are related due to the fact that they share the same father (something now trying
to be reverted). All \TEX\ spin|-|offs relate to the same parent, and those that
survived are those that were herded by user groups. We see bits and pieces of
\TEX\ end up in other applications. Hyphenation is one of them. Maybe we should
settle for that small victory in a future hall of fame.

When I sit on my balcony and look at the fruit trees in my garden, some simple
math can be applied. Say that one of the apple trees has 100 apples per year and
say that this tree survives for 25 years (it's one of those small manipulated
trees). That makes 2.500 apples. Without human intervention only a few of these
apples make it into new trees, otherwise the whole world would be dominated by
apple trees. Of course that tree now only survives because we permit it to
survive, and for that it has to be humble (something that is very hard for modern
Apples). Anyway, the apple tree doesn't look too unhappy.

A similar calculation can be done for birds that nest in the trees and under my
roof. Given that the number of birds stays the same, most of energy spent on
raising offspring is wasted. Nevertheless they seem to enjoy life. Maybe we
should be content if we get one enthusiastic new user when we demonstrate \TEX\
to thousands of potential users.

Maybe, coming back to the themes of the conference, we should not come up with
these kinds of themes. We seem to be quite happy here. Talking about the things
that we like, meeting people. We just have to make sure that we survive. Why not
stay low under the radar? That way nothing will see us as a danger. Let's be like
the ants and spiders, the invisible hive mind that carries our message, whatever
that is.

When Dennett discusses language he mentions (coined) words that survive in
language. He also mentions that children pick up language no matter what. Their
minds are made for it. Other animals don't do that: they listen but don't start
talking back. Maybe \TEX\ is just made for certain minds. Some like it and pick
it up, while for others it's just noise. There's nothing wrong with that.
Predilection can be a user property.

\stopsection

\startsection[title={The unexpected}]

In a discussion with Dawkins the well|-|spoken astrophysicist Neil deGrasse Tyson
brings up the following. We differ only a few percent in \DNA\ from a chimp but
quite a lot in brain power, so how would it be if an alien that differs a few
percent (or more) passes by earth. Just like we don't talk to ants or chimps or
whatever expecting an intelligent answer, whatever passes earth won't bother
wasting time on us. Our rambling about the quality of typesetting probably sounds
alien to many people who just want to read and who happily reflow a text on an
ebook device, not bothered by a lack of quality.

\startplacefigure[location=top]
    \startcombination[nx=4,ny=1,width=\textwidth,distance=0pt]
        {\externalfigure[covers/live-as-we-do-not-know-it.jpg][height=5cm]} {astrobiology}
        {\externalfigure[covers/life-on-the-edge.jpg]         [height=5cm]} {quantumbiology}
        {\externalfigure[covers/rare-earth.jpg]               [height=5cm]} {astrophysics}
        {\externalfigure[covers/austerity.jpg]                [height=5cm]} {economics}
    \stopcombination
\stopplacefigure

We tend to take ourselves as reference. In \quotation {Rare Earth} Ward and
Brownlee extrapolate the possibility of life elsewhere in the universe. They are
not alone in thinking that while on one hand applying statistics to these
formulas of possible life on planets there might also be a chance that we're the
only intelligent species ever evolved. In a follow up, \quotation {Life as we do
not know it} paleontologist and astrobiologist Ward (one of my favourite authors)
discusses the possibility of life not based on carbon, which is not natural for a
carbon based species. Carl Sagan once pointed out that an alien species looking
down to earth can easily conclude that cars are the dominant species on earth and
that the thingies crawling in and out them are some kind of parasites. So, when
we look at the things that somehow end up on paper (as words, sentences,
ornaments, etc.), what is dominant there? And is what we consider dominant really
that dominant in the long run? You can look at a nice page as a whole and don't
see the details of the content. Maybe beauty hides nonsense.

When \TEX ies look around they look to similar technologies. Commands in shells
and solutions done by scripting and programming. This make sense in the
perspective of survival. However, if you want to ponder alternatives, maybe not
for usage but just for fun, a completely different perspective might be needed.
You must be willing to accept that communicating with a user of a \WYSIWYG\
program might be impossible. If mutual puzzlement is a fact, then they can either
be too smart and you can be too dumb or the reverse. Or both approaches can be
just too alien, based on different technologies and assumptions. Just try to
explain \TEX\ to a kid 40 years younger or to an 80 year old grandparent for that
matter. Today you can be very clever in one area and very stupid in another.

In another debate, Neil deGrasse Tyson asks Dawkins the question why in science
fiction movies the aliens look so human and when they don't, why they look so
strange, for instance like cumbersome sluggish snails. The response to that is
one of puzzlement: the opponent has no reference of such movies. In discussions
old \TEX ies like to suggest that we should convert young users. They often don't
understand that kids live in a different universe.

How often does that happen to us? In a world of many billions \TEX\ has its place
and can happily coexist with other typesetting technologies. Users of other
technologies can be unaware of us and even create wrong images. In fact, this
also happens in the community itself: (false) assumptions turned into
conclusions. Solutions that look alien, weird and wrong to users of the same
community. Maybe something that I present as hip and modern and high|-|\TEX\ and
promising might be the opposite: backward, old|-|fashioned and of no use to
others. Or maybe it is, but the audience is in a different mindset. Does it
matter? Let's just celebrate that diversity. (So maybe, instead of discussing the
conference theme, I should have talked about how I abuse \LUATEX\ in controlling
lights in my home as part of some IoT experiments.)

\stopsection

\startsection[title=What drives us]

I'm no fan of economics and big money talk makes me suspicious. I cannot imagine
working in a large company where money is the drive. It also means that I have
not much imagination in that area. We get those calls at the office from far away
countries who are hired to convince us by phone of investments. Unfortunately
mentioning that you're not at all interested in investments or that multiplying
money is irrelevant to you does not silence the line. You have to actively kill
such calls. This is also why I probably don't understand today's publishing world
where money also dominates. Recently I ran into talks by Mark Blyth about the
crisis (what crisis?) and I wish I could argue like he does when it comes to
typesetting and workflows. He discusses quite well that most politicians have no
clue what the crisis is about.

I think that the same applies to the management of publishers: many have no clue
what typesetting is about. So they just throw lots of money into the wrong
activities, just like the central banks seem to do. It doesn't matter if we \TEX
ies demonstrate cheap and efficient solutions.

Of course there are exceptions. We're lucky to have some customers that do
understand the issues at hand. Those are also the customers where authors may use
the tools themselves. Educating publishers, and explaining that authors can do a
lot, might be a premise, predilection and prediction in one go! Forget about
those who don't get it: they will lose eventually, unfortunately not before they
have reaped and wasted the landscape.

Google, Facebook, Amazon, Microsoft and others invest a lot in artificial
intelligence (or, having all that virtual cash, just buy other companies that
do). They already have such entities in place to analyze whatever you do. It is
predicted that at some point they know more about you then you know yourself.
Reading Luke Dormehl's \quotation {The Formula} is revealing. So what will that
do with our so|-|called (disputed by some) free will? Can we choose our own
tools? What if a potential user is told that all his or her friends use
WhateverOffice so they'd better do that too? Will subtle pressure lead them or
even us users away from \TEX ? We already see arguments among \TEX ies, like
\quotation {It doesn't look updated in 3 years, is it still good?} Why update
something that is still valid? Will the community be forced to update everything,
sort of fake updates. Who sets out the rules? Do I really need to update (or
re|-|run) manuals every five years?

Occasionally I visit the Festo website. This is a (family owned) company that
does research at the level that used to be common in large companies decades ago.
If I had to choose a job, that would be the place to go to. Just google for
\quotation {festo bionic learning network} and you understand why. We lack this
kind of research in the field we talk about today: research not driven by
commerce, short term profit, long term control, but because it is fundamental
fun.

Last year Alan Braslau and I spent some time on \BIBTEX. Apart from dealing with
all the weird aspects of the \APA\ standard, dealing with the inconsistently
constructed author fields is a real pain. There have been numerous talks about
that aspect here at Bacho\TEX\ by Jean|-|Michel Hufflen. We're trying to deal
with a more than 30|-|year|-|old flawed architecture. Just look back over a curve
that backtracks 30 years of exponential development in software and databases and
you realize that it's a real waste of time and a lost battle. It's fine to have a
text based database, and stable formats are great, but the lack of structure is
appalling and hard to explain to young programmers. Compare that to the Festo
projects and you realize that there can be more challenging projects. Of course,
dealing with the old data can be a challenge, a necessity and eventually even be
fun, but don't even think that it can be presented as something hip and modern.
We should be willing to admit flaws. No wonder that Jean|-|Michel decided to
switch to talking about music instead. Way more fun.

Our brains are massively parallel bio|-|machinery. Groups of neurons cooperate
and compete for attention. Coming up with solutions that match what comes out of
our minds demands a different approach. Here we still think in traditional
programming solutions. Will new ideas about presenting information, the follow up
on books come from this community? Are we the innovative Festo or are we an old
dinosaur that just follows the fashion?

\stopsection

\startsection[title=User experience]

Here is a nice one. Harari spends many pages explaining that research shows that
when an unpleasant experience has less unpleasantness at the end of the period
involved, the overall experience is valued according to the last experience. Now,
this is something we can apply to working with \TEX: often, the more you reach
the final state of typesetting the more it feels as all hurdles are in the
beginning: initial coding, setting up a layout, figuring things out, etc.

It can only get worse if you have a few left|-|over typesetting disasters but
there adapting the text can help out. Of course seeing it in a cheap bad print
can make the whole experience bad again. It happens. There is a catch here: one
can find lots of bad|-|looking documents typeset by \TEX. Maybe there frustration
(or indifference) prevails.

I sometimes get to see what kind of documents people make with \CONTEXT\ and it's
nice to see a good looking thesis with diverse topics: science, philosophy,
music, etc. Here \TEX\ is just instrumental, as what it is used for is way more
interesting (and often also more complex) than the tool used to get it on paper.
We have conferences but they're not about rocket science or particle
accelerators. Proceedings of such conferences can still scream \TEX, but it's the
content that matters. Here somehow \TEX\ still sells itself, being silently
present in rendering and presentations. It's like a rootkit: not really
appreciated and hard to get rid of. Does one discuss the future of rootkits other
than in the perspective of extinction? So, even as an invisible rootkit, hidden
in the workings of other programs, \TEX's future is not safe. Sometimes, when you
install a Linux system, you automatically get this large \TEX\ installation,
either because of dependencies or because it is seen as a similar toolkit as for
instance Open (or is it Libre) Office. If you don't need it, that user might as
well start seeing it as a (friendly) virus.

\stopsection

\startsection[title=Conclusion]

At some point those who introduced computers in typesetting had no problem
throwing printing presses out of the window. So don't pity yourself if at some
point in the near future you figure out that professional typesetting is no
longer needed. Maybe once we let machines rule the world (even more) we will be
left alone and can make beautiful documents (or whatever) just for the joy, not
bothering if we use outdated tools. After all, we play modern music on old
instruments (and the older rock musicians get, the more they seem to like
acoustic).

There are now computer generated compositions that experienced listeners cannot
distinguish from old school. We already had copies of paintings that could only
be determined forgeries by looking at chemical properties. Both of these
(artificial) arts can be admired and bring joy. So, the same applies to fully
automated typeset novels (or runtime rendered ebooks). How bad is that really?
You don't dig channels with your hand. You don't calculate logarithmic tables
manually any longer.

However, one of the benefits of the Internet is watching and listening to great
minds. Another is seeing musicians perform, which is way more fun that watching a
computer (although googling for \quotation {animusic} brings nice visuals).
Recently I ran into a wooden musical computer made by \quotation {Wintergatan}
which reminded me of the \quotation {Paige Compositor} that we use in a \LUATEX\
cartoon. Watching something like that nicely compensates for a day of rather
boring programming. Watching how the marble machine x (mmx) evolves is yet
another nice distraction.

Now, the average age of the audience here is pretty high even if we consider that
we get older. When I see solutions of \CONTEXT\ users (or experts) posted by
(young) users on the mailing list or stack exchange I often have to smile because
my answer would have been worse. A programmable system invokes creative
solutions. My criterion is always that it has to look nice in code and has some
elegance. Many posted solutions fit. Do we really want more automation? It's more
fun to admire the art of solutions and I'm amazed how well users use the
possibilities (even ones that I already forgot).

One of my favourite artists on my weekly \quotation {check youtube} list is Jacob
Collier. Right from when I ran into him I realized that a new era in music had
begun. Just google for his name and \quotation {music theory interview} and you
probably understand what I mean. When Dennett comments on the next generation
(say up to 25) he wonders how they will evolve as they grow up in a completely
different environment of connectivity. I can see that when I watch family
members. Already long ago Greg Bear wrote the novel \quotation {Darwin's
Children}. It sets you thinking and when looking around you even wonder if there
is a truth in it.

There are folks here at Bacho\TEX\ who make music. Now imagine that this is a
conference about music and that the theme includes the word \quotation {future}.
Then, imagine watching that video. You see some young musicians, one of them
probably one of the musical masterminds of this century, others instrumental to
his success, for instance by wrapping up his work. While listening you realize
that this next generation knows perfectly well what previous generations did and
achieved and how they influenced the current. You see the future there. Just look
at how old musicians reflect on such videos. (There are lots of examples of youth
evolving into prominent musicians around and I love watching them). There is no
need to discuss the future, in fact, we might make a fool of ourselves doing so.
Now back to this conference. Do we really want to discuss the future? What we
think is the future? Our future? Why not just hope that in the flow of getting
words on a medium we play our humble role and hope we're not forgotten but
remembered as inspiration.

One more word about predicting the future. When Arthur Clarke's \quotation {2001:
A Space Odyssey} was turned into a movie in 1968, a lot of effort went into
making sure that the not so far ahead future would look right. In 1996 scientists
were asked to reflect on these predictions in \quotation {Hal's Legacy}. It
turned out that most predictions were plain wrong. For instance computers got way
smaller (and even smaller in the next 20 years) while (self|-|aware) artificial
intelligence had not arrived either. So, let's be careful in what we predict (and
wish for).

\stopsection

\startsection[title=No more themes]

We're having fun here, that's why we come to Bacho\TEX\ (predilection). That
should be our focus. Making sure that \TEX's future is not so much in the cutting
edge but in providing fun to its users (prediction). So we just have to make sure
it stays around (premise). That's how it started out. Just watch at Don Knuth's
3:16 poster: via \TEX\ and \METAFONT\ he got in contact with designers and I
wouldn't be surprised if that sub|-|project was among the most satisfying parts.
So, maybe instead of ambitious themes the only theme that matters is: show what
you did and how you did it.

\stopsection

\stopchapter

\stopcomponent
