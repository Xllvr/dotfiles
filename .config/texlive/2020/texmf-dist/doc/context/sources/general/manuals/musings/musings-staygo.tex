% language=uk

% Written with on repeat:
%
% Rai Thistlethwayte: Betty Page (the keyscape version)

% \usemodule[art-01,abr-04]
%
% \setupbodyfont[12pt]
%
% \startdocument
%   [title={What’s to stay, what’s to go},
%    subtitle={The 2018 Bacho\TeX\ theme},
%    author={Hans Hagen}]

\definedescription
  [theme]
  [before=\startnarrower,
   after=\stopnarrower,
   title=yes,
   alternative=serried,
   width=fit,
   distance=.5\emwidth,
   text={\documentvariable{title}:}]

% \starttitle[title=\documentvariable{title}\\\txx\documentvariable{subtitle}]

\startcomponent musings-staygo

\environment musings-style

\startchapter[title={What’s to stay, what’s to go}]

\startsection[title=Introduction]

The following text was written as preparation for a 2018 talk at Bacho\TEX, which
has this theme. It's mostly a collection of thoughts. It was also more meant as a
wrapup for the presentation (possibly with some discussions) than an article.

\stopsection

\startsection[title=Attraction]

There are those movies where some whiz-kid sits down behind a computer, keys in a
few commands, and miracles happen. Ten fingers are used to generate programs that
work immediately. It's no problem to bypass firewalls. There is no lag over
network connections. Checking massive databases is no big deal and there's even
processing power left for real time visualization or long logs to the terminal.

How boring and old fashioned must a regular edit||run||preview cycle look
compared to this. If we take this 2018 movie reality as reference, in a time when
one can suck a phone empty with a simple connection, pull a hard drive from a
raid five array and still get all data immediately available, when we can follow
realtime whoever we want using cameras spread over the country, it's pretty clear
that this relatively slow page production engine \TEX\ has no chance to survive,
unless we want to impress computer illiterate friends with a log flying by on the
console (which in fact is used in movies to impress as well).

On YouTube you can find these (a few hours) sessions where Jacob Collier
harmonizes live in one of these Digital Audio Workstation programs. A while later
on another channel June Lee will transcribe these masterpieces into complex
sheets of music by ear. Or you can watch the weekly Wintergatan episodes on
building the Marble Machine from wood using drilling, milling, drawing programs
etc. There are impressive videos of multi|-|dimensional led arrays made by hand
and controlled by small computers and robots that solve Rubic Cubes. You can be
impressed by these Animusic videos, musicians show their craftmanship and
interesting informative movies are all over the place. I simply cannot imagine
millions of kids watching a \TEX\ style being written in a few hours. It's a real
challenge for an attention span. I hope to be proven wrong but I fear that for
the upcoming generation it's probably already too late because the \quote {whow}
factor of \TEX\ is low at first encounter. Although: picking up one of Don Knuths
books can have that effect: a nice mixture of code, typesetting and subtle
graphics, combined with great care, only possible with a system like \TEX.

\starttheme
    Biology teaches us that \quote {cool} is not a recipe for \quote {survival}.
    Not all designs by nature look cool, and it's only efficiency and
    functionality that matters. Beauty sometimes matters too but many functional
    mechanisms can do without. So far \TEX\ and its friends were quite capable to
    survive so there must be something in it that prevents it to be discarded.
    But survival is hard to explain. So far \TEX\ just stayed around but lack of
    visual attraction is a missing competitive trait.
\stoptheme

\stopsection

\startsection[title=Satisfaction]

Biology also teaches us that chemistry can overload reason. When we go for
short|-|term pleasure instead of long|-|term satisfaction (Google for Simon Sinek
on this topic), addiction kicks in (for instance driven by crossing the dopamine
thresholds too often, Google for Robert Sapolsky). Cool might relate more to
pleasure while satisfaction relates to an effort. Using \TEX\ is not that cool
and often takes an effort. But the results can be very satisfying. Where \quote
{cool} is rewarding in the short term, \quote {satisfaction} is more a long term
effect. So, you probably get the best (experience) out of \TEX\ by using it a
lifetime. That's why we see so many old \TEX ies here: many like the rewards.

If we want to draw new users we run into the problem that humans are not that
good in long term visions. This means that we cannot rely on showing cool (and
easy) features but must make sure that the long term reward is clear. We can try
to be \quote {cool} to draw in new users, but it will not be the reason they
stay. Instant success is important for kids who have to make a report for school,
and a few days \quotation {getting acquainted with a program} doesn't fit in.
It's hard to make kids addicted to \TEX\ (which could be a dubious objective).

\starttheme
    As long as the narrative of satisfaction can be told we will see new users.
    Meetings like Bacho\TEX\ is where the narrative gets told. What will happen
    when we no longer meet?
\stoptheme

\stopsection

\startsection[title=Survival]

Survival relates to improvements, stability and discarding of weak aspects.
Unfortunately that does not work out well in practice. Fully automated
multi||columns typesetting with all other elements done well too (we just mention
images) is hard and close to impossible for arbitrary cases, so nature would have
gotten rid of it. Ligatures can be a pain especially when the language is not
tagged and some kind of intelligence is needed to selectively disable them. They
are the tail of the peacock: not that handy but meant to be impressive. Somehow
it stayed around in automated typesetting, in biology it would be called a freak
of nature: probably a goodbye in wildlife. And how about page breaks on an
electronic device: getting rid of them would make the floating figures go away
and remove boundary conditions often imposed. It would also make widows and clubs
less of a problem. One can even wonder if with page breaks the windows and clubs
are the biggest problems, and if one can simply live with them. After all, we can
live with our own bodily limitations too. After all, (depending on what country
you live in) you can also live with bad roads, bad weather, polution, taxes, lack
of healthcare for many, too much sugar in food, and more.


\starttheme
    Animals or plants that can adapt to live on a specific island might not
    survive elsewhere. Animals or plants introduced in an isolated environment
    might quickly dominate and wipe out the locals. What are the equivalents in
    our \TEX\ ecosystem?
\stoptheme

\stopsection

\startsection[title=Niches]

But arguments will not help us determine if \TEX\ is the fittest for survival.
It's not a rational thing. Humans are bad in applying statistics in their live,
and looking far ahead is not a treat needed to survive. Often nature acts in
retrospect. (Climbing mount probability by Richard Dawkins). So, it doesn't
matter if we save time in the future if it complicates the current job. If
governments and companies cannot look ahead and act accordingly, how can we
extrapolate software (usage) or more specifically typesetting demands. Just look
at the political developments in the country that hosts this conference. Could we
have predicted the diminishing popularity of the \EU\ (and disturbing retrograde
political mess in some countries) of 2018 when we celebrated the moment Poland
joining the \EU\ at a Bacho\TEX\ campfire?

Extrapolating the future quality of versions of \TEX\ or macro packages also doesn't
matter much. With machine learning and artificial intelligence around the corner and
with unavoidable new interfaces that hook into our brains, who knows what systems
we need in the future. A generic flexible typesetting system is probably not the
most important tool then. When we discuss quality and design it gets personal so
a learning system that renders neutrally coded content into a form that suits
an individual, demands a different kind of tool than we have now.

On the short term (our live span) it makes more sense to look around and see how
other software (ecosystems) fare. Maybe we can predict \TEX's future from that.
Maybe we can learn from others mistakes. In the meantime we should not flatter
ourselves with the idea that a near perfect typesetting system will draw attention
and be used by a large audience. Factors external to the community play a too
important role in this.

\starttheme
    It all depends on how well it fits into a niche. Sometimes survival is only
    possible by staying low on the radar. But just as we destroy nature and kill
    animals competing for space, programs get driven out of the software world.
    On a positive note: in a project that provides open (free) math for schools
    students expressed to favour a printed book over \WEB|-|only (one curious
    argument for \WEB\ was that it permits easier listening to music at the same
    time).
\stoptheme

\stopsection

\startsection[title=Dominance]

Last year I installed a bit clever (evohome) heating control system. It's
probably the only \quotation {working out of the box} system that supports 12
zones but at the same time it has a rather closed interface as any other. One can
tweak a bit via a web interface but that one works by a proxy outside so there is
a lock in. Such a system is a gamble because it's closed and we're talking of a
20 year investment. I was able to add a layer of control (abusing \LUATEX\ as
\LUA\ engine and \CONTEXT\ as library) so let's see. When I updated the boiler I
also reconfigured some components (like valves) and was surprised how limited
upgrading was supported. One ends up with lost settings and weird interference
and it's because I know a bit of programming that I kept going and managed to add
more control. Of course, after a few weeks I had to check a few things in the
manuals, like how to enter the right menu.

So, as the original manuals are stored somewhere, one picks up the smart phone
and looks for the manual on the web. I have no problem with proper \PDF\ as a
manual but why not provide a simple standard format document alongside the fancy
folded A3 one. Is it because it's hard to produce different instances from one
source? Is it because it takes effort? We're talking of a product that doesn't
change for years.

\starttheme
    The availability of flexible tools for producing manuals doesn't mean that
    they are used as such. They don't support the survival of tools. Bad examples
    are a threat. Dominant species win.
\stoptheme

\stopsection

\startsection[title=Extinction]

When I was writing this I happened to visit a bookshop where I always check the
SciFi section for new publications. I picked out a pocket and wondered if I had
the wrong glasses on. The text was wobbling and looked kind of weird. On close
inspection indeed the characters were kind of randomly dancing on the baseline
and looked like some 150 \DPI\ (at most) scan. (By the way, I checked this the
next time I was there by showing the book to a nephew.) I get the idea that quite
some books get published first in the (more expensive) larger formats, so
normally I wait till a pocket size shows up (which can take a year) so maybe here
I had to do with a scan of a larger print scaled down.

What does that tell us? First of all that the publisher doesn't care about the
reader: this book is just unreadable. Second, it demonstrates that the printer
didn't ask for the original \PDF\ file and then scaled down the outline copy. It
really doesn't matter in this case if you use some high quality typesetting
program then. It's also a waste of time to talk to such publishers about quality
typesetting. The printer probably didn't bother to ask for a \PDF\ file that
could be scaled down.

\starttheme
    In the end most of the publishing industry will die and this is just one of
    the symptoms. Typesetting as we know it might fade away.
\stoptheme

\stopsection

\startsection[title=Desinterest]

The newspaper that I read has a good reputation for design. But why do they need
to drastically change the layout and font setup every few years? Maybe like an
animal marking his or her territory a new department head also has to put a mark
on the layout. Who knows. For me the paper became pretty hard to read: a too
light font that suits none of the several glasses that I have. So yes, I spend
less time reading the paper. In a recent commentary about the 75 year history of
the paper there was a remark about the introduction of a modern look a few
decades ago by using a sans serif font. I'm not sure why sans is considered
modern (most handwriting is sans) and to me some of these sans fonts look pretty
old fashioned compared to a modern elegant serif (or mix).

\starttheme
    If marketing and fashion of the day dominate then a wrong decision can result
    in dying pretty fast.
\stoptheme

\stopsection

\startsection[title=Persistence]

Around the turn of the century I had to replace my \CD\ player and realized that it
made more sense to invest in ripping the \CD's to \FLAC\ files and use a decent
\DAC\ to render the sound. This is a generic approach similar to processing
documents with \TEX\ and it looks as future proof as well. So, I installed a
virtual machine running SlimServer and bought a few SlimDevices, although by that
time they were already called SqueezeBoxes.

What started as an independent supplier of hardware and an open source program
had gone the (nowadays rather predictable) route of a buy out by a larger company
(Logitech). That company later ditched the system, even if it had a decent share
of users. This \quotation {start something interesting and rely on dedicated
users}, then \quotation {sell yourself (to the highest bidder)} and a bit later
\quotation {accept that the product gets abandoned} is where open source can fail
in many aspects: loyal users are ignored and offended with the original author
basically not caring about it. The only good thing is that because the software
is open source there can be a follow up, but of course that requires that there
are users able to program.

I have 5 small boxes and a larger transporter so my setup is for now safe from
extinction. And I can run the server on any (old) \LINUX\ or \MSWINDOWS\
distribution. For the record, when I recently connected the 20 year old Cambridge
CD2 I was surprised how well it sounded on my current headphones. The only
drawback was that it needs 10 minutes for the transport to warm up and get
working.

In a similar fashion I can still use \TEX, even when we originally started using
it with the only viable quality \DVI\ to \POSTSCRIPT\ backend at that time
(\DVIPSONE). But I'm not so sure what I'd done if I had not been involved in the
development of \PDFTEX\ and later \LUATEX . As an average user I might just have
dropped out. As with the \CD\ player, maybe someone will dust off an old \TEX\
some day and maybe the only hurdle is to get it running on a virtual retro
machine. Although \unknown\ recently I ran into an issue with a virtual machine
that didn't provide a console after a \KVM\ host update, so I'm also getting
pessimistic about that escape for older programs. (Not seldom when a library
update is forced into the \LUATEX\ repository we face some issue and it's not
something the average user want (or is able to) cope with.)

\starttheme
    Sometimes it's hard to go extinct, even when commerce interfered at some
    point. But it does happen that users successfully take (back) control.
\stoptheme

\stopsection

\startsection[title=Freedom]

If you buy a book originating in academia written and typeset by the author,
there is a chance that it is produced by some flavour of \TEX\ and looks quite
okay. This is because the author could iterate to the product she or he likes.
Unfortunately the web is also a source of bad looking documents produced by \TEX.
Even worse is that many authors don't even bother to set up a document layout
properly, think about structure and choose a font setup that matches well. One
can argue that only content matters. Fine, but than also one shouldn't claim
quality simply because \TEX\ has been used.

I've seen examples of material meant for bachelor students that made me pretend
that I am not familiar with \TEX\ and cannot be held responsible. Letter based
layouts on A4 paper, or worse, meant for display (or e|-|book devices) without
bothering to remove the excessive margins. Then these students are forced to use
some collaborative \TEX\ environment, which makes them dependent on the quality
standards of fellow students. No wonder that one then sees dozens of packages
being loaded, abundant copy and paste and replace of already entered formulas and
interesting mixtures of inline and display math, skips, kerns and whatever can
help to make the result look horrible.

\starttheme
    Don't expect enthusiast new users when you impose \TEX\ but take away freedom
    and force folks to cooperate with those with lesser standards. It will not
    help quality \TEX\ to stay around. You cannot enforce survival, it just
    happens or not, probably better with no competition or with a competition so
    powerful that it doesn't bother with the niches. In fact, keeping a low
    profile might be best! The number of users is no indication of quality,
    although one can abuse that statistic selectively?
\stoptheme

\stopsection

\startsection[title=Diversity]

Diversity in nature is enormous. There are or course niches, but in general there
are multiple variants of the same. When humans started breeding stock or
companion animals diversity also was a property. No one is forcing the same dog
upon everyone or the same cow. However, when industrialization kicks in things
become worse. Many cows in our country share the same dad. And when we look at
for instance corn, tomatoes or whatever dominance is not dictated by what nature
figures out best, but by what commercially makes most sense, even if that means
that something can't reproduce by itself any longer.

In a similar way the diversity of methods and devices to communicate (on paper)
at some point turns into commercial uniformity. The diversity is simply very
small, also in typesetting. And even worse, a user even has to defend
her|/|himself for a choice of system (even in the \TEX\ community). It's just
against nature.

\starttheme
    Normally something stays around till it no longer can survive. However, we
    humans have a tendency to destroy and commerce is helping a hand here. In
    that respect it's a surprise that \TEX\ is still around. On the other hand,
    humans also have a tendency to keep things artificially alive and even
    revive. Can we revive \TEX\ in a few hundred years given the complex code
    base and Make infrastructure?
\stoptheme

\stopsection

\startsection[title=Publishing]

What will happen with publishing? In the production notes of some of my recently
bought books the author mentions that the first prints were self|-|published
(either or not sponsored). This means that when a publisher \quotation {takes
over} (which still happens when one scales up) not much work has to be done.
Basically the only thing an author needs is a distribution network. My personal
experience with for instance \CD's produced by a group of musicians is that it is
often hard to get it from abroad (if at all) simply because one needs a payment
channel and mail costs are also relatively high.

But both demonstrate that given good facilitating options it is unlikely that
publishers as we have now have not much change of survival. Add to the argument
that while in Gutenbergs time a publisher also was involved in the technology,
today nothing innovative comes from publishers: the internet, ebook devices,
programs, etc.\ all come from elsewhere. And I get the impression that even in
picking up on technology publishers lag behind and mostly just react. Even
arguments like added value in terms of peer review are disappearing with the
internet where peer groups can take over that task. Huge amounts of money are
wasted on short|-|term modern media. (I bet similar amounts were never spend on
typesetting.)

\starttheme
    Publishers, publishing, publications and their public: as they are now they
    might not stay around. Lack of long term vision and ideas and decoupling of
    technology can make sure of that. Publishing will stay but anyone can
    publish; we only need the infrastructure. Creativity can win over greed and
    exploitation, small can win over big. And tools like \TEX\ can thrive in
    there, as it already does on a small scale.
\stoptheme

\stopsection

\startsection[title=Understanding]

\quotation {Why do you use \TEX?} If we limit this question to typesetting, you
can think of \quotation {Why don't you use \MSWORD ?} \quotation {Why don't you use
Indesign?}, \quotation {Why don't you use that macro package?}, \quotation {Why
don't you use this \TEX\ engine?} and alike. I'm sure that most of the readers
had to answer questions like this, questions that sort of assume that you're not
happy with what you use now, or maybe even suggest that you must be stupid not to
use \unknown

It's not that easy to explain why I use \TEX\ and|/|or why \TEX\ is good a the
job. If you are in a one|-|to|-|one (or few) sessions you can demonstrate its
virtues but \quote {selling} it to for instance a publisher is close to
impossible because this kind of technology is rather unknown and far from the
click|-|and|-|point paradigm. It's even harder when students get accustomed to
these interactive books from wherein they can even run code snippets although one
can wonder how individual these are when a student has the web as a source of
solutions. Only after a long exposure to similar and maybe imperfect alternatives
books will get appreciated.

For instance speaking of \quotation {automated typesetting} assumes that one
knows what typesetting is and also is aware that automated has some benefits. A
simple \quotation {it's an \XML\ to \PDF\ converter} might work better but that
assumes \XML\ being used which for instance not always makes sense. And while
hyphenation, fancy font support and proper justification might impress a \TEX\
user it often is less of an argument than one thinks.

The \quotation {Why don't you} also can be heard in the \TEX\ community. In the
worst case it's accompanied by a \quotation {\unknown\ because everybody uses
\unknown} which of course makes no sense because you can bet that the same user
will not fall for that argument when it comes to using an operating system or so.
Also from outside the community there is pressure to use something else: one can
find defense of minimal markup over \TEX\ markup or even \HTML\ markup as better
alternative for dissemination than for instance \PDF\ or \TEX\ sources. The
problem here is that old||timers can reflect on how relatively wonderful a
current technique really is, given changes over time, but who wants to listen to
an old|-|timer. Progress is needed and stimulating (which doesn't mean that all
old technology is obsolete). When I watched Endre eNerd's \quotation {The Time
Capsule} blu|-|ray I noticed an Ensoniq Fizmo keyboard and looked up what it was.
I ended up in interesting reads where the bottom line was \quotation {Either you
get it or you don't}. Reading the threads rang a bell. As with \TEX, you cannot
decide after a quick test or even a few hours if you (get the concept and) like
it or not: you need days, weeks, or maybe even months, and some actually never
really get it after years.

\starttheme
    It is good to wonder why you use some program but what gets used by others
    depends on understanding. If we can't explain the benefits there is no
    future for \TEX. Or more exact: if it no longer provide benefits, it will
    just disappear. Just walk around a gallery in a science museum that deals
    with computers: it can be a bit pathetic experience.
\stoptheme

\stopsection

{\bf Who knows \unknown}

\stoptitle

\stopdocument
