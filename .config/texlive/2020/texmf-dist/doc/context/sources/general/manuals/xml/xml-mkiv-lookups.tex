\environment xml-mkiv-style

\startcomponent xml-mkiv-lookups

\startchapter[title={Lookups using lpaths}]

\startsection[title={introduction}]

There is not that much system in the following examples. They resulted from tests
with different documents. The current implementation evolved out of the
experimental code. For instance, I decided to add the multiple expressions in row
handling after a few email exchanges with Jean|-|Michel Huffen.

One of the main differences between the way \XSLT\ resolves a path and our way is
the anchor. Take:

\starttyping
/something
something
\stoptyping

The first one anchors in the current (!) element so it will only consider direct
children. The second one does a deep lookup and looks at the descendants as well.
Furthermore we have a few extra shortcuts like \type {**} in \type {a/**/b} which
represents all descendants.

The expressions (between square brackets) has to be valid \LUA\ and some
preprocessing is done to resolve the built in functions. So, you might use code
like:

\starttyping
my_lpeg_expression:match(text()) == "whatever"
\stoptyping

given that \type {my_lpeg_expression} is known. In the examples below we use the
visualizer to show the steps. Some are shown more than once as part of a set.

\stopsection

\startsection[title={special cases}]

\xmllshow{}
\xmllshow{*}
\xmllshow{.}
\xmllshow{/}

\stopsection

\startsection[title={wildcards}]

\xmllshow{*}
\xmllshow{*:*}
\xmllshow{/*}
\xmllshow{/*:*}
\xmllshow{*/*}
\xmllshow{*:*/*:*}

\xmllshow{a/*}
\xmllshow{a/*:*}
\xmllshow{/a/*}
\xmllshow{/a/*:*}

\xmllshow{/*}
\xmllshow{/**}
\xmllshow{/***}

\stopsection

\startsection[title={multiple steps}]

\xmllshow{answer}
\xmllshow{answer/test/*}
\xmllshow{answer/test/child::}
\xmllshow{answer/*}
\xmllshow{answer/*[tag()='p' and position()=1 and text()!='']}

\stopsection

\startsection[title={pitfals}]

\xmllshow{[oneof(lower(@encoding),'tex','context','ctx')]}
\xmllshow{.[oneof(lower(@encoding),'tex','context','ctx')]}

\stopsection

\startsection[title={more special cases}]

\xmllshow{**}
\xmllshow{*}
\xmllshow{..}
\xmllshow{.}
\xmllshow{//}
\xmllshow{/}

\xmllshow{**/}
\xmllshow{**/*}
\xmllshow{**/.}
\xmllshow{**//}

\xmllshow{*/}
\xmllshow{*/*}
\xmllshow{*/.}
\xmllshow{*//}

\xmllshow{/**/}
\xmllshow{/**/*}
\xmllshow{/**/.}
\xmllshow{/**//}

\xmllshow{/*/}
\xmllshow{/*/*}
\xmllshow{/*/.}
\xmllshow{/*//}

\xmllshow{./}
\xmllshow{./*}
\xmllshow{./.}
\xmllshow{.//}

\xmllshow{../}
\xmllshow{../*}
\xmllshow{../.}
\xmllshow{..//}

\stopsection

\startsection[title={more wildcards}]

\xmllshow{one//two}
\xmllshow{one/*/two}
\xmllshow{one/**/two}
\xmllshow{one/***/two}
\xmllshow{one/x//two}
\xmllshow{one//x/two}
\xmllshow{//x/two}

\stopsection

\startsection[title={special axis}]

\xmllshow{descendant::whocares/ancestor::whoknows}
\xmllshow{descendant::whocares/ancestor::whoknows/parent::}
\xmllshow{descendant::whocares/ancestor::}
\xmllshow{child::something/child::whatever/child::whocares}
\xmllshow{child::something/child::whatever/child::whocares|whoknows}
\xmllshow{child::something/child::whatever/child::(whocares|whoknows)}
\xmllshow{child::something/child::whatever/child::!(whocares|whoknows)}
\xmllshow{child::something/child::whatever/child::(whocares)}
\xmllshow{child::something/child::whatever/child::(whocares)[position()>2]}
\xmllshow{child::something/child::whatever[position()>2][position()=1]}
\xmllshow{child::something/child::whatever[whocares][whocaresnot]}
\xmllshow{child::something/child::whatever[whocares][not(whocaresnot)]}
\xmllshow{child::something/child::whatever/self::whatever}

There is also \type {last-match::} that starts with the last found set of nodes.
This can save some run time when you do lots of tests combined with a same check
afterwards. There is however one pitfall: you never know what is done with that
last match in the setup that gets called nested. Take the following example:

\starttyping
\startbuffer[test]
<something>
    <crap> <crapa> <crapb> <crapc> <crapd>
        <crape>
            done 1
        </crape>
    </crapd>  </crapc> </crapb>  </crapa>
    <crap> <crapa> <crapb> <crapc> <crapd>
        <crape>
            done 2
        </crape>
    </crapd>  </crapc> </crapb>  </crapa>
    <crap> <crapa> <crapb> <crapc> <crapd>
        <crape>
            done 3
        </crape>
    </crapd>  </crapc> </crapb>  </crapa>
</something>
\stopbuffer
\stoptyping

One way to filter the content is this:

\starttyping
\xmldoif {#1} {/crap/crapa/crapb/crapc/crapd/crape} {
    some action
}
\stoptyping

It is not unlikely that you will do something like this:

\starttyping
\xmlfirst {#1} {/crap/crapa/crapb/crapc/crapd/crape} {
    \xmlfirst{#1}{/crap/crapa/crapb/crapc/crapd/crape}
}
\stoptyping

This means that the path is resolved twice but that can be avoided as
follows:

\starttyping
\xmldoif{#1}{/crap/crapa/crapb/crapc/crapd/crape}{
    \xmlfirst{#1}{last-match::}
}
\stoptyping

But the next is now guaranteed to work:

\starttyping
\xmldoif{#1}{/crap/crapa/crapb/crapc/crapd/crape}{
    \xmlfirst{#1}{last-match::}
    \xmllast{#1}{last-match::}
}
\stoptyping

Because the first one can have done some lookup the last match can be replaced
and the second call will give unexpected results. You can overcome this with:

\starttyping
\xmldoif{#1}{/crap/crapa/crapb/crapc/crapd/crape}{
    \xmlpushmatch
    \xmlfirst{#1}{last-match::}
    \xmlpopmatch
}
\stoptyping

Does it pay off? Here are some timings of a 10.000 times text and lookup
like the previous (on a decent January 2016 laptop):

\starttabulate[|r|l|]
\NC 0.239 \NC \type {\xmldoif {...} {...}}                                     \NC \NR
\NC 0.292 \NC \type {\xmlfirst {...} {...}}                                    \NC \NR
\NC 0.538 \NC \type {\xmldoif {...} {...} + \xmlfirst {...} {...}}             \NC \NR
\NC 0.338 \NC \type {\xmldoif {...} {...} + \xmlfirst {...} {last-match::}}    \NC \NR
\NC 0.349 \NC \type {+ \xmldoif {...} {...} + \xmlfirst {...} {last-match::}-} \NC \NR
\stoptabulate

So, pushing and popping (the last row) is a bit slower than not doing that but it
is still much faster than not using \type {last-match::} at all. As a shortcut
you can use \type {=}, as in:

\starttyping
\xmlfirst{#1}{=}
\stoptyping

You can even do this:

\starttyping
\xmlall{#1}{last-match::/text()}
\stoptyping

or

\starttyping
\xmlall{#1}{=/text()}
\stoptyping


\stopsection

\startsection[title={some more examples}]

\xmllshow{/something/whatever}
\xmllshow{something/whatever}
\xmllshow{/**/whocares}
\xmllshow{whoknows/whocares}
\xmllshow{whoknows}
\xmllshow{whocares[contains(text(),'f') or contains(text(),'g')]}
\xmllshow{whocares/first()}
\xmllshow{whocares/last()}
\xmllshow{whatever/all()}
\xmllshow{whocares/position(2)}
\xmllshow{whocares/position(-2)}
\xmllshow{whocares[1]}
\xmllshow{whocares[-1]}
\xmllshow{whocares[2]}
\xmllshow{whocares[-2]}
\xmllshow{whatever[3]/attribute(id)}
\xmllshow{whatever[2]/attribute('id')}
\xmllshow{whatever[3]/text()}
\xmllshow{/whocares/first()}
\xmllshow{/whocares/last()}

\xmllshow{xml://whatever/all()}
\xmllshow{whatever/all()}
\xmllshow{//whocares}
\xmllshow{..[2]}
\xmllshow{../*[2]}

\xmllshow{/(whocares|whocaresnot)}
\xmllshow{/!(whocares|whocaresnot)}
\xmllshow{/!whocares}

\xmllshow{/interface/command/command(xml:setups:register)}
\xmllshow{/interface/command[@name='xxx']/command(xml:setups:typeset)}
\xmllshow{/arguments/*}
\xmllshow{/sequence/first()}
\xmllshow{/arguments/text()}
\xmllshow{/sequence/variable/first()}
\xmllshow{/interface/define[@name='xxx']/first()}
\xmllshow{/parameter/command(xml:setups:parameter:measure)}

\xmllshow{/(*:library|figurelibrary)/*:figure/*:label}
\xmllshow{/(*:library|figurelibrary)/figure/*:label}
\xmllshow{/(*:library|figurelibrary)/figure/label}
\xmllshow{/(*:library|figurelibrary)/figure:*/label}

\xmlshow {whatever//br[tag(1)='br']}

\stopsection

\stopchapter

\stopcomponent
