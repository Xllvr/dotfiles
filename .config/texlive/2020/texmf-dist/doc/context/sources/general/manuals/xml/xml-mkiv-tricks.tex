\environment xml-mkiv-style

\startcomponent xml-mkiv-tricks

\startchapter[title={Tips and tricks}]

\startsection[title={tracing}]

It can be hard to debug code as much happens kind of behind the screens.
Therefore we have a couple of tracing options. Of course you can typeset some
status information, using for instance:

\startxmlcmd {\cmdbasicsetup{xmlshow}}
    typeset the tree given by \cmdinternal {cd:node}
\stopxmlcmd

\startxmlcmd {\cmdbasicsetup{xmlinfo}}
    typeset the name in the element given by \cmdinternal {cd:node}
\stopxmlcmd

\startxmlcmd {\cmdbasicsetup{xmlpath}}
    returns the complete path (including namespace prefix and index) of the
    given \cmdinternal {cd:node}
\stopxmlcmd

\startbuffer[demo]
<?xml version "1.0"?>
<document>
    <section>
        <content>
            <p>first</p>
            <p><b>second</b></p>
        </content>
    </section>
    <section>
        <content>
            <p><b>third</b></p>
            <p>fourth</p>
        </content>
    </section>
</document>
\stopbuffer

Say that we have the following \XML:

\typebuffer[demo]

and the next definitions:

\startbuffer
\startxmlsetups xml:demo:base
    \xmlsetsetup{#1}{p|b}{xml:demo:*}
\stopxmlsetups

\startxmlsetups xml:demo:p
    \xmlflush{#1}
    \par
\stopxmlsetups

\startxmlsetups xml:demo:b
    \par
    \xmlpath{#1} : \xmlflush{#1}
    \par
\stopxmlsetups

\xmlregisterdocumentsetup{example-10}{xml:demo:base}

\xmlprocessbuffer{example-10}{demo}{}
\stopbuffer

\typebuffer

This will give us:

\blank \startpacked \getbuffer \stoppacked \blank

If you use \type {\xmlshow} you will get a complete subtree which can
be handy for tracing but can also lead to large documents.

We also have a bunch of trackers that can be enabled, like:

\starttyping
\enabletrackers[xml.show,xml.parse]
\stoptyping

The full list (currently) is:

\starttabulate[|lT|p|]
\NC xml.entities  \NC show what entities are seen and replaced \NC \NR
\NC xml.path      \NC show the result of parsing an lpath expression \NC \NR
\NC xml.parse     \NC show stepwise resolving of expressions \NC \NR
\NC xml.profile   \NC report all parsed lpath expressions (in the log) \NC \NR
\NC xml.remap     \NC show what namespaces are remapped \NC \NR
\NC lxml.access   \NC report errors with respect to resolving (symbolic) nodes \NC \NR
\NC lxml.comments \NC show the comments that are encountered (if at all) \NC \NR
\NC lxml.loading  \NC show what files are loaded and converted \NC \NR
\NC lxml.setups   \NC show what setups are being associated to elements \NC \NR
\stoptabulate

In one of our workflows we produce books from \XML\ where the (educational)
content is organized in many small files. Each book has about 5~chapters and each
chapter is made of sections that contain text, exercises, resources, etc.\ and so
the document is assembled from thousands of files (don't worry, runtime inclusion
is pretty fast). In order to see where in the sources content resides we can
trace the filename.

\startxmlcmd {\cmdbasicsetup{xmlinclusion}}
    returns the file where the node comes from
\stopxmlcmd

\startxmlcmd {\cmdbasicsetup{xmlinclusions}}
    returns the list of files where the node comes from
\stopxmlcmd

\startxmlcmd {\cmdbasicsetup{xmlbadinclusions}}
    returns a list of files that were not included due to some problem
\stopxmlcmd

Of course you have to make sure that these names end up somewhere visible, for
instance in the margin.

\stopsection

\startsection[title={expansion}]

For novice users the concept of expansion might sound frightening and to some
extend it is. However, it is important enough to spend some words on it here.

It is good to realize that most setups are sort of immediate. When one setup is
issued, it can call another one and so on. Normally you won't notice that but
there are cases where that can be a problem. In \TEX\ you can define a macro,
take for instance:

\starttyping
\startxmlsetups xml:foo
  \def\foobar{\xmlfirst{#1}{/bar}}
\stopxmlsetups
\stoptyping

you store the reference top node \type {bar} in \type {\foobar} maybe for later use. In
this case the content is not yet fetched, it will be done when \type {\foobar} is
called.

\starttyping
\startxmlsetups xml:foo
  \edef\foobar{\xmlfirst{#1}{/bar}}
\stopxmlsetups
\stoptyping

Here the content of \type {bar} becomes the body of the macro. But what if
\type {bar} itself contains elements that also contain elements. When there
is a setup for \type {bar} it will be triggered and so on.

When that setup looks like:

\starttyping
\startxmlsetups xml:bar
  \def\barfoo{\xmlflush{#1}}
\stopxmlsetups
\stoptyping

Here we get something like:

\starttyping
\foobar => {\def\barfoo{...}}
\stoptyping

When \type {\barfoo} is not defined we get an error and when it is known and expands
to something weird we might also get an error.

Especially when you don't know what content can show up, this can result in errors
when an expansion fails, for example because some macro being used is not defined.
To prevent this we can define a macro:

\starttyping
\starttexdefinition unexpanded xml:bar:macro #1
  \def\barfoo{\xmlflush{#1}}
\stoptexdefinition

\startxmlsetups xml:bar
  \texdefinition{xml:bar:macro}{#1}
\stopxmlsetups
\stoptyping

The setup \type {xml:bar} will still expand but the replacement text now is just the
call to the macro, think of:

\starttyping
\foobar => {\texdefinition{xml:bar:macro}{#1}}
\stoptyping

But this is often not needed, most \CONTEXT\ commands can handle the expansions
quite well but it's good to know that there is a way out. So, now to some
examples. Imagine that we have an \XML\ file that looks as follows:

\starttyping
<?xml version='1.0' ?>
<demo>
    <chapter>
        <title>Some <em>short</em> title</title>
        <content>
            zeta
            <index>
                <key>zeta</key>
                <content>zeta again</content>
            </index>
            alpha
            <index>
                <key>alpha</key>
                <content>alpha <em>again</em></content>
            </index>
            gamma
            <index>
                <key>gamma</key>
                <content>gamma</content>
            </index>
            beta
            <index>
                <key>beta</key>
                <content>beta</content>
            </index>
            delta
            <index>
                <key>delta</key>
                <content>delta</content>
            </index>
            done!
        </content>
    </chapter>
</demo>
\stoptyping

There are a few structure related elements here: a chapter (with its list entry)
and some index entries. Both are multipass related and therefore travel around.
This means that when we let data end up in the auxiliary file, we need to make
sure that we end up with either expanded data (i.e.\ no references to the \XML\
tree) or with robust forward and backward references to elements in the tree.

Here we discuss three approaches (and more may show up later): pushing \XML\ into
the auxiliary file and using references to elements either or not with an
associated setup. We control the variants with a switch.

\starttyping
\newcount\TestMode

\TestMode=0 % expansion=xml
\TestMode=1 % expansion=yes, index, setup
\TestMode=2 % expansion=yes
\stoptyping

We apply a couple of setups:

\starttyping
\startxmlsetups xml:mysetups
    \xmlsetsetup{\xmldocument}{demo|index|content|chapter|title|em}{xml:*}
\stopxmlsetups

\xmlregistersetup{xml:mysetups}
\stoptyping

The main document is processed with:

\starttyping
\startxmlsetups xml:demo
    \xmlflush{#1}
    \subject{contents}
    \placelist[chapter][criterium=all]
    \subject{index}
    \placeregister[index][criterium=all]
    \page % else buffer is forgotten when placing header
\stopxmlsetups
\stoptyping

First we show three alternative ways to deal with the chapter. The first case
expands the \XML\ reference so that we have an \XML\ stream in the auxiliary
file. This stream is processed as a small independent subfile when needed. The
second case registers a reference to the current element (\type {#1}). This means
that we have access to all data of this element, like attributes, title and
content. What happens depends on the given setup. The third variant does the same
but here the setup is part of the reference.

\starttyping
\startxmlsetups xml:chapter
    \ifcase \TestMode
        % xml code travels around
        \setuphead[chapter][expansion=xml]
        \startchapter[title=eh: \xmltext{#1}{title}]
            \xmlfirst{#1}{content}
        \stopchapter
    \or
        % index is used for access via setup
        \setuphead[chapter][expansion=yes,xmlsetup=xml:title:flush]
        \startchapter[title=\xmlgetindex{#1}]
            \xmlfirst{#1}{content}
        \stopchapter
    \or
        % tex call to xml using index is used
        \setuphead[chapter][expansion=yes]
        \startchapter[title=hm: \xmlreference{#1}{xml:title:flush}]
            \xmlfirst{#1}{content}
        \stopchapter
    \fi
\stopxmlsetups

\startxmlsetups xml:title:flush
    \xmltext{#1}{title}
\stopxmlsetups
\stoptyping

We need to deal with emphasis and the content of the chapter.

\starttyping
\startxmlsetups xml:em
    \begingroup\em\xmlflush{#1}\endgroup
\stopxmlsetups

\startxmlsetups xml:content
    \xmlflush{#1}
\stopxmlsetups
\stoptyping

A similar approach is followed with the index entries. Watch how we use the
numbered entries variant (in this case we could also have used just \type
{entries} and \type {keys}).

\starttyping
\startxmlsetups xml:index
    \ifcase \TestMode
        \setupregister[index][expansion=xml,xmlsetup=]
        \setstructurepageregister
          [index]
          [entries:1=\xmlfirst{#1}{content},
           keys:1=\xmltext{#1}{key}]
    \or
        \setupregister[index][expansion=yes,xmlsetup=xml:index:flush]
        \setstructurepageregister
          [index]
          [entries:1=\xmlgetindex{#1},
           keys:1=\xmltext{#1}{key}]
    \or
        \setupregister[index][expansion=yes,xmlsetup=]
        \setstructurepageregister
          [index]
          [entries:1=\xmlreference{#1}{xml:index:flush},
           keys:1=\xmltext{#1}{key}]
    \fi
\stopxmlsetups

\startxmlsetups xml:index:flush
    \xmlfirst{#1}{content}
\stopxmlsetups
\stoptyping

Instead of this flush, you can use the predefined setup \type {xml:flush}
unless it is overloaded by you.

The file is processed by:

\starttyping
\starttext
    \xmlprocessfile{main}{test.xml}{}
\stoptext
\stoptyping

We don't show the result here. If you're curious what the output is, you can test
it yourself. In that case it also makes sense to peek into the \type {test.tuc}
file to see how the information travels around. The \type {metadata} fields carry
information about how to process the data.

The first case, the \XML\ expansion one, is somewhat special in the sense that
internally we use small pseudo files. You can control the rendering by tweaking
the following setups:

\starttyping
\startxmlsetups xml:ctx:sectionentry
    \xmlflush{#1}
\stopxmlsetups

\startxmlsetups xml:ctx:registerentry
    \xmlflush{#1}
\stopxmlsetups
\stoptyping

{\em When these methods work out okay the other structural elements will be
dealt with in a similar way.}

\stopsection

\startsection[title={special cases}]

Normally the content will be flushed under a special (so called) catcode regime.
This means that characters that have a special meaning in \TEX\ will have no such
meaning in an \XML\ file. If you want content to be treated as \TEX\ code, you can
use one of the following:

\startxmlcmd {\cmdbasicsetup{xmlflushcontext}}
    flush the given \cmdinternal {cd:node} using the \TEX\ character
    interpretation scheme
\stopxmlcmd

\startxmlcmd {\cmdbasicsetup{xmlcontext}}
    flush the match of \cmdinternal {cd:lpath} for the given \cmdinternal
    {cd:node} using the \TEX\ character interpretation scheme
\stopxmlcmd

We use this in cases like:

\starttyping
....
  \xmlsetsetup {#1} {
      tm|texformula|
  } {xml:*}
....

\startxmlsetups xml:tm
  \mathematics{\xmlflushcontext{#1}}
\stopxmlsetups

\startxmlsetups xml:texformula
  \placeformula\startformula\xmlflushcontext{#1}\stopformula
\stopxmlsetups
\stoptyping

\stopsection

\startsection[title={collecting}]

Say that your document has

\starttyping
<table>
    <tr>
        <td>foo</td>
        <td>bar<td>
    </tr>
</table>
\stoptyping

And that you need to convert that to \TEX\ speak like:

\starttyping
\bTABLE
    \bTR
        \bTD foo \eTD
        \bTD bar \eTD
    \eTR
\eTABLE
\stoptyping

A simple mapping is:

\starttyping
\startxmlsetups xml:table
    \bTABLE \xmlflush{#1} \eTABLE
\stopxmlsetups
\startxmlsetups xml:tr
    \bTR \xmlflush{#1} \eTR
\stopxmlsetups
\startxmlsetups xml:td
    \bTD \xmlflush{#1} \eTD
\stopxmlsetups
\stoptyping

The \type {\bTD} command is a so called delimited command which means that it
picks up its argument by looking for an \type {\eTD}. For the simple case here
this works quite well because the flush is inside the pair. This is not the case
in the following variant:

\starttyping
\startxmlsetups xml:td:start
    \bTD
\stopxmlsetups
\startxmlsetups xml:td:stop
    \eTD
\stopxmlsetups
\startxmlsetups xml:td
    \xmlsetup{#1}{xml:td:start}
    \xmlflush{#1}
    \xmlsetup{#1}{xml:td:stop}
\stopxmlsetups
\stoptyping

When for some reason \TEX\ gets confused you can revert to a mechanism that
collects content.

\starttyping
\startxmlsetups xml:td:start
    \startcollect
        \bTD
    \stopcollect
\stopxmlsetups
\startxmlsetups xml:td:stop
    \startcollect
        \eTD
    \stopcollect
\stopxmlsetups
\startxmlsetups xml:td
    \startcollecting
        \xmlsetup{#1}{xml:td:start}
        \xmlflush{#1}
        \xmlsetup{#1}{xml:td:stop}
    \stopcollecting
\stopxmlsetups
\stoptyping

You can even implement solutions that effectively do this:

\starttyping
\startcollecting
    \startcollect \bTABLE \stopcollect
        \startcollect \bTR \stopcollect
            \startcollect \bTD \stopcollect
            \startcollect   foo\stopcollect
            \startcollect \eTD \stopcollect
            \startcollect \bTD \stopcollect
            \startcollect   bar\stopcollect
            \startcollect \eTD \stopcollect
        \startcollect \eTR \stopcollect
    \startcollect \eTABLE \stopcollect
\stopcollecting
\stoptyping

Of course you only need to go that complex when the situation demands it. Here is
another weird one:

\starttyping
\startcollecting
    \startcollect \setupsomething[\stopcollect
        \startcollect foo=\stopcollect
        \startcollect FOO,\stopcollect
        \startcollect bar=\stopcollect
        \startcollect BAR,\stopcollect
    \startcollect ]\stopcollect
\stopcollecting
\stoptyping

\stopsection

\startsection[title={selectors and injectors}]

This section describes a bit special feature, one that we needed for a project
where we could not touch the original content but could add specific sections for
our own purpose. Hopefully the example demonstrates its useability.

\enabletrackers[lxml.selectors]

\startbuffer[foo]
<?xml version="1.0" encoding="UTF-8"?>

<?context-directive message info 1: this is a demo file ?>
<?context-message-directive info 2: this is a demo file ?>

<one>
    <two>
        <?context-select begin t1 t2 t3 ?>
            <three>
                t1 t2 t3
                <?context-directive injector crlf t1 ?>
                t1 t2 t3
            </three>
        <?context-select end ?>
        <?context-select begin t4 ?>
            <four>
                t4
            </four>
        <?context-select end ?>
        <?context-select begin t8 ?>
            <four>
                t8.0
                t8.0
            </four>
        <?context-select end ?>
        <?context-include begin t4 ?>
            <!--
                <three>
                    t4.t3
                    <?context-directive injector crlf t1 ?>
                    t4.t3
                </three>
            -->
            <three>
                t3
                <?context-directive injector crlf t1 ?>
                t3
            </three>
        <?context-include end ?>
        <?context-select begin t8 ?>
            <four>
                t8.1
                t8.1
            </four>
        <?context-select end ?>
        <?context-select begin t8 ?>
            <four>
                t8.2
                t8.2
            </four>
        <?context-select end ?>
        <?context-select begin t4 ?>
            <four>
                t4
                t4
            </four>
        <?context-select end ?>
        <?context-directive injector page t7 t8 ?>
        foo
        <?context-directive injector blank t1 ?>
        bar
        <?context-directive injector page t7 t8 ?>
        bar
    </two>
</one>
\stopbuffer

\typebuffer[foo]

First we show how to plug in a directive. Processing instructions like the
following are normally ignored by an \XML\ processor, unless they make sense
to it.

\starttyping
<?context-directive message info 1: this is a demo file ?>
<?context-message-directive info 2: this is a demo file ?>
\stoptyping

We can define a message handler as follows:

\startbuffer
\def\MyMessage#1#2#3{\writestatus{#1}{#2 #3}}

\xmlinstalldirective{message}{MyMessage}
\stopbuffer

\typebuffer \getbuffer

When this file is processed you will see this on the console:

\starttyping
info > 1: this is a demo file
info > 2: this is a demo file
\stoptyping

The file has some sections that can be used or ignored. The recipe for
obeying \type {t1} and \type {t4} is the following:

\startbuffer
\xmlsetinjectors[t1]
\xmlsetinjectors[t4]

\startxmlsetups xml:initialize
    \xmlapplyselectors{#1}
    \xmlsetsetup {#1} {
        one|two|three|four
    } {xml:*}
\stopxmlsetups

\xmlregistersetup{xml:initialize}

\startxmlsetups xml:one
    [ONE \xmlflush{#1} ONE]
\stopxmlsetups

\startxmlsetups xml:two
    [TWO \xmlflush{#1} TWO]
\stopxmlsetups

\startxmlsetups xml:three
    [THREE \xmlflush{#1} THREE]
\stopxmlsetups

\startxmlsetups xml:four
    [FOUR \xmlflush{#1} FOUR]
\stopxmlsetups
\stopbuffer

\typebuffer \getbuffer

This typesets:

\startnarrower
\xmlprocessbuffer{main}{foo}{}
\stopnarrower

The include coding is kind of special: it permits adding content (in a comment)
and ignoring the rest so that we indeed can add something without interfering
with the original. Of course in a normal workflow such messy solutions are
not needed, but alas, often workflows are not that clean, especially when one
has no real control over the source.

\startxmlcmd {\cmdbasicsetup{xmlsetinjectors}}
    enables a list of injectors that will be used
\stopxmlcmd

\startxmlcmd {\cmdbasicsetup{xmlresetinjectors}}
    resets the list of injectors
\stopxmlcmd

\startxmlcmd {\cmdbasicsetup{xmlinjector}}
    expands an injection (command); normally this one is only used
    (in some setup) or for testing
\stopxmlcmd

\startxmlcmd {\cmdbasicsetup{xmlapplyselectors}}
    analyze the tree \cmdinternal {cd:node} for marked sections that
    will be injected
\stopxmlcmd

We have some injections predefined:

\starttyping
\startsetups xml:directive:injector:page
    \page
\stopsetups

\startsetups xml:directive:injector:column
    \column
\stopsetups

\startsetups xml:directive:injector:blank
    \blank
\stopsetups
\stoptyping

In the example we see:

\starttyping
<?context-directive injector page t7 t8 ?>
\stoptyping

When we set \type {\xmlsetinjector[t7]} a pagebreak will injected in that spot.
Tags like \type {t7}, \type {t8} etc.\ can represent versions.

\stopsection

\startsection[title=preprocessing]

% local match    = lpeg.match
% local replacer = lpeg.replacer("BAD TITLE:","<bold>BAD TITLE:</bold>")
%
% function lxml.preprocessor(data,settings)
%     return match(replacer,data)
% end

\startbuffer[pre-code]
\startluacode
    function lxml.preprocessor(data,settings)
        return string.find(data,"BAD TITLE:")
           and string.gsub(data,"BAD TITLE:","<bold>BAD TITLE:</bold>")
            or data
    end
\stopluacode
\stopbuffer

\startbuffer[pre-xml]
\startxmlsetups pre:demo:initialize
    \xmlsetsetup{#1}{*}{pre:demo:*}
\stopxmlsetups

\xmlregisterdocumentsetup{pre:demo}{pre:demo:initialize}

\startxmlsetups pre:demo:root
    \xmlflush{#1}
\stopxmlsetups

\startxmlsetups pre:demo:bold
    \begingroup\bf\xmlflush{#1}\endgroup
\stopxmlsetups

\starttext
    \xmlprocessbuffer{pre:demo}{demo}{}
\stoptext
\stopbuffer

Say that you have the following \XML\ setup:

\typebuffer[pre-xml]

and that (such things happen) the input looks like this:

\startbuffer[demo]
<root>
BAD TITLE: crap crap crap ...

BAD TITLE: crap crap crap ...
</root>
\stopbuffer

\typebuffer[demo]

You can then clean up these \type {BAD TITLE}'s as follows:

\typebuffer[pre-code]

and get as result:

\start \getbuffer[pre-code,pre-xml] \stop

The preprocessor function gets as second argument the current settings, an d
the field \type {currentresource} can be used to limit the actions to
specific resources, in our case it's \type {buffer: demo}. Afterwards you can
reset the proprocessor with:

\startluacode
lxml.preprocessor = nil
\stopluacode

Future versions might give some more control over preprocessors. For now consider
it to be a quick hack.

\stopsection

\stopchapter

\stopcomponent
