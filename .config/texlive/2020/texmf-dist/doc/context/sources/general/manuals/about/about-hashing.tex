% language=uk

\startcomponent about-hashing

\environment about-environment

\usemodule[lua-hashing]

\startchapter[title={Lua strings}]

\startsection[title=Introduction]

In the crited project \footnote {This is a project by Thomas Schmitz, Alan
Braslau, Luigi Scarso and Hans Hagen funded by the Institut für Klassische und
Romanische Philologie Universität Bonn.} we have to deal with large amounts of
data. The sources are in \TEI\ \XML\ and processed directly in \CONTEXT\ \MKIV,
and we have to filter content from different places in the \XML\ tree. Processing
relies on \LUA\ a lot because we use \LUA\ for dealing with the \XML. We're
talking about Latin and Greek texts so there is no demand for extensive font
processing in \LUA\ is moderate. But as critical editions have lots of line
specific referencing and notes there are some more complex layout elements
involved, and again these use \LUA. There is also extensive use of bibliographies
and it will be no surprise that \LUA\ comes to help too. \footnote {One of the
objectives of the project is to update and enhance the bibliographic subsystem.}

One secondary objective is to be able to process the complex documents at a speed
of at least 20 pages per second on a modern 2014 workstation laptop. One way of
achieving this is to use \LUAJITTEX\ which has a faster virtual \LUA\ machine.
However, we ran into several issues with the \LUAJIT\ interpreter, which is fully
\LUA\ language 5.1 and partly 5.2 compatible but definitely has a different low
level implementation. In the next sections I will discuss two issues that Luigi
and I ran into and for which we could come up with reasonable workarounds.

\stopsection

\startsection[title=The stacks]

A \TEX\ job is normally a multi|-|pass experience. One run can produce information
that is used in a successive one. The reason is that something can happen on page
15 that influences the typesetting of page~9. There can even be a partial chain
reaction: you typeset a document the first time the table of contents (and the
pages it refers to) is not known yet but information is saved that makes it
possible next time. That next run it gets included and it takes for instance 4
pages. This means that all page numbers shift up. This in turn will trigger a new
run because all cross references might change too: two digit page numbers can
become three digits, so paragraphs can run wider, and that again can trigger more
pages. Normally an initial three runs is enough, and with minor updates of the
source one or two runs are enough after that.

The multi|-|pass information is saved in tables in the so called utility file and
loaded a next run. Common subtables are shared in the process. In order to
determine if there has been crucial changes that demand an extra run, we have to
make sure that random order in these tables is eliminated. Normally we already
sort keys in tables when writing them to file but some tables come out in the
order the traversing \type {next} function delivers them. In the more recent 5.2
versions \LUA\ has added some randomness to the order in which hashed tables are
organized, so while in previous versions we could assume that for a specific
binary the order was the same each time, we cannot rely on that any longer. This is
not that important for normal cases, but we compare previous and current versions
of the utility file and pack shared tables in them as well, which means that we
are sensitive for a change in order. But, this could be dealt with at the cost of
some extra sorting. \footnote {In \CONTEXT\ we also pack font tables which saves
lots of memory and also some load time).}

Anyway, this kind of changes in the \LUA\ machinery is harmless apart from taking
some time to adapt to it. It is also the reason why we cannot simply push a new
update of \LUA\ into \LUATEX\ because low level changes can have an (yet unknown)
impact. Of course performance is the biggest issue here: we don't want a slower
\LUATEX.

In the past we already reported on the benefits of \LUAJITTEX, especially its
faster virtual machine. We don't benefit from jitting; on the contrary it slows
us down. One reason is that we cross the \LUA||\CCODE\ boundary often and hardly
use any of the optimized functions. Part of the speed is achieved by a different
implementation deep down and one of them is a different virtual machine
instruction set. While \LUA\ can go real big in terms of memory and table
construction, \LUAJIT\ limits us to at most 2G memory and poses some 64K
limitations in functions and table constructors. The memory is not so much the
issue in the crited project but the (nested) table constructor is. When we have a
few tens of thousands of cross references, index entries and|/|or list entries we
simply cannot load the multi|-|pass data. A few days of playing with splitting up
nested tables didn't help much: it made the code look horrible and eventually we
again ran into a maximum of 64K someplace as a \type {dofile} effectively makes a
function that gets run and \LUAJIT\ doesn't like that size. For the record: we
don't have such issues with large font tables probably because they are just one
big table. The reason why we cannot use that approach is that serializing the
potentially very large tables in the utility file also has limitations.

Eventually this could be solved by assuming only forward referencing for certain
registers. That way we only used the index entries collected in memory during the
run and as long as we don't put a register before it's entries are defined we're
okay. So here we have a typical case where one can set an option to circumvent
an engine limitation. \footnote {A decade ago similar tricks had to be used to
support hundreds of thousands of hyperlinks in \TEX\ engines with at that time
limited memory capabilities.} Explaining this in a user manual is a challenge,
because an error message like the following is not that helpful:

\starttyping
main function has more than 65536 constants
\stoptyping

But, once we could generate these indices again by posing some limitations,
\LUAJITTEX\ had other issues. This time we got excessive runtime and we spent
quite some time sorting that one out. More on that in the next section.

\stopsection

\startsection[title=Hashing]

One of the reasons why (text processing with) \LUA\ is rather fast is that it
hashes its strings so that a test for equality is real fast. This means that for
each string that enters \LUA\ a hash value is calculated and that hash is used in
comparisons. Of course hashing takes time, but especially when you work with lots
of tables the advantage of a simple hash compare outweighs this one||time
hashing. On the other hand, if you work with files and process lines, and maybe
split these in words, you might end up with a lot of unneeded hashing. But, in
\LUATEX\ and therefore \MKIV\ we benefit from hashing a lot. In \LUA\ 5.2 the
hash function was adapted so that only strings upto than (default) 40 characters
get hashed. In practice we're not affected much by this, as most keywords we use
are shorter than this boundary. And in \CONTEXT\ we do quite some keyword checking.

So, when we were conducting tests with these large registers, we were surprised
that \LUAJITTEX\ performed significantly slower (ten times or more) that stock
\LUATEX, while until then we had observed that a \LUAJITTEX\ run was normally
some 20 to 40\% faster.

The first impression was that it related to the large amount of strings that are
written from \LUA\ to \TEX. After index entries are collected, they are sorted
and the index is flushed to \TEX. This happens in one go, and \TEX\ code ends up
in the \TEX\ input stack. Some actions are delayed and create callbacks to \LUA,
so some wrapping in functions happens too. That means that some (\LUA) strings
are only freed later on, but that proved not to be the main problem.

When the entries are typeset, an interactive cross reference is kept track of and
these exist till the document is closed and the referencing information is
written to the \PDF\ file. Of course we could tweak this but once you start along
that path there is no end to writing ugly hacks.

Eventually we found that the slowdown relates to hashing, especially because that is
not the first area where you look. Why is this? The specific register concerned lots
of small greek words, pointing to locations in a text, where locations looked like
\type {1.2.3}. In case you wonder why greek is mentioned: in multi|-|byte \UTF\
sequences there is a lot of repetition:

\startluacode
local byte = string.byte
function sample(s)
    context.NC() context(s)
    context.NC() context.ttx(false)
    for b in string.utfvalues(s) do
        context("%02X ",b)
    end
    context.NC() context.ttx(false)
    for b in string.gmatch(s,".") do
        context("%02X ",byte(b))
    end
    context.NC() context.NR()
end

context.starttabulate { "||||" }
context.FL()
context.NC() context.bold("word")
context.NC() context.bold("unicode")
context.NC() context.bold("bytes")
context.NC() context.NR()
context.FL()
sample("βίον")
sample("βίου")
sample("βιοὺς")
sample("βουλὴν")
sample("βουλῆς")
context.LL()
context.stoptabulate()
\stopluacode

When cross referencing these index entries with their origin, you end up with
reference identifiers like \type {foo:1.2.3} or, because \CONTEXT\ has automated
internal references (which are rather efficient in the resulting \PDF), we get
\type {aut:1}, \type {aut:2} upto in this case some 30.000 of them.

The problem with hashing is as follows. When we write commands to \TEX\ or use
data with a repetitive property, the similarity of these strings can be hard on
the hasher as it can produce similar hash keys in which case collisions need to
be dealt with. I'm no expert on hashing but looking at the code shows that in
\LUAJIT\ (at least in the version we're talking about) the string is seen as
chunks of 4 bytes. The first, last, middle and halfway middle chunks are
consulted and after some bit juggling we get a hash value. In the case of strings
like the following it is clear that the beginning and end look quite the same:

\starttyping
foo:000001  foo:010001  foo:100001
\stoptyping

or:

\starttyping
foo:1.2.12  foo:1.3.12  foo:1.4.12  foo:1.5.12
\stoptyping

It seems that the used method of hashing is somewhat arbitrary and maybe tuned
for specific applications. In order to see what the impact is of hashing quite
similar strings, some experiments were conducted: with \LUATEX\ 0.73 using \LUA\
5.2 hashing, with \LUAJITTEX\ 0.73, and with the same \LUAJITTEX\ but using the
hash variant of native \LUA\ 5.1. For each variant we ran tests where strings of
increasing length were combined with a number (running from one to one million).

\starttabulate[|||]
\NC none   \NC <string>                   \NC \NR
\NC right  \NC <string> <number>          \NC \NR
\NC left   \NC <number> <string>          \NC \NR
\NC center \NC <string> <number> <string> \NC \NR
\NC edges  \NC <number> <string> <number> \NC \NR
\stoptabulate

The differences between engines can be seen in tables in the next page. In the
fourth table we summarize which engine performs best. Keep in mind that
\LUAJITTEX\ has the advantage of the faster virtual machine so it has an
additional speed advantage.

We show three tables with measurements. The \type {none} column shows the
baseline of the test:

\starttyping

local t = { }
for i=1,1000000 do
    t[i] = i
end
\stoptyping

The column tagged \quote {right} does this:

\starttyping
local t = { }
for i=1,1000000 do
    t[i] = text .. i
end
\stoptyping

And \quote {left} does:

\starttyping
local t = { }
for i=1,1000000 do
    t[i] = i .. text
end
\stoptyping

That leaves \quote {center}:

\starttyping
local t = { }
for i=1,1000000 do
    t[i] = text .. i .. text
end
\stoptyping

and \quote {edges}:

\starttyping
local t = { }
for i=1,1000000 do
    t[i] = i .. text .. i
end
\stoptyping

Of course there is also the loop and the concatenation involved so the last two
variants have some more overhead. We show some measurements in \in {tables}
[tab:torture-1], \in [tab:torture-2] \in {and} [tab:torture-3]. So, there we have
strings like:

\starttyping
2abc
222abc
22222abc
abc222222
222222abc222222
222222abc222222
abc2222abc
\stoptyping

and so on. Of course a million such strings makes not much sense in practice but
it serves our purpose of testing.

\startplacetable[reference=tab:torture-1,location=page,title=\type{context test.tex}]
    \scale
      [height=\the\dimexpr\textheight-3\lineheight\relax]
    % [width=\the\dimexpr\textwidth+.5\backspace\relax]
      {\vbox{\ctxlua{moduledata.luatests.showhashing { filename = "luatest-hash-luatex-073-LUA52.lua" }}}}
\stopplacetable

\startplacetable[reference=tab:torture-2,location=page,title=\type{context --jit --jithash=luajit20 test.tex}]
    \scale
      [height=\the\dimexpr\textheight-3\lineheight\relax]
    % [width=\the\dimexpr\textwidth+.5\backspace\relax]
      {\vbox{\ctxlua{moduledata.luatests.showhashing { filename = "luatest-hash-luajittex-073-JIT20.lua" }}}}
\stopplacetable

\startplacetable[reference=tab:torture-3,location=page,title=\type{context --jit --jithash=lua51 test.tex}]
    \scale
      [height=\the\dimexpr\textheight-3\lineheight\relax]
    % [width=\the\dimexpr\textwidth+.5\backspace\relax]
      {\vbox{\ctxlua{moduledata.luatests.showhashing { filename = "luatest-hash-luajittex-073-LUA51.lua" }}}}
\stopplacetable

In these tables you can see some extremes. On the average \LUA\ 5.2 performs
quite okay as does standard \LUAJIT. However, when we bring the 5.1 hash variant
into \LUAJITTEX\ we get a more predictable average performance as it deals better
with some of the extreme cases that make \LUAJITTEX\ crawl compared to \LUATEX.
We have done more tests and interesting is to see that in the 5.1 (and derived
5,2) method there are sometimes cases where odd lengths perform much worse than
even lengths. Red values are larger than two times the average, blue values
larger than average while green values indicate a less than half average value.

In \in {table} [tab:compare-1] we show which method performs best relative to each
other. Of course in many applications there will be no such extreme cases, but
we happen to ran into them. But, even if \type {JIT20} is a winner in most cases,
the fact that it has extreme slow exceptions makes it a bit of a gamble.

\startplacetable[location=page,reference=tab:compare-1,title=The best performances per engine and hasher.]
    \startcombination
        \startcontent
            \scale
              [height=\the\dimexpr\textheight-4\lineheight\relax]
              {\vbox{\ctxlua{moduledata.luatests.showhashing {
                  fileset = {
                      { tag = "JIT20", filename = "luatest-hash-luajittex-073-JIT20.lua" },
                      { tag = "JIT51", filename = "luatest-hash-luajittex-073-LUA51.lua" },
              } } }}}
        \stopcontent
        \startcaption
            \LUAJITTEX\ only
        \stopcaption
        \startcontent
            \scale
              [height=\the\dimexpr\textheight-4\lineheight\relax]
              {\vbox{\ctxlua{moduledata.luatests.showhashing {
                  fileset = {
                      { tag = "LUA52", filename = "luatest-hash-luatex-073-LUA52.lua" },
                      { tag = "JIT20", filename = "luatest-hash-luajittex-073-JIT20.lua" },
                      { tag = "JIT51", filename = "luatest-hash-luajittex-073-LUA51.lua" },
              } } }}}
        \stopcontent
        \startcaption
            Both engines.
        \stopcaption
    \stopcombination
\stopplacetable

The 5.1 hasher runs over the string with a step that depends on the length of the
string. We've seen that in 5.2 it doesn't hash strings larger than 40 characters.
The step is calculated by shifting the length (by default) over 5 bits. This
means that for strings of size 32 and more the step becomes 2 which is why we see
this odd|/|even timing issue in the tables. Basically we hash at most 32
characters of the 40. The next table shows that the less characters we take
into account (first column) the less unique keys we get (second column).

\starttabulate[|c|r|l|]
\FL
\NC \bf n \NC \bf unique \NC \bf text \NC \NR
\FL
\NC 3 \NC 22    \NC \tt\tx /Border [ 0 0 0 ] /F 4 /Subtype /Link /A * 0 R \NC \NR
\NC 3 \NC 31    \NC \tt\tx << /D [ * 0 R /Fit ] /S /GoTo >> \NC \NR
\NC 4 \NC 43    \NC \tt\tx /Border [ 0 0 0 ] /F 4 /Subtype /Link /A * 0 R \NC \NR
\NC 4 \NC 51    \NC \tt\tx << /D [ * 0 R /Fit ] /S /GoTo >> \NC \NR
\NC 5 \NC 410   \NC \tt\tx /Border [ 0 0 0 ] /F 4 /Subtype /Link /A * 0 R \NC \NR
\NC 5 \NC 210   \NC \tt\tx << /D [ * 0 R /Fit ] /S /GoTo >> \NC \NR
\NC 6 \NC 29947 \NC \tt\tx /Border [ 0 0 0 ] /F 4 /Subtype /Link /A * 0 R \NC \NR
\NC 6 \NC 29823 \NC \tt\tx << /D [ * 0 R /Fit ] /S /GoTo >> \NC \NR
\LL
\stoptabulate

In the next table we show a few cases. The characters that are taken into account
are colored red. \footnote {Again the first column indicates the shift applied to
the length in order to determine the step.}

\starttabulate[|c|l|l|]
\FL
\NC \bf n \NC \bf text \NC \bf consulted \NC \NR
\FL
\NC 3\NC \tt\tx << /D [ 8 0 R /Fit ] /S /GoTo >>  \NC \tt\tx <{\darkred <} /{\darkred D} [{\darkred \space }8 {\darkred 0} R{\darkred \space }/F{\darkred i}t {\darkred ]} /{\darkred S} /{\darkred G}oT{\darkred o} >{\darkred >} \NC \NR
\NC 3\NC \tt\tx << /D [ 9 0 R /Fit ] /S /GoTo >>  \NC \tt\tx <{\darkred <} /{\darkred D} [{\darkred \space }9 {\darkred 0} R{\darkred \space }/F{\darkred i}t {\darkred ]} /{\darkred S} /{\darkred G}oT{\darkred o} >{\darkred >} \NC \NR
\NC 3\NC \tt\tx << /D [ 10 0 R /Fit ] /S /GoTo >> \NC \tt\tx <<{\darkred \space }/D{\darkred \space}[ {\darkred 1}0 {\darkred 0} R{\darkred \space }/F{\darkred i}t {\darkred ]} /{\darkred S} /{\darkred G}oT{\darkred o} >{\darkred >} \NC \NR
\NC 3\NC \tt\tx << /D [ 11 0 R /Fit ] /S /GoTo >> \NC \tt\tx <<{\darkred \space }/D{\darkred \space}[ {\darkred 1}1 {\darkred 0} R{\darkred \space }/F{\darkred i}t {\darkred ]} /{\darkred S} /{\darkred G}oT{\darkred o} >{\darkred >} \NC \NR
\NC 3\NC \tt\tx << /D [ 12 0 R /Fit ] /S /GoTo >> \NC \tt\tx <<{\darkred \space }/D{\darkred \space}[ {\darkred 1}2 {\darkred 0} R{\darkred \space }/F{\darkred i}t {\darkred ]} /{\darkred S} /{\darkred G}oT{\darkred o} >{\darkred >} \NC \NR
\ML
\NC 4\NC \tt\tx << /D [ 8 0 R /Fit ] /S /GoTo >>  \NC \tt\tx <{\darkred <} {\darkred /}D{\darkred \space }[{\darkred \space }8{\darkred \space }0{\darkred \space }R{\darkred \space }/{\darkred F}i{\darkred t} {\darkred ]} {\darkred /}S{\darkred \space }/{\darkred G}o{\darkred T}o{\darkred \space }>{\darkred >} \NC \NR
\NC 4\NC \tt\tx << /D [ 9 0 R /Fit ] /S /GoTo >>  \NC \tt\tx <{\darkred <} {\darkred /}D{\darkred \space }[{\darkred \space }9{\darkred \space }0{\darkred \space }R{\darkred \space }/{\darkred F}i{\darkred t} {\darkred ]} {\darkred /}S{\darkred \space }/{\darkred G}o{\darkred T}o{\darkred \space }>{\darkred >} \NC \NR
\NC 4\NC \tt\tx << /D [ 10 0 R /Fit ] /S /GoTo >> \NC \tt\tx {\darkred <}<{\darkred \space}/{\darkred D} {\darkred [} {\darkred 1}0{\darkred \space }0{\darkred \space }R{\darkred \space }/{\darkred F}i{\darkred t} {\darkred ]} {\darkred /}S{\darkred \space }/{\darkred G}o{\darkred T}o{\darkred \space }>{\darkred >} \NC \NR
\NC 4\NC \tt\tx << /D [ 11 0 R /Fit ] /S /GoTo >> \NC \tt\tx {\darkred <}<{\darkred \space}/{\darkred D} {\darkred [} {\darkred 1}1{\darkred \space }0{\darkred \space }R{\darkred \space }/{\darkred F}i{\darkred t} {\darkred ]} {\darkred /}S{\darkred \space }/{\darkred G}o{\darkred T}o{\darkred \space }>{\darkred >} \NC \NR
\NC 4\NC \tt\tx << /D [ 12 0 R /Fit ] /S /GoTo >> \NC \tt\tx {\darkred <}<{\darkred \space}/{\darkred D} {\darkred [} {\darkred 1}2{\darkred \space }0{\darkred \space }R{\darkred \space }/{\darkred F}i{\darkred t} {\darkred ]} {\darkred /}S{\darkred \space }/{\darkred G}o{\darkred T}o{\darkred \space }>{\darkred >} \NC \NR
\LL
\stoptabulate

Of course, in practice, in \LUA\ 5.2 the longer string exceeds 40 characters so
is never hashed anyway. Apart from this maximum, the \LUA\ hash code looks like this:

\starttyping
/* Lua will use at most ~(2^LUAI_HASHLIMIT) bytes from
a string to compute its hash */
...
h = cast(unsigned int,len) ;
step = (len>>LUAI_HASHLIMIT) + 1 ;
for (l1=len; l1>=step; l1-=step) {
   h = h ^ ((h<<5) + (h>>2) + cast(unsigned char,str[l1-1])) ;
}
...
\stoptyping

This translates in verbose \LUA\ function as follows:

\starttyping
function string.luahash(str,shift)
    local len  = #str
    local hash = len
    local step = bit32.rshift(len,shift or 5) + 1
    for i=len,1,-step do
        hash = bit32.bxor(hash, (
            bit32.lshift(hash,5) +
            bit32.rshift(hash,2) +
            string.byte(string.sub(str,i,i))
        ) )
    end
    return hash
end
\stoptyping

The reader can argue that the following string would perform better:

\starttyping
/Subtype/Link/Border[0 0 0]/F 4/A 12 0 R
\stoptyping

but this is not the case. Also, here we use \PDF\ code, but similar cases can
happen if we flush \TEX\ commands:

\starttyping
\dothisorthat{1}
\dothisorthat{101}
\dothisorthat{10101}
\stoptyping

And in the case of \UTF\ strings, it remains a fact that when characters need two
bytes a sequence can end up with each odd or even byte being the same. This is
one more reason to support upto 64 byte (or 40 in practice) hashing.

Because of this we decided to experiment with a value of 64 instead. \footnote {Of
course, in \LUATEX, the length limit kicks in before we get to 64.} We can do the
same when we use the \LUA\ 5.1 method in \LUAJIT. In \in {table} [tab:torture-4]
\in {and} [tab:torture-5] we show the timings. Interesting is that we lost the
extremes now. The performance of the default settings are compared with the higher
values in \in {table} [tab:compare-2]. Of course the numbers are just indications
and there might be small differences between test runs. Therefore we use a threshold
of 5\% when we compare two methods.

\startplacetable[reference=tab:torture-4,location=page,title={\type{context test.tex} with len<=40 and hash<=64}]
    \scale
      [height=\the\dimexpr\textheight-3\lineheight\relax]
    % [width=\the\dimexpr\textwidth+.5\backspace\relax]
      {\vbox{\ctxlua{moduledata.luatests.showhashing { filename = "luatest-hash-luatex-073-LUA52-40-6.lua" }}}}
\stopplacetable

\startplacetable[reference=tab:torture-5,location=page,title={\type{context --jit test.tex} with hash<=64}]
    \scale
      [height=\the\dimexpr\textheight-3\lineheight\relax]
    % [width=\the\dimexpr\textwidth+.5\backspace\relax]
      {\vbox{\ctxlua{moduledata.luatests.showhashing { filename = "luatest-hash-luajittex-073-LUA51-40-6.lua" }}}}
\stopplacetable

\startplacetable[location=page,reference=tab:compare-2,title=More than 5\% difference between 32 byte or 64 byte hashing.]
    \startcombination
        \startcontent
           \scale
              [height=\the\dimexpr\textheight-4\lineheight\relax]
              {\vbox{\ctxlua{moduledata.luatests.showhashing {
                  fileset = {
                      { tag = "40 / 32", filename = "luatest-hash-luatex-073-LUA52.lua" },
                      { tag = "40 / 64", filename = "luatest-hash-luatex-073-LUA52-40-6.lua" },
              } } }}}

        \stopcontent
        \startcaption
            \LUATEX\ (size limit 40)
        \stopcaption
        \startcontent
           \scale
              [height=\the\dimexpr\textheight-4\lineheight\relax]
              {\vbox{\ctxlua{moduledata.luatests.showhashing {
                  fileset = {
                      { tag = "40 / 32", filename = "luatest-hash-luajittex-073-LUA51.lua" },
                      { tag = "40 / 64", filename = "luatest-hash-luajittex-073-LUA51-40-6.lua" },
              } } }}}

        \stopcontent
        \startcaption
            \LUAJITTEX\ (no size limit)
        \stopcaption
    \stopcombination
\stopplacetable

So how does this affect us in document production? It is not that hard to get a
processing rate of a few dozen pages per second on a modern machine, even with
somewhat complex documents, where \XML\ turns into \PDF. However, interactivity
comes somehow with a price when we use \LUAJITTEX. In \CONTEXT\ \MKIV\ we do all
\PDF\ annotations in \LUA\ and that involves assembling dictionaries. Here are
two examples, a destination:

\starttyping
<< /D [ 15 0 R /Fit ] /S /GoTo >>
\stoptyping

and a reference:

\starttyping
/Subtype /Link /Border [ 0 0 0 ] /F 4 /A 16 0 R
\stoptyping

These strings are build with small variations and at some point end up in the \PDF\
file. The same string can end up in the file several times, although sometimes we
can create a reusable object. In the last case we keep them at the \LUA\ end as
reference to such a shareable object, a key in an object reference hash.  Now imagine
that we have some 30K of such references and/or destinations, which indeed happens in
crited documents. In the next two lines we use a \type {*} to show where the
differences are:

\starttyping
<< /D [ * 0 R /Fit ] /S /GoTo >>
/Subtype /Link /Border [ 0 0 0 ] /F 4 /A * 0 R
\stoptyping

If we replace these \type {*} by a number, there are big differences between the
engines with respect to the time needed. This is summarized in the next table.
\footnote {The numbers concern 30K hash creations. The time shown is the average
over 30 runs.}

\starttabulate[|c|c|c|l|]
\FL
\NC \bf \LUA\ 5.2 \NC \bf \LUAJIT\ 2.0 \NC \bf \LUAJIT\ 2.0+5.1 \NC \NR
\FL
\NC 0.096 \NC 0.046 \NC 0.047 \NC \ttx << /D [ * 0 R /Fit ] /S /GoTo >> \NC \NR
\NC 0.054 \NC 6.017 \NC 0.055 \NC \ttx /Subtype /Link /Border [ 0 0 0 ] /F 4 /A * 0 R \NC \NR
\LL
\stoptabulate

Especially the second case behaves bad in \LUAJIT. Say that a result comes out
as:

\starttyping
/Subtype /Link /Border [ 0 0 0 ] /F 4 /A 12 0 R
/Subtype /Link /Border [ 0 0 0 ] /F 4 /A 123 0 R
/Subtype /Link /Border [ 0 0 0 ] /F 4 /A 1234 0 R
\stoptyping

The \LUAJIT\ hasher (more or less) looks at the first~4, last~4, middle~4 and
somewhere a quarter along the string, and uses these sequences for the
calculation, so you can imagine that there are clashes. The \LUA\ 5.1 hasher runs
over part of the string and sees more of the difference. The 5.2 hasher has a
threshold and doesn't hash at all when the length exceeds (by default) 40
characters, which is the case with the second string. Looking at only specific
parts of a string is somewhat arbitrary and what works for one kind of
application is not always good for another.

After these tests we decided that it makes sense to replace the \LUAJIT\ hash
calculation by the traditional \LUA\  one (or at least give users a choice at
startup. The choice of hash is a runtime option:

\starttyping
mtxrunjit --script context --jithash=lua51    ......
mtxrunjit --script context --jithash=luajit20 ......
\stoptyping

For the moment we default to the traditional \LUA\ 5.1 hashing method. Although
it can behave real bad on some large strings we think that chances are low that
this will happen in practice. An overall good performance on strings like the
hyperlink examples is more important. Using the \LUA\ 5.2 method would be even
better but it required a change in the virtual machine and that is not something
we have in mind.

\stopsection

\stopchapter

\stopcomponent

% Luatex manual:
%
% In \LUA\ strings are hashed which makes a test for equality fast and in \LUATEX\
% we benefit from that fact. Starting with \LUA\ 5.2 the hash function is no longer
% hashing strings larger than (by default) 40 characters. Of these at most 32
% characters are hashed in stock \LUA\ but for a string rich environment as \TEX\
% this can lead to many collisions. Therefore we have now set that constant limit
% to 64 characters (so in practice it's now 40 too).
%
% In \LUAJIT\ the hash function is not the same as in \LUA\ and can in some cases
% lead to a significant slowdown. We ran into cases where a \LUAJITTEX\ run was 20
% times slower than a normal \LUATEX\ run while normally such run is 30\% faster.
% For this reason we have replaced the hash code with the \LUA\ 5.1 hash code. This
% change is minimal and gives less collisions. The impact on speed can be neglected.
%
% For \LUAJITTEX\ you can control the hash method:
%
% \starttyping
% --jithash=luajit
% --jithash=lua51
% \stoptyping
%
% The current status of the hash function is available in:
%
% \starttyping
% status.list().luatex_hashtype
% status.list().luatex_hashchars
% \stoptyping
%
% The first one returns \type {lua}, \type{luajit} or \type {lua51} depending on
% the engine. The second one should always return 6. If it returns 5 then you have
% a non|-|optimized binary. Other values are suspicious.
