\documentclass{article}
\usepackage[T1]{fontenc}
\usepackage{hyperref}
\hypersetup{colorlinks}
\usepackage{iftex}

\title{The iftex package}
\author{The \LaTeX\ Project Team\thanks{%
\url{https://github.com/latex3/iftex}}}

\date{\csname ver@iftex.sty\endcsname}

\newcommand\cs[1]{{\ttfamily\textbackslash #1}}
\renewcommand*\descriptionlabel[1]{\makebox[\dimexpr\textwidth][l]{%
                                \normalfont\bfseries #1}}

\begin{document}

\maketitle
\tableofcontents


\section{Introduction}
This original \textsf{iftex} was written as part of the \textsf{bidi}
collection (by the Persian TeX Group / Vafa Khalighi) 
and provided checks for whether a document was being
processed with PDF\TeX, or Xe\TeX, or Lua\TeX. This version recodes
the package and incorporates similar tests from the \textsf{ifetex}
package by Martin Scharrer, the \textsf{ifxetex} package by Will
Robertson, the \textsf{ifluatex} and \textsf{ifvtex}
packages from Heiko Oberdiek and parts of \textsf{ifptex} by Takayuki
Yato.

For each \TeX\ variant engine supported two commands are provided:
\begin{itemize}
\item 
 a conditional, \verb|\iffootex| that is true if the \textsf(footex)
engine (or a compatible extension) is being used.

For compatibility with earlier packages which did not all use the same
naming convention all these conditionals are provided in two forms, a
lowercase name \verb|\iffootex| and a mixed case name \verb|\iffooTeX|.

\item a command \verb|RequireFooTeX| which checks that \textsf{footex}
  is being used, and stops the run with an error message if a
  different engine is detected.
\end{itemize}

\section{Loading the package}
The package can be loaded in the usual way in both Plain \TeX\ and \LaTeX.
\subsection{Loading the package in plain \TeX}
\begin{verbatim}
\input iftex.sty
\end{verbatim}

\subsection{Loading the package in \LaTeX}
\begin{verbatim}
\usepackage{iftex}
\end{verbatim}

\subsection{Loading the package in ini\TeX}
The package assumes no existing macros and may be loaded
during format setup in a format without the plain \TeX\ or \LaTeX\
format being loaded. From an initial ini\TeX\ setup the package may
be loaded as for plain \TeX.

\section{Engine test conditionals}
All the conditionals defined here are used in the same way:
\begin{verbatim}
\ifluatex
  luatex specific code
\else
  code for other engines
\fi
\end{verbatim}



\begin{description}
\item[\cs{ifetex},      \cs{ifeTeX}]
True if an e\TeX\ enabled format is in use. (This is necessarily true
in all \LaTeX\ variants.)
\item[\cs{ifpdftex},    \cs{ifPDFTeX}]
True if PDF\TeX\ is in use (whether writing PDF or DVI), so this is
true for documents processed with both the  \textsf{latex} and
\textsf{pdflatex} commands.
\item[\cs{ifxetex},     \cs{ifXeTeX}]
True if Xe\TeX\ is in use.
\item[\cs{ifluatex},    \cs{ifLuaTeX}]
True if Lua\TeX\ and extensions such as LuaHB\TeX\ are in use.
\item[\cs{ifluahbtex},  \cs{ifLuaHBTeX}]
True if the \textsf{luaharftex} Lua module is available.
This will be true in \textsf{luahbtex} and may be true in
\textsf{luatex} if a binary Lua \textsf{luaharftex} module has been
compiled and is available in Lua's search path.
\item[\cs{ifptex},      \cs{ifpTeX}]
True if any of the p\TeX\ variants are in use.
\item[\cs{ifuptex},     \cs{ifupTeX}]
True if any of the up\TeX\ variants are in use. (\verb|\ifetex| could
be used in addition to distinguish \textsf{uptex} and \textsf{euptex}.)
\item[\cs{ifptexng},    \cs{ifpTeXng}]
True if p\TeX-ng (Asiatic p\TeX) is in use.
\item[\cs{ifvtex},      \cs{ifVTeX}]
True if V\TeX\ is in use.
\item[\cs{ifalephtex},  \cs{ifAlephTeX}]
True if Aleph is in use. (The \textsf{aleph}-based \LaTeX\ command is
\textsf{lamed}.)
\item[\cs{iftutex},     \cs{ifTUTeX}]
This is not strictly an engine variant, but it is true if
\verb|\Umathchardef| is available, which essentially means that it is
true for Lua\TeX\ and Xe\TeX, allowing constructs such as
\begin{verbatim}
\iftutex
  \usepackage{fontspec}
  \setmainfont{TeX Gyre Termes}
  \usepackage{unicode-math}
  \setmathfont{Stix Two Math}
\else
  \usepackage{newtxtext,newtxmath}
\fi
\end{verbatim}

\end{description}

\section{Requiring specific engines}
For each supported engine, the package provides a command
\verb|\Require...| which checks that the document is being processed
with a suitable engine, and stops with an error message if not.

\begin{description}
\item[\cs{RequireeTeX}]
\item[\cs{RequirePDFTeX}]
\item[\cs{RequireXeTeX}]
\item[\cs{RequireLuaTeX}]
\item[\cs{RequireLuaHBTeX}]
\item[\cs{RequirepTeX}]
\item[\cs{RequireupTeX}]
\item[\cs{RequirepTeXng}]
\item[\cs{RequireVTeX}]
\item[\cs{RequireAlephTeX}]
\item[\cs{RequireTUTeX}]
\end{description}


\section{Output mode conditional}
This package also provides an \verb|\ifpdf| conditional that is true
if the format is set up to output in PDF mode rather than DVI. This is
equivalent to the test in the existing \textsf{ifpdf} package.

Unlike the engine tests above this is defined as if by \verb|\newif|
with user-documented commands \verb|\pdftrue| and \verb|\pdffalse| that can
change the boolean value. These would be needed to reset the boolean
if the output mode is reset (for example by setting
\verb|\pdfoutput=0| in PDF\LaTeX).

Unlike the original \textsf{ifpdf} package, the version here also
detects PDF output mode if running in V\TeX.

\section{Additional packages}
This extended \textsf{iftex} is designed to replace the original
\textsf{iftex} and also the packages
\textsf{ifetex},
\textsf{ifluatex},
\textsf{ifvtex},
\textsf{ifxetex},
\textsf{ifpdf}.

This collection includes small packages with these names that include
the main \textsf{iftex} package, and in some cases define additional
commands for increased compatibility. These packages should mean that
authors do not need to change existing documents, although it is
recommended that new documents use the \textsf{iftex} package
directly.


Note that while this package provides basic support for detecting
p\TeX\ (Japanese \TeX) variants and is broadly compatible with the
\textsf{ifptex} package, the \textsf{ifptex} package has many more detailed
tests for p\TeX\ variants and this package does \emph{not} replace the
\textsf{ifptex} (or \textsf{ifxptex}) packages, which are maintained
by their original authors and recommended for Japanese documents that
need fine control over the Japanese \TeX\ system in use.

\section{Compatibility with \textsf{scrbase}}
The \textsf{scrbase} package (which is automatically included in the
popular \textsf{KOMA-Script} classes) by default defines
\verb|\ifpdftex| and \verb|\ifVTeX| with a different syntax.  If you
use the \textsf{scrbase} option \verb|internalonly| then
\textsf{scrbase} will not define these and the definitions as
described here will take effect. This is recommended and will not
affect any \textsf{scrbase} package code as internally
\textsf{scrbase} uses private versions of those commands prefixed with
\verb|\scr@|.  However this package detects if the \textsf{scrbase}
definitions are in effect and if so does not redefine them, for
compatibility with existing documents. The \textsf{iftex} versions
will still be available under the names \verb|\ifPDFTeX| and
\verb|\ifvtex|.

\end{document}
