\documentclass{article}
\usepackage{graphicx}
\title{The \texttt{threeddice} package}
\author{Dan Luecking}
\date{2010/11/30}
\begin{document}

\catcode`_=12
{\catcode`\_=\active \global\let_\sb}

\maketitle
\section{Introduction}

Running \texttt{mpost} on \texttt{threeddice.mp} produces 26 figure files.
Six of them are are \texttt{die3d-1.mps}%
    \footnote{If your \texttt{mpost} version is less that 0.9, you will 
    get figure files with names like \texttt{threeddice.1}. You should 
    update your \TeX{} system.} 
through \texttt{die3d-6.mps}. These are pictures of a single die face 
with the number of dots indicated by the numerical part. Three of these 
would look the same if rotated by multiples of $90$ degrees. The other 
three could look different. It is up to the users to perform this 
rotation if they want a different view.

The twelve files with 2-digit numerical part are 3D pictures of dice with 
two faces showing, as if your were looking directly at a horizontal 
edge. You will have to rotate it $\pm90$ degrees if you want the edge to be 
vertical, or $180$ degrees if you prefer the positions of the top and 
front face to be exchanged.

The eight files with 3-digit numerical suffix are 3D pictures of dice with 
three faces showing, as if your were looking directly at one of the 
corners. They show a top face and a left and right face flanking a 
vertical edge. You can get 16 other views by rotating them $\pm120$ 
degrees. These rotations put one of the side faces on top. 

\section{Modifications}

These figures were taken from an actual right handed die, and are all
the permissable views of such a die, apart from size of dots and the
actual orientation of the patterns of dots. As near as I could determine
from internet searches, there are no official standards for the 
following. 
\begin{itemize}
  \item The 2-spot face has dots in opposite corners, but which two 
        corners? On my die, if the 1-spot is on top, the 2-spot to the 
        left and the 3-spot to the right (figure 
        \texttt{die3d-123.mps}), the dots on the 2-spot face are in 
        upper left and lower right. It seems it could just as well have 
        been the other two corners. 

  \item Similar question for the 3-spot face: three dots in a diagonal 
        between opposite corners, but which two corners? In the previous
        example (figure \texttt{die3d-123.mps}), the diagonal is from 
        lower left to upper right.

  \item The 6-spot face has two lines of three, parallel to two edges, 
        but which two edges? In \texttt{die3d-246.mps} the lines of dots are 
        horizontal, but it seems they could just as well have been 
        vertical.
\end{itemize}

If you want some other orientation of these faces you will need to edit 
\texttt{threeddice.mp}: for the 2-spot face exchange the definitions of 
\texttt{pips2} and \texttt{pips7}; for the 3-spot face exchange 
\texttt{pips3} and \texttt{pips8}; for the 6-spot face exchange 
\texttt{pips6} and \texttt{pips11}.

You can also change the size of the dots by giving a different value to 
\texttt{dot_diam}. The dice have rounded corners, you can change their
radius by changing \texttt{corner_rad}. The dots are placed so there is 
at least \texttt{face_margin} of space between the dots and the straight 
edge of the face. You can change this variable as well. 

For other effects, you are on your own. A common requirement is that 
faces with fewer dots have larger dots. This is relatively easy to do, 
but requires changing the definitions of the 9 picture variables 
\texttt{pips1} through \texttt{pips11}. 

Naturally, the \texttt{graphics} and \texttt{graphicx} packages let you 
scale these as you want. If you would prefer to change the unscaled 
sizes, edit \texttt{threeddice.mp} to change \texttt{die_size} from 1cm to 
the desired size.

Finally, to get views of a left-handed die, simply reflect the views of 
the right-handed die.

\section{Appendix}

\parindent 0pt
These are the six single faces:

\bigskip
\includegraphics{die3d-1.mps}\quad
\includegraphics{die3d-2.mps}\quad
\includegraphics{die3d-3.mps}\quad
\includegraphics{die3d-4.mps}\quad
\includegraphics{die3d-5.mps}\quad
\includegraphics{die3d-6.mps}

\bigskip
\filbreak
The two-faced dice are ugly, so let me first show you the eight 
three-faced dice:

\bigskip
\includegraphics{die3d-123.mps}\quad
\includegraphics{die3d-135.mps}\quad
\includegraphics{die3d-142.mps}\quad
\includegraphics{die3d-154.mps}

\bigskip
\includegraphics{die3d-246.mps}\quad
\includegraphics{die3d-263.mps}\quad
\includegraphics{die3d-365.mps}\quad
\includegraphics{die3d-456.mps}

\bigskip
\filbreak
Finally, the twelve two-faced dice:

\bigskip
\includegraphics{die3d-12.mps}\quad
\includegraphics{die3d-13.mps}\quad
\includegraphics{die3d-14.mps}\quad
\includegraphics{die3d-15.mps}\quad
\includegraphics{die3d-23.mps}\quad
\includegraphics{die3d-24.mps}

\bigskip
\includegraphics{die3d-26.mps}\quad
\includegraphics{die3d-35.mps}\quad
\includegraphics{die3d-36.mps}\quad
\includegraphics{die3d-45.mps}\quad
\includegraphics{die3d-46.mps}\quad
\includegraphics{die3d-56.mps}

\filbreak
\end{document}
