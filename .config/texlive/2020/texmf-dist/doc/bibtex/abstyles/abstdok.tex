								%-*-LaTeX-*-
\documentstyle[twoside,german,a4c]{article}
\pagestyle{headings}
\frenchspacing
\setlength{\doublerulesep}{0pt}

\title{Neue \BibTeX-Style-Files: Die {\em adaptable family}}
\author{Hans-Hermann Bode\thanks{Postanschrift: Arbeitsgruppe Systemforschung,
  Universit"at Osnabr"uck, Artilleriestra"se 34, D-W-4500 Osnabr"uck; e-mail:
  {\tt hhbode@dosuni1.bitnet}.}}
\date{08. April 1992}

\newcommand{\BibTeX}{{\rm B\kern-.05em{\sc i\kern-.025em b}\kern-.08em
    T\kern-.1667em\lower.7ex\hbox{E}\kern-.125emX}}

%-----------------------------------------------------------------------------

\begin{document}

\maketitle
%\tableofcontents
%\listoffigures
%\listoftables

\begin{abstract}\footnotesize
  Es wird eine neue Familie von \BibTeX-Style-Files vorgestellt, die von den
  Standard-Style-Files abgeleitet ist und sich durch nachtr"agliche
  Anpa"sbarkeit von Layout und F"ullw"ortern seitens des Benutzers oder der
  Benutzerin\footnote{Im folgenden gilt die Fu"snote auf Seite~v von Knuths
  fundamentalem Werk~\cite{art1} sinngem"a"s.} auszeichnet. Das Paket
  umfa"st neben den eigentlichen Stilen Dateien mit Definitionen, die
  automatisch eine englische oder deutsche Bibliographie mit ausgeschriebenen
  oder abgek"urzten F"ullw"ortern erzeugen und au"serdem die volle
  Kompatibilit"at bestehender \BibTeX-Datenbanken zu den neuen Stilen sowie
  umgekehrt die Vertr"aglichkeit neuer Datenbanken zu den alten Stilen
  herstellen.
\end{abstract}

%-----------------------------------------------------------------------------

Mit {\BibTeX} steht dem \LaTeX- wie dem plain-\TeX-Anwender ein
au"serordentlich leistungsf"ahiges Werkzeug zur Verf"ugung, um
Bibliographien zu Artikeln, Berichten, B"uchern und {\em literate
programs\/} zu erstellen. Mu"ste man fr"uher m"uhsam alle zitierten
Literaturstellen zusammensuchen, alle Angaben -- oft immer wieder dieselben
-- per Hand eintippen (m"oglichst, ohne Fehler bei Seitenangaben,
Bandnummern, Jahreszahlen usw.~zu machen) und sich schlie"slich auch noch um
Layout-Fragen k"ummern, so beschr"ankt sich diese Arbeit mit {\BibTeX} auf
die einmalige Eingabe der bibliographischen Informationen in eine
Literaturdatenbank und das Aufrufen von \verb|\cite|-Kommandos mit einem
Schl"ussel f"ur das zu zitierende Werk. {\BibTeX} stellt dann die
Literaturliste zusammen und sorgt auch f"ur ein konsistentes Aussehen
derselben.

{\BibTeX} ist "`offizieller"' Bestandteil des \LaTeX-Makropakets und von
Lamport in~\cite{latex} beschrieben, eine deutsche Fassung findet man etwa
bei Kopka~\cite{kopka-einf}. Dar"uberhinaus erlaubt das zur
Standard-\BibTeX-Distribution geh"orige Makropaket {\tt btxmac.tex} auch die
Verwendung unter plain-\TeX\, und zwar mit der gleichen
Benutzerschnittstelle (dieses Makropaket ist auch mit dem {\tt WEB}-System
f"ur strukturierte Dokumentation kompatibel, so da"s \BibTeX\ auch beim {\em
literate programming\/} eingesetzt werden kann). Oren Patashnik, der Autor
des \BibTeX-Programms, legt mit~\cite{bibtex-doc} ein Manual f"ur
\BibTeX-Benutzer vor, das im Gegensatz zu den zuvor genannten Stellen die
jeweils aktuelle Version dokumentiert und das in einer Datei {\tt
btxdoc.tex} jeder \BibTeX-Implementation beigegeben sein sollte. Die
Kenntnis wenigstens einer der angegebenen Quellen wird hier vorausgesetzt
und nicht weiter auf allgemeine Grundlagen der Benutzung von {\BibTeX}
eingegangen.

Das Schlu"skapitel von Patashniks Manual -- herausgegeben als separates
Dokument~\cite{bibtex-hak} -- beschreibt M"oglichkeiten der Erzeugung von
Style-Files, die das Layout der Bibliographie bestimmen. Vier solcher
Style-Files ({\tt plain}, {\tt unsrt}, {\tt alpha} und {\tt abbrv}) geh"oren
zum Standardumfang von {\BibTeX}. Weitere Styles -- wie {\tt siam}, {\tt
ieeetr} oder {\tt apalike} -- wurden von Patashnik u.\,a.~erstellt, in
erster Linie, um den Anforderungen bestimmter Zeitschriftenherausgeber zu
entsprechen. Die Erzeugung von solchen Styles, auch wenn es nur um relativ
geringf"ugige Ab"anderungen vorhandener Stile geht, ist allerdings ein recht
m"uhsamer Vorgang (man mu"s Programme in einer Spezialsprache mit
Postfixnotation schreiben), der die meisten {\BibTeX}-Anwender "uberfordern
d"urfte. Aus diesem Grund entstand die {\em adaptable family}, die eine
einfache Anpassung an pers"onliche oder fremde Vorstellungen erlaubt.

\section*{Die Styles} %-------------------------------------------------------

\begin{table}\small\centering
  \caption{\sf "Ubersicht der Stile mit ihren Attributen.\strut}
  \label{tab:stile}
  \medskip
  \begin{tabular}{lcccccc}
    \hline\hline
    Stil & numerisch & sortiert & initial & klein & kurz & titellos\\
    \hline
    {\tt aplain}  & ja   & ja   & nein & ja & nein & nein\\
    {\tt aunsrt}  & ja   & nein & nein & ja & nein & nein\\
    {\tt aalpha}  & nein & ja   & nein & ja & nein & nein\\
    {\tt aabbrv}  & ja   & ja   & ja   & ja & ja   & nein\\
    {\tt anotit}  & ja   & ja   & ja   & ja & ja   & ja\\
    {\tt aunsnot} & ja   & nein & ja   & ja & ja   & ja\\
    \hline\hline
  \end{tabular}
\end{table}

Das grunds"atzliche Aussehen der Bibliographie wird auch bei den anpa"sbaren
Versionen durch den jeweiligen, im \verb|\bibliographystyle| der \TeX-Quelle
gew"ahlten Stil festgelegt. Dieser bestimmt im wesentlichen, welche
Informationen "uberhaupt auszugeben sind und in welcher Reihenfolge diese
angeordnet werden sollen; dar"uber hinaus sind f"ur jeden Stil eine Reihe von
festen, unver"anderlichen Merkmalen vorgegeben, die in Tabelle~\ref{tab:stile}
f"ur die sechs Stile der {\em adaptable family\/} zusammengefa"st sind.

Alle Stile bis auf {\tt aalpha} erzeugen numerische Marken zur Kennzeichnung
der Zitate ("`[1]"', "`[2]"' usw.), w"ahrend {\tt aalpha} die Marken
alphanumerisch, i.\,allg.~aus K"urzeln f"ur die Autorennamen und
Erscheinungsjahr, zusammensetzt (z.\,B.~"`[knu73]"'). Bei {\tt aunsrt} und
{\tt aunsnot} erscheint die Bibliographie in der Reihenfolge der Zitate, bei
den anderen Stilen wird sie nach Autorennamen, Erscheinungsjahren
usw.~sortiert. Das Attribut "`initial"' zeigt an, ob die Vornamen der
Autoren abgek"urzt werden oder nicht; das erstere ist der Fall f"ur {\tt
aabbrv}, {\tt anotit} und {\tt aunsnot}, die zudem im Unterschied zu den
"ubrigen Stilen als "`kurz"' gekennzeichnet sind (die Bedeutung dieses
Attributs wird im n"achsten Abschnitt erkl"art). Alle Stile setzen
Gro"sbuchstaben, die in Titeln von Zeitschriftenartikeln -- sofern diese
"uberhaupt ausgegeben werden -- vorkommen und nicht am Satzanfang stehen, in
Kleinschreibung um (Attribut "`klein"'); dies ist zwar f"ur deutsche Titel
nicht geeignet, l"a"st sich aber leicht unterdr"ucken (s.~die Er"orterung am
Ende dieses Artikels). Schlie"slich zeichnen sich die Stile {\tt anotit} und
{\tt aunsnot} dadurch aus, da"s die Titel von Zeitschriftenartikeln
unterdr"uckt werden, wie es etwa in der physikalischen Literatur
durchg"angig "ublich ist.

Insgesamt entsprechen die Stile {\tt aplain}, {\tt aunsrt}, {\tt aalpha} und
{\tt aabbrv} -- wie auch aus der Namensgebung hervorgeht -- den vier
Standardstilen von \BibTeX\ (Patashniks {\em plain family\/}), w"ahrend {\tt
anotit} und {\tt aunsnot} hinzugef"ugt wurden, um eine L"ucke zu schlie"sen.

\section*{Anpassung von Layout und F"ullw"ortern} %---------------------------

Die Anpassung der {\tt a}-Stile erfolgt durch Definition bestimmter
\TeX-Makros; andererseits funktionieren die Stile nicht, wenn diese Makros
undefiniert sind. Die Definition geschieht zweckm"a"sigerweise "uber ein
\verb|@PREAMBLE|-Kommando, das sich in einer der bibliographischen
Datenbanken befindet (vgl.~das n"achste Kapitel). Die {\tt a}-Styles
definieren einen Schalter \verb|abfull|, der in Abh"angigkeit davon, ob es
sich um einen "`Kurzstil"' handelt oder nicht (s.~Tab.~\ref{tab:stile}), den
Wert \verb|false| oder \verb|true| bekommt. So k"onnen sich die Makros
automatisch darauf einstellen, F"ullw"orter abzuk"urzen oder voll
auszuschreiben.

\begin{table}\small\centering
  \caption{\sf Zusammenstellung der verwendeten Makros.\strut}
  \label{tab:makros}
  \medskip
  \begin{tabular}{lp{8cm}}
    \hline\hline
    Makroname         & Bedeutung\\
    \hline
    \verb|\abtype|    & Definiert die Schriftauszeichnung einzelner Felder
                        eines Bibliographieeintrags.\\
    \verb|\abphrase|  & Legt die F"ullw"orter fest.\\
    \verb|\abmonth|   & Enth"alt die Monatsnamen.\\
    \verb|\abedition| & Stellt Bezeichnungen f"ur Auflagen eines Buches zur
                        Verf"ugung ("`Erste"', "`Zweite"', "`Dritte"' usw.).\\
    \verb|\abchapter| & Gibt Namen f"ur bestimmte Segmente eines
                        Schriftst"ucks zur"uck ("`Abschnitt"', "`Absatz"',
                        "`Anhang"' oder "`Teil"').\\
    \hline\hline
  \end{tabular}
\end{table}

Die fraglichen Makros sind in Tab.~\ref{tab:makros} zusammengestellt. Alle
Makronamen beginnen mit {\tt ab}. W"ahrend \verb|\abtype| f"ur die
Schriftauszeichnung verantworlich ist, haben alle anderen Makros mit
F"ullw"ortern zu tun und erlauben sprachspezifische Anpassungen sowie
Unterscheidung zwischen "`Kurz-"' und "`Vollstilen"'. Es folgen detaillierte
Beschreibungen der Makros mit ihren Parametern.

\subsection*{{\tt abtype}}

\begin{table}\small\centering
  \caption{\sf Der erste Parameter von {\tt abtype}.\strut}\label{tab:abtype}
  \medskip
  \begin{tabular}{rp{5cm}}
    \hline\hline
    Code & Bedeutung\\
    \hline
      0  & Namen von Autoren und Editoren\\
      1  & Titel von B"uchern und Reihen\\
      2  & Zeitschriftentitel\\
      3  & Bandnummern von Zeitschriften\\
      4  & Heftnummern von Zeitschriften\\
      5  & Datumsangaben\\
    \hline\hline
  \end{tabular}
\end{table}

Die Layoutsteuerung wird "uber das Makro \verb|\abtype| vorgenommen. Die {\tt
a}-Stile schreiben die meisten Felder eines Eintrags indirekt "uber einen
Aufruf von \verb|\abtype| in die Bibliographie, wobei zwei Argumente
"ubergeben werden: eine Codenummer f"ur die Art des Feldes und der in das
Feld einzutragende Text. Dieser kann dann vom Makro beliebig formatiert
zur"uckgegeben werden.  Die Codewerte f"ur den ersten Parameter von
\verb|\abtype| findet man in Tab.~\ref{tab:abtype}.

Durch eine Definition der Form
\begin{verbatim}
\def\abtype##1##2{%
\ifcase##1{\sc##2}\or"`##2"'\or{\em##2\/}\or{\bf##2}\or%
(##2)\or(##2)\else##2\fi}}
\end{verbatim}
also erreichte man beispielweise, da"s Autorennamen in {\sc Kapit"alchen},
Buchtitel in "`Anf"uhrungszeichen"' und Zeitschriftentitel {\em kursiv\/}
gesetzt werden und da"s Bandnummern von Zeitschriften {\bf fett} erscheinen,
w"ahrend Heftnummern ebenso wie Datumsangaben geklammert werden.
Vorsichtshalber wurde noch eine \verb|\else|-Klausel angef"ugt, die bewirkt,
da"s bei Aufruf des Makros mit einem undefinierten Codewert das zweite
Argument unformatiert zur"uckgegeben wird (f"ur die Original-{\tt a}-Stile
w"are dies nicht erforderlich).

\subsection*{{\tt abphrase}}

\begin{table}\small\centering
  \caption{\sf Der Parameter von {\tt abphrase}.\strut}\label{tab:abphrase}
  \medskip
  \begin{tabular}{rp{9.5cm}}
    \hline\hline
    Code & Bedeutung\\
    \hline
      0  & Bindewort zwischen zwei oder vor dem letzten Namen einer Liste
           ("`und"')\\
      1  & Satzzeichen vor dem letzten Bindewort einer Namensliste (im
           Deutschen nichts, im Englischen ein Komma)\\
      2  & Ersatz f"ur Namen von Mitautoren und Mitherausgebern ("`u.\,a."' oder
           "`et~al."')\\
      3  & Bezeichnung f"ur mehrere Herausgeber (z.\,B.~"`Editoren"')\\
      4  & Bezeichnung f"ur einen Herausgeber ("`Editor"')\\
      5  & Wort nach der Bandnummer einer Reihe oder Kreuzreferenz ("`aus"')\\
      6  & Wort nach der Ausgabennummer einer Reihe ("`in"')\\
      7  & Wort vor Querverweisen auf B"ucher, Artikel usw.~("`In"')\\
      8  & Wort vor einer Bandnummer mitten im Satz ("`volume"' auf
           Englisch, aber "`Band"' auf Deutsch)\\
      9  & dto.~am Satzanfang ("`Volume"' bzw.~erneut "`Band"')\\
     10  & Wort vor einer Reihennummer mitten im Satz ("`Nummer"')\\
     11  & dto. am Satzanfang (im Deutschen dasselbe)\\
     12  & Wort hinter einer Auf"|lagennummer ("`Auf"|lage"')\\
     13  & Wort vor mehreren Seitennummern ("`Seiten"')\\
     14  & Wort vor einer einzelnen Seitennummer ("`Seite"')\\
     15  & Wort vor einer Kapitelnummer ("`Kapitel"')\\
     16  & Bezeichnung f"ur einen Forschungsbericht ("`Technical Report"')\\
     17  & Bezeichnung f"ur eine Diplomarbeit ("`Master's thesis"')\\
     18  & Bezeichnung f"ur eine Doktorarbeit ("`PhD Thesis"')\\
    \hline\hline
  \end{tabular}
\end{table}

Der gr"o"ste Teil der F"ullw"orter wird durch \verb|\abphrase| abgedeckt.
Das Makro hat einen Parameter, der wieder eine Codenummer darstellt.
Die vorkommenden Werte zeigt Tab.~\ref{tab:abphrase}.

Ein Beispiel f"ur eine geeignete Definition von \verb|\abphrase| mit
ausgeschriebenen, deutschen F"ullw"ortern w"are:
\begin{verbatim}
\def\abphrase##1{%
\ifcase##1{ und }\or{}\or{ und andere}\or%
{ (Herausgeber)}\or{ (Herausgeber)}\or%
{ aus }\or{ in }\or{In }\or%
{Band}\or{Band}\or{Nummer}\or{Nummer}\or%
{ Auf"|lage}\or{Seiten}\or{Seite}\or{Kapitel}\or%
{Bericht}\or{Diplomarbeit}\or{Dissertation}\fi}
\end{verbatim}

\subsection*{{\tt abmonth}}

Einfacher als die vorangegangenen Makros ist \verb|\abmonth|: Es wird mit
einer Zahl zwischen 1~und~12 als Argument aufgerufen und sollte die
Bezeichnung f"ur den entsprechenden Monat zur"uckgeben; also etwa
\begin{verbatim}
\def\abmonth##1{\ifcase##1\or Januar\or Februar\or M"arz\or
  April\or Mai\or Juni\or Juli\or August\or September\or
  Oktober\or November\or Dezember\fi}
\end{verbatim}
Man beachte, da"s vor \verb|Januar| bereits ein \verb|\or| stehen mu"s, da
dieser Monat den Code~1 hat und nicht~0. Desweiteren gilt, da"s die
Definition von \verb|\abmonth| nur dann einen Effekt hat, wenn in den
Bibliographiedatenbanken von den vordefinierten Strings {\tt jan}, {\tt
feb}, \dots, {\tt dec} Gebrauch gemacht wird -- es sei denn, man ruft das
Makro direkt auf, wovon man jedoch im Hinblick auf m"ogliche
Kompatibilit"atsprobleme mit anderen Styles oder zuk"unftigen Versionen der
anpa"sbaren Stile Abstand nehmen sollte.

\subsection*{{\tt abedition}}

Ein weiteres Makro mit einem numerischen Wert als Parameter ist
\verb|\abedition|. Die "ubergebene Zahl bezeichnet eine Auf"|lagennummer und
es sollte ein entsprechender String zur"uckgegeben werden, z.\,B.:
\begin{verbatim}
\def\abedition##1{\ifcase##1\or Erste\or Zweite\or Dritte\or
  Vierte\or F"unfte\or Sechste\or Siebte\or Achte\or
  Neunte\or Zehnte\else?\fi}
\end{verbatim}
In den anpa"sbaren Stilen existieren vordefinierte Strings {\tt first}, {\tt
second}, \dots, {\tt tenth}, die man aus den gleichen Gr"unden wie oben f"ur
das {\tt EDITION}-Feld der Bibliographieeintr"age verwenden sollte. Ein
Problem ergibt sich, wenn eine Auflagennummer gr"o"ser als~10 ben"otigt
wird; in diesem Fall sollte man das Makro entsprechend erweitern und
zus"atzliche Strings durch {\tt @STRING}-Kommandos definieren.

\subsection*{{\tt abchapter}}

Schlie"slich ist noch ein Makro mit einem Code zwischen 0~und~3 f"ur
Bezeichnungen verschiedener Segmente eines Dokuments zu definieren. Die
Bedeutung ergibt sich unmittelbar aus dem Beispiel
\begin{verbatim}
\def\abchapter##1{\ifcase##1Abschnitt\or Absatz\or
Anhang\or Teil\fi}
\end{verbatim}
Auch f"ur den Aufruf dieses Makros gibt es vordefinierte Strings, die man in
den Datenbanken tunlichst benutzen sollte; sie hei"sen {\tt section}, {\tt
paragraph}, {\tt appendix} und {\tt part} und sind gedacht f"ur den Einsatz
im {\tt TYPE}-Feld mancher Eintr"age, mit welchem sich das {\tt
CHAPTER}-Feld umwidmen l"a"st.

\section*{Benutzungshinweise} %-----------------------------------------------

Zum Gl"uck braucht man die im letzten Kapitel vorgestellten Makros nicht
selbst einzugeben, sie sind vielmehr bereits in der Datei {\tt
apreamble.tex} enthalten, die zum Lieferumfang des Pakets geh"ort. Sie
stellt in Abh"angigkeit vom \verb|abfull|-Schalter ausgeschriebene oder
abgek"urzte F"ullw"orter zur Verf"ugung.  Mehr noch, diese Datei enth"alt
neben den deutschen F"ullw"ortern auch "aquivalente Definitionen in
englischer Sprache. Die Auswahl zwischen den beiden Sprachversionen wird
automatisch getroffen, und zwar nach folgenden Kriterien:
\begin{itemize}
  \item Ist das Makropaket {\tt german.sty} {\em nicht\/} geladen, wird die
englische Version genommen.
  \item Wird dagegen festgestellt, da"s {\tt german.sty} eingelesen wurde,
h"angt die Sprachauswahl vom eingestellten \verb|\language|-Wert ab:
  \begin{itemize}
    \item Ist dieser gleich \verb|\german| oder \verb|\austrian|, wird die
  deutsche Version gew"ahlt.
    \item In allen anderen F"allen ist es wieder die englische.
  \end{itemize}
\end{itemize}
Auf diese Weise sollten zumindest alle diejenigen zufriedengestellt sein,
die entweder nur englische Dokumente verfassen oder unter Benutzung von {\tt
german.sty} sowohl englische als auch deutsche Texte schreiben. Nat"urlich
steht es jedem frei, {\tt apreamble.tex} um zus"atzliche Sprachen zu
erweitern oder die mittels \verb|\abtype| vorgenommenen Formatierungen nach
eigenem Gutd"unken abzu"andern.

Damit der oben vorgestellte Mechanismus zur Sprachauswahl funktioniert, mu"s
die Datei {\tt german.sty} -- wenn "uberhaupt -- {\em vor\/} {\tt
apreamble.tex} eingelesen werden. Dies sollte im Zusammenhang mit \LaTeX\
kein Problem darstellen, da \verb|german| f"ur gew"ohnlich als
Dokumentstiloption verwendet wird. Wie oben erw"ahnt l"a"st sich \BibTeX\
aber auch mit plain-\TeX\ nutzen, wenn das Makropaket {\tt btxmac.tex}
eingebunden wird; hier ist dann sicherzustellen, da"s der Befehl
\verb|\input german.sty| vor \verb|\input apreamble| (bzw., wie im n"achsten
Absatz erl"autert, vor \verb|\bibliography{...}|) erscheint. Alle zur {\em
adaptable family\/} geh"origen Makros sind "ubrigens so abgefa"st, da"s sie
sowohl mit \LaTeX\ als auch mit plain-\TeX\ funktionieren; insbesondere
vertragen sie sich auch mit dem {\tt WEB}-System f"ur strukturierte
Dokumentation.

Weiterhin ist es nat"urlich erforderlich, da"s {\tt apreamble.tex} vor der
eigentlichen Bibliographie eingelesen wird; auch daf"ur steht eine implizite
L"osung bereit. Es ist ohnehin vorteilhaft, wenn man eine oder mehrere {\tt
bib}-Dateien mit allgemeinen Definitionen (haupts"achlich
\verb|@STRING|-Kommandos) anlegt und diese stets am Anfang der
\verb|\bibliography|-Liste aufz"ahlt. Zum Lieferumfang der {\em adaptable
family\/} geh"oren zwei solche Dateien, {\tt jourfull.bib} und {\tt
jourabbr.bib}, die Namen f"ur Zeitschriften -- einmal in ausgeschriebener
und zum anderen in abgek"urzter Form -- definieren, und zwar gerade die,
welche in den Standard-Bibliographiestilen (der {\em plain family\/})
bereits vordefiniert sind\footnote{Der Autor der {\em adaptable family\/}
war der Meinung, da"s die Auswahl dieser Zeitschriften zu speziell auf ein
Fach bezogen ist, und hat die Definitionen daher in seinen Stildateien
weggelassen. Unter Benutzung dieser beiden Dateien ist jedoch die volle
Kompatibilit"at von Datenbanken, die f"ur die Standardstile erstellt wurden,
zu den anpa"sbaren Stilen gew"ahrleistet.}. Beide Dateien laden {\tt
apreamble.tex} automatisch am Anfang der Bibliographie. Alles, was man zur
Nutzung der anpa"sbaren Stile -- abgesehen vom eventuellen Laden von {\tt
btxmac} -- innerhalb eines Dokuments zu tun hat, beschr"ankt sich somit auf
zwei Kommandos, die typischerweise so aussehen:
\begin{verbatim}
\bibliography{jourfull,user1,user2,user3}\bibliographystyle{aplain}
\end{verbatim}
Dabei sind \verb|user1|, \verb|user2| und \verb|user3| irgendwelche
Benutzerdatenbanken, \verb|jourfull| kann nat"urlich auch durch
\verb|jourabbr| ersetzt werden (mu"s aber immer am Anfang der Liste stehen!)
und statt \verb|aplain| kann selbstverst"andlich ein beliebiger anderer
anpa"sbarer Stil verwendet werden. Es ist zu betonen, da"s die Auswahl
"`ausgeschriebene oder abgek"urzte Zeitschriftennamen"' nicht vom
Bibliographiestil bestimmt wird, sondern durch Angabe von \verb|jourfull|
oder \verb|jourabbr| erfolgt.

Au"serdem definieren die beiden Dateien {\tt jourfull.bib} und {\tt
jourabbr.bib} Strings namens \verb|ifger|, \verb|else| und \verb|fi|, die es
auf einfache Weise erm"oglichen, innerhalb der Datenbanken bei einzelnen
Feldern von der automatischen Sprachauswahl zu profitieren. Beispielsweise
k"onnte damit das \verb|NOTE|-Feld eines Eintrags etwa folgendes enthalten:
\begin{verbatim}
ifger # "Wird demn{\"a}chst erscheinen." # else # "To be published." # fi
\end{verbatim}
Wie man unschwer erkennt, handelt es sich um eine bedingte Anweisung:
\verb|ifger| sorgt daf"ur, da"s der nachstehende Text ausgegeben wird, wenn
die automatische Sprachauswahl sich f"ur "`deutsch"' entschieden hat,
\verb|else| leitet die Alternative f"ur den gegenteiligen Fall ein und
\verb|fi| schlie"st die ganze Konstruktion ab. Der Operator~\verb|#|
verkettet die einzelnen Teile zu einem von \TeX\ interpretierbaren
Gesamtstring.

Gelegentlich mag es vorkommen, da"s man die neuen, auf die {\em adaptable
family\/} zugeschnittenen Datenbanken mit einem der Standardstile verwenden
will. Dies st"o"st auf Schwierigkeiten, da die anpa"sbaren Stile einige
zus"atzliche Strings definieren, von denen man sicherlich Gebrauch gemacht
hat (genauer: machen sollte). Es steht aber eine Datei {\tt acompat.bib} zur
Verf"ugung, die alle zus"atzlichen Strings nachtr"aglich definiert. Bindet
man diese noch vor {\tt jourfull.bib} oder {\tt jourabbr.bib} (die man f"ur
die oben beschriebenen Mechanismen weiterhin braucht) in die
\verb|\bibliography|-Liste ein, so steht einer Nutzung der Standardstile
etwa in der Form
\begin{verbatim}
\bibliography{acompat,jourfull,user1,user2,user3}\bibliographystyle{plain}
\end{verbatim}
nichts mehr im Wege. Nat"urlich funktioniert die automatische Sprachauswahl
jetzt nicht mehr bei den F"ull"-w"or"-tern, sondern nur noch bei den
"`handgemachten"' \verb|ifger|-Konstruktionen.

Abschlie"send noch eine Anmerkung zur Gro"s-/Kleinschreibung von
Zeitschriftentiteln. Die Stile der {\em plain family\/} setzen solche stets
in Kleinschreibung um; so wird z.\,B. aus
\begin{verbatim}
TITLE="A Horror Story about Integration Methods"
\end{verbatim}
die Titelangabe "`A horror story about integration methods"'. Es wurde
verschiedentlich die Meinung vertreten, da"s ein deutscher Bibliographiestil
diese Umsetzung abschalten sollte, da sie f"ur deutsche Titel nicht ad"aquat
ist. Der Autor der {\em adaptable family\/} schlie"st sich dieser Meinung
nicht an (und hat die Umsetzung daher beibehalten): Gro"s-/Kleinschreibung
von Zeitschriftentiteln ist nicht eine Frage der Sprache, in der das
Dokument gesetzt ist, sondern eine Funktion der Sprache des Titels selbst.
So sollte die Eingabe
\begin{verbatim}
TITLE="Graphische Sprachelemente in ALGOL 68"
\end{verbatim}
{\em immer\/} zu der Ausgabe "`Graphische Sprachelemente in ALGOL~68"'
f"uhren -- auch in einem englischen Dokument; ebensowenig ist etwas dagegen
einzuwenden, wenn im obigen Beispiel mit dem englischen Titel die Umsetzung
auch in einem deutschen Dokument stattfindet.

Es ist also {\em bei den einzelnen Titelangaben\/} daf"ur zu sorgen, da"s die
Umsetzung ggf.~verhindert wird. Dies ist aber -- bei den Standard- wie bei
den anpa"sbaren Stilen -- leicht m"oglich, wie im
\BibTeX-Manual~\cite{bibtex-doc} beschrieben wird: Man schlie"se den
gesamten Titel in ein zus"atzliches Paar geschweifter Klammern ein. Also
bewirkt
\begin{verbatim}
TITLE="{Graphische Sprachelemente in ALGOL 68}"
\end{verbatim}
genau das Gew"unschte (die G"ansef"u"schen z"ahlen als "au"seres Klammerpaar
und k"onnten auch durch geschweifte Klammern ersetzt werden, die
zus"atzlichen Klammern sind aber trotzdem erforderlich). Statt dessen
k"onnte man auch nur die gro"szuschreibenden W"orter oder Buchstaben
klammern, z.\,B.
\begin{verbatim}
TITLE="{G}raphische {S}prachelemente in {ALGOL} 68"
\end{verbatim}
Diese Technik wendet man manchmal auch in englischen Titeln an, etwa in
\begin{verbatim}
TITLE="Stiff {ODE} Solvers: A Review of Current and Coming Attractions"
\end{verbatim}
so da"s sich der Titel "`Stiff ODE solvers: A review of current and coming
attractions"' ergibt (das Wort nach dem Doppelpunkt bleibt ohne besondere
Ma"snahme gro"sgeschrieben).

%-----------------------------------------------------------------------------

\footnotesize\bibliography{jfull,inf}\bibliographystyle{aplain}
%\addcontentsline{toc}{section}{\refname}

\end{document}
