
%Syntaxe : 
%\BSS{psbrace}[Options](A)(B)\AC{\TFRGB{texte}{text}}

\SbSbSSCT{Dans un environnement pspicture}{Braces in  pspicture}


\begin{tabular}{|c|c|} \hline  
 \begin{pspicture}[shift=*](-.5,-0.5)(5,4.5)
\psaxes[xticksize=4,yticksize=4,axesstyle=frame](5,4)
\pnode(1,1){B}\pnode(3,2){C}
\psbrace(B)(C){\TFRGB{texte}{text}}
\psbrace(4,3)(1,3){}
\psdots[dotstyle=*,linecolor=blue,dotscale=2](B) \nput{45}{B}{B} 
\psdots[dotstyle=*,linecolor=blue,dotscale=2](C) \nput{45}{C}{C}%
 \end{pspicture} 
& 
\parbox{3.5cm}{ 
\BS{}pnode(1,1)\AC{B} \BS{}pnode(3,2)\AC{C}

\BSS{psbrace}(B)(C)\AC{\TFRGB{texte}{text}} \BSI{psbrace}{pstricks-add} 

\bigskip
\textbf{\BS{psbrace}}(4,3)(1,3)\AC{} }
\\ \hline 
\end{tabular} 

%\begin{center}
%
%\end{center}
 
%\bigskip

 
% \psbrace[bracePos=0.25,nodesepB=10 pt,rot=90](0,2)(\linewidth,2){\fbox{Text}}
\bigskip
\begin{tabular}{|c|c|}\hline 
\multicolumn{2}{|c|}{\TFRGB{Rôle de l'astérisque}{Role of the asterisk} \footnotemark[1] } \\
\hline 
 \begin{pspicture}(-.5,-.5)(4,3)
 \psaxes[xticksize=2.5,yticksize=4,axesstyle=frame](4,2.5)
 \psbrace[braceWidth=.5cm,fillcolor=yellow](1,1)(3,2){}
 \end{pspicture}  &  
 \begin{pspicture}(-.5,-.5)(4,3)
 \psaxes[xticksize=2.5,yticksize=5,axesstyle=frame](4,2.5)
 \psbrace*[braceWidth=.5cm,fillcolor=yellow](1,1)(3,2){}
\end{pspicture} \\ \hline 
 \BS{}psbrace(1,1)(3,2)\AC{} &
  \BS{}psbrace*(1,1)(3,2)\AC{}  \\
\hline 
\end{tabular} 

\footnotetext[1]{braceWidth=.5cm,fillcolor=yellow}

\SbSbSSCT{Dans le texte}{The brace in the text}
 \TFRGB{le noeud  A est ici \pnode(0,0){A} et le noeud B est ici}{the node  A is here \pnode(0,0){A} and the node  B is here} \pnode(0,0){B} \BS{}psbrace(A)(B)\AC{\TFRGB{texte}{text}}
\psbrace(A)(B){\TFRGB{texte}{text}}
\smallskip

{\red \TFRGB{L'accolage n'a pas de dimension}{The brace has no dimension !}}

\vspace{1cm}


\pnode(0,0){A} \TFRGB{ici, se trouve le noeud}{here is the node}  A 

\BS{}vspace\AC{1cm}
\vspace{1cm}

 \pnode(0,0){B} \TFRGB{ici, se trouve le noeud}{here is the node}  B   \BS{}psbrace(A)(B)\AC{}
\psbrace[rot=90](A)(B){}

\vspace{1cm}




\subsubsection{Options}
\begin{tabular}{|c|c|c|}
\hline 
 \begin{pspicture}(-.5,-.5)(4.2,2.5)
\psaxes[xticksize=2.2,yticksize=4,axesstyle=frame](4,2.2)
\pnode(1,1){B}\pnode(3,2){C}
\psbrace[braceWidth=5pt](B)(C){\TFRGB{texte}{text}}
 \end{pspicture} 
&
 \begin{pspicture}(-.5,-.5)(4.2,2.5)
\psaxes[xticksize=2.2,yticksize=4,axesstyle=frame](4,2.2)
\pnode(1,1){B}\pnode(3,2){C}
\psbrace[braceWidthInner=.5cm](B)(C){\TFRGB{texte}{text}}
 \end{pspicture}
 &
 \begin{pspicture}(-.5,-.5)(4.2,2.5)
\psaxes[xticksize=2.2,yticksize=4,axesstyle=frame](4,2.2)
\pnode(1,1){B}\pnode(3,2){C}
\psbrace[braceWidthOuter=.5cm](B)(C){\TFRGB{texte}{text}}
 \end{pspicture}
 \\ \hline
\RDD{braceWidth}=5pt \RDI{braceWidth}{pstricks-add} & \RDD{braceWidthInner}=.5cm \RDI{braceWidthInner}{pstricks-add} &
\RDD{braceWidthOuter}=.5cm \RDI{braceWidthOuter}{pstricks-add}
\\ \hline
{\blue \dft : \BS{pslinewidth}}  & {\blue  \dft : 10\BS{pslinewidth}} & {\blue \dft : 10\BS{pslinewidth} }
  \\ \hline
 \begin{pspicture}(-.5,-.5)(4.2,2.5)
\psaxes[xticksize=2.2,yticksize=4,axesstyle=frame](4,2.2)
\pnode(1,1){B}\pnode(3,2){C}
\psbrace[bracePos=.25](B)(C){\TFRGB{texte}{text}}
 \end{pspicture}
&
 \begin{pspicture}(-.5,-.5)(4.2,2.5)
\psaxes[xticksize=2.2,yticksize=4,axesstyle=frame](4,2.2)
\pnode(1,1){B}\pnode(3,2){C}
\psbrace[nodesepA=.5cm](B)(C){\TFRGB{texte}{text}}
 \end{pspicture}
&
 \begin{pspicture}(-.5,-.5)(4.2,2.5)
\psaxes[xticksize=2.2,yticksize=4,axesstyle=frame](4,2.2)
 \pnode(1,1){B}\pnode(3,2){C}
 \psbrace[nodesepB=.5cm](B)(C){\TFRGB{texte}{text}}
  \end{pspicture}
 \\ \hline
\RDD{bracePos}=.25 \RDI{bracePos}{pstricks-add} & 
\RDD{nodesepA}=5pt \RDI{nodesepA}{pstricks-add} & \RDD{nodesepB}=5pt \RDI{nodesepB}{pstricks-add}\\
Position (\%) & \TFRGB{décalage horizontal}{horizontal  offset} & \TFRGB{décalage vertical}{vertical offset}
 \\ \hline
{\blue \dft : .5} &  {\blue\dft : 0pt  }& {\blue \dft : 0pt }
  \\ \hline
 \begin{pspicture}(-.5,-.5)(4.2,2.5)
\psaxes[xticksize=2.2,yticksize=4,axesstyle=frame](4,2.2)
 \pnode(1,1){B}\pnode(3,2){C}
 \psbrace[rot=90](B)(C){\TFRGB{texte}{text}} 
  \end{pspicture} 
&
 \begin{pspicture}(-.5,-.5)(4.2,2.5)
\psaxes[xticksize=2.2,yticksize=4,axesstyle=frame](4,2.2)
 \pnode(1,1){B}\pnode(3,2){C}
 \psbrace[rot=90,ref=r](B)(C){\TFRGB{texte}{text}}
  \end{pspicture} 
&
 \begin{pspicture}(-.5,-.5)(4.2,2.5)
\psaxes[xticksize=2.2,yticksize=4,axesstyle=frame](4,2.2)
 \pnode(1,1){B}\pnode(3,2){C}
 \psbrace[rot=90,ref=l](B)(C){\TFRGB{texte}{text}}
  \end{pspicture} 
   \\ \hline 
\RDD{rot}=90  \RDI{rot}{pstricks-add} & 
rot=90,\RDD{ref} = {\red r}  \RDI{ref}{pstricks-add} & rot=90,ref={\red l }
   \\ \hline 
 \begin{pspicture}(-.5,-.5)(4.2,2.5)
\psaxes[xticksize=2.2,yticksize=4,axesstyle=frame](4,2.2)
 \pnode(1,1){B}\pnode(3,2){C}
 \psbrace[rot=90,ref=b](B)(C){\TFRGB{texte}{text}}
  \end{pspicture} 
 &
 \begin{pspicture}(-.5,-.5)(4.2,2.5)
\psaxes[xticksize=2.2,yticksize=4,axesstyle=frame](4,2.2)
  \pnode(1,1){B}\pnode(3,2){C}
  \psbrace[rot=90,ref=t](B)(C){\TFRGB{texte}{text}}
    \end{pspicture} 
  &
 \begin{pspicture}(-.5,-.5)(4.2,2.5)
\psaxes[xticksize=2.2,yticksize=4,axesstyle=frame](4,2.2)
 \pnode(1,1){B}\pnode(3,2){C}
 \psbrace[rot=90,ref=C](B)(C){\TFRGB{texte}{text}}
  \end{pspicture}
   \\ \hline 
rot=90,ref={\red b} & rot=90,ref={\red t} & rot=90,ref={\red C}
 \\ \hline
 \begin{pspicture}(-.5,-.5)(4.2,2.5)
\psaxes[xticksize=2.2,yticksize=4,axesstyle=frame](4,2.2)
 \pnode(1,1){B}\pnode(3,2){C}
 \psbrace[rot=90,ref=B](B)(C){\TFRGB{texte}{text}}
  \end{pspicture} 
 &
 \begin{pspicture}(-.5,-.5)(4.2,2.5)
\psaxes[xticksize=2.2,yticksize=4,axesstyle=frame](4,2.2)
  \pnode(1,1){B}\pnode(3,2){C}
  \psbrace[rot=90,ref=lC](B)(C){\TFRGB{texte}{text}}
    \end{pspicture} 
  &
 \begin{pspicture}(-.5,-.5)(4.2,2.5)
\psaxes[xticksize=2.2,yticksize=4,axesstyle=frame](4,2.2)
 \pnode(1,1){B}\pnode(3,2){C}
 \psbrace[fillcolor=green](B)(C){\TFRGB{texte}{text}}
  \end{pspicture}
   \\ \hline 
rot=90,ref={\red B} & rot=90,ref={\red lC}& \RDD{fillcolor}=green  \RDI{fillcolor}{pstricks-add}\\ \hline
  \end{tabular} 
  
\bigskip
