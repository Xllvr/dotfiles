
 


\psset{fillcolor=yellow,fillstyle=solid,linecolor=blue,unit=1cm}


%
%\begin{tabular}{ll}
% \BSS{psforeach}				&   \\ 
%\AC{variable} 						& \% nom de la variable \\
%\AC{liste des valeurs} 				& \% liste des valeurs de la variable\\
%\AC{action} 						& \% qui utilise les valeurs de la variable \\
%\end{tabular}
%
%\bigskip
%\emph{Exemple: liste de valeurs}

\begin{tabular}{|l|} \hline  
%\begin{pspicture}[0,-.5](11,0.5)

\psforeach{\nA}{0, 1, 1.5, 3, 5,10}{\psdot[dotscale=2](\nA,0)}

%\end{pspicture}
\\
\\ \hline  
\BSS{psforeach}\Rnode{A}{\AC{{\red \BS{nA}}}}\Rnode{B}{\AC{0, 1, 1.5, 3, 5,10}}\Rnode{C}{\AC{\BS{}psdot[dotscale=2]({\red \BS{nA}},0)}}  \hspace{1cm}
\\
\\
 \hspace{1cm}  \Rnode{AA}{variable}  \hspace{1cm} \Rnode{BB}{\TFRGB{liste des valeurs}{list of values}} \hspace{1cm}  \Rnode{CC}{action}
\\ \hline 
\end{tabular} 
 \ncline[linecolor=blue]{A}{AA}  \ncline[linecolor=blue]{B}{BB}  \ncline[linecolor=blue]{C}{CC}
 
 
%\begin{tabular}{ll}
%$\backslash$psforeach\AC{$\backslash$nA}\AC{0, 1, 1.5, 3, 5,10}	& \% variable $\backslash$nA et ses 6 valeurs  \\  
%\AC{$\backslash$psdot[dotscale=2]($\backslash$nA,0)}			& \% un point à chaque abscisse $\backslash$nA  \\ \
%\end{tabular}
%
%\smallskip 
%
%\psforeach{\nA}{0, 1, 1.5, 3, 5,10}{\psdot[dotscale=2](\nA,0)}

\bigskip

%\emph{Exemple: }

\begin{tabular}{|l|} \hline

 \multicolumn{1}{|c|}{ \TFRGB{liste de valeurs avec pas régulier}{list of values at a regular step}  } \\  \hline  
\psforeach{\nA}{0, 1,..,10}{\psdot[dotscale=2](\nA,0)}
\\
\\  \hline  
 \hspace{1cm}  \BS{psforeach}\AC{\BS{nA}}\AC{{\red 0, 1,..,10}}\AC{\BS{psdot}[dotscale=2](\BS{nA},0)}  \hspace{1cm} 
\\ \hline 
\end{tabular} 


%\begin{tabular}{ll}
%$\backslash$psforeach\AC{$\backslash$nA}\AC{0, 1 ,..,10}	& \% variable $\backslash$nA de 0 à 10 par pas de 1  \\  
%\AC{$\backslash$psdot[dotscale=2]($\backslash$nA,0)}		& \% un point à chaque abscisse $\backslash$nA  \\ \
%\end{tabular}
% 
%\smallskip 



\bigskip


 
\begin{tabular}{|l|} \hline 

 \multicolumn{1}{|c|}{ \TFRGB{utilisation du numéro d'index}{use of the index number}  } \\  \hline  

\psforeach{\nA}{0, 1, 1.5, 2.25, 5,10}{\rput(\nA,0){\the\psLoopIndex}}
\\
\\ \hline  
\BS{psforeach}\AC{\BS{A}}\AC{0, 1, 1.5, 2.25, 5,10}\AC{\BS{rput}(\BS{nA},0)\AC{{\red\BS{the}\BS{psLoopIndex}}}}
\\ \hline  
\end{tabular} 

%$\backslash$psforeach\AC{$\backslash$nA}\AC{0, 1, 1.5, 2.25, 5,10}\AC{$\backslash$rput($\backslash$nA,0)\AC{$\backslash$the$\backslash$psLoopIndex}}
%\smallskip 
%
%\psforeach{\nA}{0, 1, 1.5, 2.25, 5,10}{\rput(\nA,0){\the\psLoopIndex}}

