\emph{Cette partie sera complétée dans uns version ultérieure}

\psset{fillstyle=none,fillcolor=yellow,linecolor=blue,incolor=green!50}


\subsection{Axes}
\psset{unit=.25cm}
\begin{tabular}{|c|c|c|} \hline 
 \begin{pspicture}(-7,-4)(7,10)
 \psframe(-7,-4)(7,10)
 \axesIIID(0,0,0)(2,2,2)
 \end{pspicture}
 &
 \begin{pspicture}(-7,-4)(7,10)
 \psframe(-7,-4)(7,10)
\axesIIID(2,2,2)(2,2,2)
\end{pspicture}
&  
 \begin{pspicture}(-7,-4)(7,10)
 \psframe(-7,-4)(7,10)
\axesIIID(1,1,1)(2,2,2)
\end{pspicture}
\\ \hline
 \BSS{axesIIID}(0,0,0)(2,2,2)  \BSI{axesIIID}{pst-sol3d}
 & 
\BSS{axesIIID}(2,2,2)(2,2,2)
&  
\BSS{axesIIID}(1,1,1)(2,2,2)
\\ \hline 
\end{tabular} 

\bigskip
\begin{tabular}{|c|c|c|} \hline 
 \begin{pspicture}(-7,-4)(7,10)
 \psframe(-7,-4)(7,10)
 \axesIIID[labelsep=0cm](0,0,0)(2,2,2)
 \end{pspicture}
 &
 \begin{pspicture}(-7,-4)(7,10)
 \psframe(-7,-4)(7,10)
\axesIIID[axisnames={a,b,c}](0,0,0)(2,2,2)
\end{pspicture}
&  
 \begin{pspicture}(-7,-4)(7,10)
 \psframe(-7,-4)(7,10)
\axesIIID[showOrigin=false](1,1,1)(2,2,2)
\end{pspicture}
\\ \hline
\RDD{labelsep}=0cm
 & 
\RDD{axisnames}=\AC{a,b,c} \RDI{axisnames}{pst-sol3d}
&  
\RDD{showOrigin}=false \RDI{showorigin}{pst-sol3d}
\\ \hline 
\dft : labelsep=5pt & \dft : axisnames=\AC{x,y,z} & \dft : showOrigin=true\\ \hline 
\end{tabular} 
   
%\subsection{\'Element en 3D}  
\SbSSCT{\'Element en 3D}{3D elements}

\psset{unit=.5cm}


%--------------------point, line, vector,
%\subsubsection{point, line, vector}
\SbSbSSCT{point, ligne, vecteur}{point, line, vector}

\begin{tabular}{|c|c|c|} \hline  
% \multicolumn{3}{|l|}{\BS{psSolid}[object=point,args=1 2 2]} \\
\begin{pspicture}(-4,-4)(4,4)
\psframe(-4,-4)(4,4)
\axesIIID(0,0,0)(1,1,1)
\psSolid[object=point,args=1 2 2,linecolor=red]			%point
\psSolid[object=line,args=0 0 0 1 2 0,linestyle=dotted,linecolor=red]
\psSolid[object=line,args=1 2 2 1 2 0,linestyle=dotted,linecolor=red]
\end{pspicture}
&  
\begin{pspicture}(-4,-4)(4,4)
\psframe(-4,-4)(4,4)
\axesIIID(0,0,0)(1,1,1)

\psSolid[object=line,args=0 -1 0 1 2 2] % line
\psSolid[object=line,args=0 0 0 0 -2 0,linecolor=red]
\psSolid[object=line,args=0 0 0 1 2 0,linestyle=dotted,linecolor=red]
\psSolid[object=line,args= 1 2 0 1 2 2,linestyle=dotted,linecolor=red]
\end{pspicture}
&  
\begin{pspicture}(-4,-4)(4,4)
\psframe(-4,-4)(4,4)
\axesIIID(0,0,0)(1,1,1)
\psSolid[object=vecteur,args=1 2 2,linecolor=red] % vecteur
\psSolid[object=line,args=0 0 0 1 2 0,linestyle=dotted,linecolor=red]
\psSolid[object=line,args=1 2 2 1 2 0,linestyle=dotted,linecolor=red]
\end{pspicture}\\ \hline
[\RDD{object=point},args=1 2 2] \RDI{object=point}{pst-sol3d} & 
[\RDD{object=line},args=0 -1 0 1 2 2] \RDI{object=line}{pst-sol3d}&
[\RDD{object=vecteur},args=1 2 2] \RDI{object=vecteur}{pst-sol3d}
\\ \hline
\end{tabular} 
\bigskip
 %------------------------- plan, grille, cube
 \psset{unit=.2cm}
 
 
% \subsubsection{ plan}
\SbSbSSCT{Plan}{Plane}
 
\begin{tabular}{|c|c|} \hline 
 \multicolumn{2}{|l|}{\BS{psSolid}[\RDD{object=plan},definition=equation,args=\Rnode*[fillcolor=yellow]{A}{\AC{[0 0 1 0]}},base=-2 2 -3 3] \RDI{object=plan}{pst-sol3d} } \\ 
 \multicolumn{2}{|l|}{ } \\ 
 \multicolumn{2}{|c|}{  \rnode{AA}{coeff de l'équation  ax+by+cz+d = 0} } \\  \hline
\begin{pspicture}(-12,-10)(12,10)
\psframe(-12,-10)(12,10)
\psSolid[object=plan,definition=equation,args={[0 0 1 0]},base=-2 2 -3 3]
 \axesIIID[linecolor=red](0,0,0)(1,1,1)
\end{pspicture}
&
\begin{pspicture}(-12,-10)(12,10)
 \psframe(-12,-10)(12,10)
\psSolid[object=plan,definition=equation,args={[0 1 0 0]},base=-2 2 -3 3]
 \axesIIID[linecolor=red](0,0,0)(1,1,1)
\end{pspicture}
\\ \hline
args=\AC{[0 0 1 0]} & args=\AC{[0 1 0 0]}  \\ \hline
\end{tabular} 
\ncline[linecolor=blue]{A}{AA}
\bigskip

\begin{tabular}{|c|c|c|} \hline 
% \multicolumn{3}{|c|}{ \BS{psSolid}[object=cube,a=3,action=draw] } \\  \hline 
\begin{pspicture}(-10,-10)(10,10)
 \psframe(-10,-10)(10,10)
%\axesIIID(1,1,1)(1,1,1)
\psSolid[object=plan,definition=equation,args={[1 0 0 0]},base=-2 2 -3 3]
 \axesIIID[linecolor=red](0,0,0)(1,1,1)
\end{pspicture}
&
\begin{pspicture}(-10,-10)(10,10)
 \psframe(-10,-10)(10,10)
%\axesIIID(1,1,1)(1,1,1)
\psSolid[object=plan,definition=equation,args={[0 0 1 1]},base=-2 2 -3 3]
 \axesIIID[linecolor=red](0,0,0)(1,1,1)
\end{pspicture}
 & 
\begin{pspicture}(-10,-10)(10,10)
 \psframe(-10,-10)(10,10)
%\axesIIID(1,1,1)(1,1,1)
\psSolid[object=plan,definition=equation,args={[1 1 0 0]},base=-2 2 -3 3]
 \axesIIID[linecolor=red](0,0,0)(1,1,1)
\end{pspicture}
\\ \hline
args={[1 0 0 0]} & args={[0 0 1 1]} & args={[1 1 0 0]} \\ \hline
\end{tabular}


%------------------------------------------

%\subsubsection{grille}
\SbSbSSCT{Grille}{Grid}

\begin{tabular}{|c|c|c|} \hline 
\multicolumn{3}{|c|}{\BS{psSolid}[\RDD{object=grille},base=-2 2 -3 3] \RDI{object=grille}{pst-sol3d}} \\ \hline
\begin{pspicture}(-12,-10)(12,10)
 \psframe(-12,-10)(12,10)
\psSolid[object=grille,base=-2 2 -3 3] % grille
 \axesIIID[linecolor=red](0,0,0)(1,1,1)
\end{pspicture}
&
\begin{pspicture}(-12,-10)(12,10)
 \psframe(-12,-10)(12,10)
\psSolid[object=grille,base=-2 2 -3 3,RotX=-90] % grille
 \axesIIID[linecolor=red](0,0,0)(1,1,1)
\end{pspicture}
&
\begin{pspicture}(-12,-10)(12,10)
 \psframe(-12,-10)(12,10)
\psSolid[object=grille,base=-2 2 -3 3,RotY=90] % grille
 \axesIIID[linecolor=red](0,0,0)(1,1,1)
\end{pspicture}

\\ \hline
\dft & RotX=90 & RotY=90  \\ \hline
\end{tabular} 
\bigskip

%------------------------------------------------------------------------------

%
%\begin{pspicture}(-10,-10)(10,10)
% \psframe(-10,-10)(10,10)
%%\axesIIID(1,1,1)(1,1,1)
%\psSolid[object=cube,action=draw] %cube
% \axesIIID[linecolor=red](0,0,0)(1,1,1)
%\end{pspicture}
%
%\begin{pspicture}(-10,-5)(10,15)
% \psframe(-10,-5)(10,15)
%\psSolid[object=cylindre,action=draw](0,0,0)%
% \axesIIID[linecolor=red](0,0,0)(1,1,1)
%\end{pspicture}

\subsubsection{cube}
\begin{tabular}{|c|c|c|} \hline 
 \multicolumn{3}{|c|}{ \BS{psSolid}[object=cube,a=3,action=draw] } \\  \hline 
\begin{pspicture}(-10,-10)(10,10)
 \psframe(-10,-10)(10,10)
%\axesIIID(1,1,1)(1,1,1)
\psSolid[object=cube,a=3,action=draw] %cube
 \axesIIID[linecolor=red](0,0,0)(1,1,1)
\end{pspicture}
&
\begin{pspicture}(-10,-10)(10,10)
 \psframe(-10,-10)(10,10)
%\axesIIID(1,1,1)(1,1,1)
\psSolid[object=cube,a=3,action=draw*] %cube
 \axesIIID[linecolor=red](0,0,0)(1,1,1)
\end{pspicture}
 & 
\begin{pspicture}(-10,-10)(10,10)
 \psframe(-10,-10)(10,10)
%\axesIIID(1,1,1)(1,1,1)
\psSolid[object=cube,a=3,action=draw**] %cube
 \axesIIID[linecolor=red](0,0,0)(1,1,1)
\end{pspicture}
\\ \hline
action=draw & action=draw* & action=draw**\\ \hline
\end{tabular} 
\bigskip


%\subsubsection{cylindre}
\SbSbSSCT{Cylindre}{Cylinder}

\begin{tabular}{|c|c|c|} \hline 
 \multicolumn{3}{|c|}{ \BS{psSolid}[\RDD{object=cylindre},h=3,r=2,action=draw](0,0,0) \RDI{object=cylindre}{pst-sol3d} } \\  \hline 
\begin{pspicture}(-10,-5)(10,15)
 \psframe(-10,-5)(10,15)
\psSolid[object=cylindre,h=3,r=2,action=draw](0,0,0)%
 \axesIIID[linecolor=red](0,0,0)(1,1,1)
\end{pspicture}
&
\begin{pspicture}(-10,-5)(10,15)
 \psframe(-10,-5)(10,15)
\psSolid[object=cylindre,h=3,r=2,action=draw*](0,0,0)%
 \axesIIID[linecolor=red](0,0,0)(1,1,1)
\end{pspicture}
&
\begin{pspicture}(-10,-5)(10,15)
 \psframe(-10,-5)(10,15)
\psSolid[object=cylindre,h=3,r=2,action=draw**](0,0,0)%
 \axesIIID[linecolor=red](0,0,0)(1,1,1)
\end{pspicture}\\ \hline
action=draw & action=draw* & action=draw**\\ \hline
\end{tabular} 


%\subsubsection{cylindrecreux}
\SbSbSSCT{cylindre creux}{Tube}

\begin{tabular}{|c|c|c|} \hline 
 \multicolumn{3}{|c|}{ \BS{psSolid}[\RDD{object=cylindrecreux},h=3,r=2,action=draw](0,0,0) \RDI{object=cylindrecreux}{pst-sol3d} } \\  \hline 
\begin{pspicture}(-10,-5)(10,15)
 \psframe(-10,-5)(10,15)
\psSolid[object=cylindrecreux,h=3,r=2,action=draw](0,0,0)%
 \axesIIID[linecolor=red](0,0,0)(1,1,1)
\end{pspicture}
&
\begin{pspicture}(-10,-5)(10,15)
 \psframe(-10,-5)(10,15)
\psSolid[object=cylindrecreux,h=3,r=2,action=draw*](0,0,0)%
 \axesIIID[linecolor=red](0,0,0)(1,1,1)
\end{pspicture}
&
\begin{pspicture}(-10,-5)(10,15)
 \psframe(-10,-5)(10,15)
\psSolid[object=cylindrecreux,h=3,r=2,action=draw**](0,0,0)%
 \axesIIID[linecolor=red](0,0,0)(1,1,1)
\end{pspicture}\\ \hline
action=draw & action=draw* & action=draw**\\ \hline
\end{tabular} 


%\subsubsection{cone}
\SbSbSSCT{Cône}{Cone}

\begin{tabular}{|c|c|c|} \hline 
 \multicolumn{3}{|c|}{ \BS{psSolid}[\RDD{object=cone},h=3,r=2,action=draw] \RDI{object=cone}{pst-sol3d} } \\  \hline 
\begin{pspicture}(-10,-10)(10,10)
 \psframe(-10,-10)(10,10)
\psSolid[object=cone,h=3,r=2,action=draw]%
 \axesIIID[linecolor=red](0,0,0)(1,1,1)
\end{pspicture}
&
\begin{pspicture}(-10,-10)(10,10)
 \psframe(-10,-10)(10,10)
\psSolid[object=cone,h=3,r=2,action=draw*]%
 \axesIIID[linecolor=red](0,0,0)(1,1,1)
\end{pspicture}
&
\begin{pspicture}(-10,-10)(10,10)
 \psframe(-10,-10)(10,10)
\psSolid[object=cone,h=3,r=2,action=draw**]%
 \axesIIID[linecolor=red](0,0,0)(1,1,1)
\end{pspicture}\\ \hline
action=draw & action=draw* & action=draw**\\ \hline
\end{tabular} 

\subsubsection{conecreux}
\SbSbSSCT{Cône creux}{Empty cone}

\begin{tabular}{|c|c|c|} \hline 
 \multicolumn{3}{|c|}{ \BS{psSolid}[\RDD{object=conecreux},h=4,r=2,action=draw] \RDI{object=conecreux}{pst-sol3d} } \\  \hline 
\begin{pspicture}(-10,-10)(10,10)
 \psframe(-10,-10)(10,10)
\psSolid[object=conecreux,h=4,r=2,action=draw]%
 \axesIIID[linecolor=red](0,0,0)(1,1,1)
\end{pspicture}
&
\begin{pspicture}(-10,-10)(10,10)
 \psframe(-10,-10)(10,10)
\psSolid[object=conecreux,h=4,r=2,action=draw*]%
 \axesIIID[linecolor=red](0,0,0)(1,1,1)
\end{pspicture}
&
\begin{pspicture}(-10,-10)(10,10)
 \psframe(-10,-10)(10,10)
\psSolid[object=conecreux,h=4,r=2,action=draw**]%
 \axesIIID[linecolor=red](0,0,0)(1,1,1)
\end{pspicture}\\ \hline
action=draw & action=draw* & action=draw**\\ \hline
\end{tabular} 
\bigskip

%\subsubsection{tronccone}
\SbSbSSCT{Tronc de cône}{Truncated cone}

\begin{tabular}{|c|c|c|} \hline 
 \multicolumn{3}{|c|}{ \BS{psSolid}[\RDD{object=tronccone},r0=2,r1=1,h=4,action=draw] \RDI{object=troncone}{pst-sol3d} } \\  \hline 
\begin{pspicture}(-10,-5)(10,15)
 \psframe(-10,-5)(10,15)
\psSolid[object=tronccone,r0=2,r1=1,h=4,action=draw]%
 \axesIIID[linecolor=red](0,0,0)(1,1,1)
\end{pspicture}
&
\begin{pspicture}(-10,-5)(10,15)
 \psframe(-10,-5)(10,15)
\psSolid[object=tronccone,r0=2,r1=1,h=4,action=draw*]%
 \axesIIID[linecolor=red](0,0,0)(1,1,1)
\end{pspicture}
&
\begin{pspicture}(-10,-5)(10,15)
 \psframe(-10,-5)(10,15)
\psSolid[object=tronccone,r0=2,r1=1,h=4,action=draw**]%
 \axesIIID[linecolor=red](0,0,0)(1,1,1)
\end{pspicture}\\ \hline
action=draw & action=draw* & action=draw**\\ \hline
\end{tabular} 

%\subsubsection{troncconecreux}
\SbSbSSCT{Tronc de cône creux creux}{Empty truncated cone}

\begin{tabular}{|c|c|c|} \hline 
 \multicolumn{3}{|c|}{ \BS{psSolid}[\RDD{object=troncconecreux},r0=2,r1=1,h=4,action=draw] \RDI{object=tronconecreux}{pst-sol3d} } \\  \hline 
\begin{pspicture}(-10,-5)(10,15)
 \psframe(-10,-5)(10,15)
\psSolid[object=troncconecreux,r0=2,r1=1,h=4,action=draw]%
 \axesIIID[linecolor=red](0,0,0)(1,1,1)
\end{pspicture}
&
\begin{pspicture}(-10,-5)(10,15)
 \psframe(-10,-5)(10,15)
\psSolid[object=troncconecreux,r0=2,r1=1,h=4,action=draw*]%
 \axesIIID[linecolor=red](0,0,0)(1,1,1)
\end{pspicture}
&
\begin{pspicture}(-10,-5)(10,15)
 \psframe(-10,-5)(10,15)
\psSolid[object=troncconecreux,r0=2,r1=1,h=4,action=draw**]%
 \axesIIID[linecolor=red](0,0,0)(1,1,1)
\end{pspicture}\\ \hline
action=draw & action=draw* & action=draw**\\ \hline
\end{tabular} 
\bigskip

\subsubsection{sphere}


\begin{tabular}{|c|c|c|} \hline 
 \multicolumn{3}{|c|}{ \BS{psSolid}[\RDD{object=sphere},r=1,action=draw] \RDI{object=sphere}{pst-sol3d} } \\  \hline 
\begin{pspicture}(-10,-10)(10,10)
 \psframe(-10,-10)(10,10)
\psSolid[object=sphere,r=3,action=draw]%
 \axesIIID[linecolor=red](0,0,0)(1,1,1)
\end{pspicture}
&
\begin{pspicture}(-10,-10)(10,10)
 \psframe(-10,-10)(10,10)
\psSolid[object=sphere,r=3,action=draw*]%
 \axesIIID[linecolor=red](0,0,0)(1,1,1)
\end{pspicture}
&
\begin{pspicture}(-10,-10)(10,10)
 \psframe(-10,-10)(10,10)
\psSolid[object=sphere,r=3,action=draw**]%
 \axesIIID[linecolor=red](0,0,0)(1,1,1)
\end{pspicture}\\ \hline
action=draw & action=draw* & action=draw**\\ \hline
\end{tabular} 
\bigskip

%\subsubsection{calottesphere}
\SbSbSSCT{Calotte sphérique}{Spherical cup}

\begin{tabular}{|c|c|c|} \hline 
 \multicolumn{3}{|c|}{ \BS{psSolid}[\RDD{object=calottesphere},r=3,action=draw] \RDI{object=calottesphere}{pst-sol3d} } \\  \hline 
\begin{pspicture}(-10,-10)(10,10)
 \psframe(-10,-10)(10,10)
\psSolid[object=calottesphere,r=3,action=draw]%
 \axesIIID[linecolor=red](0,0,0)(1,1,1)
\end{pspicture}
&
\begin{pspicture}(-10,-10)(10,10)
 \psframe(-10,-10)(10,10)
\psSolid[object=calottesphere,r=3,action=draw*]%
 \axesIIID[linecolor=red](0,0,0)(1,1,1) \axesIIID[linecolor=red](0,0,0)(1,1,1)
\end{pspicture}
&
\begin{pspicture}(-10,-10)(10,10)
 \psframe(-10,-10)(10,10)
\psSolid[object=calottesphere,r=3,action=draw**]%
 \axesIIID[linecolor=red](0,0,0)(1,1,1)
\end{pspicture}\\ \hline
action=draw & action=draw* & action=draw**\\ \hline
\end{tabular} 

%\subsubsection{calottespherecreuse}
\SbSbSSCT{calotte spherique creuse}{empty spherical cup}

\begin{tabular}{|c|c|c|} \hline 
 \multicolumn{3}{|c|}{ \BS{psSolid}[\RDD{object=calottespherecreuse},r=3,action=draw] \RDI{object=calottespherecreuse}{pst-sol3d} } \\  \hline 
\begin{pspicture}(-10,-10)(10,10)
 \psframe(-10,-10)(10,10)
\psSolid[object=calottespherecreuse,r=3,action=draw]%
\end{pspicture}
&
\begin{pspicture}(-10,-10)(10,10)
 \psframe(-10,-10)(10,10)
\psSolid[object=calottespherecreuse,r=3,action=draw*]%
\end{pspicture}
&
\begin{pspicture}(-10,-10)(10,10)
 \psframe(-10,-10)(10,10)
\psSolid[object=calottespherecreuse,r=3,action=draw**]%
\end{pspicture}\\ \hline
action=draw & action=draw* & action=draw**\\ \hline
\end{tabular} 

%\subsubsection{tore}
\SbSbSSCT{Tore}{Torus}

\begin{tabular}{|c|c|c|} \hline 
 \multicolumn{3}{|c|}{ \BS{psSolid}[r1=2,r0=1, \RDD{object=tore},ngrid=18 36,action=draw] \RDI{object=tore}{pst-sol3d} } \\  \hline 
\begin{pspicture}(-10,-10)(10,10)
 \psframe(-10,-10)(10,10)
\psSolid[r1=2,r0=1,object=tore,ngrid=18 36,action=draw]%
\end{pspicture}
&
\begin{pspicture}(-10,-10)(10,10)
 \psframe(-10,-10)(10,10)
\psSolid[r1=2,r0=1,object=tore,ngrid=18 36,action=draw*]%
\end{pspicture}
&
\begin{pspicture}(-10,-10)(10,10)
 \psframe(-10,-10)(10,10)
\psSolid[r1=2,r0=1,object=tore,ngrid=18 36,action=draw**]%
\end{pspicture}\\ \hline
action=draw & action=draw* & action=draw**\\ \hline
\end{tabular} 
\bigskip

%\subsubsection{anneau}
\SbSbSSCT{Anneau}{Ring}

\begin{tabular}{|c|c|c|} \hline 
 \multicolumn{3}{|c|}{ \BS{psSolid}[\RDD{object=anneau},h=1,R=2,r=1,action=draw] \RDI{object=anneau}{pst-sol3d} } \\  \hline 
\begin{pspicture}(-10,-10)(10,10)
 \psframe(-10,-10)(10,10)
\psSolid[object=anneau,h=1,R=2,r=1,action=draw]%
\end{pspicture}
&
\begin{pspicture}(-10,-10)(10,10)
 \psframe(-10,-10)(10,10)
\psSolid[object=anneau,h=1,R=2,r=1,action=draw*]%
\end{pspicture}
&
\begin{pspicture}(-10,-10)(10,10)
 \psframe(-10,-10)(10,10)
\psSolid[object=anneau,h=1,R=2,r=1,action=draw**]%
\end{pspicture}\\ \hline
action=draw & action=draw* & action=draw**\\ \hline
\end{tabular} 
\bigskip
%------------------------------

%\subsubsection{ conecreux,tronccone, troncconecreux}
%\begin{tabular}{|c|c|c|} \hline 
%\begin{pspicture}(-10,-5)(10,15)
% \psframe(-10,-5)(10,15)
%\axesIIID(1,1,1)(1,1,1)
%\psSolid[object=conecreux,h=3,r=2,RotY=-60,mode=4] % conecreux
%\end{pspicture}
% & 
%\begin{pspicture}(-10,-5)(10,15)
% \psframe(-10,-5)(10,15)
% \axesIIID(1,1,1)(1,1,1)
%\psSolid[object=tronccone,r0=2,r1=1,h=3,mode=4] % tronccone
%\end{pspicture}
%&
%\begin{pspicture}(-10,-5)(10,15)
% \psframe(-10,-5)(10,15)
% \axesIIID(1,1,1)(1,1,1)
%\psSolid[object=tetrahedron,r=3,action=draw**]%
%\end{pspicture}
%\\ \hline 
%\end{tabular} 

\subsubsection{tetrahedron}

\begin{tabular}{|c|c|c|} \hline 
 \multicolumn{3}{|c|}{ \BS{psSolid}[object=tetrahedron,r=1,RotZ=30,action=draw] } \\  \hline 
\begin{pspicture}(-10,-10)(10,10)
 \psframe(-10,-10)(10,10)
%\axesIIID(1,1,1)(1,1,1)
\psSolid[object=tetrahedron,r=3,action=draw,RotZ=30]%
\end{pspicture}
 & 
\begin{pspicture}(-10,-10)(10,10)
 \psframe(-10,-10)(10,10)
% \axesIIID(1,1,1)(1,1,1)
\psSolid[object=tetrahedron,r=3,action=draw*,RotZ=30]%
\end{pspicture}
&
\begin{pspicture}(-10,-10)(10,10)
 \psframe(-10,-10)(10,10)
% \axesIIID(1,1,1)(1,1,1)
\psSolid[object=tetrahedron,r=3,action=draw**,RotZ=30]%
\end{pspicture}\\ \hline 
action=draw & action=draw* & action=draw**\\ \hline

\end{tabular} 


%------------ sphere, calottesphere, calottespherecreuse









\subsubsection{parallelepiped}

\begin{tabular}{|c|c|c|} \hline 
 \multicolumn{3}{|c|}{ \BS{psSolid}[\RDD{object=parallelepiped},a=1,b=2,c=3,action=draw] \RDI{object=parallelepiped}{pst-sol3d} } \\  \hline 
\begin{pspicture}(-10,-10)(10,10)
 \psframe(-10,-10)(10,10)
\psSolid[object=parallelepiped,a=1,b=2,c=3,action=draw]%
\end{pspicture}
&
\begin{pspicture}(-10,-10)(10,10)
 \psframe(-10,-10)(10,10)
\psSolid[object=parallelepiped,a=1,b=2,c=3,action=draw*]%
\end{pspicture}
&
\begin{pspicture}(-10,-10)(10,10)
 \psframe(-10,-10)(10,10)
\psSolid[object=parallelepiped,a=1,b=2,c=3,action=draw**]%
\end{pspicture}\\ \hline
action=draw & action=draw* & action=draw**\\ \hline
\end{tabular} 
\bigskip

\subsubsection{octahedron}
\begin{tabular}{|c|c|c|} \hline 
 \multicolumn{3}{|c|}{ \BS{psSolid}[\RDD{object=octahedron},a=30,action=draw] \RDI{object=octahedron}{pst-sol3d} } \\  \hline 
\begin{pspicture}(-10,-10)(10,10)
 \psframe(-10,-10)(10,10)
\psSolid[object=octahedron,a=3,action=draw]%
\end{pspicture}
&
\begin{pspicture}(-10,-10)(10,10)
 \psframe(-10,-10)(10,10)
\psSolid[object=octahedron,a=3,action=draw*]%
\end{pspicture}
&
\begin{pspicture}(-10,-10)(10,10)
 \psframe(-10,-10)(10,10)
\psSolid[object=octahedron,a=3,action=draw**]%
\end{pspicture}\\ \hline
action=draw & action=draw* & action=draw**\\ \hline
\end{tabular} 
\bigskip

\subsubsection{dodecahedron}

\begin{tabular}{|c|c|c|} \hline 
 \multicolumn{3}{|c|}{ \BS{psSolid}[\RDD{object=dodecahedron},a=2.5,RotZ=90,action=draw] \RDI{object=dodecahedron}{pst-sol3d} } \\  \hline 
\begin{pspicture}(-10,-10)(10,10)
 \psframe(-10,-10)(10,10)
\psSolid[object=dodecahedron,a=2.5,RotZ=90,action=draw]%
\end{pspicture}
&
\begin{pspicture}(-10,-10)(10,10)
 \psframe(-10,-10)(10,10)
\psSolid[object=dodecahedron,a=2.5,RotZ=90,action=draw*]%
\end{pspicture}
&
\begin{pspicture}(-10,-10)(10,10)
 \psframe(-10,-10)(10,10)
\psSolid[object=dodecahedron,a=2.5,RotZ=90,action=draw**]%
\end{pspicture}\\ \hline
action=draw & action=draw* & action=draw**\\ \hline
\end{tabular} 
\bigskip

\subsubsection{icosahedron}
\begin{tabular}{|c|c|c|} \hline 
 \multicolumn{3}{|c|}{ \BS{psSolid}[\RDD{object=icosahedron},a=3,action=draw] \RDI{object=icosahedron}{pst-sol3d} } \\  \hline 
\begin{pspicture}(-10,-10)(10,10)
 \psframe(-10,-10)(10,10)
\psSolid[object=icosahedron,a=3,action=draw]%
\end{pspicture}
&
\begin{pspicture}(-10,-10)(10,10)
 \psframe(-10,-10)(10,10)
\psSolid[object=icosahedron,a=3,action=draw*]%
\end{pspicture}
&
\begin{pspicture}(-10,-10)(10,10)
 \psframe(-10,-10)(10,10)

\psSolid[object=icosahedron,a=3,action=draw**]%
\end{pspicture}\\ \hline
action=draw & action=draw* & action=draw**\\ \hline
\end{tabular} 
\bigskip


%\subsubsection{prisme}
\SbSbSSCT{Prisme}{prism}

\begin{tabular}{|c|c|c|} \hline 
 \multicolumn{3}{|c|}{ \BS{psSolid}[\RDD{object=prisme},action=draw,h=4] \RDI{object=prisme}{pst-sol3d} } \\  \hline 
\begin{pspicture}(-10,-5)(10,10)
 \psframe(-10,-5)(10,10)
\psSolid[object=prisme,action=draw,h=2]%
\end{pspicture}
&
\begin{pspicture}(-10,-5)(10,10)
 \psframe(-10,-5)(10,10)
\psSolid[object=prisme,action=draw*,h=2]%
\end{pspicture}
&
\begin{pspicture}(-10,-5)(10,10)
 \psframe(-10,-5)(10,10)
\psSolid[object=prisme,action=draw**,h=2]%
\end{pspicture}\\ \hline
action=draw & action=draw* & action=draw**\\ \hline
\end{tabular} 

%\subsubsection{prismecreux}
\SbSbSSCT{Prisme creux}{Empty prism}

\begin{tabular}{|c|c|c|} \hline 
 \multicolumn{3}{|c|}{ \BS{psSolid}[\RDD{object=prismecreux},action=draw,h=4] \RDI{object=prismecreux}{pst-sol3d} } \\  \hline 
\begin{pspicture}(-10,-5)(10,10)
 \psframe(-10,-5)(10,10)
\psSolid[object=prismecreux,action=draw,h=2]%
\end{pspicture}
&
\begin{pspicture}(-10,-5)(10,10)
 \psframe(-10,-5)(10,10)
\psSolid[object=prismecreux,action=draw*,h=2]%
\end{pspicture}
&
\begin{pspicture}(-10,-5)(10,10)
 \psframe(-10,-5)(10,10)
\psSolid[object=prismecreux,action=draw**,h=2]%
\end{pspicture}\\ \hline
action=draw & action=draw* & action=draw**\\ \hline
\end{tabular} 
\bigskip

\subsubsection{face,ruban}

\begin{tabular}{|c|c|c|} \hline 
\begin{pspicture}(-10,-10)(10,10)
 \psframe(-10,-10)(10,10)
\psSolid[object=face,fillcolor=yellow,incolor=blue,base=0 0 3 0 1.5 3](0,1,0)
\end{pspicture}
&
\begin{pspicture}(-10,-10)(10,10)
 \psframe(-10,-10)(10,10)
\psSolid[object=ruban,h=3,base=0 0 2 2 4 0 6 2,num=0 1 2 3,show=0 1 2 3,ngrid=3])
\end{pspicture}
&
\begin{pspicture}(-10,-10)(10,10)
 \psframe(-10,-10)(10,10)
 \axesIIID(1,1,1)(4,7,4)
\psSolid[object=ruban,h=3,base=0 0 2 2 4 0 6 2])
\end{pspicture}
\\ \hline
\end{tabular} 

%============================================== A VOIR ========================================

%\newpage
%
%\begin{pspicture}(-10,-10)(10,10)
% \psframe(-10,-10)(10,10)
%\psSurface[ngrid=.25 .25,incolor=white,axesboxed](-4,-4)(4,4){x dup mul y dup mul 3 mul sub x mul 32 div}
%\end{pspicture}
%
%\newpage
%
%\begin{pspicture}(-10,-10)(10,10)
%\psframe(-10,-10)(10,10)
%\psSolid[object=new,action=draw,sommets=
%-3 0 0
%3 0 0
%0 -3 0
%0 3 0
%0 0 -3
%0 0 3 ,
%faces={
%[0 2 1 3]
%[0 4 1 5]}]%
%\end{pspicture}
%
%\begin{pspicture}(-10,-10)(10,10)
%\psframe(-10,-10)(10,10)
%\psSolid[object=new,action=draw,sommets=
%-3 0 0
%3 0 0
%0 -3 0
%0 3 0
%0 0 -3
%0 0 3 ,
%faces={
%[0 2 1 3]
%[0 4 1 5]}
%,show=all,num=all
%]%
%\end{pspicture}
%\newpage
%
%\begin{pspicture}(-10,-10)(10,10)
%\psframe(-10,-10)(10,10)
%\psSolid[object=new,action=draw,sommets=
%2 4 3
%-2 4 3
%-2 -4 3
%2 -4 3
%2 4 0
%-2 4 0
%-2 -4 0
%2 -4 0
%0 4 5
%0 -4 5,
%faces={
%[0 1 2 3]
%[7 6 5 4]
%[0 3 7 4]
%[3 9 2]
%[1 8 0]
%[8 9 3 0]
%[9 8 1 2]
%[6 7 3 2]
%[2 1 5 6]}]%
%\end{pspicture}
%
%\newpage
%\begin{pspicture}(-10,-10)(10,10)
%\psframe(-10,-10)(10,10)
%\defFunction[algebraic]{helice}(t){3*cos(4*t)}{3*sin(4*t)}{t}
%\psSolid[object=courbe,r=0,range=0 6,linecolor=blue,linewidth=0.1,resolution=360,function=helice,fillcolor=red]
%\end{pspicture}
%%
%
%
%\newpage
%\psset{unit=1cm}
%
%
%
%\begin{pspicture}(-2,-2)(3,3)
%\psframe(-2cm,-2cm)(3cm,3cm)
%\psset{viewpoint=100 30 20,Decran=100}
%\psSolid[object=cube,a=2,action=draw*, fillcolor=magenta!20]
%\axesIIID[showOrigin=false](1,1,1)(3,2,2.5)
%\end{pspicture}
%\newpage
\subsection{Mode}

\begin{tabular}{|c|c|c|c|} \hline 
 \multicolumn{4}{|c|}{ \BS{psSolid}[object=cylindre,h=3,r=1.5,\RDD{mode}=1](0,0,0) \RDI{mode}{pst-sol3d} } \\  \hline 
\begin{pspicture}(-7,-5)(7,15)
% \psframe(-7,-5)(7,15)
\psSolid[object=cylindre,h=3,r=1.5,mode=1](0,0,0)%
 \axesIIID[linecolor=red](0,0,0)(1,1,1)
\end{pspicture}
&
\begin{pspicture}(-7,-5)(7,15)
% \psframe(-7,-5)(7,15)
\psSolid[object=cylindre,h=3,r=1.5,mode=2](0,0,0)%
 \axesIIID[linecolor=red](0,0,0)(1,1,1)
\end{pspicture}
&
\begin{pspicture}(-7,-5)(7,15)
% \psframe(-7,-5)(7,15)
\psSolid[object=cylindre,h=3,r=1.5,mode=3](0,0,0)%
 \axesIIID[linecolor=red](0,0,0)(1,1,1)
\end{pspicture}
&
\begin{pspicture}(-7,-5)(7,15)
% \psframe(-7,-5)(7,15)
\psSolid[object=cylindre,h=3,r=1.5,mode=4](0,0,0)%
 \axesIIID[linecolor=red](0,0,0)(1,1,1)
\end{pspicture}\\ \hline
mode=1 & mode=2 & mode=3 & mode=4\\ \hline
\end{tabular}
%-------------------------------------------------

\subsubsection{Options}

\begin{tabular}{|c|c|c|} \hline 
 \multicolumn{3}{|c|}{ \BS{psSolid}[object=cube,a=3,action=draw*,\RDD{trunc}=all,RotZ=30] \RDI{trunc}{pst-sol3d} } \\  \hline 
\begin{pspicture}(-10,-10)(10,10)
\psSolid[object=cube,a=3,action=draw*,trunc=all,RotZ=30] %cube
\end{pspicture}
&
\begin{pspicture}(-10,-10)(10,10)
\psSolid[object=cube,a=3,action=draw*,trunc=0  2 4,RotZ=30] %cube
\end{pspicture}
 & 
\begin{pspicture}(-10,-10)(10,10)
\psSolid[object=cube,a=3,action=draw*,trunccoeff=.5,trunc=all,RotZ=30] %cube
\end{pspicture}
\\ \hline
\RDD{trunc}=all & \RDD{trunc}=0  2 4  & \RDD{trunccoeff}=.5 \RDI{trunccoeff}{pst-sol3d}\\ \hline
\end{tabular} 

\bigskip

\begin{tabular}{|c|c|c|} \hline 
 \multicolumn{3}{|c|}{ \BS{psSolid}[object=cube,a=3,action=draw,\RDD{chanfrein},RotZ=30] \RDI{chanfrein}{pst-sol3d}} \\  \hline 
\begin{pspicture}(-10,-10)(10,10)
\psSolid[object=cube,a=3,action=draw*,chanfrein,RotZ=30] %cube
\end{pspicture}
&
\begin{pspicture}(-10,-10)(10,10)
\psSolid[object=cube,a=3,action=draw*,chanfrein,chanfreincoeff=.2,RotZ=30] %cube
\end{pspicture}
 & 
\begin{pspicture}(-10,-10)(10,10)
\psSolid[object=cube,a=3,action=draw*,chanfreincoeff=.5,chanfrein,RotZ=30] %cube
\end{pspicture}
\\ \hline
\RDD{chanfrein} & chanfrein,\RDD{chanfreincoeff}=.2 \RDI{chanfreincoeff}{pst-sol3d} & chanfrein,\RDD{chanfreincoeff}=.5 \\ \hline
\end{tabular} 
\bigskip

\bigskip

\begin{tabular}{|c|c|c|} \hline 
 \multicolumn{3}{|c|}{ \BS{psSolid}[object=cube,a=3,action=draw**,hollow,affinage=0,RotZ=30] } \\  \hline 
\begin{pspicture}(-10,-10)(10,10)
\psSolid[object=cube,a=3,action=draw**,hollow,affinage=0,RotZ=30] %cube
\end{pspicture}
&
\begin{pspicture}(-10,-10)(10,10)
\psSolid[object=cube,a=3,action=draw**,hollow,affinage=3 4,RotZ=30,] %cube
\end{pspicture}
 & 
\begin{pspicture}(-10,-10)(10,10)
\psSolid[object=cube,a=3,action=draw**,hollow,affinage=all,RotZ=30,] %cube
\end{pspicture}
\\ \hline
\RDD{hollow} \RDI{hollow}{pst-sol3d},\RDD{affinage}=3 \RDI{affinage}{pst-sol3d}& \RDD{hollow},,\RDD{affinage}=3 4  & \RDD{hollow},\RDD{affinage}=all \\ \hline
\end{tabular} 
\bigskip

%========================================================================
\subsection{Positionnement}
\psset{unit=.25cm}

\begin{tabular}{|c|c|c|} \hline 
 \multicolumn{3}{|c|}{ \BS{psSolid}[\RDD{object=parallelepiped},a=1,b=2,c=3,action=draw](1 0 0) \RDI{object=parallelepiped}{pst-sol3d}} \\  \hline 
\begin{pspicture}(-10,-10)(10,10)
 \psframe(-10,-10)(10,10)
 \psSolid[object=parallelepiped,a=1,b=2,c=3,action=draw,linestyle=dotted]% 
\psSolid[object=parallelepiped,a=1,b=2,c=3,action=draw](1 0 0)%
\axesIIID[linecolor=red](0,0,0)(1,1,1)
\end{pspicture}
&
\begin{pspicture}(-10,-10)(10,10)
 \psframe(-10,-10)(10,10)
 \psSolid[object=parallelepiped,a=1,b=2,c=3,action=draw,linestyle=dotted]% 
\psSolid[object=parallelepiped,a=1,b=2,c=3,action=draw](0 1 0)%
\axesIIID[linecolor=red](0,0,0)(1,1,1)
\end{pspicture}
&
\begin{pspicture}(-10,-10)(10,10)
 \psframe(-10,-10)(10,10)
 \psSolid[object=parallelepiped,a=1,b=2,c=3,action=draw,linestyle=dotted]% 
\psSolid[object=parallelepiped,a=1,b=2,c=3,action=draw](0 0 1)%
\axesIIID[linecolor=red](0,0,0)(1,1,1)
\end{pspicture}\\ \hline
(1 0 0) & (0 1 0) & (0 0 1)\\ \hline
\end{tabular} 
\bigskip

\begin{tabular}{|c|c|c|} \hline 
 \multicolumn{3}{|c|}{ \BS{psSolid}[\RDD{object=parallelepiped},a=1,b=2,c=3,action=draw] } \\  \hline 
\begin{pspicture}(-10,-10)(10,10)
 \psframe(-10,-10)(10,10)
 \psSolid[object=parallelepiped,a=1,b=2,c=3,action=draw,linestyle=dotted]% 
\psSolid[object=parallelepiped,a=1,b=2,c=3,action=draw,RotX=30]%
\axesIIID[linecolor=red](0,0,0)(1,1,1)
\end{pspicture}
&
\begin{pspicture}(-10,-10)(10,10)
 \psframe(-10,-10)(10,10)
\psSolid[object=parallelepiped,a=1,b=2,c=3,action=draw,linestyle=dotted]% 
\psSolid[object=parallelepiped,a=1,b=2,c=3,action=draw,RotY=30]%
\axesIIID[linecolor=red](0,0,0)(1,1,1)
\end{pspicture}
&
\begin{pspicture}(-10,-10)(10,10)
 \psframe(-10,-10)(10,10)
 \psSolid[object=parallelepiped,a=1,b=2,c=3,action=draw,linestyle=dotted]% 
\psSolid[object=parallelepiped,a=1,b=2,c=3,action=draw,RotZ=30]%
\axesIIID[linecolor=red](0,0,0)(1,1,1)
\end{pspicture}\\ \hline
RotX=30 & RotY=30 & RotZ=30\\ \hline
\end{tabular}

%\subsection{title}
%
%\psset{unit=.25cm}
%
%\begin{tabular}{|c|c|c|} \hline 
% \multicolumn{3}{|c|}{ \BS{psSolid}[\RDD{object=parallelepiped},a=1,b=2,c=3,action=draw](1 0 0) } \\  \hline 
%\begin{pspicture}(-10,-10)(10,10)
% \psframe(-10,-10)(10,10)
%
%\psSolid[object=parallelepiped,a=1,b=2,c=3,action=draw,show=all]%
%\axesIIID[linecolor=red](0,0,0)(1,1,1)
%\end{pspicture}
%&
%\begin{pspicture}(-10,-10)(10,10)
% \psframe(-10,-10)(10,10)
%
%\psSolid[object=parallelepiped,a=1,b=2,c=3,action=draw,num=all]%
%\axesIIID[linecolor=red](0,0,0)(1,1,1)
%\end{pspicture}
%&
%\begin{pspicture}(-10,-10)(10,10)
% \psframe(-10,-10)(10,10)
%\psSolid[object=parallelepiped,a=1,b=2,c=3,action=draw,show=all,num=all]%
%\axesIIID[linecolor=red](0,0,0)(1,1,1)
%\end{pspicture}\\ \hline
%show=all & num=all & show=all,num=all \\ \hline
%\end{tabular} 
%\bigskip
%
%\begin{tabular}{|c|c|c|} \hline 
% \multicolumn{3}{|c|}{ \BS{psSolid}[\RDD{object=parallelepiped},a=1,b=2,c=3,action=draw](1 0 0) } \\  \hline 
%\begin{pspicture}(-10,-10)(10,10)
% \psframe(-10,-10)(10,10)
%
%\psSolid[object=parallelepiped,a=1,b=2,c=3,action=draw,show=0 1 2 3]%
%\axesIIID[linecolor=red](0,0,0)(1,1,1)
%\end{pspicture}
%&
%\begin{pspicture}(-10,-10)(10,10)
% \psframe(-10,-10)(10,10)
%
%\psSolid[object=parallelepiped,a=1,b=2,c=3,action=draw,num=0 1 2 3]%
%\axesIIID[linecolor=red](0,0,0)(1,1,1)
%\end{pspicture}
%&
%\begin{pspicture}(-10,-10)(10,10)
% \psframe(-10,-10)(10,10)
%\psSolid[object=parallelepiped,a=1,b=2,c=3,action=draw,show=0 1 2 3,num=0 1 2 3]%
%\axesIIID[linecolor=red](0,0,0)(1,1,1)
%\end{pspicture}\\ \hline
%show=0 1 2 3 & num=0 1 2 3 & show=0 1 2 3,num=0 1 2 3 \\ \hline
%\end{tabular} 

%\bigskip
%\psset{unit=.5cm}
%\begin{pspicture}(-3,-2.5)(7,2.5)
%%\psset{viewpoint=50 20 20 rtp2xyz,Decran=40}
%\psSolid[action=draw,object=cube,RotZ=30,show=0 1 2 3,num=0 1 2 3]%
%\end{pspicture}
%
%\begin{pspicture}(-3,-2.5)(7,2.5)
%%\psset{viewpoint=50 20 20 rtp2xyz,Decran=40}
%\psSolid[action=draw,object=parallelepiped,RotZ=30,show=0 1 2 3,num=0 1 2 3]%
%\end{pspicture}


%\subsection{Coloriage numérotation}
\SbSSCT{Coloriage numérotation}coloring and numbering{}

\psset{unit=.15cm}






\begin{tabular}{|c|c|c|}\hline
 \multicolumn{3}{|c|}{ \BS{psSolid}[\RDD{fcol}=0 (green) 1 (red) 4 (cyan) 13 (blue) 40 (black), object=cube,mode=3] \RDI{fcol}{pst-sol3d}} 
 \\  \hline 
\begin{pspicture}(-10,-10)(10,10)  
 \psSolid[fcol=0 (green) 1 (red)  4 (cyan) 13 (blue) 40 (black), object=cube,mode=3]%
 \end{pspicture}
  &  
 \begin{pspicture}(-10,-10)(10,10)
  \psSolid[numfaces=all, object=cube,mode=3]%
  \end{pspicture}  
  & 
  \begin{pspicture}(-10,-10)(10,10)
   \psSolid[numfaces=0 1 2 3, object=cube,mode=3]%
   \end{pspicture} 
   \\ \hline  
\RDD{fcol}=0 (green) 1 (red)  ...  &  \RDD{numfaces}=all  \RDI{numfaces}{pst-sol3d}& \RDD{numfaces}=0 1 2 3
    \\ \hline 
\end{tabular} 
\bigskip

\begin{tabular}{|c|c|c|}\hline
 \multicolumn{3}{|c|}{ \BS{psSolid}[\RDD{fcol}=0 (green) 1 (red) 2 (cyan) 3 (magenta), object=parallelepiped,mode=3] } 
 \\  \hline 
\begin{pspicture}(-10,-10)(10,10)
 \psSolid[fcol=0 (green) 1 (red) 2 (cyan) 3 (magenta), object=parallelepiped,mode=3]%
\end{pspicture} 
&
  \begin{pspicture}(-10,-5)(10,15)
   \psSolid[fcol=0 (green) 1 (red) 2 (cyan) 3 (magenta), object=cylindre,h=4,ngrid=4,mode=3]%
   \end{pspicture}
   & 
\begin{pspicture}(-10,-10)(10,10)
\psSolid[fcol=0 (green) 1 (red) 2 (cyan) 3 (magenta) , object=tore,r1=2,r0=1,ngrid=4,mode=3]%
   \end{pspicture} 
   \\ \hline
\RDD{fcol}= 0 (green)  1 (red) ... & \RDD{numfaces}=all & \RDD{numfaces}=0 1
   \\ \hline

\end{tabular}  
 
 \newpage
 
% \subsection{Dans une prochaine version}
\SbSSCT{Dans une prochaine version}{In a future version}

 
%\subsubsection{Surface d'après une équation}
\SbSbSSCT{Surface d'après une équation}{Equation define surface}

\begin{tabular}{|c|c|c|} \hline 
%\psset{llx=-.5cm,lly=-.5cm,urx=.5cm,ury=0.5cm,fillstyle=none,linewidth=2pt}
\psset{fillcolor=yellow,fillstyle=none,linecolor=blue,unit=.4cm}

% \psset{unit=0.5}
% \psset{lightsrc=30 30 25}
 \psset{viewpoint=50 40 30 rtp2xyz,Decran=50}
\begin{pspicture}(-7,-8)(7,8)
 \psSurface[algebraic,ngrid=.25 .25,hue=0 1](-6,-6)(6,6){sin(x) *cos(y)}
\end{pspicture}
   \\ \hline
 \BSS{psSurface}[algebraic,ngrid=.25 .25,hue=0 1](-6,-6)(6,6)\AC{sin(x) *cos(y)} \BSI{psSurface}{pst-sol3d}
   \\ \hline
\end{tabular}

%\subsubsection{Fusion de 2 solides}
\SbSbSSCT{Fusion de 2 solides}Fusion of two solids{}
  \psset{unit=1cm}
\begin{center} 
%  \begin{pspicture}(-1,1)(10,8)
%  \psframe(-1,-1)(10,10)
 \begin{animateinline}[poster=first, palindrome,autoplay]{5}%
 \multiframe{30}{iAngle=10+30}{\mydessin{\iAngle} }
 \end{animateinline}
%  \end{pspicture}
 
% \begin{pspicture}(-.3\linewidth,-12)(.35\linewidth,30)
%\psframe(-.3\linewidth,-12)(.35\linewidth,30)
%\psset{unit=.15cm} 
%\psset{solidmemory}
%\psSolid[object=cylindrecreux,h=10,r=2,fillcolor=white,mode=4,name=A1,incolor=green!50](0,0,-3)
%\psSolid[object=conecreux,h=15,r=2,RotY=-60,fillcolor=white,incolor=red!50,mode=5,name=B1](4,0,0)%
%\psSolid[object=fusion,action=draw**,base=A1 B1,](0,0,0)
%\composeSolid
% \end{pspicture}

 \psset{unit=1cm}
\bigskip 
 \begin{tabular}{|l|} \hline 
\BSS{psset}\AC{solidmemory}\\
\\
\BS{psSolid}[object=cylindrecreux,h=10,r=2,fillcolor=white,mode=4,name=A1,incolor=green!50](0,0,-3)\\
\BS{psSolid}[object=conecreux,h=15,r=2,RotY=-60,fillcolor=white,incolor=red!50,mode=5,name=B1](4,0,0)\\
\BS{psSolid}[object=fusion,action=draw**,base=A1 B1,](0,0,0)\\
\BSS{composeSolid} \BSI{composeSolid}{pst-sol3d}
    \\ \hline
 \end{tabular}
\end{center} 
 

 \begin{center}

 \end{center} 



