\documentclass[t,smaller,compress]{beamer}
\usepackage[utf8]{inputenc}
\usepackage[ngerman]{babel}
\usetheme{BerlinFU}

\begin{document}
\title{Wissenschaftliches Schreiben}
\subtitle{Conference on Fabulous Presentations, 2009}
\author{Dipl. Frank.-Wiss. Jana Voß\inst{1} \and Dr.-Ing. Herbert Voß\inst{2}}
\institute{\inst{1}BTO \and \inst{2}ZEDAT}
\date{\today}
\titlegraphic{silberlaube2}
\fachbereich{ZEDAT}
\maketitle

\begin{frame}{Inhaltsverzeichnis}
\tableofcontents
\end{frame}

\section{Eine Einführung}
\subsection{Die Vorlage}
\begin{frame}{Die \LaTeX-Vorlage für \texttt{beamer}}
\begin{itemize}[<+-| alert@+>]
\item Die Vorlage folgt prinzipiell dem Corporate Design der FU
\item Im Gegensatz zu \texttt{PowerPoint} leichter erweiterbar
\item Bessere Steuerung von Inhaltsverzeichnissen.
\end{itemize}
\end{frame}
\end{document}
