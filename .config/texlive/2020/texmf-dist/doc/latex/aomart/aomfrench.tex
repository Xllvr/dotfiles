\documentclass{aomart}
\usepackage[utf8]{inputenx}
\usepackage[english,frenchb]{babel}
\title{Théorème intégral de Cauchy}
\author{Wikip\'edia}
\address{\url{https://fr.wikipedia.org/}}
\copyrightnote{Droit d'auteur: les textes sont disponibles sous
  licence Creative Commons attribution, partage dans les mêmes
  conditions; d’autres conditions peuvent s’appliquer. Voyez les
  conditions d’utilisation pour plus de détails, ainsi que les crédits
  graphiques. En cas de réutilisation des textes de cette page, voyez
  comment citer les auteurs et mentionner la licence.} 
\volumenumber{160}
\issuenumber{1}
\publicationyear{2017}
\papernumber{12}
\startpage{17}
\endpage{}

\begin{document}

{\selectlanguage{french}% We really do not need this since French is
                        % the main language of the paper
\begin{abstract}
  En analyse complexe, le théorème intégral de Cauchy, ou de
  Cauchy-Goursat, est un important résultat concernant les intégrales
  curvilignes de fonctions holomorphes dans le plan complexe. D'après
  ce théorème, si deux chemins différents relient les deux mêmes
  points et si une fonction est holomorphe «entre» les deux chemins,
  alors les deux intégrales de cette fonction suivant ces chemins sont
  égales.
\end{abstract}}

{\selectlanguage{english}%
  \begin{abstract}
    In mathematics, the Cauchy integral theorem (also known as the
    Cau\-chy–Gou\-r\-sat theorem) in complex analysis, named after
    Augustin-Louis Cauchy, is an important statement about line
    integrals for holomorphic functions in the complex
    plane. Essentially, it says that if two different paths connect
    the same two points, and a function is holomorphic everywhere in
    between the two paths, then the two path integrals of the function
    will be the same.
  \end{abstract}}

\maketitle

\tableofcontents

\section{Énoncé}

Le théorème est habituellement formulé pour les lacets (c'est-à-dire
les chemins dont le point de départ est confondu avec le point
d'arrivée) de la manière suivante.

\begin{description}
\item[Soient]
  \begin{enumerate}
  \item $U$ un ouvert simplement connexe de $\mathbb{C}$;
  \item $f : U \to \mathbb{C}$ une fonction continue sur $U$ et
    possédant une dérivée complexe sauf éventuellement en un nombre
    fini de points; 
  \item $\gamma$ un lacet rectifiable dans $U$.
  \end{enumerate}

  \item[Alors]
    \begin{displaymath}
      \int _{\gamma }f(z)\,\mathrm {d} z=0.
    \end{displaymath}
  \end{description}

\section{Condition de simple connexité}

La condition que $U$ est simplement connexe signifie que $U$ n'a pas
de «trou»; par exemple, tout disque ouvert
$U=\{z,\mid z-z_{0}\mid <r\}$, satisfait à cette condition.

La condition est cruciale; par exemple, si $\gamma$ est le cercle
unité alors l'intégrale sur ce lacet de la fonction $f(z) = 1/z$ est
non nulle; le théorème intégral de Cauchy ne s'applique pas ici
puisque $f$ n'est pas prolongeable par continuité en $0$.

\section{Démonstration}

Par des arguments de continuité uniforme de $f$ sur des
$\epsilon$-voisinages compacts de l'image de $\gamma$ dans $U$,
l'intégrale de $f$ sur $\gamma$ est limite d'intégrales de $f$ sur des
lacets polygonaux~\cite[p.~111]{Hahn96}. Il suffit alors, pour
conclure, d'invoquer le lemme de Goursat.

On peut également dans le cas où f est holomorphe en tout point de $U$
considérer la famille de lacets  $\gamma _{{\alpha
  }}(t)=z_{0}+(1-\alpha )(\gamma (t)-z_{0})$ avec $\alpha \in [0,1]$.

\section{Conséquences}

\begin{enumerate}
\item Sous les hypothèses du théorème, $f$ possède sur $U$ une
  primitive complexe $F$. En effet, quitte à remplacer $U$ par l'une
  de ses composantes connexes, on peut supposer que U est connexe. En
  fixant alors un point arbitraire $z_0$ de U et en posant
  \begin{displaymath}
    F(z)=\int _{{P(z)}}f(\xi )\,{\mathrm  d}\xi ,
  \end{displaymath}
  où $P(z)$ est n'importe quel chemin rectifiable dans $U$ de $z_0$ à
  $z$ (d'après le théorème, la valeur de $F(z)$ ne dépend pas du choix
  de $P(z)$) et en adaptant à la variable complexe la démonstration du
  premier théorème fondamental de l'analyse, on en déduit alors que
  $F$ est holomorphe sur $U$ et que $F’ = f$.


\item Pour une telle primitive on a immédiatement: pour tout chemin
  continûment différentiable par morceaux $\gamma$ de $a$ à $b$ dans
  $U$:
  \begin{displaymath}
    \int _{\gamma }f(z)\,{\mathrm  {d}}z=F(b)-F(a).
  \end{displaymath}

  \item Le peu d'hypothèses requises sur $f$ est très intéressant, parce
    qu'on peut alors démontrer la formule intégrale de Cauchy pour ces
    fonctions, et en déduire qu'elles sont en fait indéfiniment
    dérivables.
    
  \item Le théorème intégral de Cauchy est considérablement généralisé
    par le théorème des résidus.
      
  \item Le théorème intégral de Cauchy est valable sous une forme
    légèrement plus forte que celle donnée ci-dessus. Supposons que
    $U$ soit un ouvert simplement connexe de $\mathbb{C}$ dont la
    frontière est un lacet simple rectifiable $\gamma$. Si $f$ est une
    fonction holomorphe sur $U$ et continue sur l'adhérence de $U$,
    alors l'intégrale de $f$ sur $\gamma$ est nulle~\cite[p.~396
    et~420]{Lin11}.

  \end{enumerate}

\section{Exemple}

Pour tout complexe $\alpha$ , la fonction $f(z):=\frac {{\mathrm {e}
    }^{\mathrm {i} z}}{z^{\alpha }}$, où l'on a choisi la
détermination principale de la fonction puissance, est holomorphe sur
le plan complexe privé de la demi-droite $ \mathbb {R} ^{-}$. Son
intégrale sur tout lacet de ce domaine est donc nulle. Ceci permet de
montrer que les intégrales semi-convergentes
\begin{displaymath}
J_{c}(\alpha ):=\int _{0}^{\infty }{\frac {\cos t}{t^{\alpha
    }}}\,\mathrm {d} t\quad {\text{et}}\quad J_{s}(\alpha ):=\int
_{0}^{\infty }{\frac {\sin t}{t^{\alpha }}}\,\mathrm {d} t\quad
{\text{pour}}\quad \mathrm {Re} (\alpha )\in \left]0,1\right[
\end{displaymath}
(où $\mathrm {Re}$ désigne la partie réelle) sont respectivement
égales à
\begin{displaymath}
J_{c}(\alpha )=\cos((1-\alpha )\pi /2)\Gamma (1-\alpha )\quad
{\text{et}}\quad J_{s}(\alpha )=\sin((1-\alpha )\pi /2)\Gamma
(1-\alpha ),
\end{displaymath}
où $\Gamma$ \  désigne la fonction gamma et  $\cos$, $\sin$ sont les
fonctions cosinus et sinus de la variable complexe.

Notons $\alpha =a+\mathrm {i} b$ avec $a\in \left]0,1\right[$ et $b\in
\mathbb {R}$. On intègre $f$ (l'intégrale est nulle) sur le lacet
formé du segment réel $\left[\varepsilon ,R\right]$ et du segment
imaginaire pur $\mathrm {i} \left[R,\varepsilon \right]$, joints par
les quarts de cercles $R\mathrm {e} ^{\left[0,\mathrm {i} \pi
    /2\right]}$ et $\varepsilon \mathrm {e} ^{\left[\mathrm {i} \pi /2,0\right]}$, puis on fait tendre $R$ vers $ +\infty $ et $\varepsilon$  vers $0$.

Les intégrales sur les deux quarts de cercles tendent vers $0$ car
\begin{multline*}
\left|\int _{0}^{\pi /2}{\frac {{\mathrm {e} }^{\mathrm {i} R\mathrm
        {e} ^{\mathrm {i} \theta }}}{R^{\alpha }\mathrm {e} ^{\mathrm
        {i} \alpha \theta }}}\mathrm {i} R\mathrm {e} ^{\mathrm {i}
    \theta }\,\mathrm {d} \theta \right| \\
\leq R^{1-a}\int _{0}^{\pi
  /2}{\mathrm {e} }^{-R\sin \theta }\,\mathrm {d} \theta \leq
R^{1-a}\int _{0}^{\pi /2}{\mathrm {e} }^{-2R\theta /\pi }\,\mathrm {d}
\theta ={\frac {\pi }{2}}R^{-a}(1-\mathrm {e} ^{-R})
\end{multline*}
et
\begin{displaymath}
\lim _{R\to +\infty }R^{-a}(1-\mathrm {e} ^{-R})=\lim _{\varepsilon
  \to 0^{+}}\varepsilon ^{-a}(1-\mathrm {e} ^{-\varepsilon })=0.
\end{displaymath}
L'intégrale sur le segment imaginaire est égale à
\begin{displaymath}
\int _{R}^{\varepsilon }{\frac {{\mathrm {e} }^{-y}}{y^{\alpha
    }\mathrm {e} ^{\alpha \mathrm {i} \pi /2}}}\mathrm {i} \,\mathrm
{d} y=-\mathrm {e} ^{(1-\alpha )\mathrm {i} \pi /2}\int _{\varepsilon
}^{R}y^{-\alpha }\mathrm {e} ^{-y}\,\mathrm {d} y\to -\mathrm {e}
^{(1-\alpha )\mathrm {i} \pi /2}\Gamma (1-\alpha ).
\end{displaymath}
L'intégrale sur le segment réel tend vers
$J_{c}(\alpha )+\mathrm {i} J_{s}(\alpha )$, qui est donc égal à
$\mathrm {e} ^{(1-\alpha )\mathrm {i} \pi /2}\Gamma (1-\alpha )$.

De même (en rempaçant $b$ par $-b$,
$J_{c}({\overline {\alpha }})+\mathrm {i} J_{s}({\overline {\alpha
  }})=\mathrm {e} ^{(1-{\overline {\alpha }})\mathrm {i} \pi /2}\Gamma
(1-{\overline {\alpha }})$ donc (en prenant les conjugués des deux
membres) $J_{c}(\alpha )-\mathrm {i} J_{s}(\alpha
)=\mathrm {e} ^{-(1-\alpha )\mathrm {i} \pi /2}\Gamma (1-\alpha )$.

On a donc bien
\begin{multline*}
2J_{c}(\alpha )=\mathrm {e} ^{(1-\alpha )\mathrm {i} \pi /2}\Gamma
(1-\alpha )+\mathrm {e} ^{-(1-\alpha )\mathrm {i} \pi /2}\Gamma
(1-\alpha )=\\
2\cos((1-\alpha )\pi /2)\Gamma (1-\alpha )
\end{multline*}
et
\begin{multline*}
2\mathrm {i} J_{s}(\alpha )=\mathrm {e} ^{(1-\alpha )\mathrm {i} \pi
  /2}\Gamma (1-\alpha )-\mathrm {e} ^{-(1-\alpha )\mathrm {i} \pi
  /2}\Gamma (1-\alpha )=\\
2\mathrm {i} \sin((1-\alpha )\pi /2)\Gamma
(1-\alpha ).
\end{multline*}
Par exemple,
$\frac {1}{2}J_{c}(1/2)={\frac {1}{2}}J_{s}(1/2)={\frac {1}{2}}{\sqrt
{\frac {\pi }{2}}}$ (l'intégrale de Fresnel). On peut de plus
remarquer que
$\lim _{\mathrm {Re} (\alpha )<1,\alpha \to 1}J_{s}(\alpha )={\frac
  {\pi }{2}}=\int _{0}^{\infty }{\frac {\sin t}{t}}\,\mathrm {d} t$
(l'intégrale de Dirichlet).

\section{Surfaces de Riemann}

Le théorème intégral de Cauchy se généralise dans le cadre de la
géométrie des surfaces de Riemann.

\bibliography{aomsample}
\bibliographystyle{aomplain}


\end{document}
