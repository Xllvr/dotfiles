
\documentclass[a4paper]{article}

% F O N T S
\usepackage[T1]{fontenc}  % Use T1 encoded cm-super fonts.
\usepackage{microtype}    % Improve typesetting.
\usepackage{fix-cm}       % Support for arbitrary font size for cm.
\usepackage{xspace}
\usepackage{listings}
\lstset{numbers=left,basicstyle=\small,language=Tex}

% P A G E   L A Y O U T
% Use geometry package to set up margins.
% A4 paper is 8.27 × 11.69 inch.
\RequirePackage[a4paper, width=6.27in, height=9.69in, includehead]{geometry}

% Set line spacing.
\RequirePackage{setspace}
\onehalfspacing

% Setup the hyperref package for enabling links, bookmarks, and PDF properties.
\RequirePackage[hyphens]{url} % Embedding URL's in document.
\RequirePackage[bookmarks=true]{hyperref}
\hypersetup
{
	pdfstartview = {FitV},
	pdfpagemode  = UseNone,
	colorlinks   = true,
	linkcolor    = red,
	citecolor    = blue,
	filecolor    = magenta,
	urlcolor     = cyan
}

% C L E V E R E F
% 
% Must come as late as possible, especially after hyperref.
\RequirePackage[capitalize]{cleveref}

% Disable the automatic abbreviations of equations and figures.
\crefname{equation}{Equation}{Equations}
\crefname{figure}{Figure}{Figures}
\Crefname{equation}{Equation}{Equations}
\Crefname{figure}{Figure}{Figures}

% Change the way links are produced in PDF documents.
\crefformat{chapter}{#2Chapter~#1#3}
\crefformat{section}{#2Section~#1#3}
\crefformat{figure}{#2Figure~#1#3}
\crefformat{equation}{#2Equation~#1#3}
\crefformat{table}{#2Table~#1#3}
\Crefformat{chapter}{#2Chapter~#1#3}
\Crefformat{section}{#2Section~#1#3}
\Crefformat{figure}{#2Figure~#1#3}
\Crefformat{equation}{#2Equation~#1#3}
\Crefformat{table}{#2Table~#1#3}
\creflabelformat{equation}{#2#1#3}

% Set the front page matter.

\newcommand{\SgTitle}[1]{\hypersetup{pdftitle={#1}}\title{#1}}
\newcommand{\SgAuthor}[1]{\hypersetup{pdfauthor={#1}}}
\newcommand{\SgSubject}[1]{\hypersetup{pdfsubject={#1}}}
\newcommand{\SgKeywords}[1]{\hypersetup{pdfkeywords={#1}}}

\newcommand{\SgFrontMatter}[3]%
{%
	\SgTitle{#1}%
	\SgAuthor{#2}%
	\author{#2 (#3)}
	\maketitle%
	\tableofcontents%
}

\newcommand{\SPhdThesis}{\textbf{SPhdThesis}\xspace}
\newcommand{\Code}[1]{\texttt{#1}\xspace}

\begin{document}
\SgFrontMatter{SPhdThesis class documentation}{Saurabh Garg}{\url{saurabhgarg@mysoc.net}}
\SgKeywords{PhD thesis; latex template; SoC; NUS}

\section{Introduction}
I developed \SPhdThesis document class while writing my PhD thesis in School of Computing (SoC), National University of Singapore (NUS). By default, it adheres to the ``NUS Guidelines on Format of Research Thesis Submitted For Examination'' located at \url{http://www.nus.edu.sg/registrar/event/gd-thesisexam.html}\footnote{You need NUSNET username and password to access the document.}. However, it is quite easy to change it for different guidelines. Section~\ref{sec:GettingStarted} (Getting started) describes how to quickly start using \SPhdThesis with default options. The class options used to customize the formatting of the thesis are described in Section~\ref{sec:ClassOptions}. Section~\ref{sec:Tweaking} gives some hints for tweaking the class file and Section~\ref{sec:Issues} list the currently know issues. Although, not related to this document class, Section~\ref{sec:Tools} lists some tools I find useful for working with \LaTeX. \\

An important point to consider before using this document class is that it only supports using \texttt{pdflatex}. It cannot be used with \texttt{latex} to create a dvi file. The implication of using pdflatex is that all your figures have to be in pdf, png or jpg. You cannot use eps figures. However, there are tools to convert eps files to pdf on all platforms.

\section{Getting started}\label{sec:GettingStarted}
The easiest way to start using \SPhdThesis document class is to copy \texttt{SPhdThesis.cls} and \texttt{example\textbackslash{}} \texttt{thesis.tex} to your thesis folder and modify \texttt{thesis.tex}. The following is the complete listing of \texttt{thesis.tex}:

\begin{lstlisting}
\documentclass{SPhdThesis}

% Title page and PDF properties.
\SgSetTitle{A Robust Mesh-based Surface Integration Algorithm}
\SgSetAuthor{Saurabh Garg}
\SgSetAuthorDegrees{List your previous degrees here}
\SgSetYear{2013}
\SgSetDegree{Doctor of Philosophy}
\SgSetDepartment{Department of Computer Science}
\SgSetUniversity{National University of Singapore}
\SgSetDeclarationDate{June 2013}

% The document.
\begin{document}
	\begin{frontmatter}
		\SgAddTitle%
		\SgAddDeclaration%
		%%% Acknowledgements
%%% ----------------------------------------------------------------------
\begingroup
% you can erase the following three lines if you do not use an introductory quote
\cleardoublepage
\let\clearpage\relax
\let\cleardoublepage\relax
\begin{flushright}\slshape
	Put your fancy quote here. Or don't. \\ \medskip
	--- Awesome Author
\end{flushright}

\bigskip
\otherlanguage{ngerman}

\addchap{Vorwort}

\blindmathfalse
\blindtext

\bigskip

\begin{flushright}
	Place, \doctime
\end{flushright}

\endgroup%
		This document gives a quick, relatively minimal example of the use of
\texttt{uafthesis.cls}, while trying to show its features.

This section is contained in \texttt{abstract.tex}.
%
		\SgAddToc% Add table of contents.
		\SgAddLof% Add list of figures.
		\SgAddLot% Add list of tables.
		\SgAddLoa% Add list of algorithms.
	\end{frontmatter}
   
	% Sample University of Calgary Thesis
% This file contains CHAPTER ONE

\chapter{Introduction}

\epigraph{Lorem ipsum dolor sit amet,
consectetuer adipiscing elit.}{Anonymous}

The text above is an \emph{epigraph}. In this typeface, you can have
text in \textbf{boldface}, \emph{italics}, \textbf{\emph{bold
    italics}}, \textsl{slanted}, and \textsc{Small Caps}.

\section{Literature Review}

\blindtext\pagenote{\blindtext}

\blindtext[2]

\begin{defn}
A group~$G$ is said to be \emph{abelian} (or \emph{commutative}) if
for every $a, b \in G$, $a \cdot b = b \cdot a$.
\end{defn}

\blindtext[2]

\section{Contributions of the Thesis}

\blindtext[3]

\begin{table}
  \begin{center}
  \begin{tabular}{c|c}
    A & B \\
    \hline
    1 & 2
  \end{tabular}
  \end{center}
  \caption{Letters and numbers}
\end{table}

\blindtext[3]

	\SgIncludeBib{biblio}
\end{document}
\end{lstlisting}

Line 1 simply defines the document class with default options. \\

Lines 4--11 define the variables for automatically creating the title page and setting the PDF document properties. \\

Lines 15--24 defines the front matter of the thesis. \verb|\SgAddTitle| creates the title page using the values of variables defined above. \verb|\SgAddDeclaration| adds the declaration page. Declaration page must be signed. You can either sign after printing or give the path to the signature image as an argument to \verb|\SgAddDeclaration| like this \verb|\SgAddDeclaration[sign-image]|. \verb|%%% Acknowledgements
%%% ----------------------------------------------------------------------
\begingroup
% you can erase the following three lines if you do not use an introductory quote
\cleardoublepage
\let\clearpage\relax
\let\cleardoublepage\relax
\begin{flushright}\slshape
	Put your fancy quote here. Or don't. \\ \medskip
	--- Awesome Author
\end{flushright}

\bigskip
\otherlanguage{ngerman}

\addchap{Vorwort}

\blindmathfalse
\blindtext

\bigskip

\begin{flushright}
	Place, \doctime
\end{flushright}

\endgroup| includes a file containing the acknowledgments. The acknowledgments must be enclosed in \verb|\begin{acknowledgments}| \verb|...| \verb|\end{acknowledgments}|. Similarly, \verb|This document gives a quick, relatively minimal example of the use of
\texttt{uafthesis.cls}, while trying to show its features.

This section is contained in \texttt{abstract.tex}.
| includes a file containing the abstract. The abstract must be enclosed in \verb|\begin{abstract}| \verb|...| \verb|\end{abstract}|. Note that it is not necessary to create separate files for acknowledgments and abstract and they can be inserted inline. \\

Lines 20--23 inserts the table of contents, list of figures, list of tables, and list of algorithms. The order of list of figures, list of tables, and list of algorithms can be changed and not all of the them are necessary. \\

After the front matter is specified, you are free to add main content of the thesis. You can either create one tex file per chapter and include them using \verb|\input{}| or simply include everything in the main file. You can see example\textbackslash{}figure.tex, example\textbackslash{}table.tex, example\textbackslash{}algorithm.tex for examples of how to include figures, tables, and algorithms, respectively. \\

Finally, the bibliography should be included using \verb|\SgIncludeBib{}|. If your bibliography is divided in multiple files, than they can be separated by comma (,).

\section{Class options}\label{sec:ClassOptions}
\SPhdThesis document class uses key=value format for specifying options. Two options are separated by a comma (,). For example \verb|\documentclass[media=print,title=upper]{SPhdThesis}|. The following options are provided:\\

\noindent\textbf{Media type}\\
\Code{media} = \Code{screen} or \Code{print} (default value is \Code{screen}). \Code{media} internally defines the formatting settings for viewing document on screen or printing. When formatting for screen, colors are used for links, tables, and algorithms whereas only black color is used for them in the print mode.\\

\noindent\textbf{Title page casing:}\\
\Code{titlecase} = \Code{lower} or \Code{upper} (default is \Code{upper}). NUS requires the title page to be formatted in uppercase but I don't like uppercase. So, I used this option to format the title page in the lowercase everywhere expect the final submission.\\

\noindent\textbf{Line spacing:}\\
\Code{linespacing} = \Code{onehalf} or \Code{double} (default is \Code{onehalf}). NUS requires the thesis to be formatted with double line spacing which in my opinion doesn’t look very good. So, I used this option to format the thesis with one half line spacing everywhere expect the final submission.\\

\noindent\textbf{Font size:}\\
\Code{fontsize} = \Code{11pt} or \Code{12pt} (default is \Code{11pt}). NUS recommends font size to be 11 or 12 points. This option can be used to format the thesis in either 11 or 12 points. \\

\noindent\textbf{Open type:}\\
\Code{open} = \Code{right} or \Code{any} (default value is \Code{any}). This option defines which page a new chapter will start from: for \Code{open}{}=\Code{any} a new chapter will start from the next new page. However, for \Code{open}{}=\Code{right} a new chapter will only start from next right hand side (odd) page.

\section{Tweaking the document class}\label{sec:Tweaking}
The \verb|SPhdThesis.cls| file is heavily commented and should be mostly easy to understand. It is divided into sections and the order of some sections are important. For example, captions section should come before packages and hyperref section. All these dependencies are documented in the \verb|SPhdThesis.cls| file. \\

Here are some hints to customize the formatting of the thesis:
\begin{itemize}
	\item Chapter and section title formatting: The appearance of chapter and section title in various places in the thesis is defined in the fonts section. The variables defined in the font section are used by TOC (for formatting table of contents), chapter heading (for formatting the chapter heading), and fancy header (for formatting the headers and footers) sections.
	\item Chapter heading: The chapter heading is formatted using Lenny style of the fncychap package. If you like to change to another style supported by fncychap package you can do so in chapter heading section.
	\item Page layout: The paper size and margins are defined in the page layout section. Some other important properties defined is the page layout section are: paragraph indenting, footnote spacing, forcing image to be on the top on blank pages, distance between a float and text, and distance between two floats.
	\item Fancy header: The appearance of header is changed using fancyhdr package in the fancy header section. There are three parts in the header: left, center, and right. In SPhdThesis class, Chapter \#. Chapter\_Name appears on the left part and page number appears on the right part.
	\item TOC. The appearance of the table of contents is changed using tocloft package in the TOC section. Dots separating chapter titles and page numbers are removed and the appearance of the chapter title and page number is defined.
	\item Table: The appearance of tables is defined in the table section. Apart from various distance used for format table you can change the color of horizontal rules.
	\item Algorithm: The appearance of algorithms is defined in the algorithm section.
\end{itemize}

\section{Known Issues}\label{sec:Issues}
\begin{itemize}
	\item Color of rules in algorithms, unlike tables, are black in both print and screen mode.
\end{itemize}

\section{Tools}\label{sec:Tools}
Some tools which are useful for working with \LaTeX:
\begin{itemize}
	\item \textbf{The Ipe extensible drawing editor}\\
	Ipe is a drawing editor for creating figures in PDF or (encapsulated) Postscript format. It supports making small figures for inclusion into LaTeX-documents as well as making multi-page PDF presentations that can be shown on-line with Acrobat Reader.\\
Website: \url{http://ipe7.sourceforge.net/}
	\item \textbf{Jabref reference manager}\\
	JabRef is an open source bibliography reference manager. The native file format used by JabRef is BibTeX, the standard LaTeX bibliography format. JabRef runs on the Java VM (version 1.6 or newer), and should work equally well on Windows, Linux and Mac OS X.\\
	Website: \url{http://jabref.sourceforge.net/}
	\item \textbf{Gle - Graphics Layout Engine}\\
	GLE (Graphics Layout Engine) is a graphics scripting language designed for creating publication quality figures (e.g., a chart, plot, graph, or diagram). It supports various chart types (including function plot, histogram, bar chart, scatter plot, contour plot, color map, and surface plot) through a simple but flexible set of graphing commands. More complex output can be created by relying on GLE's scripting language, which is full featured with subroutines, variables, and logic control. GLE relies on LaTeX for text output and supports mathematical formulae in graphs and figures. GLE's output formats include EPS, PS, PDF, JPEG, and PNG.\\
	Website: \url{http://glx.sourceforge.net/}
	\item \textbf{Chktex - \LaTeX semantic checker}\\
	Chktex is a tool for catching typographic errors in a latex document.\\
	Website: \url{http://baruch.ev-en.org/proj/chktex/}
	\item \textbf{lacheck - \LaTeX checker}\\
	Lacheck is a tool for finding common mistakes in \LaTeX documents.\\
	Website: \url{http://www.ctan.org/tex-archive/support/lacheck}
\end{itemize}

\end{document}
