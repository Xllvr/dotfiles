\documentclass{ltxdoc}
\usepackage[svgnames,dvipsnames]{xcolor}
\usepackage{graphicx}
\usepackage{fontspec}
\setmainfont{Warnock Pro}
\setsansfont{Cronos Pro}
\usepackage[scale=.8]{FiraMono}
\usepackage{polyglossia}
\setmainlanguage{brazil}
\usepackage{tcolorbox}
\tcbuselibrary{listings,skins,xparse,breakable}
\usepackage{tikz}
\usetikzlibrary{positioning,shapes,snakes}

\newtcblisting[auto counter,list inside=codigo,number within=chapter]{codigo}[2][]{%
	%title={Exemplo~\thetcbcounter:~#2},
	%list entry={\protect\numberline{\thetcbcounter}#2},
	title={#2},
	center,
	colback=green!20,
	colbacktitle=green!50,
	colframe=green,
	listing only,
	fonttitle=\bfseries\sffamily\large,coltitle=black,titlerule=0pt,enhanced,drop fuzzy shadow,breakable,#1%halign title=center
}



\title{\tikz \node[inner sep=0pt,remember picture,overlay] at (0,1){\includegraphics[height=3cm,width=0.4\textwidth,keepaspectratio]{icone}};\\O pacote <<\texttt{abnt}>>}
\author{Youssef Cherem}
\date{2020}

\begin{document}
\frenchspacing 

\begin{center}
	\tikz\node[diamond,shape aspect=1.2,%draw=green,ultra thick,
	remember picture,overlay,%minimum size=0.6\textwidth,
	left color=SpringGreen,shading angle=45,align=center] at (0,0){\includegraphics[height=3cm,width=0.4\textwidth,keepaspectratio]{icone}\\{\Large O pacote <<\texttt{abnt}>>}\\[10pt]\large Youssef Cherem\\[6pt]2019};
\end{center}

\vspace{4cm}

%\maketitle

%\BgThispage

%\tcblistof[\section*]{codigo}{Lista de Códigos}
O pacote \tcboxverb[blank,fuzzy halo=.5mm with green]{abnt}  é uma interpretação suficiente, mas não exaustiva, das normas da ABNT. É uma implementação concisa das normas, proporcionando \textit{facilidade} e \textit{flexibilidade} ao usuário. O uso deste pacote não dispensa conhecimento básico de \LaTeX, e não há qualquer garantia de que será aceito pela sua instituição sem nenhuma modificação. Seu objetivo é reduzir ao mínimo a interferência de comandos alheios às classes comuns.
Sugere-se seu emprego com a classe \tcboxverb[blank,fuzzy halo=.5mm with teal]{book}  ou com a classe \tcboxverb[blank,fuzzy halo=.5mm with red]{scrbook}.

Comandos para elementos na capa e folha de rosto:

\begin{codigo}{}
\orientador{}   \orientadora{} \coorientador{}   \coorientadora{}

\tipotrabalho{}  \local{}   \instituicao{}  \capa  \folhaderosto
\end{codigo}

Para mudar as fontes dos elementos:

\begin{codigo}{}
\titlefont{}  \authorfont{}  \localfont{}  \datefont{}
\end{codigo}

Para elementos pré-textuais (sem números de página) e textuais (com números de página):

\begin{codigo}{}
\pretextual              \textual
\end{codigo}

Esses comandos não são obrigatórios nem aplicados por padrão. O usuário pode redefinir o cabeçalho como quiser, usando os comandos do pacote \tcboxverb[blank,fuzzy halo=.5mm with green]{scrlayer-scrpage}.

Também são disponibilizados ambientes e listas para quadros e mapas, além das usuais figuras e tabelas. Assim, podem ser inseridas listas de “quadros” e “mapas”:

\begin{codigo}{}
\listofquadros
\listofmapas
\end{codigo}

 Esses ambientes são usados da mesma forma que uma figura:

\begin{codigo}{}
\begin{quadro} ... \end{quadro}

\begin{mapa} ... \end{mapa}
\end{codigo}

Para fazer citações, use o ambiente \texttt{citacao}:

\begin{codigo}{}
\begin{citacao} ... \end{citacao}
\end{codigo}

Se o usuário quiser criar outras listas e ambientes, basta seguir o exemplo abaixo:

\begin{codigo}{Criando novos ambientes}
\DeclareFloatingEnvironment[fileext=loe,listname={Lista de esculturas},%
within=none]{escultura}
\DeclareCaptionListFormat{listaescultura}{\esculturaname\ #1#2\hfill--\hfill}
\captionsetup[escultura]{listformat=listaescultura}
\makeatletter
\renewcommand*{\l@escultura}{\@dottedtocline{1}{1.5em}{6.5em}}
\makeatother

\end{codigo}

Aqui, definimos um novo ambiente “escultura”, com um arquivo de lista de extensão \texttt{loe} e nome “Lista de Figuras”, cujo contador não recomeça em todos os capítulos (\texttt{within=none}). \verb|\l@escultura| serve para redefinir os espaços na lista. Para alternativas de formatação, referir-se à documentação dos pacotes \texttt{newfloat} e \texttt{caption}. 

Uma alternativa simplificada para essas definições está disponível ao utilizar uma das classes \tcboxverb[blank,fuzzy halo=.5mm with red]{KOMA-Script} (\texttt{scrbook}, \texttt{scrartcl} e \texttt{scrreport}):

\begin{codigo}{}
\DeclareNewTOC[%
type=exercicio,%
types=exercicios,%
float,% define a floating environment
floattype=4,%
name=Exercício,%
%counterwithin=chapter,
listname={Lista de Exercícios},
%tocentryindent=0pt,% <- added
tocentrydynnumwidth,% <- added
tocentrynumsep=0pt% <- added
]{loe}
\setuptoc{loe}{chapteratlist}

\BeforeStartingTOC[loe]{\def\autodot{\hfill~--~\hfill}}
\end{codigo}


Todos os outros comandos são os usuais das classes padrão. Recomenda-se o uso do pacote \texttt{biblatex-abnt} para as normas de citação e referências bibliográficas segundo a ABNT. 

  
 \begin{codigo}{Exemplo com a classe scrbook}
 \documentclass[footsepline=true,headsepline=true]{scrbook}
 \usepackage{abnt}
 \usepackage{xcolor,xcolor-material}
 \usepackage[sfdefault,lf]{FiraSans}
 
 %se quiser linha no rodapé
 \ModifyLayer[addvoffset=-.8ex]{scrheadings.foot.above.line}
 \ModifyLayer[addvoffset=-.8ex]{plain.scrheadings.foot.above.line}
 
%\pagestyle{scrheadings}
%\clearscrheadfoot
%\ihead{\headmark}
%\ohead{\pagemark}
% pretextual já faz isso. 
 
\setkomafont{footsepline}{\color{orange}}
\setkomafont{headsepline}{\color{teal!50!yellow}}
\setkomafont{pagenumber}{\normalfont\bfseries\sffamily\color{red}}
 
%\renewcommand*{\chapterpagestyle}{scrheadings} %não necessário - já implementado
 
%todos esses elementos iguais 
\RedeclareSectionCommands
[beforeskip=\baselineskip, afterskip=\baselineskip,font=\normalsize\bfseries]
{part,chapter,
section,subsection,subsubsection}

 
\setkomafont{chapter}{\color{orange}} \setkomafont{section}{\color{teal}}
\setkomafont{subsection}{\color{green!50!black}}
 
 %\setkomafont{disposition}{\normalsize} % não funciona, redefinição com 
 %\RedeclareSectionCommand ou \RedeclareSectionCommands
 
 \KOMAoptions{headsepline=2pt:\textwidth,footsepline=2pt:\textwidth}
 
%medidas
\DeclareNewLayer[
 background,
 %outermargin,
 %topmargin,
 contents=\layercontentsmeasure
 ]{measurelayer}
\AddLayersToPageStyle{@everystyle@}{measurelayer}
\usepackage{showframe}
 
\begin{document}

\pretextual

\textual

\chapter{Capítulo}
\section{Seção}
\subsection{Subseção}

\chapter{Capítulo}
\section{Seção}
\subsection{Subseção}
\newpage
\subsection{title}
\end{document}

\end{codigo}

 
\begin{tcolorbox}[center,width=.8\textwidth]
\centering 
\includegraphics[width=\linewidth]{koma}
\end{tcolorbox}


\end{document}
