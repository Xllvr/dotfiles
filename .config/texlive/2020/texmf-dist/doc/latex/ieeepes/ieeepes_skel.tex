%%----------------------------------------------------------------------
%% ieeepes_skel.tex
%%
%% Skeleton file for papers for the IEEE Power Engineering Society using
%% package ieeepes.
%%
%% Not copyrighted. Copy this file to a different name and fill in your
%% text.
%%
%% Volker Kuhlmann
%% c/o EEE Dept
%% University of Canterbury
%% Private Bag 4800
%% Christchurch, New Zealand
%% Email: KUHLMAV@ELEC.CANTERBURY.AC.NZ
%%
% 1.3  13Apr99	Updated for ieeepes 4.0.
% 1.2  16Nov95	Fixed discussion, closure. Added summary.
% 1.1  12Nov95	Finished first release.
% 1.02 09Nov95	Option PStimes.
% 1.0  07Nov95	Created.
%%----------------------------------------------------------------------
 
\documentclass[10pt,twoside%
		,draft%		% comment this out for final version
	]{article}


\usepackage[%
	%	psphotos,%	% uncomment those options you want
	%	photofit,%
	%	draft,%
	%	PStimes,%
	]{ieeepes}



\title{Example of the \LaTeX-Package ieeepes for IEEE PES Transactions}

\author{
	V. Kuhlmann\\
	Dept of Electrical and Electronic Engineering\\
	Christchurch, New Zealand
\and
	Second Author\\
	affiliation
}



\begin{document}


% This \maketitle command is required from ieeepes version 4.0, to make
% ieeepes work correctly with newer LaTeX versions.
\maketitle


\begin{abstract}
Put the text of your abstract here.
\end{abstract}



\section{Section}

Text for the first section.


\subsection{Subsection}

Text for the first sub-section.


\subsection{Subsection}

Text for the second sub-section.

\subsubsection{Subsubsection}
Text for the first sub-sub-section.

\subsubsection{Subsubsection}
Text for the second sub-sub-section.


\subsection{Subsection}

Text for the third sub-section.



\section{Section}

Text for the second section.



\section{Figures and Tables}

Text for the third section. This section has figures and tables in it.

\begin{figure}
\centering
\fbox{figure matter}
\caption{This is the caption for figure \#1. Make sure it goes
	\emph{after} the figure!}
\label{figure1}
\end{figure}

And more text in this section.

\begin{table}
\caption{This is the caption for table \#1. Make sure it goes
\emph{before} the table!}
\label{table1}
\centering
table matter
\end{table}

Figure and table references: \figref{figure1}, \tabref{table1}. Use
these at the beginning and within a sentence.

Much better is to use the Figure and Table environments, which will
take care of placing the caption correctly for you. See
\fref{figurelabel} and \tref{tablelabel}.

Using the Figure environment:
\begin{Figure}[h]{figurelabel}% <- don't forget this percent!
	{Caption for figure, Figure environment.}
\fbox{The figure matter goes here.}
\end{Figure}

text

Using the Table environment:
\begin{Table}{tablelabel}% <- don't forget this percent!
	{Caption for table, Table environment.}
\fbox{The table matter goes here.}
\end{Table}

And more text in this section.
And more text in this section.
And more text in this section.
And more text in this section.
And more text in this section.



\section{Equations}

Referencing equations: \equref{equation1} for whatever.
\Equref{equation1} at the beginning of a sentence. 
%
\begin{equation}
equation
\label{equation1}
\end{equation}
%
text



\bibliography{FilenameOfYourBibliography}



\section{Test of Biographies}

text

\begin{biography}{Author 1}[0mm]{file.eps}
text
% there must be enough text in the first paragraph to flow around the
% photo!

text
\end{biography}

\begin{biography}{Author 2}[0mm]{}
text
% there must be enough text in the first paragraph to flow around the
% photo!
% Leave filename empty if photo is to be pasted in.

text
\end{biography}

\begin{biography}{Author 3}[0mm]{nophoto}
text
% Use filename nophoto if you don't want to put a photo there at all.

text
\end{biography}


% The columns on the last page must be justified manually using
% \columnbreak.



\summary

text



\begin{discussion}
	{PAPER NUMBER}%
	{PAPER TITLE}%
	{AUTHOR NAMES}%
	{DISCUSSER NAME}%
	{AFFILIATION INCL ADDRESS}%
	{SHORT AFFILIATION}

text

\end{discussion}



\begin{closure}{AUTHOR NAME}

text

\end{closure}



\end{document}

%%
%% EOF ieeepes_skel.tex
%%----------------------------------------------------------------------
