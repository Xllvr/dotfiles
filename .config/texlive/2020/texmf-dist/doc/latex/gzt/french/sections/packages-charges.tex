\section{Packages chargés par la classe}\label{sec-packages-charges-par}

On a vu que, pour plusieurs de ses fonctionnalités, la \gztauthorcl{} s'appuie
sur des packages qu'elle charge automatiquement. Ceux dont les fonctionnalités
peuvent être utiles aux auteurs sont répertoriés dans la liste suivante qui
indique leur fonction et le cas échéant :
\begin{itemize}
\item la ou les options avec lesquelles ils sont chargés ;
\item les options de la \gztauthorcl{} ou leurs commandes propres permettant de les
  personnaliser.
\end{itemize}
En sus des outils propres à la \gztauthorcl, tous ceux fournis par ces
différents packages sont donc à disposition des auteurs de la \gzt{}.

\begin{description}
\item[\package{xcolor} :] couleurs ;
  \begin{description}
  \item[option par défaut :] \docAuxKey{table} et \docAuxKey{cmyk} ;
  \end{description}
\item[\package{kpfonts} :] police principale du document ;
  \begin{description}
  \item[option par défaut :] \docAuxKey{sfmath}, \docAuxKey{easyscsl},
    \docAuxKey{noDcommand} ;
  \end{description}
\item[\package*{graphicx} :] inclusion d'images ;
\item[\package*{csquotes} :] citations formelles et informelles ;
  \begin{description}
  \item[option par défaut :] \docAuxKey{autostyle} ;
  \end{description}
\item[...] (à suivre).
\end{description}

%%% Local Variables:
%%% mode: latex
%%% eval: (latex-mode)
%%% ispell-local-dictionary: "fr_FR"
%%% TeX-master: "../gzt-fr.tex"
%%% End:
