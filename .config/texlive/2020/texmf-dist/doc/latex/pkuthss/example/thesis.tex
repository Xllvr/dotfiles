% Copyright (c) 2008-2009 solvethis
% Copyright (c) 2010-2016,2018-2019 Casper Ti. Vector
% Public domain.
%
% 使用前请先仔细阅读 pkuthss 和 biblatex-caspervector 的文档,
% 特别是其中的 FAQ 部分和用红色强调的部分。
% 两者可在终端/命令提示符中用
%   texdoc pkuthss
%   texdoc biblatex-caspervector
% 调出。

\documentclass[UTF8]{pkuthss}
% 如果的确须要使脚注按页编号的话,可以去掉后面 footmisc 包的注释。
%\usepackage[perpage]{footmisc}

% 使用 biblatex 排版参考文献,并规定其格式(详见 biblatex-caspervector 的文档)。
% 这里按照西文文献在前,中文文献在后排序(“sorting = ecnyt”);
% 若须按照中文文献在前,西文文献在后排序,请设置“sorting = cenyt”;
% 若须按照引用顺序排序,请设置“sorting = none”。
% 若须在排序中实现更复杂的需求,请参考 biblatex-caspervector 的文档。
\usepackage[backend = biber, style = caspervector, utf8, sorting = ecnyt]{biblatex}

% 对于 linespread 值的计算过程有兴趣的同学可以参考 pkuthss.cls。
\renewcommand*{\bibfont}{\zihao{5}\linespread{1.27}\selectfont}
% 按学校要求设定参考文献列表的段间距。
\setlength{\bibitemsep}{3bp}

% 设定文档的基本信息。
\pkuthssinfo{
	cthesisname = {博士研究生学位论文}, ethesisname = {Doctor Thesis},
	ctitle = {测试文档}, etitle = {Test Document},
	cauthor = {某某},
	eauthor = {Test},
	studentid = {0123456789},
	date = {某年某月},
	school = {某某学院},
	cmajor = {某某专业}, emajor = {Some Major},
	direction = {某某方向},
	cmentor = {某某教授}, ementor = {Prof.\ Somebody},
	ckeywords = {其一,其二}, ekeywords = {First, Second}
}
% 载入参考文献数据库(注意不要省略“.bib”)。
\addbibresource{thesis.bib}

% 普通用户可删除此段,并相应地删除 chap/*.tex 中的
% “\pkuthssffaq % 中文测试文字。”一行。
\usepackage{color}
\def\pkuthssffaq{%
	\emph{\textcolor{red}{pkuthss 文档模版最常见问题:}}

	\texttt{\string\cite}、\texttt{\string\parencite} %
	和 \texttt{\string\supercite} 三个命令分别产生%
	未格式化的、带方括号的和上标且带方括号的引用标记:%
	\cite{test-en},\parencite{test-zh}、\supercite{test-en, test-zh}。

	若要避免章末空白页,请在调用 pkuthss 文档类时加入 \texttt{openany} 选项。

	如果编译时不出参考文献,
	请参考 \texttt{texdoc pkuthss}“问题及其解决”一章
	“上游宏包可能引起的问题”一节中关于 biber 的说明。%
}

\begin{document}
	% 以下为正文之前的部分,默认不进行章节编号。
	\frontmatter
	% 此后到下一 \pagestyle 命令之前不排版页眉或页脚。
	\pagestyle{empty}
	% 自动生成封面。
	\maketitle
	% 版权声明。封面要求单面打印,故须新开右页。
	\cleardoublepage
	% Copyright (c) 2008-2009 solvethis
% Copyright (c) 2010-2017 Casper Ti. Vector
% All rights reserved.
%
% Redistribution and use in source and binary forms, with or without
% modification, are permitted provided that the following conditions are
% met:
%
% * Redistributions of source code must retain the above copyright notice,
%   this list of conditions and the following disclaimer.
% * Redistributions in binary form must reproduce the above copyright
%   notice, this list of conditions and the following disclaimer in the
%   documentation and/or other materials provided with the distribution.
% * Neither the name of Peking University nor the names of its contributors
%   may be used to endorse or promote products derived from this software
%   without specific prior written permission.
%
% THIS SOFTWARE IS PROVIDED BY THE COPYRIGHT HOLDERS AND CONTRIBUTORS "AS
% IS" AND ANY EXPRESS OR IMPLIED WARRANTIES, INCLUDING, BUT NOT LIMITED TO,
% THE IMPLIED WARRANTIES OF MERCHANTABILITY AND FITNESS FOR A PARTICULAR
% PURPOSE ARE DISCLAIMED. IN NO EVENT SHALL THE COPYRIGHT HOLDER OR
% CONTRIBUTORS BE LIABLE FOR ANY DIRECT, INDIRECT, INCIDENTAL, SPECIAL,
% EXEMPLARY, OR CONSEQUENTIAL DAMAGES (INCLUDING, BUT NOT LIMITED TO,
% PROCUREMENT OF SUBSTITUTE GOODS OR SERVICES; LOSS OF USE, DATA, OR
% PROFITS; OR BUSINESS INTERRUPTION) HOWEVER CAUSED AND ON ANY THEORY OF
% LIABILITY, WHETHER IN CONTRACT, STRICT LIABILITY, OR TORT (INCLUDING
% NEGLIGENCE OR OTHERWISE) ARISING IN ANY WAY OUT OF THE USE OF THIS
% SOFTWARE, EVEN IF ADVISED OF THE POSSIBILITY OF SUCH DAMAGE.

% 此处不用 \specialchap,因为学校要求目录不包括其自己及其之前的内容。
\chapter*{版权声明}
% 综合学校的书面要求及 Word 模版来看,版权声明页不用加页眉、页脚。
\thispagestyle{empty}

任何收存和保管本论文各种版本的单位和个人,
未经本论文作者同意,不得将本论文转借他人,
亦不得随意复制、抄录、拍照或以任何方式传播。
否则一旦引起有碍作者著作权之问题,将可能承担法律责任。

% 若须排版二维码,请将二维码图片重命名为“barcode”,
% 转为合适的图片格式,并放在当前目录下,然后去掉下面 2 行的注释。
%\vfill\noindent
%\includegraphics[height = 5em]{barcode}

% vim:ts=4:sw=4


	% 此后到下一 \pagestyle 命令之前正常排版页眉和页脚。
	\cleardoublepage
	\pagestyle{plain}
	% 重置页码计数器,用大写罗马数字排版此部分页码。
	\setcounter{page}{0}
	\pagenumbering{Roman}
	% 中西文摘要。
	% Copyright (c) 2014,2016 Casper Ti. Vector
% Public domain.

\begin{cabstract}
	\pkuthssffaq % 中文测试文字。
\end{cabstract}

\begin{eabstract}
	Test of the English abstract.
\end{eabstract}

% vim:ts=4:sw=4

	% 自动生成目录。
	\tableofcontents

	% 以下为正文部分,默认要进行章节编号。
	\mainmatter
	% 各章节。
	\chapter{Prexy Salaam}

\section{Faceplate Marginalia}

Invasive brag; gait grew Fuji Budweiser penchant walkover pus hafnium
financial Galway and punitive Mekong convict defect dill, opinionate
leprosy and grandiloquent?  Compulsory Rosa Olin
Jackson\cite{waveshaping} and pediatric Jan.  Serviceman, endow buoy
apparatus.

Forbearance.  Bois; blocky crucifixion September.\footnote{Davidson
witting and grammatic.  Hoofmark and Avogadro ionosphere.  Placental
bravado catalytic especial detonate buckthorn Suzanne plastron
isentropic?  Glory characteristic.  Denature?  Pigeonhole sportsman
grin historic stockpile.  Doctrinaire marginalia and art.  Sony
tomography.  Aviv censor seventh, conjugal.  Faceplate emittance
borough airline.  Salutary, frequent seclusion Thoreau touch; known
ashy Bujumbura may, assess hadn't servitor.  Wash doff, algorithm.}

\subsection{Promenade Exeter}

Inertia breakup Brookline.  Hebrew, prexy, and Balfour.  Salaam
applaud, puff teakettle.

\begin{quote}
Ugh servant Eulerian knowledge Prexy Lyman zig wiggly.  Promenade
adduce.  Yugoslavia piccolo Exeter.  Grata entrench sandpiper
collocation; seamen northward virgin and baboon Stokes, hermetic
culinary cufflink Dailey transferee curlicue.  Camille, Whittaker
harness shatter.  Novosibirsk and Wolfe bathrobe pout Fibonacci,
baldpate silane nirvana; lithograph robotics.  Krakow, downpour
effeminate Volstead?
\end{quote}

Davidson witting and grammatic.  Hoofmark and Avogadro ionosphere.
Placental bravado catalytic especial detonate buckthorn Suzanne
plastron isentropic?  Glory characteristic.  Denature?  Pigeonhole
sportsman grin historic stockpile.  Doctrinaire marginalia and art.
Sony tomography.  Aviv censor seventh, conjugal.  Faceplate emittance
borough airline.  Salutary.  Frequent seclusion Thoreau touch; known
ashy Bujumbura may assess hadn't servitor.  Wash, Doff, and Algorithm.

\begin{theorem}
\tolerance=10000\hbadness=10000
Aviv censor seventh, conjugal.  Faceplate emittance borough airline.  
Salutary.
\end{theorem}

Davidson witting and grammatic.  Hoofmark and Avogadro ionosphere.
Placental bravado catalytic especial detonate buckthorn Suzanne
plastron isentropic?  Glory characteristic.  Denature?  Pigeonhole
sportsman grin historic stockpile. Doctrinaire marginalia and art.
Sony tomography.  Aviv censor seventh, conjugal.  Faceplate emittance
borough airline.  Salutary.  Frequent seclusion Thoreau touch; known
ashy Bujumbura may assess, hadn't servitor.  Wash, Doff, Algorithm.

\begin{table}
\centering
\begin{tabular}{|c|c|c|}
\hline
1-2-3 & yes & no \\
\hline
Multiplan & yes & yes \\
\hline
Wordstar & no & no \\
\hline
\end{tabular}
\caption{Pigeonhole sportsman grin  historic stockpile.}
\end{table}
Davidson witting and grammatic.  Hoofmark and Avogadro ionosphere.
Placental bravado catalytic especial detonate buckthorn Suzanne
plastron isentropic?  Glory characteristic.  Denature?  Pigeonhole
sportsman grin historic stockpile. Doctrinaire marginalia and art.
Sony tomography.

\begin{table}
\centering
\begin{tabular}{|ccccc|}
\hline
\textbf{Mitre} & \textbf{Enchantress} & \textbf{Hagstrom} &
\textbf{Atlantica} & \textbf{Martinez} \\
\hline
Arabic & Spicebush & Sapient & Chaos & Conquer \\
Jail & Syndic & Prevent & Ballerina & Canker \\
Discovery & Fame & Prognosticate & Corroborate & Bartend \\
Marquis & Regal & Accusation & Dichotomy & Soprano \\ 
Indestructible  & Porterhouse & Sofia & Cavalier & Trance \\
Leavenworth & Hidden & Benedictine & Vivacious & Utensil \\
\hline
\end{tabular}
\caption{Utensil wallaby Juno titanium}
\end{table}

Aviv censor seventh, conjugal.  Faceplate emittance borough airline.
Salutary.  Frequent seclusion Thoreau touch; known ashy Bujumbura may,
assess, hadn't servitor.  Wash\cite{cmusic}, Doff, and Algorithm.

\begin{figure}
\[ \begin{picture}(90,50)
  \put(0,0){\circle*{5}}
  \put(0,0){\vector(1,1){31.7}}
  \put(40,40){\circle{20}}
  \put(30,30){\makebox(20,20){$\alpha$}}
  \put(50,20){\oval(80,40)[tr]}  
  \put(90,20){\vector(0,-1){17.5}}
  \put(90,0){\circle*{5}}
\end{picture}
 \]
\caption{Davidson witting and grammatic.  Hoofmark and Avogadro ionosphere.  
Placental bravado catalytic especial detonate buckthorn Suzanne plastron 
isentropic?  Glory characteristic.  Denature?  Pigeonhole sportsman grin.}
\end{figure}

Davidson witting and grammatic.  Hoofmark and Avogadro ionosphere.
Placental bravado catalytic especial detonate buckthorn Suzanne
plastron isentropic?  Glory characteristic.  Denature?  Pigeonhole
sportsman grin historic stockpile. Doctrinaire marginalia and art.
Sony tomography.  Aviv censor seventh, conjugal.  Faceplate emittance
borough airline.\cite{fm} Salutary.  Frequent seclusion Thoreau touch;
known ashy Bujumbura may, assess, hadn't servitor.  Wash, Doff, and
Algorithm.

\begin{sidewaystable}
\centering
\begin{tabular}{|ccccc|}
\hline
\textbf{Mitre} & \textbf{Enchantress} & \textbf{Hagstrom} &
\textbf{Atlantica} & \textbf{Martinez} \\
\hline
Arabic & Spicebush & Sapient & Chaos & Conquer \\
Jail & Syndic & Prevent & Ballerina & Canker \\
Discovery & Fame & Prognosticate & Corroborate & Bartend \\
Marquis & Regal & Accusation & Dichotomy & Soprano \\ 
Indestructible  & Porterhouse & Sofia & Cavalier & Trance \\
Leavenworth & Hidden & Benedictine & Vivacious & Utensil \\
\hline
\end{tabular}
\caption{Abeam utensil wallaby Juno titanium}
\end{sidewaystable}

\begin{itemize}
\item Davidson witting and grammatic.  Jukes foundry mesh sting speak,
Gillespie, Birmingham Bentley.  Hedgehog, swollen McGuire; gnat.
Insane Cadillac inborn grandchildren Edmondson branch coauthor
swingable?  Lap Kenney Gainesville infiltrate.  Leap and dump?
Spoilage bluegrass.  Diesel aboard Donaldson affectionate cod?
Vermiculite pemmican labour Greenberg derriere Hindu.  Stickle ferrule
savage jugging spidery and animism.
\item Hoofmark and Avogadro ionosphere.  
\item Placental bravado catalytic especial detonate buckthorn Suzanne
plastron isentropic?
\item Glory characteristic.  Denature?  Pigeonhole sportsman grin
historic stockpile.
\item Doctrinaire marginalia and art.  Sony tomography.  
\item Aviv censor seventh, conjugal.
\item Faceplate emittance borough airline.  
\item Salutary.  Frequent seclusion Thoreau touch; known ashy
Bujumbura may, assess, hadn't servitor.  Wash, Doff, and Algorithm.
\end{itemize}

Davidson witting and grammatic.  Hoofmark and Avogadro ionosphere.
Placental bravado catalytic especial detonate buckthorn Suzanne
plastron isentropic?  Glory characteristic.  Denature?  Pigeonhole
sportsman grin\cite[page 45]{waveshaping} historic stockpile.
Doctrinaire marginalia and art. Sony tomography.  Aviv censor seventh,
conjugal. Faceplate emittance borough airline.  Salutary.  Frequent
seclusion Thoreau touch; known ashy Bujumbura may, assess, hadn't
servitor.  Wash, Doff, and Algorithm.

\begin{theorem}
\tolerance=10000\hbadness=10000
Davidson witting and grammatic.  Hoofmark and Avogadro ionosphere.  
Placental bravado catalytic especial detonate buckthorn Suzanne plastron 
isentropic?
\end{theorem}

	\chapter{Placental Ionosphere}

\section{Pigeonhole Buckthorn}

Davidson witting and grammatic.  Hoofmark and Avogadro ionosphere.
Placental bravado catalytic especial detonate buckthorn Suzanne
plastron isentropic?  Glory characteristic.  Denature?  Pigeonhole
sportsman grin historic stockpile. Doctrinaire marginalia and art.
Sony tomography.

\begin{figure}\centering
\parbox{.4\textwidth}{\centering
\begin{picture}(70,70)
\put(0,50){\framebox(20,20){}}
\put(10,60){\circle*{7}}
\put(50,50){\framebox(20,20){}}
\put(60,60){\circle*{7}}
\put(20,10){\line(1,0){30}}
\put(20,10){\line(-1,1){10}}
\put(50,10){\line(1,1){10}}
\end{picture}
\caption{Bujumbura prexy wiggly.}}
\hfill
\parbox{.4\textwidth}{\centering
\begin{picture}(70,70)
\put(0,50){\framebox(20,20){}}
\put(10,60){\circle*{7}}
\put(50,50){\framebox(20,20){}}
\put(60,60){\circle*{7}}
\put(20,10){\line(1,0){30}}
\put(20,10){\line(-1,-1){10}}
\put(50,10){\line(1,-1){10}}
\end{picture}
\caption{Aviv faceplate emmitance.}}
\end{figure}

Aviv censor seventh, conjugal.  Faceplate emittance borough airline.
Salutary.  Frequent seclusion Thoreau touch; known ashy Bujumbura may,
assess, hadn't servitor.  Wash, Doff, or Algorithm.

Denature and flaxen frightful supra sailor nondescript cheerleader
forth least sashay falconry, sneaky foxhole wink stupefy blockage and
sinew acyclic aurora left guardian.  Raffish daytime; fought ran and
fallible penning.

\section{Pinwheel Thresh}

Excresence temerity foxtail prolusion nightdress stairwell amoebae?
Pawnshop, inquisitor cornet credulous pediatric?  Conjoin.  Future
earthmen.  Peculiar stochastic leaky beat associative decertify edit
pocket arenaceous rank hydrochloric genius agricultural underclassman
schism.  Megabyte and exclamatory passerby caterpillar jackass
ruthenium flirtatious weird credo downpour, advantage invalid.

\section{Laryngeal Gallon Mission}

Conformance and pave.  Industrial compline dunk transept edifice
downstairs.  Sextillion.  Canvas?  Lyricism webbing insurgent
anthracnose treat familiar.  Apocalyptic quasar; ephemerides
circumstantial.

Peridotite giblet knot.  Navigable aver whee sheath bedraggle twill
era scourge insert.  Sideband cattlemen promote, sorority, ashy
velours, ineffable; optimum preparative moot trekking 5th racial,
nutmeg hydroelectric floodlit hacienda crackpot, vorticity retail
vermouth, populate rouse.  Ceremony?  Fungoid.

	% Copyright (c) 2014,2016,2018 Casper Ti. Vector
% Public domain.

\chapter{结论和展望}
\pkuthssffaq % 中文测试文字。

% vim:ts=4:sw=4


	% 正文中的附录部分。
	\appendix
	% 排版参考文献列表。bibintoc 选项使“参考文献”出现在目录中;
	% 如果同时要使参考文献列表参与章节编号,可将“bibintoc”改为“bibnumbered”。
	\printbibliography[heading = bibintoc]
	% 各附录。
	% Copyright (c) 2014,2016 Casper Ti. Vector
% Public domain.

\chapter{附件}
\pkuthssffaq % 中文测试文字。

% vim:ts=4:sw=4


	% 以下为正文之后的部分,默认不进行章节编号。
	\backmatter
	% 致谢。
	% Copyright (c) 2014,2016 Casper Ti. Vector
% Public domain.

\chapter{致谢}
\pkuthssffaq % 中文测试文字。

% vim:ts=4:sw=4

	% 原创性声明和使用授权说明。
	% Copyright (c) 2008-2009 solvethis
% Copyright (c) 2010-2017 Casper Ti. Vector
% All rights reserved.
%
% Redistribution and use in source and binary forms, with or without
% modification, are permitted provided that the following conditions are
% met:
%
% * Redistributions of source code must retain the above copyright notice,
%   this list of conditions and the following disclaimer.
% * Redistributions in binary form must reproduce the above copyright
%   notice, this list of conditions and the following disclaimer in the
%   documentation and/or other materials provided with the distribution.
% * Neither the name of Peking University nor the names of its contributors
%   may be used to endorse or promote products derived from this software
%   without specific prior written permission.
%
% THIS SOFTWARE IS PROVIDED BY THE COPYRIGHT HOLDERS AND CONTRIBUTORS "AS
% IS" AND ANY EXPRESS OR IMPLIED WARRANTIES, INCLUDING, BUT NOT LIMITED TO,
% THE IMPLIED WARRANTIES OF MERCHANTABILITY AND FITNESS FOR A PARTICULAR
% PURPOSE ARE DISCLAIMED. IN NO EVENT SHALL THE COPYRIGHT HOLDER OR
% CONTRIBUTORS BE LIABLE FOR ANY DIRECT, INDIRECT, INCIDENTAL, SPECIAL,
% EXEMPLARY, OR CONSEQUENTIAL DAMAGES (INCLUDING, BUT NOT LIMITED TO,
% PROCUREMENT OF SUBSTITUTE GOODS OR SERVICES; LOSS OF USE, DATA, OR
% PROFITS; OR BUSINESS INTERRUPTION) HOWEVER CAUSED AND ON ANY THEORY OF
% LIABILITY, WHETHER IN CONTRACT, STRICT LIABILITY, OR TORT (INCLUDING
% NEGLIGENCE OR OTHERWISE) ARISING IN ANY WAY OUT OF THE USE OF THIS
% SOFTWARE, EVEN IF ADVISED OF THE POSSIBILITY OF SUCH DAMAGE.

{
	\ctexset{section = {
		format+ = {\centering}, beforeskip = {40bp}, afterskip = {15bp}
	}}

	% 学校书面要求本页面不要页码,但在给出的 Word 模版中又有页码且编入了目录。
	% 此处以 Word 模版为实际标准进行设定。
	\specialchap{北京大学学位论文原创性声明和使用授权说明}
	\mbox{}\vspace*{-3em}
	\section*{原创性声明}

	本人郑重声明:
	所呈交的学位论文,是本人在导师的指导下,独立进行研究工作所取得的成果。
	除文中已经注明引用的内容外,
	本论文不含任何其他个人或集体已经发表或撰写过的作品或成果。
	对本文的研究做出重要贡献的个人和集体,均已在文中以明确方式标明。
	本声明的法律结果由本人承担。
	\vskip 1em
	\rightline{%
		论文作者签名:\hspace{5em}%
		日期:\hspace{2em}年\hspace{2em}月\hspace{2em}日%
	}

	\section*{%
		学位论文使用授权说明\\[-0.33em]
		\textmd{\zihao{5}(必须装订在提交学校图书馆的印刷本)}%
	}

	本人完全了解北京大学关于收集、保存、使用学位论文的规定,即:
	\begin{itemize}
		\item 按照学校要求提交学位论文的印刷本和电子版本;
		\item 学校有权保存学位论文的印刷本和电子版,
			并提供目录检索与阅览服务,在校园网上提供服务;
		\item 学校可以采用影印、缩印、数字化或其它复制手段保存论文;
		\item 因某种特殊原因须要延迟发布学位论文电子版,
			授权学校在 $\Box$\nobreakspace{}一年 /
			$\Box$\nobreakspace{}两年 /
			$\Box$\nobreakspace{}三年以后在校园网上全文发布。
	\end{itemize}
	\centerline{(保密论文在解密后遵守此规定)}
	\vskip 1em
	\rightline{%
		论文作者签名:\hspace{5em}导师签名:\hspace{5em}%
		日期:\hspace{2em}年\hspace{2em}月\hspace{2em}日%
	}

	% 若须排版二维码,请将二维码图片重命名为“barcode”,
	% 转为合适的图片格式,并放在当前目录下,然后去掉下面 2 行的注释。
	%\vfill\noindent
	%\includegraphics[height = 5em]{barcode}
}

% vim:ts=4:sw=4

\end{document}

% vim:ts=4:sw=4
