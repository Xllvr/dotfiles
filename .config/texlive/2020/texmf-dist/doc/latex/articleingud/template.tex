%%
%% This is file `template.tex',
%% generated with the docstrip utility.
%%
%% The original source files were:
%%
%% articleingud.dtx  (with options: `tmple')
%% 
%% -------------------------------------------------------------------
%%                           LICENSE
%% -------------------------------------------------------------------
%% 
%% This is a generated file.
%% 
%% Copyright (C) 2012-2015 by Omar Salazar
%% osalazarm@correo.udistrital.edu.co
%% Laboratory for Automation and Computational Intelligence (LAMIC)
%% Engineering Department
%% Universidad Distrital Francisco Jose de Caldas
%% Bogota, Colombia
%% http://www.udistrital.edu.co/
%% 
%% This file may be distributed and/or modified under the
%% conditions of the LaTeX Project Public License, either
%% version 1.2 of this license or (at your option) any later
%% version. The latest version of this license is in:
%% http://www.latex-project.org/lppl.txt
%% and version 1.2 or later is part of all distributions of
%% LaTeX version 1999/12/01 or later.
%% 
%% This work has the LPPL maintenance status `maintained'.
%% 
%% The Current Maintainer of this work is Omar Salazar.
%% 
%% This work consists of the source files:
%%  - articleingud.dtx (documented LaTeX file)
%%  - articleingud.ins (installer)
%% 
\documentclass[letterpaper,12pt,twoside]{articleingud}
%%----------------------------------------------------------
%%                OPTIONS
%%----------------------------------------------------------
%% Use the following options in
%% \documentclass[<options>]{articleingud}
%%
%%  -- Point size:  10pt (default),
%%                  11pt,
%%                  12pt
%%  -- Paper size:  letterpaper (default),
%%                  a4paper,
%%                  a5paper,
%%                  b5paper,
%%                  legalpaper,
%%                  executivepaper
%%  -- Orientation: portrait (default)
%%                  landscape
%%  -- Print size:  oneside (default),
%%                  twoside
%%  -- Quality:     final (default),
%%                  draft
%%  -- Columns:     onecolumn (default),
%%                  twocolumn
%%  -- peer review: peerreview
%%  -- Equation numbering (equation numbers on
%%                         the right is the default):
%%                  leqno
%%  -- Displayed equations (centered is the default):
%%                  fleqn (equations start at the same
%%                         distance from the right side)
%%  -- Open bibliography style (closed is the default):
%%                  openbib
%%----------------------------------------------------------
%%           PREAMBLE
%%----------------------------------------------------------
%%
%%      Here you can load packages with \usepackage
%%      and make (re)definitions with
%%      \newcommand, \renewcommand, \newenvironment,
%%      \renewenvironment, etc, ...
%%
%%----------------- PACKAGES -------------------------------
\usepackage{amsmath}
\usepackage{amsfonts}
\usepackage{amssymb}
\usepackage{graphicx}
\usepackage[tight,footnotesize]{subfigure}
\usepackage{cite}
%%----------------- (RE)DEFINITIONS------------------------
%%----------------------------------------------------------
%%           DOCUMENT
%%----------------------------------------------------------
\begin{document}
%%----------------------------------------------------------
%%       PAPER'S INFORMATION
%%----------------------------------------------------------
\title
  [Short title of the paper]% (Optional)
  {Main title of the paper}% (Required)
  {Secondary title of the paper}% (Required)
  {Type of paper}% (Required) Research/Review/Case-study/Short/Opinion paper
\author
  [First Author Name\and% <-- Each author should be separated with \and
   Second Author Name\and
   Third Author Name\and
   Last Author Name]% (Optional)
  {First Author Name% <-- Do not erase this percentage symbol
   \thanks%[Correspondence email: 1st-author@email.org]
          {1}% <-- Use the same label ('a-z' or '0-9') for authors with the same affiliation
          {1st Author affiliation}\and
   Second Author Name% <-- Do not erase this percentage symbol
   \thanks%[Correspondence email: 2nd-author@email.org]
          {1}% <-- Affiliation label
          {2nd Author affiliation}\and
   Third Author Name% <-- Do not erase this percentage symbol
   \thanks%[Correspondence email: 3rd-author@email.org]
          {2}% <-- Affiliation label
          {3rd Author affiliation}\and
   Last Author Name% <-- Do not erase this percentage symbol
   \thanks[Correspondence email: last-author@email.org]
          {3}% <-- Affiliation label
          {Last Author affiliation}}% (Required)
\date
  {Received: dd-mm-yyyy.
   Modified: dd-mm-yyyy.
   Accepted: dd-mm-yyyy}% (Required)
\INGUDsetciteinfo
  {\textcopyright{} The authors;
   licensee: Revista INGENIER\'IA. ISSN 0121-750X, E-ISSN 2344-8393.
   Cite this paper as: Author, F., Author, J., Author, S.:
   The Title of the Paper. INGENIER\'IA, Vol. XX, Num. XX, 2015 pp:pp.
   doi:10.14483/udistrital.jour.reving.20XX.X.aXX}% (Required)
\INGUDsetvolume
  {1}%<-- Volume (the number only)
\INGUDsetnumber
  {1}%<-- Number (the number only)
\INGUDsetinitialpage
  {1}%<-- First page of the paper (the number only)
\maketitle
\endmaketitle
%%----------------------------------------------------------
%%                ABSTRACT
%%----------------------------------------------------------
\begin{abstract} % no more than 300 words is recommended
%%-------------------------------------------
\begin{INGUDstructured}{Context}
A paragraph roughly around 2-3 sentences long,
describing briefly the problem or question that
motivated the study, a background of previous or
related work, and the proposed approach.
\end{INGUDstructured}
%%-------------------------------------------
\begin{INGUDstructured}{Method}
A paragraph roughly around 3-4 sentences long,
describing the principles, protocols, techniques,
tools and/or materials (including data, images,
databases or primary sources such as inter-views,
documents or regulations) used to conduct or implement
reliably and rigorously the pro-posed approach.
\end{INGUDstructured}
%%-------------------------------------------
\begin{INGUDstructured}{Results}
A paragraph roughly around 3-4 sentences long,
describing the findings or the outcomes obtained as a
result of the method that was conducted.
\end{INGUDstructured}
%%-------------------------------------------
\begin{INGUDstructured}{Conclusions}
A paragraph roughly around 2-3
sentences long, emphasising to what extent
the results contribute to solve the problem
or question stated in the context section,
and what avenues for future research remain open.
This is the take-home message to the readers.
\end{INGUDstructured}
%%-------------------------------------------
\begin{INGUDstructured}{Keywords}
keyword1, keyword2, keyword3.
\end{INGUDstructured}
%%-------------------------------------------
\begin{INGUDstructured}{Acknowledgements}
To other participants, funding agencies or
sponsors (optional).
\end{INGUDstructured}
%%-------------------------------------------
\begin{INGUDstructured}{Language}
(english or spanish).
\end{INGUDstructured}
\end{abstract}
%%----------------------------------------------------------
%%                RESUMEN
%%----------------------------------------------------------
\renewcommand{\abstractname}{Resumen}
\begin{abstract} % se recomienda no exceder 300 palabras
%%-------------------------------------------
\begin{INGUDstructured}{Contexto}
Un p�rrafo de entre 2-3 frases,
que describa brevemente las preguntas o problema
que motivaron el estudio, los antecedentes o trabajos
previos relacionados, y el enfoque propuesto
(voz activa, tiempo presente).
\end{INGUDstructured}
%%-------------------------------------------
\begin{INGUDstructured}{M�todo}
Un p�rrafo de entre 3-4 frases,
que describa los principios, protocolos, t�cnicas,
herramientas y/o materiales (datos, im�genes,
bases de datos, documentos, normas, entrevistas, etc.)
utilizados para efectuar o implementar de manera
confiable y rigurosa el enfoque propuesto.
\end{INGUDstructured}
%%-------------------------------------------
\begin{INGUDstructured}{Resultados}
Un p�rrafo de entre 3-4 frases, que describa
los hallazgos o productos  resultado del m�todo
como fue aplicado.
\end{INGUDstructured}
%%-------------------------------------------
\begin{INGUDstructured}{Conclusiones}
Un p�rrafo de entre 2-3 frases, que enfatice
hasta que punto los resultados contribuyen a
resolver el problema o pregunta planteada en
el contexto inicial, y que nuevas alternativas
de trabajo futuro se desprende a partir
de este. Este es el mensaje central que los
autores desean comunicar.
\end{INGUDstructured}
%%-------------------------------------------
\begin{INGUDstructured}{Palabras clave}
t�rmino1, t�rmino2, t�rmino3.
\end{INGUDstructured}
%%-------------------------------------------
\begin{INGUDstructured}{Agradecimientos}
A otros participantes o patrocinadores
del estudio, en caso de ser necesario.
\end{INGUDstructured}
\end{abstract}
%%----------------------------------------------------------
%%                INTRODUCTION
%%----------------------------------------------------------
\section{Introduction}
Introduction of your paper.
%%
%%  Add additional sections as you need with
%%  \section, \subsection, \subsubsection,
%%  \paragraph and \subparagraph commands. You
%%  can use any other \LaTeX command
%%
\section{Final section}
Final section of your paper.
%%----------------------------------------------------------
%%                CONCLUSIONS
%%----------------------------------------------------------
\section{Conclusions}
Conclusions of your paper.
%%----------------------------------------------------------
%%                APPENDIXES
%%----------------------------------------------------------
\appendix
\section{First appendix}
Text of first appendix.
\section{Second appendix}
Text of second appendix.
%%
%%   Add additional appendixes as you need.
%%
\section{Final appendix}
Text of final appendix.
%%----------------------------------------------------------
%%                ACKNOWLEDGMENTS
%%----------------------------------------------------------
\section*{Acknowledgment}% <-- Please use \section*
Acknowledgment of your paper.
%%----------------------------------------------------------
%%                REFERENCES
%%----------------------------------------------------------
\bibliography{ref}%<-- bibliography's file (*.bib).
                  % This bibligraphy should be
                  % prepared by BibTeX.
                  % You can specify more *.bib files.
                  % ref.bib should be into the same
                  % directory of your source file
\bibliographystyle{plain}%<-- Style's file (*.bst).
%%----------------------------------------------------------
%%                BIOGRAPHIES
%%----------------------------------------------------------
\begin{biography}%
%%[{\includegraphics[width=1in]{photograph}}]%<-- author's photograph
                                            % Uncomment if
                                            % you need a photograph
{Fist Author Name}% (Required)
Biography of first author.
\end{biography}
\begin{biography}%
%%[{\includegraphics[width=1in]{photograph}}]%<-- author's photograph
                                            % Uncomment if
                                            % you need a photograph
{Second Author Name}% (Required)
Biography of second author.
\end{biography}
%%
%%   Add more biographies as you need.
%%
\begin{biography}%
%%[{\includegraphics[width=1in]{photograph}}]%<-- author's photograph
                                            % Uncomment if
                                            % you need a photograph
{Last Author Name}% (Required)
Biography of last author.
\end{biography}
\end{document}
