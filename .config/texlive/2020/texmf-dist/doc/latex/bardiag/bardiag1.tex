\documentclass[12pt]{article}
%%%%%%%%%%%%%%%%%%%%%%%%%%%%%%%%%%%%%%%%
\usepackage{barddoc}

\usepackage{xr}
\externaldocument{bardiag}

\title{The \bard package: documentation part 2
       \thanks{This file has version number \barvers and describes \bard ver.\barvers}
}

\begin{document}
\maketitle

\setcounter{section}{3}
\mysection{Horizontal mode \label{sec:4}}

There is also a possibility to draw a diagram with horizontally oriented
bars. An example of such a diagram is shown in \figref{fig4:1}

\begin{figure}[!ht]\centering
  \renewcommand{\bdorientation}{\bdhor}
  \def\onecol{red}
\def\onetopcol{blue}
\def\twocol{red}
\def\twotopcol{blue}
\def\threecol{red}
\def\threetopcol{blue}
%-------------------------------------------------------
%This is the way to redefine styles
% \newpsstyle{mytickstyle}{linewidth=1pt,linecolor=blue}
%
% Style of the foreground frame
\newpsstyle{diagframestyle}{linewidth=1pt,linecolor=black,fillcolor=white}
% Style of the background frame
\newpsstyle{diagbgframestyle}{linewidth=1pt,linecolor=black,fillcolor=yellow}

% Use 3D bars
\renewcommand{\ActiveBarPrimitive}{\barTDRect}
% Make frame 3D
\renewcommand{\frameTD}{1}

% Put ticks and levellines each 10 yunits
\renewcommand{\betweenticks}{10}

% Color of the numbers on the bar-items 
\renewcommand\numbercolor{\white\bf}
% Where to put the number. Can be \bottom,\belowtop,\overtop
\renewcommand{\placenumber}{\bottom}

\renewcommand{\tdx}{1.2} % depth of 3d
\renewcommand{\tdy}{6}

% Start the diagram
\bardiagrambegin{14}{100}{2cm}{1}{5}{0.8cm}{0.1cm}
       \baritem{1990}{32}{\onecol}
          \subtopbaritem{}{40}{\onetopcol}
        \subbaritem{2000}{20}{\twocol}
          \subtopbaritem{}{30}{\twotopcol}
        \subbaritem{2010}{13}{\threecol}
          \subtopbaritem{}{50}{\threetopcol}
       %---
       \baritem{1990}{21}{\onecol}
          \subtopbaritem{}{60}{\onetopcol}
        \subbaritem{2000}{25}{\twocol} 
          \subtopbaritem{}{64}{\twotopcol}
        \subbaritem{2010}{58}{\threecol}  
          \subtopbaritem{}{32}{\threetopcol}
       %---
       \baritem{1990}{22}{\onecol}
          \subtopbaritem{}{19}{\onetopcol}
        \subbaritem{2000}{12}{\twocol}
          \subtopbaritem{}{18}{\twotopcol}
        \subbaritem{2010}{9}{\threecol} 
          \subtopbaritem{}{12}{\threetopcol}
\drawlevellines
    % Legend
    %  Let's make the background white
    %  and gray frame-line of 0.5pt
    \diagLegendoptions{white}{gray}{0.5pt}
    % 
    \renewcommand{\legendShadowColor}{darkyellow}
    %
    \diagLegendbegin{12}{65}{3}
      \diagLegenditem{girls}{\onecol}
      \diagLegenditem{boys}{\onetopcol}
    \diagLegendend
    % End of the legend
\setlength{\captionoffset}{2cm}
\bardiagramend{\parbox{11cm}{{\centering \large Year\\[0.3cm]
   \hspace{-1cm}
   \begin{tabular}{p{3.2cm}p{4.0cm}p{3.2cm}}
      \centering Alpha's & \centering Beta's & \centering Gamma's
   \end{tabular}
  }}}
  {\large Graduated (\%)}



\caption{\label{fig4:1} Example of a diagram drawn in the horizontal mode. 
          Compare with {\figref{fig:9}}. The listing can be found in 
          section \ref{sec:4:1}}
\end{figure}

The idea is that one has to be able to ``rotate'' a diagram without
changing to much in the source. For this purpose a parameter
\verb+\bdorientation+ is introduced. It should have one of two values:
\verb+\bdvert+ (default) and \verb+\bdhor+, meaning vertical and
horizontal. The \figref{fig4:1} is drawn with the following piece of
code
\begin{lstlisting}{}
\begin{figure}[!ht]\centering
  \renewcommand{\bdorientation}{\bdhor}
  \def\onecol{red}
\def\onetopcol{blue}
\def\twocol{red}
\def\twotopcol{blue}
\def\threecol{red}
\def\threetopcol{blue}
%-------------------------------------------------------
%This is the way to redefine styles
% \newpsstyle{mytickstyle}{linewidth=1pt,linecolor=blue}
%
% Style of the foreground frame
\newpsstyle{diagframestyle}{linewidth=1pt,linecolor=black,fillcolor=white}
% Style of the background frame
\newpsstyle{diagbgframestyle}{linewidth=1pt,linecolor=black,fillcolor=yellow}

% Use 3D bars
\renewcommand{\ActiveBarPrimitive}{\barTDRect}
% Make frame 3D
\renewcommand{\frameTD}{1}

% Put ticks and levellines each 10 yunits
\renewcommand{\betweenticks}{10}

% Color of the numbers on the bar-items 
\renewcommand\numbercolor{\white\bf}
% Where to put the number. Can be \bottom,\belowtop,\overtop
\renewcommand{\placenumber}{\bottom}

\renewcommand{\tdx}{1.2} % depth of 3d
\renewcommand{\tdy}{6}

% Start the diagram
\bardiagrambegin{14}{100}{2cm}{1}{5}{0.8cm}{0.1cm}
       \baritem{1990}{32}{\onecol}
          \subtopbaritem{}{40}{\onetopcol}
        \subbaritem{2000}{20}{\twocol}
          \subtopbaritem{}{30}{\twotopcol}
        \subbaritem{2010}{13}{\threecol}
          \subtopbaritem{}{50}{\threetopcol}
       %---
       \baritem{1990}{21}{\onecol}
          \subtopbaritem{}{60}{\onetopcol}
        \subbaritem{2000}{25}{\twocol} 
          \subtopbaritem{}{64}{\twotopcol}
        \subbaritem{2010}{58}{\threecol}  
          \subtopbaritem{}{32}{\threetopcol}
       %---
       \baritem{1990}{22}{\onecol}
          \subtopbaritem{}{19}{\onetopcol}
        \subbaritem{2000}{12}{\twocol}
          \subtopbaritem{}{18}{\twotopcol}
        \subbaritem{2010}{9}{\threecol} 
          \subtopbaritem{}{12}{\threetopcol}
\drawlevellines
    % Legend
    %  Let's make the background white
    %  and gray frame-line of 0.5pt
    \diagLegendoptions{white}{gray}{0.5pt}
    % 
    \renewcommand{\legendShadowColor}{darkyellow}
    %
    \diagLegendbegin{12}{65}{3}
      \diagLegenditem{girls}{\onecol}
      \diagLegenditem{boys}{\onetopcol}
    \diagLegendend
    % End of the legend
\setlength{\captionoffset}{2cm}
\bardiagramend{\parbox{11cm}{{\centering \large Year\\[0.3cm]
   \hspace{-1cm}
   \begin{tabular}{p{3.2cm}p{4.0cm}p{3.2cm}}
      \centering Alpha's & \centering Beta's & \centering Gamma's
   \end{tabular}
  }}}
  {\large Graduated (\%)}



\caption{xxxx}
\end{figure}
\end{lstlisting}
\noindent
Line 2 sets the orientation. 

The main point here is that, while calling \texttt{bardiagrambegin}
[see section~\ref{sec:4:1} for the listing], one changes the meaning
of its parameters. If before the arguments \#1 and \#2  meant X(=horizontal) and
Y(=vertical) sizes of the framed part of the diagram, then now one has
to call Y the direction along the bars. So, after\\
\verb+\renewcommand{\bdorientation}{\bdhor}+\\
X becomes vertical and Y horizontal. It sounds crazy, I know. 
In fact, it means that you don't need to change the arguments
of \verb+\beginbardiag+ to
rotate the diagram. Only  \verb+\bdorientationin+ should be
redefined. The same holds for unit lengths (args \#6 and \#7).

The same idea may work for a legend - its argument are now ``measured''
perpendicular to the bars (\#1) and along the bars (\#2). Still, you
will probably need to shift the legend by hand to make it look pretty.

Well, basically, that's it. Of course, the bar primitives, which are
used to draw the ``real things'' should be aware of a possibility to
draw them horizontally. All primitives supplied with this package work
this way. 

To demonstrate the concept let me draw the same diagram as in 
\figref{fig:1}, but now in the horizontal mode.

\begin{lstlisting}{}
\renewcommand{\bdorientation}{\bdhor}
\begin{figure}[t]\centering
  \renewcommand{\betweenticks}{50}
\bardiagrambegin{9.5}{200}{2cm}{1}{2}{1cm}{0.05cm}
    \drawlevellines
    \baritem{1998}{120}{green}
    \baritem{1999}{123}{red}
    \baritem{2000}{147}{yellow}
    \baritem{2001}{176}{green}
    \baritem{2002}{132}{red}  
\bardiagramend{{\large Year}}{{\large Income}}


\caption{\label{fig:1hor} \figref{fig:1} redrawn in the horizontal mode.}
\end{figure}
\end{lstlisting}

The result is \figref{fig:1hor}. It looks quite okay. However the same
trick does not work 
with \figref{fig:4hor} (which is \figref{fig:1} 
plus a legend) [\figref{fig:4hor}]. You need to move the legend by hand.


\renewcommand{\bdorientation}{\bdhor}
\begin{figure}[t]\centering
  \renewcommand{\betweenticks}{50}
\bardiagrambegin{9.5}{200}{2cm}{1}{2}{1cm}{0.05cm}
    \drawlevellines
    \baritem{1998}{120}{green}
    \baritem{1999}{123}{red}
    \baritem{2000}{147}{yellow}
    \baritem{2001}{176}{green}
    \baritem{2002}{132}{red}  
\bardiagramend{{\large Year}}{{\large Income}}


\caption{\label{fig:1hor} \figref{fig:1} redrawn in the horizontal mode.}
\end{figure}

\begin{figure}[t]\centering
  \renewcommand{\betweenticks}{50}
\bardiagrambegin{9.5}{200}{2cm}{1}{2}{1cm}{0.05cm}
    \drawlevellines
    \baritem{1998}{120}{green}
    \baritem{1999}{123}{red}
    \baritem{2000}{147}{yellow}
    \baritem{2001}{176}{green}
    \baritem{2002}{132}{red}  
    \diagLegendbegin{0.1}{203}{5}
      \diagLegenditem{Good statistics}{green}
      \diagLegenditem{Moderate statistics}{yellow}
      \diagLegenditem{Bad statistics}{red}
    \diagLegendend
\bardiagramend{{\large Year}}{{\large Income}}


\caption{\label{fig:4hor} \figref{fig:4} redrawn in the horizontal
  mode. As you see, the legend messes it up. Sorry, I cannot do
  anything about it.}
\end{figure}

\clearpage
\subsection{\label{sec:4:1}Listing of the example \figref{fig4:1}}
\lstinputlisting{figs/examp6.tex}



\end{document}