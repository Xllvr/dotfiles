%% GABARIT POUR THÈSE PAR ARTICLES
%%
%% Consulter la documentation de la classe ulthese pour une
%% description détaillée de la classe, de ce gabarit et des options
%% disponibles.
%%
%% [Ne pas hésiter à supprimer les commentaires après les avoir lus.]
%%
%% Déclaration de la classe avec le type de grade
%%   [l'un de LLD, DMus, DPsy, DThP, PhD]
%% et les langues les plus courantes. Le français sera la langue par
%% défaut du document. L'option 'bibsection' permet de créer des
%% bibliographies par chapitre présentées sous forme de section
%% numérotée.
\documentclass[PhD,bibsection,english,french]{ulthese}
  %% Encodage utilisé pour les caractères accentués dans les fichiers
  %% source du document. Les gabarits sont encodés en UTF-8. Inutile
  %% avec XeLaTeX, qui gère Unicode nativement.
  \ifxetex\else \usepackage[utf8]{inputenc} \fi

  %% Charger ici les autres paquetages nécessaires pour le document.
  %% Quelques exemples; décommenter au besoin.
  %\usepackage{amsmath}       % recommandé pour les mathématiques
  %\usepackage{icomma}        % gestion de la virgule dans les nombres

  %% Utilisation d'une autre police de caractères pour le document.
  %% - Sous LaTeX
  %\usepackage{mathpazo}      % texte et mathématiques en Palatino
  %\usepackage{mathptmx}      % texte et mathématiques en Times
  %% - Sous XeLaTeX
  %\setmainfont{TeX Gyre Pagella}      % texte en Pagella (Palatino)
  %\setmathfont{TeX Gyre Pagella Math} % mathématiques en Pagella (Palatino)
  %\setmainfont{TeX Gyre Termes}       % texte en Termes (Times)
  %\setmathfont{TeX Gyre Termes Math}  % mathématiques en Termes (Times)

  %% Options de mise en forme du mode français de babel. Consulter la
  %% documentation du paquetage babel pour les options disponibles.
  %% Désactiver (effacer ou mettre en commentaire) si l'option
  %% 'nobabel' est spécifiée au chargement de la classe.
  \frenchbsetup{%
    StandardItemizeEnv=true,       % format standard des listes
    ThinSpaceInFrenchNumbers=true, % espace fine dans les nombres
    og=«, fg=»                     % caractères « et » sont les guillemets
  }

  %% Suppression du numéro de section de la bibliographie. Utilisation
  %% de \extrasfrench parce que c'est la dernière langue déclarée dans
  %% \documentclass, ci-dessus.
  %\addto\extrasfrench{%
  %  \renewcommand{\bibsection}{\section*{\bibname}\prebibhook}}

  %% Composition de la page frontispice. Remplacer les éléments entre < >.
  %% Supprimer les caractères < >. Couper un long titre ou un long
  %% sous-titre manuellement avec \\.
  \titre{<Titre principal>}
  % \titre{Ceci est un exemple de long titre \\
  %   avec saut de ligne manuel}
  % \soustitre{Sous-titre le cas échéant}
  % \soustitre{Ceci est un exemple de long sous-titre \\
  %   avec saut de ligne manuel}
  \auteur{<Prénom Nom>}
  \direction{<Prénom Nom>, <directeur ou directrice> de recherche}
  % \codirection{<Prénom Nom>, <codirecteur ou codirectrice> de recherche}
  % \codirection{<Prénom Nom>, <codirecteur ou codirectrice> de recherche \\
  %              <Prénom Nom>, <codirecteur ou codirectrice> de recherche}

  %% Les commandes ci-dessous servent uniquement pour la création
  %% d'une page de titre (interdite lors du dépôt à la FESP).
  % \annee{<20xx>}

\begin{document}

\frontmatter                    % pages liminaires

\frontispice                    % production de la page frontispice

% !Mode:: "TeX:UTF-8" 

\begin{resume}
XXXX~年~XX~月~XX~日出生于~XXXX。

XXXX~年~XX~月考入~XX~大学~XX~院(系)XX~专业,XXXX~年~XX~月本科毕业并获得~XX~学学士学位。

XXXX~年~XX~月------XXXX~年~XX~月在~XX~大学~XX~院(系)XX~学科学习并获得~XX~学硕士学位。

XXXX~年~XX~月------XXXX~年~XX~月在~XX~大学~XX~院(系)XX~学科学习并获得~XX~学博士学位。

获奖情况:如获三好学生、优秀团干部、X~奖学金等(不含科研学术获奖)。

工作经历:

\textbf{( 除全日制硕士生以外,其余学生均应增列此项。个人简历一般应包含教育经历和工作经历。)}
\end{resume}
                % résumé français
This document gives a quick, relatively minimal example of the use of
\texttt{uafthesis.cls}, while trying to show its features.

This section is contained in \texttt{abstract.tex}.
              % résumé anglais
\cleardoublepage

\tableofcontents                % production de la TdM
\cleardoublepage

\listoftables                   % production de la liste des tableaux
\cleardoublepage

\listoffigures                  % production de la liste des figures
\cleardoublepage

\dedicace{<Dédicace si désiré>}
\cleardoublepage

\epigraphe{<Texte de l'épigraphe>}{<Source ou auteur>}
\cleardoublepage

\chapter*{Remerciements}        % ne pas numéroter
\label{chap:remerciements}      % étiquette pour renvois
\phantomsection\addcontentsline{toc}{chapter}{\nameref{chap:remerciements}} % inclure dans TdM

<Texte des remerciements en prose.>
         % remerciements
\chapter*{Avant-propos}         % ne pas numéroter
\label{chap:avantpropos}        % étiquette pour renvois
\phantomsection\addcontentsline{toc}{chapter}{\nameref{chap:avantpropos}} % inclure dans TdM

<Texte de l'avant-propos. Obligatoire dans une thèse ou un mémoire par
articles.>
           % avant-propos

\mainmatter                     % corps du document

\documentclass[12pt]{report}
\usepackage[utf8]{inputenc}
\usepackage[brazilian,brazil]{babel}
\usepackage{fancyhdr,setspace,float,graphicx,lscape,array,longtable,colortbl,amsmath,amssymb,booktabs,multirow,hyperref,pdfpages,tocloft,titlesec,lipsum,natbib}
\usepackage[sectionbib]{chapterbib}
\begin{document}
%***
\clearpage
\linenumbers % numeração de linhas
\modulolinenumbers[3] % numeração de linhas
%***
% Texto
%***
\chapter{Nam dui ligula}
%***
\lipsum[2-2] Nam dui ligula, fringilla a, euismod sodales, sollicitudin vel, wisi. Morbi
auctor lorem non justo \citep{lamport1986latex}.
\section{Euismod sodales}
\lipsum[2-3]
%*** APAGAR O EXEMPLO ACIMA







%*** REFERÊNCIAS
\bibliography{referencias.bib}
\bibliographystyle{apalike}
\end{document}
          % introduction
\chapter{<Titre du chapitre ou de l'article>}     % numéroté
\label{chap:}                   % étiquette pour renvois (à compléter!)

\section{Résumé}

\begin{otherlanguage*}{french}
  <Résumé de l'article en français. Obligatoire.>
\end{otherlanguage*}

\section{Abstract}

\begin{otherlanguage*}{english}
  <English abstract of the paper. Optional, but recommended.>
\end{otherlanguage*}

<Texte du chapitre ou de l'article.>

\bibliographystyle{}              % style de la bibliographie
\bibliography{}                   % production de la bibliographie
    % chapitre 1
\chapter{<Titre du chapitre ou de l'article>}     % numéroté
\label{chap:}                   % étiquette pour renvois (à compléter!)

\section{Résumé}

\begin{otherlanguage*}{french}
  <Résumé de l'article en français. Obligatoire.>
\end{otherlanguage*}

\section{Abstract}

\begin{otherlanguage*}{english}
  <English abstract of the paper. Optional, but recommended.>
\end{otherlanguage*}

<Texte du chapitre ou de l'article.>

\bibliographystyle{}              % style de la bibliographie
\bibliography{}                   % production de la bibliographie
    % chapitre 2, etc.
\chapter{The conclusion}

            % conclusion

\appendix                       % annexes le cas échéant

\chapter{<Titre de l'annexe>}     % numérotée
\label{chap:}                   % étiquette pour renvois (à compléter!)

<Texte de l'annexe.>
                % annexe A

\end{document}
