% This is a template that may be used for the articles submitted to the
% North-Western European Journal of Mathematics.
%
% The language of the article is by default English. Should it be French, German
% or Dutch instead, it would be specified as \documentclass' option.
\documentclass[
% french  % If the language of the article will be French
% german  % If the language of the article will be German
% dutch   % If the language of the article will be Dutch
]{nwejmart}
%
% Specify your own bibtex file, preferrably at `biblatex' format (don't forget
% the `.bib' extension below) in the argument of the \addbibresource command.
\addbibresource{}
%
% Should acronyms be used in the article, define them thanks to \newacronym
% command from `glossaries' package as follows:
%  - 1st argument: ⟨label⟩      of the acronym (also called key),
%  - 2nd argument: ⟨short form⟩ of the acronym (lowercase!),
%  - 3rd argument: ⟨long form⟩  of the acronym,
% and use them with \gls{⟨label⟩} (or, if needed, with \acrshort{⟨label⟩}).
% See `glossaries' package's documentation for more details.
% \newacronym{}{}{}
%
\begin{document}
%
% Title of the article. A short form (that will be displayed in the headers and
% in the volume's TOC) may be specified as optional argument.
\title{}
%
% Subtitle of the article, if any. A short form may be specified as optional
% argument.
% \subtitle{}
%
% Author(s) of the article:
% - one \author command per author,
% - mandatory argument entered as `⟨Last Name⟩, ⟨First Name⟩'.
% Use the key-value `affiliation={⟨affiliation⟩}' optional argument for each
% affiliation of the author. An affiliation can be tagged
% (`affiliation=[⟨tag⟩]{⟨affiliation⟩}') and reused later
% (affiliationtagged={⟨tag⟩}).
\author[affiliation={}]{, }
% \author[affiliation={}]{, }
%
% The abstract is entered as usually.
\begin{abstract}
  ...
\end{abstract}
%
% The keywords are entered thanks to \keywords command, as a comma separated list.
\keywords{}
%
% The Mathematical Subject Classification (MSC) are entered thanks to \msc
% command, as a comma separated list.
\msc{}
%
% The title is made as usually. Be aware that author(s) will be displayed or
% updated only if a `biber' run (cf. `nwejm''s documentation for more details).
\maketitle
%
% Acknowledgments, if any, are entered thanks to \acknowledgments command (and
% will be displayed just before the bibliography, thanks to the
% \printbibliography command).
% \acknowledgments{}
%
% Here comes the article's content.
...
%
% The \printbibliography command (from `biblatex' package) displays the list of
% references (preceded by the acknowledgments, if any)
\printbibliography
%
\end{document}
