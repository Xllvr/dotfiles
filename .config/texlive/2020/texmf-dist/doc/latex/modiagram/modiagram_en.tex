% arara: pdflatex
% arara: pdflatex
% --------------------------------------------------------------------------
% the MODIAGRAM package
%
%   easy creation of molecular orbital diagrams
%
% --------------------------------------------------------------------------
% Clemens Niederberger
% Web:    https://www.bitbucket.org/cgnieder/modiagram
% E-Mail: contact@mychemistry.eu
% --------------------------------------------------------------------------
% If you have any ideas, questions, suggestions or bugs to report, please
% feel free to contact me.
% --------------------------------------------------------------------------
% Copyright 2011--2019 Clemens Niederberger
%
% This work may be distributed and/or modified under the
% conditions of the LaTeX Project Public License, either version 1.3
% of this license or (at your option) any later version.
% The latest version of this license is in
%   http://www.latex-project.org/lppl.txt
% and version 1.3 or later is part of all distributions of LaTeX
% version 2005/12/01 or later.
%
% This work has the LPPL maintenance status `maintained'.
%
% The Current Maintainer of this work is Clemens Niederberger.
%
% This work consists of the files modiagram.sty, modiagram_en.tex,
% README and the derived file modiagram_en.pdf.
% --------------------------------------------------------------------------
\documentclass[load-preamble+]{cnltx-doc}
\usepackage[utf8]{inputenc}
\usepackage{modiagram}

\setcnltx{
  package = modiagram ,
  authors = Clemens Niederberger ,
  email   = contact@mychemistry.eu ,
  url     = https://bitbucket.org/cgnieder/modiagram/ ,
  add-cmds = {
    AO,
    atom,
    connect,    
    EnergyAxis,    
    molecule,
    setmodiagram
  } ,
  add-envs = { modiagram } ,
  add-silent-cmds = {
    ch,
    draw,
    chlewis,
    node,
    textcolor,
    chemsigma,
    chemSigma
  } ,
  abstract = {%
    \begin{modiagram}[style=fancy,distance=7cm,AO-width=15pt,labels,names]
      \atom[N]{left}{ 2p = {0;up,up,up} }
      \atom[O]{right}{ 2p = {2;pair,up,up} }
      \molecule[NO]{
        2pMO = {1.8,.4;pair,pair,pair,up},
        color = { 2piy*=red }
      }
     \end{modiagram}
  }
}

\newpackagename\modiag{modiagram}

\defbibheading{bibliography}{\addsec{References}}

\usepackage{booktabs}

\usepackage{acro}
\DeclareAcronym{AO}{
  short = AO ,
  long  = atomic orbital
}
\DeclareAcronym{MO}{
  short = MO ,
  long  = molecular orbital
}

\makeatletter
\def\libertine@figurestyle{LF}
\RequirePackage{amsmath}
\undef\lvert \undef\lVert
\undef\rvert \undef\rVert
\RequirePackage[libertine]{newtxmath}
\def\libertine@figurestyle{OsF}
\makeatother

\usepackage{chemmacros,chemformula}
\chemsetup{
  greek = newtx,
  modules = units
}
\setchemformula{format=\libertineLF}

\NewDocumentCommand \AOinline { o m }
  {%
    \begingroup
      \IfNoValueTF{#1}
        {\setmodiagram{ style=square,AO-width=8pt }}%
        {\setmodiagram{ style=square,AO-width=8pt , #1 }}%
      \begin{modiagram}
        \AO{s}{0;#2}
      \end{modiagram}%
    \endgroup
  }

\newcommand*\TikZ{Ti\textit{k}Z}

\begin{document}

\section{Licence, Requirements}
\license

\modiag\ uses \bnd{l3kernel}~\cite{bnd:l3kernel} and
\bnd{l3packages}~\cite{bnd:l3packages}. It also uses \TikZ~\cite{pkg:pgf} and
the package \pkg{chemgreek}~\cite{pkg:chemgreek} bundle.  Additionally the
\TikZ\ libraries \code{calc} and \code{arrows} are loaded.  Knowledge of
\TikZ\ is helpful.

\section{Motivation}
This package has been written as a reaction to a question on
\url{http://tex.stackexchange.com/}. To be more precise: as a reaction to the
question
``\href{http://tex.stackexchange.com/questions/13863/molecular-orbital-diagrams-in-latex}%
{Molecular orbital diagrams in LaTeX}.'' There it says
\begin{cnltxquote}
  I'm wondering if anyone has seen a package for drawing (qualitative)
  molecular orbital splitting diagrams in \LaTeX? Or if there exist any
  packages that can be easily re-purposed to this task?

  Otherwise, I think I'll have a go at it in \TikZ.
\end{cnltxquote}
The problem was solved using \TikZ, since no package existed for that purpose.
For one thing \modiag\ is intended to fill this gap. I also found it very
tedious, to make all this copying and pasting when I needed a second, third,
\ldots\ diagram. \modiag\ took care of that.


\section{Main Commands}
All molecular orbital (MO) diagrams are created using the environment
\env{modiagram}{}.

\subsection{The \cs*{atom} Command}
\begin{commands}
  \command{atom}[\oarg{name}\Marg{\meta{left}|\meta{right}}\marg{AO-spec}]
    Place an \acs{AO} in the diagram. \meta{name} is caption of the atom,
    \meta{left} and \meta{right} determine the placement in the diagram,
    \meta{AO-spec} is the specification of the \ac{AO}.
\end{commands}

Let's take a look at an example:
\begin{example}[side-by-side]
  \begin{modiagram}
    \atom{right}{
      1s = {  0; pair} ,
      2s = {  1; pair} ,
      2p = {1.5; up, down }
    }
  \end{modiagram}
\end{example}

As you can see, the argument \meta{AO-spec} is essential to create the actual
orbitals and the electrons within. You can use these key/value pairs to specify
what you need:
\begin{options}
  \keychoice{1s}{\{\meta{rel-energy}; \meta{el-spec}\}}
     Energy level and electron specifications for the 1s orbital.
  \keychoice{2s}{\{\meta{rel-energy}; \meta{el-spec}\}}
    Energy level and electron specifications for the 2s orbital.
  \keychoice{2p}{\{\meta{rel-energy}; \meta{x el-spec}{,} \meta{y el-spec}{,}
    \meta{z el-spec}\}}
    Energy level and electron specifications for the 2p orbitals.
\end{options}
\meta{el-spec} can have the values \code{pair}, \code{up} and \code{down} or
can be left empty. \meta{rel-energy} actually is the $y$ coordinate and shifts
the \ac{AO} vertically by \meta{rel-energy} \si{\centi\metre}.

The argument \meta{left}/\meta{right} is important, when p~orbitals are used.
For instance compare the following example to the one before:
\begin{example}[side-by-side]
  \begin{modiagram}
    \atom{left}{
      1s = {  0; pair} ,
      2s = {  1; pair} ,
      2p = {1.5; up, down }
    }
  \end{modiagram}
\end{example}

When both variants are used one can also see, that the right atom is shifted to
the right (hence the naming). The right atom is shifted by \SI{4}{\centi\metre}
per default and can be adjusted individually, see page~\pageref{option:distance}.
\begin{example}
  \begin{modiagram}
    \atom{left}{
      1s = {  0; pair} ,
      2s = {  1; pair} ,
      2p = {1.5; up, down }
    }
    \atom{right}{
      1s = {  0; pair} ,
      2s = {  1; pair} ,
      2p = {1.5; up, down }
    }
  \end{modiagram}
\end{example}
With the command \cs{molecule} (section~\ref{ssec:molecule}) the reason for the
shift becomes clear.

Any of the arguments for the \ac{AO} can be left empty or be omitted.
\begin{example}[side-by-side]
  Without argument: default height, full:\par
  \begin{modiagram}
    \atom{left}{1s, 2s, 2p}
  \end{modiagram}
\end{example}

\begin{example}[side-by-side]
  empty argument: default height, empty:\par
  \begin{modiagram}
    \atom{left}{1s=, 2s=, 2p=}
  \end{modiagram}
\end{example}

\begin{example}[side-by-side]
  using some values:\par
  \begin{modiagram}
    \atom{left}{1s, 2s=1, 2p={;,up} }
  \end{modiagram}
\end{example}

\subsection{The \cs*{molecule} Command}\label{ssec:molecule}
\begin{commands}
  \command{molecule}[\oarg{name}\marg{MO-spec}]
    Place a \acs{MO} in the diagram. \meta{name} is caption of the molecule,
    \meta{MO-spec} is the specification of the \ac{MO}.
\end{commands}

An example first:
\begin{example}[side-by-side]
  \begin{modiagram}
    \atom{left} { 1s = {  0; up } }
    \atom{right}{ 1s = {  0; up } }
    \molecule { 1sMO = {.75; pair } }
  \end{modiagram}
\end{example}
The command \cs{molecule} connects the \acp{AO} with the bonding and
anti-bondung \acp{MO}.  \cs{molecule} can only be used \emph{after} one has
set \emph{both} atoms since the orbitals that are to be connected must be
known.

The argument \meta{MO-spec} accepts a comma separated list of key/value pairs:
\begin{options}
  \keychoice{1sMO}{\{\meta{energy gain}/\meta{energy loss}; \meta{s
      el-spec}{,} \meta{s* el-spec}\}}
    connects the \acp{AO} specified by \option{1s}.
  \keychoice{2sMO}{\{\meta{energy gain}/\meta{energy loss}; \meta{s
      el-spec}{,} \meta{s* el-spec}\}}
    connects the \acp{AO} specified by \option{2s}
  \keychoice{2pMO}{\{\meta{s energy gain}/\meta{s energy loss}{,} \meta{p
      energy gain}/\meta{p energy loss}; \meta{s el-spec}{,} \meta{py
      el-spec}{,} \meta{pz el- spec}{,} \meta{py* el-spec}{,} \meta{pz*
      el-spec}{,} \meta{s* el-spec}\}}
    connects the \acp{AO} specified by \option{2p}.
\end{options}

Obviously the regarding \acp{AO} must have been set in order to connect them.
This for example won't work:
\begin{sourcecode}
  \begin{modiagram} 
    \atom{left} { 1s = 0 }
    \atom{right}{ 1s = 0 }
    \molecule { 2sMO = .75 }
  \end{modiagram}
\end{sourcecode}
The value used in \meta{energy gain} determines how many \si{\centi\metre} the
bonding \ac{MO} lies below the lower \ac{AO} or how many \si{\centi\metre} the
anti-bondung \ac{MO} lies above the higher \ac{AO}.

\begin{example}[side-by-side]
  same level:\par
  \begin{modiagram}
    \atom{left} { 1s = {  0; up } }
    \atom{right}{ 1s = {  0; up } }
    \molecule { 1sMO = {.75; pair } }
  \end{modiagram}

  different levels:\par
  \begin{modiagram}
    \atom{left} { 1s = {  0; up } }
    \atom{right}{ 1s = {  1; up } }
    \molecule { 1sMO = {.25; pair } }
  \end{modiagram}
\end{example}

If you specify \meta{energy loss} you can create non-symmetrical splittings.
Then, the first value (\meta{energy gain}) is used for the bonding \ac{MO} and
the second value (\meta{energy loss}) is used for the anti-bonding \ac{MO}.
\begin{example}[side-by-side]
  \begin{modiagram}
    \atom{left} { 1s = {  0; up } }
    \atom{right}{ 1s = {  0; up } }
    \molecule { 1sMO = {.75/.25; pair } }
  \end{modiagram}

  \begin{modiagram}
    \atom{left} { 1s = {  0; up } }
    \atom{right}{ 1s = {  1; up } }
    \molecule { 1sMO = {.25/.75; pair } }
  \end{modiagram}
\end{example}

Please be aware, that you have to specify \emph{two} such values or pairs with
\option{2pMO}: the splitting of the \chemsigma\ orbitals and the splitting of
the \chempi\ orbitals.
\begin{example}
  \begin{modiagram}
    \atom{left} { 2p = { 0; up, up } }
    \atom{right}{ 2p = { 1; up, up } }
    \molecule { 2pMO = { 1.5, .75; pair, up, up } } 
  \end{modiagram}
\end{example}

The complete \ac{MO} diagram for triplett dioxygen now could look something
like that:
\begin{example}
  \begin{modiagram}
    \atom{left}{
      1s, 2s, 2p = {;pair,up,up}
    }
    \atom{right}{
      1s, 2s, 2p = {;pair,up,up}
    }
    \molecule{
      1sMO, 2sMO, 2pMO = {;pair,pair,pair,up,up}
    }
  \end{modiagram}
\end{example}

\subsection{The Naming Scheme}\label{Namensgebung}
Since one wants to be able to put labels to the orbitals and since they are
nodes in a \env*{tikzpicture}, the internal naming scheme is important.  It
closely follows the function:
\begin{center}
  \begin{modiagram}[
    distance = 6cm,
    AO-width = 20pt,
    labels-fs = \ttfamily\footnotesize,
    labels-style = {yshift=10pt}
  ]
   \atom{left}{
     1s = 0 ,
     2s = 2 ,
     2p = 5 ,
     label = {
       1sleft  = {1sleft} ,
       2sleft  = {2sleft} ,
       2pxleft = {2pxleft} ,
       2pyleft = {2pyleft} ,
       2pzleft = {2pzleft}
     }
   }
   \atom{right}{
     1s = 0 ,
     2s = 2 ,
     2p = 5 ,
     label = {
       1sright  = {1sright} ,
       2sright  = {2sright} ,
       2pxright = {2pxright} ,
       2pyright = {2pyright} ,
       2pzright = {2pzright}
     }
   }
   \molecule{
     1sMO = .5 ,
     2sMO = .5 ,
     2pMO = {1.5,.5} ,
     label = {
       1sigma   = {1sigma} ,
       1sigma*  = {1sigma*} ,
       2sigma   = {2sigma} ,
       2sigma*  = {2sigma*} ,
       2psigma  = {2psigma} ,
       2psigma* = {2psigma*} ,
       2piy     = {2piy} ,
       2piy*    = {2piy*} ,
       2piz     = {2piz} ,
       2piz*    = {2piz*}
     }
   }
  \end{modiagram}
\end{center}

With these names it is possible to reference the orbitals with the known \TikZ
commands:
\begin{example}[side-by-side]
  \begin{modiagram}
    \atom{left} { 1s = 0 }
    \atom{right}{ 1s = 0 }
    \molecule { 1sMO = .75 }
    \draw[<->,red,semithick]
      (1sigma.center) -- (1sigma*.center) ;
    \draw[red]
      (1sigma*) ++ (2cm,.5cm) node {splitting} ;
  \end{modiagram}
\end{example}

\begin{example}
  \begin{modiagram}
    \atom{left} { 1s = 0 }
    \atom{right}{ 1s = 0 }
    \molecule { 1sMO = .75 }
    \draw[draw=blue,very thick,fill=blue!40,opacity=.5]
      (1sigma*) circle (8pt);
    \draw[<-,shorten <=8pt,shorten >=15pt,blue]
      (1sigma*) --++(2,1) node {anti-bonding MO};
  \end{modiagram}
\end{example}

\subsection{Placing AOs and MOs Arbitrarily}\label{ssec:AO_MO_irgendwo}
The standard orbitals are not always sufficient in order to draw a correct
\ac{MO} diagram. For example in the \ac{MO} diagram of \ch{XeF2} one would
need the part that illustrates the interaction between the bonding and
anti-bonding combination of two p orbitals of Flourine with one p orbital of
Xenon:

\begin{center}
  \begin{modiagram}[names]
    \atom[\chlewis{0.}{F}\hspace*{5mm}\chlewis{180.}{F}]{left}{
      1s=.2;up,up-el-pos={1sleft=.5}
    }
    \atom[Xe]{right}{1s=1.25;pair}
    \molecule[\ch{XeF2}]{1sMO={1/.25;pair}}
    \AO(1cm){s}{0;up}
    \AO(3cm){s}{0;pair}
    \connect{ AO1 & AO2 }
    \node[right,xshift=4mm] at (1sigma) {\footnotesize bonding};
    \node[above] at (AO2.90) {\footnotesize non-bonding};
    \node[above] at (1sigma*.90) {\footnotesize anti-bonding};
  \end{modiagram}
\end{center}

To create diagrams like this there is the following command, which draws a single
\ac{AO}:
\begin{commands}
 \command{AO}[\oarg{name}\darg{xshift}\marg{type}\oarg{options}\Marg{\meta{energy};
   \meta{el-spec}}]
   Place an \acs{AO} in the diagram. \meta{<name>} (optional) is the name of
   the node; if not specified, \code{AO\#} is used where \code{\#} is a
   consecutive number.  \meta{xshift} is the vertical position of the orbital(s),
   a \TeX\ dimension.  \meta{type} can be \code{s} or \code{p}.
   \meta{options} is a list of key/value pairs with which the \ac{AO} can be
   customized, see section~\ref{ssec:AO_anpassen}.  \meta{AO-spec} is the
   specification of the \ac{AO}.
\end{commands}

Depending on the \meta{type} one s or three p orbitals are drawn.
\begin{example}[side-by-side]
  \begin{modiagram}
    \AO{s}{0;}
    \AO(-20pt){p}{1;pair,up,down}
  \end{modiagram}
\end{example}

If one wants to place such an \ac{AO} at the position of an atom, one has to
know their \meta{xshift}.  They have predefined values (also see
section~\ref{orbital-positionen}):\label{xshift}
\begin{itemize}
  \item atom left: \SI{1}{\centi\metre}
  \item molecule: \SI{3}{\centi\metre}
  \item atom right: \SI{5}{\centi\metre}
\end{itemize}

\begin{example}[side-by-side]
  \begin{modiagram}
    \atom{left} {1s=0}
    \atom{right}{1s=0}
    \molecule {1sMO=1}
    \AO(1cm){s}{2}
    \AO(3cm){s}{2}
    \AO(5cm){s}{2}
  \end{modiagram}
\end{example}

Within the p orbitals there is an additional shift by \SI{20}{pt} per orbital.
This is equivalent to a double shift by the length \code{AO-width} (see
section~\ref{option:AO-width}):
\begin{example}
  \begin{modiagram}
    \atom{left} {2p=0}
    \atom{right}{2p=0}
    % above the left atom:
    \AO(1cm)     {s}{ .5}
    \AO(1cm-20pt){s}{  1;up}
    \AO(1cm-40pt){s}{1,5;down}
    % above the right atom:
    \AO(1cm)     {s}{ .5}
    \AO(5cm+20pt){s}{  1;up}
    \AO(5cm+40pt){s}{1.5;down}
  \end{modiagram}
\end{example}

The \acp{AO} created with \cs{AO} also can be connected.  For this you can use
the \TikZ\ command \cs*{draw}, of course.  You can use the predefined node
names\ldots
\begin{example}
  \begin{modiagram}
    \AO{s}{0} \AO(2cm){s}{1}
    \AO{s}{2} \AO(2cm){s}{1.5}
    \draw[red] (AO1.0) -- (AO2.180) (AO3.0) -- (AO4.180);
  \end{modiagram}
\end{example}
\ldots\ or use own node names
\begin{example}
  \begin{modiagram}
    \AO[a]{s}{0} \AO[b](2cm){s}{1}
    \AO[c]{s}{2} \AO[d](2cm){s}{1.5}
    \draw[red] (a.0) -- (b.180) (c.0) -- (d.180);
  \end{modiagram}
\end{example}

The predefined names are \code{AO1}, \code{AO2} \etc for the type \code{s} and
\code{AO1x}, \code{AO1y}, \code{AO1z}, \code{AO2x} \etc\ for the type
\code{p}.  Nodes of the type \code{p} get an \code{x}, \code{y} or \code{z} if
you specify your own name, too.
\begin{example}
  \begin{modiagram}
    \AO{p}{0}
    \draw[<-,shorten >=5pt] (AO1y.-90) -- ++ (.5,-1) node {y};
   \end{modiagram}
  and
  \begin{modiagram}
    \AO[A]{p}{0}
    \draw[<-,shorten >=5pt] (Ay.-90) -- ++ (.5,-1) node {y};
  \end{modiagram}
\end{example}

However, if you want the lines to be drawn in the same style as the ones
created by \cs{molecule}\footnote{which can be customized, see
  page~\pageref{option:lines}}, you should use the command \cs{connect}.
\begin{commands}
  \command{connect}[\marg{AO-connect}]
    Connects the specified \acp{AO}.  \meta{AO-connect} is comma separated
    list of node name pairs connected with \code{\&}.
\end{commands}
This command expects a comma separated list of node name pairs that are to be
connected. The names have to be connected with a \code{\&}:
\begin{example}[side-by-side]
  \begin{modiagram}
    \AO{s}{0;} \AO(2cm){s}{1;}
    \AO{s}{2;} \AO(2cm){s}{1.5;}
    \connect{ AO1 & AO2, AO3 & AO4 }
  \end{modiagram}
\end{example}

Some things still need to be said: \cs{connect} adds the anchor \code{east} to
the first name and the anchor \code{west} to the second one.  This means a
connection only makes sense from the left to the right.  However, you can add
own anchors using the usual \TikZ\ way:
\begin{example}
  \begin{tikzpicture}
    \draw (0,0) node (a) {a} ++ (1,0) node (b) {b}
          ++ (0,1) node (c) {c} ++ (-1,0) node (d) {d} ;
    \connect{ a.90 & d.-90, c.180 & d.0 }
  \end{tikzpicture}
\end{example}

\subsection{The Positioning Scheme}\label{orbital-positionen}
The figure below shows the values of the $x$ coordinates of the orbitals
depending on the values of \meta{distance} (\meta{dist}) and \meta{AO-width}
(\meta{AO}). In sections~\ref{option:distance} and \ref{option:AO-width} these
lengths and how they can be changed are discussed.
\begin{center}
  \begin{modiagram}[
    AO-width = 22pt ,
    labels-fs = \ttfamily\tiny ,
    labels-style = {text width=40pt,align=center,yshift=11pt}]
   \atom{left}{
     1s = 0 ,
     2s = 2 ,
     2p = 5.5 ,
     label = {
       1sleft  = {1cm} ,
       2sleft  = {1cm} ,
       2pxleft = {1cm - 4*\meta{AO}} ,
       2pyleft = {1cm - 2*\meta{AO}} ,
       2pzleft = {1cm}
     }}
   \atom{right}{
     1s = 0 ,
     2s = 2 ,
     2p = 5.5 ,
     label = {
       1sright  = {1cm + \meta{dist}} ,
       2sright  = {1cm + \meta{dist}} ,
       2pxright = {1cm+ \meta{dist}} ,
       2pyright = {1cm + \meta{dist} + 2*\meta{AO}} ,
       2pzright = {1cm + \meta{dist} + 4*\meta{AO}}
     }}
   \molecule{
     1sMO = .5 ,
     2sMO = .5 ,
     2pMO = {2,.75} ,
     label = {
       1sigma   = {.5*\meta{dist} + 1cm} ,
       1sigma*  = {.5*\meta{dist} + 1cm} ,
       2sigma   = {.5*\meta{dist} + 1cm} ,
       2sigma*  = {.5*\meta{dist} + 1cm} ,
       2psigma  = {.5*\meta{dist} + 1cm} ,
       2psigma* = {.5*\meta{dist} + 1cm} ,
       2piy     = {.5*\meta{dist} + 1cm - \meta{AO}} ,
       2piy*    = {.5*\meta{dist} + 1cm - \meta{AO}} ,
       2piz     = {.5*\meta{dist} + 1cm + \meta{AO}} ,
       2piz*    = {.5*\meta{dist} + 1cm + \meta{AO}}
     }
   }
  \end{modiagram}
\end{center}

\subsection{Default Values}
If you leave the arguments (or better: values) for the specification of the
\ac{AO} or \ac{MO} empty or omit them, default values are used. The table below
shows you, which ones.
\begin{center}
  \small
  \begin{tabular}{l>{\ttfamily}l>{\ttfamily}l>{\ttfamily}l}
    \toprule &
      \normalfont\bfseries\ac{AO}/\ac{MO} &
      \normalfont\bfseries omitted &
      \normalfont\bfseries empty \\
    \midrule
      syntax:
      &      & 1s                                       & 1s= \\
    \midrule
      & 1s   & \{0;pair\}                               & \{0;\} \\
      & 2s   & \{2;pair\}                               & \{2;\} \\
      & 2p   & \{5;pair,pair,pair\}                     & \{5;,{},\} \\
    \midrule
      & 1sMO & \{.5;pair,pair\}                         & \{.5;,\} \\
      & 2sMO & \{.5;pair,pair\}                         & \{.5;,\} \\
      & 2pMO & \{1.5,.5;pair,pair,pair,pair,pair,pair\} & \{1.5,.5;{},{},{},{},{},\} \\
    \bottomrule
  \end{tabular}
\end{center}

This is similar for the \cs{AO} command (page~\pageref{ssec:AO_MO_irgendwo});
it needs a value for \meta{energy}, though.

\begin{center}
  \small
  \begin{tabular}{>{\ttfamily}l>{\ttfamily}l>{\ttfamily}l}
    \toprule
      \bfseries\meta{type} & \bfseries \meta{el-spec} \\
    \midrule
      s & pair \\
      p & pair,pair,pair \\
    \bottomrule
  \end{tabular}
\end{center}

Compare these examples:
\begin{example}[side-by-side]
  \begin{modiagram}
    \atom{left} { 1s={0;pair} }
    \atom{right}{ 1s }
  \end{modiagram}

  \hrulefill
 
  \begin{modiagram}
    \atom{left}{ 1s=1 }
    \atom{right}{ 1s= }
  \end{modiagram}
\end{example}

\section{Customization}
The options of the section~\ref{ssec:umgebungs_optionen} can be set global as
package option, \ie, with \cs*{usepackage}\oarg{options}\Marg{modiagram}, or
via the setup command \cs{setmodiagram}\marg{options}.

\subsection{Environment Options}\label{ssec:umgebungs_optionen}
There are some options with which the layout of the \ac{MO} diagrams can be
changed:
\begin{options}
  \keyval{style}{type}
    change the style of the orbitals and the connecting lines,
    section~\ref{option:style}.
  \keyval{distance}{dim}
    distance betwen left and right atom,
    section~\ref{option:distance}.
  \keyval{AO-width}{dim}
    change the width of orbitals,
    section~\ref{option:AO-width}.
  \keyval{el-sep}{num}
    distance between the electron pair arrows,
    section~\ref{option:electrons}.
  \keyval{up-el-pos}{num}
    position of the spin-up arrow,
    section~\ref{option:electrons}.
  \keyval{down-el-pos}{num}
    position of the spin-down arrow,
    section~\ref{option:electrons}.
  \keyval{lines}{tikz}
    change the \TikZ\ style of the connecting lines,
    section~\ref{option:lines}.
  \keybool{names}
    add captions to the atoms and the molecule,
    section~\ref{option:names}.
  \keyval{names-style}{tikz}
    change the \TikZ\ style of the captions,
    section~\ref{option:names_style}.
  \keyval{names-style-add}{tikz}
    change the \TikZ\ style of the captions,
    section~\ref{option:names_style}.
  \keybool{labels}
    add default labels to the orbitals,
    section~\ref{option:labels}.
  \keyval{labels-fs}{cs}
    change the font size of the labels,
    section~\ref{option:labels-fs}.
  \keyval{labels-style}{tikz}
    change the \TikZ\ style of the labels,
    section~\ref{option:labels-style}.
\end{options}
They all are discussed in the following sections.  If they're used as options
for the environment, they're set locally and only change that environment.
\begin{sourcecode}
  \begin{modiagram}[options]
    ...
  \end{modiagram}
\end{sourcecode}

\subsubsection{Option \option*{style}}\label{option:style}
There are five different styles which can be chosen.
\begin{itemize}
  \item\keyis{style}{plain} \AOinline[style=plain]{pair} (default)
  \item\keyis{style}{square} \AOinline[style=square]{pair}
  \item\keyis{style}{circle} \AOinline[style=circle]{pair}
  \item\keyis{style}{round} \AOinline[style=round]{pair}
  \item\keyis{style}{fancy} \AOinline[style=fancy]{pair}
\end{itemize}

Let's take the \ac{MO} diagram of \ch{H2} to illustrate the different styles:
\begin{example}[side-by-side]
  % use package `chemmacros'
  \begin{modiagram}[style=plain]% default
    \atom[H]{left} { 1s = {;up} }
    \atom[H]{right}{ 1s = {;up} }
    \molecule[\ch{H2}]{ 1sMO = {.75;pair} }
  \end{modiagram}
\end{example}

\begin{example}[side-by-side]
  % use package `chemmacros'
  \begin{modiagram}[style=square]
    \atom[H]{left} { 1s = {;up} }
    \atom[H]{right}{ 1s = {;up} }
    \molecule[\ch{H2}]{ 1sMO = {.75;pair} }
  \end{modiagram}
\end{example}

\begin{example}[side-by-side]
  % use package `chemmacros'
  \begin{modiagram}[style=circle]
    \atom[H]{left} { 1s = {;up} }
    \atom[H]{right}{ 1s = {;up} }
    \molecule[\ch{H2}]{ 1sMO = {.75;pair} }
  \end{modiagram}
\end{example}

\begin{example}[side-by-side]
  % use package `chemmacros'
  \begin{modiagram}[style=round]
    \atom[H]{left} { 1s = {;up} }
    \atom[H]{right}{ 1s = {;up} }
    \molecule[\ch{H2}]{ 1sMO = {.75;pair} }
  \end{modiagram}
\end{example}

\begin{example}[side-by-side]
  % use package `chemmacros'
  \begin{modiagram}[style=fancy]
    \atom[H]{left} { 1s = {;up} }
    \atom[H]{right}{ 1s = {;up} }
    \molecule[\ch{H2}]{ 1sMO = {.75;pair} }
  \end{modiagram}
\end{example}

\subsubsection{Option \option*{distance}}\label{option:distance}
Depending on labels and captions the \SI{4}{\centi\metre} by which the right
and left atom are separated can be too small. With \key{distance}{dim} the
length can be adjusted. This will change the position of the right atom to
\code{1cm + \meta{dim}} and the position of the molecule is changed to
\code{0.5*(1cm + \meta{dim})}, also see page~\pageref{xshift} and
section~\ref{orbital-positionen}.
\begin{example}[side-by-side]
  % use package `chemmacros'
  \begin{modiagram}[distance=6cm]
    \atom[H]{left} { 1s = {;up} }
    \atom[H]{right}{ 1s = {;up} }
    \molecule[\ch{H2}]{ 1sMO = {.75;pair} }
  \end{modiagram}
\end{example}

\subsubsection{Option \option*{AO-width}}\label{option:AO-width}
The length \option{AO-width} sets the length of the horizontal line in a
orbital displayed with the \code{plain} style. It's default value is
\SI{10}{pt}.
\begin{example}[side-by-side]
  % use package `chemmacros'
  \begin{modiagram}[AO-width=15pt]
    \atom[H]{left} { 1s = {;up} }
    \atom[H]{right}{ 1s = {;up} }
    \molecule[\ch{H2}]{ 1sMO = {.75;pair} }
  \end{modiagram}
\end{example}

\begin{example}[side-by-side]
  % use package `chemmacros'
  \begin{modiagram}[style=fancy,AO-width=15pt]
    \atom[H]{left} { 1s = {;up} }
    \atom[H]{right}{ 1s = {;up} }
    \molecule[\ch{H2}]{ 1sMO = {.75;pair} }
  \end{modiagram}
\end{example}
By changing the value of \option{AO-width} the positions of the p and the
\chempi\ orbitals also change, see section~\ref{orbital-positionen}.

\subsubsection{Optionen \option*{el-sep}, \option*{up-el-pos} und
  \option*{down-el-pos}}
\label{option:electrons}

These three options change the horizontal positions of the arrows representing
the electrons in an \ac{AO}/\ac{MO}. The option \key{el-sep}{num} needs a
value between \code{0} and \code{1}. \code{0} means \emph{no} distance between
the arrows and \code{1} \emph{full} distance (with respect to the length
\option{AO-width}, see section~\ref{option:AO-width}).

\begin{example}[side-by-side]
  % use package `chemmacros'
  \begin{modiagram}[el-sep=.2]% default
    \atom[H]{left} { 1s = {;up} }
    \atom[H]{right}{ 1s = {;up} }
    \molecule[\ch{H2}]{ 1sMO = {.75;pair} }
  \end{modiagram}
\end{example}

\begin{example}[side-by-side]
  % use package `chemmacros'
  \begin{modiagram}[el-sep=0]
    \atom[H]{left} { 1s = {;up} }
    \atom[H]{right}{ 1s = {;up} }
    \molecule[\ch{H2}]{ 1sMO = {.75;pair} }
  \end{modiagram}
\end{example}

\begin{example}[side-by-side]
  % use package `chemmacros'
   \begin{modiagram}[el-sep=1]
    \atom[H]{left} { 1s = {;up} }
    \atom[H]{right}{ 1s = {;up} }
    \molecule[\ch{H2}]{ 1sMO = {.75;pair} }
  \end{modiagram}
\end{example}

The options \key{up-el-pos}{<num>} and \key{down-el-pos}{<num>} can be used
alternatively to place the spin-up and spin-down electron, respectively.
Again they need values between \code{0} and \code{1}.  This time \code{0}
means \emph{on the left} and \code{1} means \emph{on the right}.

\begin{example}
  % use package `chemmacros'
  \begin{modiagram}[up-el-pos=.4,down-el-pos=.6]% default
    \atom[H]{left} { 1s = {;up} }
    \atom[H]{right}{ 1s = {;up} }
    \molecule[\ch{H2}]{ 1sMO = {.75;pair} }
  \end{modiagram}
\end{example}

\begin{example}
  % use package `chemmacros'
  \begin{modiagram}[up-el-pos=.333,down-el-pos=.667]
    \atom[H]{left} { 1s = {;up} }
    \atom[H]{right}{ 1s = {;up} }
    \molecule[\ch{H2}]{ 1sMO = {.75;pair} }
  \end{modiagram}
\end{example}

\begin{example}
  % use package `chemmacros'
  \begin{modiagram}[up-el-pos=.7,down-el-pos=.3]
    \atom[H]{left} { 1s = {;up} }
    \atom[H]{right}{ 1s = {;up} }
    \molecule[\ch{H2}]{ 1sMO = {.75;pair} }
  \end{modiagram}
\end{example}

\subsubsection{Option \option*{lines}}\label{option:lines}
The option \option{lines} can be used to modify the \TikZ\ style of the
connecting lines:
\begin{example}[side-by-side]
  % use package `chemmacros'
  \begin{modiagram}[lines={gray,thin}]
    \atom[H]{left} { 1s = {;up} }
    \atom[H]{right}{ 1s = {;up} }
    \molecule[\ch{H2}]{ 1sMO = {.75;pair} }
  \end{modiagram}
\end{example}

\subsubsection{Option \option*{names}}\label{option:names}
If you use the option \option{names} the atoms and the molecule get captions
provided you have used the optional \meta{name} argument of \cs{atom} and/or
\cs{molecule}.
\begin{example}[side-by-side]
  % use package `chemmacros'
  \begin{modiagram}[names]
    \atom[H]{left} { 1s = {;up} }
    \atom[H]{right}{ 1s = {;up} }
    \molecule[\ch{H2}]{ 1sMO = {.75;pair} }
  \end{modiagram}
\end{example}

\subsubsection{Options \option*{names-style} and \option*{names-style-add}}\label{option:names_style}
These options enable to customize the style of the captions of the atoms and
of the molecule. By default this setting is used:
\key{names-style}{anchor=base}\footnote{Please see ``\TikZ\ and PGF -- Manual
  for Version 2.10'' p.\,183 section 16.4.4 (pgfmanual.pdf) for the meaning}.
\begin{example}[side-by-side]
  % use package `chemmacros'
  \begin{modiagram}[names,names-style={draw=blue}]
    \atom[p]{left} { 1s = {;up} }
    \atom[b]{right}{ 1s = {;up} }
    \molecule[\ch{X2}]{ 1sMO = {.75;pair} }
  \end{modiagram}
\end{example}

With this the default setting is overwritten. As you can see it destroys the
vertical alignment of the nodes. In order to avoid that you can for example
specify \code{text height} and \code{text depth} yourself \ldots
\begin{example}
  % use package `chemmacros'
  \begin{modiagram}[names,names-style={text height=1.5ex, text depth=.25ex, draw=blue}]
    \atom[p]{left} { 1s = {;up} }
    \atom[b]{right}{ 1s = {;up} }
    \molecule[\ch{X2}]{ 1sMO = {.75;pair} }
  \end{modiagram}
\end{example}

\ldots, add the \code{anchor} again \ldots
\begin{example}
  % use package `chemmacros'
  \begin{modiagram}[names,names-style={anchor=base, draw=blue}]
    \atom[p]{left} { 1s = {;up} }
    \atom[b]{right}{ 1s = {;up} }
    \molecule[\ch{X2}]{ 1sMO = {.75;pair} }
  \end{modiagram}
\end{example}

\ldots\ or use the option \key{names-style-add}. It doesn't overwrite the
current setting but appends the new declaration:
\begin{example}
  % use package `chemmacros'
  \begin{modiagram}[names,names-style-add={draw=blue}]
    \atom[p]{left} { 1s = {;up} }
    \atom[b]{right}{ 1s = {;up} }
    \molecule[\ch{X2}]{ 1sMO = {.75;pair} }
  \end{modiagram}
\end{example}

\begin{example}
  % use package `chemmacros'
  \setmodiagram{
    names,
    names-style = {
      text height = 2.5ex,
      text depth = .5ex,
      draw = blue!80,
      rounded corners
    }
  }
  \begin{modiagram}
    \atom[p]{left} { 1s = {;up} }
    \atom[b]{right}{ 1s = {;up} }
    \molecule[\ch{X2}]{ 1sMO = {.75;pair} }
  \end{modiagram}
  \begin{modiagram}[names-style-add={fill=blue!20}]
    \atom[p]{left} { 1s = {;up} }
    \atom[b]{right}{ 1s = {;up} }
    \molecule[\ch{X2}]{ 1sMO = {.75;pair} }
  \end{modiagram}
\end{example}

\subsubsection{Option \option*{labels}}\label{option:labels}
If you use the option \option{labels} predefined labels are written below the
orbitals.  These labels can be changed, see section~\ref{sec:key:label}.
\begin{example}[side-by-side]
  % use package `chemmacros'
  \begin{modiagram}[labels]
    \atom[H]{left} { 1s = {;up} }
    \atom[H]{right}{ 1s = {;up} }
    \molecule[\ch{H2}]{ 1sMO = {.75;pair} }
  \end{modiagram}
\end{example}

\subsubsection{Option \option*{labels-fs}}\label{option:labels-fs}
Labels are set with the font size \cs*{small}.  If you want to change that you
can use the option \option{labels-fs}.
\begin{example}
  % use package `chemmacros'
  \begin{modiagram}[labels,labels-fs=\footnotesize]
    \atom[H]{left} { 1s = {;up} }
    \atom[H]{right}{ 1s = {;up} }
    \molecule[\ch{H2}]{ 1sMO = {.75;pair} }
  \end{modiagram}
\end{example}

This also allows you to change the font style or font shape of the labels.
\begin{example}
  % use package `chemmacros'
  \begin{modiagram}[labels,labels-fs=\sffamily\footnotesize]
    \atom[H]{left} { 1s = {;up} }
    \atom[H]{right}{ 1s = {;up} }
    \molecule[\ch{H2}]{ 1sMO = {.75;pair} }
  \end{modiagram}
\end{example}

\subsubsection{Option \option*{labels-style}}\label{option:labels-style}
The option \option{labels-style} changes the \TikZ\ style of the nodes within
which the labels are written.
\begin{example}
  % use package `chemmacros'
  \begin{modiagram}[labels,labels-style={blue,yshift=4pt}]
    \atom[H]{left} { 1s = {;up} }
    \atom[H]{right}{ 1s = {;up} }
    \molecule[\ch{H2}]{ 1sMO = {.75;pair} }
  \end{modiagram}
\end{example}

\subsection{\cs*{atom} and \cs*{molecule} Specific Customizations}
\subsubsection{The \option*{label} Key}\label{sec:key:label}
If you don't want to use the predefined labels, change single labels or use
only one or two labels, you can use the key \option{label}.  This option is
used in the \cs{atom} and \cs{molecule} commands in the \meta{AO-spec} or
\meta{MO-spec} argument, respectively.  The key awaits a comma separated
key/value list.  The names mentioned in section~\ref{Namensgebung} are used as
keys to specify the \ac{AO} that you want to label.
\begin{example}[side-by-side]
  % use package `chemmacros'
  \begin{modiagram}[labels-fs=\footnotesize]
    \atom[H]{left} { 1s = {;up} }
    \atom[H]{right}{ 1s = {;up} }
    \molecule[\ch{H2}]{
      1sMO  = {.75;pair},
      label = { 1sigma = {bonding MO} }
    }
  \end{modiagram}
\end{example}

\begin{example}[side-by-side]
  \begin{modiagram}[style=square,distance=6cm]
    \atom{left} { 1s = {;up} }
    \atom{right}{ 1s = {;up} }
    \molecule{
      1sMO  = {.75;pair} ,
      label = {
        1sigma  = \chemsigma,
        1sigma* = \chemsigma$^*$
      }
    }
    \node[right] at (1sigma.-45) {bonding};
    \node[right] at (1sigma*.45) {anti-bonding};
  \end{modiagram}
\end{example}

If the option is used together with the \option{labels} option
(page~\pageref{option:labels}) single labels are overwritten:
\begin{example}[side-by-side]
  % use package `chemmacros'
  \begin{modiagram}[labels]
    \atom[H]{left} { 1s = {;up} }
    \atom[H]{right}{ 1s = {;up} }
    \molecule[\ch{H2}]{
      1sMO  = {.75;pair},
      label = { 1sigma = \textcolor{red}{??} }
    }
  \end{modiagram}
\end{example}

\subsubsection{The \option*{color} Key}\label{sec:key:color}
Analogous to the \option{label} key the \option{color} key can be used to
display coloured electrons:
\begin{example}[side-by-side]
  % use package `chemmacros'
  \begin{modiagram}[labels-fs=\footnotesize]
    \atom[H]{left}{
      1s, color = { 1sleft = blue }
    }
    \atom[H]{right}{
      1s, color = { 1sright = red }
    }
    \molecule[\ch{H2}]{
      1sMO,
      label = { 1sigma = {bonding MO} },
      color = { 1sigma = green, 1sigma* = cyan }
    }
  \end{modiagram}
\end{example}

\subsubsection{The \option*{up-el-pos} and \option*{down-el-pos} keys}\label{sec:key:electrons}
The options \option{up-el-pos} and \option{down-el-pos} allow it to shift the
arrows representing the electrons in a single \ac{AO} or \ac{MO} individually.
You need to use values between \code{0} and \code{1}, also see
section~\ref{option:electrons}.
\begin{example}[side-by-side]
  % use package `chemmacros'
  \begin{modiagram}
    \atom[H]{left}{
      1s        = {;up},
      up-el-pos = { 1sleft=.5 }
    }
    \atom[H]{right}{ 1s = {;up} }
    \molecule[\ch{H2}]{
      1sMO        = {.75;pair} ,
      up-el-pos   = { 1sigma=.15 } ,
      down-el-pos = { 1sigma=.85 }
    }
  \end{modiagram}
\end{example}

\subsection{\cs*{AO} Specific Customizations}\label{ssec:AO_anpassen}
These keys enable to customize orbitals created with \cs{AO}.

\subsubsection{The \option*{label} Key}\label{key:AO_label}
The key \option{label}\Oarg{\meta{x}/\meta{y}/\meta{z}} allows you to put a
label to the \ac{AO}/\ac{MO}.  If you use the type \code{p} you can specify
the orbital you want to label in square brackets:
\begin{example}[side-by-side]
  \begin{modiagram}[style=square]
    \AO{s}[label={s orbital}]{0}
    \AO{p}[label[y]=py,label[z]=pz]{1.5}
  \end{modiagram}
\end{example}

\subsubsection{The \option*{color} Key}\label{key:AO_color}
Analogous to the \option{label} key there is the key
\option{color}\Oarg{\meta{x}/\meta{y}/\meta{z}} which enables you to choose a
color for the electrons.  If you use the type \code{p} you can specify the
orbital in square brackets:
\begin{example}[side-by-side]
  \begin{modiagram}[style=square]
    \AO{s}[color=red]{0}
    \AO{p}[color[y]=green,color[z]=cyan]{1.5}
  \end{modiagram}
\end{example}

\subsubsection{The \option*{up-el-pos} and \option*{down-el-pos} Keys}\label{key:AO_electrons}
Then there are the keys \option{up-el-pos}\Oarg{\meta{x}/\meta{y}/\meta{z}}
and \option{down-el-pos}\Oarg{\meta{x}/\meta{y}/\meta{z}} with which the
electrons can be shifted horizontally.  You can use values between \code{0}
and \code{1}, also see section~\ref{option:electrons}.  If you use the type
\code{p} you can specify the orbital in square brackets:
\begin{example}[side-by-side]
  \begin{modiagram}[style=square]
    \AO{s}[up-el-pos=.15]{0}
    \AO{p}[up-el-pos[y]=.15,down-el-pos[z]=.15]{1.5}
  \end{modiagram}
\end{example}

\subsection{Energy Axis}
Last but not least one might want to add an energy axis to the diagram.  For
this there is the command \cs{EnergyAxis}.

\begin{commands}
  \command{EnergyAxis}[\oarg{option}]
    Adds an energy axis to the diagram.  \meta{options} are key/value pairs to
    modify the axis.
\end{commands}

\begin{example}[side-by-side]
  \begin{modiagram}
    \atom{left} { 1s = {;up} }
    \atom{right}{ 1s = {;up} }
    \molecule{ 1sMO = {.75;pair} }
    \EnergyAxis
  \end{modiagram}
\end{example}

For the time being there are two options to modify the axis.
\begin{options}
  \keyval{title}{title}\Default{energy}
    the axis label.  If used without value the default is used.
  \keyval{head}{tikz arrow head}\Default{>}
    the arrow head; you can use the arrow heads specified in the \TikZ\ library
    \code{arrows} (pgfmanual v2.10 pages 256ff.)
\end{options}

\begin{example}[side-by-side]
  \begin{modiagram}
    \atom{left} { 1s = {;up} }
    \atom{right}{ 1s = {;up} }
    \molecule{ 1sMO = {.75;pair} }
    \EnergyAxis[title]
  \end{modiagram}
\end{example}

\begin{example}[side-by-side]
  \begin{modiagram}
    \atom{left} { 1s = {;up} }
    \atom{right}{ 1s = {;up} }
    \molecule{ 1sMO = {.75;pair} }
    \EnergyAxis[title=E,head=stealth]
  \end{modiagram}
\end{example}

\section{Examples}
The example from the beginning of section \ref{ssec:AO_MO_irgendwo}.
\begin{example}
  % use package `chemmacros'
  \begin{modiagram}[names]
    \atom[\chlewis{0.}{F}\hspace*{5mm}\chlewis{180.}{F}]{left}{
      1s=.2;up,up-el-pos={1sleft=.5}
    }
    \atom[Xe]{right}{1s=1.25;pair}
    \molecule[\ch{XeF2}]{1sMO={1/.25;pair}}
    \AO(1cm){s}{0;up}
    \AO(3cm){s}{0;pair}
    \connect{ AO1 & AO2 }
    \node[right,xshift=4mm] at (1sigma) {\footnotesize bonding};
    \node[above] at (AO2.90) {\footnotesize non-bonding};
    \node[above] at (1sigma*.90) {\footnotesize anti-bonding};
  \end{modiagram}
\end{example}

\begin{example}[outside]
  % use package `chemmacros'
  \begin{figure}[p]
    \centering
    \begin{modiagram}[style=square,labels,names,AO-width=8pt,labels-fs=\footnotesize]
      \atom[\ch{O_a}]{left}{
        1s, 2s, 2p = {;pair,up,up}
      }
      \atom[\ch{O_b}]{right}{
        1s, 2s, 2p = {;pair,up,up}
      }
      \molecule[\ch{O2}]{
        1sMO, 2sMO, 2pMO = {;pair,pair,pair,up,up},
        color = { 2piy*=red, 2piz*=red }
      }
      \EnergyAxis
    \end{modiagram}
    \caption{MO diagram of \ch{^3 "\chemSigma-" O2}.}
  \end{figure}
\end{example}

\begin{example}[outside]
  % use package `chemmacros'
  \begin{figure}[p]
    \centering
    \setmodiagram{style = fancy, distance = 7cm, AO-width = 15pt, labels}
    \begin{modiagram}
      \atom[N]{left}{
        2p = {0;up,up,up}
      }
      \atom[O]{right}{
        2p = {2;pair,up,up}
      }
      \molecule[NO]{
        2pMO = {1.8,.4;pair,pair,pair,up},
        color = { 2piy*=red }
      }
      \EnergyAxis
    \end{modiagram}
    \caption{Part of the MO diagram of \chlewis{180.}{NO}.}
  \end{figure}
\end{example}

\clearpage

\end{document}
