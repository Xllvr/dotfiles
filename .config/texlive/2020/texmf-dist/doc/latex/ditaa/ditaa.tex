\documentclass[pagesize=auto]{scrartcl}

\addtokomafont{title}{\rmfamily}

\title{The \textsf{ditaa} package}
\author{Hiroshi Ukai}
\date{2018/10/18}


\usepackage{verbatim}
\def\imagepath{./resources}
\usepackage[imagepath=\imagepath]{ditaa}
\graphicspath{ {\imagepath/} }

\begin{document}

\maketitle

\noindent
The \LaTeX\ \textsf{ditaa} package renders your ascii art inside your LaTeX files into pretty image files which you can use them in your papers.

\minisec{Usage:}
To \texttt{usepackage} \textbf{ditaa}, typically do following.
\begin{verbatim}
  \def\imagepath{./resources}
  \usepackage[imagepath=\imagepath]{ditaa}
  \graphicspath{ {\imagepath/} }
\end{verbatim}

A vaiable \texttt{\textbackslash imagepath} points a directory that your images reside.
Under it, you need to create a directory \texttt{ditaa} in advance.
In this directory, \texttt{ditaa} environment creates its working files.

\minisec{Features:}
%
This is a ditaa diagram example.
\begin{verbatim}
  \begin{ditaa}[8cm]{ditaa caption example}{ditaaexample}
    +--------+   +-------+    +-------+
    |        | --+ ditaa +--> |       |
    |  Text  |   +-------+    |diagram|
    |Document|   |!magic!|    |       |
    |     {d}|   |       |    |       |
    +---+----+   +-------+    +-------+
        :                         ^
        |       Lots of work      |
        +-------------------------+
  \end{ditaa}
\end{verbatim}

The fragment above will be rendered into following diagram.

\begin{ditaa}[8cm]{ditaa caption example}{ditaaexample}
  +--------+   +-------+    +-------+
  |        | --+ ditaa +--> |       |
  |  Text  |   +-------+    |diagram|
  |Document|   |!magic!|    |       |
  |     {d}|   |       |    |       |
  +---+----+   +-------+    +-------+
      :                         ^
      |       Lots of work      |
      +-------------------------+
\end{ditaa}
\pagebreak
And you can specify image width like this Figure.\ref{fig:ditaaexample2}.

\begin{verbatim}
  \begin{ditaa}[6cm]{ditaa caption example2}{ditaaexample2}
    +---------+
    | cBLU    |
    |         |
    |    +----+
    |    |cPNK|
    |    |    |
    +----+----+
  \end{ditaa}
\end{verbatim}

To denote your diagram, you can do

\begin{verbatim}

  Figure~\ref{fig:ditaaexample2}

\end{verbatim}

, where \texttt{ditaaexample2} is the second argument you gave to the \texttt{ditaa} environment as you see above.
The prefix \texttt{fig:} is given by \texttt{ditaa} environment automatically.

\begin{ditaa}[6cm]{ditaa caption example2}{ditaaexample2}
  +---------+
  | cBLU    |
  |         |
  |    +----+
  |    |cPNK|
  |    |    |
  +----+----+
\end{ditaa}

Enjoy!

\minisec{ChangeLog:}
%
\begin{labeling}[\hspace{\labelsep}--]{0.9}
\item[0.9] Preparation for first \textsf{ditaa} release.
\end{labeling}

\end{document}
