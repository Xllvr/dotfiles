\chapter{绪论}
\section{欢迎!}
欢迎来到\cquthesis{}示例文档!

本文档使用\cquthesis{}本身作为模板,即\pkg{cquthesis.cls}, \pkg{cquthesis.sty}和\pkg{cquthesis.cfg},旨在展现\cquthesis{}的使用方法。请结合\cquthesis{}用户手册和本文档源代码进行学习和操作。

祝毕设成功,答辩拿优!Happy Texing!

本文档编译时使用的\cquthesis{}版本为\version{}。

请留意到本链接检查更新:\url{https://github.com/nanmu42/CQUThesis}

\section{关于\LaTeX{}}
\noindent{\heiti{}提示:}{\kaishu{}下面是一些基本思路和知识,如果你已经对\LaTeX{}比较熟悉,请直接跳转到第\ref{txt:FreqCmd}节。}

\subsection{关于推荐重庆大学开设\TeX 相关课程并推广其运用的提议}
这一小节是对\href{http://jq.qq.com/?_wv=1027&k=2HvYu95}{重庆大学\TeX 用户组}所撰写的提案的简介。

本提案从介绍排版系统\TeX 的背景和特点开始,从研究生期刊论文投稿以及毕业生毕业论文排版工作这两个维度阐述了引入\TeX 作为一种与Office Word平行的写作系统的优势和必要性,最终提出一套基于我校重庆大学实际情况,有效可行的实施方案。

这份提案可以作为新手从全局认识\TeX 的入门材料,提案的下载地址是:\url{https://github.com/CQUtug/TeXProposal}

\subsection{\LaTeX{}小传}
\LaTeX{}是\TeX{}的改进版本,后者由Knuth(高德纳)在上世纪七十年代研发,包含\TeX{}排版程序和Plain \TeX{}宏集这两部分。Plain \TeX{}可以看做是一种既定语法的编程语言,源代码对应文件后缀为\pkg{.tex},而\TeX{}程序对源代码进行解析,编译,得到排版结果。上世纪八十年代,\LaTeX{}对Plain \TeX{}的语言体系进行了升级和重构,使得\TeX{}的易用性获得了质的提升。

\TeX{}有着很多分支,比如\LaTeX{}, \LuaTeX{}和\XeTeX{}。每个分支的产生都是为了解决不同的问题。其中,\XeTeX{}提供了对东亚字体(中日韩)的原生支持。


\subsection{\LaTeX{}背后的思路}

\LaTeX{}的思路也许你已经有所耳闻,即{\heiti{}内容和样式分离。}它有些像HTML编辑器,遵循\textsf{WYTIWYG}原则\footnote{What you think is what you get. 所想即所得。}。这也是它和Word这一类遵循\qthis{所见即所得}\footnote{\textsf{WYSIWYG} -- What you see is what you get.}原则的文档编辑器的最大不同。

在你使用\LaTeX{}写作的时候,大部分时间你只需要关心内容本身,而\LaTeX{}就像你的编辑,按照样式要求(模板和宏包)为你排版。


\section{\cquthesis{}背后的思路}

\cquthesis{}秉承\LaTeX{}的思路,旨在为你解决论文内容以外的大部分问题。

出于性能和管理方面的考虑,\cquthesis{}使用分布式的源文件方案,将论文的各个部分(通常以章为单位)分散到tex文件中,然后在主文档\pkg{main.tex}中统一处理。\figref{fig:filetree}展示了一个可能的文件目录情况。
\begin{figure}[htb]
	\dirtree{%
		.1 \myfolder{pink}{工作文件夹}.
		.2 \myfolder{cyan}{\pkg{cquthesis.cls}}.
		.2 \myfolder{cyan}{\pkg{cquthesis.cfg}}.
		.2 \myfolder{cyan}{\pkg{cquthesis.sty}}.
		.2 \myfolder{cyan}{main.tex}.
		.2 \myfolder{cyan}{contents}.
		.3 \myfolder{lime}{introduction.tex}.
		.3 \myfolder{lime}{experiment.tex}.
		.3 \myfolder{lime}{analysis.tex}.
		.3 \myfolder{lime}{conclusion.tex}.
		.2 \myfolder{cyan}{figures}.
		.3 \myfolder{lime}{myCat.png}.
		.3 \myfolder{lime}{dogEatsSandwiches.jpg}.
		.2 \myfolder{cyan}{ref}.
		.3 \myfolder{lime}{refs.bib}.
	}%\dirtree
\caption[\cquthesis{}文件结构图示]{\cquthesis{}文件结构图示,出于测试的原因,这个标题被故意填充得很长,这里,你可以结合本文文档代码看到插图索引中是如何处理这个问题的。}
\label{fig:filetree}
\end{figure}






