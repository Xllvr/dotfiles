%%
%% This is file `gabarit-these-articles.tex',
%% generated with the docstrip utility.
%%
%% The original source files were:
%%
%% hecthese.dtx  (with options: `gabarit,phd,articles,francais')
%% 
%% This is a stripped version of the original file.
%% 
%% Copyright 2017-2019 HEC Montreal
%% 
%% This work may be distributed and/or modified under the
%% conditions of the LaTeX Project Public License, either version 1.3c
%% of this license or (at your option) any later version.
%% 
%% The latest version of this license is in
%% http://www.latex-project.org/lppl.txt
%% and version 1.3c or later is part of all distributions of LaTeX
%% version 2008/05/04 or later.
%% 
%% This work has the LPPL maintenance status `maintained'.
%% 
%% The Current Maintainer of this work is Benoit Hamel
%% <benoit.2.hamel@hec.ca>.
%% 
%% This work consists of the files hecthese.dtx, hecthese-fr.ins,
%% hecthese-en.ins, hecthese.pdf, hecthese-en.pdf
%% and the derived files listed in the README file.
%% 
%% GABARIT POUR UNE THÈSE PAR ARTICLES
%%
%% Ceci est le fichier maître dans lequel vous inscrivez les métadonnées
%% relatives à votre travail, vous créez vos commandes et environnements
%% personnalisés, et à partir duquel vous lancez vos compilations.
%%
%% NE RÉDIGEZ PAS VOTRE THÈSE OU MÉMOIRE DANS CE FICHIER!
%%
%% Consultez la documentation de la classe hecthese pour de plus amples
%% informations.
%%
%% DÉCLARATION DE LA CLASSE DE DOCUMENT
%%
%% La classe est déclarée avec le type de document et les langues par
%% défaut. Inscrivez dans la liste d'options la taille de police de
%% caractères (10pt, 11pt, 12pt) ou laissez la classe charger la taille
%% par défaut : 12pt.
\documentclass[phdarticles,english,frenchb]{hecthese}
%%
%% PACKAGES À CHARGER
%%
%% Ajoutez tous les packages nécessaires à la rédaction de votre travail.
%% Consultez la documentation de la classe pour connaître la liste des
%% packages qui sont chargés par défaut. Assurez-vous cependant de suivre
%% les consignes suivantes :
%%
%% 1) Le package hyperref doit être chargé EN DERNIER si vous voulez qu'il
%% fonctionne correctement.
%% 2) Le package geometry est INCOMPATIBLE avec la classe memoir. Vous ne
%% devez pas l'utiliser dans votre travail. Consultez la documentation
%% de la classe memoir pour de plus amples informations.
%%
%% CHOIX D'UNE POLICE DE CARACTÈRES
%%
%% Choisissez le package mathptmx si vous voulez utiliser une police de type
%% Times, avec empattements, et le package mathpazo si vous voulez utiliser
%% une police de type Arial, sans empattements. Choisissez-en une et supprimez
%% l'autre, ou mettez-la en commentaires.
\usepackage{mathptmx}
%% \usepackage{mathpazo}

\usepackage{hyperref}
%%
%% PRODUCTION DE L'INDEX
%%
\makeindex
%%
%% GÉNÉRATION DES TITRES
%%
%% On change les titres de l'introduction et de la conclusion générales.
%%
\HECgenererTitres
%%
%% MISE EN FORME DE LA TABLE DES MATIÈRES
%%
%% On inclut dans la table des matières toutes les divisions de document
%% jusqu'aux sous-sections. Si vous désirez avoir une table des matières
%% plus détaillée, indiquez dans les deux commandes ci-dessous jusqu'à
%% quel niveau vous voulez voir répertoriés dans la TDM.
%%
\setsecnumdepth{subsection} % Numérotation des sous-sections / Subsection numbering
\settocdepth{subsection} % Inclusion des sous-sections dans la TDM / Including subsections in the TOC
%%
%% MÉTADONNÉES DU DOCUMENT
%%
%% Le titre de votre travail. Si le titre est long, utilisez la commande \\
%% pour le mettre sur plusieurs lignes.
\HECtitre{Titre de la thèse ou du mémoire}
%% Le sous-titre de votre travail. S'il ne comporte pas de sous-titre, videz
%% le contenu des accolades.
\HECsoustitre{Sous-titre de la thèse}
%% L'auteur, c'est vous...
\HECauteur{Prénom Nom}
%% Nom de l'option de votre grade
\HECoption{Nom de l'option}
%% Mois du dépôt final de votre travail
\HECmoisDepot{Mai}
%% Année du dépôt final de votre travail
\HECanneeDepot{2017}
%% Le nom complet du président rapporteur et son genre (M ou F)
\HECpresidentRapporteur{Prénom Nom}{M ou F}
%% Le nom complet de votre directeur de recherche et son genre (M ou F)
\HECdirecteurRecherche{Prénom Nom}{M ou F}
%% Le nom complet de votre codirecteur de recherche et son genre (M ou F)
\HECcodirecteurRecherche{Prénom Nom}{M ou F}
%% L'université de provenance de votre codirecteur de recherche
\HECuniversiteCodirecteur{HEC Montréal}
%% Le nom complet du membre du jury
\HECmembreJury{Prénom Nom}
%% L'université de provenance du membre du jury
\HECuniversiteMembreJury{HEC Montréal}
%% Le nom complet de l'examinateur externe et son genre (M ou F)
\HECexaminateurExterne{Prénom Nom}{M ou F}
%% L'université de provenance de l'examinateur externe
\HECuniversiteExaminateur{HEC Montréal}
%% Le nom complet du représentant du directeur et son genre (M ou F)
\HECrepresentantDirecteur{Prénom Nom}{M ou F}
%%
%% OPTIONS DES PACKAGES CHARGÉS
%%
%% Si vos packages ont des options spécifiques à charger avant le début
%% du document, inscrivez-les ci-dessous. Si vous voulez outrepasser les
%% options des packages chargés par défaut avec la classe hecthese,
%% consultez la documentation pour en connaître la procédure.
%%
%% Options du package hyperref (inclure les métadonnées pdf dans les options) /
%% hyperref package option (including pdf metadata)
\hypersetup{%
colorlinks=true,
allcolors=black,
pdfauthor=\HECpdfauteur,
pdftitle=\HECpdftitre
}
%% Options de babel / babel options
\frenchbsetup{% 
og=«, fg=» % caractères « et » sont les guillemets
}
%%
%% DÉBUT DE LA THÈSE OU DU MÉMOIRE
%%
\begin{document}

%% Pages liminaires / frontmatter
\frontmatter

%% Page de garde / cover page
\mbox{}
\thispagestyle{empty}
\cleardoublepage

%% Page de titre
 \HECpagestitre

%% Configuration des bibliographies des articles /
%% Articles' bibliographies configuration
\HECbibliographieArticle

%% Résumé français / french abstract
 %% Fichier contenant le résumé français, les mots-clés et les méthodes de recherche.
\chapter*{Résumé}
\phantomsection\addcontentsline{toc}{chapter}{Résumé}
\thispagestyle{empty} % Première page non paginée / Unnumbered first page

%% Rédigez votre résumé français ici (350 à 500 mots).

\section*{Mots-clés}

%% Inscrivez vos mots-clés ici (15 maximum, incluant les méthodes de recherche ci-dessous).

\section*{Méthodes de recherche}

%% Inscrivez vos méthodes de recherche ici.

%% THÈSES ET MÉMOIRES PAR ARTICLES SEULEMENT
%% Si vous avez inséré des citations dans cette section, retirez les signes
%% de commentaires (%%) devant la commande ci-dessous et inscrivez le style
%% bibliographique et le nom du fichier .bib utilisés pour vos références.
%% \HECreferences{style}{nom-du-fichier-file-name}


%% Résumé anglais / english abstract
 %% Fichier contenant le résumé anglais, les mots-clés et les méthodes de recherche.
\chapter*{Abstract}
\phantomsection\addcontentsline{toc}{chapter}{Abstract}

%% Rédigez votre résumé anglais ici (350 à 500 mots).

\section*{Keywords}

%% Inscrivez vos mots-clés en anglais ici (15 maximum, incluant les méthodes de recherche ci-dessous).

\section*{Research methods}

%% Inscrivez vos méthodes de recherche en anglais ici.

%% THÈSES ET MÉMOIRES PAR ARTICLES SEULEMENT
%% Si vous avez inséré des citations dans cette section, retirez les signes
%% de commentaires (%%) devant la commande ci-dessous et inscrivez le style
%% bibliographique et le nom du fichier .bib utilisés pour vos références.
%% \HECreferences{style}{nom-du-fichier-file-name}


%% Table des matières (* pour ne pas inclure la TDM dans la TDM) /
%% Table of contents (* for excluding the TOC from the TOC)
\tableofcontents*
\cleardoublepage

%% Liste des tableaux / list of table
\listoftables
\cleardoublepage

%% Liste des figures / list of figures
\listoffigures
\cleardoublepage

%% Liste des abréviations / acronyms list
 %% Fichier contenant la liste des abréviations. Vous retrouverez ci-dessous un exemple
%% de liste. L'environnement HECabreviations prend en argument la plus longue des
%% abréviations. À la compilation, une liste en deux colonnes alignées est générée.
\chapter*{\HECtdmAbreviations}
\phantomsection\addcontentsline{toc}{chapter}{\HECtdmAbreviations}

\begin{HECabreviations}{ABBR}
\item[ABBR] Abréviation
\item[BAA] Baccalauréat en administration des affaires
\item[DESS] Diplôme d'études supérieures spécialisées
\item[HEC] Hautes études commerciales
\item[MBA] Maîtrise en administration des affaires
\item[MSc] Maîtrise
\item[PhD] Doctorat
\end{HECabreviations}


%% Dédicace / dedication
 %% Fichier contenant la dédicace
%% N'inscrivez rien entre les accolades de la commande \chapter*{},
%% sauf si vous voulez voir la dédicace dans la table des matières.

\chapter*{}

\begin{HECdedicace}
%% Rédigez votre dédicace ici.
\end{HECdedicace}


%% Remerciements / acknowledgements
 \chapter*{Remerciements}        % ne pas numéroter
\label{chap:remerciements}      % étiquette pour renvois
\phantomsection\addcontentsline{toc}{chapter}{\nameref{chap:remerciements}} % inclure dans TdM

<Texte des remerciements en prose.>


%% Avant-propos / preface
 %% Fichier contenant l'avant-propos
\chapter*{\HECtdmAvantPropos}
\phantomsection\addcontentsline{toc}{chapter}{\HECtdmAvantPropos}

%% Rédigez votre avant-propos ici.

%% THÈSES ET MÉMOIRES PAR ARTICLES SEULEMENT
%% Si vous avez inséré des citations dans cette section, retirez les signes
%% de commentaires (%%) devant la commande ci-dessous et inscrivez le style
%% bibliographique et le nom du fichier .bib utilisés pour vos références.
%% \HECreferences{style}{nom-du-fichier-file-name}


%% Partie principale de la thèse ou du mémoire / mainmatter
\mainmatter

%% Introduction
\documentclass[12pt]{report}
\usepackage[utf8]{inputenc}
\usepackage[brazilian,brazil]{babel}
\usepackage{fancyhdr,setspace,float,graphicx,lscape,array,longtable,colortbl,amsmath,amssymb,booktabs,multirow,hyperref,pdfpages,tocloft,titlesec,lipsum,natbib}
\usepackage[sectionbib]{chapterbib}
\begin{document}
%***
\clearpage
\linenumbers % numeração de linhas
\modulolinenumbers[3] % numeração de linhas
%***
% Texto
%***
\chapter{Nam dui ligula}
%***
\lipsum[2-2] Nam dui ligula, fringilla a, euismod sodales, sollicitudin vel, wisi. Morbi
auctor lorem non justo \citep{lamport1986latex}.
\section{Euismod sodales}
\lipsum[2-3]
%*** APAGAR O EXEMPLO ACIMA







%*** REFERÊNCIAS
\bibliography{referencias.bib}
\bibliographystyle{apalike}
\end{document}


%% Cadre théorique / theoretical framework
 %% Fichier contenant le cadre théorique
\chapter*{\HECtdmCadreTheorique}
\phantomsection\addcontentsline{toc}{chapter}{\HECtdmCadreTheorique}
\thispagestyle{empty}

%% Rédigez votre cadre théorique ici.

%% THÈSES ET MÉMOIRES PAR ARTICLES SEULEMENT
%% Si vous avez inséré des citations dans cette section, retirez les signes
%% de commentaires (%%) devant la commande ci-dessous et inscrivez le style
%% bibliographique et le nom du fichier .bib utilisés pour vos références.
%% \HECreferences{style}{nom-du-fichier-file-name}


%% Articles de développement / articles
%% Fichier contenant un article. La classe génère trois fichiers
%% d'articles par défaut. Une thèse compte généralement trois articles
%% et un mémoire, un. Si vous en avez besoin davantage,
%% enregistrez ce fichier sous un autre nom et incluez-le dans
%% votre gabarit avec la commande \include.
%%
%% Les articles sont structurés tel que vous le voyez ci-dessous :
%% un résumé non numéroté, une introduction, des sections et une
%% conclusion numérotées, et une bibliographie. Vous pouvez ajouter
%% ou supprimer des sections de développement selon vos besoins.
\chapter{Titre de l'article / Article title}
\thispagestyle{empty} % Première page non paginée / First page is unnumbered

\section*{\HECtdmResumeArticle}
\phantomsection\addcontentsline{toc}{section}{\HECtdmResumeArticle}

%% Rédigez votre résumé ici.

\section{Introduction}

%% Rédigez votre introduction d'article ici.

\section{Titre de la section de développement 1 / Section 1 title}

%% Rédigez votre section de développement ici.

\section{Titre de la section de développement 2 / Section 2 title}

%% Rédigez votre section de développement ici.

\section{Titre de la section de développement 3 / Section 3 title}

%% Rédigez votre section de développement ici.

\section{Conclusion}

%% Rédigez votre conclusion d'article ici.

\bibliographystyle{francais}
%% Inscrivez le nom de votre fichier .bib entre les accolades.
\bibliography{}

%% Fichier contenant un article. La classe génère trois fichiers
%% d'articles par défaut. Une thèse compte généralement trois articles
%% et un mémoire, un. Si vous en avez besoin davantage,
%% enregistrez ce fichier sous un autre nom et incluez-le dans
%% votre gabarit avec la commande \include.
%%
%% Les articles sont structurés tel que vous le voyez ci-dessous :
%% un résumé non numéroté, une introduction, des sections et une
%% conclusion numérotées, et une bibliographie. Vous pouvez ajouter
%% ou supprimer des sections de développement selon vos besoins.
\chapter{Titre de l'article / Article title}
\thispagestyle{empty} % Première page non paginée / First page is unnumbered

\section*{\HECtdmResumeArticle}
\phantomsection\addcontentsline{toc}{section}{\HECtdmResumeArticle}

%% Rédigez votre résumé ici.

\section{Introduction}

%% Rédigez votre introduction d'article ici.

\section{Titre de la section de développement 1 / Section 1 title}

%% Rédigez votre section de développement ici.

\section{Titre de la section de développement 2 / Section 2 title}

%% Rédigez votre section de développement ici.

\section{Titre de la section de développement 3 / Section 3 title}

%% Rédigez votre section de développement ici.

\section{Conclusion}

%% Rédigez votre conclusion d'article ici.

\bibliographystyle{francais}
%% Inscrivez le nom de votre fichier .bib entre les accolades.
\bibliography{}

%% Fichier contenant un article. La classe génère trois fichiers
%% d'articles par défaut. Une thèse compte généralement trois articles
%% et un mémoire, un. Si vous en avez besoin davantage,
%% enregistrez ce fichier sous un autre nom et incluez-le dans
%% votre gabarit avec la commande \include.
%%
%% Les articles sont structurés tel que vous le voyez ci-dessous :
%% un résumé non numéroté, une introduction, des sections et une
%% conclusion numérotées, et une bibliographie. Vous pouvez ajouter
%% ou supprimer des sections de développement selon vos besoins.
\chapter{Titre de l'article / Article title}
\thispagestyle{empty} % Première page non paginée / First page is unnumbered

\section*{\HECtdmResumeArticle}
\phantomsection\addcontentsline{toc}{section}{\HECtdmResumeArticle}

%% Rédigez votre résumé ici.

\section{Introduction}

%% Rédigez votre introduction d'article ici.

\section{Titre de la section de développement 1 / Section 1 title}

%% Rédigez votre section de développement ici.

\section{Titre de la section de développement 2 / Section 2 title}

%% Rédigez votre section de développement ici.

\section{Titre de la section de développement 3 / Section 3 title}

%% Rédigez votre section de développement ici.

\section{Conclusion}

%% Rédigez votre conclusion d'article ici.

\bibliographystyle{francais}
%% Inscrivez le nom de votre fichier .bib entre les accolades.
\bibliography{}


%% Conclusion
\chapter{The conclusion}



%% Index analytique / analytical index
\printindex

%% BIBLIOGRAPHIE / BIBLIOGRAPHY
 %% Configuration de la bibliographie générale
\HECbibliographieGenerale
\bibliographystyle{francais}
%% Inscrivez le nom de votre fichier .bib entre les accolades.
\bibliography{}

\backmatter

%% Retour à la pagination romaine / Back to roman page numbering
\pagenumbering{roman}

%% Annexes / appendices
\appendix
 \chapter{<Titre de l'annexe>}     % numérotée
\label{chap:}                   % étiquette pour renvois (à compléter!)

<Texte de l'annexe.>


%% Page de garde de fin / back cover page
\mbox{}
\thispagestyle{empty}

\end{document}
\endinput
%%
%% End of file `gabarit-these-articles.tex'.
