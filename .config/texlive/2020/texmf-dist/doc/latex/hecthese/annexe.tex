%% Fichier contenant une annexe. La classe ne génère qu'un seul
%% fichier annexe.tex par défaut. Si vous en avez besoin davantage,
%% enregistrez ce fichier sous un autre nom et incluez-le dans votre
%% gabarit avec la commande \include.
%%
%% Pour que la bibliographie et la (les) annexe(s) soient paginées
%% correctement, c'est-à-dire en chiffres arabes pour la bibliographie
%% et en chiffres romains pour les annexes, ces dernières doivent être
%% placées après la commande \backmatter. Ce faisant, la numérotation
%% des annexes s'en trouve désactivée. Vous devrez donc numéroter
%% vos annexes manuellement à l'intérieur de la commande \chapter.
\chapter{Annexe A -- Titre de l'annexe}

%% Rédigez votre annexe ici.

%% THÈSES ET MÉMOIRES PAR ARTICLES SEULEMENT
%% Si vous avez inséré des citations dans cette section, retirez les signes
%% de commentaires (%%) devant la commande ci-dessous et inscrivez le style
%% bibliographique et le nom du fichier .bib utilisés pour vos références.
%% \HECreferences{style}{nom-du-fichier-file-name}
