\section{Physik}
	\label{fach:physik}
	Zur Zeit ist das Paket \module{Physik} noch leer, bis auf das Einbinden der
	Pakete \pkg{units}, \pkg{circuittikz} und \pkg{mhchem}.

	Ein kurzes Beispiel zur Dartstellung von Schaltplänen mit dem
	Paket \pkg{circuittikz} soll an dieser Stelle genügen. Ausführlichere Hinweise können den	entsprechenden Dokumentationen der einzelnen Pakete entnommen werden.

\begin{example}[gobble=0]
\begin{circuitikz}
  \draw
    (0,0)--(1,0) to[european resistor,l=$47$\,k$\Omega$] (3,0)--(5,0)
    to[C, l=$470$\,$\mu$F] (7,0) -- (8,0)
    (4.5,0) to[short, -*] (4.5,0) -- (4.5, -2)
    (4.5,-2) -- (5,-2) to[voltmeter, l=$U_C$] (7,-2) -- (7.5,-2)
    (7.5, -2) to[short, -*] (7.5,0)
    (8,1) node[spdt, rotate=90] (Ums) {}
    (Ums) node[right=0.4cm] {$WS$}
    (Ums.out 1) node[left] {1}
    (Ums.out 2) node[right] {2}
    (0,0) |- (2,4) to[closing switch, l=$S$] (3,4) to[battery1,
    l=$U$] (5,4) -| (Ums.out 2)
    (Ums.in) -- (8,0)
    (Ums.out 1) |- (0,2) to[short, -*] (0,2)
  ;
\end{circuitikz}	
\end{example}