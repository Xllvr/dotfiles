\section{Unterrichtsbesuch}
\label{typ:ub}
Dieser Dokumenttyp dient als Grobvorlage für Unterrichtsbesuche. Eine komplette Vorlage wird nicht angeboten, da die Studienseminare unterschiedliche Anforderungen stellen und es auch in den einzelnen Seminaren sehr häufig Änderungen an den layouttechnischen Aspekten gibt. Die Hauptanwendungen dieses Dokumenttyps sind daher Unterrichtsbesuche, bei denen es keine festen Vorgaben gibt, wie z.\,B. bei Revisionen oder der Materialsammlung für Informatik, die vollständig dieses \LaTeX-Paket nutzt. Für die Verwendung dieses Dokumententyps ist \verbcode|ub| anzugeben.

Als Dokumentklasse wird für diesen Typ \cls{scrartcl} empfohlen. Darin wird durch den Dokumenttyp Kopf- und Fußzeile gesetzt, sowie eine Titelseite erzeugt, die mit Angaben gefüllt wird, die für einen Unterrichtsbesuch typisch sind und entsprechend angegeben werden müssen.

%\subsection{Paketoptionen}
%\begin{options}
%	\opt{X}
%		Y
%\end{options}

\subsection{Befehle für Angaben zum Unterrichtsbesuch}
Mit den folgenden Befehlen werden Angaben gesetzt, die auf der Titelseite des Unterrichtsbesuch angezeigt werden.
\begin{commands}
    \command{besuchtitel}[\marg{Titel}]
        setzt den Eintrag, um was für eine Art es sich bei dem Unterrichtsbesuch handelt. Dieses kann z.\,B. sein: "`2. Unterichtsbesuch im Fach Informatik"'.
    \command{lehrer}[\marg{Lehrername}]
        setzt den Namen des Lehrers, der neben der Titelseite auch im Seitenkopf angezeigt wird.

    \command{schulform}[\marg{Schulform}]
        setzt den Eintrag für die Schulform wie z.\,B. Gesamtschule.

    \command{lerngruppe}[\oarg{Kurzform der Lerngruppe} \marg{Name der Lerngruppe} \marg{Anzahl weiblich} \marg{Anzahl männlich}]
        sorgt dafür, dass die Angaben zur Lerngruppe gesetzt werden. Der Name wird auf dem Titelblatt und im Seitenkopf angegeben, außer die optionale Möglichkeit der Kurzform wurde genutzt. In diesem Fall wird die Kurzform im Seitenkopf angegeben. Aus der Anzahl der weiblichen und männlichen Schülerinnen und Schüler wird automatisch die Gesamtzahl bestimmt, daher sind für diese Angaben nur Zahlen erlaubt.

    \command{zeit}[\marg{Startzeit} \marg{Endzeit} \marg{Stunde}]
        bietet die Möglichkeit, die Zeiten der Besuchsstunde anzugeben. Neben der Uhrzeit des Beginns und des Endes muss angegeben werden, um welche Stunde es sich an dem entsprechenden Tag handelt.

    \command{schule}[\marg{Name der Schule}]
        hierüber lässt sich der Name der Schule angeben, der auf der Titelseite angezeigt wird.

    \command{raum}[\marg{Raumbezeichnung}]
        bietet die Möglichkeit die Bezeichnung des Raumes anzugeben, in dem die Besuchsstunde stattfinden soll.

\end{commands}