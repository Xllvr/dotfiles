\section{Klausur}
\label{typ:kl}
Dieser Dokumenttyp wird für Klausuren und Klassenarbeiten verwendet. Für die Verwendung der Klausur oder Klassenarbeit ist als Typ \verbcode|kl| anzugeben.

Die empfohlene Dokumentklasse ist \cls{scrartcl}.

Die Vorgaben des Dokumenttyps definieren Kopf- und Fußzeilen in der üblichen Darstellung des Schule-Pakets. Außerdem werden Namens- und Datumsfelder erzwungen.

\subsection{Paketoptionen}
\begin{options}
    \keychoice{klausurtyp}{klausur,klasse,kurs}\Default{klausur}
        legt fest, ob die Klausur als Klausur (Standard), Kursarbeit (\keyis{klausurtyp}{kurs}) oder Klassenarbeit (\keyis{klausurtyp}{klasse}) bezeichnet wird.
\end{options}

%\subsection{Befehle}
%\begin{commands}
%
%\command{X}
%	Y
%
%\end{commands}