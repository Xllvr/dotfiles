\section{Kuerzel}
\label{modul:kuerzel}
Das Modul \module{Kuerzel} stellt einige Makros bereit, die
Kurzschreibweisen für häufig verwendete Schreibweisen
bereitstellen. Sofern relevant, wird dabei die Schreibweise an
die gewählte Variante des Genderings, im Sinne einer
geschlechtergerechten Sprache angepasst.

\subsection{Paketoptionen}\begin{options}
	\keychoice{gendering}{binneni,fem,gap,mas,split,star}\Default{split}
	Standardmäßig wird die amtlich geforderte Schreibweise des Splittings (etwa Schülerinnen und Schüler) verwendet. 
	\begin{smallitemize}
		\item\hspace{1ex} Splitting: \keyis{gendering}{split}
	\end{smallitemize}
			
	Außerdem werden folgende Varianten unterstützt:
	\begin{smallitemize}
		\item\hspace{1ex} Gender-Gap: \keyis{gendering}{gap}
		\item\hspace{1ex} Gender-Star: \keyis{gendering}{star}
		\item\hspace{1ex} Binnen-I: \keyis{gendering}{binneni}
	\end{smallitemize}
			
	Für spezielle Fälle, kann auch die ausschließliche Nutzung einer Geschlechtsform erzwungen werden:
	\begin{smallitemize}
		\item \hspace{1ex}Generisches Femininum: \keyis{gendering}{fem}
		\item \hspace{1ex}Generisches Maskulinum: \keyis{gendering}{mas}
	\end{smallitemize}
	
\end{options}
\subsection{Befehle}\hfill

\begin{multicols}{2}
\begin{commands}
	\command{Lkr}
	Lehr\-kraft
	\command{Lkre}
	Lehr\-kräf\-te
	\command{Lpr}
	Lehr\-per\-son
	\command{Lprn}
	Lehr\-per\-so\-nen
	\command{EuE}
	El\-tern und Er\-zie\-hungs\-be\-recht\-ig\-te
	\command{EuEn}
	El\-tern und Er\-zie\-hungs\-be\-recht\-ig\-ten
	\command{EK}
	Er\-wei\-ter\-ungs\-kurs
	\command{EKe}
	Er\-wei\-ter\-ungs\-kurse
	\command{EKen}Er\-wei\-ter\-ungs\-kursen
	\command{GK}
	Grund\-kurs
	\command{GKe}
	Grund\-kurse
	\command{GKen}
	Grund\-kursen
	\command{LK}
	Leis\-tungs\-kurs
	\command{LKe}
	Leis\-tungs\-kurse
	\command{LKen}
	Leis\-tungs\-kursen
	\command{SuS}
{\tiny 			\begin{tabular}{ll}
		\toprule \textbf{Gendering} & \textbf{Ergebnis} \\ 
		\midrule binneni & SchülerInnen \\ 
		fem & Schülerinnen \\ 
		gap & Schüler\_innen \\ 
		mas & Schüler \\ 
		split & Schülerinnen und Schüler \\ 
		star & Schüler*innen \\ 
		\bottomrule 
	\end{tabular} }
	\command{SuSn}
	{\tiny \begin{tabular}{ll}
		\toprule \textbf{Gendering} & \textbf{Ergebnis} \\ 
		\midrule binneni & SchülerInnen \\ 
		fem & Schülerinnen \\ 
		gap & Schüler\_innen \\ 
		mas & Schülern \\ 
		split & Schülerinnen und Schülern \\ 
		star & Schüler*innen \\ 
		\bottomrule 
	\end{tabular} }
	
	\command{LuL}
	{\tiny \begin{tabular}{ll}
		\toprule \textbf{Gendering} & \textbf{Ergebnis} \\ 
		\midrule binneni & LehrerInnen \\ 
		fem & Lehrerinnen \\ 
		gap & Lehrer\_innen \\ 
		mas & Lehrer \\ 
		split & Lehrerinnen und Lehrer \\ 
		star & Lehrer*innen \\ 
		\bottomrule 
	\end{tabular} }
	
	\command{LuLn}
	{\tiny \begin{tabular}{ll}
		\toprule \textbf{Gendering} & \textbf{Ergebnis} \\ 
		\midrule binneni & LehrerInnen \\ 
		fem & Lehrerinnen \\ 
		gap & Lehrer\_innen \\ 
		mas & Lehrern \\ 
		split & Lehrerinnen und Lehrern \\ 
		star & Lehrer*innen \\ 
		\bottomrule 
	\end{tabular} }
	
	\command{KuK}
	{\tiny \begin{tabular}{ll}
		\toprule \textbf{Gendering} & \textbf{Ergebnis} \\ 
		\midrule binneni & KollegInnen \\ 
		fem & Kolleginnen \\ 
		gap & Kolleg\_innen \\ 
		mas & Kollegen \\ 
		split & Kolleginnen und Kollegen \\ 
		star & Kolleg*en*innen \\ 
		\bottomrule 
	\end{tabular} }
	
\end{commands}
\end{multicols}