% Kompatibilität mit dem Schule-Paket sicherstellen
\PassOptionsToPackage{colaction}{multicol}

\documentclass[a4paper,add-index]{cnltx-doc}

\usepackage[
    typ=ohne,
    fach=ohne,
    weitereFaecher={
        Informatik,
        Physik
    },
    farbig,
    module={
        Kuerzel,
        Papiertypen,
        Symbole,
        Texte,
    }
]{schule}

\usepackage{hyperref}
\usepackage{booktabs}
\usepackage{blindtext}
\usepackage{weva}
\usepackage{schule-dokumentation}

\usepackage{todonotes}

\usepackage{pdfpages}
\usepackage{prettyref}
\newrefformat{sec}{Abschnitt\,\ref{#1}, S.\,\pageref{#1}}
\newrefformat{paket}{Paket~\ref{#1}, S.\,\pageref{#1}}
\newrefformat{modul}{Modul~\ref{#1}, S.\,\pageref{#1}}
\newrefformat{klasse}{Klasse~\ref{#1}, S.\,\pageref{#1}}
\newrefformat{fig}{Abb.\,\ref{#1}}
\newrefformat{tab}{Tab.\,\ref{#1}}
\newrefformat{ex}{Bsp.\,\ref{#1}, S.\,\pageref{#1}}
\newrefformat{example}{Bsp.\,\ref{#1}, S.\,\pageref{#1}}

\newcommand{\materialsammlung}{
    \website{http://ddi.uni-wuppertal.de/material/materialsammlung/index.html}
}

\newcommand{\includepdfpage}[2][page=1,scale=0.3]{
    \fbox{
        \includegraphics[#1]{Beispiele/#2}
    }
}

\setcnltx{
    name     = schule ,
    title    = schule ,
    version  = 0.8.1 ,
    date     = 2018-08-22 ,
    subtitle = {\LaTeX-Klassen und Pakete für den Einsatz im Bereich der Schule},
    info     = Paketdokumentation ,
    authors  =  {Johannes Pieper, Johannes Kuhaupt, Ludger Humbert, Andr\'e Hilbig, Adrian Salamon, Daniel Spittank} ,
    email    = schulepaket@zfsl.ham.nw.schule.de ,
    url     = http://ddi.uni-wuppertal.de/material/schulepaket.html ,
    abstract = {%
        Diese Zusammenstellung wird entwickelt, um Pakete und damit Befehle bereit zu stellen, die für den Textsatz von Dokumenten zur Unterrichtsvorbereitung für den (Informatik)Unterricht nützlich sind. Zur Zeit liegt der Schwerpunkt auf dem Informatikunterricht, eine Ergänzung für den Physikunterricht wird nach und nach eingearbeitet. Weitere Ergänzungen für andere Fächer werden gerne entgegen genommen.

        Diese Sammlung umfasst Pakete und Klassen zum Setzen von speziellen Dokumenten für Klausuren, Lernzielkontrollen, Unterrichtsbesuche, Arbeits-, Informations- und Lösungsblättern, sowie speziellen Elementen, wie Struktogramme, Syntax-, Sequenz-, Objekt- und Klassendiagramme.

        Ein besonderer Dank geht an Martin Weise für seine Hilfe bei der Übersetzung der Readme-Dateien und Zusammenfassung auf \ctan\footnote{\CTAN}\space ins Englische.
    } ,
    index-setup = { othercode=\footnotesize,level=\section},
    add-listings-options= {
        morekeywords={
                skaliereSequenzdiagramm, skaliereTikz, weva, uline, newthread, newthreadtwo, newinst, node, chainin, EuEn, draw, to, Titel, Autor, Datum, Klasse, Lerngruppe, EK, minisec, subsection, glqq, grqq, so, person, uuline, erwartung, newboolean, setboolean, hinweis, achtung, teilaufgabe, enquote, includegraphics, dotuline, Lprn, dashuline, uwave, xout, diastring, Lkr, Lkre, Lpr, GK, EuE, EKe, EKen, GKe, GKen, LK, LKe, LKen, SuS, SuSn, LuL, LuLn, KuK, Autor, Titel, Datum, Fach, Lerngruppe, feldLin, feldKar, feldMil, resetZeilenNr, anchormark, assign, while, ifthenelse, ifend, change, whileend, usym, usymH, usymW, blindtext, punktUebersicht, notenverteilung, flushleft, flushright, pmglyph, chb, textpmhg, aufgabeMC, luecke, textluecke, usetikzlibrary, pgfmathtruncatemacro, pgflinewidth, multirow,
        }
    },
}

% \includeonly{module}

\begin{document}

\part{Allgemeines}

\section{Allgemeines zum Paket}

\subsection{Wichtiger Hinweis zur neuen Version}
	Das Schule-Paket wurde vollständig überarbeitet. Diese
	Version enthält grundlegende, strukturelle Veränderungen.
	So wird unter anderem die Vielzahl an Dokumentenklassen 
	stark reduziert und die Konfiguration erfolgt nun über
	Paketoptionen.
	
	Dies führt zu großen Veränderungen der Schnittstelle. Die
	neue Version ist damit \textbf{nicht kompatibel} zu allen
	vorhergehenden Versionen. Es besteht allerdings ein
	\textbf{Kompatibilitätsmodus}, der automatisch für alle alten
	Dokumentenklassen aktiv ist. Alte Dokumente lassen sich somit
	weiterhin setzen, die Schnittstelle wird aber nicht
	weiterentwickelt. Bestehende Fehler werden in der alten
	Version nicht behoben. 

	Diese Änderungen ermöglichen die Lösung einiger bestehender
	Probleme (u.\,a. Quelltexte in Aufgaben und Lösungen).
	Zusätzlich wurde die Nutzung des Pakets vereinheitlicht und
	die Nutzung in anderen Dokumentenklassen ermöglicht, sodass
	etwa die Aufgabenumgebungen auch in Beamer-Präsentationen
	übernommen werden können. Klausuren unterstützen nun die
	automatische Erzeugung von Erwartungshorizonten.

	Eine weitere große Veränderung ist die Ausgliederung der
	ausbildungsrelevanten Teile (Unterrichtsbesuche,
	Stundenverläufe etc.) des Pakets. In der
	Vergangenheit hat sich gezeigt, dass sich die Anforderungen der
	verschiedenen, an der Lehrerbildung beteiligten Stellen
	stark voneinander unterscheiden.
	Daher werden die entsprechenden Funktionen des Pakets
	ausgegliedert, sodass sie einfach in eigenen Dokumenten
	genutzt werden können. Die bestehenden Vorlagen werden
	als eigenständige Klassen mitgeliefert.

\subsection{Manuelle Installation}
	Um die Pakete und Klassen nutzen zu können, gibt es drei
	Varianten. In der folgenden Beschreibung dieser Möglichkeiten
	wird von einer standardisierten \LaTeX-Installation
	ausgegangen -- weitere Hinweise können der Dokumentation der
	jeweiligen \TeX-Distribution entnommen werden:
		
\begin{description}
\item[Global]
	Für die globale/systemweite Installation der Pakete und
	Klassen	müssen diese in das globale \LaTeX-Verzeichnis der
	\TeX-Installation kopiert werden: unter Linux in der Regel
	\texttt{/usr/share/texmf/tex/latex/}. In diesem kann ein
	weiteres Verzeichnis wie z.\,B. \texttt{schule} angelegt
	werden, in das alle Dateien des Schulepakets
	kopiert werden. 
	
	Damit die Quellen anschließend dem System bekannt sind, muss
	der Cache von \LaTeX{} neu aufgebaut werden. Bei den meisten
	Linux-Installationen geschieht dieses durch den Aufruf von
	\texttt{texhash}.

\item[Benutzer]
	Damit ein Nutzer auf die Quellen zugreifen kann,
	müssen diese im Benutzerverzeichnis (Home directory)
	abgelegt werden. Dies geschieht durch das Kopieren der
	Pakete und Klassen in das Verzeichnis
	\texttt{texmf/tex/latex/} im Benutzerverzeichnis, das ggf.
	erst angelegt werden muss. Auch hier sollte -- wie bei der
	globalen Installation -- ein eigenes Unterverzeichnis
	angelegt werden.

\item[Lokal]
	Um die Klassen und Pakete ohne weitere Installation
	nutzen zu können, ist es darüber hinaus möglich, die
	benötigten Dateien in das Verzeichnis zu kopieren, in dem
	die Datei liegt, die übersetzt werden soll. Dies ist jedoch
	aufgrund des Umfangs des Schulepakets weniger empfehlenswert.
\end{description}

\subsubsection{Voraussetzungen}
Ein Grund für die Nutzung des Schule-Pakets und der damit
verbundenen speziellen Klassen und Pakete liegt
darin, viele der häufig  benötigten Pakete zusammen zu fassen.
Daher müssen diese für die Benutzung vorhanden sein. Die meisten
sind Standardpakete, die mit jeder normalen Installation
mitgeliefert sind.
Es folgt eine Aufstellung der Voraussetzungen
für das Paket \pkg{schule} und die vorhandenen Module. 
Mit einem Stern (*) markierte Pakete sind im Paket 
\pkg{schule} bereits enthalten:
\begin{multicols}{4}
	\begin{smallitemize}
		\item \pkg{amsmath}
		\item \pkg{babel}
		\item \pkg{environ}
		\item \pkg{fontenc}
		\item \pkg{forarray}
		\item \pkg{graphicx}
		\item \pkg{hyperref}
		\item \pkg{ifthen}
		\item \pkg{inputenc}
		\item \pkg{pgfopts}
		\item \pkg{schulealt} *
		\item \pkg{tikz}
		\item \pkg{xcolor}
		\item \pkg{xparse}
		\item \pkg{xstring}
		\item \pkg{zref-totpages}
	\end{smallitemize}
\end{multicols}

Folgende Pakete werden zusätzlich für das Fach \enquote{Informatik} benötigt: 
\begin{multicols}{4}
	\begin{smallitemize}
		\item \pkg{listings}
		\item \pkg{pgf-umlcd}
		\item \pkg{pgf-umlsd}
		\item \pkg{relaycircuit} *
		\item \pkg{struktex}
		\item \pkg{syntaxdi} *
	\end{smallitemize}
\end{multicols}

Folgende TikZ-Bibliotheken werden für das Fach \enquote{Informatik} benötigt:
\begin{smallitemize}
	\item er
\end{smallitemize}

Folgende Pakete werden zusätzlich für das Fach \enquote{Physik} benötigt:
\begin{multicols}{4}
	\begin{smallitemize}
	\item \pkg{circuittikz}
	\item \pkg{units}
	\item \pkg{mhchem}
	\end{smallitemize}
\end{multicols}

Folgende Pakete werden zusätzlich für das Fach \enquote{Geschichte} benötigt:
\begin{multicols}{4}
	\begin{smallitemize}
		\item \pkg{uni-wtal-ger}
		\item \pkg{marginnote}
	\end{smallitemize}
\end{multicols}

Folgende Pakete werden zusätzlich für das Modul \enquote{Aufgaben} benötigt: 
\begin{multicols}{4}
	\begin{smallitemize}
		\item \pkg{xsim}
		\item \pkg{utfsym} *
	\end{smallitemize}
\end{multicols}

Folgende Pakete werden zusätzlich für das Modul \enquote{Format} benötigt: 
\begin{multicols}{4}
	\begin{smallitemize}
		\item \pkg{amssymb}
		\item \pkg{array}
		\item \pkg{colortbl}
		\item \pkg{csquotes}
		\item \pkg{ctable}
		\item \pkg{enumitem}
		\item \pkg{eurosym}
		\item \pkg{graphicx}
		\item \pkg{longtable}
		\item \pkg{multicol}
		\item \pkg{multirow}
		\item \pkg{setspace}
		\item \pkg{tikz}
		\item \pkg{ulem}
		\item \pkg{utfsym} *
		\item \pkg{xspace}
	\end{smallitemize}
\end{multicols}

Folgende Pakete werden zusätzlich für das Modul \enquote{Symbole} benötigt: 
\begin{multicols}{4}
	\begin{smallitemize}
		\item \pkg{utfsym} *
	\end{smallitemize}
\end{multicols}

Folgende Pakete werden zusätzlich für das Modul \enquote{Texte} benötigt: 
\begin{multicols}{4}
	\begin{smallitemize}
		\item \pkg{lineno}
		\item \pkg{multicol}
	\end{smallitemize}
\end{multicols}

Folgende TikZ-Bibliotheken werden für das Zusatzpaket \pkg{syntaxdi} benötigt:
\begin{multicols}{4}
	\begin{smallitemize}
		\item arrows
		\item chains
		\item scopes
		\item shadows
		\item shapes.misc
	\end{smallitemize}
\end{multicols}

Folgende TikZ-Bibliotheken werden für das Zusatzpaket \pkg{relaycircuit} benötigt:
\begin{multicols}{4}
	\begin{smallitemize}
		\item arrows
		\item scopes
		\item shadows
		\item shapes.misc
	\end{smallitemize}
\end{multicols}

\subsection{Begriffsklärungen}
	\label{sec:begriffe}
	\begin{description}
		\item[Zusatzpaket] 
			Das Paket \pkg{schule} liefert einige \LaTeX-Pakete mit,
			die für das Paket entwickelt wurden, aber von diesem
			unabhängig nutzbar sind.
		
			Diese Pakete werden im Folgenden als Zusatzpaket
			bezeichnet.
		
		\item[Modul]
			Im Gegensatz zu einem Zusatzpaket ist ein Modul
			enger mit dem Hauptpaket verzahnt. Es lässt sich nicht
			unabhängig von diesem nutzen.
		
			Module bestehen aus einer oder mehreren
			\LaTeX-Quelldateien, die in das Paket eingebunden werden.
			
			Siehe auch die Beschreibung in der 
			Entwicklungsdokumentation im 
			\prettyref{sec:devmoduleModul}.
			
		\item[Fachmodul]
			Ein Fachmodul ist ähnlich aufgebaut wie ein	normales 
			Modul für das Schulepaket, wird allerdings für 
			fachspezifische Erweiterungen genutzt und erfüllt somit 
			einen anderen Zweck.
			
			Siehe auch die Beschreibung in der 
			Entwicklungsdokumentation im 
			\prettyref{sec:devmoduleFachmodul}.
						
		\item[Dokumenttyp]
			Ein Dokumenttyp ist ähnlich aufgebaut wie ein normales 
			Modul für das Schulepaket, wird allerdings für 
			typspezifische Erweiterungen genutzt und erfüllt somit 
			einen anderen Zweck.
			
			Siehe auch die Beschreibung in der 
			Entwicklungsdokumentation im 
			\prettyref{sec:devDokumenttyp}.
					
	\end{description}
	
\subsection{Arten der Nutzung}
	\subsubsection{Nutzung für Dokumente}
		Wenn zumindest ein \option{typ} in den Paketoptionen angegeben
		wird, werden viele Module und mit diesen auch viele externe
		Pakete geladen und konfiguriert, von denen einige auch die
		grundlegende Struktur der zu setzenden Dokumente verändern.
		
		Außerdem werden Entscheidungen für das Aussehen der Dokumente 
		getroffen. Man hat hier noch viele Freiheiten, ist jedoch auf
		die grundlegenden Vorgaben des Schule-Pakets festgelegt.
		
		Dies kann auch zu Inkompatibilitäten mit bestimmten
		Dokumentenklassen oder externen Paketen führen, \zB\space
		könnten Option-Clashes auftreten.
	
	\subsubsection{Eingebettete Nutzung}
		Es trat immer wieder der Wunsch auf, dass Funktionen aus dem
		Schulepaket auch in anderen Dokumenten oder gar in
		Dokumentenklassen oder anderen Paketen nutzen zu können.
		
		Das war aus den oben genannten Gründen schwierig. Inzwischen
		ist dies möglich, in dem man beim Laden des Pakets die Option
		\keyis{typ}{ohne} angibt.
		
		Damit wird das Paket in einen \enquote{minimalinvasiven} Modus
		geschaltet, der nur die nötigsten Module lädt und so wenig
		Vorgaben macht wie möglich.
		
		Weitere Module können dann natürlich geladen werden.
	
	\subsubsection{Nutzung über die Dokumentenklassen}
		Die Nutzungsvariante mit den wenigsten Freiheiten ist die
		über eine der Dokumentenklassen. Anpassungen sind hier nur
		sehr eingeschränkt möglich und es werden sehr viele Vorgaben
		gemacht. Sie ist allerdings gleichzeitig die Variante, bei
		der man am wenigsten konfigurieren und eigene Einstellungen
		vornehmen muss. Siehe auch \prettyref{subsubsec:paketoptionen}.
		
\subsection{Kompilieren der Dokumente}
	Das Schulepaket ist für die Nutzung von \texttt{pdflatex} 
	optimiert und wurde nur damit getestet.  
	
	Aufgrund des komplexen Aufbaus kann es besonders bei der Nutzung 
	des Moduls \module{Aufgaben} notwendig sein, mindestens zwei
	Läufe von \texttt{pdflatex} durchzuführen. Dies liegt daran, das 
	eine Menge Zwischendateien mit Punkten und anderen Metadaten zu 
	den Aufgaben erstellt werden müssen und dann damit Berechnungen
	durchgeführt werden.
	
	Es kann deshalb passieren, dass nach der Änderung der Anzahl der 
	Aufgaben der erste Durchlauf mit sehr vielen Fehlern fehlschlägt.


\part{Module}
    \label{sec:module}
    \section{Nutzung der Module}
        \subsection{Standardmodule}
        Standardmäßig wird davon ausgegangen, dass ein Dokument mit Schulkontext gesetzt werden soll (Arbeitsblatt, Klausur, etc). Dann lädt das Schule-Paket die Module \module{Metadaten}, \module{Format} und \module{Aufgaben}.

        Wird das Paket eingebettet verwendet, also mit der Paketoption \keyis{typ}{ohne} geladen, lädt das Schule-Paket nur die Module \module{Metadaten} und \module{Format}.

        Wird ein nicht definierter Typ angegeben, wird ein Arbeitsblatt gesetzt und der angegebene Typ wird als Bezeichner verwendet.

        \subsection{Laden weiterer Module}
        \begin{options}
            \keyval{module}{Modul1,Modul2,\dots}\Default Weitere Module können geladen werden, indem Sie der Paketoption \option{module} als kommaseparierte Liste übergeben werden.
        \end{options}

    \section{Aufgaben}
\label{modul:aufgaben}
Das Modul \module{Aufgaben} ist das umfangreichste Modul des Schule-Pakets. Es umfasst alles, was zum Setzen von verschiedenen Arbeitsblättern, Klausuren, Klassenarbeiten, Lernzielkontrollen\usw. notwendig ist.

Im Kern baut das Modul auf dem Paket \pkg{xsim} auf, sodass alle Funktionen dieses Pakets nutzbar sind.

Die vom Schulepaket gemachten Ergänzungen sind voll kompatibel zu \pkg{xsim}, so werden die Hinweise etwa in den Eigenschaften der Aufgaben gespeichert.

\subsection{Aufgaben}

\subsubsection{Befehle}
\begin{commands}
    \command{setzeSymbol}[\marg{Symbol}] kann nur innerhalb der Aufgabenumgebung genutzt werden und stellt der jeweiligen Aufgabe ein Symbol voran. Dies kann etwa genutzt werden, um die Arbeitsform oder bestimmte Aufgabentypen zu kennzeichnen.

    Eine Kombination mit dem Modul \module{Symbole} bietet sich an. So könnte etwa zur Kennzeichnung von Höraufgaben \verbcode|\setzeSymbol{\symOhr}| genutzt werden.

    Alternativ lässt sich das Symbol auch als Eigenschaft der Aufgabe direkt setzen. Dieses erfolgt z.\,B. durch \verbcode|\begin{aufgabe}[symbol=\symOhr]|.

    \command{punkteAufgabe} liefert die Punkte der aktuellen Aufgabe inkl. der Bezeichnung.

    \command{punkteTotal} liefert die Gesamtpunktezahl aller Aufgaben inkl. der Bezeichnung.

    \command{punktuebersicht}[\oarg{Darstellungsart}]\Default{kurz} setzt eine Übersichtstabelle über die in allen Aufgaben erreichbaren Punkte und Zusatzpunkte, sowie einer Leerzeile für die erreichten Punkte. Als optionalen Parameter kann zwischen verschiedenen Darstellungen gewählt werden. Alternativ zur Standardoptionen ist \verbcode{default}.
\end{commands}

\subsubsection{Umgebungen}
\begin{environments}
    \environment{aufgabe\sarg}[\oarg{Optionen}]\envidx{aufgabe}%
        setzt eine Aufgabe. Alle Aufgaben werden automatisch durchnummeriert. Wird der optionale Stern angegeben, wird die Aufgabe als Zusatzaufgabe gesetzt.

        Bei den optionales Argument können alle von \pkg{xsim} bereitgestellen Optionen angegeben werden. Dazu gehören unter anderem folgende:
        \begin{options}
            \keyval-{points}{Punkte} legt die Punkte der Aufgabe fest.
            \keyval-{bonus-points}{Zusatzaufgabe} legt die Punkte der Aufgabe fest.
            \keyval{subtitle}{Titel} setzt den Title der Aufgabe.
        \end{options}
\end{environments}

\subsubsection{Aufgabentemplates}
    Die Darstellung der Aufgaben erfolgt auf der Grundlage verschiedener Templates. Das Paket \pkg{schule} liefert dabei folgende Templates mit, die in darunter dargestellt sind.
    \begin{itemize}
        \item \verbcode|schule-binnen|
        \item \verbcode|schule-default|
        \item \verbcode|schule-keinenummer|
        \item \verbcode|schule-keinepunkte|
        \item \verbcode|schule-keintitel|
        \item \verbcode|schule-randpunkte|
        \item \verbcode|schule-tcolorbox|
    \end{itemize}

    \includepdfpage[page=1, scale=0.6, trim={3cm, 16cm, 1cm, 2.5cm}, clip] {beispiel-aufgabentemplates}

    \begin{commands}
        \command{setzeAufgabentemplate}[\marg{Templatename}] setzt das Template mit dem die folgenden Aufgaben dargestellt werden.
    \end{commands}

\subsection{Teilaufgaben}

\subsubsection{Befehle}
\begin{commands}
    \command{teilaufgabe}[\oarg{Punkte}] leitet innerhalb einer \env{teilaufgaben}-Umgebung eine Teilaufgabe ein. Teilaufgaben werden mit den Kleinbuchstaben von \textit{a} bis \textit{z} gekennzeichnet.

    Über den optionalen Parameter kann eine Punktzahl angegeben werden.

    \command{teilaufgabeOhneLoesung} Dient als Platzhalter bei Teilaufgaben, bei den keine Lösung angegeben wird. Die entsprechende Nummer wird bei den Lösungen nicht aufgeführt und die folgende Teilaufgaben bekommt den nächsten Buchstaben, so dass es übereinstimmt mit der Aufgabenstellung.
\end{commands}

\subsubsection{Umgebung}
\begin{environments}
    \environment{teilaufgaben} bietet die Möglichkeit, eine Aufgabe in verschiedene Teilaufgaben zu unterteilen.
    \begin{sourcecode}[gobble=8]
        \begin{aufgabe}
            Inhalt...
            \begin{teilaufgaben}
                \teilaufgabe Erstens.
                \teilaufgabe[5] Zweitens.
            \end{teilaufgaben}
        \end{aufgabe}
    \end{sourcecode}
    \hinweis{Teilaufgaben können auch in einer \env{loesung}- und \env{bearbeitungshinweis}-Umgebung verwendet werden!}
\end{environments}

\subsection{Lösungen}

\subsubsection{Paketoptionen}\label{subsubsec:paketoptionen}
\begin{options}
    \keychoice{loesungen}{folgend, keine, seite}\Default{keine}
        legt fest, ob die Lösungen direkt hinter die Aufgaben, als eigenständige Lösungsseite oder gar nicht gesetzt werden.

        \achtung{Die Option \keyis{loesungen}{seite} ist nur für eigenständige Dokumente, \zB\space mit der Dokumentenklasse \cls{scrartcl} gedacht. Sie greift tief in den Übersetzungsprozess ein und ist geeignet Fehler im Zusammenspiel mit anderen Paketen zu provozieren.}
\end{options}

\begin{commands}
    \command{printsolutions} Wenn keine der Standardoptionen genutzt wird, kann der Befehl zur Ausgabe der Lösungen aus dem \pkg{exsheets}-Paket genutzt werden.
\end{commands}

\subsubsection{Umgebungen}
\begin{environments}
    \environment{loesung\sarg}[] wird innerhalb oder direkt hinter einer \env{aufgabe} verwendet, um eine Lösung dazu anzugeben. Die Inhalte dieser Umgebung werden standardmäßig nicht gesetzt, sondern durch die entsprechende Konfiguration von \option{loesungen} an der entsprechenden Stelle gesetzt. Wichtig ist, dass bei Zusatzaufgaben auch bei der Lösung der Stern gesetzt werden muss.

    \begin{sourcecode}[gobble=8]
        \begin{aufgabe}
            Inhalt...
            \begin{teilaufgaben}
                \teilaufgabe Erstens.
                \teilaufgabe[5] Zweitens.
            \end{teilaufgaben}
            \begin{loesung}
                \begin{teilaufgaben}
                    \teilaufgabe Erste Lösung.
                    \teilaufgabe Zweite Lösung.
                \end{teilaufgaben}
            \end{loesung}
        \end{aufgabe}
    \end{sourcecode}
\end{environments}

\subsection{Lückentexte}

\subsubsection{Befehle}
\begin{commands}
    \command{luecke}[\oarg{Optionen für blank}\marg{Länge}]
        Setzt eine Lücke mit der angegebenen Länge. Der Befehl nutzt dazu den \cs{blank}-Befehl aus \pkg{xsim}. Mit dem optionalen Parameter können zusätzliche Optionen an diesen weitergereicht werden, \zB\space kann mit \texttt{style = \choices{line,wave,dline,dotted,dashed}} der Stil der Unterstreichung festgelegt werden.
    \command{textluecke}[\oarg{Optionen für blank}\marg{Text}]
        Setzt eine Lücke für den angegebenen Text, die Länge wird durch den angegebenen Text vorgegeben. Standardmäßig wird als Korrekturfaktor für das handschriftliche Ausfüllen $2$ genutzt.

        Der Befehl nutzt dazu den \cs{blank}-Befehl aus \pkg{xsim}. Mit dem optionalen Parameter können zusätzliche Optionen an diesen weitergereicht werden, \zB\space kann mit \verbcode|style=\dashuline{#1}| eine unterstrichelte Linie gesetzt werden. Mit \texttt{scale = 3} ließe sich der Korrekturfaktor auf $3$ anpassen. Wird als Option \texttt{nichts} angegeben, so wird die Lücke ohne Inhalt und Weite eingesetzt.

        Innerhalb von Lösungsumgebungen wird der Text in die Lücke eingesetzt.

        %%%%% Mit xsim nicht möglich
%     \command{aufgabeLueckentext}[\oarg{Punkte}\marg{Lückentext} \marg{Extras}\oarg{Symbol}\oarg{Optionen für die Aufgabenumgebung}]
%         Um Lückentexte mit Lösungen anzugeben, müsste der gesamte Text mit den Lücken zweimal im Dokument stehen: einmal in der Aufgaben- und einmal in der Lösungsumgebung. Um dies einfacher zu gestalten, wurde als Abkürzung dieser Befehl eingeführt, der intern eine entsprechende Aufgaben- und Lösungsumgebung erzeugt.
%
%         Der Parameter \enquote{Extras} dient dem Anhängen von Inhalten an den Aufgabentext und kann somit etwa für Erwartungen und Hinweise genutzt werden.

% \begin{sourcecode}
% \aufgabeLueckentext[4]{
%     Das ist ein total verrückter \textluecke{Lückentext}. Für alle
%     \textluecke{Menschen}, die \textluecke{Lücken} lieben.
% }{
%     \begin{erwartungen}
%         \erwartung{füllt die Lücken richtig aus.}{4}
%     \end{erwartungen}
% }[\symBleistift][name=Lückentext]
% \end{sourcecode}
\end{commands}

\subsection{Multiple-Choice}
Zwar ist es über das \module{Format}-Modul möglich, einzelne Kästchen zum Ankreuzen zu setzen. In der Regel sollten allerdings echte Multiple-Choice-Aufgaben vorgezogen werden, da diese besser formatiert werden können.% und sich auch direkt Lösungen angeben lassen.

\subsubsection{Befehle}
\begin{commands}
    \command{choice}[\oarg{richtig}!] Innerhalb einer \env{mcumgebung} können mit \cs{choice} die einzelnen Wahlmöglichkeiten angegeben werden.

        Falls im optionalen Parameter \cs{mcrichtig} steht, wird die Wahlmöglichkeit als richtig markiert und in Lösungsumgebungen entsprechend gesetzt.

        Ein optionales Ausrufezeichen hinter dem Befehl sorgt dafür, dass die Wahlmöglichkeit einzeln gesetzt und somit hervorgehoben wird.
    \command{mcrichtig} markiert innerhalb einer \env{mcumgebung} eine Wahlmöglichkeit als richtig.

%     %%%% Mit xsim nicht möglich
%     \command{aufgabeMC}[\oarg{Punkte}\marg{Auswahlmöglichkeiten} \oarg{Spaltenzahl}\marg{Extras}\oarg{Symbol}
%         \oarg{Optionen für die Aufgabenumgebung}]
%         Genau wie bei Lückentexten müsste die jeweilige \env{mcumgebung}  zweimal im Dokument stehen, um automatisch Lösungen zu generieren: einmal in der Aufgaben- und einmal in der Lösungsumgebung. Um dies einfacher zu gestalten, wurde als Abkürzung dieser Befehl eingeführt, der intern eine entsprechende Aufgaben- und Lösungsumgebung erzeugt.
%
%         Der Parameter \enquote{Extras} dient dem Anhängen von Inhalten an den Aufgabentext und kann somit etwa für Erwartungen und Hinweise genutzt werden.
% \begin{sourcecode}
%   \aufgabeMC[4]{
%     \choice Erstens
%     \choice Zweitens
%     \choice Drittens
%     \choice[\mcrichtig] Viertens
%     \choice! Fünftens
%     \choice[\mcrichtig] Sechstens
%     \choice Siebtens
%     \choice[\mcrichtig] Achtens
%   }[4]{
%     \begin{erwartungen}
%         \erwartung{kreuzt alles richtig an.}{4}
%     \end{erwartungen}
%   }[\symHaken]
% \end{sourcecode}
\end{commands}

\subsection{Umgebungen}
    \begin{environments}
        \environment{mcumgebung}[\darg{Spaltenzahl}] ermöglicht es Multiple-Choice-Aufgaben zu setzen.
    \end{environments}

\subsection{Bearbeitungshinweise}
Die Bearbeitungshinweise sind dazu gedacht, dass man den Lernenden Tipps zu den Aufgaben mitgibt. Dieses ist z.\,B. bei der Bearbeitung von Leitprogrammen (siehe \ref{typ:Leitprogramm}) der Fall. Dabei ist es angedacht, diese nicht direkt bei den Aufgaben stehen zu haben, sondern an einer anderen Stelle, damit sie nur bei Bedarf genutzt werden.

\subsubsection{Umgebungen}
\begin{environments}
    \environment{bearbeitungshinweis} erlaubt es, zu einzelnen Aufgaben Hinweise anzugeben. Der Hinweis kann dabei fast beliebigen \LaTeX-Code enthalten. Verbatim-Elemente, wie z.\,B. die Verwendung von Quellcode machen an Probleme. Es kann aber \cs{lstinputlisting} genutzt werden.
\end{environments}

\subsubsection{Befehle}
\begin{commands}
    \command{bearbeitungshinweisZuAufgabe}[\oarg{Aufgabentyp}\marg{AufgabenId}]\Default{aufgabe} Setzt die Bearbeitungshinweise für die angegebene Aufgabe. Die ID ist dabei fortlaufend über alle Aufgabentypen. Der optionale Parameter erlaubt es auch für andere Aufgabentypen wie der Zusatzaufgabe mit \verbcode|aufgabe*| den Hinweis direkt auszugeben. Wird als AufgabenId nichts angegeben, so wird die aktuelle Aufgabe genommen.
    \command{bearbeitungshinweisliste} Setzt die Bearbeitungshinweise zu allen Aufgaben als Liste.
\end{commands}
    \section{Bewertung}
\label{modul:bewertung}
Das Modul \module{Bewertung} ergänzt das Modul \module{Aufgaben} um die Möglichkeiten eines Erwartungshorizonts und der Berechnung der Notenverteilung. Die Punkteangaben beim Erwartungshorizont werden auch als Punkte für die Aufgaben herangezogen und müssen so nicht doppelt angegeben werden.

\achtung{Soll in einem Dokument ein Erwartungshorizont gesetzt werden, müssen \uline{alle} Aufgaben Erwartungen enthalten!}

\subsection{Paketoptionen}
\begin{options}
    \opt{erwartungshorizontAnzeigen} hängt den Erwartungshorizont im gewählten Stil automatisch an das Dokument an, setzt vorher die Seitennummerierung und die Dokumentbezeichnung in der Kopfzeile zurück.

    Unter dem Erwartungshorizont wird automatische eine Notenverteilung gesetzt.

    \achtung{Diese Option ist nur für eigenständige Dokumente, \zB\space mit der Dokumentenklasse \cls{scrartcl} gedacht. Sie greift tief in den Übersetzungsprozess ein und ist geeignet Fehler im Zusammenspiel mit anderen Paketen zu provozieren.}

    \keychoice{erwartungshorizontStil}{einzeltabellen,simpel,standard}
        \Default{standard} legt den Stil des Erwartungshorizonts fest.
        Bisher gibt es drei verschiedene Stile:
        \begin{description}
            \item[\keyis-{erwartungshorizontStil}{einzel}\space] setzt für jede Aufgabe eine eigene Überschrift und darunter eine Tabelle mit den einzelnen Erwartungen. Unter die Erwartungen aller Aufgaben wird mit \cs{punktuebersicht} eine Übersicht über die erreichten Punkte gesetzt.

                \includepdfpage[page=3, scale=0.3, trim={3cm, 12.5cm, 3cm, 1.5cm}, clip] {minimal-kl-et}

            \item[\keyis-{erwartungshorizontStil}{simpel}\space] setzt einen Bewertungsbogen ohne Punkte mit drei Smiley-Feldern zum Ankreuzen. Die Notenverteilung wird hier ebenfalls nicht gesetzt.

                \includepdfpage[page=3, scale=0.3, trim={3cm, 22cm, 3cm, 1.5cm}, clip]{minimal-ka}

            \item[\keyis-{erwartungshorizontStil}{standard}\space] setzt einen klassischen Erwartungshorizont in einer zusammenhängenden Tabelle. Die Umgebung ist \env{longtable}, die Tabelle bricht also bei längeren Erwartungshorizonten auf die nächste Seite um.

                \includepdfpage[page=3, scale=0.3, trim={3cm, 18cm, 3cm, 1.5cm}, clip]{minimal-kl}
        \end{description}

    \opt{kmkPunkte} schaltet alle benotungsrelevanten Funktionen vom normalen Notensystem (\textit{ungenügend} bis \textit{sehr gut}) auf KMK-Notenpunkte ($0$ bis $15$) um.

    \keyval{notenschema}{15=.95,\dots} gibt ein Notenschema für die Berechnung der Notenverteilung an. Es muss eine Liste mit der Zuordnung von Notenpunkten zu Prozentwerten übergeben werden. Die Prozentwerte geben dabei jeweils die untere Grenze für die jeweilige Note an.

    Das Standardnotenschema ist \texttt{15 = .95, 14 = .9, 13 = .85, 12 = .8, 11 = .75, 10 = .7, 9 = .65, 8 = .6, 7 = .55, 6 = .5, 5 = .45, 4 = .39, 3 = .33, 2 = .27, 1 = .2}
\end{options}

\subsubsection{Umgebungen}
\begin{environments}
    \environment{erwartungen} erlaubt es, zu einzelnen Aufgaben Erwartungen anzugeben. Die einzelnen Erwartungen werden dabei mit dem Makro \cs{erwartung} angegeben.
\end{environments}

\subsubsection{Befehle}
\begin{commands}
    \command{erwartung}[\marg{Erwartung}\marg{Punkte}\oarg{Zusatzpunkte}] definiert eine einzelne Erwartung innerhalb der Umgebung \env{erwartungen}. Der Parameter kann beliebigen \LaTeX-Code enthalten bis auf Verbatim-Elemente. Des weiteren werden die Punkte für diese Erwartung als Parameter erwartet. Als optionalen Parameter können Zusatzpunkte angegeben werden.
    \command{erwartungshorizont} setzt den Erwartungshorizont im gewählten Stil, falls die automatische Erzeugung über die Paketoption \option{erwartungshorizontAnzeigen} nicht genutzt wird.
    \command{notenverteilung} setzt die Notenverteilung, falls die automatische Erzeugung über den Erwartungshorizont nicht genutzt wird. Die Verteilung wird über die Gesamtpunkte aller Aufgaben unter Berücksichtigung des gewählten Notenschemas ermittelt.
\end{commands}

    \section{Format}
\label{modul:format}
Dieses Modul definiert einige grundlegende Paketoptionen für die Formatierung von Dokumenten und stellt passende Makros bereit. Außerdem bindet es das Paket \pkg{ulem} für verschiedene Textformatierungen ein.

\subsection{Formatierungen}
    Über verschiedene Paketoptionen kann das Aussehen der vom Schule-Paket erstellten Dokumente beeinflusst werden. Es sind zudem einige Makros vorhanden, die häufig verwendete Formatierungen und Sonderzeichen bereitstellen.

\subsubsection{Paketoptionen}
\begin{options}
    \opt{farbig} aktiviert die farbige Darstellung.
    \opt{sprache} fügt eine Liste von CSV Sprachen dem Babelpaket hinzu. \texttt{ngerman} ist immer geladen (als Hauptsprache)
\end{options}

\subsubsection{Befehle}
\begin{commands}
    \command{achtung}[\marg{Text}] Der Befehl \cs{achtung} stellt den angegebenen Text mit einem vorangestellte Warnsymbol und einem fettgedruckten \enquote{Achtung:} dar.
\begin{sidebyside}[gobble=4]
    \achtung{Dies ist ein Beispiel.}
\end{sidebyside}

    \command{chb}[\sarg] setzt eine ankreuzbares Kästchen, der optionale Stern markiert dieses.
\begin{sidebyside}[gobble=4]
    \chb \chb*
\end{sidebyside}

    \command{dashuline}[\marg{Text}] Der Befehl \cs{dashuline} stellt den angegebenen Text unterstrichelt dar.
\begin{sidebyside}[gobble=4]
    \dashuline{Dies ist ein Beispiel.}
\end{sidebyside}

    \command{dotuline}[\marg{Text}] Der Befehl \cs{dotuline} stellt den angegebenen Text unterpunktet dar.
\begin{sidebyside}[gobble=4]
    \dotuline{Dies ist ein Beispiel.}
\end{sidebyside}

    \command{hinweis}[\marg{Text}] Der Befehl \cs{hinweis} stellt den angegebenen Text mit einem vorangestellte Warnsymbol und einem fettgedruckten \enquote{Hinweis:} dar.
\begin{sidebyside}[gobble=4]
    \hinweis{Dies ist ein Beispiel.}
\end{sidebyside}

    \command{person}[\marg{Name}] Der Name einer Person wird mit dem Befehl \cs{person}\marg{Name} hervorgehoben.
\begin{sidebyside}[gobble=4]
    \person{Einstein}
\end{sidebyside}

    \command{so}[\marg{Text}] Der Befehl \cs{so} stellt den angegebenen Text durchgestrichen dar und ermöglicht es so in Wertetabellen bzw. Schreibtischtests einzelne Werte durchzustreichen.
\begin{sidebyside}[gobble=4]
    \so{Dies ist ein Beispiel.}
\end{sidebyside}

    \command{uline}[\marg{Text}] Der Befehl \cs{uline} stellt den angegebenen Text unterstrichen dar.
\begin{sidebyside}[gobble=4]
    \uline{Dies ist ein Beispiel.}
\end{sidebyside}

    \command{uuline}[\marg{Text}] Der Befehl \cs{uuline} stellt den angegebenen Text doppelt unterstrichen dar.
\begin{sidebyside}[gobble=4]
    \uuline{Dies ist ein Beispiel.}
\end{sidebyside}

    \command{uwave}[\marg{Text}] Der Befehl \cs{uwave} stellt den angegebenen Text unterschlängelt dar.
\begin{sidebyside}[gobble=4]
    \uwave{Dies ist ein Beispiel.}
\end{sidebyside}

    \command{xout}[\marg{Text}] Der Befehl \cs{xout} stellt den angegebenen Text durchgekreuzt dar.
\begin{sidebyside}[gobble=4]
    \xout{Dies ist ein Beispiel.}
\end{sidebyside}

\end{commands}

\subsection{Kopf- Fußzeilen}
    Das Modul stellt einige Standardformatierungen für Kopf- und Fußzeilen bereit, die in den Dokumenttypen verwendet werden.

\subsubsection{Paketoptionen}
\begin{options}
    \opt{datumAnzeigen} aktiviert die Darstellung des Datums in der Kopfzeile.
    \opt{namensfeldAnzeigen} aktiviert die Darstellung eines Namensfelds in der Kopfzeile.
\end{options}

\subsubsection{Befehle}
\begin{commands}
    \command{schule@kopfUmbruch} setzt einen Umbruch, wenn die Kopfzeile durch eine gesetzte Option mehrzeilig wird. Kann in Kopfzeilen verwendet werden, um sie gleichmäßig auszurichten.
    \command{schule@kopfInnen} setzt eine Kopfzeile mit der vollständigen Lerngruppenbezeichnung (Fach, Lerngruppe) und je nach Paketoption einem Namensfeld in der zweiten Zeile. Ein etwaiger Umbruch der anderen Kopfzeilenfelder wird berücksichtigt.
    \command{schule@kopfMitte} setzt eine Kopfzeile mit dem Titel des Dokuments. Ein etwaiger Umbruch der anderen Kopfzeilenfelder wird berücksichtigt.
    \command{schule@kopfAussen} setzt eine Kopfzeile mit dem gegebenen Parameter, üblicherweise dem Dokumenttypbezeichner und je nach Paketoption einem Datumsfeld in der zweiten Zeile. Ein etwaiger Umbruch der anderen Kopfzeilenfelder wird berücksichtigt.
\end{commands}

\subsection{Seitenzahlen}
    Die Darstellung von Seitenzahlen in Dokumenten kann ebenfalls beeinflusst werden.

\subsubsection{Paketoptionen}
\begin{options}
    \keychoice{seitenzahlen}{auto,autoGesamt,immer,immerGesamt,keine} \Default{autoGesamt}
        legt die Art der Darstellung von Seitenzahlen fest. Die verschiedenen Varianten sind davon abhängig, ob es sich um ein einseitiges oder mehrseitiges Dokument handelt:

        \begin{tabular}{lcc}
            \toprule
            \textbf{Variante}& \textbf{einseitig} & \textbf{mehrseitig} \\
            \midrule
            auto &  & $1$ \\
            autoGesamt &  & $1$ von $n$ \\
            immer & $1$ & $1$ \\
            immerGesamt & $1$ von $1$ &  $1$ von $n$ \\
            keine &  &  \\
            \bottomrule
        \end{tabular}
\end{options}

\subsubsection{Befehle}
\begin{commands}
     \command{Seitenzahlen} setzt die Seitenzahlen gemäß der über die Paketoption \option{seitenzahlen} gewählten Variante. Dieser Befehl kann in Kopf- oder Fußzeilen verwendet werden.
\end{commands}


\subsection{Strukturelemente}
    Verschiedene, häufig verwendete Strukturelemente gehören ebenfalls zum Umfang des Pakets. Darunter sind verschiedene Listen und Platzhalter.

\subsubsection{Umgebungen}
\begin{environments}%
    \environment{smalldescription} Die Listenumgebung \env{smalldescription} ist identisch zur \env*{description}-Standardumgebungen von \LaTeX, bis auf die Tatsache, dass zwischen den einzelnen Punkten der Abstand verkleinert wurde.
    \environment{smallenumerate} Die Listenumgebung \env{smallenumerate} ist identisch zur \env*{enumerate}-Standardumgebungen von \LaTeX, bis auf die Tatsache, dass zwischen den einzelnen Punkten der Abstand verkleinert wurde.
    \environment{smallitemize} Die Listenumgebung \env{smallitemize} ist identisch zur \env*{itemize}-Standardumgebungen von \LaTeX, bis auf die Tatsache, dass zwischen den einzelnen Punkten der Abstand verkleinert wurde.

    Der Unterschied wird besonders dann deutlich, wenn man die Umgebungen nebeneinander sieht:
\begin{example}[gobble=4]
    \begin{minipage}[t]{.4\textwidth}
        \texttt{itemize}-Umgebung:
        \begin{itemize}
            \item Punkt
            \item Punkt
            \item Punkt
        \end{itemize}
    \end{minipage}
    \begin{minipage}[t]{.4\textwidth}
        \texttt{smallitemize}-Umgebung:
        \begin{smallitemize}
            \item Punkt
            \item Punkt
            \item Punkt
        \end{smallitemize}
    \end{minipage}
\end{example}

\end{environments}


\subsection{Wörtliche Rede, Zitate und Anführungszeichen}
\subsubsection{Paketoptionen}
\begin{options}
    \keychoice{zitate}{guillemets,quotes,swiss} \Default{guillemets} Standardmäßig werden die deutschen \enquote{Möwchen}
    geladen. Über \verb|quotes| können doppelte \glqq Hochkommata\grqq\ (99-66) geladen werden. Die Darstellung
    von doppelten Hochkommata im ``Modus 66-99'' kann mittels \verb|swiss| erreicht werden.
\end{options}

\subsubsection{Befehle}
\begin{commands}
    \command{enquote}[\marg{Text}] Setzen von Passagen in typographische Anführungszeichen.
\begin{sidebyside}[gobble=4]
    \enquote{Beispiel}
\end{sidebyside}

    \command{diastring}[\marg{Zeichenkette}] Darstellung von Zeichenketten (strings) in Diagrammen usw.
\begin{sidebyside}[gobble=4]
    \diastring{Beispiel}
\end{sidebyside}
\end{commands}

    \textbf{Hinweis:} Teilweise kann es zu Fehlern kommen, wenn das Paket \pkg{csquotes} mit eigenen Optionen geladen wird.

    \section{Kuerzel}
\label{modul:kuerzel}
Das Modul \module{Kuerzel} stellt einige Makros bereit, die
Kurzschreibweisen für häufig verwendete Schreibweisen
bereitstellen. Sofern relevant, wird dabei die Schreibweise an
die gewählte Variante des Genderings, im Sinne einer
geschlechtergerechten Sprache angepasst.

\subsection{Paketoptionen}\begin{options}
	\keychoice{gendering}{binneni,fem,gap,mas,split,star}\Default{split}
	Standardmäßig wird die amtlich geforderte Schreibweise des Splittings (etwa Schülerinnen und Schüler) verwendet. 
	\begin{smallitemize}
		\item\hspace{1ex} Splitting: \keyis{gendering}{split}
	\end{smallitemize}
			
	Außerdem werden folgende Varianten unterstützt:
	\begin{smallitemize}
		\item\hspace{1ex} Gender-Gap: \keyis{gendering}{gap}
		\item\hspace{1ex} Gender-Star: \keyis{gendering}{star}
		\item\hspace{1ex} Binnen-I: \keyis{gendering}{binneni}
	\end{smallitemize}
			
	Für spezielle Fälle, kann auch die ausschließliche Nutzung einer Geschlechtsform erzwungen werden:
	\begin{smallitemize}
		\item \hspace{1ex}Generisches Femininum: \keyis{gendering}{fem}
		\item \hspace{1ex}Generisches Maskulinum: \keyis{gendering}{mas}
	\end{smallitemize}
	
\end{options}
\subsection{Befehle}\hfill

\begin{multicols}{2}
\begin{commands}
	\command{Lkr}
	Lehr\-kraft
	\command{Lkre}
	Lehr\-kräf\-te
	\command{Lpr}
	Lehr\-per\-son
	\command{Lprn}
	Lehr\-per\-so\-nen
	\command{EuE}
	El\-tern und Er\-zie\-hungs\-be\-recht\-ig\-te
	\command{EuEn}
	El\-tern und Er\-zie\-hungs\-be\-recht\-ig\-ten
	\command{EK}
	Er\-wei\-ter\-ungs\-kurs
	\command{EKe}
	Er\-wei\-ter\-ungs\-kurse
	\command{EKen}Er\-wei\-ter\-ungs\-kursen
	\command{GK}
	Grund\-kurs
	\command{GKe}
	Grund\-kurse
	\command{GKen}
	Grund\-kursen
	\command{LK}
	Leis\-tungs\-kurs
	\command{LKe}
	Leis\-tungs\-kurse
	\command{LKen}
	Leis\-tungs\-kursen
	\command{SuS}
{\tiny 			\begin{tabular}{ll}
		\toprule \textbf{Gendering} & \textbf{Ergebnis} \\ 
		\midrule binneni & SchülerInnen \\ 
		fem & Schülerinnen \\ 
		gap & Schüler\_innen \\ 
		mas & Schüler \\ 
		split & Schülerinnen und Schüler \\ 
		star & Schüler*innen \\ 
		\bottomrule 
	\end{tabular} }
	\command{SuSn}
	{\tiny \begin{tabular}{ll}
		\toprule \textbf{Gendering} & \textbf{Ergebnis} \\ 
		\midrule binneni & SchülerInnen \\ 
		fem & Schülerinnen \\ 
		gap & Schüler\_innen \\ 
		mas & Schülern \\ 
		split & Schülerinnen und Schülern \\ 
		star & Schüler*innen \\ 
		\bottomrule 
	\end{tabular} }
	
	\command{LuL}
	{\tiny \begin{tabular}{ll}
		\toprule \textbf{Gendering} & \textbf{Ergebnis} \\ 
		\midrule binneni & LehrerInnen \\ 
		fem & Lehrerinnen \\ 
		gap & Lehrer\_innen \\ 
		mas & Lehrer \\ 
		split & Lehrerinnen und Lehrer \\ 
		star & Lehrer*innen \\ 
		\bottomrule 
	\end{tabular} }
	
	\command{LuLn}
	{\tiny \begin{tabular}{ll}
		\toprule \textbf{Gendering} & \textbf{Ergebnis} \\ 
		\midrule binneni & LehrerInnen \\ 
		fem & Lehrerinnen \\ 
		gap & Lehrer\_innen \\ 
		mas & Lehrern \\ 
		split & Lehrerinnen und Lehrern \\ 
		star & Lehrer*innen \\ 
		\bottomrule 
	\end{tabular} }
	
	\command{KuK}
	{\tiny \begin{tabular}{ll}
		\toprule \textbf{Gendering} & \textbf{Ergebnis} \\ 
		\midrule binneni & KollegInnen \\ 
		fem & Kolleginnen \\ 
		gap & Kolleg\_innen \\ 
		mas & Kollegen \\ 
		split & Kolleginnen und Kollegen \\ 
		star & Kolleg*en*innen \\ 
		\bottomrule 
	\end{tabular} }
	
\end{commands}
\end{multicols}
    \section{Lizenzen}
\label{modul:lizenzen}
Dieses Modul definiert die Paketoptionen zum Festlegen der Lizenz des
Dokuments und bietet Makros zum Setzen des Lizenznamens und der
Lizenzsymbole an.

Außerdem werden automatisch passende XMP-Dateien in die PDF-Datei
eingebunden.

\subsection{Paketoptionen}
\begin{options}
	\keyval{lizenz}{Lizenzcode}\Default{cc-by-nc-sa-4}
		legt die Lizenz für das Dokument fest. Aktuell werden
		folgende Codes unterstützt: \begin{smallitemize}
			\item cc-by-4
			\item cc-by-sa-4
			\item cc-by-nc-sa-4
		\end{smallitemize}
\end{options}

\subsection{Befehle}
\begin{commands}
	\command{lizenzName}
		gibt den vollständigen Namen der Lizenz des Dokuments zurück.
	\command{lizenzNameKurz}
		gibt den gekürzten Namen der Lizenz des Dokuments zurück.
	\command{lizenzSymbol}
		setzt das Symbol der Lizenz des Dokuments.

\end{commands}
    \section{Metadaten}
\label{modul:metadaten}
Dieses Modul definiert die Paketoptionen zum Setzen bestimmter Metadaten 
und bietet Makros zum Zugriff darauf an.

Im Gegensatz zu älteren Versionen des Schule-Paketes werden für Metadaten
immer die Standardmakros von \LaTeX\space eingesetzt, soweit dies
möglich ist. Dies gilt etwa für Autor (\cs{author}\marg{Autor}), Datum 
(\cs{date}\marg{Datum}) und Titel (\cs{titel}\marg{Titel}) des 
Dokumentes. Es werden allerdings auch für diese Metadaten Makros zum
 einfachen Zugriff aus dem Dokument heraus definiert.

\subsection{Paketoptionen}
\begin{options}
	\keyval{fach}{Fach}
		legt das Fach für das Dokument fest, siehe \prettyref{sec:faecher}.
	\keyval{lerngruppe}{Lerngruppe}
		legt die Lerngruppe für das Dokument fest.
	\keyval{nummer}{Dokumentnummer}
		legt die Dokumentnummer fest.
\end{options}

\subsection{Befehle}
\begin{commands}
	\command{Autor}
		gibt den Autor des Dokuments zurück.
	\command{Datum}
		gibt das Datum des Dokuments zurück.
	\command{Fach}
		gibt das Fach des Dokuments zurück.
	\command{Lerngruppe}
		gibt die Lerngruppe des Dokuments zurück. Als Alternative ist auch \cs{Kurs} definiert.
	\command{Nummer}
		gibt die Nummer des Dokuments zurück.
	\command{Titel}
		gibt den Titel des Dokuments zurück.	
\end{commands}
    \section{Papiertypen}
\label{modul:papiertypen}
Das Modul \module{Papiertypen} stellt einige Makros bereit, die
es erlauben, Freiräume zum Bearbeiten von Aufgaben zu setzen. Hierzu
stehen verschiedene Muster zur Auswahl. Die entsprechenden Felder werden
dabei in der Breite jeweils auf \verbcode|\linewidth| skaliert, 
allerdings so, dass ein vollständiges Muster entsteht. Die zur Verfügung
stehende Breite wird also optimal genutzt.

\subsection{Befehle}
\begin{commands}
	\command{feldLin}[\oarg{Abstand}\marg{Anzahl}]
		setzt die angegebene Anzahl Linien mit dem angegebenen Abstand zueinander. Der Standardabstand beträgt $1cm$.
\begin{example}
  \feldLin[1cm]{4}
\end{example}

	\command{feldKar}[\oarg{Seitenlänge}\marg{Anzahl}]
		setzt die angegebene Anzahl von Karo-Kästchen mit einer
		gegebenen Seitenlänge. Der Standard für die Seitenlänge beträgt
		$0,5cm$.
\begin{example}
  \feldKar[0.5cm]{5}
\end{example}

	\command{feldMil}[\marg{Anzahl}]
		setzt die angegebene Anzahl von Kästchen im
		Millimeterpapiermuster untereinander. Die Farbe wird von der
		Paketoption \option{farbig} beeinflusst.
\begin{example}
  \feldMil{2}
\end{example}
\end{commands}
    \section{Symbole}
\label{modul:symbole}
Dieses Modul stellt einige für den Schulkontext relevante Unicode-Symbole aus den vom Paket \pkg{utfsym} unterstützten Blöcken als benannte Makros zur Verfügung.

\subsection{Befehle}
\begin{multicols}{4}

\paragraph{Körperteile}
\begin{commands}
	\command{symNase}
		{\Huge\usym{1F443}} (1F443)
	\command{symAuge}
		{\Huge\usym{1F441}} (1F441)
	\command{symAugen}
		{\Huge\usym{1F440}} (1F440)
	\command{symMund}
		{\Huge\usym{1F444}} (1F444)
	\command{symZunge}
		{\Huge\usym{1F445}} (1F445)
	\command{symOhr}
		{\Huge\usym{1F442}} (1F442)
	\command{symDaumenHoch}
		{\Huge\usym{1F44D}} (1F44D)
	\command{symDaumenRunter}
		{\Huge\usym{1F44E}} (1F44E)
	\command{symZeigefinger}
		{\Huge\usym{1F446}} (1F446)
	\command{symApplaus}
		{\Huge\usym{1F44F}} (1F44F)
	
\end{commands}

\paragraph{Kommunikation}
\begin{commands}
	\command{symSprechblase}
		{\Huge\usym{1F5E9}} (1F5E9)
	\command{symZweiSprechblasen}
		{\Huge\usym{1F5EA}} (1F5EA)
	\command{symDreiSprechblasen}
		{\Huge\usym{1F5EB}} (1F5EB)
	\command{symDenkblase}
		{\Huge\usym{1F5ED}} (1F5ED)
	
\end{commands}

\paragraph{Kunst}
\begin{commands}
	\command{symPalette}
		{\Huge\usym{1F3A8}} (1F3A8)
	
\end{commands}

\paragraph{Material}
\begin{commands}
	\command{symBleistift}
		{\Huge\usym{1F589}} (1F589)
	\command{symFueller}
		{\Huge\usym{1F58B}} (1F58B)
	\command{symKuli}
		{\Huge\usym{1F58A}} (1F58A)
	\command{symBuntstift}
		{\Huge\usym{1F58D}} (1F58D)
	\command{symLineal}
		{\Huge\usym{1F4CF}} (1F4CF)
	\command{symGeodreieck}
		{\Huge\usym{1F4D0}} (1F4D0)
	\command{symBueroklammer}
		{\Huge\usym{1F4CE}} (1F4CE)
	\command{symBueroklammern}
		{\Huge\usym{1F587}} (1F587)
	\command{symPin}
		{\Huge\usym{1F4CC}} (1F4CC)
	\command{symNadel}
		{\Huge\usym{1F4CD}} (1F4CD)
	\command{symPinsel}
		{\Huge\usym{1F58C}} (1F58C)
	\command{symBuch}
		{\Huge\usym{1F56E}} (1F56E)
	\command{symBild}
		{\Huge\usym{1F5BC}} (1F5BC)
	\command{symMikroskop}
		{\Huge\usym{1F52C}} (1F52C)
	\command{symHeft}
		{\Huge\usym{1F4D3}} (1F4D3)
	\command{symBuecher}
		{\Huge\usym{1F4DA}} (1F4DA)
	\command{symKlemmbrett}
		{\Huge\usym{1F4CB}} (1F4CB)
	\command{symCD}
		{\Huge\usym{1F4BF}} (1F4BF)
	\command{symZeitung}
		{\Huge\usym{1F4F0}} (1F4F0)
	\command{symThermometer}
		{\Huge\usym{1F321}} (1F321)
	\command{symSchere}
		{\Huge\usym{2700}} (2700)
	\command{symSchloss}
		{\Huge\usym{1F512}} (1F512)
	\command{symSchlossOffen}
		{\Huge\usym{1F513}} (1F513)
	\command{symSchluessel}
		{\Huge\usym{1F511}} (1F511)
	\command{symGlocke}
		{\Huge\usym{1F514}} (1F514)
	\command{symKeineGlocke}
		{\Huge\usym{1F515}} (1F515)
	\command{symLupe}
		{\Huge\usym{1F5FD}} (1F5FD)
	
\end{commands}

\paragraph{Musik}
\begin{commands}
	\command{symNote}
		{\Huge\usym{1F39C}} (1F39C)
	\command{symNoten}
		{\Huge\usym{1F3B6}} (1F3B6)
	
\end{commands}

\paragraph{Smileys}
\begin{commands}
	\command{symSmileyLachend}
		{\Huge\usym{1F642}} (1F642) 
	\command{symSmileyNeutral}
		{\Huge\usym{1F610}} (1F610) 
	\command{symSmileyTraurig}
		{\Huge\usym{1F641}} (1F641) 
	\command{symSmileyGrinsend}
		{\Huge\usym{1F600}} (1F600) 
	\command{symSmileySchlafend}
		{\Huge\usym{1F614}} (1F614) 
	\command{symSmileyZwinkernd}
		{\Huge\usym{1F609}} (1F609) 
	
	
\end{commands}

\paragraph{Sonstiges}
\begin{commands}
	\command{symKlee}
		{\Huge\usym{1F340}} (1F340)
	\command{symSonne}
		{\Huge\usym{1F323}} (1F323)
	\command{symMond}
		{\Huge\usym{1F319}} (1F319)
	\command{symStern}
		{\Huge\usym{1F31F}} (1F31F)
	\command{symUhr}
		{\Huge\usym{1F551}} (1F551)
	\command{symHaken}
		{\Huge\usym{1F5F8}} (1F5F8)
	
	
\end{commands}

\paragraph{Spielkarten}
\begin{commands}
	\command{symSpielkarte}
		{\Huge\usym{1F0A0}} (1F0A0)
	\command{symPik}
		{\Huge\usym{2660}} (2660)
	\command{symHerz}
		{\Huge\usym{2665}} (2665)
	\command{symKaro}
		{\Huge\usym{2666}} (2666)
	\command{symKreuz}
		{\Huge\usym{2663}} (2663)
	
	\command{symPikAss}
		{\Huge\usym{1F0A1}} (1F0A1) 
	\command{symPikZwei}
		{\Huge\usym{1F0A2}} (1F0A2) 
	\command{symPikDrei}
		{\Huge\usym{1F0A3}} (1F0A3) 
	\command{symPikVier}
		{\Huge\usym{1F0A4}} (1F0A4) 
	\command{symPikFuenf}
		{\Huge\usym{1F0A5}} (1F0A5) 
	\command{symPikSechs}
		{\Huge\usym{1F0A6}} (1F0A6)
	\command{symPikSieben}
		{\Huge\usym{1F0A7}} (1F0A7) 
	\command{symPikAcht}
		{\Huge\usym{1F0A8}} (1F0A8) 
	\command{symPikNeun}
		{\Huge\usym{1F0A9}} (1F0A9) 
	\command{symPikZehn}
		{\Huge\usym{1F0AA}} (1F0AA) 
	\command{symPikBube}
		{\Huge\usym{1F0AB}} (1F0AB) 
	\command{symPikDame}
		{\Huge\usym{1F0AD}} (1F0AD) 
	\command{symPikKoenig}
		{\Huge\usym{1F0AE}} (1F0AE) 
	
	\command{symHerzAss}
		{\Huge\usym{1F0B1}} (1F0B1) 
	\command{symHerzZwei}
		{\Huge\usym{1F0B2}} (1F0B2) 
	\command{symHerzDrei}
		{\Huge\usym{1F0B3}} (1F0B3) 
	\command{symHerzVier}
		{\Huge\usym{1F0B4}} (1F0B4) 
	\command{symHerzFuenf}
		{\Huge\usym{1F0B5}} (1F0B5) 
	\command{symHerzSechs}
		{\Huge\usym{1F0B6}} (1F0B6)
	\command{symHerzSieben}
		{\Huge\usym{1F0B7}} (1F0B7) 
	\command{symHerzAcht}
		{\Huge\usym{1F0B8}} (1F0B8) 
	\command{symHerzNeun}
		{\Huge\usym{1F0B9}} (1F0B9) 
	\command{symHerzZehn}
		{\Huge\usym{1F0BA}} (1F0BA) 
	\command{symHerzBube}
		{\Huge\usym{1F0BB}} (1F0BB) 
	\command{symHerzDame}
		{\Huge\usym{1F0BD}} (1F0BD) 
	\command{symHerzKoenig}
		{\Huge\usym{1F0BE}} (1F0BE) 
	
	\command{symKaroAss}
		{\Huge\usym{1F0C1}} (1F0C1) 
	\command{symKaroZwei}
		{\Huge\usym{1F0C2}} (1F0C2) 
	\command{symKaroDrei}
		{\Huge\usym{1F0C3}} (1F0C3) 
	\command{symKaroVier}
		{\Huge\usym{1F0C4}} (1F0C4) 
	\command{symKaroFuenf}
		{\Huge\usym{1F0C5}} (1F0C5) 
	\command{symKaroSechs}
		{\Huge\usym{1F0C6}} (1F0C6)
	\command{symKaroSieben}
		{\Huge\usym{1F0C7}} (1F0C7) 
	\command{symKaroAcht}
		{\Huge\usym{1F0C8}} (1F0C8) 
	\command{symKaroNeun}
		{\Huge\usym{1F0C9}} (1F0C9) 
	\command{symKaroZehn}
		{\Huge\usym{1F0CA}} (1F0CA) 
	\command{symKaroBube}
		{\Huge\usym{1F0CB}} (1F0CB) 
	\command{symKaroDame}
		{\Huge\usym{1F0CD}} (1F0CD) 
	\command{symKaroKoenig}
		{\Huge\usym{1F0CE}} (1F0CE) 
	
	\command{symKreuzAss}
		{\Huge\usym{1F0D1}} (1F0D1) 
	\command{symKreuzZwei}
		{\Huge\usym{1F0D2}} (1F0D2) 
	\command{symKreuzDrei}
		{\Huge\usym{1F0D3}} (1F0D3) 
	\command{symKreuzVier}
		{\Huge\usym{1F0D4}} (1F0D4) 
	\command{symKreuzFuenf}
		{\Huge\usym{1F0D5}} (1F0D5) 
	\command{symKreuzSechs}
		{\Huge\usym{1F0D6}} (1F0D6)
	\command{symKreuzSieben}
		{\Huge\usym{1F0D7}} (1F0D7) 
	\command{symKreuzAcht}
		{\Huge\usym{1F0D8}} (1F0D8) 
	\command{symKreuzNeun}
		{\Huge\usym{1F0D9}} (1F0D9) 
	\command{symKreuzZehn}
		{\Huge\usym{1F0DA}} (1F0DA) 
	\command{symKreuzBube}
		{\Huge\usym{1F0DB}} (1F0DB) 
	\command{symKreuzDame}
		{\Huge\usym{1F0DD}} (1F0DD) 
	\command{symKreuzKoenig}
		{\Huge\usym{1F0DE}} (1F0DE) 
	
\end{commands}

\paragraph{Sport}
\begin{commands}
	\command{symBaseball}
		{\Huge\usym{26BE}} (26BE)
	\command{symBasketball}
		{\Huge\usym{1F3C0}} (1F3C0)
	\command{symFussball}
		{\Huge\usym{26BD}} (26BD)
	\command{symVolleyball}
		{\Huge\usym{1F3D0}} (1F3D0)
	\command{symHockey}
		{\Huge\usym{1F3D1}} (1F3D1)
	
	\command{symLaufen}
		{\Huge\usym{1F3C3}} (1F3C3)
	\command{symReiten}
		{\Huge\usym{1F3C7}} (1F3C7)
	\command{symSchwimmen}
		{\Huge\usym{1F3CA}} (1F3CA)
	
	\command{symSki}
		{\Huge\usym{26F7}} (26F7)
	\command{symSnowboard}
		{\Huge\usym{1F3C2}} (1F3C2)
	
	\command{symSurfen}
		{\Huge\usym{1F3C4}} (1F3C4)
	
	\command{symTennis}
		{\Huge\usym{1F3BE}} (1F3BE)
	\command{symTischtennis}
		{\Huge\usym{1F3D3}} (1F3D3)
	
	\command{symPokal}
		{\Huge\usym{1F3C6}} (1F3C6)
	\command{symMedaille}
		{\Huge\usym{1F3C5}} (1F3C5)
	\command{symZielflagge}
		{\Huge\usym{1F3C1}} (1F3C1)
	
\end{commands}

\paragraph{Technik}
\begin{commands}
	\command{symHandy}
		{\Huge\usym{1F4F1}} (1F4F1)
	\command{symKeinHandy}
		{\Huge\usym{1F4F5}} (1F4F5)
	
\end{commands}

\paragraph{Theater}
\begin{commands}
	\command{symTheater}
		{\Huge\usym{1F0DD}} (1F0DD)
	
\end{commands}

\paragraph{Verkehrsmittel}
\begin{commands}
	\command{symAuto}
		{\Huge\usym{1F698}} (1F698)
	\command{symBus}
		{\Huge\usym{1F68C}} (1F68C)
	\command{symBahn}
		{\Huge\usym{1F682}} (1F682)
	\command{symStrassenbahn}
		{\Huge\usym{1F68B}} (1F68B)
	\command{symSchwebebahn}
		{\Huge\usym{1F69F}} (1F69F)
	\command{symSeilbahn}
		{\Huge\usym{1F6A1}} (1F6A1)
	\command{symSchiff}
		{\Huge\usym{1F6A2}} (1F6A2)
	\command{symBoot}
		{\Huge\usym{1F6A3}} (1F6A3)
	\command{symFahrrad}
		{\Huge\usym{1F6B2}} (1F6B2)
	\command{symFussgaenger}
		{\Huge\usym{1F6B8}} (1F6B8)
	\command{symRollstuhl}
		{\Huge\usym{267F}} (267F)
	
	
\end{commands}

\paragraph{Würfel}
\begin{commands}
	\command{symWuerfelEins}
		{\Huge\usym{2680}} (2680)
	\command{symWuerfelZwei}
		{\Huge\usym{2681}} (2681)
	\command{symWuerfelDrei}
		{\Huge\usym{2682}} (2682)
	\command{symWuerfelVier}
		{\Huge\usym{2683}} (2683)
	\command{symWuerfelFuenf}
		{\Huge\usym{2684}} (2684)
	\command{symWuerfelSechs}
		{\Huge\usym{2685}} (2685)	
\end{commands}
\end{multicols}
    \section{Texte}
\label{modul:texte}
Dieses Modul definiert einige Umgebungen, die für die Formatierung
längerer Texte hilfreich sind. 

\achtung{Da die Umgebungen mit Zeilennummern
nicht ohne große Klimmzüge in umrahmte Boxen gesetzt werden können,
sehen die Beispiele hier ein wenig anders aus.}

\subsection{Befehle}
\begin{commands}
	\command{resetZeilenNr}
		Standardmäßig werden die Zeilennummern über die
		Umgebungsgrenzen hinweg vergeben. Möchte man in jeder neuen
		Umgebung mit 1 beginnen, so muss man die Zeilennummer mit
		diesem Befehl zunächst zurücksetzen.
\end{commands}

\subsection{Umgebungen}
\begin{environments}

	\environment{mehrspaltig}[\oarg{Anzahl}]
		Setzt einen gegebenen Text mehrspaltig, wobei die Anzahl der
		Spalten angegeben werden kann. Die Standardanzahl ist 2.
\begin{example}
  \begin{mehrspaltig}
    \blindtext
  \end{mehrspaltig}
\end{example}
\resetZeilenNr
	\environment{zeilenNr}[\oarg{Modulo}]
		Setzt einen gegebenen Text mit Zeilennummern, wobei
		ein	Modulo für den Abstand der Zeilennummern angegeben werden
		kann.
		
		Der Standardmodulo beträgt 5.
\begin{example}[outside=true]
  \begin{zeilenNr}[1]
    \blindtext
  \end{zeilenNr}
\end{example}
\resetZeilenNr

	\environment{zeilenNrMehrspaltig}[\oarg{Modulo}\marg{Anzahl}]
		Setzt einen gegebenen Text mehrspaltig mit Zeilennummern, 
		wobei die Anzahl der Spalten angegeben werden muss und 
		zusätzlich ein Modulo für den Abstand der Zeilennummern
		angegeben werden kann.
		
		Die Zeilennummern stehen jeweils links neben der jeweiligen
		Spalte.	Der Standardmodulo beträgt 5.
\begin{example}[outside=true]
  \begin{zeilenNrMehrspaltig}[5]{3}
    \blindtext
  \end{zeilenNrMehrspaltig}
\end{example}
\resetZeilenNr

	\environment{zeilenNrZweispaltig}[\oarg{Modulo}]
		Setzt einen gegebenen Text zweispaltig mit Zeilennummern,
		wobei ein Modulo für den Abstand der Zeilennummern angegeben
		werden kann.
		
		Die Zeilennummern stehen links neben der linken Spalte und
		rechts neben der rechten Spalte. Der Standardmodulo beträgt 5.
		
\begin{example}[outside=true]
  \begin{zeilenNrZweispaltig}[3]
    \blindtext
  \end{zeilenNrZweispaltig}
\end{example}
\resetZeilenNr
\end{environments}

\part{Fächer}
	\label{sec:faecher}
	
	\section{Nutzung der Fachmodule}
	
	Die fachspezifischen Funktionen und Vorgaben sind in sogenannte
	Fachmodule aufgeteilt, die über Paketoptionen flexibel geladen
	werden können.
	
	\begin{options}
		\keyval-{fach}{Fach}
			Mit der Paketoption \option{fach} kann das Fach für das
			Dokument festgelegt werden. Es werden dann alle
			fachspezifischen Funktionen und Vorgaben für das Fach
			geladen. 
			
			Mit der Angabe von \keyis-{fach}{ohne} kann auf die Angabe
			eines Faches verzichtet werden, etwa für die Einbindung in
			Dokumentationen \etc.
			
			Wird ein Fach angegeben, zu dem kein Fachmodul existiert,
			so wird dieses nur als Bezeichner verwendet.
		\keyval{weitereFaecher}{Fach 1,Fach 2,\dots}
			Für fächerübergreifenden Unterricht können weitere
			Fachmodule geladen werden, indem eine kommaseparierte
			Liste von Fachmodulen angegeben wird.
			
			Hierbei wird auf das Laden möglicher
			\enquote{Standalone-Abschnitte} der Fachmodule verzichtet, 
			\vgl\space\prettyref{sec:devmodule:fachmodul}
	\end{options}
	
\section{Informatik}
\label{fach:informatik}
Das Fachmodul \module{Informatik} bindet Pakete ein, um Klassen- sowie Objektdiagramme (\texttt{pgf-umlcd}), Syntaxdiagramme (\pkg{syntaxdi}, siehe \prettyref{paket:syntaxdi}), Struktogramme (\pkg{struktex}) und Sequenzdiagramme (\texttt{pgf-umlsd}) setzen zu können. Die entsprechenden Dokumentationen sind bei den jeweiligen Paketen zu finden. Hier sind lediglich Abweichungen und Erweiterungen vom Standardumfang der Pakete dokumentiert.

\subsection{Objektdiagramme}
\begin{commands}
    \command{anchormark}[\oarg{Horizontale Verschiebung} \marg{Nodename}\oarg{Skalierung}] Durch den Befehl \cs{anchormark} können Objektdiagramme mit Beziehungsattributen ausgestattet werden, die an der korrekten Stelle hinter dem Attributbezeichner beginnen.

    \achtung{Dieser Befehl ist nicht skalierungssicher!}

    % % Der folgende kenntlich gemachte Abschnitt ist in der Zusammenarbeit
    % % von Informatikreferendaren und ehemaligen Informatikreferendaren
    % % der Studienseminare (heute ZfsL) Arnsberg, Hamm und Solingen
    % % entstanden.
    % %
    % % Der Abschnitt steht unter der Lizenz: Creative Commons by-nc-sa
    % % Version 4.0
    % % http://creativecommons.org/licenses/by-nc-sa/4.0/deed.de
    % %
    % % Nach dieser Lizenz darf der Abschnitt beliebig kopiert und
    % % bearbeitet werden, sofern das Folgeprodukt wiederum unter
    % % gleichen Lizenzbedingungen vertrieben und auf die ursprünglichen
    % % Urheber verwiesen wird.  Eine kommerzielle Nutzung ist
    % % ausdrücklich ausgeschlossen.
    % %
    % % Die Namensnennung durch einen Verweis und die Lizenzangabe der
    % % ursprünglichen Urheber auf den Materialien für Schülerinnen und
    % % Schüler ist erforderlich.
    % %
    % % Die vollständige Sammlung der Dokumente steht unter
    % % http://ddi.uni-wuppertal.de/material/materialsammlung/ zur
    % % Verfügung.
    % %
    % % Das LaTeX-Paket zum Setzen der Dokumente der Sammlung steht unter
    % % http://www.ctan.org/pkg/ zur Verfügung.
    % %
    % % ----- BEGIN ------------------------------------------------------
        \begin{example}[gobble=12]
            \begin{tikzpicture}[remember picture]
                \begin{object}[text width=5.5cm]{gustavsRadiowecker}{-3,0}
                    \attribute{standort = \diastring{Gustavs Zimmer}}
                    \attribute{weckzeit = \diastring{6:30}}
                    \attribute{weckmodusAktiv = \diastring{Wahr}}
                    \attribute{hatLautsprecher = \anchormark{hatLautsprecher}[0.025]}
                    \operation{einschalten()}
                    \operation{ausschalten()}
                    \operation{alarmAusloesen()}
                \end{object}
                \begin{object}[text width=4.5cm]{gustav}{-10,0}
                    \attribute{name = \diastring{Gustav Grabert}}
                    \attribute{geburtstag = \diastring{3.10.1998}}
                    \attribute{besitzt =\anchormark{besitzt}[0.025]}
                    \attribute{kennt =\anchormark{gKennt}[0.025]}
                \end{object}
                \begin{object}[text width=4.5cm]{fridolin}{-10,-4}
                    \attribute{name = \diastring{Fridolin Wagner}}
                    \attribute{geburtstag = \diastring{1.4.1999}}
                    \attribute{kennt =\anchormark{fKennt}[0.025]}
                \end{object}
                \begin{object}[text width=5.2cm]{lautsprecher}{-3,-5}
                    \attribute{untereFrequenzInHertz = 100}
                    \attribute{obereFrequenzInHertz = 18000}
                \end{object}

                \draw (hatLautsprecher) -- (lautsprecher.north);
                \draw (gKennt.south east) -- (fridolin.north);
                \draw (besitzt.east) -- (gustavsRadiowecker.west);
                \draw (fKennt.east) -- ($(fKennt.east)+(3.5,0)$)
                    -| ($(gustav.south)+(3,0.2)$) -- ($(gustav.south east)
                    +(-0.01,0.2)$);
            \end{tikzpicture}
        \end{example}
        % % ----- END ------------------------------------------------------
\end{commands}

\subsection{Sequenzdiagramme}
\label{fach:informatik:sequenz}
\begin{commands}
    \command{skaliereSequenzdiagramm}[\marg{Faktor}]
        \achtung{Sollte nicht mehr verwendet werden: Besser resizebox oder scalebox}

        Da es vorkommen kann, dass Sequenzdiagramme zu breit für eine Seite sind, kann mit dem Befehl \cs{skaliereSequenzdiagramm}\marg{Faktor} die Größe des Sequenzdiagramms angepasst werden, wenn er innerhalb der Umgebung \env{sequencediagram} ausgeführt wird.

    \command{newthreadtwo}[\oarg{Farbe}\marg{Bezeichnung}\marg{Name}\marg{Abstand}]
        Threads haben im Gegensatz zu Instanzen im Paket \pkg{pgf-umlsd} immer einen festen Abstand zu den Nachbarn. Durch den neuen Befehl \cs{newthreadtwo} ist es über den dritten Parameter möglich, diesen Abstand zu verändern. Dabei verhält sich der neue Parameter für den Abstand genauso wie der zugehörige optionale Parameter bei Instanzen.

        \begin{example}[gobble=12]
            \begin{sequencediagram}
                \newthread{fritz}{fritz}
                \newthreadtwo{mutter}{mutter}{3cm}
                \newinst[2]{wecker}{wecker}
                \newinst[2]{lampe}{lampe}

                \begin{callself}[2]{fritz}{schlafe()}{}
                \end{callself}
                \begin{call}{fritz}{gibUhrzeit()}{wecker}{\diastring{5:30}}
                \end{call}
                \begin{callself}[2]{fritz}{schlafe()}{}
                    \begin{call}{mutter}{gibUhrzeit()}{wecker}{\diastring{6:30}}
                    \end{call}
                \begin{call}{mutter}{schalteAn()}{lampe}{}
                    \end{call}
                    \begin{call}{mutter}{weckeAuf()}{fritz}{}
                    \end{call}
                \end{callself}
            \end{sequencediagram}
        \end{example}

    \command{nextlevel} Im Paket für Sequenzdiagramme ist vorgesehen, dass man mit \cs{prevlevel} wieder einen Schritt nach oben gehen kann. Zusätzlich wird ein Befehl \cs{nextlevel} bereitgestellt, mit dem man auch einen zusätzlichen Schritt nach unten gehen kann, um ggf. etwas mehr Platz und Abstand zu schaffen.

\end{commands}

\subsection{Struktogramme}
Mit dem Paket \pkg{struktex} lassen sich sehr einfach Struktogramme setzen:

\begin{example}[gobble=4]
    \begin{struktogramm}(130,60)[koche Kaffee]
    \assign{F"ulle 1 Liter Wasser in die Kaffekanne}
    \assign{Gie"se das Wasser in den Wasserbeh"alter}
    \assign{Lege eine Filtert"ute in den Filter}
    \ifthenelse{5}{5}{Sind die Kaffeetrinker m"ude?}{Ja}{Nein}
        \assign{Gib 6 L"offel Pulver hinein}
    \change
        \assign{Gib 5 L"offel Pulver hinein}
    \ifend
    \assign{Dr"ucke auf den Start-Knopf}
    \while{Solange der Kaffee noch nicht durchgelaufen ist}
        \assign{Warte ungeduldig}
    \whileend
    \assign{Gie"se den Kaffee in die Tasse}
    \assign{Trinke den Kaffee aus der Tasse}
    \end{struktogramm}
\end{example}

\subsection{Syntaxdiagramme}
Für ein Beispiel siehe \prettyref{paket:syntaxdi}.

\subsection{Flussdiagramme}
Für Flussdiagramme, bzw. Programmablaufpläne steht der Style \verbcode|pap| bereit, der in \env{tikzpicture} genutzt werden kann. Damit werden \cs{node} ein entsprechendes Aussehen gegeben. Es stehen zur Verfügung:
\begin{description}
    \item[startstop] Für den Beginn bzw. das Ende eines Ablaufs, als Rechteck mit runden Ecken.
    \item[verzweigung] Für Abfragen und Wiederholungen als Diamant.
    \item[aktion] Einfache Aktionen in einem Rechteck.
    \item[einausgabe] Ein Rhomboid wird für Ein- oder Ausgaben genutzt.
    \item[unterprogramm] Ein Rechteck ergänzt um freie Flächen auf der linken und rechten Seite stellt den Aufruf eines Unterprogramms dar.
\end{description}

Weiterhin können Linien mit dem Style \verbcode|linie| versehen werden, um diese deutlicher darzustellen.

\begin{example}[gobble=4]
    \begin{tikzpicture}[pap]
        \node[startstop] (s1){Los!};
        \node[verzweigung, below = of s1] (v1) {Lieblingsfach Informatik?};
        \node[unterprogramm, right = of v1] (up1) {\nodepart[text width=7em]{two} Pro\-gram\-mie\-re ein Spiel};
        \node[aktion, below = of up1] (a1) {fuehre es aus!};
        \node[einausgabe, below = of v1] (ea1) {ERROR 1337};
        \node[startstop, below = of ea1] (e1){Ende};

        \draw[linie] (s1)--(v1);
        \draw[linie] (v1)--(up1) node[near start, above] {ja};
        \draw[linie] (v1)--(ea1) node[near start, right] {nein};
        \draw[linie] (up1)--(a1);

        \draw[linie] (a1) |- ($(e1.north) + (0,0.5)$);
        \draw[linie] (ea1)--(e1);
    \end{tikzpicture}
\end{example}

\section{Physik}
	\label{fach:physik}
	Zur Zeit ist das Paket \module{Physik} noch leer, bis auf das Einbinden der
	Pakete \pkg{units}, \pkg{circuittikz} und \pkg{mhchem}.

	Ein kurzes Beispiel zur Dartstellung von Schaltplänen mit dem
	Paket \pkg{circuittikz} soll an dieser Stelle genügen. Ausführlichere Hinweise können den	entsprechenden Dokumentationen der einzelnen Pakete entnommen werden.

\begin{example}[gobble=0]
\begin{circuitikz}
  \draw
    (0,0)--(1,0) to[european resistor,l=$47$\,k$\Omega$] (3,0)--(5,0)
    to[C, l=$470$\,$\mu$F] (7,0) -- (8,0)
    (4.5,0) to[short, -*] (4.5,0) -- (4.5, -2)
    (4.5,-2) -- (5,-2) to[voltmeter, l=$U_C$] (7,-2) -- (7.5,-2)
    (7.5, -2) to[short, -*] (7.5,0)
    (8,1) node[spdt, rotate=90] (Ums) {}
    (Ums) node[right=0.4cm] {$WS$}
    (Ums.out 1) node[left] {1}
    (Ums.out 2) node[right] {2}
    (0,0) |- (2,4) to[closing switch, l=$S$] (3,4) to[battery1,
    l=$U$] (5,4) -| (Ums.out 2)
    (Ums.in) -- (8,0)
    (Ums.out 1) |- (0,2) to[short, -*] (0,2)
  ;
\end{circuitikz}	
\end{example}
\section{Geschichte}
\label{fach:geschichte}
Das Fachmodul \module{Geschichte} bindet das Paket \pkg{biblatex} mit den Einstellungen für die humanwissenschaftliche Zitierweise ein. Des weiteren werden Befehle für durchnummerierte Quellen, Materialien und Verfassertexte zur Verfügung gestellt.

\subsection{Befehle}
    \begin{commands}
        \command{material}[\oarg{Ebene}\marg{Titel}]\Default{\textbackslash subsection}
            Erzeugt eine Überschrift, mit der Material markiert werden kann. Dazu wird am rechten Rand ein M mit einer fortlaufenden Nummer gesetzt. Standardmäßig ist als Überschriftsebene \verbcode|\subsection| gesetzt, dass über den optionalen Parameter geändert werden kann.

        \command{quelle}[\oarg{Ebene}\marg{Titel}]\Default{\textbackslash subsection}
            Erzeugt eine Überschrift, mit der eine Quelle markiert werden kann. Dazu wird am rechten Rand ein Q mit einer fortlaufenden Nummer gesetzt. Standardmäßig ist als Überschriftsebene \verbcode|\subsection| gesetzt, dass über den optionalen Parameter geändert werden kann.

        \command{vt}[\oarg{Ebene}\marg{Titel}]\Default{\textbackslash subsection}
            Erzeugt eine Überschrift, mit der ein Verfassertext markiert werden kann. Dazu wird am rechten Rand ein VT mit einer fortlaufenden Nummer gesetzt. Standardmäßig ist als Überschriftsebene \verbcode|\subsection| gesetzt, dass über den optionalen Parameter geändert werden kann.
    \end{commands}
    
    Die so erstellten Textabschnitte können mit \verb|\nameref{sec:$REFERENZ_ART$NUMMER}| referenziert werden, z.\,B. per \verb|\nameref{sec:vt1}|. Für weitere Hinweise siehe \prettyref{example:bsp-geschichte}.
    
    
% \begin{example}[gobble=0]
% %
% \end{example}

\part{Dokumenttypen}
	\label{sec:typen}
	Dokumenttypen dienen dazu, Vorgaben für spezielle Arten von
	Dokumenten zu machen und entsprechende Makros bereitzustellen.
	Üblicherweise wird eine bestimmte Dokumentklasse für die
	Verwendung empfohlen, grundsätzlich sind die Dokumenttypen aber
	unabhängig von der verwendeten Klasse.

	Die Typen werden über die Paketoption \option{typ} geladen.
	Ist der Typ unbekannt, wird ein Arbeitsblatt gesetzt und der
	die Bezeichnung des Dokuments wird auf den angegebenen Typ
	eingestellt.

	Wird kein Typ angegeben, ist der Kompatibilitätsmodus zum
	alten Schule-Paket aktiv. Es wird folglich kein Dokumenttyp
	geladen.
	Ebensowenig wird ein Dokumenttyp geladen, wenn der Typ auf
	\keyis{typ}{ohne} gesetzt wird, da sich das Paket dann im
	eingebetteten Modus befindet, also kein eigenständiges
	Dokument gesetzt werden soll.

	\section{Arbeitsblatt}
\label{typ:ab}
Dieser Dokumenttyp ist der Standard des Schulepakets. Wird ein unbekannter Dokumenttyp verwendet, wird stattdessen ein Arbeitsblatt gesetzt. Um gezielt das Arbeitsblatt zu verwenden ist als Typ \verbcode|ab| anzugeben.

Die empfohlene Dokumentklasse ist \cls{scrartcl}.

Die Vorgaben des Dokumenttyps definieren Kopf- und Fußzeilen in der üblichen Darstellung des Schule-Pakets. Darüber hinaus werden keine weiteren Vorgaben gemacht.

%\subsection{Paketoptionen}
%\begin{options}
%	\opt{X}
%		Y
%\end{options}

%\subsection{Befehle}
%\begin{commands}
%
%\command{X}
%	Y
%
%\end{commands}
%	\input{typBrief}
	\section{Klausur}
\label{typ:kl}
Dieser Dokumenttyp wird für Klausuren und Klassenarbeiten verwendet. Für die Verwendung der Klausur oder Klassenarbeit ist als Typ \verbcode|kl| anzugeben.

Die empfohlene Dokumentklasse ist \cls{scrartcl}.

Die Vorgaben des Dokumenttyps definieren Kopf- und Fußzeilen in der üblichen Darstellung des Schule-Pakets. Außerdem werden Namens- und Datumsfelder erzwungen.

\subsection{Paketoptionen}
\begin{options}
    \keychoice{klausurtyp}{klausur,klasse,kurs}\Default{klausur}
        legt fest, ob die Klausur als Klausur (Standard), Kursarbeit (\keyis{klausurtyp}{kurs}) oder Klassenarbeit (\keyis{klausurtyp}{klasse}) bezeichnet wird.
\end{options}

%\subsection{Befehle}
%\begin{commands}
%
%\command{X}
%	Y
%
%\end{commands}
%	\input{typKursbuch}
    \section{Leitprogramm}
\label{typ:Leitprogramm}
Als Leitprogramm wird eine Grundlage für den Unterricht bezeichnet, mit dem \SuS sich ein größeres Thema erarbeiten können. Ein Leitprogramm enthält dafür erklärende Texte sowie Aufgaben mit Hinweisen und Lösungen. Diese werden von Elemente werden von den Lernenden selbstständig gelesen und bearbeitet. Zum Abschluss eines Kapitels gehört in der Regel ein Kapiteltest. Dieses holen sich die \SuS bei der Lehrkraft ab um ihn zu bearbeiten und ihn anschließend direkt von der Lehrkraft kontrollieren zu lassen. Dieser Test wird dabei nur auf Grundlage des Erlernten und ohne direktes Hinzunehmen des Leitprogramms absolviert.

Der Dokumententyp Leitprogramm, als Typ ist \verbcode|leit| anzugeben, stellt die layouttechnischen Grundlagen bereit und sorgt für die Verknüpfungen zwischen den Aufgaben und den dazugehörenden Hinweisen und Lösungen. Der Dokumententyp lässt sich aber auch für ein Skript nutzen, dass aus verschiedenen Kapiteln besteht. Die empfohlene Dokumentklasse ist \cls{scrreprt}. Ein Beispiel ist unter \prettyref{example:beispiel-leitprogramm} aufgeführt.

\subsection{Paketoptionen}
Beim Leitprogramm werden standardmäßig von der Aufgabe Links zu möglichen vorhanden Lösungen oder Bearbeitungshinweisen gesetzt. Da dieses Schaltflächen auch angezeigt werden, wenn die Lösungen bzw. Hinweise nicht eingebunden wurden, kann die Anzeige über Paketoptionen ausgeschaltet werden.
\begin{options}
    \opt{hinweisLinkVerbergen} verbirgt Links bei der Aufgabe zu möglichen Bearbeitungshinweisen.
    \opt{loesungLinkVerbergen} verbirgt Links bei der Aufgabe zu möglichen Lösungen.
\end{options}

\subsection{Befehle}
\begin{commands}
    \command{TextFeld}[\marg{Höhe}] Erstellt ein Formularfeld mit der angegebenen Höhe und der aktuellen Spaltenbreite. Mit passenden Anzeigeprogrammen kann dann an dieser Stelle im PDF-Dokument Text eingegeben werden.
    \command{monatWort}[\marg{Monatszahl}] Übersetzt den als Zahl angegeben Monat in den deutschen Namen. Sollte die Zahl nicht erkannt werden, wird \enquote{unbekannter Monat} ausgegeben.
    \command{uebungBild} Erstellt ein Symbol für eine Übung, dass allen Aufgaben innerhalb eines Leitprogramms vorangestellt wird.
    \command{hinweisBild} Erstellt ein Symbol für ein Hinweis.
\end{commands}

\subsection{Umgebungen}
\begin{environments}
    \environment{hinweisBox} Erzeugt eine optisch hervorgehobene Box, die mit dem Symbol für einen Hinweis gekennzeichnet ist.
\end{environments}
%	\input{typLogbuch}
	\section{Lernzielkontrolle}
\label{typ:lzk}
Dieser Dokumenttyp wird für Lernzielkontrollen verwendet. Für die Verwendung der Lernzielkontrolle ist als Typ \verbcode|lzk| anzugeben.

Die empfohlene Dokumentklasse ist \cls{scrartcl}.

Die Vorgaben des Dokumenttyps definieren Kopf- und Fußzeilen in der üblichen Darstellung des Schule-Pakets. Außerdem werden Namens- und Datumsfelder erzwungen.

%\subsection{Paketoptionen}
%\begin{options}
%	\opt X
%		Y
%\end{options}

%\subsection{Befehle}
%\begin{commands}
%
%\command{X}
%	Y
%
%\end{commands}
	\section{Übungsblatt}
\label{typ:ueb}
Dieser Dokumenttyp wird für Übungsblätter verwendet. Der Hauptunterschied zum Typ \enquote{Arbeitsblatt} liegt darin, dass Aufgaben als \enquote{Übungen} bezeichnet werden. Für die Verwendung des Übungsblatts ist als Typ \verbcode|ueb| anzugeben.

Die empfohlene Dokumentklasse ist \cls{scrartcl}.

Die Vorgaben des Dokumenttyps definieren Kopf- und Fußzeilen in der üblichen Darstellung des Schule-Pakets. Darüber hinaus werden keine weiteren Vorgaben gemacht.

%\subsection{Paketoptionen}
%\begin{options}
%	\opt{X}
%		Y
%\end{options}

%\subsection{Befehle}
%\begin{commands}
%
%\command{X}
%	Y
%
%\end{commands}
    \section{Unterrichtsbesuch}
\label{typ:ub}
Dieser Dokumenttyp dient als Grobvorlage für Unterrichtsbesuche. Eine komplette Vorlage wird nicht angeboten, da die Studienseminare unterschiedliche Anforderungen stellen und es auch in den einzelnen Seminaren sehr häufig Änderungen an den layouttechnischen Aspekten gibt. Die Hauptanwendungen dieses Dokumenttyps sind daher Unterrichtsbesuche, bei denen es keine festen Vorgaben gibt, wie z.\,B. bei Revisionen oder der Materialsammlung für Informatik, die vollständig dieses \LaTeX-Paket nutzt. Für die Verwendung dieses Dokumententyps ist \verbcode|ub| anzugeben.

Als Dokumentklasse wird für diesen Typ \cls{scrartcl} empfohlen. Darin wird durch den Dokumenttyp Kopf- und Fußzeile gesetzt, sowie eine Titelseite erzeugt, die mit Angaben gefüllt wird, die für einen Unterrichtsbesuch typisch sind und entsprechend angegeben werden müssen.

%\subsection{Paketoptionen}
%\begin{options}
%	\opt{X}
%		Y
%\end{options}

\subsection{Befehle für Angaben zum Unterrichtsbesuch}
Mit den folgenden Befehlen werden Angaben gesetzt, die auf der Titelseite des Unterrichtsbesuch angezeigt werden.
\begin{commands}
    \command{besuchtitel}[\marg{Titel}]
        setzt den Eintrag, um was für eine Art es sich bei dem Unterrichtsbesuch handelt. Dieses kann z.\,B. sein: "`2. Unterichtsbesuch im Fach Informatik"'.
    \command{lehrer}[\marg{Lehrername}]
        setzt den Namen des Lehrers, der neben der Titelseite auch im Seitenkopf angezeigt wird.

    \command{schulform}[\marg{Schulform}]
        setzt den Eintrag für die Schulform wie z.\,B. Gesamtschule.

    \command{lerngruppe}[\oarg{Kurzform der Lerngruppe} \marg{Name der Lerngruppe} \marg{Anzahl weiblich} \marg{Anzahl männlich}]
        sorgt dafür, dass die Angaben zur Lerngruppe gesetzt werden. Der Name wird auf dem Titelblatt und im Seitenkopf angegeben, außer die optionale Möglichkeit der Kurzform wurde genutzt. In diesem Fall wird die Kurzform im Seitenkopf angegeben. Aus der Anzahl der weiblichen und männlichen Schülerinnen und Schüler wird automatisch die Gesamtzahl bestimmt, daher sind für diese Angaben nur Zahlen erlaubt.

    \command{zeit}[\marg{Startzeit} \marg{Endzeit} \marg{Stunde}]
        bietet die Möglichkeit, die Zeiten der Besuchsstunde anzugeben. Neben der Uhrzeit des Beginns und des Endes muss angegeben werden, um welche Stunde es sich an dem entsprechenden Tag handelt.

    \command{schule}[\marg{Name der Schule}]
        hierüber lässt sich der Name der Schule angeben, der auf der Titelseite angezeigt wird.

    \command{raum}[\marg{Raumbezeichnung}]
        bietet die Möglichkeit die Bezeichnung des Raumes anzugeben, in dem die Besuchsstunde stattfinden soll.

\end{commands}
    \section{Folie}
\label{typ:folie}
Bei der Nutzung des Dokumenttyps Folie wird eine Seite mit wenig Rand zur Verfügung gestellt, der die Fuß und Kopfzeile fehlt. So kann möglichst viel auf eine Folie gedruckt werden, wenn diese im Unterricht zum Einsatz kommen soll. Um den Dokumenttyp verwenden zu können muss \verbcode|folie| als Typ angegeben werden.

%\subsection{Paketoptionen}
%\begin{options}
% \opt{X}
%   Y
%\end{options}

% \subsection{Befehle}
% \begin{commands}
%     \command{x}[]
%         ...
% \end{commands}

\part{Zusatzpakete}
\label{sec:zusatzpakete}
\section{Nutzung der Zusatzpakete}
Die Zusatzpakete sind normale \LaTeX-Pakete und können auch so
eingesetzt werden. Allerdings erfolgt die Nutzung innerhalb des
Schule-Pakets in der Regel über die Einbindung in Modulen oder
Fachmodulen. Dennoch ist die direkte Einbindung möglich, auch
unabhängig vom Schule-Paket.

\section{Schaltungen mit Relais}
\label{paket:relaycircuit}
Durch das Paket \texttt{relaycircuit} ist es möglich Schaltungen
mit Relais zu zeichnen. Dazu wird die neue Knotenform
\textit{relais} deklariert, die sich in \textit{arbeits relais}
(Bezeichnung: AK) und \textit{ruhe relais} (Bezeichnung: RK)
aufteilen. So kann der Schaltplan eines logischen NAND mittels Relais
wie folgt gesetzt werden:

\begin{example}[gobble=0]
\begin{tikzpicture}
  \draw (0,6.8) node [left] {\(+\)} -- (9,6.8);
  \draw (0,0) node [left] {\(-\)} -- (9,0);
  \draw (4.5,0) to[short, *-] (4.5,0) node [ground] {};

  \draw (7.4,2.5) to[short,*-] (7.5,2.5) to[lamp] (9,2.5) 
    node[ground] {};

  \draw (2.5,5.8) node[arbeits relais] (a1) {};
  \draw (2.5,4) node[arbeits relais] (a2) {};
  \draw (2.4,6.8) to[short,*-] (a1.anschluss);
  \draw (a1.ausgabe) -- (a2.anschluss);

  \draw (2.5,1) node[ruhe relais] (r1) {};
  \draw (a2.ausgabe) -- (r1.anschluss);
  \draw (r1.ausgabe) to[short,-*] (2.4,0);
  \draw (5,1) node[ruhe relais] (r2) {};
  \draw (r2.ausgabe) to[short,-*] (4.9,0);

  \draw (7.5,1) node[arbeits relais] (a3) {};
  \draw (7.5,4) node[ruhe relais] (r3) {};
  \draw (a3.anschluss) -- (r3.ausgabe);
  \draw (a3.ausgabe) to[short,-*] (7.4,0);
  \draw (r3.anschluss) to[short,-*] (7.4,6.8);

  \draw (2.4,2.5) to[short,*-*] (4.9,2.5) -| (a3.eingabe);
  \draw (r2.anschluss) |- (r3.eingabe);

  \draw (0,4.7) node [left] {A} to[short,-*] (0.2,4.7) --
    (a2.eingabe);
  \draw (0.2,4.7) |- (r1.eingabe);

  \draw (0,2.1) node [left] {B} to[short,-*] (0.4,2.1) -|
    (r2.eingabe);
  \draw (0.4,2.1) |- (a1.eingabe);
\end{tikzpicture}
\end{example}
\section{Syntaxdiagramme}
\label{paket:syntaxdi}

Mit den Paketen \pkg{syntaxdi} und \pkg{tikz} ist es möglich,
einfache Syntaxdiagramme zu erstellen. Dazu sind einige Elemente
definiert worden, die automatisch durch Pfeile miteinander
verbunden werden.

Hierzu definiert das Paket \pkg{syntaxdi} einige TikZ-Stile, die
einfach genutzt werden können.

\subsection{TikZ-Stile}
\begin{options}
\opt{nonterminal} definiert ein Non-Terminal.
	\opt{terminal} definiert ein Terminal.
\opt{fnonterminal} definiert ein Non-Terminal ohne automatische
			Verzweigung.
\opt{fterminal} definiert ein Terminal ohne automatische
				Verzweigung.
\opt{point} definiert einen Punkt, der ohne ankommenden Pfeil
				gezeichnet wird.
\opt{endpoint} definiert einen Punkt, der mit ankommenden Pfeil
				gezeichnet wird.
\end{options}
		
\subsection{Beispiel}

Damit kann \zB\xspace das folgende Syntaxdiagramm gezeichnet werden.

\begin{example}[gobble=0]
\begin{tikzpicture}[syntaxdiagramm]
  \node [] {};
  \node [terminal] {if};
  \node [nonterminal] {Bedingung};
  \node [terminal] {:};
  \node [nonterminal] {Anweisungsblock};
  \node (ersteReiheEnde) [point] {};
  \node (ersteReiheEndeUnten) [point, below=of ersteReiheEnde] {};
  \node (zweiteReiheStartOben) [point, left=of ersteReiheEndeUnten,
         xshift=-75mm] {};
  \node (zweiteReiheStart) [point, below=of zweiteReiheStartOben] {};
  {
    [start chain=elif going right]
    \chainin (zweiteReiheStart);
    \node [terminal] {elif};
    \node [nonterminal] {Bedingung};
    \node [terminal] {:};
    \node [nonterminal] {Anweisungsblock};
    \node (elifEnde) [point] {};
    \node (elifEndeOben) [point, above=of elifEnde] {};
    \draw[->,left] (elifEndeOben) -- (ersteReiheEndeUnten);
  }
  \node (dritteReiheStart) [point, below=of zweiteReiheStart,
    yshift=-5mm] {}; 
  \node (vierteReiheStart) [point, below=of	dritteReiheStart,
    yshift=-5mm] {};
  \node (vierteReiheEnde) [point, xshift=84mm] {};
  {
    [start chain=else going right]
    \chainin (dritteReiheStart);
    \node [terminal] {else};
    \node [terminal] {:};
    \node (elseEnde) [nonterminal] {Anweisungsblock};
    \draw[->] (elseEnde) -| (vierteReiheEnde);
  }
  \node (ende) [endpoint] {};
\end{tikzpicture}
\end{example}
\section{Unicode-Symbole}
\label{paket:utfsym}

Im Schulkontext benötigt man häufig verschiedene Symbole, sei es
für die Kennzeichnung von Partner- und Gruppenarbeit, für die
Auflistung von Materialien oder einfach für die Verschönerung
von Arbeitsblättern.

Im Unicode sind einige tausend Symbole bereits definiert, von
denen sich viele für den Einsatz in der Schule aufdrängen. 
Leider ist der Einsatz in Latex nicht ohne weiteres möglich.

Deswegen wurde für das Schule-Paket das Zusatzpaket \pkg{utfsym}
entwickelt. Dieses ermöglicht die Verwendung von Zeichen aus
den folgenden Unicode-Blöcken:

\begin{smallitemize}
	\item Miscellaneous Symbols and Arrows (2600-26FF),
		siehe S.~\pageref{utab:02600-026FF}
	\item Dingbats (2700-27BF),
		siehe S.~\pageref{utab:02700-027BF}
	\item Mahjong Tiles (1F000-1F02F),
		siehe S.~\pageref{utab:1F000-1F02F}
	\item Domino Tiles (1F030-1F09F),
		siehe S.~\pageref{utab:1F030-1F09F}
	\item Playing Cards (1F0A0-1F0FF),
		siehe S.~\pageref{utab:1F0A0-1F0FF}
	\item Miscellaneous Symbols and Pictographs (1F300-1F5FF),
		siehe S.~\pageref{utab:1F300-1F5FF}
	\item Emoticons / Emoji (1F600-1F64F),
		siehe S.~\pageref{utab:1F600-1F64F}
	\item Transport and Map Symbols (1F680-1F6FF),
		siehe S.~\pageref{utab:1F680-1F6FF}
\end{smallitemize}

Die vollständigen Symboltabellen, mit
allen über 1600 unterstützten Symbolen, finden sich im Anhang, siehe
\prettyref{sec:unicodesymbole}.

Die entsprechenden Symbole können direkt in den LaTeX-Quelltext
eingefügt oder per Befehl eingebunden werden.

Die Symbole stammen dabei aus der Public-Domain-Schrift
\textbf{Symbola.ttf} von \person{George Douros}. Sie werden mit
TikZ gesetzt und passen sich jeweils an die Schriftgröße und
Farbe der Umgebung an, können aber auch als Bild mit
benutzerdefinierter Skalierung eingebunden werden.

\subsection{Befehle}
\begin{commands}
	\command{usym}[\marg{Code}]
	Bindet das Symbol mit dem gegebenen Code in das Dokument ein.
	Die Darstellung erfolgt im Fließtext. Größe und Farbe werden an
	den umgebenden Text angepasst.
\begin{sidebyside}
  \usym{1F642}
  \usym{1F609}
  \Huge\color{yellow}\usym{1F60A}
\end{sidebyside}
	\command{usymH}[\marg{Code}\marg{Höhe}]
	Bindet das Symbol mit dem gegebenen Code in das Dokument ein.
	Die Größe wird an die angegebene Höhe angepasst.
\begin{sidebyside}
  \usymH{1F0A1}{2cm}
\end{sidebyside}
	\command{usymW}[\marg{Code}\marg{Breite}]
	Bindet das Symbol mit dem gegebenen Code in das Dokument ein.
	Die Größe wird an die angegebene Breite angepasst.
\begin{sidebyside}
  \usymW{1F68C}{3cm}
\end{sidebyside}
\end{commands}
\section{Das alte Schule-Paket}
\label{paket:schulealt}

	Bei der mit Version 0.7 verbundenen Neuentwicklung wurde das alte 
	Schule-Paket in \pkg{schulealt} umbenannt und die Dokumentklassen
	entsprechend angepasst.
	
	Um zu vermeiden, dass alle alten Dokumente nicht mehr gesetzt
	werden können, wird das alte Paket mitgeliefert. Zudem wurde ein 
	Kompatibilitätsmodus implementiert, der dafür sorgt, dass das
	Einbinden des Schule-Pakets ohne die Paketoption \option{typ} dazu
	führt, dass das Laden des neuen Pakets frühzeitig unterbrochen
	und stattdessen das alte Paket geladen wird. Somit sollten alle alten Dokumente unverändert gesetzt werden können.
	
	Für die Befehle des alten Schule-Pakets kann in der Doku nachgesehen
	werden, die dem Zusatzpaket \pkg{schulealt} beiliegt.
	
	\achtung{Wahrscheinlich wird das alte Paket in der Zukunft entfernt
	werden. Es sollte also keinesfalls für neue Dokumente verwendet
	werden.}

\chapter{Questions fréquemment posées}\label{cha-faq}

Ce chapitre est une \gls{faq} \aside{autrement dit une liste des questions
  fréquemment posées} sur la \yatCl{}.

\section{Problèmes d'utilisation}

\begin{dbfaq}{Comment faire en cas de problème d'utilisation de la \yatCl{} ?}{probleme-utilisation}
  \index{problème d'utilisation}%
  La \yatCl{} est vraiment formidable, mais je rencontre un problème en
  l'utilisant.  Comment faire ?
  %
  \tcblower
  %
  En cas de problème d'utilisation\footnote{À ne pas confondre avec un
    bogue ou une fonctionnalité manquante, cf. \vref{faq-bogues}.} :
  \begin{enumerate}
  \item commencer par chercher s'il n'a pas déjà été signalé
    (et surtout solutionné) en consultant par exemple la liste des questions
    concernant la \yatCl{} sur les sites de questions \& réponses dédiés
    à \LaTeX{} :
      \begin{itemize}
      \item \url{http://texnique.fr/osqa/tags/yathesis/}\footnote{Site francophone.} ;
      \item
        \url{http://tex.stackexchange.com/questions/tagged/yathesis}\footnote{Site anglophone.} ;
      \end{itemize}
    \item s'il semble inédit (ou n'est pas \aside{ou mal} solutionné), poser
      soi-même une question sur un des lieux d'entraide dédiés à \LaTeX{}, par
      exemple sur l'un des sites ci-dessus\ecm{}.
    \end{enumerate}
\end{dbfaq}

\section{Communication}
\label{sec-communication}

\begin{dbfaq}{Comment communiquer avec l'auteur de la \yatCl{} ?}{bogues}
  \index{bogue}%
  \index{bogue!rapport}%
  \indexsee{bug}{bogue}%
  \index{fonctionnalité!demande}%
  La \yatCl{} est vraiment formidable, mais :
  \begin{enumerate}
  \item je souhaite signaler un dysfonctionnement (un bogue) ou suggérer une amélioration
    (par exemple en demandant une nouvelle fonctionnalité) ;
  \item je souhaite communiquer avec son auteur.
  \end{enumerate}
  Comment faire ?
  %
  \tcblower
  %
  \begin{enumerate}
  \item Pour un dysfonctionnement\footnote{À ne pas confondre avec un
      \enquote{simple} problème d'utilisation, cf. \vref{faq-probleme-utilisation}.}
    ou une amélioration :
    \begin{enumerate}
    \item avant de le signaler ou de la suggérer, s'assurer qu'ils n'ont pas
      déjà été répertoriés :
      \begin{enumerate}
      \item en consultant la liste de ceux qui le sont déjà\issues ;
      \item en lisant la suite du présent chapitre ;
      \item en lisant l'\vref{cha-incomp-conn} ;
        % des dysfonctionnements déjà répertoriés à l'adresse
        % \url{https://github.com/dbitouze/yathesis/issues?q=is%3Aopen+is%3Aissue+label%3Abug} ;
      \end{enumerate}
    \item s'ils n'ont pas déjà été répertoriés, signaler ce
      dysfonctionnement\ecm{} ou suggérer cette amélioration\newissues{}.
    \end{enumerate}
  % \item Pour une nouvelle fonctionnalité (ou suggestion d'amélioration) :
  %   \begin{enumerate}
  %   \item avant de la demander, s'assurer qu'elle n'a pas déjà été répertoriée :
  %     \begin{enumerate}
  %     \item en consultant la liste de celles qui le sont déjà\issues ;
  %       % \url{https://github.com/dbitouze/yathesis/issues?q=is%3Aopen+is%3Aissue+label%3Aenhancement} ;
  %     \item en lisant la suite du présent chapitre ;
  %     \end{enumerate}
  %   \item si elle n'a pas déjà été répertoriée, demander cette fonctionnalité
  %     (ou suggérer une amélioration)\newissues.
  %   \end{enumerate}
  \item Pour communiquer avec l'auteur de la classe, il est possible d'utiliser
    l'adresse indiquée à la page \url{https://github.com/dbitouze/yathesis/}.
  \end{enumerate}
\end{dbfaq}

\section{Avertissements}
\label{sec-avertissements}

\begin{dbfaq}{Puis-je ignorer un avertissement signalant une version trop
    ancienne d'un package ?}{}
  \index{avertissement de compilation}%
  \index{compilation!avertissement}%
  \index{package!ancien}%
  Je suis confronté à un avertissement de la forme \enquote{You have requested,
    on input line \meta{numéro}, version `\meta{date plus récente}' of package
    \meta{nom d'un package}, but only version `\meta{date moins récente} ...'
    is available.}. Est-ce grave, docteur ?
  %
  \tcblower
  %
  Ça peut être grave. Cf. \vref{rq-packages-anciens} pour plus de précisions.
\end{dbfaq}

\section{Erreurs}
\label{sec-erreurs}%
\index{erreur de compilation}%
\index{compilation!erreur}%

\begin{dbfaq}{Comment éviter l'erreur \enquote{Option clash for package
      \meta{package}} ?}{option-clash}
  Je suis confronté à l'erreur \enquote{Option clash for package
    \meta{package}} (notamment avec \meta{package}|=|\package{babel}). Comment
  l'éviter ?
  %
  \tcblower
  %
  Cette erreur est probablement due au fait que le \meta{package} a été
  manuellement chargé au moyen de la commande
  |\usepackage[...]{|\meta{package}|}|, alors que la \yatCl{} le charge déjà
  automatiquement (cf. l'\vref{sec-packages-charges-par} pour la liste des
  packages automatiquement chargés). Supprimer cette commande devrait résoudre
  le problème (cf. également l'\vref{wa-packages-a-ne-pas-charger}).
\end{dbfaq}

\begin{dbfaq}{Comment éviter l'erreur \enquote{Command
      \protect\docAuxCommand*{nobreakspace} unavailable in encoding T1} ?}{}
  Lorsque je compile ma thèse avec \hologo{XeLaTeX} ou \hologo{LuaLaTeX}, je
  suis confronté à l'erreur \enquote{Command
    \docAuxCommand*{nobreakspace} unavailable in encoding T1}. Comment
  l'éviter ?
  %
  \tcblower
  %
  (Cette question ne concerne pas directement la \yatCl{}.) Il suffit d'insérer,
  en préambule du fichier (maître) de la thèse, la
  ligne :
\begin{preamblecode}[title=Par exemple dans le \File{\configurationfile}]
\DeclareTextCommand{\nobreakspace}{T1}{\leavevmode\nobreak\ }
\end{preamblecode}
\end{dbfaq}

% \begin{dbfaq}{Comment éviter l'erreur \enquote{No room for a new%
%   \protect\docAuxCommand*{write}} ?}{}%
%   Je suis confronté à l'erreur \enquote{no room for a new%
%   \docAuxCommand{write}}. Comment l'éviter ?%
%  %
%   \tcblower%
%  %
%   Il devrait suffire de charger le \Package{morewrites} (plutôt parmi%
%   les premiers packages).%
%   \end{dbfaq}

\section{Mise en page}
\label{sec-mise-en-page}

\subsection{Pages de titre}
\label{sec-pages-de-titre}
\index{page de titre!mise en page}%
\index{page de titre!apparence}%

\begin{dbfaq}{Comment modifier l'apparence de la page de titre ?}{}
  L'apparence par défaut de la page de titre ne me convient pas et je voudrais
  la modifier. Comment faire ?
  %
  \tcblower
  %
  Il est prévu de permettre de modifier certains aspects de la mise en page de
  la page de titre, et même de fournir une documentation permettant d'obtenir
  une apparence complètement personnalisée, mais ce n'est pas encore
  implémenté.  En attendant que ça le soit, il faut composer cette page soit
  même, en y resaisissant manuellement toutes les caractéristiques nécessaires
  définies au \vref{cha-caract-du-docum}.
\end{dbfaq}

\subsection{Table des matières}
\label{sec-table-des-matieres-faq}

\begin{dbfaq}{Pourquoi les glossaire, listes d'acronymes et de symboles
    apparaissent en double dans la table des matières et dans les signets ?}{}
  \index{table des matières!globale!entrée en double}%
  \index{signets!entrée en double}%
  Les glossaire, listes d'acronymes et de symboles apparaissent en double dans
  la table des matières et dans les signets. Comment éviter cela ?%
  \tcblower
  %
  La \yatCl{} fait d'elle-même figurer les glossaire, listes d'acronymes et de
  symboles à la fois dans la table des matières et dans les signets. Pour
  régler le problème, il devrait donc suffire de \emph{ne pas} explicitement
  demander que ce soit le cas, en \emph{ne} recourant \emph{ni} à l'option
  \docAuxKey*{toc}, \emph{ni} à la commande \docAuxCommand*{glstoctrue} du
  \Package{glossaries}.
\end{dbfaq}

\begin{dbfaq}{Comment faire en sorte que, dans la table des matières, seuls
    les numéros de page soient des liens hypertextes ?}{}
  \index{table des matières!hyperlien}%
  J'ai chargé le \Package{hyperref} et, par défaut, les entrées de la table des
  matières sont toutes entières des liens hypertextes, ce qui est trop
  envahissant. Comment faire en sorte que seuls les numéros de page soient des
  liens hypertextes ?
  %
  \tcblower
  %
  (Cette question ne concerne pas directement la \yatCl{}.) Il suffit de passer
  l'option \lstinline|linktoc=false| au \Package{hyperref}.
\end{dbfaq}

\begin{dbfaq}{Comment supprimer la bibliographie des sommaire, table des
    matières et signets ?}{}
  \index{table des matières!globale!bibliographie}%
  \index{signets!bibliographie}%
  Par défaut, la bibliographie figure dans les sommaire, table des matières et
  signets du document. Comment éviter cela ?
  %
  \tcblower
  %
  (Cette question ne concerne pas directement la \yatCl{}.) Il suffit de passer
  à la commande \docAuxCommand{printbibliography} l'option
  |heading=|\meta{entête}, où \meta{entête} vaut par exemple
  \docValue*{bibliography} (cf. la documentation du \Package{biblatex} pour plus
  de détails).
\end{dbfaq}

\begin{dbfaq}{Comment affecter des profondeurs différentes aux signets et à la
    table des matières ?}{}
  \index{table des matières!globale!signet}%
  \index{table des matières!globale!profondeur}%
  \index{signets!profondeur}%
  \index{profondeur!signets}%
  Grâce au chargement du \Package{hyperref}, mon fichier \glsxtrshort{pdf} dispose
  de signets mais, par défaut, ceux-ci ont même niveau de profondeur que la
  table des matières. Comment leur affecter une profondeur différente ?
  %
  \tcblower
  %
  (Cette question ne concerne pas directement la \yatCl{}.) L'option
  \docAuxKey{depth} du \Package*{bookmark} permet d'affecter aux signets un
  autre niveau que celui par défaut.
\begin{preamblecode}[title=Par exemple dans le \File{\configurationfile}]
\bookmarksetup{depth="\meta{autre niveau}"}
\end{preamblecode}
où \meta{autre niveau} est l'une des valeurs possibles de la clé
\refKey{depth}.
\end{dbfaq}

\begin{dbfaq}{Dans la table des matières, des numéros de pages débordent dans
    la marge de droite}{}
  \index{table des matières!globale!débordement dans la marge}%
  Dans la table des matières, certains numéros de pages (en chiffres romains
  notamment) débordent dans la marge de droite. Comment l'éviter ?
  %
  \tcblower
  %
  Il suffit d'insérer, en préambule du fichier (maître) de la thèse, les
  lignes :
\begin{preamblecode}[title=Par exemple dans le \File{\configurationfile}]
\makeatletter
\renewcommand*\@pnumwidth{"\meta{distance}"}
\makeatother
\end{preamblecode}
  où \meta{distance}, à exprimer par exemple en points (par exemple |27pt|),
  est à déterminer par \enquote{essais/erreurs} de sorte que \meta{distance}
  soit :
  \begin{enumerate}
  \item suffisamment grande, pour empêcher les débordements de numéros de
    pages ;
  \item aussi petite que possible, pour éviter les lignes de pointillés trop
    courtes.
  \end{enumerate}
\end{dbfaq}

\subsection[Titres courants]{\texorpdfstring{\Glsentryplural{titrecourant}}{Titres courants}}
\label{sec-titres-courants}

\glsreset{tdm}
\begin{dbfaq}{Est-il possible d'obtenir des \glsplural{titrecourant} distincts
    des titres figurant en table(s) des matières ?}{}
  \index{table des matières!entrée différente du titre courant}%
  \index{titre courant!différent de l'entrée en table des matières}%
  %
  Les titres des chapitres et des sections sont reproduits en
  \glsplural{titrecourant} et, si certains d'entre eux sont longs au point de
  déborder de l'entête, je peux recourir à l'argument optionnel des commandes
  \docAuxCommand{chapter} et \docAuxCommand{section} pour qu'ils y soient
  remplacés par des titres alternatifs \enquote{courts}.

  Mais ce remplacement a alors lieu \emph{aussi} en \gls{tdm}, et cela me pose
  problème dans les cas suivants.
  % Mais alors les titres \enquote{normaux} de ces chapitres sont remplacés par
  % leurs versions alternatives \enquote{courtes} \emph{également} en table(s) des
  % matières, et cela me pose problème dans les cas suivants.
  \begin{description}
  \item[Cas 1.] Je souhaite que, en \gls{tdm}, figurent systématiquement les
    titres \enquote{normaux}, et pas d'éventuels titres alternatifs
    \enquote{courts}.
  \item[Cas 2.] Je souhaite que les titres alternatifs \enquote{courts} des
    titres des chapitres puissent être différentes en \gls{tdm} et en
    \glsplural{titrecourant}.
  \end{description}
  Est-il donc possible d'obtenir des \glsplural{titrecourant} distincts des
  titres figurant en \gls{tdm} ?%
  \tcblower
  %
  Il suffit de recourir à l'argument optionnel supplémentaire des commandes
  \refCom{chapter} et \refCom{section} fourni par la \yatCl{}
  (cf. \vref{sec-intit-altern}).
%   (Cette question ne concerne pas directement la \yatCl{}.) Il suffit de
%   recourir à la commande \docAuxCommand{chaptermark}, comme suit.
%   \begin{description}
%   \item[Cas 1.]\
% \begin{bodycode}[listing options={deletekeywords={chapter},deletekeywords={[3]chapter}}]
% \chapter{"\meta{titre}"}
% \chaptermark{"\meta{titre alternatif en titre courant}"}
% \end{bodycode}
%   \item[Cas 2.]\
% \begin{bodycode}[listing options={deletekeywords={chapter},deletekeywords={[3]chapter}}]
% \chapter["\meta{titre alternatif en table(s) des matières}"]{"\meta{titre}"}
% \chaptermark{"\meta{titre alternatif en titre courant}"}
% \end{bodycode}
%   \end{description}
%   La commande analogue \docAuxCommand{sectionmark} peut être utilisée si le
%   problème se pose pour les sections.
\end{dbfaq}

\subsection{Divers}
\label{sec-divers}

\begin{dbfaq}{Pourquoi mes signes de ponctuation haute ne sont pas précédés des
    espaces adéquates ?}{}
  \index{espace!avant \enquote{?;:"!}}%
  Certains éléments que j'ai saisis en préambule contiennent des signes de
  ponctuation haute ({\NoAutoSpacing?;:!}) mais, dans le \pdf{} produit, ces
  derniers ne sont pas précédés des espaces adéquates. Comment régler ce
  problème ?
  %
  \tcblower
  %
  (Cette question ne concerne pas directement la \yatCl{}.) Le problème est dû
  aux caractères actifs du module \package*+{babel-french} du
  \Package{babel}. Si ces éléments concernent :
  \begin{enumerate}
  \item les caractéristiques du document (cf. \vref{cha-caract-du-docum}), il
    suffit de les saisir\footnote{Cf. \vref{sec-lieu-de-saisie}.} :
    \begin{itemize}
    \item soit dans le \emph{corps} du fichier (maître) de la
      thèse\footnote{Mais cf. alors \vref{wa-avant-maketitle}.} (et donc
      \emph{pas} dans son \emph{préambule}) ;
    \item soit dans le \File{\characteristicsfile} prévu à cet effet ;
    \item soit entre |\shorthandon{;:!?}| et |\shorthandoff{;:!?}| si on tient
      absolument à ce qu'ils soient saisis en préambule.
    \end{itemize}
  \item les termes du glossaire, des acronymes ou des symboles, il suffit de
    définir les entrées correspondantes ou d'utiliser la ou les commandes
    \docAuxCommand{loadglsentries} :
    \begin{itemize}
    \item soit dans le \File{\configurationfile}
      (cf. \vref{rq-configurationfile}) ;
    \item soit entre |\shorthandon{;:!?}| et |\shorthandoff{;:!?}|. Cette
      solution peut être préférée à la précédente pour ne pas perdre les
      fonctionnalités de complétion pour les labels des termes de glossaire
      fournies par certains éditeurs de texte orientés \LaTeX{}.
    \end{itemize}
  \end{enumerate}
\end{dbfaq}

\begin{dbfaq}{Pourquoi \protect\docAuxCommand*{setcounter} n'a-t-elle pas
    d'effet sur \protect\docAuxKey*{secnumdepth} ?}{}
  \index{profondeur!numérotation des unités}%
  \index{numérotation!des unités!profondeur}%
  J'essaie de modifier la profondeur de numérotation de mon document en
  spécifiant la valeur du compteur \docAuxKey*{secnumdepth} au moyen de la
  commande :
\begin{preamblecode}
\setcounter{secnumdepth}{"\meta{nombre}"}
\end{preamblecode}
  mais cela n'a aucun effet. Pourquoi ?
  %
  \tcblower
  %
  La profondeur de numérotation d'un document composé avec la \yatCl{} est
  à spécifier au moyen de l'option de classe
  \refKey{secnumdepth}. Cf. \vref{sec-profondeur-de-la} pour plus de
  précisions.
\end{dbfaq}

\section{Validation}
\label{sec-validation}

\begin{dbfaq}{Le \glsxtrshort{pdf} de mon mémoire n'est pas valide au yeux du
    \glsxtrshort{cines}. Comment y remédier ?}{}
  \index{pdf@\glsxtrshort{pdf}!valide}
  \index{validité!pdf@\glsxtrshort{pdf}}
  Conformément aux dispositions propres au dépôt sur support électronique
  \autocite{guidoct-abes}, j'ai testé sur le site \url{http://facile.cines.fr/}
  la validité du fichier \glsxtrshort{pdf} de mon mémoire de thèse créé avec la
  \yatCl{}, et il s'avère que celui-ci n'est pas valide. Comment y remédier ?
  %
  \tcblower
  %
  (Cette question ne concerne pas directement la \yatCl{}.) Le problème vient de
  ce que le site \url{http://facile.cines.fr/} reconnaît mal les méta-données
  des fichiers \glsxtrshort{pdf} produits par \hologo{pdfLaTeX}, \hologo{XeLaTeX} ou
  \hologo{LuaLaTeX}.
  %
  Pour pallier cela, il devrait suffire\footnote{Plus de précisions à l'adresse
    \url{https://facile.cines.fr/\#latex}.} d'insérer en introduction du fichier
  (maître) \file{.tex}, avant même la déclaration
  \docAuxCommand{documentclass} :
\begin{preamblecode}
\pdfobjcompresslevel 0
\end{preamblecode}
\end{dbfaq}

%%% Local Variables:
%%% mode: latex
%%% TeX-master: "../yathesis-fr"
%%% End:


\part{Beispiele}

\section{Minimalbeispiele}
    \Beispieldatei{minimal-ab}{Arbeitsblatt}
    \Beispieldatei{minimal-ib}{Informationsblatt}
    \Beispieldatei{minimal-kl}{Klausur}
    \Beispieldatei{minimal-ka}{Klassenarbeit}
    \Beispieldatei{beispiel-leitprogramm}{Leitprogramm}

\section{Praxisbeispiele}
    \subsection{Geschichte}
        \Beispieldatei{bsp-geschichte}{Beispieldokument Geschichte}
        \Beispieldatei{beispiel-ib-hieroglyphen}{Informationsblatt Hieroglyphen}

    \subsection{Informatik}
        \Beispieldatei{beispiel-ab-abbott}{Arbeitsblatt Fahrkartenauskunft}
        \Beispieldatei{beispiel-kl}{Klausur Objektorientierte Modellierung}

    \subsection{Physik}
        \Beispieldatei{beispiel-ab-schiefeebene}{Arbeitsblatt Schiefe Ebene}



\part{Entwicklungsdokumentation}
\section{Lizenzen}
	\license
	
	Einzelne Code-Beispiele in dieser Dokumentation stammen aus der Materialsammlung Informatik\footnote{\materialsammlung} und unterliegen damit der CC-BY-NC-SA (Creative Commons Attribution-NonCommercial-ShareAlike License).
	
	Das beiliegende Zusatzpaket \pkg{utfsym} von \person{Daniel Spittank} fällt unter die Public Domain im Sinne der CC0 (Creative Commons Zero).
\section{Richtlinien}
    Um die Quelltexte des Pakets über einen längeren Zeitraum halbwegs konsistent und dokumentiert zu halten und somit die leichtere  Einarbeitung neuer Betreuer zu ermöglichen, wurden folgende Richtlinien vereinbart:
    \begin{itemize}
        \item Bei sämtliche Klassen- und Paketquellen sollten lange Zeilen vermieden werden.
        \item Bei der Dokumentation (dokumentation.tex) soll kein automatischer Umbruch im Dokument stattfinden, um Änderungen im Text einfacher nachhalten zu können.
        \item Auf Einrückungen sollte geachtet werden. Zeilenumbrüche  sind ggf. entsprechend auszukommentieren.
        \item Einrückungen erfolgen mit Leerzeichen in einer Weite von 4.
        \item Alle internen Makros und Variablen sollen im Namensraum  \code{schule\cnltxat} stehen, \zB\space\code{schule\cnltxat  ergebnishorizontAnzeigen}.
        \item Sprechende Bezeichner sollten verwendet werden.
        \item Für neue Funktionen sollte ein neues Modul oder ein neuer Dokumenttyp angelegt werden, sofern es sich um keine klare Ergänzung handelt.
        \item Neue Funktionen und Änderungen sind in der Dokumentation mit den Mitteln von \pkg{cnltx-doc} zu kennzeichnen, \zB\space mit \cs{changedversion}\marg{version} für Änderungen oder \cs{sinceversion}\marg{version} für neue Funktionen. Außerdem sind sie im Changelog zu vermerken.
    \end{itemize}
\section{Modularität}
	\label{sec:devmodule}
	\subsection{Erläuterungen zum Modulsystem}
	Ein zentrales Problem des alten Schule-Pakets bis Version 0.6
	war, dass es sehr monolithisch aufgebaut und alles integrierte,
	was im Schulalltag und in der Lehrerausbildung nützlich sein
	könnte. So wurden die Weiterentwicklung und Übergabe an neue
	Maintainer schwierig, da stets eine Einarbeitung in alle Bereiche
	erforderlich war.

	Die grundlegende Entscheidung für das neue Schule-Paket ist es
	also, eine Modularisierung zu etablieren, die, zusätzlich zu
	einem stabilen Kern, verschiedene Funktionen und Fachspezifika
	entsprechend kapselt und von diesem Kern trennt.

	Ein eindeutiger Aufbau dieser Module soll dafür sorgen, dass es
	leichter wird, neue Funktionen zu ergänzen, ohne den gesamten
	Quelltext des Pakets kennen und verstehen zu müssen.

	Grundsätzlich gibt es drei verschiedene Arten von Modulen im
	Schule-Paket: \textbf{Module}, \textbf{Fachmodule} und
	\textbf{Zusatzpakete}, \vgl\space \prettyref{sec:begriffe}.

	Erweiterungen ohne direkten Bezug (also alles, das auch ohne das
	Schule-Paket sinnvoll genutzt werden kann) zum Paket sollten in
	Form unabhängiger Zusatzmodule implementiert werden.

	\subsection{Aufbau eines Moduls}

	\label{sec:devmoduleModul}

    Ein Modul für das Schule-Paket besteht aus mehreren Dateien, deren erster Teil des Namens \texttt{schule.mod.Modulname} identisch ist. Je nach Funktion werden dort drei Möglichkeiten angehangen, die als Abschnitte bezeichnet werden. Sobald eine der Dateien mit diesem Schema vorhanden ist, kann das Modul über seinen Namen eingebunden werden, \vgl\space\prettyref{sec:module}. Alle Dateien vorgegebenen Dateien liegen dazu im Verzeichnis \enquote{latex}.

    Der Modulmechanismus sorgt dafür, dass die entsprechenden Abschnitte des Moduls an den richtigen Stellen des Quelltextes eingebunden werden. Es sind die folgenden drei Abschnitte definiert:

    \begin{description}
        \item[\texttt{optionen.tex}] $\rightarrow$ Paketoptionen des Moduls, \vgl\space Paket \pkg{pgfopts}
        \item[\texttt{pakete.tex}] $\rightarrow$ Paketabhängigkeiten des Moduls
        \item[\texttt{code.tex}] $\rightarrow$ Implementierung des Moduls
    \end{description}

    All diese Abschnitte sind optional und werden geladen, wenn sie vorhanden sind. Ein Modul kann also beispielsweise nur aus einer \texttt{code.tex}-Datei bestehen, wenn es nur einige Makros definiert.

	\subsubsection{Beispiel}
	\label{sec:devmoduleModulBeispiel}
	\paragraph{schule.mod.HalloWelt.optionen.tex}
\begin{sourcecode}[gobble=0]
% ********************************************************************
% * Paketoptionen                                                    *
% ********************************************************************

% Boolesche Optionen
% ********************************************************************
\newboolean{schule@nutzeGoodbye}

% Standardwerte
% ********************************************************************
\newcommand{\schule@weltname}{Welt}

% Definition der Paketoptionen
% ********************************************************************
\pgfkeys{
  /schule/.cd,
  weltname/.store in=\schule@weltname,
  nutzeGoodbye/.value forbidden,
  nutzeGoodbye/.code=\setboolean{schule@datumAnzeigen}{true},
}
\end{sourcecode}

	\paragraph{schule.mod.HalloWelt.pakete.tex}
\begin{sourcecode}[gobble=0]
% ********************************************************************
% * Paketabhängigkeiten                                              *
% ********************************************************************

\RequirePackage{ifthenelse}
\end{sourcecode}

	\paragraph{schule.mod.HalloWelt.code.tex}
\begin{sourcecode}[gobble=0]
% ********************************************************************
% * Hallo Welt!                                                      *
% ********************************************************************

\newcommand{\halloWelt}{
  \ifthenelse{\boolean{schule@nutzeGoodbye}}{
    Goodbye \schule@weltname!
  }{
    Hallo \schule@weltname!
  }
}
\end{sourcecode}


    \subsection{Aufbau eines Fachmoduls}

    \label{sec:devmoduleFachmodul}

    Ein Fachmodul für das Schule-Paket wird analog zu einem normalen Modul erstellt, \vgl\space\prettyref{sec:devmoduleModul}. Hier steht nur im Dateinamen nach dem zweiten Punkt \texttt{fach} anstatt \texttt{mod}.

    Die Dateien für das Fach Geschichte sind dann beispielsweise folgende:
    \begin{itemize}
        \item \texttt{schule.fach.Geschichte.optionen.tex}
        \item \texttt{schule.fach.Geschichte.pakete.tex}
        \item \texttt{schule.fach.Geschichte.code.tex}
    \end{itemize}


    \subsection{Aufbau eines Dokumenttyps}

    \label{sec:devDokumenttyp}

    Auch ein Dokumenttyp für das Schule-Paket hat den gleichen Aufbau wie ein normales Modul, \vgl\space\prettyref{sec:devmoduleModul}. Hier steht nur im Dateinamen nach dem zweiten Punkt \texttt{typ} anstatt \texttt{mod}.

    Die Dateien für das Arbeitsblatt (Typ ab) sind dann beispielsweise folgende:
    \begin{itemize}
        \item \texttt{schule.typ.ab.optionen.tex}
        \item \texttt{schule.typ.ab.pakete.tex}
        \item \texttt{schule.typ.ab.code.tex}
    \end{itemize}

\section{Funktionen für Entwickler}
	\subsection{Fehlerbehandlung und Debugging}
		\subsubsection{Paketoptionen}
			\begin{options}
				\opt{debug}
				schaltet das Paket in den Debugmodus. Verhindert zudem
				die Unterdrückung ungefährlicher Warnungen, die
				standardmäßig aktiv ist.
				
				Diese Paketoption setzt zudem die boolesche Variable 
				\code{schule\cnltxat debug} auf \code{true}, sodass
				jederzeit geprüft werden kann, ob der Debugmodus aktiv
				ist oder nicht.
			\end{options}
		\subsubsection{Befehle}
			\begin{commands}
				\command{sinfo}[\marg{Text}]
					schreibt den angegebenen Text in die Logdatei.
				\command{swarnung}[\marg{Text}]
					schreibt den angegebenen Text als Warnung in die
					Logdatei.
				\command{sfehler}[\marg{Text}]
					schreibt den angegebenen Text als Fehler in die
					Logdatei und beendet die Kompilierung.
				\command{sdinfo}[\marg{Text}]
					schreibt den angegebenen Text in die Logdatei,
					falls der Debugmodus aktiv ist.
				\command{sdwarnung}[\marg{Text}]
					schreibt den angegebenen Text als Warnung in die
					Logdatei, falls der Debugmodus aktiv ist.
			\end{commands}
	
	\subsection{Interne Makros}
		\subsubsection{Befehle}
			\begin{commands}
				\command{schule\cnltxat kopfUmbruch}
					liefert einen Zeilenumbruch, wenn die Kopfzeile an
					einer Stelle zweizeilig ist, \zB\space durch die
					Darstellung eines Namensfelds oder die Anzeige
					des Datums.
				\command{schule\cnltxat modulDateiLaden}[
				\marg{Kategorie}\marg{Modulname}\marg{Abschnitt}]
				Lädt einen Abschnitt eines Moduls. Interne
				Hilfsfunktion für das Laden von Modulen. 
				
				\achtung{Sollte nicht manuell verwendet werden,
				sondern nur an den entsprechenden Stellen der
				\code{schule.sty}.}
				
				\achtung{Kann mit viel Sorgfalt zur Erfüllung von 
				Abhängigkeiten zwischen Modulen genutzt werden.
				Es ist darauf zu achten, dass dann zu Beginn der
				Abschnitte des ladenden Moduls die entsprechenden
				Abschnitte des zu ladenden Moduls eingebunden 
				werden.
				
				Dies führt dazu, dass die Ladereihenfolge der Module
				nicht mehr sichergestellt werden kann.}
				
				\command{schule@modulNachladen}[\marg{Modulname}]
				Lädt ein Modul mit allen Abschnitten. Interne
				Hilfsfunktion für die Erfüllung von Abhängigkeiten in
				Modulen. 
				
				\achtung{Die Verwendung zu anderen Zwecken wird nicht
				empfohlen, da die Reihenfolge der zu ladenden Module
				hier nicht beachtet werden kann. Ebenfalls kann so
				nicht sichergestellt werden, dass der geladene Code
				an der richtigen Stelle im Quelltext landet, vielmehr
				wird er genau an der Stelle des Befehls geladen.}
			\end{commands}
%DO NOT EDIT THIS AUTOMATICALLY GENERATED FILE, run "make changelog" at toplevel!!!
The major changes among the different circuitikz versions are listed
here. See \url{https://github.com/circuitikz/circuitikz/commits} for a
full list of changes.

\begin{itemize}
\item
  Version 1.1.1 (2020-04-24)

  One-line bugfix release for the IEEE ports ``not'' circle thickness
\item
  Version 1.1.0 (2020-04-19)

  Version bumped to 1.1 because the new logic ports are quite a big
  addition: now there is a new style for logic ports, conforming to IEEE
  recommendations.

  Several minor additions all over the map too.

  \begin{itemize}
  \tightlist
  \item
    added IEEE standard logic ports suggested by user Jason-s on GitHub
  \item
    added configurability to european logic port ``not'' output symbol,
    suggested by j-hap on GitHub
  \item
    added \texttt{inerter} component by user Tadashi on GitHub
  \item
    added variable outer base height for IGBT, suggested by user RA-EE
    on GitHub
  \item
    added configurable ``+'' and ``-'' signs on american-style voltage
    generators
  \item
    text on amplifiers can be positioned to the left or centered
  \end{itemize}
\item
  Version 1.0.2 (2020-03-22)

  \begin{itemize}
  \tightlist
  \item
    added Schottky transistors (thanks to a suggestion by Jérôme
    Monclard on GitHub)
  \item
    fixed formatting of \texttt{CHANGELOG.md}
  \end{itemize}
\item
  Version 1.0.1 (2020-02-22)

  Minor fixes and addition to 1.0, in time to catch the freeze for
  TL2020.

  \begin{itemize}
  \tightlist
  \item
    add v1.0 version snapshots
  \item
    added crossed generic impedance (suggested by Radványi Patrik Tamás)
  \item
    added open barrier bipole (suggested by Radványi Patrik Tamás)
  \item
    added two flags to flip the direction of light's arrows on LED and
    photodiode (suggested by karlkappe on GitHub)
  \item
    added a special key to help with precision loss in case of
    fractional scaling (thanks to AndreaDiPietro92 on GitHub for
    noticing and reporting, and to Schrödinger's cat for finding a fix)
  \item
    fixed a nasty bug for the flat file generation for ConTeXt
  \end{itemize}
\item
  Version 1.0 (2020-02-04)

  And finally\ldots{} version 1.0 (2020-02-04) of \texttt{circuitikz} is
  released.

  The main updates since version 0.8.3, which was the last release
  before Romano started co-maintaining the project, are the following
  --- part coded by Romano, part by several collaborators around the
  internet:

  \begin{itemize}
  \tightlist
  \item
    The manual has been reorganized and extended, with the addition of a
    tutorial part; tens of examples have been added all over the map.
  \item
    Around 74 new shapes where added. Notably, now there are chips,
    mux-demuxes, multi-terminal transistors, several types of switches,
    flip-flops, vacuum tubes, 7-segment displays, more amplifiers, and
    so on.
  \item
    Several existing shapes have been enhanced; for example, logic gates
    have a variable number of inputs, transistors are more configurable,
    resistors can be shaped more, and more.
  \item
    You can style your circuit, changing relative sizes, default
    thickness and fill color, and more details of how you like your
    circuit to look; the same you can do with labels (voltages,
    currents, names of components and so on).
  \item
    A lot of bugs have been squashed; especially the (very complex)
    voltage direction conundrum has been clarified and you can choose
    your preferred style here too.
  \end{itemize}
\end{itemize}

A detailed list of changes can be seen below.

\begin{itemize}
\item
  Version 1.0.0-pre3 (not released)

  \begin{itemize}
  \tightlist
  \item
    Added a Reed switch
  \item
    Put the copyright and license notices on all files and update them
  \item
    Fixed the loading of style; we should not guard against reload
  \end{itemize}
\item
  Version 1.0.0-pre2 (2020-01-23)

  \textbf{Really} last additions toward the 1.0.0 version. The most
  important change is the addition of multiplexer and de-multiplexers;
  also added the multi-wires (bus) markers.

  \begin{itemize}
  \tightlist
  \item
    Added mux-demux shapes
  \item
    Added the possibility to suppress the input leads in logic gates
  \item
    Added multiple wires markers
  \item
    Added a style to switch off the automatic rotation of instruments
  \item
    Changed the shape of the or-type american logic ports (reversible
    with a flag)
  \end{itemize}
\item
  Version 1.0.0-pre1 (2019-12-22)

  Last additions before the long promised 1.0! In this pre-release we
  feature a flip-flop library, a revamped configurability of amplifiers
  (and a new amplifier as a bonus) and some bug fix around the clock.

  \begin{itemize}
  \tightlist
  \item
    Added a flip-flop library
  \item
    Added a single-input generic amplifier with the same dimension as
    ``plain amp''
  \item
    Added border anchors to amplifiers
  \item
    Added the possibility (expert only!) to add transparency to poles
    (after a suggestion from user @matthuszagh on GitHub)
  \item
    Make plus and minus symbol on amplifiers configurable
  \item
    Adjusted the position of text in triangular amplifiers
  \item
    Fixed ``plain amp'' not respecting ``noinv input up''
  \item
    Fixed minor incompatibility with ConTeXt and Plain TeX
  \end{itemize}
\item
  Version 0.9.7 (2019-12-01)

  The important thing in this release is the new position of
  transistor's labels; see the manual for details.

  \begin{itemize}
  \tightlist
  \item
    Fix the position of transistor's text. There is an option to revert
    to the old behavior.
  \item
    Added anchors for adding circuits (like snubbers) to the flyback
    diodes in transistors (after a suggestion from @EdAlvesSilva on
    GitHub).
  \end{itemize}
\item
  Version 0.9.6 (2019-11-09)

  The highlights of this release are the new multiple terminals BJTs and
  several stylistic addition and fixes; if you like to pixel-peep, you
  will like the fixed transistors arrows. Additionally, the transformers
  are much more configurable now, the ``pmos'' and ``nmos'' elements
  have grown an optional bulk connection, and you can use the ``flow''
  arrows outside of a path.

  Several small and less small bugs have been fixed.

  \begin{itemize}
  \tightlist
  \item
    Added multi-collectors and multi-emitter bipolar transistors
  \item
    Added the possibility to style each one of the two coils in a
    transformer independently
  \item
    Added bulk connection to normal MOSFETs and the respective anchors
  \item
    Added ``text'' anchor to the flow arrows, to use them alone in a
    consistent way
  \item
    Fixed flow, voltage, and current arrow positioning when ``auto'' is
    active on the path
  \item
    Fixed transistors arrows overshooting the connection point, added a
    couple of anchors
  \item
    Fixed a spelling error on op-amp key ``noinv input down''
  \item
    Fixed a problem with ``quadpoles style=inner'' and ``transformer
    core'' having the core lines running too near
  \end{itemize}
\item
  Version 0.9.5 (2019-10-12)

  This release basically add features to better control labels, voltages
  and similar text ``ornaments'' on bipoles, plus some other minor
  things.

  On the bug fixes side, a big incompatibility with ConTeXt has been
  fixed, thanks to help from \texttt{@TheTeXnician} and \texttt{@hmenke}
  on \texttt{github.com}.

  \begin{itemize}
  \tightlist
  \item
    Added a ``midtap'' anchor for coils and exposed the inner coils
    shapes in the transformers
  \item
    Added a ``curved capacitor'' with polarity coherent with
    ``ecapacitor''
  \item
    Added the possibility to apply style and access the nodes of
    bipole's text ornaments (labels, annotations, voltages, currents and
    flows)
  \item
    Added the possibility to move the wiper in resistive potentiometers
  \item
    Added a command to load and set a style in one go
  \item
    Fixed internal font changing commands for compatibility with ConTeXt
  \item
    Fixed hardcoded black color in ``elko'' and ``elmech''
  \end{itemize}
\item
  Version 0.9.4 (2019-08-30)

  This release introduces two changes: a big one, which is the styling
  of the components (please look at the manual for details) and a change
  to how voltage labels and arrows are positioned. This one should be
  backward compatible \emph{unless} you used \texttt{voltage\ shift}
  introduced in 0.9.0, which was broken when using the global
  \texttt{scale} parameter.

  The styling additions are quite big, and, although in principle they
  are backward compatible, you can find corner cases where they are not,
  especially if you used to change parameters for
  \texttt{pgfcirc.defines.tex}; so a snapshot for the 0.9.3 version is
  available.

  \begin{itemize}
  \tightlist
  \item
    Fixed a bug with ``inline'' gyrators, now the circle will not
    overlap
  \item
    Fixed a bug in input anchors of european not ports
  \item
    Fixed ``tlinestub'' so that it has the same default size than
    ``tline'' (TL)
  \item
    Fixed the ``transistor arrows at end'' feature, added to styling
  \item
    Changed the behavior of ``voltage shift'' and voltage label
    positioning to be more robust
  \item
    Added several new anchors for ``elmech'' element
  \item
    Several minor fixes in some component drawings to allow fill and
    thickness styles
  \item
    Add 0.9.3 version snapshots.
  \item
    Added styling of relative size of components (at a global or local
    level)
  \item
    Added styling for fill color and thickeness
  \item
    Added style files
  \end{itemize}
\item
  Version 0.9.3 (2019-07-13)

  \begin{itemize}
  \tightlist
  \item
    Added the option to have ``dotless'' P-MOS (to use with arrowmos
    option)
  \item
    Fixed a (puzzling) problem with coupler2
  \item
    Fixed a compatibility problem with newer PGF (\textgreater{}3.0.1a)
  \end{itemize}
\item
  Version 0.9.2 (2019-06-21)

  \begin{itemize}
  \tightlist
  \item
    (hopefully) fixed ConTeXt compatibility. Most new functionality is
    not tested; testers and developers for the ConTeXt side are needed.
  \item
    Added old ConTeXt version for 0.8.3
  \item
    Added tailless ground
  \end{itemize}
\item
  Version 0.9.1 (2019-06-16)

  \begin{itemize}
  \tightlist
  \item
    Added old LaTeX versions for 0.8.3, 0.7, 0.6 and 0.4
  \item
    Added the option to have inline transformers and gyrators
  \item
    Added rotary switches
  \item
    Added more configurable bipole nodes (connectors) and more shapes
  \item
    Added 7-segment displays
  \item
    Added vacuum tubes by J. op den Brouw
  \item
    Made the open shape of dcisources configurable
  \item
    Made the arrows on vcc and vee configurable
  \item
    Fixed anchors of diamondpole nodes
  \item
    Fixed a bug (\#205) about unstable anchors in the chip components
  \item
    Fixed a regression in label placement for some values of scaling
  \item
    Fixed problems with cute switches anchors
  \end{itemize}
\item
  Version 0.9.0 (2019-05-10)

  \begin{itemize}
  \tightlist
  \item
    Added Romano Giannetti as contributor
  \item
    Added a CONTRIBUTING file
  \item
    Added options for solving the voltage direction problems.
  \item
    Adjusted ground symbols to better match ISO standard, added new
    symbols
  \item
    Added new sources (cute european versions, noise sources)
  \item
    Added new types of amplifiers, and option to flip inputs and outputs
  \item
    Added bidirectional diodes (diac) thanks to Andre Lucas Chinazzo
  \item
    Added L,R,C sensors (with european, american and cute variants)
  \item
    Added stacked labels (thanks to the original work by Claudio
    Fiandrino)
  \item
    Make the position of voltage symbols adjustable
  \item
    Make the position of arrows in FETs and BJTs adjustable
  \item
    Added chips (DIP, QFP) with a generic number of pins
  \item
    Added special anchors for transformers (and fixed the wrong center
    anchor)
  \item
    Changed the logical port implementation to multiple inputs (thanks
    to John Kormylo) with border anchors.
  \item
    Added several symbols: bulb, new switches, new antennas,
    loudspeaker, microphone, coaxial connector, viscoelastic element
  \item
    Make most components fillable
  \item
    Added the oscilloscope component and several new instruments
  \item
    Added viscoelastic element
  \item
    Added a manual section on how to define new components
  \item
    Fixed american voltage symbols and allow to customize them
  \item
    Fixed placement of straightlabels in several cases
  \item
    Fixed a bug about straightlabels (thanks to @fotesan)
  \item
    Fixed labels spacing so that they are independent on scale factor
  \item
    Fixed the position of text labels in amplifiers
  \end{itemize}
\item
  Version 0.8.3 (2017-05-28)

  \begin{itemize}
  \tightlist
  \item
    Removed unwanted lines at to-paths if the starting point is a node
    without a explicit anchor.
  \item
    Fixed scaling option, now all parts are scaled by bipoles/length
  \item
    Surge arrester appears no more if a to path is used without
    {[}{]}-options
  \item
    Fixed current placement now possible with paths at an angle of
    around 280°
  \item
    Fixed voltage placement now possible with paths at an angle of
    around 280°
  \item
    Fixed label and annotation placement (at some angles position not
    changable)
  \item
    Adjustable default distance for straight-voltages:
    `bipoles/voltage/straight label distance'
  \item
    Added Symbol for bandstop filter
  \item
    New annotation type to show flows using f=\ldots{} like currents,
    can be used for thermal, power or current flows
  \end{itemize}
\item
  Version 0.8.2 (2017-05-01)

  \begin{itemize}
  \tightlist
  \item
    Fixes pgfkeys error using alternatively specified mixed colors(see
    pgfplots manual section ``4.7.5 Colors'')
  \item
    Added new switches ``ncs'' and ``nos''
  \item
    Reworked arrows at spst-switches
  \item
    Fixed direction of controlled american voltage source
  \item
    ``v\textless{}='' and ``i\textless{}='' do not rotate the sources
    anymore(see them as ``counting direction indication'', this can be
    different then the shape orientation); Use the option ``invert'' to
    change the direction of the source/apperance of the shape.
  \item
    current label ``i='' can now be used independent of the regular
    label ``l='' at current sources
  \item
    rewrite of current arrow placement. Current arrows can now also be
    rotated on zero-length paths
  \item
    New DIN/EN compliant operational amplifier symbol ``en amp''
  \end{itemize}
\item
  Version 0.8.1 (2017-03-25)

  \begin{itemize}
  \tightlist
  \item
    Fixed unwanted line through components if target coordinate is a
    name of a node
  \item
    Fixed position of labels with subscript letters.
  \item
    Absolute distance calculation in terms of ex at rotated labels
  \item
    Fixed label for transistor paths (no label drawn)
  \end{itemize}
\item
  Version 0.8 (2017-03-08)

  \begin{itemize}
  \tightlist
  \item
    Allow use of voltage label at a {[}short{]}
  \item
    Correct line joins between path components (to{[}\ldots{}{]})
  \item
    New Pole-shape .-. to fill perpendicular joins
  \item
    Fixed direction of controlled american current source
  \item
    Fixed incorrect scaling of magnetron
  \item
    Fixed: Number of american inductor coils not adjustable
  \item
    Fixed Battery Symbols and added new battery2 symbol
  \item
    Added non-inverting Schmitttrigger
  \end{itemize}
\item
  Version 0.7 (2016-09-08)

  \begin{itemize}
  \tightlist
  \item
    Added second annotation label, showing, e.g., the value of an
    component
  \item
    Added new symbol: magnetron
  \item
    Fixed name conflict of diamond shape with tikz.shapes package
  \item
    Fixed varcap symbol at small scalings
  \item
    New packet-option ``straightvoltages, to draw straight(no curved)
    voltage arrows
  \item
    New option ``invert'' to revert the node direction at paths
  \item
    Fixed american voltage label at special sources and battery
  \item
    Fixed/rotated battery symbol(longer lines by default positive
    voltage)
  \item
    New symbol Schmitttrigger
  \end{itemize}
\item
  Version 0.6 (2016-06-06)

  \begin{itemize}
  \tightlist
  \item
    Added Mechanical Symbols (damper,mass,spring)
  \item
    Added new connection style diamond, use (d-d)
  \item
    Added new sources voosource and ioosource (double zero-style)
  \item
    All diode can now drawn in a stroked way, just use globel option
    ``strokediode'' or stroke instead of full/empty, or D-. Use this
    option for compliance with DIN standard EN-60617
  \item
    Improved Shape of Diodes:tunnel diode, Zener diode, schottky diode
    (bit longer lines at cathode)
  \item
    Reworked igbt: New anchors G,gate and new L-shaped form Lnigbt,
    Lpigbt
  \item
    Improved shape of all fet-transistors and mirrored p-chan fets as
    default, as pnp, pmos, pfet are already. This means a
    backward-incompatibility, but smaller code, because p-channels
    mosfet are by default in the correct direction(source at top). Just
    remove the `yscale=-1' from your p-chan fets at old pictures.
  \end{itemize}
\item
  Version 0.5 (2016-04-24)

  \begin{itemize}
  \tightlist
  \item
    new option boxed and dashed for hf-symbols
  \item
    new option solderdot to enable/disable solderdot at source port of
    some fets
  \item
    new parts: photovoltaic source, piezo crystal, electrolytic
    capacitor, electromechanical device(motor, generator)
  \item
    corrected voltage and current direction(option to use old behaviour)
  \item
    option to show body diode at fet transistors
  \end{itemize}
\item
  Version 0.4

  \begin{itemize}
  \tightlist
  \item
    minor improvements to documentation
  \item
    comply with TDS
  \item
    merge high frequency symbols by Stefan Erhardt
  \item
    added switch (not opening nor closing)
  \item
    added solder dot in some transistors
  \item
    improved ConTeXt compatibility
  \end{itemize}
\item
  Version 0.3.1

  \begin{itemize}
  \tightlist
  \item
    different management of color\ldots{}
  \item
    fixed typo in documentation
  \item
    fixed an error in the angle computation in voltage and current
    routines
  \item
    fixed problem with label size when scaling a tikz picture
  \item
    added gas filled surge arrester
  \item
    added compatibility option to work with Tikz's own circuit library
  \item
    fixed infinite in arctan computation
  \end{itemize}
\item
  Version 0.3.0

  \begin{itemize}
  \tightlist
  \item
    fixed gate node for a few transistors
  \item
    added mixer
  \item
    added fully differential op amp (by Kristofer M. Monisit)
  \item
    now general settings for the drawing of voltage can be overridden
    for specific components
  \item
    made arrows more homogeneous (either the current one, or latex' bt
    pgf)
  \item
    added the single battery cell
  \item
    added fuse and asymmetric fuse
  \item
    added toggle switch
  \item
    added varistor, photoresistor, thermocouple, push button
  \item
    added thermistor, thermistor ptc, thermistor ptc
  \item
    fixed misalignment of voltage label in vertical bipoles with names
  \item
    added isfet
  \item
    added noiseless, protective, chassis, signal and reference grounds
    (Luigi «Liverpool»)
  \end{itemize}
\item
  Version 0.2.4

  \begin{itemize}
  \tightlist
  \item
    added square voltage source (contributed by Alistair Kwan)
  \item
    added buffer and plain amplifier (contributed by Danilo Piazzalunga)
  \item
    added squid and barrier (contributed by Cor Molenaar)
  \item
    added antenna and transmission line symbols contributed by Leonardo
    Azzinnari
  \item
    added the changeover switch spdt (suggestion of Fabio Maria
    Antoniali)
  \item
    rename of context.tex and context.pdf (thanks to Karl Berry)
  \item
    updated the email address
  \item
    in documentation, fixed wrong (non-standard) labelling of the axis
    in an example (thanks to prof. Claudio Beccaria)
  \item
    fixed scaling inconsistencies in quadrupoles
  \item
    fixed division by zero error on certain vertical paths
  \item
    introduced options straighlabels, rotatelabels, smartlabels
  \end{itemize}
\item
  Version 0.2.3

  \begin{itemize}
  \tightlist
  \item
    fixed compatibility problem with label option from tikz
  \item
    Fixed resizing problem for shape ground
  \item
    Variable capacitor
  \item
    polarized capacitor
  \item
    ConTeXt support (read the manual!)
  \item
    nfet, nigfete, nigfetd, pfet, pigfete, pigfetd (contribution of
    Clemens Helfmeier and Theodor Borsche)
  \item
    njfet, pjfet (contribution of Danilo Piazzalunga)
  \item
    pigbt, nigbt
  \item
    \emph{backward incompatibility} potentiometer is now the standard
    resistor-with-arrow-in-the-middle; the old potentiometer is now
    known as variable resistor (or vR), similarly to variable inductor
    and variable capacitor
  \item
    triac, thyristor, memristor
  \item
    new property ``name'' for bipoles
  \item
    fixed voltage problem for batteries in american voltage mode
  \item
    european logic gates
  \item
    \emph{backward incompatibility} new american standard inductor. Old
    american inductor now called ``cute inductor''
  \item
    \emph{backward incompatibility} transformer now linked with the
    chosen type of inductor, and version with core, too. Similarly for
    variable inductor
  \item
    \emph{backward incompatibility} styles for selecting shape variants
    now end are in the plural to avoid conflict with paths
  \item
    new placing option for some tripoles (mostly transistors)
  \item
    mirror path style
  \end{itemize}
\item
  Version 0.2.2 - 20090520

  \begin{itemize}
  \tightlist
  \item
    Added the shape for lamps.
  \item
    Added options \texttt{europeanresistor}, \texttt{europeaninductor},
    \texttt{americanresistor} and \texttt{americaninductor}, with
    corresponding styles.
  \item
    FIXED: error in transistor arrow positioning and direction under
    negative \texttt{xscale} and \texttt{yscale}.
  \end{itemize}
\item
  Version 0.2.1 - 20090503

  \begin{itemize}
  \tightlist
  \item
    Op-amps added
  \item
    added options arrowmos and noarrowmos, to add arrows to pmos and
    nmos
  \end{itemize}
\item
  Version 0.2 - 20090417 First public release on CTAN

  \begin{itemize}
  \tightlist
  \item
    \emph{Backward incompatibility}: labels ending with
    \texttt{:}\textit{angle} are not parsed for positioning anymore.
  \item
    Full use of \TikZ~keyval features.
  \item
    White background is not filled anymore: now the network can be drawn
    on a background picture as well.
  \item
    Several new components added (logical ports, transistors, double
    bipoles, \ldots).
  \item
    Color support.
  \item
    Integration with \{\ttfamily siunitx\}.
  \item
    Voltage, american style.
  \item
    Better code, perhaps. General cleanup at the very least.
  \end{itemize}
\item
  Version 0.1 - 2007-10-29 First public release
\end{itemize}

\section{ToDo}
    Die folgende Liste soll die nächsten geplanten Funktionen bzw. Entwicklungsschritte angeben.

    \subsection{Must-have}
        \begin{itemize}
            \item \ldots
        \end{itemize}
    \subsection{Nice-to-have}
        \begin{itemize}
            \item Weitere für die Schule nützliche Dokumenttypen integrieren, z.\,B. Lerntagebücher.
            \item Praxisbeispiel Klausur: Objekt- und Klassendiagramm mit TikZ setzen.
            \item TikZ-Stile nicht als Paketoptionen formatieren.
            \item Dokumentieren des Generierens von Klausuren aus Aufgabendatenbanken wie es mit \pkg{exsheets} möglich war.
            \item Bessere Trennung von Inhalt und Formatierung (Modul Format sollte alle notwendigen Makros enthalten, Modul Aufgaben sollte sich nicht mehr um die Formatierung kümmern)
        \end{itemize}

%
% bgteubner class bundle
%
% anhang.tex
% Copyright 2003--2012 Harald Harders
%
% This program may be distributed and/or modified under the
% conditions of the LaTeX Project Public License, either version 1.3
% of this license or (at your opinion) any later version.
% The latest version of this license is in
%    http://www.latex-project.org/lppl.txt
% and version 1.3 or later is part of all distributions of LaTeX
% version 1999/12/01 or later.
%
% This program consists of all files listed in manifest.txt.
% ===================================================================

% ===================================================================


\end{document}
