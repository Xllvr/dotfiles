\documentclass[a5paper, 12pt,headheight=50pt]{scrartcl}
\usepackage[
    typ=lzk,
    fach = DAZ,
    lerngruppe=5g,
    loesungen=seite,
    module={Symbole},
    sprache={english,french},
]{schule}
\usepackage[sfdefault]{comicneue}

% Dieses Dokument gehört zu den Beispiel des LaTeX Paketes Schule und ist von den Autoren
% des Pakets erstellt worden.
%
% Das Dokument steht unter der Lizenz: Creative Commons by-nc-sa Version 4.0
% http://creativecommons.org/licenses/by-nc-sa/4.0/deed.de
%
% Nach dieser Lizenz darf das Dokument beliebig kopiert und bearbeitet werden,
% sofern das Folgeprodukt wiederum unter gleichen Lizenzbedingungen vertrieben
% und auf die ursprünglichen Urheber verwiesen wird.
% Eine kommerzielle Nutzung ist ausdrücklich ausgeschlossen.

\author{Test Person}
\date{\today}
\title{1. Vokabeltest}


\newcommand{\eintrag}[3]{\\\hline \usymH{#1}{1cm}& \textluecke[nichts]{#2}& \textluecke[nichts]{#3}}
\renewcommand{\arraystretch}{3}
\begin{document}

\xsimsetup{
	aufgabe/template=schule-keintitel,
}

\section*{\Titel am \today}
\sffamily
\begin{aufgabe}[points=12]
    \begin{tabularx}{\linewidth}{c|X|X}
            & \textbf{deutsch}   & \textbf{english}
        \eintrag{1F34F}    {der Apfel}          {the apple}
        \eintrag{1F3CA}    {schwimmen}          {to swim}
        \eintrag{1F408}    {die Katze}          {the cat}
        \eintrag{1F56E}    {das Buch}           {the book}
        \eintrag{1F68C}    {der Bus}            {the bus}
        \eintrag{1F697}    {das Auto}           {the car}
    \end{tabularx}
\end{aufgabe}
\begin{loesung}
    \begin{tabularx}{\linewidth}{c|X|X}
            & \textbf{deutsch}   & \textbf{english}
        \eintrag{1F34F}    {der Apfel}          {the apple}
        \eintrag{1F3CA}    {schwimmen}          {to swim}
        \eintrag{1F408}    {die Katze}          {the cat}
        \eintrag{1F56E}    {das Buch}           {the book}
        \eintrag{1F68C}    {der Bus}            {the bus}
        \eintrag{1F697}    {das Auto}           {the car}
    \end{tabularx}
\end{loesung}

\vfill
\hinweis{Denke an der/die/das!}
\vfill

\begin{center}
    \large Viel Erfolg! \symKlee{} %wünscht \Autor!
\end{center}

%Du hast \luecke{1cm} von \punkteAufgabe{n} (\punkteTotal{n}) erreicht!\\[0.5cm]
Note:

\end{document}
