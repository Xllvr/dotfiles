\documentclass[a4paper]{scrartcl}
\usepackage[left=3cm,right=2cm]{geometry}
\usepackage[
    fach=Informatik,
    lerngruppe={GK EF},
    typ=ab,
    farbig,
    loesungen=seite,
    module={
        Kuerzel,
        Papiertypen,
        Symbole,
        Texte,
        Bewertung
    },
]{schule}

% Dieses Dokument gehört zu den Beispiel des LaTeX Paketes Schule und ist von den Autoren
% des Pakets erstellt worden.
%
% Das Dokument steht unter der Lizenz: Creative Commons by-nc-sa Version 4.0
% http://creativecommons.org/licenses/by-nc-sa/4.0/deed.de
%
% Nach dieser Lizenz darf das Dokument beliebig kopiert und bearbeitet werden,
% sofern das Folgeprodukt wiederum unter gleichen Lizenzbedingungen vertrieben
% und auf die ursprünglichen Urheber verwiesen wird.
% Eine kommerzielle Nutzung ist ausdrücklich ausgeschlossen.

\usepackage{weva}

\title{Ein Arbeitsblatt}
\author{Daniel Spittank}
\date{12.08.2016}

\begin{document}
\section*{Text}
\begin{zeilenNrMehrspaltig}{4}
    Dies hier ist ein Blindtext \so{zum Testen von Textausgaben}. Wer diesen Text liest, ist natürlich
    selbst schuld. Der Text gibt lediglich den Grauwert der Schrift an. Ist das wirklich so?
    Ist es gleichgültig, ob ich schreibe: \enquote{Dies ist ein Blindtext} oder \enquote{Huardest gefburn}?
    Kjift – mitnichten! Ein Blindtext liefert mir wichtige Hinweise. An ihm messe ich
    die Lesbarkeit einer Schrift, ihre Anmutung, wie harmonisch die Figuren zueinander
    stehen und prüfe, wie breit oder schmal sie läuft. Ein Blindtext sollte möglichst viele
    verschiedene Buchstaben enthalten und in der Originalsprache gesetzt sein. Er muss
    keinen Sinn ergeben, sollte aber lesbar sein. Fremdsprachige Texte wie \enquote{Lorem ipsum}
    dienen nicht dem eigentlichen Zweck, da sie eine falsche Anmutung vermitteln. € \enquote{Ein Zitat}.  \SuS oder \KuK 😀 \symUhr
\end{zeilenNrMehrspaltig}

\section*{Aufgaben}Für alle Aufgaben zusammen gibt es \punkteTotal
    \begin{aufgabe}[points=2,bonus-points=1]
        Eine einfache Aufgabe zum Einstieg.
    \end{aufgabe}
    \begin{aufgabe}
        Hier steht eine Aufgabe. Die ist wirklich schwierig und hat auch noch Teilaufgaben.
        \begin{teilaufgaben}
            \teilaufgabe Checkboxen: \chb\chb\chb*\chb\chb*
            \teilaufgabe Das ist eine feste Lücke: \luecke[style=\uwave{#1}]{2cm}
            \teilaufgabe Das ist eine \textluecke[style=\dashuline{#1}]{Lücke für einen Text}.
            \teilaufgabe Keine Aufgabe
        \end{teilaufgaben}
        \begin{loesung}
            \begin{teilaufgaben}
                \teilaufgabe Checkboxen: \chb\chb\chb*\chb\chb*
                \teilaufgabe Das ist eine feste Lücke: \luecke[style=\uwave{#1}]{2cm}
                \teilaufgabe Das ist eine \textluecke[style=\dashuline{#1}]{Lücke für einen Text}.
                \teilaufgabe Keine Aufgabe
            \end{teilaufgaben}
        \end{loesung}
        \begin{erwartungen}
            \erwartung{\weva gibt eine Lösung an.}{1}
            \erwartung{erfüllt eine \so{ehrlich völlig überzogene} Erwartung.}{10}
            \erwartung{gibt nicht auf.}{}[10]
        \end{erwartungen}
        \begin{bearbeitungshinweis}
            \begin{teilaufgaben}
                \teilaufgabe Man kann auch Hinweise geben.
                \teilaufgabe Sogar mehrere.
                \teilaufgabeOhneLoesung
                \teilaufgabe Wirklich viele.
            \end{teilaufgaben}
            Oder auch allgemeine Hinweise.
        \end{bearbeitungshinweis}
    \end{aufgabe}
    \begin{aufgabe}
        \setzeSymbol{\symZweiSprechblasen}
        Hier steht eine Aufgabe. Die ist wirklich schwierig und hat auch noch Teilaufgaben.
        \begin{teilaufgaben}
            \teilaufgabe Erstens
            \teilaufgabe Zweitens
        \end{teilaufgaben}
        \begin{loesung}
            Eine Lösung.
        \end{loesung}
        \begin{erwartungen}
            \erwartung{hat den {\LARGE Text} \textcolor{red}{gelesen}.}{1}
            \erwartung{gibt eine Lösung an.}{4}
            \erwartung{erfüllt eine ehrlich völlig überzogene Erwartung.
                \begin{tabular}{|c|c|}
                    \hline
                    a & b  \\
                    \hline
                    c & d \\
                    \hline
                \end{tabular}
            }{10}
        \end{erwartungen}
    \end{aufgabe}

    \begin{aufgabe}
        Aufgaben können auch Punkte erhalten.
        \begin{teilaufgaben}
            \teilaufgabe[5] erstens
            \teilaufgabe[10] zweitens
            \teilaufgabe[15] drittens
        \end{teilaufgaben}
        \begin{loesung}
            Eine weitere Lösung.
            \begin{teilaufgaben}
                \teilaufgabe erstens
                \teilaufgabe zweitens
                \teilaufgabe drittens
            \end{teilaufgaben}
        \end{loesung}
        \begin{erwartungen}
            \erwartung{sollte eine weitere \symAuge Lösung angeben.}{30}
        \end{erwartungen}
        \begin{bearbeitungshinweis}
            Man kann auch Hinweise geben.
            \begin{itemize}
                \item Sogar mehrere.
                \item Wirklich viele.
            \end{itemize}
        \end{bearbeitungshinweis}
    \end{aufgabe}

    \begin{aufgabe}[subtitle=Eine Kreuzchenaufgabe,symbol=\symHaken]
        \begin{mcumgebung} %Lösung wird automatisch ergänzt
            \choice Erstens
            \choice Zweitens
            \choice Drittens
            \choice[\mcrichtig] Viertens
            \choice! Fünftens
            \choice[\mcrichtig] Sechstens
            \choice Siebtens
            \choice[\mcrichtig] Achtens
        \end{mcumgebung}
        \begin{loesung}
            \begin{mcumgebung} %Lösung wird automatisch ergänzt
                \choice Erstens
                \choice Zweitens
                \choice Drittens
                \choice[\mcrichtig] Viertens
                \choice! Fünftens
                \choice[\mcrichtig] Sechstens
                \choice Siebtens
                \choice[\mcrichtig] Achtens
            \end{mcumgebung}
        \end{loesung}
        \begin{erwartungen}
            \erwartung{kreuzt \textbf{alles} richtig an.}{10}
        \end{erwartungen}
    \end{aufgabe}

    \begin{aufgabe}[symbol=\symBleistift,subtitle={Lückentext}]
        Das ist ein total verrückter \textluecke{Lückentext}. Für alle \textluecke{Menschen}, die \textluecke{Lücken} lieben.
        \begin{loesung}
            Das ist ein total verrückter \textluecke{Lückentext}. Für alle \textluecke{Menschen}, die \textluecke{Lücken} lieben.
        \end{loesung}
        \begin{erwartungen}
            \erwartung{füllt die Lücken richtig aus.}{10}
        \end{erwartungen}
    \end{aufgabe}

    \begin{aufgabe*}
        Eine Zusatzaufgabe.
        \begin{loesung*}
            Eine weitere Lösung.
        \end{loesung*}
        \begin{erwartungen}
            \erwartung{sollte eine $7+4=11$ weitere Lösung angeben.}{}[2]
        \end{erwartungen}
        \begin{bearbeitungshinweis}
            Noch ein Hinweis. Und noch einer.
        \end{bearbeitungshinweis}
    \end{aufgabe*}

    \begin{aufgabe}
        Ergänzen Sie den Quelltext unten so um Funktionsaufrufe, dass die Ausgabe des Programms danach lautet:\\
        Goodbye Erde!\\
        Hallo Mars!
        \begin{lstlisting}[language=python,gobble=10]
            # Ein Beispiel
            def halloWelt (nutzeGoodbye, weltname):
                if (nutzeGoodbye):
                    tmp = "Goodbye "
                else:
                    tmp = "Hallo "
                print(tmp + weltname + "!")
        \end{lstlisting}
        \begin{erwartungen}
            \erwartung{Ergänzt die Funktionsaufrufe korrekt.}{10}
        \end{erwartungen}
        \begin{bearbeitungshinweis}
            Denke gut darüber nach wie man vorgeht.
        \end{bearbeitungshinweis}
        \begin{loesung}
            \begin{lstlisting}[language=python,gobble=14]
                # Ein Beispiel
                def halloWelt (nutzeGoodbye, weltname):
                    if (nutzeGoodbye):
                        tmp = "Goodbye "
                    else:
                        tmp = "Hallo "
                    print(tmp + weltname + "!")

                halloWelt(True, "Erde")
                halloWelt(False, "Mars")
            \end{lstlisting}
        \end{loesung}
    \end{aufgabe}

\section*{Bearbeitungshinweise}
    \bearbeitungshinweisZuAufgabe{4}
    \bearbeitungshinweisliste

\section*{Punktübersicht}
    \punktuebersicht

\section*{Notenverteilung}
    \notenverteilung

\section*{Erwartungen}
    \erwartungshorizont

\section*{Papiertypen}
    \subsection*{liniert}
        \feldLin[1cm]{5}
    \subsection*{kariert}
        \feldKar[0.5cm]{5}
    \subsection*{Millimeterpapier}
        \feldMil{4}
    \subsection*{Beispiel in Tabelle}
        \begin{tabular}{|p{0.48\linewidth}|p{0.45\linewidth}|}
            \hline Millimeterpapier & kariert \\
            \hline \feldMil{2} & \feldKar[0.5cm]{4} \\
            \hline
        \end{tabular}

\end{document}
