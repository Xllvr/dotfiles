\documentclass[a4paper]{scrartcl}
\usepackage[
    typ=kl,
    klausurtyp=klausur,
    fach=Informatik,
    lerngruppe=9g,
    loesungen=seite,
    erwartungshorizontAnzeigen,
    erwartungshorizontStil=einzeltabellen,
]{schule}

% Dieses Dokument gehört zu den Beispiel des LaTeX Paketes Schule und ist von den Autoren
% des Pakets erstellt worden.
%
% Das Dokument steht unter der Lizenz: Creative Commons by-nc-sa Version 4.0
% http://creativecommons.org/licenses/by-nc-sa/4.0/deed.de
%
% Nach dieser Lizenz darf das Dokument beliebig kopiert und bearbeitet werden,
% sofern das Folgeprodukt wiederum unter gleichen Lizenzbedingungen vertrieben
% und auf die ursprünglichen Urheber verwiesen wird.
% Eine kommerzielle Nutzung ist ausdrücklich ausgeschlossen.

\author{Test Person}
\date{\today}
\title{Eine Klausur}

\usepackage{blindtext}

\begin{document}

\begin{aufgabe}
    \blindtext
    \begin{teilaufgaben}
        \teilaufgabe[5] Lies den Text!
        \teilaufgabe[10] Nimm dazu begründet Stellung.
    \end{teilaufgaben}
    \begin{loesung}
        Es handelt sich eindeutig um einen Blindtext.
    \end{loesung}
    \begin{erwartungen}
        \erwartung{hat den Text gelesen.}{5}
        \erwartung{nimmt begründet Stellung zum Text und berücksichtigt, dass es sich um einen Blindtext handelt.}{10}
    \end{erwartungen}
\end{aufgabe}

\begin{aufgabe*}
    Eine Zusatzaufgabe mit zwei Teilaufgaben
    \begin{teilaufgaben}
        \teilaufgabe Erste Teilaufgabe.
        \teilaufgabe Zweite Teilaufgabe.
    \end{teilaufgaben}
    \begin{loesung*}
        \begin{teilaufgaben}
            \teilaufgabe Die Lösung lautet 1.
            \teilaufgabe Die Lösung lautet 2.
        \end{teilaufgaben}
    \end{loesung*}
    \begin{erwartungen}
        \erwartung{gibt eine richtige Lösung zu Teilaufgabe a) an.}{}[5]
        \erwartung{gibt eine richtige Lösung zu Teilaufgabe b) an.}{}[5]
    \end{erwartungen}
\end{aufgabe*}

\vspace{1cm}
{\Huge Viel Erfolg!}

\end{document}