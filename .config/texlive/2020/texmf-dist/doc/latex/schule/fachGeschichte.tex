\section{Geschichte}
\label{fach:geschichte}
Das Fachmodul \module{Geschichte} bindet das Paket \pkg{biblatex} mit den Einstellungen für die humanwissenschaftliche Zitierweise ein. Des weiteren werden Befehle für durchnummerierte Quellen, Materialien und Verfassertexte zur Verfügung gestellt.

\subsection{Befehle}
    \begin{commands}
        \command{material}[\oarg{Ebene}\marg{Titel}]\Default{\textbackslash subsection}
            Erzeugt eine Überschrift, mit der Material markiert werden kann. Dazu wird am rechten Rand ein M mit einer fortlaufenden Nummer gesetzt. Standardmäßig ist als Überschriftsebene \verbcode|\subsection| gesetzt, dass über den optionalen Parameter geändert werden kann.

        \command{quelle}[\oarg{Ebene}\marg{Titel}]\Default{\textbackslash subsection}
            Erzeugt eine Überschrift, mit der eine Quelle markiert werden kann. Dazu wird am rechten Rand ein Q mit einer fortlaufenden Nummer gesetzt. Standardmäßig ist als Überschriftsebene \verbcode|\subsection| gesetzt, dass über den optionalen Parameter geändert werden kann.

        \command{vt}[\oarg{Ebene}\marg{Titel}]\Default{\textbackslash subsection}
            Erzeugt eine Überschrift, mit der ein Verfassertext markiert werden kann. Dazu wird am rechten Rand ein VT mit einer fortlaufenden Nummer gesetzt. Standardmäßig ist als Überschriftsebene \verbcode|\subsection| gesetzt, dass über den optionalen Parameter geändert werden kann.
    \end{commands}
    
    Die so erstellten Textabschnitte können mit \verb|\nameref{sec:$REFERENZ_ART$NUMMER}| referenziert werden, z.\,B. per \verb|\nameref{sec:vt1}|. Für weitere Hinweise siehe \prettyref{example:bsp-geschichte}.
    
    
% \begin{example}[gobble=0]
% %
% \end{example}