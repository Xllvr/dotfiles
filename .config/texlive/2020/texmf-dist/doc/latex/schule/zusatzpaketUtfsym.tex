\section{Unicode-Symbole}
\label{paket:utfsym}

Im Schulkontext benötigt man häufig verschiedene Symbole, sei es
für die Kennzeichnung von Partner- und Gruppenarbeit, für die
Auflistung von Materialien oder einfach für die Verschönerung
von Arbeitsblättern.

Im Unicode sind einige tausend Symbole bereits definiert, von
denen sich viele für den Einsatz in der Schule aufdrängen. 
Leider ist der Einsatz in Latex nicht ohne weiteres möglich.

Deswegen wurde für das Schule-Paket das Zusatzpaket \pkg{utfsym}
entwickelt. Dieses ermöglicht die Verwendung von Zeichen aus
den folgenden Unicode-Blöcken:

\begin{smallitemize}
	\item Miscellaneous Symbols and Arrows (2600-26FF),
		siehe S.~\pageref{utab:02600-026FF}
	\item Dingbats (2700-27BF),
		siehe S.~\pageref{utab:02700-027BF}
	\item Mahjong Tiles (1F000-1F02F),
		siehe S.~\pageref{utab:1F000-1F02F}
	\item Domino Tiles (1F030-1F09F),
		siehe S.~\pageref{utab:1F030-1F09F}
	\item Playing Cards (1F0A0-1F0FF),
		siehe S.~\pageref{utab:1F0A0-1F0FF}
	\item Miscellaneous Symbols and Pictographs (1F300-1F5FF),
		siehe S.~\pageref{utab:1F300-1F5FF}
	\item Emoticons / Emoji (1F600-1F64F),
		siehe S.~\pageref{utab:1F600-1F64F}
	\item Transport and Map Symbols (1F680-1F6FF),
		siehe S.~\pageref{utab:1F680-1F6FF}
\end{smallitemize}

Die vollständigen Symboltabellen, mit
allen über 1600 unterstützten Symbolen, finden sich im Anhang, siehe
\prettyref{sec:unicodesymbole}.

Die entsprechenden Symbole können direkt in den LaTeX-Quelltext
eingefügt oder per Befehl eingebunden werden.

Die Symbole stammen dabei aus der Public-Domain-Schrift
\textbf{Symbola.ttf} von \person{George Douros}. Sie werden mit
TikZ gesetzt und passen sich jeweils an die Schriftgröße und
Farbe der Umgebung an, können aber auch als Bild mit
benutzerdefinierter Skalierung eingebunden werden.

\subsection{Befehle}
\begin{commands}
	\command{usym}[\marg{Code}]
	Bindet das Symbol mit dem gegebenen Code in das Dokument ein.
	Die Darstellung erfolgt im Fließtext. Größe und Farbe werden an
	den umgebenden Text angepasst.
\begin{sidebyside}
  \usym{1F642}
  \usym{1F609}
  \Huge\color{yellow}\usym{1F60A}
\end{sidebyside}
	\command{usymH}[\marg{Code}\marg{Höhe}]
	Bindet das Symbol mit dem gegebenen Code in das Dokument ein.
	Die Größe wird an die angegebene Höhe angepasst.
\begin{sidebyside}
  \usymH{1F0A1}{2cm}
\end{sidebyside}
	\command{usymW}[\marg{Code}\marg{Breite}]
	Bindet das Symbol mit dem gegebenen Code in das Dokument ein.
	Die Größe wird an die angegebene Breite angepasst.
\begin{sidebyside}
  \usymW{1F68C}{3cm}
\end{sidebyside}
\end{commands}