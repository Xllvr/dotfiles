\section{Übungsblatt}
\label{typ:ueb}
Dieser Dokumenttyp wird für Übungsblätter verwendet. Der Hauptunterschied zum Typ \enquote{Arbeitsblatt} liegt darin, dass Aufgaben als \enquote{Übungen} bezeichnet werden. Für die Verwendung des Übungsblatts ist als Typ \verbcode|ueb| anzugeben.

Die empfohlene Dokumentklasse ist \cls{scrartcl}.

Die Vorgaben des Dokumenttyps definieren Kopf- und Fußzeilen in der üblichen Darstellung des Schule-Pakets. Darüber hinaus werden keine weiteren Vorgaben gemacht.

%\subsection{Paketoptionen}
%\begin{options}
%	\opt{X}
%		Y
%\end{options}

%\subsection{Befehle}
%\begin{commands}
%
%\command{X}
%	Y
%
%\end{commands}