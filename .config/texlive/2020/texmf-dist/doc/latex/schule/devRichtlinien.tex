\section{Richtlinien}
    Um die Quelltexte des Pakets über einen längeren Zeitraum halbwegs konsistent und dokumentiert zu halten und somit die leichtere  Einarbeitung neuer Betreuer zu ermöglichen, wurden folgende Richtlinien vereinbart:
    \begin{itemize}
        \item Bei sämtliche Klassen- und Paketquellen sollten lange Zeilen vermieden werden.
        \item Bei der Dokumentation (dokumentation.tex) soll kein automatischer Umbruch im Dokument stattfinden, um Änderungen im Text einfacher nachhalten zu können.
        \item Auf Einrückungen sollte geachtet werden. Zeilenumbrüche  sind ggf. entsprechend auszukommentieren.
        \item Einrückungen erfolgen mit Leerzeichen in einer Weite von 4.
        \item Alle internen Makros und Variablen sollen im Namensraum  \code{schule\cnltxat} stehen, \zB\space\code{schule\cnltxat  ergebnishorizontAnzeigen}.
        \item Sprechende Bezeichner sollten verwendet werden.
        \item Für neue Funktionen sollte ein neues Modul oder ein neuer Dokumenttyp angelegt werden, sofern es sich um keine klare Ergänzung handelt.
        \item Neue Funktionen und Änderungen sind in der Dokumentation mit den Mitteln von \pkg{cnltx-doc} zu kennzeichnen, \zB\space mit \cs{changedversion}\marg{version} für Änderungen oder \cs{sinceversion}\marg{version} für neue Funktionen. Außerdem sind sie im Changelog zu vermerken.
    \end{itemize}