\section{Funktionen für Entwickler}
	\subsection{Fehlerbehandlung und Debugging}
		\subsubsection{Paketoptionen}
			\begin{options}
				\opt{debug}
				schaltet das Paket in den Debugmodus. Verhindert zudem
				die Unterdrückung ungefährlicher Warnungen, die
				standardmäßig aktiv ist.
				
				Diese Paketoption setzt zudem die boolesche Variable 
				\code{schule\cnltxat debug} auf \code{true}, sodass
				jederzeit geprüft werden kann, ob der Debugmodus aktiv
				ist oder nicht.
			\end{options}
		\subsubsection{Befehle}
			\begin{commands}
				\command{sinfo}[\marg{Text}]
					schreibt den angegebenen Text in die Logdatei.
				\command{swarnung}[\marg{Text}]
					schreibt den angegebenen Text als Warnung in die
					Logdatei.
				\command{sfehler}[\marg{Text}]
					schreibt den angegebenen Text als Fehler in die
					Logdatei und beendet die Kompilierung.
				\command{sdinfo}[\marg{Text}]
					schreibt den angegebenen Text in die Logdatei,
					falls der Debugmodus aktiv ist.
				\command{sdwarnung}[\marg{Text}]
					schreibt den angegebenen Text als Warnung in die
					Logdatei, falls der Debugmodus aktiv ist.
			\end{commands}
	
	\subsection{Interne Makros}
		\subsubsection{Befehle}
			\begin{commands}
				\command{schule\cnltxat kopfUmbruch}
					liefert einen Zeilenumbruch, wenn die Kopfzeile an
					einer Stelle zweizeilig ist, \zB\space durch die
					Darstellung eines Namensfelds oder die Anzeige
					des Datums.
				\command{schule\cnltxat modulDateiLaden}[
				\marg{Kategorie}\marg{Modulname}\marg{Abschnitt}]
				Lädt einen Abschnitt eines Moduls. Interne
				Hilfsfunktion für das Laden von Modulen. 
				
				\achtung{Sollte nicht manuell verwendet werden,
				sondern nur an den entsprechenden Stellen der
				\code{schule.sty}.}
				
				\achtung{Kann mit viel Sorgfalt zur Erfüllung von 
				Abhängigkeiten zwischen Modulen genutzt werden.
				Es ist darauf zu achten, dass dann zu Beginn der
				Abschnitte des ladenden Moduls die entsprechenden
				Abschnitte des zu ladenden Moduls eingebunden 
				werden.
				
				Dies führt dazu, dass die Ladereihenfolge der Module
				nicht mehr sichergestellt werden kann.}
				
				\command{schule@modulNachladen}[\marg{Modulname}]
				Lädt ein Modul mit allen Abschnitten. Interne
				Hilfsfunktion für die Erfüllung von Abhängigkeiten in
				Modulen. 
				
				\achtung{Die Verwendung zu anderen Zwecken wird nicht
				empfohlen, da die Reihenfolge der zu ladenden Module
				hier nicht beachtet werden kann. Ebenfalls kann so
				nicht sichergestellt werden, dass der geladene Code
				an der richtigen Stelle im Quelltext landet, vielmehr
				wird er genau an der Stelle des Befehls geladen.}
			\end{commands}