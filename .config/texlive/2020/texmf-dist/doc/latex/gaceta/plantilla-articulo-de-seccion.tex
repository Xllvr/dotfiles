\documentclass[autocontact]{gaceta}
% (* o \documentclass{gaceta} si el art\'{\i}culo no tiene firma *)

%---------------------
%%%%%% NOTA T\'ECNICA (ign\'orala si no la entiendes) %%%%%%
% Si, en lugar de pdflatex (opci\'on por defecto), vas a usar 
% latex, puedes reproducir el tama\~no final del papel con el 
% que se imprimir\'a la Gaceta a\~nadiendo la opci\'on dvips:
% \documentclass[dvips]{gaceta}
%---------------------

%---------------------
% No olvides descargar el archivo gaceta.cls y ponerlo 
% en la misma carpeta que este documento, o donde LaTeX 
% lo sepa encontrar.
% Recuerda que La Gaceta prefiere art\'{\i}culos que 
% no superen las 20 p\'aginas.
%---------------------

%---------------------
% La Gaceta se har\'a con esto activado, 
% pero no hace falta ahora si no quieres:
%\usepackage[T1]{fontenc}
%\usepackage{lmodern}

% Puedes usar esto (u otra codificaci\'on) si quieres:
%\usepackage[latin1]{inputenc}

\usepackage[spanish]{babel}
%---------------------

%---------------------
% Esto ya lo carga el estilo de La Gaceta:
%\usepackage{amsmath, amsthm, amssymb}
%\usepackage{url} % <-- Para p\'aginas web o similar: \url{...}
%\usepackage{graphicx}
%---------------------

%---------------------
% Carga esto para gr\'aficos si lo necesitas:
% \usepackage{wrapfig}
%---------------------

%---------------------
% <<< ESTO SE AJUSTAR\'A AL EDITAR CADA N\'UMERO DE LA GACETA >>>
\setcounter{page}{1} 
\journame{La Gaceta de la RSME}
\yearofpublication{0000}
\volume{00}
\issuenumber{0}
%---------------------

%---------------------
% Deja esto as\'{\i}:
%\belongstopart{Art\'{\i}culos} 
\belongstopart{Secciones} 
%\belongstopart{Noticias de la Sociedad} 
%\belongstopart{Actualidad} 
%\nopart
%---------------------


%---------------------
% En caso de que el art\'{\i}culo sea de una secci\'on fija,
% rell\'enalo con el nombre de la secci\'on y su responsable
% (o no hagas nada si tienes dudas).
\supertitle{Autoexplicaciones de La Gaceta}
\editor{Un Responsable de Secci\'on}
% Opcional:
% \shortsupertitle{Autoexplicaciones de La Gaceta}
%---------------------

%---------------------
\title{Un ejemplo de c\'omo escribir un art\'{\i}culo que 
se publicar\'a en La Gaceta de la RSME}
\author{Un Autor} % o \author{Un Autor y Otro Autor}, etc.
% \shorttitle{....} % <--- No tiene ning\'un efecto si hay \supertitle
%---------------------

%---------------------
\contact{Un autor, Dpto. de Matem\'aticas, Universidad de \dots}
{autor@uni.es}{http://www.uni.es/personales/autor.html}
%% SINTAXIS (\'usense tantos de estos como autores haya):
%\contact{nombre y direcci\'on autor 1}{email autor 1}{p\'agina web autor 1}
%\contact{nombre y direcci\'on autor 2}{email autor 2}{p\'agina web autor 2}
% (IMPORTANTE: se puede dejar vac\'{\i}o lo que se quiera)
%---------------------

%---------------------------
\begin{document}
%---------------------------

%---------------------
% Descomentar si hay \supertitle y \editor
\makesupertitle
%---------------------

%---------------------
%%% PARA QUE LO USE EL RESPONSABE DE SECCI\'ON (si lo hubiera) SI QUIERE 
%%% INCLUIR ALGUNA INTRODUCCI\'ON PEQUE\~NITA:
% \begin{explanation}
% Una explicaci\'on sobre el art\'{\i}culo que va a ir a continuaci\'on,
% o lo que el responsable de secci\'on considere oportuno contar aqu\'{\i}.
% No hay ninguna necesidad de poner esta explicaci\'on; 
% es s\'olo una posibilidad que algunos desean usar.
% \end{explanation}
%---------------------


%---------------------
%%% EL RESPONSABLE DE LA SECCI\'ON CORRESPONDIENTE PUEDE
%%% PONER UAN EXPLICACI\'ON MUCHO M\'AS GRANDE (UN PAR
%%% DE P\'AGINAS) DE ESTA MANERA:
% \section*{En este n\'umero\dots}
% En este n\'umero de \textit{La Gaceta}, presentamos\dots
% \newpage 
%---------------------


%---------------------
\maketitle
%---------------------


%---------------------
%%% Se puede poner si se quiere:
%\section*{Introducci\'on} % (o similar)
%---------------------

\textit{La Gaceta} de la Real Sociedad Matem\'atica Espa\~nola se compone con \LaTeX, mediante la clase \textit{gaceta}. Disponer de una clase propia espec\'{\i}ficamente dise\~nada es muy \'util, pues se consigue uniformizar el aspecto de todos los art\'{\i}culos de \textit{La Gaceta} minimizando el trabajo, tanto los autores, como de los que tienen que maquetar la revista para enviarla a la imprenta. No es el autor el que tiene que dar el aspecto al art\'{\i}culo; es la clase \textit{gaceta} la que se encarga de hacerlo de manera autom\'atica.

La primera pregunta que te puede surgir es <<?`d\'onde pongo el archivo \texttt{gaceta.cls} que tengo que utilizar?>> La instalaci\'on de \TeX\ de tu ordenador tiene unos lugares relativamente estrictos en los que es necesario situar los diversos archivos que \LaTeX\ va a usar. No se puede explicar esto de manera general, pues depende de la distribuci\'on de \TeX\ de cada uno; sin duda tienes alg\'un documento que te explica en qu\'e directorio de tu disco duro poner ese archivo \texttt{gaceta.cls} para que \LaTeX\ lo encuentre siempre y ya no tengas que preocuparte m\'as de \'el. Pero una cosa es segura: si lo colocas en la misma carpeta que el documento que vas a escribir, todo te funcionar\'a perfectamente.

Esto que tienes delante es una peque\~na muestra de c\'omo escribir un art\'{\i}culo para \textit{La Gaceta}. Por supuesto ~---y lo advertimos por si ahora est\'as leyendo el documento pdf--- tienes que mirar el c\'odigo tex para ver lo que se ha hecho. De esa manera, lo podr\'as usar como plantilla para redactar tu art\'{\i}culo.

Son muy pocas las cosas propias del estilo \textit{gaceta} que debes hacer (y, esencialmente, todas ellas en el pre\'ambulo).
Una vez puestos los t\'{\i}tulos, el art\'{\i}culo se escribe como har\'{\i}as con \textit{article} o con \textit{amsart}. En particular, puedes usar secciones (numeradas o no), subsecciones, listas, entornos para teoremas o similares (que deber\'as definir previamente), etc.

En estas explicaciones, el c\'odigo est\'a escrito con acentos tradicionales de \TeX\ para que nadie tenga problemas. Pero te animamos a que, si quieres, cargues el paquete \textit{inputenc} (con \textit{latin1} o como necesites) y utilices los acentos, e\~nes, admiraciones e interrogaciones normales, los que usar\'{\i}as con cualquier programa de escritura. Te recordamos que la orden para cargar este paquete figura en las primeras del c\'odigo de este documento, pero comentado con un s\'{\i}mbolo de tanto por ciento para que no se active salvo que lo quites.

Si eres un experto en el uso de \LaTeX, te pedimos disculpas por la p\'erdida de tiempo que te puede suponer leer algunas de las cosas que aqu\'{\i} comentamos. Pero, como sabes, hay usuarios de \LaTeX\ de todos los niveles.

Esto \textbf{no} es un manual de \LaTeX. Uno bastante completo en espa\~nol es~\cite{latex-imprenta}. Y, desde luego, el que m\'as informaci\'on proporciona es~\cite{latex-companion}.

\section{Unas pocas peticiones de los directores}

Por favor, no cambies adrede el tama\~no de las p\'aginas para que el art\'{\i}culo ocupe menos. S\'olo es una manera de hacernos perder el tiempo a todos. Recuerda que \textit{La Gaceta} prefiere art\'{\i}culos que no superen las 20 p\'aginas con \textbf{este} formato.

Tampoco te molestes en poner montones de \'ordenes para ampliar espacios entre p\'arrafos o cosas parecidas. \textit{La Gaceta} se publica tal como queda con la clase \textit{gaceta} (que es la que se est\'a usando en esta plantilla). Tendremos que quitar todo lo que pongas para intentar modificar el formato. Y no te empe\~nes en que los gr\'aficos (si los hay) caigan donde t\'u quieras; d\'ejalos que floten y c\'{\i}talos con etiquetas y referencias.

As\'{\i} mismo te pedimos que no cargues una inmensa lista de paquetes o de comandos que no vas a usar, copiados y pegados de otros sitios. Aunque, si no sabes si los necesitas o no, y siempre los usas en tus art\'{\i}culos, d\'ejalos sin preocuparte demasiado.

\textit{La Gaceta} se imprime en tama\~no de letra <<10pt>>, as\'{\i} que las opciones <<11pt>> y <<12pt>> est\'an desactivadas en el estilo. Si, mientras est\'as escribiendo el art\'{\i}culo, prefieres usar <<12pt>>, puedes lograrlo poniendo las opciones <<allow12pt>> y <<12pt>> (y an\'alogamente para <<11pt>>).


\section{Unas pocas instrucciones y consejos}

\subsection{Direcciones de p\'aginas web}

Escribir direcciones de p\'aginas web siempre es complicado es un papel, pues pueden no caber bien en las l\'{\i}neas de texto al paginar. Adem\'as, hay caracteres como el de tanto por ciento o el gui\'on bajo que dan problemas, ya que \LaTeX\ los interpreta como caracteres de control suyos.

Para sortear estas dificultades, existe el comando <<url>>. As\'{\i}, si en tu art\'{\i}culo citas alguna p\'agina web, hazlo como
\url{http://www.sitio.es/facultad/departamento/grupo/~pepe.html}
y todos los problemas se resuelven solos. Haz lo mismo con las direcciones de correo electr\'onico: \url{jose_ramirez@universidad.es} funciona la mar de bien.


\subsection{F\'ormulas de varias l\'{\i}neas}

Para escribir f\'ormulas que ocupan varias l\'{\i}neas, \LaTeX\ tiene el entorno <<eqnarray>>. Pero no es muy vers\'atil, y los resultados que proporciona no son est\'eticamente brillantes. El paquete \textit{amsmath} (que el estilo de \textit{La Gaceta} siempre carga) dispone de muchos otros entornos, como <<align>>, <<multline>>, <<gather>>, <<aligned>>, <<split>>,\dots, algunos de ellos tambi\'en con forma estrellada (como <<align*>>). Las versiones con estrellas no ponen n\'umeros a las ecuaciones.

As\'{\i} mismo, los comandos para escribir matrices que proporciona \textit{amsmath} son mucho m\'as c\'omodos de usar que los que incluye \LaTeX\ a secas.

\subsection{Escribir la bibliograf\'{\i}a}

Habitualmente, para adaptar un art\'{\i}culo al estilo de una revista, lo que m\'as trabajo cuesta es adaptar el formato de la bibliograf\'{\i}a. 
Por favor, intenta seguir el que aqu\'{\i} estamos usando. Tienes un ejemplo de c\'omo citar un art\'{\i}culo de una revista (\cite{Kn}), un libro (\cite{latex-imprenta} o \cite{latex-companion}), y un art\'{\i}culo en las actas de un congreso (\cite{Tao-ICM}). En este \'ultimo caso, o con cap\'{\i}tulos de libros, no pretendemos seguir un estilo estricto, pues la casu\'{\i}stica es demasiado grande. Si lo que citas est\'a en alguna p\'agina web, puedes hacerlo como en \cite{Be-spanish, Be-orto, Do, MeSl} (de paso, observa que quiz\'as alguna de esas referencias te puede resultar \'util para redactar documentos en~\LaTeX).


Te recordamos que los nombres de las revistas no se abrevian de cualquier manera, sino que hay formas est\'andar de hacerlo. Lo m\'as c\'omodo es seguir lo que hace la AMS,\footnote{Aunque a \textit{La Gaceta de la RSME} no la nombre como solemos hacerlo nosotros.} que puedes encontrar en \url{http://www.ams.org/mathweb/mi-annser.html}
o \url{http://www.ams.org/msnhtml/serials.pdf}. Por supuesto, es lo que se usa en MathSciNet; te puede resultar c\'omodo copiar y pegar desde all\'{\i} si tienes acceso.

\begin{thebibliography}{9}

\bibitem{Be-spanish}
\textsc{J. Bezos}, 
Estilo spanish para el sistema babel,
\url{http://www.ctan.org/tex-archive/language/spanish/babel/spanish.pdf}

\bibitem{Be-orto}
\textsc{J. Bezos}, 
Ortotipograf\'{\i}a y notaciones matem\'aticas,
\url{http://www.texytipografia.com/archive/ortomatem.pdf}

\bibitem{latex-imprenta}
\textsc{B. Cascales, P. Lucas, J. M. Mira, A. Pallar\'es y S. S\'anchez-Pedre\~no}, 
\textit{LaTeX, una imprenta en tus manos}, 
Aula Documental de Investigaci\'on, Madrid, 2000.

\bibitem{Do}
\textsc{M. Downes},
Short math guide for \LaTeX,
\url{ftp://ftp.ams.org/pub/tex/doc/amsmath/short-math-guide.pdf}

\bibitem{Kn}
\textsc{D. E. Knuth}, 
Mathematical typography, 
\textit{Bull. Amer. Math. Soc. (N.S.)} 
\textbf{1} (1979), 337--372. 

\bibitem{MeSl}
\textsc{A. Mertz y W. Slough},
Graphics with PGF and Ti\textit{k}Z,
\textit{The Prac\TeX\ Journal} 2007, n.\textsuperscript{o}~1.
Disponible en
\url{http://www.tug.org/pracjourn/2007-1/mertz/}

\bibitem{latex-companion}
\textsc{F. Mittelbach, M. Goossens, J. Braams, D. Carlisle y C. Rowley}, 
\textit{The \LaTeX\ Companion}, 
2.\textsuperscript{a} ed., 
Addison-Wesley, 2004. 

\bibitem{Tao-ICM} 
\textsc{T. Tao}, 
The dichotomy between structure and randomness, arithmetic progressions, and the primes, 
\textit{International Congress of Mathematicians} (Madrid, 2006), 
Vol. I, 581--608, \textit{Eur. Math. Soc.}, Zurich, 2007.

\end{thebibliography}

%---------------------------
\end{document}
%---------------------------

