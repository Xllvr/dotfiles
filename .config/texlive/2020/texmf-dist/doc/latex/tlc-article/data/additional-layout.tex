% This use case demonstrates tlc-article being extended.  All definitions are
% process during preamble phase.  In other words, before your \begin{document}
% statement.

% ------------------------------------------------------------------------------
% \makeatletter is used so we can reference commands and definitions defined by
% tlc-article, which are all prefaced with tlc@.
\makeatletter

% ------------------------------------------------------------------------------
% tlc-article.tex (Getting Starting) definitions.

\def\tlcProduct{tlc-article}%

\def\tlcA{\tlcDarkblue{\tlcProduct}}%

\def\tlcAL{\tlcDarkblue{\tlc@additionalLayout}}%
\def\tlcBL{\tlcDarkblue{tlcBeginLandscape}}%
\def\tlcDB{\tlcDarkblue{tlcDarkblue}}%
\def\tlcEL{\tlcDarkblue{tlcEndLandScape}}%
\def\tlcHF{\tlcDarkblue{\tlc@headerFooter}}%
\def\tlcLG{\tlcDarkblue{\tlc@logoFile}}%
\def\tlcNCT{newcolumn type: \tlcDarkblue{L, C} \& \tlcDarkblue{R}}%
\def\tlcTOC{\tlcDarkblue{tlcTitlePageAndTableOfContents}}%
\def\tlcVE{\tlcDarkblue{\tlc@versionFile}}%

\def\tlcVC{\tlcDarkblue{tlc@version}}%
\def\tlcDC{\tlcDarkblue{tlc@date}}%
\def\tlcSC{\tlcDarkblue{tlc@status}}%
\def\tlcIC{\tlcDarkblue{tlc@instatution}}%
\def\tlcPC{\tlcDarkblue{tlc@permission}}%

\def\kpse{\$(kpsewhich -var-value TEXMFLOCAL)}%
\def\texDist{\kpse}%
\def\tlcDist{/tex/latex/\tlcProduct}%
\def\tlcGlobalDist{\texDist\tlcDist}%

\def\tlcHome{\$HOME}%
\def\tlcMyDoc{\tlcHome/mydoc}%

\def\gitHub{GitHub.com}%
\def\gitHubUrl{http://\gitHub}%

\def\tlcRepo{git@\gitHub:Traap/\tlcProduct.git}%

\def\tlcPkgFile{data/required-packages.csv}%
\def\tlcNote{\tlcDarkblue{Note}}%

% ------------------------------------------------------------------------------%
% Define the column names used by csvreader when reading \packageFile.
\csvnames{tlcPkgNames}{
   1=\name
  ,2=\description
}

% Define the table style used to report the required package names and
% descriptions.
\csvstyle{tlcPkgStyle}{
  longtable=|L{3cm}|L{12cm}|
  ,table head=\hline Name & Description\\\hline\hline\endhead
  ,late after line=\\\hline
  ,tlcPkgNames
}

% ------------------------------------------------------------------------------%
