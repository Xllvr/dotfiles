%
% bgteubner class bundle
%
% vorwort.tex
% Copyright 2003--2012 Harald Harders
%
% This program may be distributed and/or modified under the
% conditions of the LaTeX Project Public License, either version 1.3
% of this license or (at your opinion) any later version.
% The latest version of this license is in
%    http://www.latex-project.org/lppl.txt
% and version 1.3 or later is part of all distributions of LaTeX
% version 1999/12/01 or later.
%
% This program consists of all files listed in manifest.txt.
% ===================================================================
\preface{Vorwort zu Version~1.21}%

Als ich gemeinsam mit Prof.\ Joachim R�sler und Dr.\ Martin B�ker das
im Teubner Verlag erschienene Buch "`Mechanisches Verhalten der
Werkstoffe"' \cite{roesler2003a} geschrieben habe, war mir keine
Dokumentklasse f�r \LaTeX\ zu Erstellung von B�chern f�r den Teubner
Verlag bekannt. 
Auch die Autorenrichtlinien kannte ich nicht.
Daher habe ich selbst ein Layout entwickelt, mit dem ich das
Manuskript Herrn Dr.\ Feuchte von Teubner Verlag zur Ansicht
�bergeben habe.

Obwohl es Autorenrichtlinien und eine Dokumentklasse f�r den Teubner
Verlag gab, bat Herr Dr.\ Feuchte nur um geringf�gige �nderungen wie
z.\,B.\ die Anpassung an den festgelegten Satzspiegel.  

Als ich Herrn Dr.\ Feuchte das endg�ltige Manuskript �bergab, bot ich
ihm an, eine Dokumentklasse f�r den Teubner Verlag basierend auf dem
Layout unseres Buches zu entwickeln.
Das Lektorat des Teubner Verlags hat daraufhin einige weitere
Ver�nderungen gegen�ber meinem Layout beschlossen (z.\,B.\ Verwendung
der Times als Brotschrift sowie Kolumnentitel au�en) und mir den
Auftrag erteilt, eine Dokumentklasse zu entwickeln.  

Die hier vorliegenden Autorenrichtlinien beschreiben zum einen das
Layout, das der Dokumentklasse zugrunde liegt, zum anderen, wie
Autoren die Dokumentklasse verwenden k�nnen.

Dieses Dokument ist zwar sehr lang, arbeiten Sie aber wenigstens den
zweiten Teil ("`Durchf�hrung mit \LaTeX"') ab
Seite~\pageref{part:latex} durch und beachten Sie die Checkliste ab
Seite~\pageref{sec:checkliste}.
Dadurch k�nnen Sie viele Probleme vermeiden, die dann entstehen,
wenn Sie Ihre normale Vorlage mit wenigen �nderungen �bernehmen.

Bei diesen Autorenrichtlinien ist genau das aufgetreten, was
eigentlich nie passieren darf:
Das Stichwortverzeichnis ist bei weitem nicht ausf�hrlich genug.
Mir war an dieser Stelle wichtiger, dass alle Dinge korrekt erkl�rt
sind und die Dokumentklasse m�glichst fehlerfrei ist, als dass der
letzte Feinschliff an diesen Autorenrichtlinien durchgef�hrt wurde.
Durch die laufenden Erweiterungen kann es an einigen Stellen
vorkommen, dass �berg�nge etwas holprig klingen. 
Wenn es dadurch schwer verst�ndliche Dinge geben sollte, bitte ich
Sie, mir bescheid zu sagen (\url{h.harders@tu-bs.de}).

An dieser Stelle m�chte ich Herrn Dr.\ Feuchte danken, zum einen, dass
er mein Angebot angenommen hat, zum anderen f�r die sehr gute
Zusammenarbeit.

\signature{Braunschweig}{im M�rz 2004}{Harald Harders}


% ===================================================================
\preface{Vorwort zu Version~2.02}%

Im Laufe der Zeit sind durch die Benutzung der Klasse durch die
Autoren immer wieder kleine M�ngel zu Tage getreten, und es haben sich
Verbesserungsm�glichkeiten ergeben.

Ich bitte zu entschuldigen, wenn sich dadurch hin und wieder Umbr�che
�ndern.
Sofern Sie nicht ein Problem haben, sollten Sie in der Endphase eines
Projekts lieber bei der laufenden Version bleiben.

Au�erdem wurde die mehrmalige Umbenennung des Verlags �ber
Vieweg+Teubner in Springer Vieweg ber�cksichtigt.

\signature{M�lheim an der Ruhr}{im April 2012}{Harald Harders}

% ===================================================================
\preface{Vorwort zu Version~2.11}%

Aufgrund von Ver�nderungen anderer Pakete (insbesondere Koma-Script) wurde es
notwendig, diese Klasse auch anzupassen.

\signature{M�lheim an der Ruhr}{im Mai 2015}{Harald Harders}

% ===================================================================

%%% Local Variables: 
%%% mode: latex
%%% TeX-master: bgteubner.tex
%%% TeX-master: "bgteubner"
%%% End: 
