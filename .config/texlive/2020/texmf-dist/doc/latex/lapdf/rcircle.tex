\input preamble.tex

\Defdim(\a,0)
\Defdim(\d,0)
\Defdim(\x,0)
\Defdim(\y,0)

\def\Ncirc(#1){\Rad(#1,\a)
 \Cos(\Np\a,\x) \Add(\x,\x)
 \Sin(\Np\a,\y) \Add(\y,\y)
 \Circle(64)(\Np\x,\Np\y,2) \Stroke}

\def\ctitle#1{{\huge\bf{#1}}}

\title{\Huge \bf{Drawing Circles \\
       with \\
       Rational Quadratic Bezier Curves}}
\author{Detlef Reimers, detlefreimers@gmx.de}
\date{\today}

% ---------------------------------------------------------------------------
\begin{document}
\maketitle

\begin{center}
\begin{lapdf}(8, 10)(-4,-5)
 \Whiledim{\d<360}{\Nextcol(0,23) \Ncirc(\Np\d) \Dadd(\d,15)}
\end{lapdf}

\ctitle{Description}
\end{center}
This document explains, how to calculate the bezier points for
complete circles. These can be drawn with the \texttt{Rcurve}
commmand from the \texttt{lapdf.sty}. If the weight of the point
$P_1$ is $w=\cos(\alpha)$, where $\alpha$ ist the angle between
$P_{0}P_{1}$ and $P_{1}P_{2}$, then the conic will be a circular
arc, if also both length $P_{0}P_{1}$ and $P_{1}P_{2}$ are equal.

We have to smothly join several of these arcs together, to get
a full circle. Only in the case of two segments, we have have to
use one negative weight. In all other cases we only have positive
weights. In all of the following calculations and drawings we assume,
that the center of the circle lies at the origin.

\pagebreak
\parskip0.3cm

\begin{center}
\ctitle{General calculation scheme}

\begin{lapdf}(14,13.5)(-7,-6.5)
 \Dash(1)
 \Setwidth(0.01)
 \Polygon(+0.00,-5.00)
 (+3.63,-5.00)(+4.76,-1.55)
 (+5.88,+1.91)(+2.94,+4.05)
 (+0.00,+6.18)(-2.94,+4.05)
 (-5.88,+1.91)(-4.76,-1.55)
 (-3.63,-5.00)(+0.00,-5.00) \Stroke
 \Line(+0.00,+0.00)(+0.00,-5.00) \Stroke
 \Line(+0.00,+0.00)(+3.63,-5.00) \Stroke
 \Line(+0.00,+0.00)(+4.76,-1.55) \Stroke
 \Line(+0.00,+0.00)(+5.88,+1.91) \Stroke
 \Line(+0.00,+0.00)(+2.94,+4.05) \Stroke
 \Line(+0.00,+0.00)(+0.00,+6.18) \Stroke
 \Line(+0.00,+0.00)(-2.94,+4.05) \Stroke
 \Line(+0.00,+0.00)(-5.88,+1.91) \Stroke
 \Line(+0.00,+0.00)(-4.76,-1.55) \Stroke
 \Line(+0.00,+0.00)(-3.63,-5.00) \Stroke
 \Dash(0)
 \Setwidth(0.02)
 \Blue
 \Circle(96)(0,0,6.18) \Stroke
 \Red
 \Circle(96)(0,0,5) \Stroke
 \Black
 \Rcurve(64)(+0.00,-1.30,1)(+0.42,-1.30,0.5)(0.76,-1.05,1) \Stroke
 \Point(1)(+3.63,-5.00)
 \Point(1)(+4.76,-1.55)
 \Point(1)(+5.88,+1.91)
 \Point(1)(+2.94,+4.05)
 \Point(1)(+0.00,+6.18)
 \Point(1)(-2.94,+4.05)
 \Point(1)(-5.88,+1.91)
 \Point(1)(-4.76,-1.55)
 \Point(1)(-3.63,-5.00)
 \Point(1)(+0.00,-5.00)
 \Text(+3.63,-5.10,tl){$P_1$}
 \Text(+4.86,-1.55,tl){$P_2$}
 \Text(+5.98,+2.01,cl){$P_3$}
 \Text(+3.04,+4.15,bl){$P_4$}
 \Text(+0.00,+6.38,bc){$P_5$}
 \Text(-3.04,+4.15,br){$P_6$}
 \Text(-5.98,+2.01,cr){$P_7$}
 \Text(-4.86,-1.55,tr){$P_8$}
 \Text(-3.63,-5.10,tr){$P_9$}
 \Text(+0.00,-5.10,tc){$P_0=P_{10}$}
 \Text(0.10,-2.90,cl){$r$}
 \Text(0.10,+3.10,cl){$R$}
 \Text(0.25,-0.80,cc){$\alpha$}
\end{lapdf}
\end{center}
We always put $P_0$ at the bottom of the circle and all other points
follow counterclockwise.
This is the general procedure for circle construction with rational
quadratic bezier curves (see picture):
\begin{enumerate}
\item Set the radius $r$.
\item Set the number of bezier segments $n$.
\item Calculate $\alpha = \displaystyle {360^\circ \over 2n}$.
\item Calculate outer radius $R=\displaystyle {r \over \cos(\alpha)}$.
\item Calculate all even bezier points
 $P_{2i} = \displaystyle {+r \cdot \sin(2i\cdot\alpha)
           \choose -r \cdot \cos(2i\cdot\alpha)}$ for $i=0 \dots n$.
\item Calculate odd bezier points
 $P_{2i+1} = \displaystyle {+R \cdot \sin((2i+1)\cdot\alpha)
             \choose -R \cdot \cos((2i+1)\cdot\alpha)}$ for $i=0 \dots n-1$.
\end{enumerate}
You can control your calculations, if you check your endpoint $P_{2n}$.
This point is equal with $P_0$. All curves are drawn with the
{\tt Rmoveto()} and {\tt Rcurveto()} combination.

\pagebreak

\parskip1cm
\begin{center}
\ctitle{2 Segments}

\begin{lapdf}(6,7.5)(-3,-3)
 \Lingrid(5)(0,1)(-3,3)(-3,5)
 \Dash(1)
 \Setwidth(0.01)
 \Polygon(+2.165,+1.25)
 (+0.00,+5.0)(-2.165,+1.25) \Stroke
 \Dash(0)
 \Red
 \Setwidth(0.02)
 \Rmoveto(+2.165,+1.25,1)
 \Rcurveto(64)(+0.00,+5.0,+0.5)(-2.165,+1.25,1)
 \Rcurveto(64)(+0.00,+5.0,-0.5)(+2.165,+1.25,1) \Stroke
 \Black
 \Point(1)(+2.165,+1.25)
 \Point(1)(+0.000,+5.00)
 \Point(1)(-2.165,+1.25)
\end{lapdf}

Circle with $2n+1=5$ points ($w_{2n}=1$ and $w_{2n+1} = \pm \cos(60^\circ) = \pm 0.5$).
\end{center}
\begin{center}
\ctitle{3 Segments}

\begin{lapdf}(6,7.5)(-3,-3)
 \Lingrid(5)(0,1)(-5,5)(-3,5)
 \Dash(1)
 \Setwidth(0.01)
 \Polygon(+0.00,-2.50)
 (+4.333,-2.50)(+2.165,+1.25)
 (+0.000,+5.00)(-2.165,+1.25)
 (-4.333,-2.50)(+0.000,-2.50) \Stroke
 \Dash(0)
 \Setwidth(0.02)
 \Red
 \Rmoveto(+0.000,-2.50,1)
 \Rcurveto(64)(+4.33,-2.50,0.5)(+2.165,+1.25,1)
 \Rcurveto(64)(+0.00,+5.00,0.5)(-2.165,+1.25,1)
 \Rcurveto(64)(-4.33,-2.50,0.5)(+0.000,-2.50,1) \Stroke
 \Black
 \Point(1)(+0.000,-2.50)
 \Point(1)(+4.333,-2.50)
 \Point(1)(+2.165,+1.25)
 \Point(1)(+0.000,+5.00)
 \Point(1)(-2.165,+1.25)
 \Point(1)(-4.333,-2.50)
\end{lapdf}

Circle with $2n+1=7$ points ($w_{2n}=1$ and $w_{2n+1} = \cos(60^\circ) = 0.5$).
\end{center}

\pagebreak

\begin{center}
\ctitle{4 Segments}

\begin{lapdf}(6,6)(-3,-3)
 \Lingrid(5)(0,1)(-3,3)(-3,3)
 \Dash(1)
 \Setwidth(0.01)
 \Polygon(+2.50,+0.00)
 (+2.50,+2.50)(+0.00,+2.50)
 (-2.50,+2.50)(-2.50,+0.00)
 (-2.50,-2.50)(+0.00,-2.50)
 (+2.50,-2.50)(+2.50,+0.00) \Stroke
 \Dash(0)
 \Setwidth(0.02)
 \Red
 \Rmoveto(+2.50,+0.00,1)
 \Rcurveto(64)(+2.50,+2.50,0.707)(+0.00,+2.50,1)
 \Rcurveto(64)(-2.50,+2.50,0.707)(-2.50,+0.00,1)
 \Rcurveto(64)(-2.50,-2.50,0.707)(+0.00,-2.50,1)
 \Rcurveto(64)(+2.50,-2.50,0.707)(+2.50,+0.00,1) \Stroke
 \Black
 \Point(1)(+2.50,+0.00)
 \Point(1)(+2.50,+2.50)
 \Point(1)(+0.00,+2.50)
 \Point(1)(-2.50,+2.50)
 \Point(1)(-2.50,+0.00)
 \Point(1)(-2.50,-2.50)
 \Point(1)(+0.00,-2.50)
 \Point(1)(+2.50,-2.50)
\end{lapdf}

Circle with $2n+1=9$ points ($w_{2n}=1$ and $w_{2n+1} = \cos(45^\circ) = 0.707$).
\end{center}
\begin{center}
\ctitle{5 Segments}

\begin{lapdf}(6,6)(-3,-3)
 \Lingrid(5)(0,1)(-3,3)(-3,3)
 \Dash(1)
 \Setwidth(0.01)
 \Polygon(+0.00,-2.50)
 (+1.815,-2.500)(+2.38,-0.775)
 (+2.940,+0.905)(+1.47,+2.025)
 (+0.000,+3.090)(-1.47,+2.025)
 (-2.940,+0.905)(-2.38,-0.775)
 (-1.815,-2.500)(+0.00,-2.500) \Stroke
 \Dash(0)
 \Setwidth(0.02)
 \Red
 \Rmoveto(+0.00,-2.500,1)
 \Rcurveto(64)(+1.815,-2.500,0.809)(+2.380,-0.775,1)
 \Rcurveto(64)(+2.940,+0.905,0.809)(+1.470,+2.025,1)
 \Rcurveto(64)(+0.000,+3.090,0.809)(-1.470,+2.025,1)
 \Rcurveto(64)(-2.940,+0.905,0.809)(-2.380,-0.775,1)
 \Rcurveto(64)(-1.815,-2.500,0.809)(+0.000,-2.500,1) \Stroke
 \Black
 \Point(1)(+1.815,-2.500)
 \Point(1)(+2.380,-0.775)
 \Point(1)(+2.940,+0.905)
 \Point(1)(+1.470,+2.025)
 \Point(1)(+0.000,+3.090)
 \Point(1)(-1.470,+2.025)
 \Point(1)(-2.940,+0.905)
 \Point(1)(-2.380,-0.775)
 \Point(1)(-1.815,-2.500)
 \Point(1)(+0.000,-2.500)
\end{lapdf}

Circle with $2n+1=11$ points ($w_{2n}=1$ and $w_{2n+1} = \cos(36^\circ) = 0.809$).
\end{center}

\pagebreak

\begin{center}
\ctitle{6 Segments}

\begin{lapdf}(6,6)(-3,-3)
 \Lingrid(5)(0,1)(-3,3)(-3,3)
 \Dash(1)
 \Setwidth(0.01)
 \Polygon(+2.50,+0.00)
 (+2.50,+1.445)(+1.25,+2.165)
 (+0.00,+2.885)(-1.25,+2.165)
 (-2.50,+1.445)(-2.50,+0.000)
 (-2.50,-1.445)(-1.25,-2.165)
 (+0.00,-2.885)(+1.25,-2.165)
 (+2.50,-1.445)(+2.50,+0.000) \Stroke
 \Dash(0)
 \Setwidth(0.02)
 \Red
 \Rmoveto(+2.50,+0.000,1)
 \Rcurveto(64)(+2.50,+1.445,0.866)(+1.25,+2.165,1)
 \Rcurveto(64)(+0.00,+2.885,0.866)(-1.25,+2.165,1)
 \Rcurveto(64)(-2.50,+1.445,0.866)(-2.50,+0.000,1)
 \Rcurveto(64)(-2.50,-1.445,0.866)(-1.25,-2.165,1)
 \Rcurveto(64)(+0.00,-2.885,0.866)(+1.25,-2.165,1)
 \Rcurveto(64)(+2.50,-1.445,0.866)(+2.50,+0.000,1) \Stroke
 \Black
 \Point(1)(+2.50,+0.000)
 \Point(1)(+2.50,+1.445)
 \Point(1)(+1.25,+2.165)
 \Point(1)(+0.00,+2.885)
 \Point(1)(-1.25,+2.165)
 \Point(1)(-2.50,+1.445)
 \Point(1)(-2.50,+0.000)
 \Point(1)(-2.50,-1.445)
 \Point(1)(-1.25,-2.165)
 \Point(1)(+0.00,-2.885)
 \Point(1)(+1.25,-2.165)
 \Point(1)(+2.50,-1.445)
\end{lapdf}

Circle with $2n+1=13$ points ($w_{2n}=1$ and $w_{2n+1} = \cos(30^\circ) = 0.866$).
\end{center}

\begin{center}
\ctitle{7 Segments}

\begin{lapdf}(6,6)(-3,-3)
 \Lingrid(5)(0,1)(-3,3)(-3,3)
 \Dash(1)
 \Setwidth(0.01)
 \Polygon(+0.00,-2.50)
 (+1.205,-2.50)(+1.955,-1.560)
 (+2.705,-0.62)(+2.440,+0.555)
 (+2.170,+1.73)(+1.085,+2.255)
 (+0.000,+2.78)(-1.085,+2.255)
 (-2.170,+1.73)(-2.440,+0.555)
 (-2.705,-0.62)(-1.955,-1.560)
 (-1.205,-2.50)(+0.000,-2.500) \Stroke
 \Dash(0)
 \Setwidth(0.02)
 \Red
 \Rmoveto(+0.000,-2.500,1)
 \Rcurveto(64)(+1.205,-2.50,0.901)(+1.905,-1.560,1)
 \Rcurveto(64)(+2.705,-0.62,0.901)(+2.440,+0.555,1)
 \Rcurveto(64)(+2.170,+1.73,0.901)(+1.085,+2.255,1)
 \Rcurveto(64)(+0.000,+2.78,0.901)(-1.085,+2.255,1)
 \Rcurveto(64)(-2.170,+1.73,0.901)(-2.440,+0.555,1)
 \Rcurveto(64)(-2.705,-0.62,0.901)(-1.905,-1.560,1)
 \Rcurveto(64)(-1.205,-2.50,0.901)(+0.000,-2.500,1) \Stroke
 \Black
 \Point(1)(+0.000,-2.500)
 \Point(1)(+1.205,-2.500)
 \Point(1)(+1.905,-1.560)
 \Point(1)(+2.705,-0.620)
 \Point(1)(+2.440,+0.555)
 \Point(1)(+2.170,+1.730)
 \Point(1)(+1.085,+2.255)
 \Point(1)(+0.000,+2.780)
 \Point(1)(-1.085,+2.255)
 \Point(1)(-2.170,+1.730)
 \Point(1)(-2.440,+0.555)
 \Point(1)(-2.705,-0.620)
 \Point(1)(-1.905,-1.560)
 \Point(1)(-1.205,-2.500)
\end{lapdf}

Circle with $2n+1=15$ points ($w_{2n}=1$ and $w_{2n+1} = \cos(25.71^\circ) = 0.901$).
\end{center}
\end{document}
