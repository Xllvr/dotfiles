%%
%% This is file `ekaia_example.tex',
%% generated with the docstrip utility.
%%
%% The original source files were:
%%
%% ekaia.dtx  (with options: `example')
%% 
%% Copyright (C) 2014-2019, Edorta Ibarra a and the Ekaia Journal (UPV/EHU)
%% -----------------------------------------------------------------------
%% 
%% This work may be distributed and/or modified under the
%% conditions of the LaTeX Project Public License, either version 1.3c
%% of this license or (at your option) any later version.
%% The latest version of this license is in
%% http://www.latex-project.org/lppl.txt
%% and version 1.3c or later is part of all distributions of LaTeX
%% version 2008-05-04 or later.
%% 
%% This work has the LPPL maintenance status `maintained'.
%% 
%% The Current Maintainer of this work is Edorta Ibarra.
%% 
%% This work consists of the files ekaia.dtx, ekaia.ins
%% and the derived files ekaia.sty, ekaia.pdf and
%% ekaia_EUS.pdf.
%% 
%% 
%% Description
%% 
%% This package configures the document layout and provides a set of commands and
%% environments to generate an article following the style of the University of the
%% Basque Country Science and Technology Journal ``Ekaia''.
%% 
%% \CharacterTable
%%  {Upper-case    \A\B\C\D\E\F\G\H\I\J\K\L\M\N\O\P\Q\R\S\T\U\V\W\X\Y\Z
%%   Lower-case    \a\b\c\d\e\f\g\h\i\j\k\l\m\n\o\p\q\r\s\t\u\v\w\x\y\z
%%   Digits        \0\1\2\3\4\5\6\7\8\9
%%   Exclamation   \!     Double quote  \"     Hash (number) \#
%%   Dollar        \$     Percent       \%     Ampersand     \&
%%   Acute accent  \'     Left paren    \(     Right paren   \)
%%   Asterisk      \*     Plus          \+     Comma         \,
%%   Minus         \-     Point         \.     Solidus       \/
%%   Colon         \:     Semicolon     \;     Less than     \<
%%   Equals        \=     Greater than  \>     Question mark \?
%%   Commercial at \@     Left bracket  \[     Backslash     \\
%%   Right bracket \]     Circumflex    \^     Underscore    \_
%%   Grave accent  \`     Left brace    \{     Vertical bar  \|
%%   Right brace   \}     Tilde         \~}
%%

 %%%Example of an Ekaia article created in LaTeX
 \documentclass[twoside,a4paper,11pt]{article}
 \usepackage[review]{ekaia}
 %\usepackage[final]{ekaia} %Aldatu aukera 'final'era onartutako bertsioa bidaltzean
 \begin{document}

 \izenburua{Ekaia Aldizkariko egileentzako gidalerroak}
 \azpiizenburua{Ekaia: Guidelines for authors}

 \begin{autoreak}
 \textit{Egile guztien izenak$^1$}
 \linebreak
 $^1$Egile guztien afilizazioa (instituzioa, hiria, herrialdea)
 \linebreak
 Egile nagusiaren helbide elektronikoa
 \linebreak
 Egile nagusiaren helbide osoa (Instituzioa, kalea, zenbakia, posta kutxatila,
 hiria, herrialdea)
 \linebreak
 Egile nagusiaren ORCID zenbakia
 \end{autoreak}

 \datak{XXXX-XX-XX}{XXXX-XX-XX}

 \begin{laburpena}
 (250 hitz gehienez eta paragrafo batean)
 \end{laburpena}

 \begin{hitz-gakoak}
 Ekaia, \LaTeX{}, euskara.
 \end{hitz-gakoak}

 \begin{abstract}
 (Maximum 250 words and one paragraph)
 \end{abstract}

 \begin{keywords}
 Ekaia, \LaTeX{}, Basque language.
 \end{keywords}

 \section{Egileentzako gidalerro orokorrak}

 Jarraian aurkezten dira Ekaia Aldizkarian publikatzeko autoreek kontuan izan behar
 dituzten gidalerro orokorrak.

 \begin{enumerate}
 \item Ekaia aldizkarian Osasun Zientziak, Natur Zientziak, Zientzia Zehatzak eta
 Teknologiari buruzko idazlan originalak argitaratuko dira, goi-mailako dibulgazioaren
 eta ikerketaren esparruan.
 \item Ale bakoitzean 20 artikulu argitaratuko dira gehienez eta urtean ale arrunt bat
 eta beste ale berezi bat (gehienez 10 artikulurekin) argitaratuko dira. Urte berean
 argitaratuko den alerako lanak urte horretako maiatzaren 31 baino lehen bidalitakoak
 izango dira; data horretatik kanpo bidalitako lanak hurrengo urteko aleko
 edizio-prozesuan sartuko dira.
 \item Artikulu-egileek, dokumentuaren bertsio elektronikoa ``doc'' edo ``pdf''
 artxibo"-formatoan (gure kasuan, ``doc'' formatuarekin lan egitea erosoagoa zaigu)
 bidali dezakete OJS on-line sistemara (http://www.ehu.es/ekaia).
 \item Hizkuntza matematikorako (formulak,\ldots) \LaTeX{} formatoa ere onartuko da,
 dena den, aldi berean euskara zuzentzailearentzako ``pdf'' formatoan ere bidali
 behar da.
 \item Testua euskara batuaz idatzita egongo da, hizkera argia eta zehatza erabiliz,
 eta literatura zientifikoaren usadio eta konbentzioak betez. Nomenklatura, laburdurak,
 ikurrak etab., nazioarteko kodeen arabera idatziko dira. XUXEN zuzentzaile elektronikoa
 pasatuta bidaltzea ere eskatzen da.
 Artikuluaren hasieran, izenburua, egileen izen-abizenak osorik (ezin da izenaren
 laburdura hizkia jarri), afiliazioak eta harremanetarako pertsonaren helbide
 elektronikoa jarriko dira.
 Artikuluaren testu orokorraren aurretik laburpena (abstract-a) eta hitz gakoak
 (key-words-ak) jarri behar dira euskaraz eta ingelesez.
 Artikuluaren atalak era librean aukeratu daitezke. Dena den, ezinbestean lehenengo
 atala eta azkenengoa sarrera eta bibliografia izan beharko dira, hurrenez hurren.
 Testuaren formatoaren ezaugarriak ezagutzeko on-line sisteman jarri diren txantiloiak
 jarraitu: ''doc” formatoarentzako eta \LaTeX formatoarentzako txantiloiak dituzue
 eskuragarri. 10.000 eta 15.000 karaktere (tarterik gabe) bitarteko lana izan behar da.
 Testuan, irudi, argazki eta taulen kokapena adieraziko da. Egileak irudi, argazki eta
 taula bakoitza deskribatzen duen oina bidali beharko du eta irudia, argazkia edo taula
 norberak egina ez bada nondik jaso den ere adierazi beharko da. Noski, berriz
 argitaratzeko eskubideak dituen pertsona edo erakundearen baimenarekin. Egile eskubideak
 urratzen badira artikuluaren egilea izango da egindakoaren arduradun, Ekaiak ez du kasu
 horietan ardurarik hartuko.
 Aipamen bibliografikoak testuan azaltzen diren ordenean zenbatuko dira. Posible den
 kasuan erreferentzietan egileen izen eta abizenak osorik jartzea gomendatzen da. eta
 lanaren amaieran, ondorengo eran idatziko dira:
 \begin{itemize}
 \item[a)] Liburuetarako: [Zenbakia]. Egilea(k). Urtea. Liburuaren izenburua kurtsibaz.
 Argitaletxea, Argitalpen-herria.

 Adibidez, [3] SCHMIDT K. 1975. \textit{Respiration in air}. Cambridge University Press,
 London.
 \item[b)] Artikuluetarako: [Zenbakia]. Egilea(k). Urtea. "<Artikuluaren izenburua">.
 Aldizkariaren izena kurtsibaz, bolumena beltzez, orriak.

 Adibidez: [5] PUSKA M.J. eta NIEMINEN R.M. 1994. "<Theory of positrons in solids and
 no solid surfaces">. \textit{Reviews of Modern Physics}, 66, 841-896.
 \item[c)] Sareko erreferentzien URLarekin batera web orrialdearen eguneratze-data
 adierazi beharko da.
  \end{itemize}
 \item Bidalitako artikulua ez da beste inon argitaratua izan ezta argitaratua izan
 dadin beste aldizkari batera bidalia ere.
 \item Egileek gaian adituak diren eta euskaraz dakiten bi ebaluatzaile zientifiko
 proposatu beharko dituzte.
 \item Erredakzio-batzordeko zuzendariak arloko koordinatzaileari bidaliko dio artikulua
 eta arloko koordinatzaileak gaian aditua den ebaluatzaile bati. Ebaluatzailearen
 iruzkinak eta arloko koordinatzailearen erabakia (argitaratzeko prest, zuzenketa
 txikiak, zuzenketa nagusiak) egileari itzuliko zaizkio. Egileak zuzenketak egin eta
 gero, artikuluaren argitalpenaz azken erabakia erredakzio-batzordeak izango du.
 \item Ebaluazio zientifikoa gainditu duten artikulu guztiak euskara zuzentzailearen
 eskuetatik pasatuko dira. Euskara zuzentzaileak ezinbestez zuzendu behar dena gorriz eta
  iradokizunak berdez adieraziko ditu testuan bertan. Egilearen lehenengo zuzenketaren
 ostean euskara zuzentzaileak erabakiko du bigarren zuzenketa bat beharrezkoa den edo ez.
 \item Azkenik, Ekaia aldizkariak eta Kultura Zientifikoko Katedrak adostutako elkarlana
 bultzatzeko, egileek artikuluaren 400-500 hitzetako laburpen dibulgatiboa egiteko
 konpromezua hartuko dute, Zientzia Kaiera hedabide digitalean argitaratzeko.
 \end{enumerate}

 \section{\LaTeX{}eko \texttt{ekaia.sty} paketea erabiltzeari buruzko oharrak}

 Titulua, egileak, autoreak, etab. zehazteko, beharrezkoa da txantiloiak emandako
 formatua eta komando propioak jarraitzea. Behin derrigorrezko atal horiek bete ondoren,
 \LaTeX{}en ohikoa den bezala idatzi daiteke dokumentua. Hau da, atalak, taulak,
 ekuazioak, irudiak etab. sortzeko, \LaTeX{}en ohikoak diren komando eta inguruneak
 erabil daitezke.

 Beharrezkoa balitz, posible da pakete gehigarriak erabiltzea \verb|\usepackage{}|
 komandoaren bidez, beti ere txantiloiaren formatu originala errespetatzen badute.

 \texttt{ekaia.sty} paketearen dokumentazioan (euskaraz eta ingelesez) aurki daitezke
 pakete horri buruzko informazio gehigarria eta xehetasun teknikoak. Nahi izanez gero,
 \LaTeX{} bidezko euskarazko dokumentu zientifiko-teknikoei buruzko informazio gehigarria
 lor daiteke \cite{ibarra} erreferentzian.

 \renewcommand\refname{\indent Bibliografia}
 \begin{thebibliography}{99}

 \bibitem{ibarra}
 IBARRA E. eta ETXEBARRIA J.R. 2014. "<\LaTeX{}: euskarazko dokumentu zientifiko-teknikoen
  ediziorako baliabideak">. \textit{Ekaia}, 27, 329-343.
 \end{thebibliography}
 \end{document}

 %%OHARRA. Ondorengo inplementazio-kodea txantiloitik borratu daiteke, ez baita beharrezkoa.
 %%Adibide-fitxategia .dtx fitxategi batetatik automatikoki sortu delako agertzen da
 %%hurrengo inplementazio-kodea.
\RequirePackage[english,basque]{babel}
\selectlanguage{basque}
\frenchspacing
\RequirePackage{basque-date}
\linespread{1.3}
\RequirePackage[top=2.6cm,bottom=2.8cm,left=3cm,right=3cm]{geometry}
\RequirePackage{sectsty}
\sectionfont{\fontsize{12}{15}\selectfont}
\subsectionfont{\fontsize{12}{15}\selectfont}
\subsubsectionfont{\fontsize{12}{15}\selectfont}
\RequirePackage{ccicons}
\newcommand{\ekaiafoot}{dummy}
\DeclareOption{review}{\renewcommand{\ekaiafoot}{
\tiny Errebisiorako bertsioa\\
Ekaia aldizkaria (UPV/EHU), \eusdata}
}
\DeclareOption{final}{\renewcommand{\ekaiafoot}{
\tiny\copyright UPV/EHU Press\\
                              ISSN: 0214-9001\\
                              e-ISSN: 2444-3255\\
                              \ccLogo\ccAttribution\ccNonCommercial\ccShareAlike\\
                              Attribution-NonCommercial-ShareAlike\\
                              4.0 international (CC BY-NC-SA 4.0)}
}
\ExecuteOptions{review}
\ProcessOptions\relax
\RequirePackage{fancyhdr}
\pagestyle{fancy}
\renewcommand{\headrulewidth}{0pt}
\fancyhead[L]{ }
\fancyhead[C]{ }
\fancyhead[R]{ }
\fancyfoot[L]{ }
\fancyfoot[C]{\thepage}
\fancyfoot[R]{\ekaiafoot}

\makeatletter
\let\ps@plain\ps@fancy
\makeatother

\makeatletter
\def\ps@headings{%
    \let\@oddfoot\@empty
    \def\@oddhead{{\slshape\rightmark}\hfil\thepage}%
    \let\@mkboth\markboth
    \def\sectionmark##1{%
      \markright {\MakeUppercase{%
        \ifnum \c@secnumdepth >\m@ne
          \thesection\quad
        \fi
        ##1}}}}
\makeatother

\RequirePackage{indentfirst}

\makeatletter
\def\fnum@figure{\textbf{\fontsize{10}{15}\selectfont\thefigure .~irudia}}
\makeatother

\makeatletter
\def\fnum@table{\textbf{\fontsize{10}{15}\selectfont \thetable .~taula}}
\makeatother

\makeatletter
\long\def\@makecaption#1#2{%
\vskip\abovecaptionskip
\sbox\@tempboxa{#1\textbf{.} \fontsize{10}{15}\selectfont #2}%
\ifdim\wd\@tempboxa >\hsize
#1\textbf{.} \fontsize{10}{15}\selectfont #2\par
\else
\global \@minipagefalse
\hb@xt@\hsize{\hfil\box\@tempboxa\hfil}%
\fi
\vskip\belowcaptionskip}
\makeatother

\renewcommand\thesection {\indent \arabic{section}.}
\renewcommand\thesubsection {\thesection \arabic{subsection}.}
\renewcommand\thesubsubsection {\thesubsection \arabic{subsubsection}.}
\renewcommand\theparagraph {\thesubsubsection \arabic{paragraph}.}
\renewcommand\thesubparagraph {\theparagraph \arabic{subparagraph}.}

\newcommand{\izenburua}[1]{
\begin{flushleft}
\fontsize{16}{15}\textbf{#1}\linebreak\fontsize{11}{15}
\end{flushleft}
}

\newcommand{\azpiizenburua}[1]{
\begin{flushleft}
\fontsize{12}{11}\textit{(#1)}\linebreak\fontsize{12}{11}
\end{flushleft}
}

\newcommand{\datak}[2]{
\begin{flushleft}
Jasoa:~{#1}
\linebreak
Onartua:~{#2}
\linebreak
\end{flushleft}
}

\newenvironment{autoreak}
{
\flushright
}

\newenvironment{laburpena}
{
\selectlanguage{basque}
\setlength{\parindent}{0pt}
\textbf{Laburpena:}~\itshape
}{\setlength{\parindent}{0.8cm}\\ }

\renewenvironment{abstract}
{
\selectlanguage{english}
\setlength{\parindent}{0pt}
\textbf{Abstract:}~\bfseries
}{\setlength{\parindent}{0.8cm}\selectlanguage{basque}\\ }

\newenvironment{hitz-gakoak}{
\setlength{\parindent}{0pt}
\textbf{Hitz gakoak:}~
}{\setlength{\parindent}{0.8cm}\\ }

\newenvironment{keywords}{
\setlength{\parindent}{0pt}
\textbf{Keywords:}~
}{\setlength{\parindent}{0.8cm}\\ }

\endinput
%%
%% End of file `ekaia_example.tex'.
