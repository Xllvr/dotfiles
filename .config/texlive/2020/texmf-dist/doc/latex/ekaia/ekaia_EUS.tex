%%
%% This is file `ekaia_EUS.tex',
%% generated with the docstrip utility.
%%
%% The original source files were:
%%
%% ekaia.dtx  (with options: `basque')
%% 
%% Copyright (C) 2014-2019, Edorta Ibarra a and the Ekaia Journal (UPV/EHU)
%% -----------------------------------------------------------------------
%% 
%% This work may be distributed and/or modified under the
%% conditions of the LaTeX Project Public License, either version 1.3c
%% of this license or (at your option) any later version.
%% The latest version of this license is in
%% http://www.latex-project.org/lppl.txt
%% and version 1.3c or later is part of all distributions of LaTeX
%% version 2008-05-04 or later.
%% 
%% This work has the LPPL maintenance status `maintained'.
%% 
%% The Current Maintainer of this work is Edorta Ibarra.
%% 
%% This work consists of the files ekaia.dtx, ekaia.ins
%% and the derived files ekaia.sty, ekaia.pdf and
%% ekaia_EUS.pdf.
%% 
%% 
%% Description
%% 
%% This package configures the document layout and provides a set of commands and
%% environments to generate an article following the style of the University of the
%% Basque Country Science and Technology Journal ``Ekaia''.
%% 
%% \CharacterTable
%%  {Upper-case    \A\B\C\D\E\F\G\H\I\J\K\L\M\N\O\P\Q\R\S\T\U\V\W\X\Y\Z
%%   Lower-case    \a\b\c\d\e\f\g\h\i\j\k\l\m\n\o\p\q\r\s\t\u\v\w\x\y\z
%%   Digits        \0\1\2\3\4\5\6\7\8\9
%%   Exclamation   \!     Double quote  \"     Hash (number) \#
%%   Dollar        \$     Percent       \%     Ampersand     \&
%%   Acute accent  \'     Left paren    \(     Right paren   \)
%%   Asterisk      \*     Plus          \+     Comma         \,
%%   Minus         \-     Point         \.     Solidus       \/
%%   Colon         \:     Semicolon     \;     Less than     \<
%%   Equals        \=     Greater than  \>     Question mark \?
%%   Commercial at \@     Left bracket  \[     Backslash     \\
%%   Right bracket \]     Circumflex    \^     Underscore    \_
%%   Grave accent  \`     Left brace    \{     Vertical bar  \|
%%   Right brace   \}     Tilde         \~}
%%


 %%%Documentation of the ekaia package in Basque
 \documentclass{ltxdoc}
 \begin{document}

 \title{\textsf{ekaia} paketea\thanks{Fitxategi honek (\textsf{ekaia.dtx})
 1.06. bertsioa du. Azken aldiz errebisatua: 19-01-02.}}
 \author{Edorta Ibarra eta Ekaia Aldizkaria (UPV/EHU)\\\texttt{ekaia@ehu.eus}}
 \date{2019-01-02}
 \renewcommand{\contentsname}{Aurkibidea}
 \renewcommand{\refname}{Bibliografia}
 \renewcommand\thesection{\arabic{section}.}
 \renewcommand\thesubsection{\thesection \arabic{subsection}.}
 \maketitle

 \begin{abstract}
  Pakete honek \texttt{article} dokumentu-klasearen itxura aldatzen du
  eta zenbait komando eta ingurune eskaintzen ditu Euskal Herriko Unibertsitateko
  Ekaia Zientzia eta Teknologi aldizkarirako artikuluak sortzeko.
 \end{abstract}

 \tableofcontents

 \section{Sarrera}
 Azken mendeotako aurrerapen zientifikoek eta teknologiaren hedapenak sekulako
 eragina izan dute gizartean, eta funtsezkoak bilakatu dira guztiontzako.
 Horretarako ezinbestekoa izan da zientzia eta gizartearen artean dibulgazioaren
 zubia eraikitzea, ikerketaren emaitzak modu ulergarrian plazaratzea, alegia.

 Euskal herrian zientziaren eta teknikaren dibulgazioak arazo ugari izan ditu,
 euskara bera ez baitzegoen maila teknikorako egokituta. Zorionez, lan handia
 egin da azken urteotan, eta gaur egun unibertsitateko gai ugari euskaraz irakasteaz
 gain, euskaraz diharduten ikerketa-taldeak ere eratu dira. Honek areagotu egin du
 euskarazko dibulgazioaren premia, inguru zientifikoetan sorturiko lexikoak eta
 indartutako tradizioak ez baitute, tamalez, kalean nahikoa oihartzunik izan.
 He\-dabide desberdinetan gai zientifikoak maiz agertzen dira, baina oraindik hutsune
 asko geratzen dira.

 Euskal Herriko Unibertsitateak zientzia- eta teknika-dibulgazioa bultzatzeko Ekaia
 aldizkaria sortu zuen 1989. urtean. Lehen saio honen iraupena laburra zen, tamalez,
 eta Ekaiaren bidea eten zen, 2. alean, hain zuzen ere. 1995. urtean EHUko irakasle-talde
 batek, indarberriturik eta hornikuntza hobearekin, inoiz utziriko bideari berrekin zion,
 bateratasun terminologikoak eta kontzeptu zein aurkikuntza berrien etengabeko agerpenak
 horrela eskatzen dutelakoan.

 Ekaia aldizkaria oinarrizko formazio zientifikoa duten irakurleei bideratuta dago.
 Haren helburuen artean ondokoak daude: zientzia eta teknikaren alorretan egiten diren
 aurrerapenak plazaratzea, unibertsitateko ikasleei zein irakaskuntza ertaineko irakasleei
 testuliburuen osagarriak izango diren materialak eskaintzea, esparru zientifiko-teknikoan
 euskararen estandarizazioa bultzatzea, eta esparru honetan hizkuntzaren erabilerak sortzen
 dituen arazoak konpontzen laguntzea. Xede honekin, urtean bitan Ekaia plazaratzen da, alor
 zientifiko desberdinetako artikuluak biltzen ditu.

 \texttt{article} dokumentu-klasearen itxura moldatzen du \texttt{ekaia} paketeak,
 eta Ekaia aldizkarirako \LaTeX{} bidezko artikuluak prestatzeko erabilgarriak
 diren komando eta inguruneak sortzen ditu.

 \section{Paketea nola deitu}

 \verb|\usepackage| komandoa erabiliz deitzen da
 \verb|ekaia| paketea:\\ \verb|\usepackage{ekaia}|.

 \verb|ekaia| paketearen bertsio honek hurrengo aukerak ditu:
 \begin{itemize}
 \item \texttt{review}: Errebisiorako bertsioa sortzeko hautatu behar da aukera hori.
 \item \texttt{final}: Onartutako bertsioa sortzeko hautatu behar da aukera hori.
 \end{itemize}
 eskaintzen. Hurrengo pakete gehigarriak behar ditu
 \texttt{ekaia} paketeak: \texttt{babel}, \texttt{geometry}, \texttt{sectsty},
 \texttt{fancyhdr}, \texttt{indentfirst}, \texttt{basque-date} eta \texttt{ccicons}.

 \section{Paketearen erabilera}
 \subsection{Paketeak eskaintzen dituen \LaTeX{}en komando eta inguruneak}

 Hurrengo komandoak eskaintzen ditu \texttt{ekaia} paketeak Ekaia aldizkarirako
 artikuluak prestatzeko:

 \begin{itemize}
 \item \verb|\izenburua{}|: Artikuluaren izenburua euskaraz sortzeko erabiltzen da
 komando hori.
 \item \verb|\azpiizenburua{}|: Artikuluaren izenburua ingelesez sortzeko erabiltzen da
 komando hori.
 \item \verb|\datak{}{}|: bidalpen- eta onarpen-datak inprimatzen ditu
 komando ho\-rrek, hurrenez hurren.
 \end{itemize}

 Horrez gain, hurrengo inguruneak eskaintzen ditu \texttt{ekaia}
 paketeak:
 \begin{itemize}
 \item \texttt{autoreak}: Autoreei buruzko informazioa (izen-abizenak, afiliazioa,
 eta kontakturako informazioa) sortzeko erabiltzen da ingurune hori.
 \item \texttt{laburpena}: Artikuluaren laburpena euskaraz sortzeko erabiltzen da
 ingurune hori.
 \item \texttt{hitz-gakoak}: Artikuluaren hitz gakoak euskaraz sortzeko erabiltzen
  da ingurune hori.
 \item \texttt{abstract}: Artikuluaren laburpena ingelesez sortzeko erabiltzen da
 ingurune hori.
 \item \texttt{keywords}: Artikuluaren hitz gakoak ingelesez sortzeko erabiltzen
  da ingurune hori.
 \end{itemize}

 \subsection{Paketearen erabileraren adibidea}

  LaTeX{} bidez Ekaia aldizkarirako
  dokumentuak prestatzeko \verb|ekaia.sty| paketeak eskaintzen dituen komandoen
  eta inguruneen erabilera erakusten du hurrengo \texttt{.tex} kodearen adibideak .\\

 \noindent
 \verb|\documentclass[twoside,a4paper,11pt]{article}|\\
 \verb|\usepackage[review]{ekaia}|\\
 \verb|\begin{document}|\\
 \verb|\izenburua{Ekaia Aldizkariko egileentzako gidalerroak}|\\
 \verb|\azpiizenburua{Ekaia: Guidelines for authors}|\\
 \verb||\\
 \verb|\begin{autoreak}|\\
 \verb|\textit{Egile guztien izenak$^1$}|\\
 \verb|\linebreak|\\
 \verb|$^1$Egile guztien afilizazioa (instituzioa, hiria, herrialdea)|\\
 \verb|\linebreak|\\
 \verb|Egile nagusiaren helbide elektronikoa|\\
 \verb|\linebreak|\\
 \verb|Egile nagusiaren helbide osoa (Instituzioa, kalea, zenbakia,|\\
 \verb|posta kutxatila, hiria, herrialdea)|\\
 \verb|\linebreak|\\
 \verb|Egile nagusiaren ORCID zenbakia|\\
 \verb|\end{autoreak}|\\
 \verb| |\\
 \verb|\datak{XXXX-XX-XX}{XXXX-XX-XX}|\\
 \verb| |\\
 \verb|\begin{laburpena}|\\
 \verb|(250 hitz gehienez eta paragrafo batean)|\\
 \verb|\end{laburpena}|\\
 \verb| |\\
 \verb|\begin{hitz-gakoak}|\\
 \verb|Ekaia, \LaTeX{}, euskara.|\\
 \verb|\end{hitz-gakoak}|\\
 \verb| |\\
 \verb|\begin{abstract}|\\
 \verb| (Maximum 250 words and one paragraph)|\\
 \verb|\end{abstract}|\\
 \verb| |\\
 \verb|\begin{keywords}|\\
 \verb|Ekaia, \LaTeX{}, Basque language.|\\
 \verb|\end{keywords}|\\
 \verb| |\\
 \verb|\section{Sarrera}|\\

 \subsection{Bibliografia sortzen}

 \texttt{thebibliography} ingurunea erabiltzea gomendatzen da
 artikuluaren bibliografia Ekaia aldizkariaren bibliografia-estiloarekin
 bat izan dadin\footnote{http://www.ehu.es/ojs/index.php/ekaia/about/submissions\#authorGuidelines}.
 Hurrengo adibidea baliagarria da prozedura hori ulertzeko.\\

 \noindent
 \verb|\renewcommand\refname{\indent Bibliografia}|\\
 \verb|\begin{thebibliography}{99}|\\
 \verb| |\\
 \verb|\bibitem{ibarra}|\\
 \verb|IBARRA E. eta ETXEBARRIA J.R. 2014. "<\LaTeX{}: euskarazko dokumentu|\\
 \verb|zientifiko-teknikoen ediziorako baliabideak">. \textit{Ekaia},|\\
 \verb|27, 329-343.|\\
 \verb|\end{thebibliography}|\\

 \appendix

 \renewcommand\thesection{A.}
 \renewcommand\thesubsection{\thesection \arabic{subsection}.}

 \section{Eranskinak}

 \subsection{Lizentzia}

 Copyright 2014-2019 Edorta Ibarra eta Ekaia Aldizkaria (UPV/EHU).

 CTAN fitxategietan banatutako \LaTeX\ proiektuko lizentzia
 publikoaren terminoetan birbanatu edota alda daiteke
 programa hau:

 macros/latex/basee/lppl.txe; bai lizentziaren 1.2. bertsioaren
 terminoetan, edota ondorengo edozein bertsioren terminoetan.

 \subsection{Bertsioen historia}

 \begin{itemize}
 \item \textbf{v1.00. bertsioa (14/07/27).} Garapenerako bertsio
 ez publikoa.
 \item \textbf{v1.02. bertsioa (14/12/26).} Lehen bertsio
 publikoa.
 \item \textbf{v1.04. bertsioa (16/11/04).} \verb|azpiizenburua| komandoa
 gehitu da izenburua ingelesez sortzeko. Errore txikiak zuzenduta.
 \item \textbf{v1.06. bertsioa (19/01/02).} Aldaketak txantiloian UPV/EHUko argitalpen
 zerbitzuaren beharrizanak jarraituz.
 \end{itemize}

 \subsection{Inplementazioa}
 Ingelesezko dokumentazioan daude irakurgai paketearen
 inplementazioari buruzko xehetasun teknikoak.
 \end{document}

\RequirePackage[english,basque]{babel}
\selectlanguage{basque}
\frenchspacing
\RequirePackage{basque-date}
\linespread{1.3}
\RequirePackage[top=2.6cm,bottom=2.8cm,left=3cm,right=3cm]{geometry}
\RequirePackage{sectsty}
\sectionfont{\fontsize{12}{15}\selectfont}
\subsectionfont{\fontsize{12}{15}\selectfont}
\subsubsectionfont{\fontsize{12}{15}\selectfont}
\RequirePackage{ccicons}
\newcommand{\ekaiafoot}{dummy}
\DeclareOption{review}{\renewcommand{\ekaiafoot}{
\tiny Errebisiorako bertsioa\\
Ekaia aldizkaria (UPV/EHU), \eusdata}
}
\DeclareOption{final}{\renewcommand{\ekaiafoot}{
\tiny\copyright UPV/EHU Press\\
                              ISSN: 0214-9001\\
                              e-ISSN: 2444-3255\\
                              \ccLogo\ccAttribution\ccNonCommercial\ccShareAlike\\
                              Attribution-NonCommercial-ShareAlike\\
                              4.0 international (CC BY-NC-SA 4.0)}
}
\ExecuteOptions{review}
\ProcessOptions\relax
\RequirePackage{fancyhdr}
\pagestyle{fancy}
\renewcommand{\headrulewidth}{0pt}
\fancyhead[L]{ }
\fancyhead[C]{ }
\fancyhead[R]{ }
\fancyfoot[L]{ }
\fancyfoot[C]{\thepage}
\fancyfoot[R]{\ekaiafoot}

\makeatletter
\let\ps@plain\ps@fancy
\makeatother

\makeatletter
\def\ps@headings{%
    \let\@oddfoot\@empty
    \def\@oddhead{{\slshape\rightmark}\hfil\thepage}%
    \let\@mkboth\markboth
    \def\sectionmark##1{%
      \markright {\MakeUppercase{%
        \ifnum \c@secnumdepth >\m@ne
          \thesection\quad
        \fi
        ##1}}}}
\makeatother

\RequirePackage{indentfirst}

\makeatletter
\def\fnum@figure{\textbf{\fontsize{10}{15}\selectfont\thefigure .~irudia}}
\makeatother

\makeatletter
\def\fnum@table{\textbf{\fontsize{10}{15}\selectfont \thetable .~taula}}
\makeatother

\makeatletter
\long\def\@makecaption#1#2{%
\vskip\abovecaptionskip
\sbox\@tempboxa{#1\textbf{.} \fontsize{10}{15}\selectfont #2}%
\ifdim\wd\@tempboxa >\hsize
#1\textbf{.} \fontsize{10}{15}\selectfont #2\par
\else
\global \@minipagefalse
\hb@xt@\hsize{\hfil\box\@tempboxa\hfil}%
\fi
\vskip\belowcaptionskip}
\makeatother

\renewcommand\thesection {\indent \arabic{section}.}
\renewcommand\thesubsection {\thesection \arabic{subsection}.}
\renewcommand\thesubsubsection {\thesubsection \arabic{subsubsection}.}
\renewcommand\theparagraph {\thesubsubsection \arabic{paragraph}.}
\renewcommand\thesubparagraph {\theparagraph \arabic{subparagraph}.}

\newcommand{\izenburua}[1]{
\begin{flushleft}
\fontsize{16}{15}\textbf{#1}\linebreak\fontsize{11}{15}
\end{flushleft}
}

\newcommand{\azpiizenburua}[1]{
\begin{flushleft}
\fontsize{12}{11}\textit{(#1)}\linebreak\fontsize{12}{11}
\end{flushleft}
}

\newcommand{\datak}[2]{
\begin{flushleft}
Jasoa:~{#1}
\linebreak
Onartua:~{#2}
\linebreak
\end{flushleft}
}

\newenvironment{autoreak}
{
\flushright
}

\newenvironment{laburpena}
{
\selectlanguage{basque}
\setlength{\parindent}{0pt}
\textbf{Laburpena:}~\itshape
}{\setlength{\parindent}{0.8cm}\\ }

\renewenvironment{abstract}
{
\selectlanguage{english}
\setlength{\parindent}{0pt}
\textbf{Abstract:}~\bfseries
}{\setlength{\parindent}{0.8cm}\selectlanguage{basque}\\ }

\newenvironment{hitz-gakoak}{
\setlength{\parindent}{0pt}
\textbf{Hitz gakoak:}~
}{\setlength{\parindent}{0.8cm}\\ }

\newenvironment{keywords}{
\setlength{\parindent}{0pt}
\textbf{Keywords:}~
}{\setlength{\parindent}{0.8cm}\\ }

\endinput
%%
%% End of file `ekaia_EUS.tex'.
