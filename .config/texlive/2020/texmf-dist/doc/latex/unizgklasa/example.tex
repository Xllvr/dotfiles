\documentclass{unizgklasa}
%\graphicspath{} %folder za slike, pročitati upute za pisanje u LaTeXu
%na računalu je potrebno imati pakete:
%upgreek
%titlesec
%tocbibind
%babel
%hypperref

%ubacivanje Unizg fonta ako je potrebno

%\usepackage{fontspec} 
%\setmainfont{UniZGLight.ttf}[
%BoldFont=UniZgBold.ttf,
%ItalicFont=UniZgLightItalic.ttf]

\begin{document}

\imestudenta{Matea Milin}
\sveuciliste{Sveučilište u Zagrebu}
\fakultet{Grafički fakultet}
\naslovrada{\textit{Unizgklasa} - primjer i kratke upute za korištenje}
\rad{\textit{Unizgklasa} - example file and short instructions}
\mentor{Ante Poljičak}
\grad{Zagreb}
\logo{LOGO} % ubaci logo svog fakulteta i definiraj dimenzije
%provjeri kako se adresira folder sa slikama u uputama

\prvanaslovna
\druganaslovna

\section{Uvod/Introduction}
Ova klasa namijenjena je generiranju diplomskih i završnih radova prema uputama Grafičkog fakulteta  Sveučilišta u Zagrebu. Ne odgovara nužno uvjetima svake sastavnice Sveučilišta no izrađena je kao ideja za povezivanje i uniformiranje izgleda svih diplomskih radova. \\

\verb=unizgklasa= distribuira se preko LaTeX Project Public Licence (LPPL) i moguće ju je mijenjati i ponovo distribuirati uz obavezno mijenjanje imena klase kako ne bi dolazilo do zabune. \\

\textit{\textbf{English}}
This class is intended for generating graduate and final theses according to the instructions of the Faculty of Graphic Arts, University of Zagreb. It does not necessarily correspond to the requirements of each component of the University, but it is designed as an idea for linking and uniformizing the look of all graduate papers.\\

\verb=unizgklasa = is distributed through the LaTeX Project Public License (LPPL) and can be changed and redistributed with the mandatory changing of the class name so that there is no confusion.

\section{Upute/Instructions}

Ova klasa zahtjeva pakete  \begin{verbatim} upgreek, titlesec , tocbibind , babel , hypperref
\end{verbatim}


Da bi klasa funkcionirala važno je pravilno popuniti tokene \begin{itemize}
\item imestudenta - upisuje se ime i prezime studenta
\item sveuciliste - upisuje se inaziv sveučilišta
\item fakultet - upisuje se inaziv fakulteta
\item naslovrada - upisuje se naslov rada
\item rad - upisuje se vrsta rada (diplomski, završni, seminar,...)
\item mentor - upisuje se ime, prezime i titula mentora
\item grad - upisuje se grad u kojem je se nalazi fakultet
\item logo - dodaje se logotip sveučilišta i sastavnice s naredbom \verb=\includegraphics=
\end{itemize}
 
Da bi se generirale početne naslovne stranice pozivaju se naredbe \verb=\prvanaslovna= i \verb=\druganaslovna=.
 
Detaljnija objašnjenja klase nalaze se u diplomskom radu \LaTeX\ program za oblikovanje dokumenata i izrada predloška za diplomski rad u \LaTeX\-u dostupnom na repozitoriju Grafičkog fakulteta.\\

\textit{\textbf{English}}\\
This class requires packages  \begin{verbatim} upgreek, titlesec , tocbibind , babel , hypperref
\end{verbatim}

To make the class work, it is important to properly fill out the tokens \begin{itemize}
\item imestudenta - write the name and surname of the student
\item sveuciliste - write the name of univerity
\item fakultet - name of faculty
\item naslovrada - the title of the paper
\item rad - write the type of paper (graduate, final, seminar, ...)
\item mentor - write the name, last name and title of the mentor
\item grad - write the city where the faculty is located
\item logo - add the logo of the university and its component with \verb=\includegraphics= command.
\end{itemize}

In order to generate the title pages, the commands  \verb=\ prvanaslovna= and \verb=\druganaslovna= need to be called.

More detailed explanations of the class are provided in the "\LaTeX{} program za oblikovanje dokumenata i izrada \LaTeX{} predloška za diplomski rad" graduate thesis available on Repository of Faculty of Graphic Arts.
\section{Kontakt/Contact}

Za sva pitanja, komentare i probleme u vezi klase javite se na matea.milin@yahoo.com ili ante.poljicak@grf.hr .\\

For any questions, comments, and class file issues, please contact matea.milin@yahoo.com or ante.poljicak@grf.hr.

\end{document}