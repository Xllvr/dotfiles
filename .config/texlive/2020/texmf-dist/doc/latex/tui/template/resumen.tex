% !TEX root = main.tex
% !TEX encoding = Windows Latin 1
% !TEX TS-program = pdflatex
%
% Archivo: resumen.tex (en espanol)


\chapter{Resumen} % No cambiar el titulo
\selectlanguage{spanish}
\noindent
Duis tristique sollicitudin leo nec consequat. Praesent et dui convallis velit tincidunt fermentum. Mauris cursus purus at sem viverra sed imperdiet sapien imperdiet. Aliquam mattis, elit eget rutrum vulputate, tortor sem pulvinar justo, sit amet mollis felis sem at nibh. Donec malesuada, neque id interdum eleifend, arcu augue porta elit, nec tristique libero metus at massa. Fusce fringilla laoreet rhoncus. Suspendisse potenti. Phasellus dignissim sodales mauris at pharetra. Donec gravida fringilla velit ac rutrum.

Curabitur ornare lectus id diam molestie eu imperdiet nulla tempus. Maecenas vestibulum enim et dui ornare blandit. Vivamus fermentum faucibus viverra. Maecenas at justo sapien. Aenean rhoncus augue mattis purus rhoncus venenatis. Suspendisse metus felis, porttitor in varius in, vulputate at tortor. Aliquam molestie, turpis et malesuada porta, tortor sapien pharetra sapien, ac rhoncus quam dolor a sapien. Pellentesque varius laoreet enim ut auctor. Nullam nec ultricies nisi. Nullam porta lectus et ante consectetur posuere.


Curabitur ornare lectus id diam molestie eu imperdiet nulla tempus. Maecenas vestibulum enim et dui ornare blandit. Vivamus fermentum faucibus viverra. Maecenas at justo sapien. Aenean rhoncus augue mattis purus rhoncus venenatis. Suspendisse metus felis, porttitor in varius in, vulputate at tortor. Aliquam molestie, turpis et malesuada porta, tortor sapien pharetra sapien, ac rhoncus quam dolor a sapien. Pellentesque varius laoreet enim ut auctor. Nullam nec ultricies nisi. Nullam porta lectus et ante consectetur posuere.


\bigskip
\noindent
\textit{Palabras clave:} palabra uno; palabra dos; palabra tres.

\checklanguage
% Fin archivo resumen.tex
\endinput 