%% ----------------------------------------------------------------------
%% START OF FILE
%% ----------------------------------------------------------------------
%% 
%% Filename: ch02-options.tex
%% Author: Fred Qi
%% Created: 2012-12-26 09:15:56(+0800)
%% 
%% ----------------------------------------------------------------------
%%% CHANGE LOG
%% ----------------------------------------------------------------------
%% Last-Updated: 2015-05-14 20:48:32(+0300) [by Fred Qi]
%%     Update #: 77
%% ----------------------------------------------------------------------


\chapter{\texorpdfstring{\XeLaTeX{}}{XeLaTeX}%
          论文模板的功能选项}
\label{cha:options}

本章介绍本论文模板提供的可用功能选项。在使用模板时,最先声明的是使用文档类与及选
项,典型的形式如下:
\begin{lstlisting}[emph={doctor,print}, emphstyle=\textbf]
  \documentclass[doctor,print]{xduthesis}
\end{lstlisting}
其中使用逗号分割开的内容\texttt{doctor,print}为模板的调用选项。下面分别进行介绍。

\section{学位选项}
\label{sec:degree}

学校对不同学位论文模板格式有不同的规定。为此提供学位选项使模板按照对应学位论文的
格式要求进行排版。目前模板支持从本科到博士各个阶段的学位论文格式,提供了下述选
项:

\begin{description}
\item [\texttt{bachelor}] 学士学位论文(即本科毕业设计论文)
\item [\texttt{master}] 学术型硕士学位论文
\item [\texttt{masterpro}] 专业型硕士学位论文
\item [\texttt{doctor}] 博士学位论文
\end{description}

\section{打印选项}
\label{sec:print}

使用\texttt{print}选项后论文章节排版为单开(即章的首页保持为奇数页);为双面打印
方便,还在需要的部分插入了空白页;论文中的超链接使用黑色,以便清晰打印。在电脑上
保存与查看时,可以不选该选项。

该选项默认为关闭,此时章节排版连续,章节之间不会插入空白页;论文中的公式、图表及
参考文献等引用会加入超链接,实现跳转;超链接使用彩色显示。这些设置非常便于在电脑
上查看,建议在编辑阶段使用。当最终打印交付纸质版本时,
请\textbf{务必}打开\texttt{print}选项,以便生成符合学校排版要求的论文。

电子阅读的时代已经到来,\texttt{print}开关选项的设置是为了更多的使用电子文档的便
利特性。因此后续的某些特性可能会导致与学校排版要求的不一致。请以打印选项开启的排
版结果为准检查。


\section{其他选项}
\label{sec:others}


\begin{description}
\item [\texttt{english}] 此选项供使用英文撰写学位论文时使用。启用此选项后,论文
  中除页眉页脚中的“西安电子科技大学某某学位论文”字样保持中文之外,其他部分均使用
  英文。
\item [\texttt{msfonts}] 如果操作系统中安装了微软公司的中文字体时,使用相应字体排版
  论文。为确保论文能够正确排版,需要确认操作系统中安装了宋体(SimSun)、黑体
  (SimHei)、以及楷体(Kaiti\_GB2312)等字体。进行确认时可以使用命令 \texttt{fc-list
  :lang=zh-cn} 查看确认。此选项不能与 \texttt{adobefonts} 选项同时使用。
\item [\texttt{adobefonts}] 如果操作系统中安装了 Adobe 公司的中文字体时,使用相
  应字体排版论文。为了正确排版论文,需要安装宋体(Adobe Song Std)与黑体(Adobe
  Heiti Std)两种字体。此选项与 \texttt{msfonts} 不可同时使用。此选项为默认使用的
  字体选项。
\item [\texttt{secret}] 是否为涉密论文,目前此选项对论文排版没有作用。只要使
  用 \texttt{$\backslash$secretlevel} 命令设置了密级,论文封面的相应部分就会显示
  密级。
\end{description}
% \DeclareOption{secret}{\xdu@secrettrue}
% \DeclareOption{english}{\xdu@englishtrue}
% \DeclareOption{print}{\xdu@printtrue}
% \DeclareOption{msfonts}{\xdu@msfontstrue}
% \DeclareOption{adobefonts}{\xdu@msfontsfalse}


%% ----------------------------------------------------------------------
%%% END OF FILE 
%% ----------------------------------------------------------------------