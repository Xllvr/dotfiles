%% EU FET Open  Proposal LaTeX template
%% V1.0
%% Based on the h2020proposal.cls LaTeX class for writing EU H2020 RIA proposals.
%% 
%% Copyright (c) 2010, Giacomo Indiveri
%%
%%  This latex class is free software: you can redistribute it and/or modify
%%  it under the terms of the GNU General Public License as published by
%%  the Free Software Foundation, either version 3 of the License, or
%%  (at your option) any later version.
%%
%%  h2020proposal.cls is distributed in the hope that it will be useful,
%%  but WITHOUT ANY WARRANTY; without even the implied warranty of
%%  MERCHANTABILITY or FITNESS FOR A PARTICULAR PURPOSE.  See the
%%  GNU General Public License for more details.
%%
%%  You should have received a copy of the GNU General Public License
%%  along with h2020proposal.cls.  If not, see <http://www.gnu.org/licenses/>.
%%
%% Contributors: Elisabetta Chicca
%%
%% Disclaimer: The template is based on the document provided by the EU Participants Portal 
%% "Part B  Template FETOPEN sections 1-3 Final.doc"
%%
%% Use the original source and the http://ec.europa.eu/ documentation for reference. We make no
%% representations or warranties of any kind, express or implied, about the completeness, accuracy,
%% reliability, suitability or availability with respect to the original template.
%% In no event will we be liable for any loss or damage including without limitation, indirect or
%% consequential loss or damage, or any loss or damage whatsoever arising out of, or in connection
%% with, the use of this template and/or class.
%%
%% Makes use of the memoir class. Read the optimum memman documentation for
%% info on how to customize your proposal.


%\documentclass[]{h2020proposal}     % Remove 'draft' option for final version
\documentclass[draft]{h2020proposal} % Use 'draft' option to show comments and labels

% For in-line comments use:
% \marginpar{comment text}

%% Extra Packages
%% ========
%\usepackage{fontspec}% Latin Modern by default with xelatex

%% LaTeX Font encoding -- DO NOT CHANGE
\usepackage[OT1]{fontenc}

%% Input encoding 'utf8'. In some cases you might need 'utf8x' for
%% extra symbols. Not all editors, especially on Windows, are UTF-8
%% capable, so you may want to use 'latin1' instead.
%\usepackage[utf8,latin1]{inputenc}

%% Babel provides support for languages.  'english' uses British
%% English hyphenation and text snippets like "Figure" and
%% "Theorem". Use the option 'ngerman' if your document is in German.
%% Use 'american' for American English.  Note that if you change this,
%% the next LaTeX run may show spurious errors.  Simply run it again.
%% If they persist, remove the .aux file and try again.
\usepackage[english]{babel}

%% For underlined wrapped text.
\usepackage{soul}

%% This changes default fonts for both text and math mode to use Herman Zapfs
%% excellent Palatino font.  Do not change this.
\usepackage[sc]{mathpazo} % Not needed with xelatex

%% The AMS-LaTeX extensions for mathematical typesetting.  Do not
%% remove.
\usepackage{amsmath,amssymb,amsfonts,mathrsfs}

%% Gantt Charts in LaTeX
\usepackage{pgfgantt}

%% LaTeX' own graphics handling
\usepackage{graphicx}

%% Fancy character protrusion.  Must be loaded after all fonts.
\usepackage[activate]{pdfcprot}

%% Nicer tables.  Read the excellent documentation.
\usepackage{booktabs}

% Compressed itemized lists (with a * at the end)
\usepackage{mdwlist}

%% Nicer URLs.  
\usepackage{url}

%% Configure citation styles
\usepackage[numbers,sort&compress,square]{natbib}
\def\bibfont{\footnotesize}     %for smaller fonts in the biblio section

%% Hyper Ref package. In order to operate correctly, it must be the last package declared
\usepackage[colorlinks,pagebackref,breaklinks]{hyperref} 

%% Extra package options
\hypersetup{
  hypertexnames=true, linkcolor=blue, anchorcolor=black,
  citecolor=blue, urlcolor=blue  
}

\urlstyle{rm} %so it doesn't use a typewriter font for urls.
\DeclareGraphicsExtensions{.jpg,.pdf,.mps,.png} % for pdflatex
\graphicspath{{img/} {./}} %put all figures in these dirs

\newcommand{\alert}[1]{{\color{red}\textbf{#1}}}

%%%%%%%%%%%%%%%%%%%%%%%%%%%%%%%%%%%%%%%%%%%%%%%%%%%%%%%%%%%%%%%%%%%%%%

%% ========================================================
%% IMPORTANT store proposal information in global variables
%% ========================================================
\title{Full Title of the Proposal ($<$200 chars)}
\shortname{ACRONYM} 
\titlelogo{}{0.25} % file name and scale
\fundingscheme{Research and Innovation Action}
\topic{Work Programme topic addressed}
\coordinator{Name of coordinator}{email}{fax}
\participant{University of Coordinator}{UoC}{Country1} % First participant is the coordinator
\participant{University of partner 2}{UoP2}{Country2} % as example...
\participant{University of partner 3}{UoP3}{Country3} % as example...
% etc.

% Page Headers
%\makeoddhead{proposal}{\disptoken{@acronym}}{}{\rightmark}
%\makeevenhead{proposal}{\leftmark}{}{\disptoken{@acronym}}

%Page Footers
%\makeevenfoot{proposal}{ \thepage }{ \date{\today} }{ \disptoken{@acronym} }
%\makeoddfoot{proposal}{  }{ \date{\today} }{ \thepage }

%Page Style
\pagestyle{proposal} %use \pagestyle{showlocs} for debugging

%Heading style 
\makeheadstyles{default}{%
\renewcommand*{\chapnamefont}{\normalfont\bfseries}
\renewcommand*{\chapnumfont}{\normalfont\bfseries}
\renewcommand*{\chaptitlefont}{\normalfont\bfseries}
\renewcommand*{\secheadstyle}{\normalfont\bfseries}
}%
\headstyles{default}

%Chapter Style
\chapterstyle{section} %Avoid writing the word "Chapter" at the beginning of each proposal section
% other possible valid styles:
% article, bringhurst, crosshead, culver, dash, demo2, ell, southall, tandh, verville, wilsondob
\renewcommand*{\chaptitlefont}{\normalfont\Large\bfseries}
\renewcommand*{\chapnumfont}{\normalfont\Large\bfseries}



\begin{document}

\instructions{\centerline{\textbf{Proposal template}}}
\instructions{\centerline{\textbf{(Technical annex)}}}
\vskip0.5cm
\instructions{\centerline{\textit{\textbf{Research and Innovation actions}}}}
\vskip0.5cm
\instructions{\centerline{\textit{\textbf{Future and Emerging Technologies:}}}}
\instructions{\centerline{\textit{\textbf{Call FETPROACT adn FETOPEN}}}}
\vskip0.5cm
\instructions{Please follow the structure of this template when preparing your proposal. It has been designed to ensure that the important aspects of your planned work are presented in a way that will enable the experts to make an effective assessment against the evaluation criteria. Sections 1, 2 and 3 each correspond to an evaluation criterion.}
\vskip0.5cm
\instructions{\textbf{Page limit:}}
\vskip0.5cm
\instructions{\textbf{The part B (cover page and sections 1, 2 and 3) is strictly limited to 16 A4 pages and shall consist of:}
\begin{itemize}
\item \textbf{A single A4 title page with acronym, title and abstract of the proposal.}
\item \textbf{Maximum 15 A4 pages consisting of an S\&T section (section 1), an Impact section (section 2) and an Implementation section (section 3).}
\end{itemize}
}
\vskip0.5cm
\instructions{All tables in these sections must be included within this limit. The minimum font size allowed is 11 points.  The page size is A4, and all margins (top, bottom, left, right) should be at least 15 mm (not including any footers or headers).}
\vskip0.5cm
\instructions{\textbf{A proposal that does not comply with these page limits will be declared ineligible.}}
\vskip0.5cm
\instructions{\textit{Important remarks:}
\begin{itemize}
\item \textit{This strict page limitation does not apply to the other additional sections that contain information related to the description of the participating organisations and to the ethics self-assessment.}
\item \textit{The list of the participants’ main scientific publications relevant to the proposal is to be included in section 4. Any other list of scientific publications relevant to the proposal must be included in sections 1-3, that is within the strict page limit.}
\end{itemize}
}

%% TITLE
\maketitle
\instructions{
Maximum length for Sections 1,2,3: 16 pages including all tables.\\ 
First stage proposals have a limit of Cover page + 15 pages.\\
Use the same participant numbering as that used in the administrative proposal forms.
}


\vspace{-4em}
\renewcommand\contentsname{\normalsize Table of Contents \vspace{-4em}}
\setlength{\cftbeforechapterskip}{1.0em plus 0.3em minus 0.1em}
\renewcommand{\cftchapterbreak}{\addpenalty{-4000}}
%\makeparticipantstable          %for the ICT RIA proposals
%\tableofcontents*               %for the ICT RIA proposals
This document gives a quick, relatively minimal example of the use of
\texttt{uafthesis.cls}, while trying to show its features.

This section is contained in \texttt{abstract.tex}.

\thispagestyle{empty}
\pagebreak



%% Main proposal

%% Fixed proposal structure - Do not change
%%% Important. To have correct table numberings
\renewcommand{\thetable}{\thesection\alph{table}}

\chapter[Excellence]{Excellence}
\label{cha:excellence}
\instructions{
Your proposal must address a work programme topic for this call for proposals. \\
\textit{This section of your proposal will be assessed only to the extent that it is relevant to that topic.}\\
}

\section{Objectives}
\label{sec:objectives}
\instructions{
\begin{itemize}
\item Describe the specific objectives for the project\footnote{The term ‘project’ used in this template equates to an ‘action’ in certain other Horizon 2020 documentation.}, which should be clear, measurable, realistic and achievable within the duration of the project. Objectives should be consistent with the expected exploitation and impact of the project (see section 2). 
\end{itemize}
}



\section{Relation to the work programme}
\label{sec:relation-to-work-programme}
\instructions{
\begin{itemize}
\item Indicate the work programme topic to which your proposal relates, and explain how your proposal addresses the specific challenge and scope of that topic, as set out in the work programme.
\end{itemize}
}


\section{Concept and approach}
\label{sec:concept}
\instructions{
\begin{itemize}
\item Describe and explain the overall concept underpinning the project. Describe the main ideas, models or assumptions involved. Identify any trans-disciplinary considerations;\\
\item Describe the positioning of the project e.g. where it is situated in the spectrum from idea to application, or from ``lab to market''. Refer to Technology Readiness Levels where relevant. (See General Annex G of the work programme);\\
\item Describe any national or international research and innovation activities which will be linked with the project, especially where the outputs from these will feed into the project;\\
\item Describe and explain the overall approach and methodology, distinguishing, as appropriate, activities indicated in the relevant section of the work programme, e.g. for research, demonstration, piloting, first market replication, etc;\\
\item Where relevant, describe how sex and/or gender analysis is taken into account in the project's content.
\end{itemize}
\emph{Sex and gender refer to biological characteristics and social/cultural factors respectively. For guidance
on methods of sex / gender analysis and the issues to be taken into account, please refer to
http://ec.europa.eu/research/science-society/gendered-innovations/index\_en.cfm}
}

\section{Ambition}
\label{sec:ambition}
\instructions{
\begin{itemize}
\item Describe the advance your proposal would provide beyond the state-of-the-art, and the extent the proposed work is ambitious. Your answer could refer to the ground-breaking nature of the objectives, concepts involved, issues and problems to be addressed, and approaches and methods to be used.\\
\item Describe the innovation potential which the proposal represents. Where relevant, refer to products and services already available on the market. Please refer to the results of any patent search carried out.\\
\end{itemize}
}
 % Section I
\chapter{Impact}
\label{cha:impact}


\section{Expected impact} 
\label{sec:expected-impact}
\instructions{
\textit{Please be specific, and provide only information that applies to the proposal and its objectives. Wherever possible, use quantified indicators and targets.}\\
\begin{itemize}
\item Describe how your project will contribute to:
\begin{itemize}
\item the expected impacts set out in the work programme under the relevant topic. 
\item improving innovation capacity and the integration of new knowledge (strengthening the competitiveness and growth of companies by developing innovations meeting the needs of European and global markets; and, where relevant, by delivering such innovations to the markets;\\
\item any other environmental and socially important impacts (if not already covered above).
\end{itemize}
\item Describe any barriers/obstacles, and any framework conditions (such as regulation and standards), that may determine whether and to what extent the expected impacts will be achieved. (This should not include any risk factors concerning implementation, as covered in section 3.2.) 
\end{itemize}
}

\section{Measures to maximize impact} 
\label{sec:maximize-impact}

\subsection{Dissemination and exploitation of results}
\label{sec:dissemination-exploitation}
\instructions{
\begin{itemize}
\item Provide a draft ``plan for the dissemination and exploitation of the project's results'' (unless the work programme topic explicitly states that such a plan is not required). For innovation actions describe a credible path to deliver the innovations to the market. The plan, which should be proportionate to the scale of the project, should contain measures to be implemented both during and after the project.\\ 
\emph{Dissemination and exploitation measures should address the full range of potential users and uses including research, commercial, investment, social, environmental, policy making, setting standards, skills and educational training.\\
The approach to innovation should be as comprehensive as possible, and must be tailored to the specific technical, market and organisational issues to be addressed.}\\
\item Explain how the proposed measures will help to achieve the expected impact of the project. Include a business plan where relevant.\\
\item Where relevant, include information on how the participants will manage the research data generated and/or collected during the project, in particular addressing the following issues:\footnote{For further guidance on research data management, please refer to the H2020 Online Manual on the Participant Portal.}
\begin{itemize}
\item What types of data will the project generate/collect?
\item What standards will be used?
\item How will this data be exploited and/or shared/made accessible for verification and re-use? If data cannot be made available, explain why.
\item How will this data be curated and preserved?
\end{itemize}
\emph{You will need an appropriate consortium agreement to manage (amongst other things) the ownership and access to key knowledge (IPR, data etc.). Where relevant, these will allow you, collectively and individually, to pursue market opportunities arising from the project's results.} \\
\emph{The appropriate structure of the consortium to support exploitation is addressed in section 3.3.}\\
\item Outline the strategy for knowledge management and protection. Include measures to provide open access (free on-line access, such as the ``green'' or ``gold'' model) to peer-reviewed scientific publications which might result from the project\footnote{Open access must be granted to all scientific publications resulting from Horizon 2020 actions. Further guidance on open access is available in the H2020 Online Manual on the Participant Portal.}.\\
\emph{Open access publishing (also called 'gold' open access) means that an article is immediately provided in open access mode by the scientific publisher. The associated costs are usually shifted away from readers, and instead (for example) to the university or research institute to which the researcher is affiliated, or to the funding agency supporting the research.}\\
\emph{Self-archiving (also called 'green' open access) means that the published article or the final peer-reviewed manuscript is archived by the researcher - or a representative - in an online repository before, after or alongside its publication. Access to this article is often - but not necessarily - delayed (``embargo period''), as some scientific publishers may wish to recoup their investment by selling subscriptions and charging pay-per-download/view fees during an exclusivity period.}
\end{itemize}
}

\subsection{Communication activities}
\label{sec:communication}
\instructions{
\begin{itemize}
\item Describe the proposed communication measures for promoting the project and its findings during the period of the grant. Measures should be proportionate to the scale of the project, with clear objectives. They should be tailored to the needs of various audiences, including groups beyond the project's own community. Where relevant, include measures for public/societal engagement on issues related to the project. 
\end{itemize}
} % Section II
\chapter{Implementation}\label{chap:implementation}

\section{Management Structure and Procedures}\label{chap:management}
\begin{todo}{from the proposal template}
  Describe the organizational structure and decision-making mechanisms
  of the project. Show how they are matched to the nature, complexity
  and scale of the project.  Maximum length of this section: five pages.
\end{todo}

The Project Management of {\pn} is based on its Consortium Agreement, which will be
signed before the Contract is signed by the Commission. The Consortium Agreement will
enter into force as from the date the contract with the European Commission is signed.
\subsection{Organizational structure}\label{sec:management-structure}
\subsection{Milestones}\label{sec:milestones}
\milestonetable
\subsection{Risk Assessment and Management}
\subsection{Information Flow and Outreach}\label{sec:spread-excellence}
\subsection{Quality Procedures}\label{sec:quality-management}
\subsection{Internal Evaluation Procedures}
\newpage
\section{Individual Participants}\label{sec:partners}
\begin{todo}{from the proposal template}
For each participant in the proposed project, provide a brief description of the legal entity, the main
tasks they have been attributed, and the previous experience relevant to those tasks. Provide also a
short profile of the individuals who will be undertaking the work.\\
Maximum length for Section 2.2: one page per participant. However, where two or more departments within
an organisation have quite distinct roles within the proposal, one page per department is acceptable.\\
The maximum length applying to a legal entity composed of several members, each of which is a separate
legal entity (for example an EEIG1), is one page per member, provided that the members have quite distinct
roles within the proposal.
\end{todo}
\newpage
\begin{sitedescription}{jacu}

\paragraph{Organization} Jacobs University Bremen is a private research university patterned
after the Anglo-Saxon university system.  The university opened in
2001 and has an international student body ($1,245$ students from 102
nations as of 2011, admitted in a highly selective process).

The KWARC (KnoWledge Adaptation and Reasoning for
Content\footnote{\url{http://kwarc.info}}) Group headed by
{\emph{Prof.\ Dr.\ Michael Kohlhase}} specializes in building
knowledge management systems for e-science applications, in particular
for the natural and mathematical sciences.  Formal logic, natural
language semantics, and semantic web technology provide the
foundations for the research of the group.
  
  Since doing research and developing systems is much more fun than writing proposals,
  they try go do that as efficiently as possible, hence this meta-proposal. 

\paragraph{Main tasks}

\begin{itemize}
\item creating {\LaTeX} class files
\end{itemize}

\paragraph{Relevant previous experience}

The KWARC group is the main center and lead implementor of the OMDoc
(Open Mathematical Document) format for representing mathematical
knowledge.  The group has developed added-value services powered by such semantically rich representations, different paths to obtaining them, as well as platforms that integrate both aspects.  Services include the adaptive context-sensitive presentation framework JOMDoc and the semantic search engine MathWebSearch.  For obtaining rich mathematical content, the group has been pursuing the two alternatives of assisting manual editing (with the sTeXIDE editing environment) and automatic annotation using natural language processing techniques.  The latter is work in progress but builds on the arXMLiv system, which is currently capable of converting 70\% out of the 600,000 scientific publications in the arXiv from {\LaTeX} to XHTML+MathML without errors.  Finally, the KWARC group has been developing the Planetary integrated environment.

\paragraph{Specific expertise}

\begin{itemize}
\item writing intelligent proposals
\end{itemize}

\paragraph{Staff members involved}

\textbf{Prof.\ Dr.\ Michael Kohlhase} is head of the KWARC research
group.  He is the head developer of the OMDoc mathematical markup
language.  He was a member of the Math Working Group at W3C, which finished its work with the publication of the MathML 3 recommendation.  He is president of the OpenMath society and trustee of the MKM
interest group.

\keypubs{KohDavGin:psewads11,Kohlhase:pdpl10,Kohlhase:omdoc1.2,CarlisleEd:MathML10,StaKoh:tlcspx10}
\end{sitedescription}

%%% Local Variables: 
%%% mode: LaTeX
%%% TeX-master: "propB"
%%% End: 

% LocalWords:  site-jacu.tex sitedescription emph textbf keypubs KohDavGin
% LocalWords:  psewads11 pdpl10 StaKoh tlcspx10
\newpage
\begin{sitedescription}{efo}
\paragraph{Organization}
 The EFO is the world leader in futurology, \ldots
\paragraph{Main tasks}
\paragraph{Relevant previous experience}
\paragraph{Specific expertise}
\paragraph{Staff members undertaking the work}
\keypubs{providemore}
\end{sitedescription}

%%% Local Variables: 
%%% mode: LaTeX
%%% TeX-master: "propB"
%%% End: 
\newpage
\begin{sitedescription}{bar}

\paragraph{Organization}
  Universit\'e de BAR specializes on drinking lots of red wine. It is a partner in the
  consortium, because it has a very nice chateau on the Cote d'Azure, where it can host
  gorgeous project meetings.

\paragraph{Main tasks}
\paragraph{Relevant previous experience}
\paragraph{Specific expertise}
\paragraph{Staff members undertaking the work}
\keypubs{providemore}

\end{sitedescription}

%%% Local Variables: 
%%% mode: LaTeX
%%% TeX-master: "propB"
%%% End: 
\newpage
\begin{sitedescription}{baz}
\paragraph{Organization}
\paragraph{Main tasks}
\paragraph{Relevant previous experience}
\paragraph{Specific expertise}
\paragraph{Staff members undertaking the work}
\keypubs{providemore}
\end{sitedescription}

%%% Local Variables: 
%%% mode: LaTeX
%%% TeX-master: "propB"
%%% End: 
\newpage

\section{The {\protect\pn} consortium as a whole}
\begin{todo}{from the proposal template}
  Describe how the participants collectively constitute a consortium capable of achieving
  the project objectives, and how they are suited and are committed to the tasks assigned
  to them. Show the complementarity between participants. Explain how the composition of
  the consortium is well-balanced in relation to the objectives of the project.  

  If appropriate describe the industrial/commercial involvement to ensure exploitation of
  the results. Show how the opportunity of involving SMEs has been addressed
\end{todo}

The project partners of the \pn project have a long history of successful collaboration;
Figure~\ref{tab:collaboration} gives an overview over joint projects (including proposals) and
joint publications (only international, peer reviewed ones).

\jointorga{jacu,efo,baz}
\jointpub{efo,baz,jacu}
\jointproj{efo,bar}
\jointsup{jacu,bar}
\jointsoft{baz,efo}
\coherencetable

\subsection{Subcontracting}\label{sec:subcontracting}
\begin{todo}{from the proposal template}
  If any part of the work is to be sub-contracted by the participant responsible for it,
  describe the work involved and explain why a sub-contract approach has been chosen for
  it.
\end{todo}
\subsection{Other Countries}\label{sec:other-countries}
\begin{todo}{from the proposal template}
  If a one or more of the participants requesting EU funding is based outside of the EU
  Member states, Associated countries and the list of International Cooperation Partner
  Countries\footnote{See CORDIS web-site, and annex 1 of the work programme.}, explain in
  terms of the project’s objectives why such funding would be essential.
\end{todo}

\subsection{Additional Partners}\label{sec:assoc-partner}
\begin{todo}{from the proposal template}
  If there are as-yet-unidentified participants in the project, the expected competences,
  the role of the potential participants and their integration into the running project
  should be described
\end{todo}
\section{Resources to be Committed}\label{sec:resources}
\begin{todo}{from the proposal template}
Maximum length: two pages

Describe how the totality of the necessary resources will be mobilized, including any resources that
will complement the EC contribution. Show how the resources will be integrated in a coherent way,
and show how the overall financial plan for the project is adequate.

In addition to the costs indicated on form A3 of the proposal, and the effort shown in Section 1.3
above, please identify any other major costs (e.g. equipment). Ensure that the figures stated in Part B
are consistent with these.
\end{todo}

\subsection{Travel Costs and Consumables}\label{sec:travel-costs}
\subsection{Subcontracting Costs}
\subsection{Other Costs}

%%% Local Variables: 
%%% mode: LaTeX
%%% TeX-master: "propB"
%%% End: 

% LocalWords:  pn newpage site-jacu site-efo site-baz jointpub efo baz
% LocalWords:  jointproj coherencetable assoc-partner
 % Section III

\bibliographystyle{plain}       %References go to end of Section III
\bibliography{refs}

\clearpage

\chapter{Members of the consortium}
\label{cha:members}

\instructions{
\textit{This section is not covered by the page limit.}
\vskip0.2cm
\textit{The information provided here will be used to judge the operational capacity.}
}

\section{Participants (applicants)}
\label{sec:participants}

\instructions{
Please provide, for each participant, the following (if available):\\
\begin{itemize}
\item a description of the legal entity and its main tasks, with an explanation of how its profile matches the tasks in the proposal;
\item a curriculum vitae or description of the profile of the persons, including their gender, who will be primarily responsible for carrying out the proposed research and/or innovation activities;
\item a list of up to 5 relevant publications, and/or products, services (including widely-used datasets or software), or other achievements relevant to the call content;
\item a list of up to 5 relevant previous projects or activities, connected to the subject of this proposal;
\item a description of any significant infrastructure and/or any major items of technical equipment, relevant to the proposed work;
\item any other supporting documents specified in the work programme for this call.
\end{itemize}
}

\section{Third parties involved in the project (including use of third party resources)}
\label{sec:third-parties}

\instructions{
\textit{Please complete, for each participant, the following table (or simply state "No third parties involved", if applicable).} \\
If yes in first row, please describe and justify the tasks to be subcontracted. If yes in second row, please describe the third party, the link of the participant to the third party, and describe and justify the foreseen tasks to be performed by the third party\footnote{A third party that is an affiliated entity or has a legal link to a participant implying a collaboration not limited to the action. (Article 14 of the Model Grant Agreement).}. If yes in third row, please describe the third party and their contributions.}

\begin{tabular}{|p{.85\textwidth}|p{.05\textwidth}|}
  \hline  
  \multicolumn{2}{|l|}{\cellcolor[gray]{0.8}\textbf{UoC}}\\
  \hline
  Does the participant plan to subcontract certain tasks (please note that core tasks of the project should not be sub-contracted) &
  \textbf{Y/N} \\
  \hline
Does the participant envisage that part of its work is performed by linked
third parties &
  \textbf{Y/N} \\
  \hline
  Does the participant envisage the use of contributions in kind provided by
third parties (Articles 11 and 12 of the General Model Grant Agreement) &
  \textbf{Y/N}\\
  \hline
\end{tabular}

\begin{tabular}{|p{.85\textwidth}|p{.05\textwidth}|}
  \hline  
  \multicolumn{2}{|l|}{\cellcolor[gray]{0.8}\textbf{UoP1}}\\
  \hline
  Does the participant plan to subcontract certain tasks (please note that core tasks of the project should not be sub-contracted) &
  \textbf{Y/N} \\
  \hline
Does the participant envisage that part of its work is performed by linked
third parties &
  \textbf{Y/N} \\
  \hline
  Does the participant envisage the use of contributions in kind provided by
third parties (Articles 11 and 12 of the General Model Grant Agreement) &
  \textbf{Y/N}\\
  \hline
\end{tabular}


\begin{tabular}{|p{.85\textwidth}|p{.05\textwidth}|}
  \hline  
  \multicolumn{2}{|l|}{\cellcolor[gray]{0.8}\textbf{UoP2}}\\
  \hline
  Does the participant plan to subcontract certain tasks (please note that core tasks of the project should not be sub-contracted) &
  \textbf{Y/N} \\
  \hline
Does the participant envisage that part of its work is performed by linked
third parties &
  \textbf{Y/N} \\
  \hline
  Does the participant envisage the use of contributions in kind provided by
third parties (Articles 11 and 12 of the General Model Grant Agreement) &
  \textbf{Y/N}\\
  \hline
\end{tabular}
 % Section IV
\chapter{Ethics and Security}
\label{cha:ethics}
\instructions{
\textit{This section is not covered by the page limit.}
}

\section{Ethics}
\label{sec:ethics}
\instructions{
If you have entered any ethics issues in the ethical issue table in the administrative proposal forms, you must:
\begin{itemize}
\item submit an ethics self-assessment, which:
\begin{itemize}
\item describes how the proposal meets the national legal and ethical requirements of the country or countries where the tasks raising ethical issues are to be carried out; 
\item explains in detail how you intend to address the issues in the ethical issues table, in particular as regard:
\begin{itemize}
\item research objectives (e.g. study of vulnerable populations, dual use, etc.)
\item research methodology (e.g. clinical trials, involvement of children and related consent procedures, protection of any data collected, etc.) 
\item the potential impact of the research (e.g. dual use issues, environmental damage, stigmatisation of particular social groups, political or financial retaliation, benefit-sharing,  malevolent use, etc.).
\end{itemize}
\end{itemize}
\item provide the documents that you need under national law(if you already have them), e.g.:
\begin{itemize}
\item an ethics committee opinion;
\item the document notifying activities raising ethical issues or authorising such activities;
\end{itemize}
\end{itemize}
\textit{\indent If these documents are not in English, you must also submit an English summary of them (containing, if available, the conclusions of the committee or authority concerned).}
\vskip0.2cm
\textit{If you plan to request these documents specifically for the project you are proposing, your request must contain an explicit reference to the project title.}
}

\section{Security}\footnote{Article 37.1 of the Model Grant Agreement: Before disclosing results of activities raising security issues to a third party (including affiliated entities), a beneficiary must inform the coordinator -- which must request written approval from the Commission/Agency. Article 37.2: Activities related to ``classified deliverables'' must comply with the ``security requirements'' until they are declassified. Action tasks related to classified deliverables may not be subcontracted without prior explicit written approval from the Commission/Agency. The beneficiaries must inform the coordinator -- which must immediately inform the Commission/Agency -- of
any changes in the security context and --if necessary -- request for Annex 1 to be amended (see Article 55).
}
\label{sec:security}
\instructions{
Please indicate if your project will involve:
\begin{itemize}
\item activities or results raising security issues: (YES/NO)
\item ``EU-classified information'' as background or results: (YES/NO)
\end{itemize}
}
 % Section V

\appendix

\chapter{The first appendix}
\section{a section}

\chapter{The second appendix}
\section{a section}
\subsection{a subsection}
 % Appendix

\backmatter

\end{document}
