\newcommand{\frenchabstract}{%
  La présente classe, \yatcl, a pour objet de faciliter la composition de
  mémoires de thèses préparées en France, quels que soient les champs
  disciplinaires et instituts. Elle implémente notamment l'essentiel des
  recommandations émanant du \citeauthor{guidoct} et ce, de façon transparente
  pour l'utilisateur.  Elle a en outre été conçue pour (facultativement) tirer
  profit de plusieurs outils puissants disponibles sous \LaTeX{}, notamment les
  packages :
  \begin{itemize}
  \item \package{biblatex} pour la bibliographie ;
  \item \package{glossaries} pour les glossaire, liste d'acronymes et liste
    de symboles.
  \end{itemize}
  %
  La \yatCl{}, basée sur la \Class{book}, se veut à la fois simple d'emploi et,
  dans une certaine mesure, (aisément) personnalisable.%
}

\begin{abstract}
  \medskip

  \frenchabstract
\end{abstract}
%
\begin{abstract}
  \medskip

  The purpose of the current class, \yatcl, is to facilitate dissertations'
  typesetting of theses prepared in France, whatever disciplines and
  institutes. It implements most notably recommendations from the Ministry of
  Higher Education and Research and this, transparently to the user. It has also
  been designed to (optionally) take advantage of powerful tools available in
  \LaTeX{}, including packages:
  \begin{itemize}
  \item \package{biblatex} for the bibliography ;
  \item \package{glossaries} for the glossary, list of acronyms and symbols list.
  \end{itemize}
  The \yatCl{}, based on the \Class{book}, aims to be both simple to use and, to
  some extent, (easily) customizable.
\end{abstract}
%
\makeabstract

%%% Local Variables:
%%% mode: latex
%%% TeX-master: "../yathesis-fr"
%%% End:
