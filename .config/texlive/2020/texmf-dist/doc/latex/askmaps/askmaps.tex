%%
%%
%%   askmaps.tex
%%
%%   (c)2013, J.E.J. op den Brouw <J.E.J.opdenBrouw@hhs.nl>
%%
%%   document v0.1 - 2013/12/04
%%

\documentclass[a4paper,10pt]{ltxdoc}
\usepackage[a4paper]{geometry}

%% PDF Version and compression...
\pdfminorversion=5
\pdfobjcompresslevel=2

%% English spelling of chapter, section, etc.
\usepackage[english]{babel}

%% Use some packages
\usepackage[scaled=0.92]{helvet}
\usepackage{sansmath}
\usepackage{color}
\usepackage{float}
\usepackage{listings}
\usepackage{array}
%% The next lines are commented out by default. If you want these to appear
%% in the resulting PDF, you have to copy the kvmacros.tex file in the
%% current directory, uncomment the next lines and LaTex this file.
%%
%%\input kvmacros

%% Use askmaps
\usepackage{askmaps}

%% Use hyperrefs in PDFs
\usepackage{hyperref}
\hypersetup{
  colorlinks=true,
  linkcolor=blue,
  pdftitle={American Style Karnaugh Maps},
  pdfauthor={J.E.J op den Brouw},
  pdfsubject={Drawing 2, 3, 4 and 5 variables K-maps},
  pdfkeywords={K-maps, Karnaugh maps, Digital Design}
}
\definecolor{red}{rgb}{1,0,0}
\definecolor{blue}{rgb}{0,0,1}
\definecolor{darkgreen}{rgb}{0,0.4,0}
\definecolor{orange}{rgb}{1,0.5,0}
\definecolor{gray}{rgb}{0.5,0.5,0.5}
\definecolor{lightgray}{rgb}{0.95,0.95,0.95}

%% Making captions nicer...
\usepackage[font=footnotesize,format=plain,labelfont=bf,up,textfont=it,up]
        {caption}

%% Set up the LaTex language layout
\lstset{ %
  language=[LaTeX]tex,
  basicstyle=\footnotesize\fontfamily{pcr}\selectfont,
  commentstyle=\itshape,
  numbers=left,
  numberstyle=\tiny\color{gray},
  stepnumber=1,
  numbersep=5pt,
  backgroundcolor=\color{lightgray},
  showspaces=false,
  showstringspaces=false,
  showtabs=false,
  frame=lines,
  rulecolor=\color{black},
  tabsize=4,
  captionpos=b,
  breaklines=true,
  breakatwhitespace=false,
  title=\lstname,
  morekeywords={RequirePackage,ProvidesPackage},
  aboveskip=\bigskipamount
}

%% Need one more footnote...
%% Display package name boldface
\renewcommand*{\thefootnote}{\fnsymbol{footnote}}
\newcommand\Package[1]{\texttt{#1}}

%% \cmdinvoke et al politely reused from the miscdoc.sty file
\makeatletter
\DeclareRobustCommand\cmdinvoke{\@ifstar
  {\let\@tempa\emph\@scmdinvoke}%
  {\let\@tempa\relax\@scmdinvoke}%
}
\def\@scmdinvoke#1{\texttt{\symbol{92}#1}%
  \futurelet\@let@token\@cmdinvoke
}
\def\@cmdinvoke{\ifx\@let@token\bgroup
    \let\@tempb\@cmdinvoke@lbrace
  \else
    \ifx\@let@token[% ]
      \let\@tempb\@cmdinvoke@lbrack
    \else
      \ifx\@let@token(% )
        \let\@tempb\@cmdinvoke@lparen
      \else
        \let\@tempb\@empty
      \fi
    \fi
  \fi
  \@tempb
}
\def\@cmdinvoke@lbrace#1{\penalty0\hskip0pt\relax
  \texttt{\symbol{123}\@tempa{#1}\symbol{125}}%
  \futurelet\@let@token\@cmdinvoke
}
\def\@cmdinvoke@lbrack[#1]{\penalty-150\hskip0pt\relax
  \texttt{[\@tempa{#1}]}%
  \futurelet\@let@token\@cmdinvoke
}
\def\@cmdinvoke@lparen(#1){\penalty-150\hskip0pt\relax
  \texttt{(\@tempa{#1})}%
  \futurelet\@let@token\@cmdinvoke
}
\makeatother

%% No indent
\setlength{\parindent}{0pt}

%% At last, the document...
\begin{document}
\title{The \Package{askmaps} package}
\author{Jesse op den Brouw\thanks{The Hague University of Applied Sciences,
        \url{J.E.J.opdenBrouw@hhs.nl}}}
\date{\askmapsdate, version \askmapsversion}
\maketitle
\setcounter{footnote}{1}

\section{Abstract}
\label{sec:abstract}
This package generates configurable American style Karnaugh maps for
2, 3, 4 and 5 variables as can be found in numerous books on digital design.
Note that there are many ways to draw correct Karnaugh
maps\footnote{E.g., a four-variable Karnaugh map can be drawn in 384 different
and correct ways} and this packages creates only one version.

There are more packages and examples that produce Karnaugh maps. A highly
recommendable package is written by Andreas W. Wieland\footnote{See
\url{http://www.ctan.org/tex-archive/macros/latex/contrib/karnaugh}}.
However, this package creates maps that are frequently found in dutch
textbooks\footnote{E.g. Digitale Techniek, A.P. Thijssen}
but not in American textbooks.

\section{Overview}
\label{sec:overview}
After \cmdinvoke{usepackage}{askmaps}, four new commands are loaded
to draw Karnaugh maps using the picture environment:
\begin{table}[H]
\begin{tabular}{p{2cm}p{10cm}}
\cs{askmapii}  & draws a two-variable Karnaugh map \\ 
\cs{askmapiii} & draws a three-variable Karnaugh map \\ 
\cs{askmapiv} & draws a four-variable Karnaugh map \\ 
\cs{askmapv} & draws a five-variable Karnaugh map \\
\end{tabular} 
\end{table}

These commands have five parameters which have the same meaning for each
commands. This will be explained in section \ref{sec:outline}. As can be seen
later on, much of the parameters can include typesetting commands such as
coloring and math notation.

The new dimension \cs{askmapunitlength} is available and specifies
the length of the squares. It defaults to 1 cm (about 0.4 in) which is a good
value for 12 pt font size.

The global definitions \cs{askmapsversion} and \cs{askmapsdate} will render
to the current version and date respectively.

The command \cs{askmap} provides drop-in replacement for the \cs{karnaughmap}
command available in the \Package{kvdocs} package (although it is not really a
package) and internally calls one of the four
commands, but there are some drawbacks on using this command.

\newpage
\section{Outline}
\label{sec:outline}
All four functions take the form
\cs{askmap<romannumber>\{\#1\}\{\#2\}\{\#3\}\{\#4\}\{\#5\}}
where \texttt{<romannumber>} is one of \texttt{ii}, \texttt{iii}, \texttt{iv}
or \texttt{v} and the parameters \texttt{\#1} to \texttt{\#5} have the
following meaning:

\begin{itemize}
	\itemsep0pt
	%\labelsep0pt
	\item[\texttt{\#1}] is the function output variable.
	\item[\texttt{\#2}] is a list of function input variables.
	\item[\texttt{\#3}] contains a list op options, see table below.
	\item[\texttt{\#4}] is a list of function values.
	\item[\texttt{\#5}] can be used to display user defined picture commands.
\end{itemize}

If a parameter contains fewer elements than needed, you will get empty spaces
in the Karnaugh map. If a parameter contains more elements than needed, all
elements in excess are not printed. In both situations, you will not be
notified.\newline

The third parameter contains a list of options as explained below.
\begin{itemize}
    \itemsep0pt
    \item[\texttt{i}] index numbers are printed in the lower left corner of
                      each square.
    \item[\texttt{I}] no index numbers are printed (default).
    \item[\texttt{f}] function output variable is printed at the upper right
                      corner of the Karnaugh map together with a small line
                      extending from the square to the function name (default).
    \item[\texttt{F}] no function output variable is printed.
    \item[\texttt{b}] bit combinations of the function input variables are
                      printed on top of the columns and at the right of the the
                      rows (default).
    \item[\texttt{B}] bit combinations are not printed.
    \item[\texttt{c}] shortcut for \texttt{ifb}.
    \item[\texttt{C}] shortcut for \texttt{IFB}.
\end{itemize}

Options are evaluated from left to right, so a combination of \texttt{iI} will
yield in no index numbers being printed. Please note that macros are not
expanded to a list of tokens but to one token.\newline

If you look at American style Karnaugh maps, you will see that the function
output variable usually isn't printed with the map, but is printed below the
map (if it is printed at all). This behavior can be simulated by using the
\texttt{F} option, in which case the first parameter (the function output
variable) may be left unused.


\section{Use}
\label{sec:use}
Let's say we have the following truth table for a function S with two variables
a and b and the function values 1, 0, 1 and 1 respectively.

\begin{table}[H]
\centering
\caption{Truth table}
\begin{tabular}{cc|c}
\hline 
a & b & S \\ \hline 
0 & 0 & 1 \\ 
0 & 1 & 0 \\ 
1 & 0 & 1 \\ 
1 & 1 & 1 \\ 
\hline 
\end{tabular} 
\end{table}

Using the command
\begin{verbatim}
  \askmapii{S}{ab}{i}{1011}{}
\end{verbatim}
will produce the two-variable Karnaugh map as presented in
figure~\ref{fig:askmapii}.
\begin{figure}[h]
\centering
\askmapunitlength=0.88cm
\askmapii{S}{ab}{i}{1011}{}%
\caption{Karnaughmap for two variables}
\label{fig:askmapii}
\end{figure}

As you can see, the function output variable is printed top right of the
Karnaugh map, with a small line from the variable to the top right square.
The most significant function input variable is printed on the top left just
above the small line extending from the top left square. The least significant
function input variable is printed just below that line. The bit combinations
for the most significant function input variable is printed on top of the
columns, the bit combinations of the least significant function input variable
is printed on the left of the rows. You can see how the function values are
placed in the squares by observing the small index numbers in the lower left
corners of the squares.
\medskip

The command 
\begin{verbatim}
  \askmapiii{S}{abc}{}{10110110}{}
\end{verbatim}
will produce a three-variable Karnaugh map as presented in
figure~\ref{fig:askmapiii}. In this case, the index numbers are not printed.

\begin{figure}[H]
\centering
\askmapunitlength=0.88cm
\askmapiii{S}{abc}{}{10110110}{}%
\caption{Karnaugh map for three variables}
\label{fig:askmapiii}
\end{figure}

Let's draw a Karnaugh map for a logic function with four inputs and one output.
The output will become logic 1 if three or four inputs are logic 1, otherwise
the output is logic 0. (Of course you can set up a truth table first; this is
left as an exercise to the reader).
The command 
\begin{verbatim}
  \askmapiv{S}{abcd}{}{0000000100010111}{}
\end{verbatim}
will produce a four-variable Karnaugh map for the given logic function and is
presented in figure~\ref{fig:askmapiv}.

\begin{figure}[H]
\centering
\askmapunitlength=0.88cm
  \askmapiv{S}{abdc}{}{0000000100010111}{}
\caption{Karnaugh map for four variables}
\label{fig:askmapiv}
\end{figure}

Using the command
\begin{verbatim}
  \askmapv{S}{abcde}{}{00101101010010111011110011101011}{}
\end{verbatim}
will produce a five-variable Karnaugh map as shown in figure \ref{fig:askmapv}.

\begin{figure}[H]
\centering
\askmapunitlength=0.88cm
  \askmapv{S}{abcde}{}{00101101010010111011110011101011}{}
\caption{Karnaugh map for five variables}
\label{fig:askmapv}
\end{figure}

Of course you can do a lot more with the parameters of the commands. You
can use inline math to produce sophisticated variable names, you can
use coloring for the function values. However, please note that the second
and fourth parameter act as a list of tokens, so you have to use braces to
separate the tokens in the lists if they consists of multiple characters.
A list of tokens such as
\begin{verbatim}
  {abcd}
\end{verbatim}
will yield a, b, c and d as separate tokens, but
\begin{verbatim}
  {$m_{0}$$m_{1}$$m_{0}$$m_{1}$}
\end{verbatim}
will not work. You have to write
\begin{verbatim}
  {{$m_{0}$}{$m_{1}$}{$m_{0}$}{$m_{1}$}}
\end{verbatim}

The fifth parameter can be used to supply user created picture commands
such as dashed rectangles of ovals to emphasize the simplifications
in the function. Note that the origin
of the picture (0,0) is at the lower left corner of the lower left square.
Section \ref{sec:examples} shows some examples.
\bigskip

The size of the indexes is scriptsize, the size of the bit combinations
on the edges of the Karnaugh maps is footnotesize. All other text is
in normalsize. Note that there are no commands to change the font sizes
in the Karnaugh diagrams.


The command \cs{askmap} provides a drop-in replacement for the command
\cs{karnaughmap} from the Karnaugh package. The command
\begin{verbatim}
  \askmap{4}{F}{abcd}{0100101010100011}{}
\end{verbatim}
produces the Karnaugh map in figure \ref{fig:askmap}.
\begin{figure}[H]
\centering
\askmapunitlength=0.88cm
  \askmap{4}{F}{abcd}{0100101010100011}{}
\caption{Karnaugh map using drop-in replacement}
\label{fig:askmap}
\end{figure}

Only Karnaugh maps of two to five variables are supported, any other number
will trigger a warning. Please note that the \cs{kvindex}, \cs{kvindexsize}
and \cs{kvcontentsize} commands are not supported.
\medskip

%% The next lines are commented out by default. If you want these to appear
%% in the resulting PDF, you have to copy the kvmacros.tex file in the
%% current directory, uncomment the next lines and LaTex this file.
%%
%%The packages \Package{kvmacros} and \Package{askmaps} can be used in the same
%%document.
%%\begin{figure}[H]
%%\centering
%%  \karnaughmap{4}{f(a,b,c,d):}{abcd}{1110011001100110}{}
%%\caption{Karnaugh map using the \Package{kvmacros} package}
%%\label{fig:karnaughmap}
%%\end{figure}

\newpage
\section{Examples}
\label{sec:examples}
\askmapunitlength=0.88cm
The Karnaugh maps for two variables: plain, with indexes, without function
output variable.
\begin{verbatim}
  \askmapii{S}{ab}{}{1010}{}
  \askmapii{S}{ab}{i}{1110}{}
  \askmapii{S}{ab}{F}{1110}{}
\end{verbatim}
\askmapii{S}{ab}{}{1010}{}
\askmapii{S}{ab}{i}{1110}{}
\askmapii{S}{ab}{F}{1110}{}
\bigskip\bigskip

The Karnaugh maps for three variables: plain, with indexes but without function
output variable.
\begin{verbatim}
  \askmapiii{S}{abc}{}{00111010}{}
  \askmapiii{F}{xyz}{iF}{11111010}{}
\end{verbatim}
\askmapiii{S}{abc}{}{00111010}{}
\askmapiii{F}{xyz}{iF}{11111010}{}
\bigskip\bigskip

The Karnaugh maps for four variables: without function output variable, with
indexes.
\begin{verbatim}
  \askmapiv{f}{wxyz}{F}{1110001101000101}{}
  \askmapiv{S}{abcd}{i}{1100100100011101}{}
\end{verbatim}
\askmapiv{f}{wxyz}{F}{1110001101000101}{}
\askmapiv{S}{abcd}{i}{1100100100011101}{}
\bigskip\bigskip

You can also do math things by using the known \$ signs... and you can make an
empty map for your exams...
\newpage
\begin{verbatim}
  \askmapii{$S_{0}$}{{$a_{1}$}{$a_{0}$}}{}{1010}{}
  \askmapii{}{}{}{}{}
\end{verbatim}
\askmapii{$S_{0}$}{{$a_{1}$}{$a_{0}$}}{}{1010}{}
\askmapii{}{}{}{}{}
\bigskip\bigskip

You can do the math thing in roman font... and of course you can make a
\textsl{really} empty map for your exams...
\begin{verbatim}
  \begin{verbatim}
  \askmapiii{$\mathrm{M^{n+1}_{0}}$}{{$\mathrm{M^{n}_{2}}$}{$\mathrm{M^{n}_{1}}$}
            {$\mathrm{M^{n}_{0}}$}}{}{11100111}{}
  \askmapiii{}{}{BF}{}{}
\end{verbatim}
\askmapiii{$\mathrm{M^{n+1}_{0}}$}{{$\mathrm{M^{n}_{2}}$}{$\mathrm{M^{n}_{1}}$}
          {$\mathrm{M^{n}_{0}}$}}{}{11100111}{}
\askmapiii{}{}{BF}{}{}
\bigskip\bigskip

You can do things with don't cares...
\begin{verbatim}
  \askmapii{S}{ab}{}{011-}{}
  \askmapiii{S}{abc}{F}{001--10-}{}
\end{verbatim}
\askmapii{S}{ab}{}{011-}{}
\askmapiii{S}{abc}{F}{001--10-}{}
\bigskip\bigskip

You can use colors and empty function values and variables as values too...
\begin{verbatim}
  \askmapiii{S}{abc}{}{{\color{blue}{0}}{\color{blue}{0}}{\color{red}{1}}{ }{ }
               {\color{red}{1}}{\color{blue}{0}}{ }}{}
  \askmapii{S}{ab}{}{{$i_{0}$}{$i_{1}$}{$i_{2}$}{$i_{3}$}}{}
\end{verbatim}
\askmapiii{S}{abc}{}{{\color{blue}{0}}{\color{blue}{0}}{\color{red}{1}}{ }{ }
          {\color{red}{1}}{\color{blue}{0}}{ }}{}
\askmapii{S}{ab}{}{{$i_{0}$}{$i_{1}$}{$i_{2}$}{$i_{3}$}}{}
\bigskip

\newpage
You can set the font to something else ... and use sans math font ...
\begin{verbatim}
  {\fontfamily{phv}\selectfont%
  \askmapiv{S}{abcd}{}{0110111011110011}{}%
  }
  {\fontfamily{phv}\selectfont\sansmath
  \askmapiv{$Q^{n+1}_{0}$}{{$p^{n}_{1}$}{$p^{n}_{0}$}{$q^{n}_{1}$}{$q^{n}_{0}$}}
           {}{0110111011110011}{}%
  }
\end{verbatim}
{\fontfamily{phv}\selectfont%
\askmapiv{S}{abcd}{}{0110111011110011}{}%
}
{\fontfamily{phv}\selectfont\sansmath
\askmapiv{$Q^{n+1}_{0}$}{{$p^{n}_{1}$}{$p^{n}_{0}$}{$q^{n}_{1}$}{$q^{n}_{0}$}}
         {}{0110111011110011}{}%
}
\bigskip\bigskip

You can use the last parameter to create to your own picture commands. Note
that the origin of the picture (0,0) is at the lower left corner of the lower
left square. It is also possible to use \cs{raisebox} in conjuction with
horizontal spacing to adjust the placing of the left (least significant)
variable(s).
\begin{verbatim}
  {\fontfamily{phv}\selectfont\sansmath
  \askmapiii{F}{xy{\raisebox{2ex}{\ \ z}}}{}{11100111}{% raise Z input
  \color{red}\put(0.1,0.1){\dashbox{0.1}(0.8,1.8){}}%
  \color{blue}\put(1.1,1.1){\dashbox{0.1}(1.8,0.8){}}%
  \color{darkgreen}\put(2.1,0.1){\dashbox{0.1}(1.8,0.8){}}%
  \color{orange}\put(0.15,1.15){\dashbox{0.1}(1.7,0.7){}}%
  }}%
  {\fontfamily{phv}\selectfont\sansmath
  \askmapiii{F}{xyz}{}{11100111}{%
  \put(0.5,1.0){\oval(0.8,1.8)}%
  \put(2.0,1.5){\oval(1.8,0.8)}%
  \put(2.5,0.5){\oval(0.8,0.8)}% circle
  \put(3.5,0.5){\oval(0.8,0.8)[l]}% these three create
  \put(3.5,0.9){\line(1,0){0.6}}% a oval open to the left
  \put(3.5,0.1){\line(1,0){0.6}}%
  }}%
\end{verbatim}
{\fontfamily{phv}\selectfont\sansmath
\askmapiii{F}{xy{\raisebox{1.5ex}{\ \ z}}}{}{11100111}{% raise Z input
\color{red}\put(0.1,0.1){\dashbox{0.1}(0.8,1.8){}}%
\color{blue}\put(1.1,1.1){\dashbox{0.1}(1.8,0.8){}}%
\color{darkgreen}\put(2.1,0.1){\dashbox{0.1}(1.8,0.8){}}%
\color{orange}\put(0.15,1.15){\dashbox{0.1}(1.7,0.7){}}%
}}%
{\fontfamily{phv}\selectfont\sansmath
\askmapiii{F}{xyz}{}{11100111}{%
\put(0.5,1.0){\oval(0.8,1.8)}%
\put(2.0,1.5){\oval(1.8,0.8)}%
\put(2.5,0.5){\oval(0.8,0.8)}%    circle
\put(3.5,0.5){\oval(0.8,0.8)[l]}% these three create
\put(3.5,0.9){\line(1,0){0.7}}%   an oval open to the left
\put(3.5,0.1){\line(1,0){0.7}}%
}}%
\bigskip\bigskip

You can set the length of the squares, but please note that this will also
affect the rendering of the text...
\begin{verbatim}
  \askmapunitlength=1cm
  \askmapii{S}{ab}{}{1011}{}
  \askmapunitlength=0.88cm
  \askmapii{S}{ab}{}{1011}{}
  \askmapunitlength=0.8cm
  \askmapii{S}{ab}{}{1011}{}
  \askmapunitlength=0.5cm
  \askmapii{S}{ab}{}{1011}{}
\end{verbatim}

{\askmapunitlength=1cm
\askmapii{S}{ab}{}{1011}{}
\askmapunitlength=0.88cm
\askmapii{S}{ab}{}{1011}{}
\askmapunitlength=0.8cm
\askmapii{S}{ab}{}{1011}{}
\askmapunitlength=0.5cm
\askmapii{S}{ab}{}{1011}{}}
\bigskip\bigskip

An example directly from the Karnaugh package:

\begin{verbatim}
  \askmap{4}{$f(a,b,c,d):$}{{$a$}{$b$}{$c$}{$d$}}%
  {0110011001100110}%
  {%
  \put(0,2){\oval(1.9,1.9)[r]}
  \put(4,2){\oval(1.9,1.9)[l]}
  \put(2,0){\oval(1.9,1.9)[t]}
  \put(2,4){\oval(1.9,1.9)[b]}
  }
\end{verbatim}
{\askmapunitlength=0.88cm
\askmap{4}{$f(a,b,c,d):$}{{$a$}{$b$}{$c$}{$d$}}%
{0110011001100110}%
{%
\put(0,2){\oval(1.9,1.9)[r]}
\put(4,2){\oval(1.9,1.9)[l]}
\put(2,0){\oval(1.9,1.9)[t]}
\put(2,4){\oval(1.9,1.9)[b]}
}}

\bigskip
but please note that the ovals do not cover the logic 1's in the Karnaughmap
because the function values are placed differently.

\newpage
\section{The source code}
\label{sec:sourcecode}
The source code is pretty straight forward. It uses a lot of picture primitives
for drawing the maps. The package uses three macros for processing
variable-length parameters, which were completely reused from the karnaugh
package.

\lstinputlisting{askmaps.sty}

\end{document}
