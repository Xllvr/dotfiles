% \iffalse meta-comment
%
% Copyright (C) 2020 by zhaofeng-shu33 <616545598@qq.com>
%
% This work may be distributed and/or modified under the
% conditions of the LaTeX Project Public License, either version 1.3
% of this license or (at your option) any later version.
% The latest version of this license is in
%   http://www.latex-project.org/lppl.txt
% and version 1.3 or later is part of all distributions of LaTeX
% version 2005/12/01 or later.
%
% This work has the LPPL maintenance status `maintained'.
%
% \fi
%
%\iffalse
%<*internal>
\begingroup
\input docstrip.tex
\askforoverwritefalse
\showprogress
\keepsilent
\preamble

This is a generated file.

Copyright (C) \the\year by zhaofeng-shu33 <616545598@qq.com>

This work may be distributed and/or modified under the
conditions of the LaTeX Project Public License, either version 1.3
of this license or (at your option) any later version.
The latest version of this license is in
  http://www.latex-project.org/lppl.txt
and version 1.3 or later is part of all distributions of LaTeX
version 2005/12/01 or later.

To produce the documentation run the original source files ending with `.dtx'
through LaTeX.

\endpreamble
\nopostamble
\generate{
\file{dtx-style.sty}{\from{\jobname.dtx}{dtx-style}}  
\file{iidef.sty}{\from{\jobname.dtx}{sty}}
}

\ifToplevel{%
  \Msg{***********************************************************}
  \Msg{*}
  \Msg{* To produce the documentation run the files ending with}
  \Msg{* `.dtx' through XeLaTeX.}
  \Msg{*}
  \Msg{* Happy TeXing!}
  \Msg{***********************************************************}}
\endgroup
%</internal>
%<*driver>
\ProvidesFile{thucoursework.dtx}[2020/02/23 2.5.1 Tsinghua University Coursework Template]
\documentclass{ltxdoc}
\usepackage{dtx-style}

\EnableCrossrefs
\CodelineIndex
\RecordChanges

\begin{document}
  \DocInput{\jobname.dtx}
\end{document}
%</driver>
% \fi
% \DoNotIndex{\NeedsTeXFormat,\ProvidesClass,\RequirePackage}
% \DoNotIndex{\renewcommand,\underline,\newcommand,\def}
% \DoNotIndex{\setlist,\hfill,\hrule,\vspace}
% \DoNotIndex{\begin,\end,\DeclareStringOption}
% \DoNotIndex{\fancypagestyle,\fancyhead,\pagestyle,\thispagestyle}
% \DoNotIndex{\protect,\mathpalette,\rlap,\mathrel,\mkern}
% \GetFileInfo{\jobname.dtx}
%^^A theses changes are moved from update.md
%^^A \changes{hversioni}{hdatei}{htexti}
% \def\xiangxiang{\textit{Xiangxiang}}
% \def\zhaofeng{\textit{zhaofeng-shu33}}
% \changes{v1.1}{2017/09/26}{[\xiangxiang] Add mathcal and eucal package.}
% \changes{v1.1}{2017/09/26}{[\xiangxiang] Add aligned equations in \texttt{iihw.tex}.}
% \changes{v1.2}{2017/10/09}{[\xiangxiang] Add example for figures in \texttt{iihw.tex}}
% \changes{v1.2}{2017/10/09}{[\xiangxiang] Add example fro input matlab codes in \texttt{iihw.tex}}
% \changes{v1.2}{2017/10/09}{[\xiangxiang] No need to set counter style manually}
% \changes{v1.3}{2017/10/12}{[\xiangxiang] Add example for vector and random vectors(underline) in \texttt{iihw.tex}}
% \changes{v1.4}{2017/10/30}{[\xiangxiang] Add an example of hypothesis testing in \texttt{iihw.tex}}
% \def\indexname{索引}
% \def\glossaryname{修改记录}
% \IndexPrologue{\section{\indexname}}
% \GlossaryPrologue{\section{\glossaryname}}
% \title{\bfseries\color{violet}\thucoursework:清华大学课程作业模板\thanks{本文档在 \textit{xueruini} 的 \href{https://github.com/xueruini/thuthesis}{\enspace \textsc{Thu}\-\textsc{Thesis}} 手册基础上进行改动}}
% \author{{\fangsong 赵丰\thanks{师兄徐祥祥维护到v1.4}}\\[5pt]\texttt{616545598@qq.com}}
% \date{v\fileversion\ (\filedate)}
% \maketitle\thispagestyle{empty}
%
%
% \begin{abstract}\noindent
%   此宏包旨在建立一个简单易用的清华大学课程作业模板,作业主要针对理论物理、数学、信息科学、通信、理论力学等方面。
% \end{abstract}
%
% \vskip2cm
% \def\abstractname{免责声明}
% \begin{abstract}
% \noindent
% \begin{enumerate}
% \item 本模板的发布遵守 \LaTeX\ Project Public License,使用前请认真阅读协议内
%   容。
% \item 任何个人或组织以本模板为基础进行修改、扩展而生成的新的专用模板,请严格遵
%   守 \LaTeX\ Project Public License 协议。由于违犯协议而引起的任何纠纷争端均与
%   本模板作者无关。
% \end{enumerate}
% \end{abstract}
%
%
% \clearpage
% \pagestyle{fancy}
% \begin{multicols}{2}[
%   \setlength{\columnseprule}{.4pt}
%   \setlength{\columnsep}{18pt}]
%   \tableofcontents
% \end{multicols}
% \clearpage
%
% \section{模板介绍}
% \thucoursework\ (\textbf{T}sing\textbf{h}ua \textbf{U}niversity \LaTeX\
% \textbf{Coursework} Template) 是为了帮助清华大学学生撰写课程作业而编写
% 的 \LaTeX\ 模板。最初由徐祥祥师兄担任\textit{Information Inference} 这门课助教期间设计出来,当时命名为\texttt{idef.sty}。后来由赵丰接手进行拓展和完善。
% 本文档将尽量完整的介绍模板的使用方法,如有不清楚之处可以参考示例文档或者根据
% 第~\ref{sec:howtoask} 节说明提问,有兴趣者都可以参与完善此手册,也非常欢迎对代
% 码的贡献。

% \section{安装}
% \label{sec:installation}
%
%^^A 发行版 updated , 2018/5/17 https://mirror.ctan.org/systems/texlive/tlnet/tlpkg
% 阅读文档可以使用以下命令:
% \begin{shell}
% $ texdoc thucoursework
% \end{shell}
%
% 如果要使用开发版,需自己下载,\thucoursework\ 相关链接:
% \begin{itemize}
% \item 主页:\href{https://github.com/zhaofeng-shu33/THU-coursework-template}{GitHub}
% \item 下载:\href{https://www.ctan.org/pkg/thucoursework}{CTAN}
% \end{itemize}
%
% \subsection{模板的组成}
% 下表列出了 \thucoursework\ 的主要文件及其功能介绍:
%
% \begin{longtable}{l|p{8cm}}
% \toprule
% {\heiti 文件(夹)} & {\heiti 功能描述}\\\midrule
% \endfirsthead
% \midrule
% {\heiti 文件(夹)} & {\heiti 功能描述}\\\midrule
% \endhead
% \endfoot
% \endlastfoot
% thucoursework.dtx & \textsc{DocStrip} 源文件(开发用)\\\midrule
% iihw.tex & \xiangxiang 编写的使用模板的英文作业例子\\
% ithw.tex & \zhaofeng 编写的使用模板的中文作业例子\\ \midrule
% Makefile & Makefile\\
%   & latexmk 配置文件 \\
% README.md & Readme\\
% \textbf{thucoursework.pdf} & 用户手册(本文档)\\\bottomrule
% \end{longtable}
%
%
% \subsection{生成模板sty文件和用户手册}
% \label{sec:generate-cls}
% \note[提示:]{若使用 linux 操作系统,可在含有\file{Makefile}的目录下直接运行 \texttt{make doc} 命令
% 生成模板sty文件和用户手册}
%
% 在使用之前可以用下面的代码生成模板sty文件和用户手册:
%
% \begin{shell}
% $ xelatex thucoursework.dtx
% $ makeindex -s gind.ist -o thucoursework.ind thucoursework.idx
% $ makeindex -s gglo.ist -o thucoursework.gls thucoursework.glo
% $ xelatex thucoursework.dtx
% $ xelatex thucoursework.dtx  % 生成说明文档 thucoursework.pdf
% \end{shell}
% 
%
%
% \subsection{生成作业}
% \label{sec:generate-coursework}
% 本节介绍几种常见的生成作业的方法。用户可根据自己的情况选择,使用前请仿照例子里的tex文件,使用 
% \begin{latex}
% \usepackage{iidef}
% \end{latex}
% 引用宏包。
% \subsubsection{\XeLaTeX}
% \label{sec:xelatex}
% 很多用户对 \LaTeX\ 命令执行的次数不太清楚。一个基本的原则是多次运行 \LaTeX\ 命
% 令直至不再出现警告。下面给出生成示例文档的详细过程(\texttt{\#} 开头的行为注
% 释),首先来看推荐的 \texttt{xelatex} 方式:
% \begin{shell}
% # 1. 发现里面的引用关系,文件后缀 .tex 可以省略
% $ xelatex main
%
% # 2. 编译参考文件源文件,生成 bbl 文件
% $ bibtex main
%
% # 3. 下面解决引用
% $ xelatex main
% $ xelatex main   # 此时生成完整的 pdf 文件
% \end{shell}
%
% \subsubsection{PDF\LaTeX}
% \label{sec:pdflatex}
% 当仅使用英文完成课程作业时,可以使用 PDF\LaTeX,编译速度会快很多。按照第~\ref{sec:xelatex} 节的顺序执行即可,只是将命令中
% \texttt{xelatex} 替换为 \texttt{pdflatex}。
%
% 需要注意的是 PDF\LaTeX\ 不能处理常见的 EPS 图形,需要先用 \texttt{epstopdf} 将其转化
% 成 PDF。不过 PDF\LaTeX\ 增加了对 png,jpg 等格式的支持,比较方便。\TeX Live
% 自从 2010 版本起自动调用 \texttt{epstopdf} 将 EPS 图形转化为 PDF。
%
% \subsubsection{latexmk}
% \label{sec:latexmk}
% \texttt{latexmk} 命令支持全自动生成 \LaTeX\ 编写的文档,并且支持使用不同的工具
% 链来进行生成,它会自动运行多次工具直到交叉引用都被解决。下面给出了一个用
% \texttt{latexmk} 调用 \texttt{xelatex} 生成最终文档的示例:
% \begin{shell}
% # 一句话就够了!
% $ latexmk -xelatex Assignment
% \end{shell}
%
% \subsubsection{make}
% \label{sec:make}
% \note[提示:]{使用 \texttt{make} 编译,需要配合linux 操作系统。}
%
%
% \subsection{升级}
% \label{sec:updgrade}
%
% \section{使用说明}
% \label{sec:usage}
% 本手册假定用户已经能处理一般的 \LaTeX\ 文档。如果
% 从来没有接触过 \TeX\ 和 \LaTeX,建议先学习相关的基础知识。
%
% \subsection{关于提问}
% \label{sec:howtoask}
% 按照优先级推荐提问的位置如下:
%
% \begin{itemize}
% \item \href{http://github.com/zhaofeng-shu33/thucoursework/issues}{Github Issues}
% \item \href{http://bbs.ctex.org/nForum/#!board/TeX}{ctex 论坛}
% \end{itemize}
%
% \subsection{示例文件}
% \label{sec:userguide}
%
% 推荐新用户从模板自带的示例文档入手,里面包括了写作业用到的所有命令及其使用方法,
% 只需要用自己的内容进行相应替换就可以。对于不清楚的命令可以查阅本手册。下面的例
% 子描述了模板中作业的组织形式,来自于示例文档,具体内容可以参考模板附带
% \changes{v2.3}{2018/03/30}{[\zhaofeng] Add sample file \file{ithw.tex}}
% 的 \file{iihw.tex} 和 \file{ithw.tex}。
%
% \lstinputlisting[style=lstStyleLaTeX]{iihw.tex}
%
%
% \subsection{作业选项}
% \label{sec:option}
%
% \DescribeOption{thehwcnt}
%   指定作业序号,
%   为可选项。
% \begin{latex}
% % 使用默认序号1
% \usepackage{iidef}
%
% % 指定序号为6
% \usepackage[thehwcnt=6]{iidef}
% \end{latex}
%
% \subsection{基本信息}\label{sec:basic_info}
% \DescribeMacro{\thecourseinstitute}
% 指定学校或学院,如果使用中文,请引用\pkg{ctex} 宏包。
% \begin{latex}
% % 指定学院是TBSI
% \thecourseinstitute{Tsinghua-Berkeley Shenzhen Institute}
%
% % 指定中文名
% \thecourseinstitute{清华大学电子工程系}
% \end{latex}
% 
% \DescribeMacro{\thecoursename}
% 指定课程名
% \begin{latex}
% % 指定课程名是 Information Inference
% \thecoursename{Information Inference}
% % 指定课程名是 "应用信息论"
% \thecoursename{应用信息论}
% \end{latex}
%
% \DescribeMacro{\theterm}
% 指定学期
% \begin{latex}
% % English example
% \theterm{Fall 2017}
% % 中文示例
% \theterm{2018年春季学期}
% \end{latex}
%
% \DescribeMacro{\hwname}
% 指定作业名,默认是 Homework
% \begin{latex}
% % English example
% \hwname{Coursework}
% % 中文示例
% \hwname{作业}
% \end{latex}
% 
% \subsection{标题}
% \DescribeMacro{\courseheader}
% 将其放在正文前面生成作业标题
%
% \note[提示:]{使用 \cs{courseheader} 前需使用~\ref{sec:basic_info}节描述的宏指定相关基本信息。并且使用了\cs{courseheader}无需再用 \LaTeX\ 里的\cs{maketitle}}
% \begin{latex}
% \begin{document}
% \courseheader
% \name{YOUR NAME}
% \end{latex}    
% \DescribeMacro{\name}
%^^A manually indentation 
% \indent 生成姓名、作业名、日期
%
% \note[提示:]{使用 \cs{name} 会直接在文档中输出姓名、作业名、日期,因此如果需要自定义作业名,需要在使用\cs{name} 前用 \cs{hwname} 指定作业名称}
% \subsection{题目自动编号}
% 建议使用嵌套的\texttt{enumerate} 环境对题目自动编号,可比较好的解决自动缩进的问题。
% 本模板设置了一级列表为作业序号加题目序号,比如第6次作业第3题会排版成 \enspace 6.3.\enspace ,对于某一道题目的不同小问,默认是用(a),(b),(c) 进行
% 顺序编排。
% \begin{latex}
%   \begin{enumerate}
%    \item First Problem
%    \item Second Problem
%       \begin{enumerate}
%          \item first question of the second problem
%          \item second question of the second problem
%       \end{enumerate}
%   \end{enumerate}
% \end{latex}
% 
% 如果需要改变默认的设置,比如不同小问使用罗马数字,则可以在正文开始之前使用
% \begin{latex}
% \setlist[enumerate,2]{label=\roman*.}
% \end{latex}
% 进行设置。
%
% \subsection{证明和解的环境}
% \myentry{证明}
% \DescribeEnv{proof}
% \texttt{proof} 环境是 \pkg{amsthm} 宏包提供的,引用本模板后可以直接使用。
% \begin{latex}
%    \begin{proof}
%      proof contents go here
%    \end{proof}
% \end{latex}
%  在中文作业中,引用\pkg{ctex} 宏包
%  后 “证明” 的环境会变成 中文,如果需要改变默认文字,可以在序言中使用 \cs{ctexset} 配置
% \begin{latex}
%   \ctexset{
%       proofname = \heiti{证明} 
%   }  
%   \begin{document}
% \end{latex}
% \myentry{解}
% \DescribeEnv{solution}
% \texttt{solution} 环境是本模板提供的,用于回答“求解XXX”之类的问题。
% \begin{latex}
%    \begin{solution}
%      solution contents go here
%    \end{solution}
% \end{latex}
% \DescribeMacro{slname}
% 在中文作业中,可能需要将 \texttt{solution} 环境默认的 \textit{solution.} 变成中文,为此可以将
% \begin{latex}
%    \slname{\heiti{解}}
% \end{latex}          
% 放在使用 \texttt{solution} 环境之前。
% \subsection{常用数学符号}
% 本文档预先定义了常用的数学符号,在\file{iihw.tex} 示例文件中对这些符号的使用有详细说明。
% \section{致谢}
% \label{sec:thanks}
% 感谢师兄徐祥祥的最初的作业模板设计, \textsc{Thu}\-\textsc{Thesis} 宏包的文档源码。
% \changes{v2.4}{2018/04/09}{[\zhaofeng] finish the chinese pdf manual}
% \changes{v2.5}{2018/05/20}{[\zhaofeng] make documentation more compact by including \texttt{dtx-style.sty} in the \texttt{dtx} file itself.}
% \changes{v2.5.1}{2020/02/23}{[\zhaofeng] remove dtx-style source code from manual}
% \StopEventually{\PrintChanges\PrintIndex}
% \clearpage
%
% \section{实现细节}
%
% \subsection{基本信息}
%    \begin{macrocode}
% \changes{v2.0}{2018/03/22}{[\zhaofeng] Make it a latex package(.sty)}
%<*sty>
\NeedsTeXFormat{LaTeX2e}[1999/12/01]
\ProvidesClass{iidef}
[2020/02/23 2.5.1 Tsinghua University Coursework Template]
%    \end{macrocode}
% \subsection{装载宏包}
% \label{sec:loadpackage}
%
% 引用的宏包和相应的定义。
%    \begin{macrocode}
%% configuration of nested enumerate env
\RequirePackage{enumitem}
%% set hwcount key-value option
\RequirePackage{kvoptions}
%% required by macro DeclareMathOperator
\RequirePackage{amsmath}
%% Set up page headers using with fancyhdr
\RequirePackage{fancyhdr}
%define the solution env with amsthm proof
\RequirePackage{amsthm}
%    \end{macrocode}
%
% 定义学期、学院、课程名,作业名
% \changes{v2.3}{2018/03/30}{[\zhaofeng] User configuration of \cs{hwname}}
%    \begin{macrocode}
%% semester
\def\@term{term}
\newcommand{\theterm}[1]{\renewcommand\@term{#1}}
%% institute
\newcommand{\@courseinstitute}[1]{institute}
\newcommand{\thecourseinstitute}[1]{\renewcommand\@courseinstitute{#1}}
%% coursename
\newcommand{\@coursename}[1]{coursename}
\newcommand{\thecoursename}[1]{\renewcommand\@coursename{\textsc{#1}}}
%% user can rewrite homework name
\def\@hwname{Homework}
\def\hwname#1{\renewcommand\@hwname{#1}}
%    \end{macrocode}
% \subsection{作业选项}
% \changes{v2.1}{2018/03/22}{[\zhaofeng] Use kvoptions to set the coursework counter.}
%    \begin{macrocode}
%% \iidef@thehwcnt = 1
\DeclareStringOption[1]{thehwcnt}
\ProcessKeyvalOptions*
\def\thehwcnt{\iidef@thehwcnt}
%    \end{macrocode}
% \subsection{标题}
% 定义 \cs{courseheader}
% \changes{v2.2}{2018/03/27}{[\zhaofeng] Use fancyhdr to set the coursework user file header.}
%    \begin{macrocode}
%% page header setup, distinguish between first page(plain style) 
%% and second page on (runningpage style)
%%***************************************************************************
\newcommand{\courseheader}{
\thispagestyle{plain}%first page use native plain style to suppress header
\vspace*{-1in}
\begin{center}
\@courseinstitute\\
\@coursename\\
\@term
\vspace*{0.1in}
\hrule
\end{center}
\begin{center}
  \underline{\bf \@hwname\;\thehwcnt} \\
\end{center}
}
\fancypagestyle{runningpage}
{
  \fancyhead[L]{\small\@coursename}
  \fancyhead[R]{\small\@courseinstitute}
}
%% use runningpage style from second page on 
\pagestyle{runningpage}
%% *********************************************************************************************
%    \end{macrocode}
%
% 定义 \cs{name}
%    \begin{macrocode}
%%name command macro
%%*************************
\newcommand{\name}[1]{
\begin{flushleft}
  #1\hfill
  \today
\end{flushleft}
\hrule

\vspace{2em}

\flushleft
}
%%*************************
%    \end{macrocode}
% \subsection{题目自动编号}
%    \begin{macrocode}
%% enumitem related configuration
\setlist[enumerate,1]{label=\thehwcnt.\arabic*.}
\setlist[enumerate,2]{label=(\alph*)}
\setlist[enumerate,3]{label=\roman*.}
\setlist[enumerate,4]{label=\greek*}
%    \end{macrocode}
% \subsection{解的环境}
% \changes{v2.3}{2018/03/30}{[\zhaofeng] Add \texttt{solution} environment}
% \changes{v2.3}{2018/03/30}{[\zhaofeng] user configuration of \texttt{solution} environment name}
% 包装 \pkg{amsthm} 的 \cs{proof},可以通过 \cs{slname} 对默认”解“的文字进行修改。 
%    \begin{macrocode}
%%******************************
\def\@slname{Solution}
\def\slname#1{\renewcommand\@slname{#1}}

\newenvironment{solution}
{
\proof[\@slname]
}
{
%% no qed symbol in solution env
\renewcommand{\qedsymbol}{}
\endproof
}
%%******************************
%    \end{macrocode}
% \subsection{常用数学符号}
%    \begin{macrocode}
%%common math symbols go here
%%*************************************************
\def\v#1{\underline{#1}}
\newcommand{\uc}{\underline{c}}    % c, vec
\newcommand{\uv}{\underline{v}}    % x, vec
\newcommand{\uw}{\underline{w}}    % w, vec
\newcommand{\ux}{\underline{x}}    % x, vec
\newcommand{\uy}{\underline{y}}    % y, vec
\newcommand{\uz}{\underline{z}}    % z, vec
\newcommand{\um}{\underline{m}}    % m, vec
\newcommand{\rvx}{\mathsf{x}}    % x, r.v.
\newcommand{\rvy}{\mathsf{y}}    % y, r.v.
\newcommand{\rvz}{\mathsf{z}}    % z, r.v.
\newcommand{\rvw}{\mathsf{w}}    % w, r.v.
\newcommand{\rvH}{\mathsf{H}}    % H, r.v.
\newcommand{\urvx}{\underline{\mathsf{x}}}    % x, r.v. vec
\newcommand{\urvy}{\underline{\mathsf{y}}}    % y, r.v. vec
\newcommand{\urvz}{\underline{\mathsf{z}}}    % z, r.v. vec
\newcommand{\urvw}{\underline{\mathsf{w}}}    % w, r.v. vec

\newcommand{\defas}{\triangleq} %\coloneqq
\newcommand{\reals}{\mathbb{R}}
\newcommand{\TT}{\mathrm{T}}    % transpose
\DeclareMathOperator*{\argmax}{arg\,max}
\DeclareMathOperator*{\argmin}{arg\,min}
\DeclareMathOperator*{\argsup}{arg\,sup}
\DeclareMathOperator*{\arginf}{arg\,inf}
\DeclareMathOperator{\diag}{diag}
\DeclareMathOperator{\Var}{Var}
\DeclareMathOperator{\Cov}{Cov}
\DeclareMathOperator{\MSE}{MSE}
\DeclareMathOperator{\1}{\mathds{1}}
\DeclareMathOperator{\In}{\mathbb{I}}
\DeclareMathOperator{\E}{\mathbb{E}}
\DeclareMathOperator{\Prob}{\mathbb{P}}
\newcommand\independent{\protect\mathpalette{\protect\independenT}{\perp}}
\def\independenT#1#2{\mathrel{\rlap{$#1#2$}\mkern2mu{#1#2}}}
%%************************************************************************************
%</sty>
%    \end{macrocode}
% \iffalse
%    \begin{macrocode}
%<*dtx-style>
\ProvidesPackage{dtx-style}
\RequirePackage{hypdoc}
\RequirePackage[UTF8,scheme=chinese]{ctex}
\RequirePackage{newpxtext}
\RequirePackage{newpxmath}
\RequirePackage[
  top=2.5cm, bottom=2.5cm,
  left=4cm, right=2cm,
  headsep=3mm]{geometry}
\RequirePackage{array,longtable,booktabs}
\RequirePackage{listings}
\RequirePackage{fancyhdr}
\RequirePackage{xcolor}
\RequirePackage{enumitem}
\RequirePackage{etoolbox}
\RequirePackage{metalogo}

\colorlet{thu@macro}{blue!60!black}
\colorlet{thu@env}{blue!70!black}
\colorlet{thu@option}{purple}
\patchcmd{\PrintMacroName}{\MacroFont}{\MacroFont\bfseries\color{thu@macro}}{}{}
\patchcmd{\PrintDescribeMacro}{\MacroFont}{\MacroFont\bfseries\color{thu@macro}}{}{}
\patchcmd{\PrintDescribeEnv}{\MacroFont}{\MacroFont\bfseries\color{thu@env}}{}{}
\patchcmd{\PrintEnvName}{\MacroFont}{\MacroFont\bfseries\color{thu@env}}{}{}

\def\DescribeOption{%
  \leavevmode\@bsphack\begingroup\MakePrivateLetters%
  \Describe@Option}
\def\Describe@Option#1{\endgroup
  \marginpar{\raggedleft\PrintDescribeOption{#1}}%
  \thu@special@index{option}{#1}\@esphack\ignorespaces}
\def\PrintDescribeOption#1{\strut \MacroFont\bfseries\sffamily\color{thu@option} #1\ }
\def\thu@special@index#1#2{\@bsphack
  \begingroup
    \HD@target
    \let\HDorg@encapchar\encapchar
    \edef\encapchar usage{%
      \HDorg@encapchar hdclindex{\the\c@HD@hypercount}{usage}%
    }%
    \index{#2\actualchar{\string\ttfamily\space#2}
           (#1)\encapchar usage}%
    \index{#1:\levelchar#2\actualchar
           {\string\ttfamily\space#2}\encapchar usage}%
  \endgroup
  \@esphack}

\lstdefinestyle{lstStyleBase}{%
   basicstyle=\small\ttfamily,
   aboveskip=\medskipamount,
   belowskip=\medskipamount,
   lineskip=0pt,
   boxpos=c,
   showlines=false,
   extendedchars=true,
   upquote=true,
   tabsize=2,
   showtabs=false,
   showspaces=false,
   showstringspaces=false,
   numbers=none,
   linewidth=\linewidth,
   xleftmargin=4pt,
   xrightmargin=0pt,
   resetmargins=false,
   breaklines=true,
   breakatwhitespace=false,
   breakindent=0pt,
   breakautoindent=true,
   columns=flexible,
   keepspaces=true,
   gobble=2,
   framesep=3pt,
   rulesep=1pt,
   framerule=1pt,
   backgroundcolor=\color{gray!5},
   stringstyle=\color{green!40!black!100},
   keywordstyle=\bfseries\color{blue!50!black},
   commentstyle=\slshape\color{black!60}}

\lstdefinestyle{lstStyleShell}{%
   style=lstStyleBase,
   frame=l,
   rulecolor=\color{purple},
   language=bash}

\lstdefinestyle{lstStyleLaTeX}{%
   style=lstStyleBase,
   frame=l,
   rulecolor=\color{violet},
   language=[LaTeX]TeX}

\lstnewenvironment{latex}{\lstset{style=lstStyleLaTeX}}{}
\lstnewenvironment{shell}{\lstset{style=lstStyleShell}}{}

\setlist{nosep}

\DeclareDocumentCommand{\option}{m}{\textsf{#1}}
\DeclareDocumentCommand{\env}{m}{\texttt{#1}}
\DeclareDocumentCommand{\pkg}{s m}{%
  \texttt{#2}\IfBooleanF#1{\thu@special@index{package}{#2}}}
\DeclareDocumentCommand{\file}{s m}{%
  \texttt{#2}\IfBooleanF#1{\thu@special@index{file}{#2}}}
\newcommand{\myentry}[1]{%
  \marginpar{\raggedleft\color{purple}\bfseries\strut #1}}
\newcommand{\note}[2][Note]{{%
  \color{magenta}{\bfseries #1}\emph{#2}}}

\def\thucoursework{\textsc{Thu}\-\textsc{Coursework}}
\endinput
%</dtx-style>
%    \end{macrocode}
% \fi
% \Finale
%
\endinput
