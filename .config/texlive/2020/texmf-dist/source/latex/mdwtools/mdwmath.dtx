% \begin{meta-comment}
%
% $Id: mdwmath.dtx,v 1.1 1996/11/19 20:53:21 mdw Exp $
%
% Various nicer mathematical things
%
% (c) 1996 Mark Wooding
%
%----- Revision history -----------------------------------------------------
%
% $Log: mdwmath.dtx,v $
% Revision 1.1  1996/11/19 20:53:21  mdw
% Initial revision
%
%
% \end{meta-comment}
%
% \begin{meta-comment} <general public licence>
%%
%% mdwmath package -- various nicer mathematical things
%% Copyright (c) 1996 Mark Wooding
%%
%% This program is free software; you can redistribute it and/or modify
%% it under the terms of the GNU General Public License as published by
%% the Free Software Foundation; either version 2 of the License, or
%% (at your option) any later version.
%%
%% This program is distributed in the hope that it will be useful,
%% but WITHOUT ANY WARRANTY; without even the implied warranty of
%% MERCHANTABILITY or FITNESS FOR A PARTICULAR PURPOSE.  See the
%% GNU General Public License for more details.
%%
%% You should have received a copy of the GNU General Public License
%% along with this program; if not, write to the Free Software
%% Foundation, Inc., 675 Mass Ave, Cambridge, MA 02139, USA.
%%
% \end{meta-comment}
%
% \begin{meta-comment} <Package preamble>
%<+package>\NeedsTeXFormat{LaTeX2e}
%<+package>\ProvidesPackage{mdwmath}
%<+package>                [1996/04/11 1.1 Nice mathematical things]
%<+oldeqnarray>\NeedsTeXFormat{LaTeX2e}
%<+oldeqnarray>\ProvidesPackage{eqnarray}
%<+oldeqnarray>                [1996/04/11 1.1 Old enhanced eqnarray]
% \end{meta-comment}
%
% \CheckSum{259}
% \begin{old-eqnarray}
% \CheckSum{484}
% \end{old-eqnarray}
%% \CharacterTable
%%  {Upper-case    \A\B\C\D\E\F\G\H\I\J\K\L\M\N\O\P\Q\R\S\T\U\V\W\X\Y\Z
%%   Lower-case    \a\b\c\d\e\f\g\h\i\j\k\l\m\n\o\p\q\r\s\t\u\v\w\x\y\z
%%   Digits        \0\1\2\3\4\5\6\7\8\9
%%   Exclamation   \!     Double quote  \"     Hash (number) \#
%%   Dollar        \$     Percent       \%     Ampersand     \&
%%   Acute accent  \'     Left paren    \(     Right paren   \)
%%   Asterisk      \*     Plus          \+     Comma         \,
%%   Minus         \-     Point         \.     Solidus       \/
%%   Colon         \:     Semicolon     \;     Less than     \<
%%   Equals        \=     Greater than  \>     Question mark \?
%%   Commercial at \@     Left bracket  \[     Backslash     \\
%%   Right bracket \]     Circumflex    \^     Underscore    \_
%%   Grave accent  \`     Left brace    \{     Vertical bar  \|
%%   Right brace   \}     Tilde         \~}
%%
%
% \begin{meta-comment}
%
%<*driver>
%
% $Id: mdwtools.ins,v 1.3 1996/11/19 20:57:26 mdw Exp $
%
% Installer for the mdwtools packages
%
% (c) 1996 Mark Wooding
%

%----- Revision history -----------------------------------------------------
%
% $Log: mdwtools.ins,v $
% Revision 1.3  1996/11/19 20:57:26  mdw
% Entered into RCS
%

% --- Licence note ---
%
% mdwtools installer
% Copyright (c) 1996 Mark Wooding
%
% This program is free software; you can redistribute it and/or modify
% it under the terms of the GNU General Public License as published by
% the Free Software Foundation; either version 2 of the License, or
% (at your option) any later version.
%
% This program is distributed in the hope that it will be useful,
% but WITHOUT ANY WARRANTY; without even the implied warranty of
% MERCHANTABILITY or FITNESS FOR A PARTICULAR PURPOSE.  See the
% GNU General Public License for more details.
%
% You should have received a copy of the GNU General Public License
% along with this program; if not, write to the Free Software
% Foundation, Inc., 675 Mass Ave, Cambridge, MA 02139, USA.

% --- Sort out how to do all this ---

\def\batchfile{mdwtools.ins}
\input docstrip
\keepsilent

\preamble

IMPORTANT NOTICE
\endpreamble

{ % --- This is a group so that docstrip \reads it all in one go ---

  \ifx\generate\mdwxxnotdef
    \gdef\mdwgen#1{#1}
    \gdef\mdwf#1#2{\generateFile{#1}{n}{#2}}
    \gdef\needed#1{}
  \else
    \global\let\mdwgen\generate
    \global\def\mdwf{\file}
    \global\askforoverwritefalse
  \fi

}

\mdwgen{\mdwf {at.sty}		{\from {at.dtx}	      {package}}
	\mdwf {mdwlist.sty}	{\from {mdwlist.dtx}  {package}}
	\mdwf {mdwtab.sty}	{\from {mdwtab.dtx}   {mdwtab}
				 \needed{syntax.dtx}
				 \from {footnote.dtx} {macro}
				 \from {doafter.dtx}  {macro}}
	\mdwf {syntax.sty}	{\from {syntax.dtx}   {package}
				 \from {doafter.dtx}  {macro}}
	\mdwf {mathenv.sty}	{\from {mdwtab.dtx}   {mathenv}}
	\mdwf {mdwmath.sty}	{\from {mdwmath.dtx}  {package}}
	\mdwf {sverb.sty}	{\from {sverb.dtx}    {package}}
	\mdwf {footnote.sty}	{\from {footnote.dtx} {package}}
	\mdwf {doafter.sty}	{\from {doafter.dtx}  {package,latex2e}}
	\mdwf {doafter.tex}	{\from {doafter.dtx}  {package,plain}}
	\mdwf {cmtt.sty}	{\from {cmtt.dtx}     {sty}}
	\mdwf {mTTenc.def}	{\from {cmtt.dtx}     {def}}
	\mdwf {mTTcmtt.fd}	{\from {cmtt.dtx}     {fd}}
}

\Msg{Done!}

\describespackage{mdwmath}
% \describespackage{eqnarray}
\ignoreenv{old-eqnarray}
% \unignoreenv{old-eqnarray}
\mdwdoc
%</driver>
%
% \end{meta-comment}
%
% \section{User guide}
%
% \subsection{Square root typesetting}
%
% \DescribeMacro{\sqrt}
% The package supplies a star variant of the |\sqrt| command which omits the
% vinculum over the operand (the line over the top).  While this is most
% useful in simple cases like $\sqrt*{2}$ it works for any size of operand.
% The package also re-implements the standard square root command so that it
% positions the root number rather better.
%
% \begin{figure}
% \begin{demo}[w]{Examples of the new square root command}
%\[ \sqrt*{2} \quad \mbox{rather than} \quad \sqrt{2} \]
%\[ \sqrt*[3]{2} \quad \mbox{ rather than } \quad \sqrt[3]{2} \]
%\[ \sqrt{x^3 + \sqrt*[y]{\alpha}} - \sqrt*[n+1]{a} \]
%\[ x = \sqrt*[3]{\frac{3y}{7}} \]
%\[ q = \frac{2\sqrt*{2}}{5}+\sqrt[\frac{n+1}{2}]{2x^2+3xy-y^2} \]
% \end{demo}
% \end{figure}
%
% [Note that omission of the vinculum was originally a cost-cutting exercise
% because the radical symbol can just fit in next to its operand and
% everything ends up being laid out along a line.  However, I find that the
% square root without vinculum is less cluttered, so I tend to use it when
% it doesn't cause ambiguity.]
%
% \subsection{Some maths symbols you already have}
%
% Having just tried to do some simple things, I've found that there are maths
% symbols missing.  Here they are, in all their glory:
%
% \begin{center} \unverb\| \begin{tabular}{cl|cl|cl}
% $\&$ & "\&"		& $\bitor$ & "\bitor"	& $\dbland$ & "\dbland" \\
% $\bitand$ & "\bitand"	& $\dblor$ & "\dblor"	& 
% \end{tabular} \end{center}
%
% \begin{ignore}
% There used to be an eqnarray here, but that's migrated its way into the
% \package{mdwtab} package.  Maybe the original version, without dependency
% on \package{mdwtab} ought to be releasable separately.  I'll keep it around
% just in case.
%
% The following is the documentation for the original version.  There's an
% updated edition in \package{mdwtab}.
% \end{ignore}
%
% \begin{old-eqnarray}
%
% \subsection{A new \env{eqnarray} environment}
%
% \LaTeX's built-in \env{eqnarray} is horrible -- it puts far too much space
% between the items in the array.  This environment is rather nearer to the
% \env{amsmath} \env{align} environments, although rather less capable.
%
% \bigskip
% \DescribeEnv{eqnarray}
% {\synshorts
% \setbox0\hbox{"\\begin{eqnarray}["<preamble>"]" \dots "\\end{eqnarray}"}
% \leavevmode \hskip-\parindent \fbox{\box0}
% }
% \smallskip
%
% The new version of \env{eqnarray} tries to do everything which you really
% want it to.  The \synt{preamble} string allows you to define the column
% types in a vaguely similar way to the wonderful \env{tabular} environment.
% The types provided (and it's easy-ish to add more) are:
%
% \def\ch{\char`}
% \begin{description} \def\makelabel{\hskip\labelsep\normalfont\ttfamily}
% \item [r] Right aligned equation
% \item [c] Centre-aligned equation
% \item [l] Left aligned equation
% \item [\textrm{\texttt{Tr}, \texttt{Tc} and \texttt{Tl}}] Right, centre and
%       left aligned text (not maths)
% \item [L] Left aligned zero-width equation
% \item [x] Centred entire equation
% \item [:] Big gap separating sets of equations
% \item [q] Quad space
% \item [>\ch\{\synt{text}\ch\}] Insert text before column
% \item [<\ch\{\synt{text}\ch\}] Insert text after column
% \end{description}
%
% Some others are also defined: don't use them because they do complicated
% things which are hard to explain and they aren't much use anyway.
%
% The default preamble, if you don't supply one of your own, is \lit{rcl}.
% Most of the time, \lit{rl} is sufficient, although compatibility is more
% important to me.
%
% By default, there is no space between columns, which makes formul\ae\ in an
% \env{eqnarray} environment look just like formul\ae\ typeset on their own,
% except that things get aligned in columns.  This is where the default
% \env{eqnarray} falls down: it leaves |\arraycolsep| space between each
% column making the thing look horrible.
%
% An example would be good here, I think.  This one's from exercise 22.9 of
% the \textit{\TeX book}.
%
% \begin{demo}[w]{Simultaneous equations}
%\begin{eqnarray}[rcrcrcrl]
%  10w & + &  3x & + & 3y & + & 18z & = 1 \\
%   6w & - & 17x &   &    & - &  5z & = 2
%\end{eqnarray}
% \end{demo}
%
% Choosing a more up-to-date example, here's one demonstrating the \lit{:}
% column specifier from the \textit{\LaTeX\ Companion}.
%
% \begin{demo}[w]{Lots of equations}
%\begin{eqnarray}[rl:rl:l]
% V_i &= v_i - q_i v_j,	& X_i &= x_i - q_i x_j,	&
%       U_i = u_i, \qquad \mbox{for $i \ne j$}  \label{eq:A} \\
% V_j &= v_j,           & X_j &= x_j            &
%       U_j u_j + \sum_{i \ne j} q_i u_i.
%\end{eqnarray}
% \end{demo}
%
% We can make things more interesting by adding a plain text column.  Here we
% go:
%
% \begin{demo}[w]{Plain text column}
%\begin{eqnarray}[rlqqTl]
%     x  &= y           & by (\ref{eq:A}) \\
%     x' &= y'          & by definition \\
% x + x' &= y + y'      & by Axiom~1
%\end{eqnarray}
% \end{demo}
%
% The new features also mean that you don't need to mess about with
% |\lefteqn| any more.  This is handled by the \lit{L} column type:
%
% \begin{demo}{Splitting example}
%\begin{eqnarray*}[Ll]
%   w+x+y+z = \\
%    & a+b+c+d+e+ \\
%    & f+g+h+i+j
%\end{eqnarray*}
% \end{demo}
%
% Finally, just to prove that the spacing's right at last, here's another one
% from the \textit{Companion}.
%
% \begin{demo}{Spacing demonstration}
%\begin{equation}
%  x^2 + y^2 = z^2
%\end{equation}
%\begin{eqnarray}[rl]
%  x^2 + y^2 &= z^2 \\
%        y^2 &< z^2
%\end{eqnarray}
% \end{demo}
%
% Well, that was easy enough.  Now on to numbering.  As you've noticed, the
% equations above are numbered.  You can use the \env{eqnarray$*$}
% environment to turn off the numbering in the whole environment, or say
% |\nonumber| on a line to suppress numbering of that one in particular.
% More excitingly, you can say \syntax{"\\nonumber["<text>"]"} to choose
% what text to display.
%
% A note for cheats: you can use the sparkly new \env{eqnarray} for simple
% equations simply by specifying \lit{x} as the column description.  Who
% needs \AmSTeX? |;-)|
%
% \end{old-eqnarray}
%
% \implementation
%
% \section{Implementation}
%
% This isn't really complicated (honest) although it is a lot hairier than I
% think it ought to be.
%
%    \begin{macrocode}
%<*package>
%    \end{macrocode}
%
% \subsection{Square roots}
%
% \subsubsection{Where is the square root sign?}
%
% \LaTeX\ hides the square root sign away somewhere without telling anyone
% where it is.  I extract it forcibly by peeking inside the |\sqrtsign| macro
% and scrutinising the contents.  Here we go: prepare for yukkiness.
%
%    \begin{macrocode}
\newcount\sq@sqrt
\begingroup
  \catcode`\|0 \catcode`\\12
  |def|sq@readrad#1"#2\#3|relax{|global|sq@sqrt"#2|relax}
  |expandafter|sq@readrad|meaning|sqrtsign|relax
|endgroup
\def\sq@delim{\delimiter\sq@sqrt\relax}
%    \end{macrocode}
%
% \subsubsection{Drawing fake square root signs}
%
% \TeX\ absolutely insists on drawing square root signs with a vinculum over
% the top.  In order to get the same effect, we have to attempt to emulate
% \TeX's behaviour.
%
% \begin{macro}{\sqrtdel}
%
% This does the main job of typesetting a vinculum-free radical.\footnote{^^A
%   Note for chemists: this is nothing to do with short-lived things which
%   don't have their normal numbers of electrons.  And it won't reduce the
%   appearance of wrinkles either.}
% It's more or less a duplicate of what \TeX\ does internally, so it might be
% a good plan to have a copy of Appendix~G open while you examine this.
%
% We start off by using |\mathpalette| to help decide how big things should
% be.
%
%    \begin{macrocode}
\def\sqrtdel{\mathpalette\sqrtdel@i}
%    \end{macrocode}
%
% Read the contents of the radical into a box, so we can measure it.
%
%    \begin{macrocode}
\def\sqrtdel@i#1#2{%
  \setbox\z@\hbox{$\m@th#1#2$}% %%% Bzzzt -- uncramps the mathstyle
%    \end{macrocode}
%
% Now try and sort out the values needed in this calculation.  We'll assume
% that $\xi_8$ is 0.6\,pt, the way it usually is.  Next try to work out the
% value of $\varphi$.
%
%    \begin{macrocode}
  \ifx#1\displaystyle%
    \@tempdima1ex%
  \else%
    \@tempdima.6\p@%
  \fi%
%    \end{macrocode}
%
% That was easy.  Now for $\psi$.
%
%    \begin{macrocode}
  \@tempdimb.6\p@%
  \advance\@tempdimb.25\@tempdima%
%    \end{macrocode}
%
% Build the `delimiter' in a box of height $h(x)+d(x)+\psi+\xi_8$, as
% requested.  Box~2 will do well for this purpose.
%
%    \begin{macrocode}
  \dimen@.6\p@%
  \advance\dimen@\@tempdimb%
  \advance\dimen@\ht\z@%
  \advance\dimen@\dp\z@%
  \setbox\tw@\hbox{%
    $\left\sq@delim\vcenter to\dimen@{}\right.\n@space$%
  }%
%    \end{macrocode}
%
% Now we need to do some more calculating (don't you hate it?).  As far as
% Appendix~G is concerned, $\theta=h(y)=0$, because we want no rule over the
% top.  
%
%    \begin{macrocode}
  \@tempdima\ht\tw@%
  \advance\@tempdima\dp\tw@%
  \advance\@tempdima-\ht\z@%
  \advance\@tempdima-\dp\z@%
  \ifdim\@tempdima>\@tempdimb%
    \advance\@tempdima\@tempdimb%
    \@tempdimb.5\@tempdima%
  \fi%
%    \end{macrocode}
%
% Work out how high to raise the radical symbol.  Remember that Appendix~G
% thinks that the box has a very small height, although this is untrue here.
%
%    \begin{macrocode}
  \@tempdima\ht\z@%
  \advance\@tempdima\@tempdimb%
  \advance\@tempdima-\ht\tw@%
%    \end{macrocode}
%
% Build the output (finally).  The brace group is there to turn the output
% into a mathord, one of the few times that this is actually desirable.
%
%    \begin{macrocode}
  {\raise\@tempdima\box\tw@\vbox{\kern\@tempdimb\box\z@}}%
}
%    \end{macrocode}
%
% \end{macro}
%
% \subsubsection{The new square root command}
%
% This is where we reimplement all the square root stuff.  Most of this stuff
% comes from the \PlainTeX\ macros, although some is influenced by \AmSTeX\
% and \LaTeXe, and some is original.  I've tried to make the spacing vaguely
% automatic, so although it's not configurable like \AmSTeX's version, the
% output should look nice more of the time.  Maybe.
%
% \begin{macro}{\sqrt}
%
% \LaTeX\ says this must be robust, so we make it robust.  The first thing to
% do is to see if there's a star and pass the appropriate squareroot-drawing
% command on to the rest of the code.
%
%    \begin{macrocode}
\DeclareRobustCommand\sqrt{\@ifstar{\sqrt@i\sqrtdel}{\sqrt@i\sqrtsign}}
%    \end{macrocode}
%
% Now we can sort out an optional argument to be displayed on the root.
%
%    \begin{macrocode}
\def\sqrt@i#1{\@ifnextchar[{\sqrt@ii{#1}}{\sqrt@iv{#1}}}
%    \end{macrocode}
%
% Stages~2 and~3 below are essentially equivalents of \PlainTeX's
% |\root|\dots|\of| and |\r@@t|.  Here we also find the first wrinkle: the
% |\rootbox| used to store the number is spaced out on the left if necessary.
% There's a backspace after the end so that the root can slip underneath, and
% everything works out nicely.  Unfortunately size is fixed here, although
% doesn't actually seem to matter.
%
%    \begin{macrocode}
\def\sqrt@ii#1[#2]{%
  \setbox\rootbox\hbox{$\m@th\scriptscriptstyle{#2}$}%
  \ifdim\wd\rootbox<6\p@%
    \setbox\rootbox\hb@xt@6\p@{\hfil\unhbox\rootbox}%
  \fi%
  \mathpalette{\sqrt@iii{#1}}%
}
%    \end{macrocode}
%
% Now we can actually build everything.  Note that the root is raised by its
% depth -- this prevents a common problem with letters with descenders.
%
%    \begin{macrocode}
\def\sqrt@iii#1#2#3{%
  \setbox\z@\hbox{$\m@th#2#1{#3}$}%
  \dimen@\ht\z@%
  \advance\dimen@-\dp\z@%
  \dimen@.6\dimen@%
  \advance\dimen@\dp\rootbox%
  \mkern-3mu%
  \raise\dimen@\copy\rootbox%
  \mkern-10mu%
  \box\z@%
}
%    \end{macrocode}
%
% Finally handle a non-numbered root.  We read the rooted text in as an
% argument, to stop problems when people omit the braces.  (\AmSTeX\ does
% this too.)
%
%    \begin{macrocode}
\def\sqrt@iv#1#2{#1{#2}}
%    \end{macrocode}
%
% \end{macro}
%
% \begin{macro}{\root}
%
% We also re-implement \PlainTeX's |\root| command, just in case someone uses
% it, and supply a star-variant.  This is all very trivial.
%
%    \begin{macrocode}
\def\root{\@ifstar{\root@i\sqrtdel}{\root@i\sqrtsign}}
\def\root@i#1#2\of{\sqrt@ii{#1}[#2]}
%    \end{macrocode}
%
% \end{macro}
%
% \subsection{Some magic new maths characters}
%
% This is all really easy.
%
%    \begin{macrocode}
\DeclareMathSymbol{&}{\mathbin}{operators}{`\&}
\DeclareMathSymbol{\bitand}{\mathbin}{operators}{`\&}
\def\bitor{\mathbin\mid}
\def\dblor{\mathbin{\mid\mid}}
\def\dbland{\mathbin{\mathrel\bitand\mathrel\bitand}}
%    \end{macrocode}
%
% \subsection{Biggles}
%
% Now for some user-controlled delimiter sizing.  The standard bigness of
% plain \TeX's delimiters are all right, but it's a little limiting.
%
% The biggness of delimiters is based on the size of the current |\strut|,
% which \LaTeX\ keeps up to date all the time.  This will make the various
% delimiters grow in proportion when the text gets bigger.  Actually, I'm
% not sure that this is exactly right -- maybe it should be nonlinear,
%
% \begin{macro}{\bbigg}
% \begin{macro}{\bbiggl}
% \begin{macro}{\bbiggr}
% \begin{macro}{\bbiggm}
%
% This is where the bigness is done.  This is more similar to the plain \TeX\
% big delimiter stuff than to the \package{amsmath} stuff, although there's
% not really a lot of difference.
%
% The two arguments are a multiplier for the delimiter size, and a small
% increment applied \emph{before} the multiplication (which is optional).
%
% This is actually a front for a low-level interface which can be called
% directly for efficiency.
%
%    \begin{macrocode}
\def\bbigg{\@bbigg\mathord}
\def\bbiggl{\@bbigg\mathopen}
\def\bbiggr{\@bbigg\mathclose}
\def\bbiggm{\@bbigg\mathrel}
%    \end{macrocode}
%
% \end{macro}
% \end{macro}
% \end{macro}
% \end{macro}
%
% \begin{macro}{\@bbigg}
%
% This is an optional argument parser providing a front end for the main
% macro |\bbigg@|.
%
%    \begin{macrocode}
\def\@bbigg#1{\@ifnextchar[{\@bigg@i{#1}}{\@bigg@i{#1}[\z@]}}
\def\@bigg@i#1[#2]#3#4{#1{\bbigg@{#2}{#3}{#4}}}
%    \end{macrocode}
%
% \end{macro}
%
% \begin{macro}{\bbigg@}
%
% This is it, at last.  The arguments are as described above: an addition
% to be made to the strut height, and a multiplier.  Oh, and the delimiter,
% of course.
%
% This is a bit messy.  The smallest `big' delimiter, |\big|, is the same
% height as the current strut box.  Other delimiters are~$1\frac12$, $2$
% and~$2\frac12$ times this height.  I'll set the height of the delimiter by
% putting in a |\vcenter| of the appropriate size.
%
% Given an extra height~$x$, a multiplication factor~$f$ and a strut
% height~$h$ and depth~$d$, I'll create a vcenter with total height
% $f(h+d+x)$.  Easy, isn't it?
%
%    \begin{macrocode}
\def\bbigg@#1#2#3{%
  \hbox{$%
    \dimen@\ht\strutbox\advance\dimen@\dp\strutbox%
    \advance\dimen@#1%
    \dimen@#2\dimen@%
    \left#3\vcenter to\dimen@{}\right.\n@space%
  $}%
}
%    \end{macrocode}
%
% \end{macro}
%
% \begin{macro}{\big}
% \begin{macro}{\Big}
% \begin{macro}{\bigg}
% \begin{macro}{\Bigg}
%
% Now for the easy macros.
%
%    \begin{macrocode}
\def\big{\bbigg@\z@\@ne}
\def\Big{\bbigg@\z@{1.5}}
\def\bigg{\bbigg@\z@\tw@}
\def\Bigg{\bbigg@\z@{2.5}}
%    \end{macrocode}
%
% \end{macro}
% \end{macro}
% \end{macro}
% \end{macro}
% 
%
% \begin{ignore}
% The following is the original definition of the enhanced eqnarray
% environment.  It's not supported, although if you can figure out how to
% extract it, it's all yours.
% \end{ignore}
%
% \begin{old-eqnarray}
%
% \subsection{The sparkly new \env{eqnarray}}
%
% Start off by writing a different package.
%
%    \begin{macrocode}
%</package>
%<*oldeqnarray>
%    \end{macrocode}
%
% \subsubsection{Options handling}
%
% We need to be able to cope with \textsf{fleqn} and \textsf{leqno} options.
% This will adjust our magic modified \env{eqnarray} environment
% appropriately.
%
%    \begin{macrocode}
\newif\if@fleqn
\newif\if@leqno
\DeclareOption{fleqn}{\@fleqntrue}
\DeclareOption{leqno}{\@leqnotrue}
\ProcessOptions
%    \end{macrocode}
%
% This is all really different to the \LaTeX\ version.  I've looked at the
% various \env{tabular} implementations, the original \env{eqnarray} and the
% \textit{\TeX book} to see how best to do this, and then went my own way.
% If it doesn't work it's all my fault.
%
% \subsubsection{Some useful registers}
%
% The old \LaTeX\ version puts the equation numbers in by keeping a count of
% where it is in the alignment.  Since I don't know how may columns there are
% going to be, I'll just use a switch in the preamble to tell me to stop
% tabbing.
%
%    \begin{macrocode}
\newif\if@eqalast
%    \end{macrocode}
%
% Now define some useful length parameters.  First allocate them:
%
%    \begin{macrocode}
\newskip\eqaopenskip
\newskip\eqacloseskip
\newskip\eqacolskip
\newskip\eqainskip
%    \end{macrocode}
%
% Now assign some default values.  Users can play with these if they really
% want although I can't see the point myself.
%
%    \begin{macrocode}
\if@fleqn
  \AtBeginDocument{\eqaopenskip\leftmargini}
\else
  \eqaopenskip\@centering
\fi
\eqacloseskip\@centering
\eqacolskip\@centering
\eqainskip\z@
%    \end{macrocode}
%
% We allow the user to play with the style if this is really wanted.  I dunno
% why, really.  Maybe someone wants very small alignments.
%
%    \begin{macrocode}
\let\eqa@style\displaystyle
%    \end{macrocode}
%
% \subsubsection{The main environments}
%
% We define the toplevel commands here.  They just add in default arguments
% and then call |\@eqnarray| with a preamble string.  The only difference is
% the last column they add in -- \env{eqnarray$*$} throws away the last
% column by sticking it in box~0.  (I used to |\@gobble| it but that caused
% the |\cr| to be lost.)
%
%    \begin{macrocode}
\def\eqnarray{\@ifnextchar[\eqnarray@i{\eqnarray@i[rcl]}}
\def\eqnarray@i[#1]{%
  \@eqnarray{#1!{\hb@xt@\z@{\hss##}\tabskip\z@}}
}
\@namedef{eqnarray*}{\@ifnextchar[\eqnarray@s@i{\eqnarray@s@i[rcl]}}
\def\eqnarray@s@i[#1]{%
  \@eqnarray{#1!{\nonumber\setbox\z@\hbox{##}\tabskip\z@}}%
}
%    \end{macrocode}
%
% \subsubsection{Set up the initial display}
%
% \begin{macro}{\@eqnarray}
%
% The |\@eqnarray| command does most of the initial work.  It sets up some
% flags and things, builds the |\halign| preamble, and returns.
%
%    \begin{macrocode}
\def\@eqnarray#1{%
%    \end{macrocode}
%
% Start playing with the counter here.  The original does some icky internal
% playing, which isn't necessary.  The |\if@eqnsw| switch is |true| if the
% user hasn't supplied an equation number.  The |\if@eqalast| switch is
% |true| in the final equation-number column.
%
%    \begin{macrocode}
  \refstepcounter{equation}%
  \@eqalastfalse%
  \global\@eqnswtrue%
  \m@th%
%    \end{macrocode}
%
% Set things up for the |\halign| which is coming up.
%
%    \begin{macrocode}
  \openup\jot%
  \tabskip\eqaopenskip%
  \let\\\@eqncr%
  \everycr{}%
  $$%
%    \end{macrocode}
%
% We'll build the real |\halign| and preamble in a token register.  All we
% need to do is stuff the header in the token register, clear a switch
% (that'll be explained later), parse the preamble and then expand the
% tokens we collected.  Easy, no?
%
%    \begin{macrocode}
  \toks@{\halign to\displaywidth\bgroup}%
  \@tempswafalse%
  \eqa@preamble#1\end%
  \the\toks@\cr%
}
%    \end{macrocode}
%
% \end{macro}
%
% \subsubsection{Parsing the preamble}
%
% All this actually involves is reading the next character and building a
% command from it.  That can pull off an argument if it needs it.  Just make
% sure we don't fall off the end and we'll be OK.
%
%    \begin{macrocode}
\def\eqa@preamble#1{%
  \ifx\end#1\else\csname eqa@char@#1\expandafter\endcsname\fi%
}
%    \end{macrocode}
%
% Adding stuff to the preamble tokens is a simple matter of using
% |\expandafter| in the correct way.\footnote{^^A
%   I have no idea why \LaTeX\ uses \cmd\edef\ for building its preamble.  It
%   seems utterly insane to me -- the amount of bodgery that \env{tabular}
%   has to go through to make everything expand at the appropriate times is
%   scary.  Maybe Messrs~Lamport and Mittelbach just forgot about token
%   registers when they were writing the code.  Maybe I ought to rewrite the
%   thing properly some time.  Sigh.
%
%   As a sort of postscript to the above, I \emph{have} rewritten the
%   \env{tabular} environment, and made a damned fine job of it, in my
%   oh-so-humble opinion.  All this \env{eqnarray} stuff has been remoulded
%   in terms of the generic column-defining things in \package{mdwtab}.
%   You're reading the documentation of the old version, which isn't
%   supported any more, so any bugs here are your own problem.}
%
%    \begin{macrocode}
\def\eqa@addraw#1{\expandafter\toks@\expandafter{\the\toks@#1}}
%    \end{macrocode}
%
% Now for some cleverness again.  In order to put all the right bits of
% |\tabskip| glue in the right places we must \emph{not} terminate each
% column until we know what the next one is.  We set |\if@tempswa| to be
% |true| if there's a column waiting to be closed (so it's initially
% |false|).  The following macro adds a column correctly, assuming we're in
% a formula.  Other column types make their own arrangements.
%
%    \begin{macrocode}
\def\eqa@add#1{%
  \if@tempswa%
    \eqa@addraw{\tabskip\eqainskip&#1}%
  \else%
    \eqa@addraw{#1}%
  \fi%
  \@tempswatrue%
}
%    \end{macrocode}
%
% Now to defining column types.  Let's define a macro which allows us to
% define column types:
%
%    \begin{macrocode}
\def\eqa@def#1{\expandafter\def\csname eqa@char@#1\endcsname}
%    \end{macrocode}
%
% Now we can define the column types.  Each column type must loop back to
% |\eqa@preamble| once it's finished, to read the rest of the preamble
% string.  Note the positioning of ord atoms in the stuff below.  This will
% space out relations and binops correctly when they occur at the edges of
% columns, and won't affect ord atoms at the edges, because ords pack
% closely.
%
% First the easy onces.  Just stick |\hfil| in the right places and
% everything will be all right.
%
%    \begin{macrocode}
\eqa@def r{\eqa@add{\hfil$\eqa@style##{}$}\eqa@preamble}
\eqa@def c{\eqa@add{\hfil$\eqa@style{}##{}$\hfil}\eqa@preamble}
\eqa@def l{\eqa@add{$\eqa@style{}##$\hfil}\eqa@preamble}
\eqa@def x{\eqa@add{\hfil$\eqa@style##$\hfil}\eqa@preamble}
%    \end{macrocode}
%
% Now for the textual ones.  This is also fairly easy.
%
%    \begin{macrocode}
\eqa@def T#1{%
  \eqa@add{}%
  \if#1l\else\eqa@addraw{\hfil}\fi%
  \eqa@addraw{##}%
  \if#1r\else\eqa@addraw{\hfil}\fi%
  \eqa@preamble%
}
%    \end{macrocode}
%
% Sort of split types of equations.  I mustn't use |\rlap| here, or
% everything goes wrong -- |\\| doesn't get noticed by \TeX\ in the same way
% as |\cr| does.
%
%    \begin{macrocode}
\eqa@def L{\eqa@add{\hb@xt@\z@{$\eqa@style##$\hss}\qquad}\eqa@preamble}
%    \end{macrocode}
%
% The \lit{:} column type is fairly simple.  We set |\tabskip| up to make
% lots of space and close the current column, because there must be one.^^A
% \footnote{This is an assumption.}
%
%    \begin{macrocode}
\eqa@def :{%
  \eqa@addraw{\tabskip\eqacolskip&}\@tempswafalse\eqa@preamble%
}
\eqa@def q{\eqa@add{\quad}\@tempswafalse\eqa@preamble}
%    \end{macrocode}
%
% The other column types just insert given text in an appropriate way.
%
%    \begin{macrocode}
\eqa@def >#1{\eqa@add{#1}\@tempswafalse\eqa@preamble}
\eqa@def <#1{\eqa@addraw{#1}\eqa@preamble}
%    \end{macrocode}
%
% Finally, the magical \lit{!} column type, which sets the equation number.
% We set up the |\tabskip| glue properly, tab on, and set the flag which
% marks the final column.
%
%    \begin{macrocode}
\eqa@def !#1{%
  \eqa@addraw{\tabskip\eqacloseskip&\@eqalasttrue#1}\eqa@preamble%
}
%    \end{macrocode}
%
% \subsubsection{Newline codes}
%
% Newline sequences (|\\|) get turned into calls of |\@eqncr|.  The job is
% fairly simple, really.  However, to avoid reading `|&|' characters
% prematurely, we set up a magic brace (from the \package{array} package --
% this avoids creating ord atoms and other nastyness).
%
%    \begin{macrocode}
\def\@eqncr{%
  \iffalse{\fi\ifnum0=`}\fi%
  \@ifstar{\eqacr@i{\@M}}{\eqacr@i{\interdisplaylinepenalty}}%
}
\def\eqacr@i#1{\@ifnextchar[{\eqacr@ii{#1}}{\eqacr@ii{#1}[\z@]}}
\def\eqacr@ii#1[#2]{%
  \ifnum0=`{}\fi%
  \eqa@eqnum%
  \noalign{\penalty#1\vskip#2\relax}%
}
%    \end{macrocode}
%
% \subsubsection{Setting equation numbers}
%
% Before we start, we need to generalise the flush-left number handling bits.
% The macro |\eqa@eqpos| will put its argument in the right place.
%
%    \begin{macrocode}
\if@leqno
  \def\eqa@eqpos#1{%
    \hb@xt@.01\p@{}\rlap{\normalfont\normalcolor\hskip-\displaywidth#1}%
  }
\else
  \def\eqa@eqpos#1{\normalfont\normalcolor#1}
\fi
%    \end{macrocode}
%
% First we need to move into the right column.  Then we just set the equation
% number appropriately.  There is some subtlety here, ish.  The |\relax| is
% important, to delay expansion of the |\if|\dots\ until the new column has
% been started.  The two helper macros are important too, to hide `|&|'s and
% `|\cr|'s from \TeX's scanner until the right time.
%
%    \begin{macrocode}
\def\eqa@eqnum{%
  \relax%
  \if@eqalast\expandafter\eqa@eqnum@i\else\expandafter\eqa@eqnum@ii\fi%
}
\def\eqa@eqnum@i{%
  \if@eqnsw%
    \eqa@eqpos{(\theequation)}\stepcounter{equation}%
  \else%
    \eqa@eqpos\eqa@number%
  \fi%
  \global\@eqnswtrue%
  \cr%
}
\def\eqa@eqnum@ii{&\eqa@eqnum}
%    \end{macrocode}
%
% \subsubsection{Numbering control}
%
% This is trivial.  We set the |\if@eqnsw| flag to be |false| and store the
% text in a macro.
%
%    \begin{macrocode}
\let\nonumber\relax
\newcommand\nonumber[1][]{\global\@eqnswfalse\global\def\eqa@number{#1}}
%    \end{macrocode}
%
% \subsubsection{Closing the environments off}
%
% This is really easy.  Set the final equation number, close the |\halign|,
% tidy up the equation counter (it's been stepped once too many times) and
% close the display.
%
%    \begin{macrocode}
\def\endeqnarray{%
  \eqa@eqnum%
  \egroup%
  \global\advance\c@equation\m@ne%
  $$%
  \global\@ignoretrue%
}
\expandafter\let\csname endeqnarray*\endcsname\endeqnarray
%    \end{macrocode}
%
% Now start up the other package again.
%
%    \begin{macrocode}
%</oldeqnarray>
%<*package>
%    \end{macrocode}
%
% \end{old-eqnarray}
%
% That's all there is.  Byebye.
%
%    \begin{macrocode}
%</package>
%    \end{macrocode}
%
% \hfill Mark Wooding, \today
%
% \Finale
\endinput
