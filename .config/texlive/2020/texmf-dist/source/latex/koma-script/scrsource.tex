% ======================================================================
% scrsource.tex
% Copyright (c) Markus Kohm, 2002-2019
%
% This file is part of the LaTeX2e KOMA-Script bundle.
%
% This work may be distributed and/or modified under the conditions of
% the LaTeX Project Public License, version 1.3c of the license.
% The latest version of this license is in
%   http://www.latex-project.org/lppl.txt
% and version 1.3c or later is part of all distributions of LaTeX 
% version 2005/12/01 or later and of this work.
%
% This work has the LPPL maintenance status "author-maintained".
%
% The Current Maintainer and author of this work is Markus Kohm.
%
% This work consists of all files listed in manifest.txt.
% ----------------------------------------------------------------------
% scrsource.tex
% Copyright (c) Markus Kohm, 2002-2019
%
% Dieses Werk darf nach den Bedingungen der LaTeX Project Public Lizenz,
% Version 1.3c, verteilt und/oder veraendert werden.
% Die neuste Version dieser Lizenz ist
%   http://www.latex-project.org/lppl.txt
% und Version 1.3c ist Teil aller Verteilungen von LaTeX
% Version 2005/12/01 oder spaeter und dieses Werks.
%
% Dieses Werk hat den LPPL-Verwaltungs-Status "author-maintained"
% (allein durch den Autor verwaltet).
%
% Der Aktuelle Verwalter und Autor dieses Werkes ist Markus Kohm.
% 
% Dieses Werk besteht aus den in manifest.txt aufgefuehrten Dateien.
% ======================================================================

% Try to get the version of KOMA-Script from scrkernel-version.dtx.
\begingroup
  \def\ProvidesFile#1[#2]{\csname endinput\endcsname}%
  \input scrkernel-version.dtx
  \newcommand*{\ParseKOMAScriptVersion}{}
  \def\ParseKOMAScriptVersion#1/#2/#3 v#4 #5\ParseKOMAScriptVersion{%
    \gdef\filedate{#1\kern1pt-\kern0pt#2\kern1pt-\kern0pt#3}%
    \gdef\fileversion{#4}%
  }
  \expandafter
  \ParseKOMAScriptVersion\KOMAScriptVersion\ParseKOMAScriptVersion
  \gdef\filename{\jobname}
\endgroup

\ProvidesFile{scrsource.tex}[\KOMAScriptVersion (source)]
\documentclass{scrdoc}
\usepackage[T1]{fontenc}
\usepackage{lmodern}
\usepackage[english,ngerman]{babel}
\newcommand*{\Environment}[1]{\texttt{\mbox{#1}}}
\CodelineIndex
\RecordChanges\setcounter{GlossaryColumns}{1}
\title{\KOMAScript{} -- The Source\footnote{Ich weiß natürlich, dass ein
    Englischer Titel eines vorwiegend deutschsprachigen Dokuments ein
    wenig eigentümlich ist, aber "`\KOMAScript{} -- Der
    Quelltext"' wollte mir einfach nicht gefallen.}%
}
\date{\filedate\\[1ex] Version \fileversion}
\author{Markus Kohm}

\begin{document}
  \maketitle
  \begin{abstract}\noindent
    In diesem Dokument finden Sie \emph{nicht} die Anleitung zu
    \KOMAScript{}. Wenn Sie diese suchen, dann suchen Sie bitte nach
    \texttt{scrguide} (Deutsch) oder \texttt{scrguien} (Englisch). In
    diesem Dokument ist die Implementierung von \KOMAScript{} -- in
    erster Linie der \KOMAScript{}-Klassen -- dokumentiert.
  \end{abstract}
  \tableofcontents

  \clearpage
  \addsec{Vorwort}
  
  Sie finden im Folgenden die Dokumentation der Implementierung von
  \KOMAScript.  Diese kann für Paketautoren von Interesse sein. Bevor
  ein Paketautor ein Makro umdefiniert oder ein internes Makro
  verwendet, sollte er jedoch Rücksprache mit dem Autor von
  \KOMAScript{} halten, damit so weit wie möglich sichergestellt ist,
  dass bei zukünftigen Versionen von \KOMAScript{} die notwendige
  Kompatibilität erhalten bleibt oder der Paketautor über notwendige
  Änderungen an den entsprechenden Makros informiert wird.

  Teile der Dokumentation sind in Englisch. Der größte Teil ist jedoch
  in Deutsch.

  Der Quelltext ist in logische Gruppen eingeteilt und in dieser Form
  dokumentiert. Die Reihenfolge in der Dokumentation entspricht jedoch
  \emph{nicht} der Reihenfolge des Codes in den Klassen und Paketen.

  \DocInclude{scrkernel-version}
  \DocInclude{scrkernel-basics}
%  \DocInclude{tocbasic}
  \DocInclude{scrkernel-miscellaneous}
  \DocInclude{scrkernel-language} 
  \DocInclude{scrkernel-fonts}
  \DocInclude{scrkernel-typearea} 
  \DocInclude{scrkernel-floats}
  \DocInclude{scrkernel-footnotes} 
  \DocInclude{scrkernel-pagestyles}
  \DocInclude{scrkernel-paragraphs}
  \DocInclude{scrkernel-title}
  \DocInclude{scrkernel-sections} 
  \DocInclude{scrkernel-listsof}
  \IfFileExists{scrkernel-tocstyle.dtx}{\DocInclude{scrkernel-tocstyle}}{}
  \DocInclude{scrkernel-bibliography} 
  \DocInclude{scrkernel-index}
  \DocInclude{scrkernel-listsandtabulars}
  \DocInclude{scrlfile} 
  \DocInclude{scrlogo}
  \DocInclude{scrkernel-compatibility}
  \DocInclude{scrkernel-notepaper}
  \DocInclude{scrkernel-variables}
  \DocInclude{scrkernel-pseudolengths}
  \DocInclude{scrkernel-letterclassoptions}
  \DocInclude{japanlco}
  \DocInclude{scrkernel-circularletters}
  \DocInclude{scrextend}
  \DocInclude{scrhack}

  \PrintIndex
  \PrintChanges
\end{document}
%
% end of file `scrsource.tex'
%%% Local Variables:
%%% mode: latex
%%% TeX-master: t
%%% End:
