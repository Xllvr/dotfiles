\ProvidesFile{fnlineno.tex}[2011/02/14 documenting fnlineno.sty (UL)]
\title{\textsf{fnlineno.sty}\\---\\Numbering Footnote Lines\thanks{This 
       document %%% manual %% 2010/12/28
       describes version 
       \textcolor{blue}{\UseVersionOf{fnlineno.sty}} 
       of \textsf{fnlineno.sty} as of \UseDateOf{fnlineno.sty}.}}
% \listfiles                                          %% 2010/12/22
{ \RequirePackage{makedoc}[2010/12/20] \ProcessLineMessage{} 
  \MakeJobDoc{19}{\SectionLevelThreeParseInput}     %% 2010/12/16
}
\documentclass{article}%% TODO paper dimensions!?
\ProvidesFile{makedoc.cfg}[{2013/03/25 documentation settings}] 
%%
\author{Uwe L\"uck\thanks{%
        \url{http://contact-ednotes.sty.de.vu}}}
%%
%% 'hyperref':
\RequirePackage{ifpdf}
\usepackage[%
  \ifpdf
%     bookmarks=false,                  %% 2010/12/22
%     bookmarksnumbered,
    bookmarksopen,                      %% 2011/01/24!?
    bookmarksopenlevel=2,               %% 2011/01/23
%     pdfpagemode=UseNone,
%     pdfstartpage=10,
    pdfstartview=FitH,                  %% 2012/11/26 again
%     pdfstartview=0 0 100,             %% 2011/08/22
%     pdfstartview={XYZ null null 1},   %% 2011/08/25
%     pdfstartview={XYZ null null null},%% 2011/08/25
%     pdfstartview={XYZ null null .5},    %% 2011/08/26
%     pdffitwindow=true,          %% 2011/08/22
    citebordercolor={ .6 1    .6},
    filebordercolor={1    .6 1},
    linkbordercolor={1    .9  .7},
     urlbordercolor={ .7 1   1},   %% playing 2011/01/24
  \else
    draft
  \fi
]{hyperref}
\hypersetup{% 
    pdfauthor={Uwe L\374ck}% 
}
%% metadata, |\MDkeywords{<text>}|, |\MDkeywordsstring|:
%% %% 2011/08/22:
\makeatletter
  \newcommand*{\MDkeywords}[1]{%
    \gdef\MDkeywordsstring{#1}%
    \hypersetup{pdfkeywords=\MDkeywordsstring}%% TODO!?
  }
  \@onlypreamble\MDkeywords
%% |\MDaddtoabstract{<par-head>}|, `:' added:
  \newcommand*{\MDaddtoabstract}[1]{%           %% 2012/05/10
    \par\smallskip\noindent
    \strong{#1:}\quad\ignorespaces}
%% \pagebreak[2]
%% |\printMDkeywords|:
  \newcommand*{\printMDkeywords}{%
    \MDaddtoabstract{Keywords}%
    \MDkeywordsstring 
%     \global\let\MDkeywordsstring\relax    %% `%' 2012/11/12
  }
%% The previous definitions mainly are useful with a variant 
%% |\begin{MDabstract}| of \LaTeX's `{abstract}' environment:
  \newenvironment{MDabstract}
                 {\abstract\noindent
                  \hspace{1sp}%% for niceverb
                  \ignorespaces}
                 {\@ifundefined{MDkeywordsstring}%
                               {}%
                               {\printMDkeywords}%
                  \global\let\MDabstract\relax    %% 2012/11/12
                  \global\let\endMDabstract\relax %% 2012/11/12
                  \endabstract}
%% |\[MD]docnewline| 2012/11/12 from `readprov.tex':
  \newcommand*{\MDdocnewline}{\leavevmode\@normalcr[\topsep]}
%% <- `\leavevmode' for use with `\paragraph'.
%%    Sometimes needs to be preceded by a space.
%% 
%% |\MDfinaldatechecks[<tex-script>]| with \ctanpkgref{filedate}:
  \newcommand*{\MDfinaldatechecks}[1][fdatechk]{%
    \AtEndDocument{%
%       \clearpage %% 2013/03/25 no avail -- with `filedate'!
      \def\@pkgextension{sty}%
      \def\NeedsTeXFormat##1[##2]{}%
      \noNiceVerb                       %% 2013/03/22
      \input{#1}%
    }}
  \@onlypreamble\MDfinaldatechecks
\makeatother
%% Use other packages:
\RequirePackage{niceverb}[2011/01/24] 
\RequirePackage{readprov}               %% 2010/12/08
\RequirePackage{hypertoc}               %% 2011/01/23
\RequirePackage{texlinks}               %% 2011/01/24
\RequirePackage{relsize}                %% 2011/06/27
\RequirePackage{color}                  %% 2011/08/06
\RequirePackage{lmodern}                %% 2012/10/29
\RequirePackage{filedate}               %% 2012/11/12
\RequirePackage{filesdo}                %% 2013/03/22 
%% \pagebreak[3]
%% Logical markup:\qquad  |\strong{<chars>}|, |\meta{<chars>}|, 
%% |\acro{<chars>}|, |\pkg{<chars>}|, 
%% |\code{<chars>}|, |\file{<chars>}|:{\sloppy\par}
\makeatletter
  \def\do#1#2{\@ifdefinable#1{\let#1#2}}%% 2012/07/13
  \do\strong\textbf \do\file\texttt \do\acro\textsmaller 
  %% <- wrong tests before 2012/07/13
  \do\meta\textit   \do \pkg\textsf \do\code\texttt
  \ifpdf
    \pdfstringdefDisableCommands{%
        \let\acro\textrm 
        \let\file\textrm                            %% 2011/11/09
        \let\code\textrm                            %% 2011/11/20
        \let\pkg \textrm                            %% 2012/03/23
    }
  \fi
  %% TODO 2011/07/22 -> `htlogml.sty'
\makeatother
%% |\qtdcode{<text>}|: 2012/10/24:
    \newcommand*{\qtdcode}[1]{`\code{#1}'} 
%% |\pkgtitle{<package-name>}{<caption>}| 
\newcommand*{\pkgtitle}[2]{%            %% 2012/07/13
    \global\let\pkgtitle\relax
    \pkg{\huge #1}\\---\\#2\thanks{This 
       document describes version 
       \textcolor{blue}{\UseVersionOf{\jobname.sty}} 
       of \textsf{\jobname.sty} as of \UseDateOf{\jobname.sty}.}}
%% TODO: %% |\TODO| bad with `mdoccorr.cfg'
\newcommand*{\TODO}{\textcolor{blue}{\acro{TODO}}}  %% 2012/11/06
%% `\MDsampleinput[{<file>}' was added 2012/11/06. 
%% Problems with `myfilist.tex' were due to 'parskip.sty'
%% there. On 2012/11/12, we change the former simple macro to a 
%% much more complex
%% |\MDsamplecodeinput[<add-hfuss>]{<file>}| 
\newcommand*{\MDsamplecodeinput}[2][]{%
    \begingroup
        \vskip\bigskipamount \hrule
        \nobreak\vskip-\parskip 
%         \nobreak\vskip\medskipamount
%% Previous mistake (same below) due to manual change 
%% of `\topsep' in the file `myfilist.tex' (2012/11/30).
        \ifx\\#1\\\else
            \hfuzz=\textwidth \advance\hfuzz#1\relax
        \fi
        \noNiceVerb \verbatiminput{#2}%
%         \nobreak\vskip\medskipamount 
        \hrule \vskip-\parskip 
        \bigskip %%% \bigbreak
%% `\bigbreak' made much larger space in `myfilist.tex'.
    \endgroup
}
%% |\ctanpkgdref{<pkg-id>}| adds the printed link to 
%% `ctan.org/pkg' as a footnote. There is a little space 
%% for coloured link borders:
\newcommand*{\ctanpkgdref}[1]{%
    \ctanpkgref{#1}\,\urlfoot{CtanPkgRef}{#1}}
\errorcontextlines=4
\pagestyle{headings}

\endinput 
 %% shared formatting settings
\newcommand*{\lt}{<} \newcommand*{\gt}{>}           %% 2010/12/22
\providecommand*{\strong}{\textbf}                  %% 2010/12/15
\ReadPackageInfos{fnlineno}
\usepackage{color}
% \hypersetup{bookmarksopen}    %% rm. 2010/12/21, cf. .cfg
%% 2010/12/21: %% 2010/12/26 not sure, splits code
% \makeatletter \@beginparpenalty\@highpenalty \makeatother
%% 2010/12/27:
\makeatletter \@beginparpenalty\@lowpenalty \makeatother
\sloppy
\begin{document}
\maketitle
\begin{abstract}\noindent
'fnlineno.sty' extends 
% Stephan~I. B\"ottcher's 
\CtanPkgRef{lineno}{lineno.sty}\urlpkgfoot{lineno}
(created by Stephan~I. B\"ottcher) 
such that even 
`\footnote'                                 %% `\' 2010/12/09
lines are numbered and can be referred to 
using `\linelabel', `\ref', etc. 
%% rm. 2011/02/09:
% Version v0.5 aims at working as a user expects 
% (just cf.~``Limitations"), otherwise please complain!

Making the package was motivated as support for 
\emph{critical editions}
% of scientific work from an age when footnotes 
% were a standard in publishing in print, 
%% <- 2011/02/09 ->
of \emph{printed works with footnotes} 
as opposed to scholarly critical editions of \emph{manuscripts.} 
For this purpose, an extension 'edfnotes' of the \ctanpkgref{ednotes}
package for critical editions, building on 'fnlineno', is provided 
by the \textit{ednotes} bundle.\urlfoot{CtanPkgRef}{ednotes}

'lineno.sty' has also been used for the revision process 
of \emph{submissions.} 
With 'fnlineno.sty', reference to footnotes 
in the submitted work may become possible. 

%% rm. 2011/02/09:
% Another standalone package 'finstrut' is described. 
As to \emph{implementation:}    %% 2011/02/14
1.~Some included tools for 
\emph{storing and restoring global settings} 
may be ``exported" as standalone packages later. 
2.~The method of typesetting footnotes on the main vertical list 
may later lead to applying the line numbering method to 
several \emph{parallel} texts (with footnotes) and to 
`inner' material such as table cells.
%% <- 2011/02/14 ->
% \dots

%% new 2011/02/09:
  \smallskip\noindent
\strong{Keywords:}\quad line numbers; footnotes, pagewise, 
critical editions, revision
\end{abstract}
\tableofcontents

%   \newpage                    %% rm. 2011/02/09
\section{Usage and Features}
\subsection{Package File Header (Legalize)}
\ProvidesFile{fnlineno.tex}[2011/02/14 documenting fnlineno.sty (UL)]
\title{\textsf{fnlineno.sty}\\---\\Numbering Footnote Lines\thanks{This 
       document %%% manual %% 2010/12/28
       describes version 
       \textcolor{blue}{\UseVersionOf{fnlineno.sty}} 
       of \textsf{fnlineno.sty} as of \UseDateOf{fnlineno.sty}.}}
% \listfiles                                          %% 2010/12/22
{ \RequirePackage{makedoc}[2010/12/20] \ProcessLineMessage{} 
  \MakeJobDoc{19}{\SectionLevelThreeParseInput}     %% 2010/12/16
}
\documentclass{article}%% TODO paper dimensions!?
\ProvidesFile{makedoc.cfg}[{2013/03/25 documentation settings}] 
%%
\author{Uwe L\"uck\thanks{%
        \url{http://contact-ednotes.sty.de.vu}}}
%%
%% 'hyperref':
\RequirePackage{ifpdf}
\usepackage[%
  \ifpdf
%     bookmarks=false,                  %% 2010/12/22
%     bookmarksnumbered,
    bookmarksopen,                      %% 2011/01/24!?
    bookmarksopenlevel=2,               %% 2011/01/23
%     pdfpagemode=UseNone,
%     pdfstartpage=10,
    pdfstartview=FitH,                  %% 2012/11/26 again
%     pdfstartview=0 0 100,             %% 2011/08/22
%     pdfstartview={XYZ null null 1},   %% 2011/08/25
%     pdfstartview={XYZ null null null},%% 2011/08/25
%     pdfstartview={XYZ null null .5},    %% 2011/08/26
%     pdffitwindow=true,          %% 2011/08/22
    citebordercolor={ .6 1    .6},
    filebordercolor={1    .6 1},
    linkbordercolor={1    .9  .7},
     urlbordercolor={ .7 1   1},   %% playing 2011/01/24
  \else
    draft
  \fi
]{hyperref}
\hypersetup{% 
    pdfauthor={Uwe L\374ck}% 
}
%% metadata, |\MDkeywords{<text>}|, |\MDkeywordsstring|:
%% %% 2011/08/22:
\makeatletter
  \newcommand*{\MDkeywords}[1]{%
    \gdef\MDkeywordsstring{#1}%
    \hypersetup{pdfkeywords=\MDkeywordsstring}%% TODO!?
  }
  \@onlypreamble\MDkeywords
%% |\MDaddtoabstract{<par-head>}|, `:' added:
  \newcommand*{\MDaddtoabstract}[1]{%           %% 2012/05/10
    \par\smallskip\noindent
    \strong{#1:}\quad\ignorespaces}
%% \pagebreak[2]
%% |\printMDkeywords|:
  \newcommand*{\printMDkeywords}{%
    \MDaddtoabstract{Keywords}%
    \MDkeywordsstring 
%     \global\let\MDkeywordsstring\relax    %% `%' 2012/11/12
  }
%% The previous definitions mainly are useful with a variant 
%% |\begin{MDabstract}| of \LaTeX's `{abstract}' environment:
  \newenvironment{MDabstract}
                 {\abstract\noindent
                  \hspace{1sp}%% for niceverb
                  \ignorespaces}
                 {\@ifundefined{MDkeywordsstring}%
                               {}%
                               {\printMDkeywords}%
                  \global\let\MDabstract\relax    %% 2012/11/12
                  \global\let\endMDabstract\relax %% 2012/11/12
                  \endabstract}
%% |\[MD]docnewline| 2012/11/12 from `readprov.tex':
  \newcommand*{\MDdocnewline}{\leavevmode\@normalcr[\topsep]}
%% <- `\leavevmode' for use with `\paragraph'.
%%    Sometimes needs to be preceded by a space.
%% 
%% |\MDfinaldatechecks[<tex-script>]| with \ctanpkgref{filedate}:
  \newcommand*{\MDfinaldatechecks}[1][fdatechk]{%
    \AtEndDocument{%
%       \clearpage %% 2013/03/25 no avail -- with `filedate'!
      \def\@pkgextension{sty}%
      \def\NeedsTeXFormat##1[##2]{}%
      \noNiceVerb                       %% 2013/03/22
      \input{#1}%
    }}
  \@onlypreamble\MDfinaldatechecks
\makeatother
%% Use other packages:
\RequirePackage{niceverb}[2011/01/24] 
\RequirePackage{readprov}               %% 2010/12/08
\RequirePackage{hypertoc}               %% 2011/01/23
\RequirePackage{texlinks}               %% 2011/01/24
\RequirePackage{relsize}                %% 2011/06/27
\RequirePackage{color}                  %% 2011/08/06
\RequirePackage{lmodern}                %% 2012/10/29
\RequirePackage{filedate}               %% 2012/11/12
\RequirePackage{filesdo}                %% 2013/03/22 
%% \pagebreak[3]
%% Logical markup:\qquad  |\strong{<chars>}|, |\meta{<chars>}|, 
%% |\acro{<chars>}|, |\pkg{<chars>}|, 
%% |\code{<chars>}|, |\file{<chars>}|:{\sloppy\par}
\makeatletter
  \def\do#1#2{\@ifdefinable#1{\let#1#2}}%% 2012/07/13
  \do\strong\textbf \do\file\texttt \do\acro\textsmaller 
  %% <- wrong tests before 2012/07/13
  \do\meta\textit   \do \pkg\textsf \do\code\texttt
  \ifpdf
    \pdfstringdefDisableCommands{%
        \let\acro\textrm 
        \let\file\textrm                            %% 2011/11/09
        \let\code\textrm                            %% 2011/11/20
        \let\pkg \textrm                            %% 2012/03/23
    }
  \fi
  %% TODO 2011/07/22 -> `htlogml.sty'
\makeatother
%% |\qtdcode{<text>}|: 2012/10/24:
    \newcommand*{\qtdcode}[1]{`\code{#1}'} 
%% |\pkgtitle{<package-name>}{<caption>}| 
\newcommand*{\pkgtitle}[2]{%            %% 2012/07/13
    \global\let\pkgtitle\relax
    \pkg{\huge #1}\\---\\#2\thanks{This 
       document describes version 
       \textcolor{blue}{\UseVersionOf{\jobname.sty}} 
       of \textsf{\jobname.sty} as of \UseDateOf{\jobname.sty}.}}
%% TODO: %% |\TODO| bad with `mdoccorr.cfg'
\newcommand*{\TODO}{\textcolor{blue}{\acro{TODO}}}  %% 2012/11/06
%% `\MDsampleinput[{<file>}' was added 2012/11/06. 
%% Problems with `myfilist.tex' were due to 'parskip.sty'
%% there. On 2012/11/12, we change the former simple macro to a 
%% much more complex
%% |\MDsamplecodeinput[<add-hfuss>]{<file>}| 
\newcommand*{\MDsamplecodeinput}[2][]{%
    \begingroup
        \vskip\bigskipamount \hrule
        \nobreak\vskip-\parskip 
%         \nobreak\vskip\medskipamount
%% Previous mistake (same below) due to manual change 
%% of `\topsep' in the file `myfilist.tex' (2012/11/30).
        \ifx\\#1\\\else
            \hfuzz=\textwidth \advance\hfuzz#1\relax
        \fi
        \noNiceVerb \verbatiminput{#2}%
%         \nobreak\vskip\medskipamount 
        \hrule \vskip-\parskip 
        \bigskip %%% \bigbreak
%% `\bigbreak' made much larger space in `myfilist.tex'.
    \endgroup
}
%% |\ctanpkgdref{<pkg-id>}| adds the printed link to 
%% `ctan.org/pkg' as a footnote. There is a little space 
%% for coloured link borders:
\newcommand*{\ctanpkgdref}[1]{%
    \ctanpkgref{#1}\,\urlfoot{CtanPkgRef}{#1}}
\errorcontextlines=4
\pagestyle{headings}

\endinput 
 %% shared formatting settings
\newcommand*{\lt}{<} \newcommand*{\gt}{>}           %% 2010/12/22
\providecommand*{\strong}{\textbf}                  %% 2010/12/15
\ReadPackageInfos{fnlineno}
\usepackage{color}
% \hypersetup{bookmarksopen}    %% rm. 2010/12/21, cf. .cfg
%% 2010/12/21: %% 2010/12/26 not sure, splits code
% \makeatletter \@beginparpenalty\@highpenalty \makeatother
%% 2010/12/27:
\makeatletter \@beginparpenalty\@lowpenalty \makeatother
\sloppy
\begin{document}
\maketitle
\begin{abstract}\noindent
'fnlineno.sty' extends 
% Stephan~I. B\"ottcher's 
\CtanPkgRef{lineno}{lineno.sty}\urlpkgfoot{lineno}
(created by Stephan~I. B\"ottcher) 
such that even 
`\footnote'                                 %% `\' 2010/12/09
lines are numbered and can be referred to 
using `\linelabel', `\ref', etc. 
%% rm. 2011/02/09:
% Version v0.5 aims at working as a user expects 
% (just cf.~``Limitations"), otherwise please complain!

Making the package was motivated as support for 
\emph{critical editions}
% of scientific work from an age when footnotes 
% were a standard in publishing in print, 
%% <- 2011/02/09 ->
of \emph{printed works with footnotes} 
as opposed to scholarly critical editions of \emph{manuscripts.} 
For this purpose, an extension 'edfnotes' of the \ctanpkgref{ednotes}
package for critical editions, building on 'fnlineno', is provided 
by the \textit{ednotes} bundle.\urlfoot{CtanPkgRef}{ednotes}

'lineno.sty' has also been used for the revision process 
of \emph{submissions.} 
With 'fnlineno.sty', reference to footnotes 
in the submitted work may become possible. 

%% rm. 2011/02/09:
% Another standalone package 'finstrut' is described. 
As to \emph{implementation:}    %% 2011/02/14
1.~Some included tools for 
\emph{storing and restoring global settings} 
may be ``exported" as standalone packages later. 
2.~The method of typesetting footnotes on the main vertical list 
may later lead to applying the line numbering method to 
several \emph{parallel} texts (with footnotes) and to 
`inner' material such as table cells.
%% <- 2011/02/14 ->
% \dots

%% new 2011/02/09:
  \smallskip\noindent
\strong{Keywords:}\quad line numbers; footnotes, pagewise, 
critical editions, revision
\end{abstract}
\tableofcontents

%   \newpage                    %% rm. 2011/02/09
\section{Usage and Features}
\subsection{Package File Header (Legalize)}
\ProvidesFile{fnlineno.tex}[2011/02/14 documenting fnlineno.sty (UL)]
\title{\textsf{fnlineno.sty}\\---\\Numbering Footnote Lines\thanks{This 
       document %%% manual %% 2010/12/28
       describes version 
       \textcolor{blue}{\UseVersionOf{fnlineno.sty}} 
       of \textsf{fnlineno.sty} as of \UseDateOf{fnlineno.sty}.}}
% \listfiles                                          %% 2010/12/22
{ \RequirePackage{makedoc}[2010/12/20] \ProcessLineMessage{} 
  \MakeJobDoc{19}{\SectionLevelThreeParseInput}     %% 2010/12/16
}
\documentclass{article}%% TODO paper dimensions!?
\ProvidesFile{makedoc.cfg}[{2013/03/25 documentation settings}] 
%%
\author{Uwe L\"uck\thanks{%
        \url{http://contact-ednotes.sty.de.vu}}}
%%
%% 'hyperref':
\RequirePackage{ifpdf}
\usepackage[%
  \ifpdf
%     bookmarks=false,                  %% 2010/12/22
%     bookmarksnumbered,
    bookmarksopen,                      %% 2011/01/24!?
    bookmarksopenlevel=2,               %% 2011/01/23
%     pdfpagemode=UseNone,
%     pdfstartpage=10,
    pdfstartview=FitH,                  %% 2012/11/26 again
%     pdfstartview=0 0 100,             %% 2011/08/22
%     pdfstartview={XYZ null null 1},   %% 2011/08/25
%     pdfstartview={XYZ null null null},%% 2011/08/25
%     pdfstartview={XYZ null null .5},    %% 2011/08/26
%     pdffitwindow=true,          %% 2011/08/22
    citebordercolor={ .6 1    .6},
    filebordercolor={1    .6 1},
    linkbordercolor={1    .9  .7},
     urlbordercolor={ .7 1   1},   %% playing 2011/01/24
  \else
    draft
  \fi
]{hyperref}
\hypersetup{% 
    pdfauthor={Uwe L\374ck}% 
}
%% metadata, |\MDkeywords{<text>}|, |\MDkeywordsstring|:
%% %% 2011/08/22:
\makeatletter
  \newcommand*{\MDkeywords}[1]{%
    \gdef\MDkeywordsstring{#1}%
    \hypersetup{pdfkeywords=\MDkeywordsstring}%% TODO!?
  }
  \@onlypreamble\MDkeywords
%% |\MDaddtoabstract{<par-head>}|, `:' added:
  \newcommand*{\MDaddtoabstract}[1]{%           %% 2012/05/10
    \par\smallskip\noindent
    \strong{#1:}\quad\ignorespaces}
%% \pagebreak[2]
%% |\printMDkeywords|:
  \newcommand*{\printMDkeywords}{%
    \MDaddtoabstract{Keywords}%
    \MDkeywordsstring 
%     \global\let\MDkeywordsstring\relax    %% `%' 2012/11/12
  }
%% The previous definitions mainly are useful with a variant 
%% |\begin{MDabstract}| of \LaTeX's `{abstract}' environment:
  \newenvironment{MDabstract}
                 {\abstract\noindent
                  \hspace{1sp}%% for niceverb
                  \ignorespaces}
                 {\@ifundefined{MDkeywordsstring}%
                               {}%
                               {\printMDkeywords}%
                  \global\let\MDabstract\relax    %% 2012/11/12
                  \global\let\endMDabstract\relax %% 2012/11/12
                  \endabstract}
%% |\[MD]docnewline| 2012/11/12 from `readprov.tex':
  \newcommand*{\MDdocnewline}{\leavevmode\@normalcr[\topsep]}
%% <- `\leavevmode' for use with `\paragraph'.
%%    Sometimes needs to be preceded by a space.
%% 
%% |\MDfinaldatechecks[<tex-script>]| with \ctanpkgref{filedate}:
  \newcommand*{\MDfinaldatechecks}[1][fdatechk]{%
    \AtEndDocument{%
%       \clearpage %% 2013/03/25 no avail -- with `filedate'!
      \def\@pkgextension{sty}%
      \def\NeedsTeXFormat##1[##2]{}%
      \noNiceVerb                       %% 2013/03/22
      \input{#1}%
    }}
  \@onlypreamble\MDfinaldatechecks
\makeatother
%% Use other packages:
\RequirePackage{niceverb}[2011/01/24] 
\RequirePackage{readprov}               %% 2010/12/08
\RequirePackage{hypertoc}               %% 2011/01/23
\RequirePackage{texlinks}               %% 2011/01/24
\RequirePackage{relsize}                %% 2011/06/27
\RequirePackage{color}                  %% 2011/08/06
\RequirePackage{lmodern}                %% 2012/10/29
\RequirePackage{filedate}               %% 2012/11/12
\RequirePackage{filesdo}                %% 2013/03/22 
%% \pagebreak[3]
%% Logical markup:\qquad  |\strong{<chars>}|, |\meta{<chars>}|, 
%% |\acro{<chars>}|, |\pkg{<chars>}|, 
%% |\code{<chars>}|, |\file{<chars>}|:{\sloppy\par}
\makeatletter
  \def\do#1#2{\@ifdefinable#1{\let#1#2}}%% 2012/07/13
  \do\strong\textbf \do\file\texttt \do\acro\textsmaller 
  %% <- wrong tests before 2012/07/13
  \do\meta\textit   \do \pkg\textsf \do\code\texttt
  \ifpdf
    \pdfstringdefDisableCommands{%
        \let\acro\textrm 
        \let\file\textrm                            %% 2011/11/09
        \let\code\textrm                            %% 2011/11/20
        \let\pkg \textrm                            %% 2012/03/23
    }
  \fi
  %% TODO 2011/07/22 -> `htlogml.sty'
\makeatother
%% |\qtdcode{<text>}|: 2012/10/24:
    \newcommand*{\qtdcode}[1]{`\code{#1}'} 
%% |\pkgtitle{<package-name>}{<caption>}| 
\newcommand*{\pkgtitle}[2]{%            %% 2012/07/13
    \global\let\pkgtitle\relax
    \pkg{\huge #1}\\---\\#2\thanks{This 
       document describes version 
       \textcolor{blue}{\UseVersionOf{\jobname.sty}} 
       of \textsf{\jobname.sty} as of \UseDateOf{\jobname.sty}.}}
%% TODO: %% |\TODO| bad with `mdoccorr.cfg'
\newcommand*{\TODO}{\textcolor{blue}{\acro{TODO}}}  %% 2012/11/06
%% `\MDsampleinput[{<file>}' was added 2012/11/06. 
%% Problems with `myfilist.tex' were due to 'parskip.sty'
%% there. On 2012/11/12, we change the former simple macro to a 
%% much more complex
%% |\MDsamplecodeinput[<add-hfuss>]{<file>}| 
\newcommand*{\MDsamplecodeinput}[2][]{%
    \begingroup
        \vskip\bigskipamount \hrule
        \nobreak\vskip-\parskip 
%         \nobreak\vskip\medskipamount
%% Previous mistake (same below) due to manual change 
%% of `\topsep' in the file `myfilist.tex' (2012/11/30).
        \ifx\\#1\\\else
            \hfuzz=\textwidth \advance\hfuzz#1\relax
        \fi
        \noNiceVerb \verbatiminput{#2}%
%         \nobreak\vskip\medskipamount 
        \hrule \vskip-\parskip 
        \bigskip %%% \bigbreak
%% `\bigbreak' made much larger space in `myfilist.tex'.
    \endgroup
}
%% |\ctanpkgdref{<pkg-id>}| adds the printed link to 
%% `ctan.org/pkg' as a footnote. There is a little space 
%% for coloured link borders:
\newcommand*{\ctanpkgdref}[1]{%
    \ctanpkgref{#1}\,\urlfoot{CtanPkgRef}{#1}}
\errorcontextlines=4
\pagestyle{headings}

\endinput 
 %% shared formatting settings
\newcommand*{\lt}{<} \newcommand*{\gt}{>}           %% 2010/12/22
\providecommand*{\strong}{\textbf}                  %% 2010/12/15
\ReadPackageInfos{fnlineno}
\usepackage{color}
% \hypersetup{bookmarksopen}    %% rm. 2010/12/21, cf. .cfg
%% 2010/12/21: %% 2010/12/26 not sure, splits code
% \makeatletter \@beginparpenalty\@highpenalty \makeatother
%% 2010/12/27:
\makeatletter \@beginparpenalty\@lowpenalty \makeatother
\sloppy
\begin{document}
\maketitle
\begin{abstract}\noindent
'fnlineno.sty' extends 
% Stephan~I. B\"ottcher's 
\CtanPkgRef{lineno}{lineno.sty}\urlpkgfoot{lineno}
(created by Stephan~I. B\"ottcher) 
such that even 
`\footnote'                                 %% `\' 2010/12/09
lines are numbered and can be referred to 
using `\linelabel', `\ref', etc. 
%% rm. 2011/02/09:
% Version v0.5 aims at working as a user expects 
% (just cf.~``Limitations"), otherwise please complain!

Making the package was motivated as support for 
\emph{critical editions}
% of scientific work from an age when footnotes 
% were a standard in publishing in print, 
%% <- 2011/02/09 ->
of \emph{printed works with footnotes} 
as opposed to scholarly critical editions of \emph{manuscripts.} 
For this purpose, an extension 'edfnotes' of the \ctanpkgref{ednotes}
package for critical editions, building on 'fnlineno', is provided 
by the \textit{ednotes} bundle.\urlfoot{CtanPkgRef}{ednotes}

'lineno.sty' has also been used for the revision process 
of \emph{submissions.} 
With 'fnlineno.sty', reference to footnotes 
in the submitted work may become possible. 

%% rm. 2011/02/09:
% Another standalone package 'finstrut' is described. 
As to \emph{implementation:}    %% 2011/02/14
1.~Some included tools for 
\emph{storing and restoring global settings} 
may be ``exported" as standalone packages later. 
2.~The method of typesetting footnotes on the main vertical list 
may later lead to applying the line numbering method to 
several \emph{parallel} texts (with footnotes) and to 
`inner' material such as table cells.
%% <- 2011/02/14 ->
% \dots

%% new 2011/02/09:
  \smallskip\noindent
\strong{Keywords:}\quad line numbers; footnotes, pagewise, 
critical editions, revision
\end{abstract}
\tableofcontents

%   \newpage                    %% rm. 2011/02/09
\section{Usage and Features}
\subsection{Package File Header (Legalize)}
\ProvidesFile{fnlineno.tex}[2011/02/14 documenting fnlineno.sty (UL)]
\title{\textsf{fnlineno.sty}\\---\\Numbering Footnote Lines\thanks{This 
       document %%% manual %% 2010/12/28
       describes version 
       \textcolor{blue}{\UseVersionOf{fnlineno.sty}} 
       of \textsf{fnlineno.sty} as of \UseDateOf{fnlineno.sty}.}}
% \listfiles                                          %% 2010/12/22
{ \RequirePackage{makedoc}[2010/12/20] \ProcessLineMessage{} 
  \MakeJobDoc{19}{\SectionLevelThreeParseInput}     %% 2010/12/16
}
\documentclass{article}%% TODO paper dimensions!?
\input{makedoc.cfg} %% shared formatting settings
\newcommand*{\lt}{<} \newcommand*{\gt}{>}           %% 2010/12/22
\providecommand*{\strong}{\textbf}                  %% 2010/12/15
\ReadPackageInfos{fnlineno}
\usepackage{color}
% \hypersetup{bookmarksopen}    %% rm. 2010/12/21, cf. .cfg
%% 2010/12/21: %% 2010/12/26 not sure, splits code
% \makeatletter \@beginparpenalty\@highpenalty \makeatother
%% 2010/12/27:
\makeatletter \@beginparpenalty\@lowpenalty \makeatother
\sloppy
\begin{document}
\maketitle
\begin{abstract}\noindent
'fnlineno.sty' extends 
% Stephan~I. B\"ottcher's 
\CtanPkgRef{lineno}{lineno.sty}\urlpkgfoot{lineno}
(created by Stephan~I. B\"ottcher) 
such that even 
`\footnote'                                 %% `\' 2010/12/09
lines are numbered and can be referred to 
using `\linelabel', `\ref', etc. 
%% rm. 2011/02/09:
% Version v0.5 aims at working as a user expects 
% (just cf.~``Limitations"), otherwise please complain!

Making the package was motivated as support for 
\emph{critical editions}
% of scientific work from an age when footnotes 
% were a standard in publishing in print, 
%% <- 2011/02/09 ->
of \emph{printed works with footnotes} 
as opposed to scholarly critical editions of \emph{manuscripts.} 
For this purpose, an extension 'edfnotes' of the \ctanpkgref{ednotes}
package for critical editions, building on 'fnlineno', is provided 
by the \textit{ednotes} bundle.\urlfoot{CtanPkgRef}{ednotes}

'lineno.sty' has also been used for the revision process 
of \emph{submissions.} 
With 'fnlineno.sty', reference to footnotes 
in the submitted work may become possible. 

%% rm. 2011/02/09:
% Another standalone package 'finstrut' is described. 
As to \emph{implementation:}    %% 2011/02/14
1.~Some included tools for 
\emph{storing and restoring global settings} 
may be ``exported" as standalone packages later. 
2.~The method of typesetting footnotes on the main vertical list 
may later lead to applying the line numbering method to 
several \emph{parallel} texts (with footnotes) and to 
`inner' material such as table cells.
%% <- 2011/02/14 ->
% \dots

%% new 2011/02/09:
  \smallskip\noindent
\strong{Keywords:}\quad line numbers; footnotes, pagewise, 
critical editions, revision
\end{abstract}
\tableofcontents

%   \newpage                    %% rm. 2011/02/09
\section{Usage and Features}
\subsection{Package File Header (Legalize)}
\input{fnlineno.doc}
\end{document}

VERSION HISTORY

2010/12/08  for v0.1    very first
2010/12/09  for v0.2    moved much to .sty
2010/12/15  for v0.4    \strong
2010/12/16              \SectionLevelThree... 
2010/12/22ff.           beginparpenalty varied
2010/12/28  for v0.5    abstract extended
2011/02/09              removing `finstrut'; mention `edfnotes'
2011/02/10              using \urlpkgfoot etc.
2011/02/14              abstract modified

\end{document}

VERSION HISTORY

2010/12/08  for v0.1    very first
2010/12/09  for v0.2    moved much to .sty
2010/12/15  for v0.4    \strong
2010/12/16              \SectionLevelThree... 
2010/12/22ff.           beginparpenalty varied
2010/12/28  for v0.5    abstract extended
2011/02/09              removing `finstrut'; mention `edfnotes'
2011/02/10              using \urlpkgfoot etc.
2011/02/14              abstract modified

\end{document}

VERSION HISTORY

2010/12/08  for v0.1    very first
2010/12/09  for v0.2    moved much to .sty
2010/12/15  for v0.4    \strong
2010/12/16              \SectionLevelThree... 
2010/12/22ff.           beginparpenalty varied
2010/12/28  for v0.5    abstract extended
2011/02/09              removing `finstrut'; mention `edfnotes'
2011/02/10              using \urlpkgfoot etc.
2011/02/14              abstract modified

\end{document}

VERSION HISTORY

2010/12/08  for v0.1    very first
2010/12/09  for v0.2    moved much to .sty
2010/12/15  for v0.4    \strong
2010/12/16              \SectionLevelThree... 
2010/12/22ff.           beginparpenalty varied
2010/12/28  for v0.5    abstract extended
2011/02/09              removing `finstrut'; mention `edfnotes'
2011/02/10              using \urlpkgfoot etc.
2011/02/14              abstract modified
