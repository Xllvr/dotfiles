% \iffalse meta-comment
%
% File: ltxcmds.dtx
% Version: 2019/12/15 v1.24
% Info: LaTeX kernel commands for general use
%
% Copyright (C)
%    2009-2011 Heiko Oberdiek
%    2016-2019 Oberdiek Package Support Group
%    https://github.com/ho-tex/ltxcmds/issues
%
% This work may be distributed and/or modified under the
% conditions of the LaTeX Project Public License, either
% version 1.3c of this license or (at your option) any later
% version. This version of this license is in
%    https://www.latex-project.org/lppl/lppl-1-3c.txt
% and the latest version of this license is in
%    https://www.latex-project.org/lppl.txt
% and version 1.3 or later is part of all distributions of
% LaTeX version 2005/12/01 or later.
%
% This work has the LPPL maintenance status "maintained".
%
% The Current Maintainers of this work are
% Heiko Oberdiek and the Oberdiek Package Support Group
% https://github.com/ho-tex/ltxcmds/issues
%
% The Base Interpreter refers to any `TeX-Format',
% because some files are installed in TDS:tex/generic//.
%
% This work consists of the main source file ltxcmds.dtx
% and the derived files
%    ltxcmds.sty, ltxcmds.pdf, ltxcmds.ins, ltxcmds.drv,
%    ltxcmds-test1.tex, ltxcmds-test-gobble.tex,
%    ltxcmds-test-ifempty.tex, ltxcmds-test-zapspace.tex,
%    ltxcmds-test-ifboxempty.tex, ltxcmds-test-nextchar.tex,
%    ltxcmds-test-carcdr.tex.
%
% Distribution:
%    CTAN:macros/latex/contrib/ltxcmds/ltxcmds.dtx
%    CTAN:macros/latex/contrib/ltxcmds/ltxcmds.pdf
%
% Unpacking:
%    (a) If ltxcmds.ins is present:
%           tex ltxcmds.ins
%    (b) Without ltxcmds.ins:
%           tex ltxcmds.dtx
%    (c) If you insist on using LaTeX
%           latex \let\install=y% \iffalse meta-comment
%
% File: ltxcmds.dtx
% Version: 2019/12/15 v1.24
% Info: LaTeX kernel commands for general use
%
% Copyright (C)
%    2009-2011 Heiko Oberdiek
%    2016-2019 Oberdiek Package Support Group
%    https://github.com/ho-tex/ltxcmds/issues
%
% This work may be distributed and/or modified under the
% conditions of the LaTeX Project Public License, either
% version 1.3c of this license or (at your option) any later
% version. This version of this license is in
%    https://www.latex-project.org/lppl/lppl-1-3c.txt
% and the latest version of this license is in
%    https://www.latex-project.org/lppl.txt
% and version 1.3 or later is part of all distributions of
% LaTeX version 2005/12/01 or later.
%
% This work has the LPPL maintenance status "maintained".
%
% The Current Maintainers of this work are
% Heiko Oberdiek and the Oberdiek Package Support Group
% https://github.com/ho-tex/ltxcmds/issues
%
% The Base Interpreter refers to any `TeX-Format',
% because some files are installed in TDS:tex/generic//.
%
% This work consists of the main source file ltxcmds.dtx
% and the derived files
%    ltxcmds.sty, ltxcmds.pdf, ltxcmds.ins, ltxcmds.drv,
%    ltxcmds-test1.tex, ltxcmds-test-gobble.tex,
%    ltxcmds-test-ifempty.tex, ltxcmds-test-zapspace.tex,
%    ltxcmds-test-ifboxempty.tex, ltxcmds-test-nextchar.tex,
%    ltxcmds-test-carcdr.tex.
%
% Distribution:
%    CTAN:macros/latex/contrib/ltxcmds/ltxcmds.dtx
%    CTAN:macros/latex/contrib/ltxcmds/ltxcmds.pdf
%
% Unpacking:
%    (a) If ltxcmds.ins is present:
%           tex ltxcmds.ins
%    (b) Without ltxcmds.ins:
%           tex ltxcmds.dtx
%    (c) If you insist on using LaTeX
%           latex \let\install=y% \iffalse meta-comment
%
% File: ltxcmds.dtx
% Version: 2019/12/15 v1.24
% Info: LaTeX kernel commands for general use
%
% Copyright (C)
%    2009-2011 Heiko Oberdiek
%    2016-2019 Oberdiek Package Support Group
%    https://github.com/ho-tex/ltxcmds/issues
%
% This work may be distributed and/or modified under the
% conditions of the LaTeX Project Public License, either
% version 1.3c of this license or (at your option) any later
% version. This version of this license is in
%    https://www.latex-project.org/lppl/lppl-1-3c.txt
% and the latest version of this license is in
%    https://www.latex-project.org/lppl.txt
% and version 1.3 or later is part of all distributions of
% LaTeX version 2005/12/01 or later.
%
% This work has the LPPL maintenance status "maintained".
%
% The Current Maintainers of this work are
% Heiko Oberdiek and the Oberdiek Package Support Group
% https://github.com/ho-tex/ltxcmds/issues
%
% The Base Interpreter refers to any `TeX-Format',
% because some files are installed in TDS:tex/generic//.
%
% This work consists of the main source file ltxcmds.dtx
% and the derived files
%    ltxcmds.sty, ltxcmds.pdf, ltxcmds.ins, ltxcmds.drv,
%    ltxcmds-test1.tex, ltxcmds-test-gobble.tex,
%    ltxcmds-test-ifempty.tex, ltxcmds-test-zapspace.tex,
%    ltxcmds-test-ifboxempty.tex, ltxcmds-test-nextchar.tex,
%    ltxcmds-test-carcdr.tex.
%
% Distribution:
%    CTAN:macros/latex/contrib/ltxcmds/ltxcmds.dtx
%    CTAN:macros/latex/contrib/ltxcmds/ltxcmds.pdf
%
% Unpacking:
%    (a) If ltxcmds.ins is present:
%           tex ltxcmds.ins
%    (b) Without ltxcmds.ins:
%           tex ltxcmds.dtx
%    (c) If you insist on using LaTeX
%           latex \let\install=y% \iffalse meta-comment
%
% File: ltxcmds.dtx
% Version: 2019/12/15 v1.24
% Info: LaTeX kernel commands for general use
%
% Copyright (C)
%    2009-2011 Heiko Oberdiek
%    2016-2019 Oberdiek Package Support Group
%    https://github.com/ho-tex/ltxcmds/issues
%
% This work may be distributed and/or modified under the
% conditions of the LaTeX Project Public License, either
% version 1.3c of this license or (at your option) any later
% version. This version of this license is in
%    https://www.latex-project.org/lppl/lppl-1-3c.txt
% and the latest version of this license is in
%    https://www.latex-project.org/lppl.txt
% and version 1.3 or later is part of all distributions of
% LaTeX version 2005/12/01 or later.
%
% This work has the LPPL maintenance status "maintained".
%
% The Current Maintainers of this work are
% Heiko Oberdiek and the Oberdiek Package Support Group
% https://github.com/ho-tex/ltxcmds/issues
%
% The Base Interpreter refers to any `TeX-Format',
% because some files are installed in TDS:tex/generic//.
%
% This work consists of the main source file ltxcmds.dtx
% and the derived files
%    ltxcmds.sty, ltxcmds.pdf, ltxcmds.ins, ltxcmds.drv,
%    ltxcmds-test1.tex, ltxcmds-test-gobble.tex,
%    ltxcmds-test-ifempty.tex, ltxcmds-test-zapspace.tex,
%    ltxcmds-test-ifboxempty.tex, ltxcmds-test-nextchar.tex,
%    ltxcmds-test-carcdr.tex.
%
% Distribution:
%    CTAN:macros/latex/contrib/ltxcmds/ltxcmds.dtx
%    CTAN:macros/latex/contrib/ltxcmds/ltxcmds.pdf
%
% Unpacking:
%    (a) If ltxcmds.ins is present:
%           tex ltxcmds.ins
%    (b) Without ltxcmds.ins:
%           tex ltxcmds.dtx
%    (c) If you insist on using LaTeX
%           latex \let\install=y\input{ltxcmds.dtx}
%        (quote the arguments according to the demands of your shell)
%
% Documentation:
%    (a) If ltxcmds.drv is present:
%           latex ltxcmds.drv
%    (b) Without ltxcmds.drv:
%           latex ltxcmds.dtx; ...
%    The class ltxdoc loads the configuration file ltxdoc.cfg
%    if available. Here you can specify further options, e.g.
%    use A4 as paper format:
%       \PassOptionsToClass{a4paper}{article}
%
%    Programm calls to get the documentation (example):
%       pdflatex ltxcmds.dtx
%       makeindex -s gind.ist ltxcmds.idx
%       pdflatex ltxcmds.dtx
%       makeindex -s gind.ist ltxcmds.idx
%       pdflatex ltxcmds.dtx
%
% Installation:
%    TDS:tex/generic/ltxcmds/ltxcmds.sty
%    TDS:doc/latex/ltxcmds/ltxcmds.pdf
%    TDS:source/latex/ltxcmds/ltxcmds.dtx
%
%<*ignore>
\begingroup
  \catcode123=1 %
  \catcode125=2 %
  \def\x{LaTeX2e}%
\expandafter\endgroup
\ifcase 0\ifx\install y1\fi\expandafter
         \ifx\csname processbatchFile\endcsname\relax\else1\fi
         \ifx\fmtname\x\else 1\fi\relax
\else\csname fi\endcsname
%</ignore>
%<*install>
\input docstrip.tex
\Msg{************************************************************************}
\Msg{* Installation}
\Msg{* Package: ltxcmds 2019/12/15 v1.24 LaTeX kernel commands for general use (HO)}
\Msg{************************************************************************}

\keepsilent
\askforoverwritefalse

\let\MetaPrefix\relax
\preamble

This is a generated file.

Project: ltxcmds
Version: 2019/12/15 v1.24

Copyright (C)
   2009-2011 Heiko Oberdiek
   2016-2019 Oberdiek Package Support Group

This work may be distributed and/or modified under the
conditions of the LaTeX Project Public License, either
version 1.3c of this license or (at your option) any later
version. This version of this license is in
   https://www.latex-project.org/lppl/lppl-1-3c.txt
and the latest version of this license is in
   https://www.latex-project.org/lppl.txt
and version 1.3 or later is part of all distributions of
LaTeX version 2005/12/01 or later.

This work has the LPPL maintenance status "maintained".

The Current Maintainers of this work are
Heiko Oberdiek and the Oberdiek Package Support Group
https://github.com/ho-tex/ltxcmds/issues


The Base Interpreter refers to any `TeX-Format',
because some files are installed in TDS:tex/generic//.

This work consists of the main source file ltxcmds.dtx
and the derived files
   ltxcmds.sty, ltxcmds.pdf, ltxcmds.ins, ltxcmds.drv,
   ltxcmds-test1.tex, ltxcmds-test-gobble.tex,
   ltxcmds-test-ifempty.tex, ltxcmds-test-zapspace.tex,
   ltxcmds-test-ifboxempty.tex, ltxcmds-test-nextchar.tex,
   ltxcmds-test-carcdr.tex.

\endpreamble
\let\MetaPrefix\DoubleperCent

\generate{%
  \file{ltxcmds.ins}{\from{ltxcmds.dtx}{install}}%
  \file{ltxcmds.drv}{\from{ltxcmds.dtx}{driver}}%
  \usedir{tex/generic/ltxcmds}%
  \file{ltxcmds.sty}{\from{ltxcmds.dtx}{package}}%
}

\catcode32=13\relax% active space
\let =\space%
\Msg{************************************************************************}
\Msg{*}
\Msg{* To finish the installation you have to move the following}
\Msg{* file into a directory searched by TeX:}
\Msg{*}
\Msg{*     ltxcmds.sty}
\Msg{*}
\Msg{* To produce the documentation run the file `ltxcmds.drv'}
\Msg{* through LaTeX.}
\Msg{*}
\Msg{* Happy TeXing!}
\Msg{*}
\Msg{************************************************************************}

\endbatchfile
%</install>
%<*ignore>
\fi
%</ignore>
%<*driver>
\NeedsTeXFormat{LaTeX2e}
\ProvidesFile{ltxcmds.drv}%
  [2019/12/15 v1.24 LaTeX kernel commands for general use (HO)]%
\documentclass{ltxdoc}
\usepackage{holtxdoc}[2011/11/22]
\usepackage{zref-savepos}
\usepackage{paralist}
\hfuzz=1pt
\begin{document}
  \DocInput{ltxcmds.dtx}%
\end{document}
%</driver>
% \fi
%
%
%
% \GetFileInfo{ltxcmds.drv}
%
% \title{The \xpackage{ltxcmds} package}
% \date{2019/12/15 v1.24}
% \author{Heiko Oberdiek\thanks
% {Please report any issues at \url{https://github.com/ho-tex/ltxcmds/issues}}}
%
% \maketitle
%
% \begin{abstract}
% The package \xpackage{ltxcmds} exports some utility macros
% from the \LaTeX\ kernel into a separate namespace and
% also provides them for other formats such as plain-\TeX.
% \end{abstract}
%
% \tableofcontents
%
% \newcounter{to}
% \renewcommand*{\theto}{to\arabic{to}}
% \newlength\tolength
% \tolength=\dimexpr .5\linewidth+\oddsidemargin+1in+15mm\relax
% \newcommand*{\TO}{^^A
%   \quad
%   \stepcounter{to}^^A
%   \zsavepos\theto
%   \ifdim\zposx\theto sp>0pt
%     \ifdim\zposx\theto sp<\tolength
%       \kern\dimexpr\tolength-\zposx\theto sp\relax
%     \fi
%   \fi
%   $\rightarrow$\quad
%   \ignorespaces
% }
%
% \section{Documentation}
%
% \subsection{Introduction}
%
% Many of my packages also support other formats such as plain-\TeX.
% Because I am rather familiar with the utility macros from
% \LaTeX's kernel (e.g. \cs{@gobble}, \cs{@firstoftwo}), I found
% myself rewriting them again and again, because they are lacking
% in plain-\TeX.
%
% Therefore this package provides often used macros and similar
% ones with the name prefix \cs{ltx@}. This avoids also faulty
% redefinitions. I remember an example where a package redefined
% \cs{@firstoftwo} with forgetting \cs{long}.
%
% \subsection{Numbers}
%
% \begin{declcs}{ltx@zero} \TO 0\\
%   \cs{ltx@one} \TO 1\\
%   \cs{ltx@two} \TO 2\\
%   \cs{ltx@cclv} \TO 255\\
%   \cs{ltx@minusone} \TO -1
% \end{declcs}
% These commands are numbers 0, 1, 2, 255 and -1. They are not digits
% and a space is not gobbled afterwards. Macro \cs{ltx@minusone}
% is available since version 2010/12/12 v1.15.
%
% \subsection{Scratch registers}
%
% Following the conventions of \hologo{plainTeX} and \hologo{LaTeX}
% the first ten registers are free to use. Even numbered registers
% are for local, odd numbered for global use.
%
% \begin{declcs}{ltx@(Loc,Glob)(Toks,Dimen,Skip)(A,B,C,D,E)}
% \end{declcs}
% The name consists of the prefix \cs{ltx@}, then
% \texttt{Loc} or \texttt{Glob} for local or global usage follows.
% The register type is given by \texttt{Toks} for token register,
% \texttt{Dimen} for dimen register and \texttt{Skip} for skip
% register. As last part the registers are numbered from \texttt{A}
% to \texttt{E}. Example: \cs{ltx@LocToksA}.
%
% Since 2011/04/14 v1.19.
%
% \subsection{Argument killers}
%
% \begin{declcs}{ltx@gobble} \M{1} \TO\\
%   \cs{ltx@gobbletwo} \M{1} \M{2} \TO\\
%   \cs{ltx@gobblethree} \M{1} \M{2} \M{3} \TO\\
%   \cs{ltx@gobblefour} \M{1} \M{2} \M{3} \M{4} \TO
% \end{declcs}
%
% \begin{declcs}{ltx@GobbleNum} \M{num} \M{1} \M{2} \dots\space
%   \M{\meta{num}} \TO
% \end{declcs}
% The first argument \meta{num} of macro \cs{ltx@GobbleNum} specifies,
% how many following arguments are eaten. Macro \cs{ltx@GobbleNum}
% is expandable in exact two expansion steps.
%
% \subsection{Argument grabbers}
%
% \begin{declcs}{ltx@firstofone} \M{1} \TO \meta{1}\\[1ex]
%  \cs{ltx@firstoftwo} \M{1} \M{2} \TO \meta{1}\\
%  \cs{ltx@secondoftwo} \M{1} \M{2} \TO \meta{2}\\[1ex]
%  \cs{ltx@firstofthree} \M{1} \M{2} \M{3} \TO \meta{1}\\
%  \cs{ltx@secondofthree} \M{1} \M{2} \M{3} \TO \meta{2}\\
%  \cs{ltx@thirdofthree} \M{1} \M{2} \M{3} \TO \meta{3}\\
%  \cs{ltx@firstoffour} \M{1} \M{2} \M{3} \M{4} \TO \meta{1}\\
%  \cs{ltx@secondoffour} \M{1} \M{2} \M{3} \M{4} \TO \meta{2}\\
%  \cs{ltx@thirdoffour} \M{1} \M{2} \M{3} \M{4} \TO \meta{3}\\
%  \cs{ltx@fourthoffour} \M{1} \M{2} \M{3} \M{4} \TO \meta{4}
% \end{declcs}
% Macros \cs{ltx@firstofthree}, \cs{ltx@secondofthree}
% and \cs{ltx@thirdofthree} were added in version
% 2010/11/12 v1.11.
% Macros \cs{ltx@firstoffour}, \dots, \cs{ltx@fourthoffour}
% were added in version 2011/02/04 v1.16.
%
% \subsection{List helpers}
%
% \begin{declcs}{ltx@carzero} \dots\ \cs{@nil} \TO\\
%   \cs{ltx@cdrzero} \dots\ \cs{@nil} \TO \dots
% \end{declcs}
%
% \begin{declcs}{ltx@car} \M{1} \dots\ \cs{@nil} \TO \meta{1}\\
%   \cs{ltx@cdr} \M{1} \dots\ \cs{@nil} \TO \dots
% \end{declcs}
%
% \begin{declcs}{ltx@cartwo} \M{1} \M{2} \dots\ \cs{@nil}
%   \TO \meta{1}\meta{2}\\
%   \cs{ltx@carsecond} \M{1} \M{2} \dots\ \cs{@nil} \TO \meta{2}\\
%   \cs{ltx@cdrtwo} \M{1} \M{2} \dots\ \cs{@nil} \TO \dots
% \end{declcs}
%
% \begin{declcs}{ltx@carthree} \M{1} \M{2} \M{3} \dots\ \cs{@nil}
%       \TO \meta{1}\meta{2}\meta{3}\\
%   \cs{ltx@carthird} \M{1} \M{2} \M{3} \dots\ \cs{@nil}
%       \TO \meta{3}\\
%   \cs{ltx@cdrthree} \M{1} \M{2} \M{3} \dots\ \cs{@nil} \TO \dots
% \end{declcs}
%
% \begin{declcs}{ltx@carfour} \M{1} \M{2} \M{3} \M{4} \dots\ \cs{@nil}
%       \TO \meta{1}\meta{2}\meta{3}\meta{4}\\
%   \cs{ltx@carfourth} \M{1} \M{2} \M{3} \M{4} \dots\ \cs{@nil}
%       \TO \meta{4}\\
%   \cs{ltx@cdrfour} \M{1} \M{2} \M{3} \M{4} \dots\ \cs{@nil} \TO \dots
% \end{declcs}
%
% \begin{declcs}{ltx@CarNum} \M{num} \M{1} \dots\ \M{\meta{num}}
%   \M{\meta{num}+1} \dots\ \cs{@nil}\\
%   \TO \M{1} \dots\ \M{\meta{num}} \dots\\
%   \cs{ltx@CarNumth} \M{num} \M{1} \dots\ \M{\meta{num}}
%   \M{\meta{num}+1} \dots\ \cs{@nil}\\
%   \TO \M{\meta{num}} \dots\\
%   \cs{ltx@CdrNum} \M{num} \M{1} \dots\ \M{\meta{num}}
%   \M{\meta{num}+1} \dots\ \cs{@nil}\\
%   \TO \M{\meta{num}+1} \dots
% \end{declcs}
%
% Macros with uppercase letters are expandable in two expansion steps.
% Changes in version 2019/12/15 v1.24:
% \begin{itemize}
% \item
%   Macros \cs{ltx@carsecond}, \cs{ltx@carthird}, \cs{ltx@carfourth},
%   \cs{ltx@CarNumth} added.
% \item
%   Macros \cs{ltx@cdr}, \cs{ltx@cdrtwo}, \cs{ltx@cdrthree}, \cs{ltx@cdrfour},
%   \cs{ltx@CdrNum} are expandable in two expansion steps and retain
%   spaces and braces after the first gobbled arguments.
% \end{itemize}
%
% \subsection{Tail recursion}
%
% \begin{declcs}{ltx@ReturnAfterFi} \M{1} \cs{fi} \TO \cs{fi} \meta{1}\\
%   \cs{ltx@ReturnAfterElseFi} \M{1} \cs{else} \M{2} \cs{fi}
%     \TO \cs{fi} \meta{1}
% \end{declcs}
%
% \subsection{Empty macro}
%
% \begin{declcs}{ltx@empty} \TO
% \end{declcs}
%
% \subsection{Characters}
%
% \begin{declcs}{ltx@space} \TO \verb*| |\\
%   \cs{ltx@percentchar} \TO \texttt{\%}\\
%   \cs{ltx@backslashchar} \TO
%      \makeatletter\texttt{\@backslashchar}\\
%   \cs{ltx@hashchar} \TO \texttt{\#}\qquad(since v1.7)\\
%   \cs{ltx@leftbracechar} \TO \texttt{\textbraceleft}\qquad(since v1.8)\\
%   \cs{ltx@rightbracechar} \TO \texttt{\textbraceright}\qquad(since v1.8)
% \end{declcs}
%
% \subsection{Boolean switch}
%
% \begin{declcs}{ltx@newif} \M{cmd}
% \end{declcs}
% \cs{ltx@newif} defines a new boolean switch \meta{cmd} like \cs{newif}.
% Unlike \plainTeX's \cs{newif}, \cs{ltx@newif} is not \cs{outer}.
% The command \meta{cmd} must start with the two characters |if|.
%
% \begin{declcs}{ltx@newglobalif} \M{cmd}
% \end{declcs}
% \cs{ltx@newglobalif} defines a new boolean switch \meta{cmd}
% like \cs{ltx@newif}. However the switch setting commands,
% \meta{cmd} without the prefix |if| and followed by |true| or |false|
% are acting globally.
%
% \subsection{Command definitions}
%
% \begin{declcs}{ltx@ifundefined} \M{cmd} \M{yes} \M{no}
% \end{declcs}
% If \eTeX\ is available, \cs{ifcsname} is used that does
% not have the side effect of defining undefined commands
% with meaning of \cs{relax}. This command is always expandable.
% Change in version 1.1: Also the meaning \cs{relax} is
% always considered ``undefined''.
%
% \begin{declcs}{ltx@IfUndefined} \M{cmd} \M{yes} \M{no}
% \end{declcs}
% If \eTeX\ is available, \cs{ifcsname} is used that does
% not have the side effect of defining undefined commands
% with meaning of \cs{relax}. Also it always checks for
% the meaning of \cs{relax} and considers this as undefined.
% This macro is not expandable without \eTeX.
%
% \begin{declcs}{ltx@LocalExpandAfter}
% \end{declcs}
% It expands the token after the next token but in a local
% context. That is the difference to \cs{expandafter}.
% The local context discards the side effect of \cs{csname}
% and let the command undefined after the expansion step.
%
% \subsection{Stripping}
%
% \begin{declcs}{ltx@RemovePrefix}\\
%   \cs{ltx@StripPrefix}
% \end{declcs}
% All tokens up to and including the next available character
% `\textgreater' are thrown away. Usually it is used to strip
% the first part of the output of the commands \cs{meaning}
% or \cs{pdflastmatch}. Macro \cs{ltx@RemovePrefix}
% has the same meaning as \LaTeX's \cs{strip@prefix}, whereas
% macro \cs{ltx@StripPrefix} expands the next token
% once before stripping the prefix.
%
% \begin{declcs}{ltx@onelevel@sanitize} \M{macro}
% \end{declcs}
% Macro \cs{ltx@onelevel@sanitize} provides \hologo{LaTeX}'s
% \cs{@onelevel@sanitize}. The macro is expanded once and
% the contents is converted to characters with catcode 12 (other)
% and space tokens with catcode 10 (space). Then then sanitized
% contents is stored into the macro again. Since version 1.12.
%
% \subsection{File management}
%
% All macros in this section are expandable like
% the counterparts of the \LaTeX\ kernel. Also they
% can be used after the preamble.
%
% \subsubsection{File extensions}
%
% \begin{declcs}{ltx@clsextension}\\
%   \cs{ltx@pkgextension}
% \end{declcs}
% Macros \cs{ltx@clsextension} and \cs{ltx@styextension}
% stores the strings |cls| and |sty|. In opposite to \hologo{LaTeX}'s
% \cs{@clsextension} and \cs{@styextension} they can also be
% used after |\begin{document}|.
%
% \subsubsection{Load check}
%
% \begin{declcs}{ltx@ifclassloaded} \M{class} \M{yes} \M{no}\\
%   \cs{ltx@ifpackageloaded} \M{package} \M{yes} \M{no}
% \end{declcs}
% Macros \cs{ltx@ifclassloaded}\slash\cs{ltx@ifpackageloaded}
% execute \meta{yes}, if the \meta{class} or \meta{package} is loaded,
% otherwise \meta{no} is called.
% Both \meta{class} and \meta{package} are specified without
% extension. The macros can also be used after |\begin{document}|.
%
% \begin{declcs}{ltx@iffileloaded} \M{file} \M{yes} \M{no}
% \end{declcs}
% If \LaTeX's \cs{ProvidesFile} macro was called before
% using \meta{file} as argument, then \cs{ltx@iffileloaded}
% calls \meta{yes}, otherwise \meta{no}.
% Therefore it is possible
% that the \meta{file} is loaded, but \meta{no} is executed because
% of a missing \cs{ProvidesFile}. The \LaTeX\ kernel does not have
% a counterpart of \cs{ltx@iffileloaded}.
%
% Note that the file name used in \cs{ProvidesFile} and
% \cs{ltx@iffileloaded} must match. For example, if \TeX's
% default extension \xext{tex} was given in the first command,
% then it must also specified in the latter command and vice versa.
%
% \subsubsection{Version date check}
%
% \begin{declcs}{ltx@ifclasslater} \M{class} \M{date} \M{yes} \M{no}\\
%   \cs{ltx@ifpackagelater} \M{package} \M{date} \M{yes} \M{no}\\
%   \cs{ltx@iffilelater} \M{file} \M{date} \M{yes} \M{no}
% \end{declcs}
% If a \cs{ProvidesClass}\slash\cs{ProvidesPackage}\slash
% \cs{ProvidesFile} command with exact the same class\slash package\slash
% file was executed before with an optional
% argument that starts with a \LaTeX\ version date, then
% this version date is compared with the argument \meta{date}.
% If they are equal or if the version date is the later date,
% then \meta{yes} is called. In all other cases \meta{no} is executed.
%
% A \LaTeX\ date has the format |YYYY/MM/DD| with |YYYY| as year
% with four digits, |MM| as month with two digits and |DD| as day
% with two digits. If \pdfTeX's \cs{pdfmatch} is available, then
% it is used to detect the version date, to reject invalid
% date formats and to reject some invalid dates.
% Dates before 1994/01/01 are always invalid, because version
% dates are introduced with \LaTeXe\ in 1994.
%
% \subsection{Macro additions}
%
% \begin{declcs}{ltx@GlobalAppendToMacro} \M{cmd} \M{addition}\\
%   \cs{ltx@LocalAppendToMacro} \M{cmd} \M{addition}
% \end{declcs}
% The \meta{addition} is appended to the parameterless macro \meta{cmd}.
% If \meta{cmd} is undefined or has the meaning \cs{relax}, then
% it will be initialized as empty macro beforehand. Due to a bug \meta{addition}
% must not contain \cs{par} before version 2010/10/25 v1.9.
%
% \begin{declcs}{ltx@GlobalPrependToMacro} \M{cmd} \M{addition}\\
%   \cs{ltx@LocalPrependToMacro} \M{cmd} \M{addition}
% \end{declcs}
% The \meta{addition} is prepended to the parameterless macro \meta{cmd}.
% If \meta{cmd} is undefined or has the meaning \cs{relax}, then
% it will be initialized as empty macro beforehand. The macros
% were added in version 2011/08/22 v1.21.
%
% \subsection{Next character detection}
%
% \begin{declcs}{ltx@ifnextchar} \M{char} \M{yes} \M{no}
% \end{declcs}
% If next character is \meta{char} then \meta{yes} is called,
% otherwise \meta{no}. The character is not removed.
% Spaces are silently removed when looking for \meta{char}
% as \hologo{LaTeX}'s version \cs{kernel@ifnextchar} does.
% But there are also small differences:
% \begin{itemize}
% \item The space can be used as \meta{char}. In this case
%   optional spaces before \meta{char} are not supported of course.
% \item If the optional space is a command that is a character
%   (defined by \cs{let} or \cs{futurelet}), then
%   \cs{kernel@ifnextchar} breaks with an \hologo{TeX} error.
%   \cs{ltx@ifnextchar} silently removes this token as
%   optional space.
% \end{itemize}
% Since 2010/03/01 v1.3.
%
% \begin{declcs}{ltx@ifnextchar@nospace} \M{char} \M{yes} \M{no}
% \end{declcs}
% Macro \cs{ltx@ifnextchar@nospace} behaves like
% macro \cs{ltx@ifnextchar} with the exception that
% optional spaces are not supported before \meta{char}.
% Since 2011/04/14 v1.19.
%
% \subsection{\cs{ltx@leavevmode}, \cs{ltx@mbox}}
%
% \begin{declcs}{ltx@leavevmode}
% \end{declcs}
% Macro \cs{ltx@leavevmode} calls \hologo{pdfTeX}'s \cs{quitvmode}.
% Otherwise \cs{leavevmode} is used and defined if it is necessary.
%
% \begin{declcs}{ltx@mbox}
% \end{declcs}
% Macro \cs{ltx@mbox} reimplements \cs{mbox} with two changes.
% Instead of \cs{leavevmode} it uses \cs{ltx@leavevmode} and
% stops right after \cs{hbox}. Especially it does not grab
% the argument and allows the extended syntax of \cs{hbox}.
%
% \subsection{Expandable test for emptiness}
%
% \begin{declcs}{ltx@ifempty} \M{stuff} \M{yes} \M{no}
% \end{declcs}
% Macro \cs{ltx@ifempty} checks in exact two expansion steps
% whether \meta{stuff}
% is empty or contains token. Depending on the result
% \meta{yes} or \meta{no} is executed. The token in \meta{stuff}
% may contain \cs{par} and unmatched conditionals (\cs{if}, \cs{else},
% \cs{fi}, \dots). Since version 2010/11/12 v1.11.
%
% \begin{declcs}{ltx@ifblank} \M{stuff} \M{yes} \M{no}
% \end{declcs}
% Macro \cs{ltx@ifblank} tests in exact two expansion steps
% if \meta{stuff} is empty or contain only blank spaces.
% In this case argument \meta{yes} is called. If \meta{stuff}
% contains other tokens than spaces then \meta{no} is executed.
% Since version 2010/12/04 v1.13.
%
% \subsection{Stripping spaces}
%
% \begin{declcs}{ltx@zapspace} \M{stuff}
% \end{declcs}
% Macro \cs{ltx@zapspace} strips spaces from \meta{stuff}
% that are not hidden inside curly braces. Like \hologo{LaTeX}'s
% \cs{zap@space} it is expandable. Differences:
% \begin{itemize}
% \item
% Syntax: \cs{zap@space} also expects a space token
% and \cs{@empty} after \meta{stuff}.
% \item
% Macro \cs{ltx@zapspace} is expandable in exact two
% expansion steps.
% \item
% Macro \cs{ltx@zapspace} always retains curly braces.
% \item
% Macro \cs{zap@space} has a bug. It stops stripping spaces
% after a token group in curly braces if the first two tokens
% inside the group are equal.
% \item
% Macro \cs{ltx@zapspace} also works with \cs{par}
% and conditionals (\cs{if}, \cs{else}, \cs{fi}, \dots).
% \end{itemize}
% Macro \cs{ltx@zapspace} is available since version
% 2010/12/07 v1.14.
%
% \subsection{Check for emptiness of boxes}
%
% \begin{declcs}{ltx@IfBoxEmpty} \M{box register number}
%   \M{yes} \M{no}
% \end{declcs}
% Macro \cs{ltx@IfBoxEmpty} calls \meta{yes} if
% the box exists (\cs{ifvoid} returns false) and the box
% does not contain any content. Otherwise if the box
% is void or contains something, then \meta{no} is executed.
% Thus being empty means that the box exists and is either
% an \cs{hbox} or a \cs{vbox} and may even have dimensions other than
% 0.0\,pt, but the box does not contain anything.
% Macro \cs{ltx@IfBoxEmpty} is available since 2010/02/04 v1.16.
%
% \begin{declcs}{ltx@IfBoxVoidOrEmpty} \M{box register number}
%   \M{yes} \M{no}
% \end{declcs}
% Macro \cs{ltx@IfBoxVoidOrEmpty} calls \meta{yes} if
% the box is either void or does not contain any content.
% Otherwise \meta{no} is executed.
% Macro \cs{ltx@IfBoxVoidOrEmpty} is available since 2010/02/04 v1.16.
%
% \StopEventually{
% }
%
% \section{Implementation}
%
% \subsection{Identification}
%
%    \begin{macrocode}
%<*package>
%    \end{macrocode}
%    Reload check, especially if the package is not used with \LaTeX.
%    \begin{macrocode}
\begingroup\catcode61\catcode48\catcode32=10\relax%
  \catcode13=5 % ^^M
  \endlinechar=13 %
  \catcode35=6 % #
  \catcode39=12 % '
  \catcode44=12 % ,
  \catcode45=12 % -
  \catcode46=12 % .
  \catcode58=12 % :
  \catcode64=11 % @
  \catcode123=1 % {
  \catcode125=2 % }
  \expandafter\let\expandafter\x\csname ver@ltxcmds.sty\endcsname
  \ifx\x\relax % plain-TeX, first loading
  \else
    \def\empty{}%
    \ifx\x\empty % LaTeX, first loading,
      % variable is initialized, but \ProvidesPackage not yet seen
    \else
      \expandafter\ifx\csname PackageInfo\endcsname\relax
        \def\x#1#2{%
          \immediate\write-1{Package #1 Info: #2.}%
        }%
      \else
        \def\x#1#2{\PackageInfo{#1}{#2, stopped}}%
      \fi
      \x{ltxcmds}{The package is already loaded}%
      \aftergroup\endinput
    \fi
  \fi
\endgroup%
%    \end{macrocode}
%    Package identification:
%    \begin{macrocode}
\begingroup\catcode61\catcode48\catcode32=10\relax%
  \catcode13=5 % ^^M
  \endlinechar=13 %
  \catcode35=6 % #
  \catcode39=12 % '
  \catcode40=12 % (
  \catcode41=12 % )
  \catcode44=12 % ,
  \catcode45=12 % -
  \catcode46=12 % .
  \catcode47=12 % /
  \catcode58=12 % :
  \catcode64=11 % @
  \catcode91=12 % [
  \catcode93=12 % ]
  \catcode123=1 % {
  \catcode125=2 % }
  \expandafter\ifx\csname ProvidesPackage\endcsname\relax
    \def\x#1#2#3[#4]{\endgroup
      \immediate\write-1{Package: #3 #4}%
      \xdef#1{#4}%
    }%
  \else
    \def\x#1#2[#3]{\endgroup
      #2[{#3}]%
      \ifx#1\@undefined
        \xdef#1{#3}%
      \fi
      \ifx#1\relax
        \xdef#1{#3}%
      \fi
    }%
  \fi
\expandafter\x\csname ver@ltxcmds.sty\endcsname
\ProvidesPackage{ltxcmds}%
  [2019/12/15 v1.24 LaTeX kernel commands for general use (HO)]%
%    \end{macrocode}
%
%    \begin{macrocode}
\begingroup\catcode61\catcode48\catcode32=10\relax%
  \catcode13=5 % ^^M
  \endlinechar=13 %
  \catcode123=1 % {
  \catcode125=2 % }
  \catcode64=11 % @
  \def\x{\endgroup
    \expandafter\edef\csname LTXcmds@AtEnd\endcsname{%
      \endlinechar=\the\endlinechar\relax
      \catcode13=\the\catcode13\relax
      \catcode32=\the\catcode32\relax
      \catcode35=\the\catcode35\relax
      \catcode61=\the\catcode61\relax
      \catcode64=\the\catcode64\relax
      \catcode123=\the\catcode123\relax
      \catcode125=\the\catcode125\relax
    }%
  }%
\x\catcode61\catcode48\catcode32=10\relax%
\catcode13=5 % ^^M
\endlinechar=13 %
\catcode35=6 % #
\catcode64=11 % @
\catcode123=1 % {
\catcode125=2 % }
\def\TMP@EnsureCode#1#2{%
  \edef\LTXcmds@AtEnd{%
    \LTXcmds@AtEnd
    \catcode#1=\the\catcode#1\relax
  }%
  \catcode#1=#2\relax
}
\TMP@EnsureCode{36}{3}% $
\TMP@EnsureCode{38}{4}% &
\TMP@EnsureCode{40}{12}% (
\TMP@EnsureCode{41}{12}% )
\TMP@EnsureCode{45}{12}% -
\TMP@EnsureCode{46}{12}% .
\TMP@EnsureCode{47}{12}% /
\TMP@EnsureCode{60}{12}% <
\TMP@EnsureCode{62}{12}% >
\TMP@EnsureCode{91}{12}% [
\TMP@EnsureCode{96}{12}% `
\TMP@EnsureCode{93}{12}% ]
\TMP@EnsureCode{94}{12}% ^ (superscript) (!)
\TMP@EnsureCode{124}{12}% |
\edef\LTXcmds@AtEnd{\LTXcmds@AtEnd\noexpand\endinput}
%    \end{macrocode}
%
% \subsection{Numbers}
%
%    \begin{macro}{\ltx@zero}
%    \begin{macrocode}
\chardef\ltx@zero=0 %
%    \end{macrocode}
%    \end{macro}
%    \begin{macro}{\ltx@one}
%    \begin{macrocode}
\chardef\ltx@one=1 %
%    \end{macrocode}
%    \end{macro}
%    \begin{macro}{\ltx@two}
%    \begin{macrocode}
\chardef\ltx@two=2 %
%    \end{macrocode}
%    \end{macro}
%    \begin{macro}{\ltx@active}
%    \begin{macrocode}
\chardef\ltx@active=13 %
%    \end{macrocode}
%    \end{macro}
%    \begin{macro}{\ltx@cclv}
%    \begin{macrocode}
\chardef\ltx@cclv=255 %
%    \end{macrocode}
%    \end{macro}
%
%    \begin{macro}{\ltx@minusone}
%    \begin{macrocode}
\def\ltx@minusone{%
  -\ltx@one
}
%    \end{macrocode}
%    \end{macro}
%
% \subsection{Scratch registers}
%
%    \begin{macro}{\ltx@LocToksA}
%    \begin{macrocode}
\toksdef\ltx@LocToksA=0 %
%    \end{macrocode}
%    \end{macro}
%    \begin{macro}{\ltx@LocToksB}
%    \begin{macrocode}
\toksdef\ltx@LocToksB=2 %
%    \end{macrocode}
%    \end{macro}
%    \begin{macro}{\ltx@LocToksC}
%    \begin{macrocode}
\toksdef\ltx@LocToksC=4 %
%    \end{macrocode}
%    \end{macro}
%    \begin{macro}{\ltx@LocToksD}
%    \begin{macrocode}
\toksdef\ltx@LocToksD=6 %
%    \end{macrocode}
%    \end{macro}
%    \begin{macro}{\ltx@LocToksE}
%    \begin{macrocode}
\toksdef\ltx@LocToksE=8 %
%    \end{macrocode}
%    \end{macro}
%
%    \begin{macro}{\ltx@GlobToksA}
%    \begin{macrocode}
\toksdef\ltx@GlobToksA=1 %
%    \end{macrocode}
%    \end{macro}
%    \begin{macro}{\ltx@GlobToksB}
%    \begin{macrocode}
\toksdef\ltx@GlobToksB=3 %
%    \end{macrocode}
%    \end{macro}
%    \begin{macro}{\ltx@GlobToksC}
%    \begin{macrocode}
\toksdef\ltx@GlobToksC=5 %
%    \end{macrocode}
%    \end{macro}
%    \begin{macro}{\ltx@GlobToksD}
%    \begin{macrocode}
\toksdef\ltx@GlobToksD=7 %
%    \end{macrocode}
%    \end{macro}
%    \begin{macro}{\ltx@GlobToksE}
%    \begin{macrocode}
\toksdef\ltx@GlobToksE=9 %
%    \end{macrocode}
%    \end{macro}
%
%    \begin{macro}{\ltx@LocDimenA}
%    \begin{macrocode}
\dimendef\ltx@LocDimenA=0 %
%    \end{macrocode}
%    \end{macro}
%    \begin{macro}{\ltx@LocDimenB}
%    \begin{macrocode}
\dimendef\ltx@LocDimenB=2 %
%    \end{macrocode}
%    \end{macro}
%    \begin{macro}{\ltx@LocDimenC}
%    \begin{macrocode}
\dimendef\ltx@LocDimenC=4 %
%    \end{macrocode}
%    \end{macro}
%    \begin{macro}{\ltx@LocDimenD}
%    \begin{macrocode}
\dimendef\ltx@LocDimenD=6 %
%    \end{macrocode}
%    \end{macro}
%    \begin{macro}{\ltx@LocDimenE}
%    \begin{macrocode}
\dimendef\ltx@LocDimenE=8 %
%    \end{macrocode}
%    \end{macro}
%
%    \begin{macro}{\ltx@GlobDimenA}
%    \begin{macrocode}
\dimendef\ltx@GlobDimenA=1 %
%    \end{macrocode}
%    \end{macro}
%    \begin{macro}{\ltx@GlobDimenB}
%    \begin{macrocode}
\dimendef\ltx@GlobDimenB=3 %
%    \end{macrocode}
%    \end{macro}
%    \begin{macro}{\ltx@GlobDimenC}
%    \begin{macrocode}
\dimendef\ltx@GlobDimenC=5 %
%    \end{macrocode}
%    \end{macro}
%    \begin{macro}{\ltx@GlobDimenD}
%    \begin{macrocode}
\dimendef\ltx@GlobDimenD=7 %
%    \end{macrocode}
%    \end{macro}
%    \begin{macro}{\ltx@GlobDimenE}
%    \begin{macrocode}
\dimendef\ltx@GlobDimenE=9 %
%    \end{macrocode}
%    \end{macro}
%
%    \begin{macro}{\ltx@LocSkipA}
%    \begin{macrocode}
\skipdef\ltx@LocSkipA=0 %
%    \end{macrocode}
%    \end{macro}
%    \begin{macro}{\ltx@LocSkipB}
%    \begin{macrocode}
\skipdef\ltx@LocSkipB=2 %
%    \end{macrocode}
%    \end{macro}
%    \begin{macro}{\ltx@LocSkipC}
%    \begin{macrocode}
\skipdef\ltx@LocSkipC=4 %
%    \end{macrocode}
%    \end{macro}
%    \begin{macro}{\ltx@LocSkipD}
%    \begin{macrocode}
\skipdef\ltx@LocSkipD=6 %
%    \end{macrocode}
%    \end{macro}
%    \begin{macro}{\ltx@LocSkipE}
%    \begin{macrocode}
\skipdef\ltx@LocSkipE=8 %
%    \end{macrocode}
%    \end{macro}
%
%    \begin{macro}{\ltx@GlobSkipA}
%    \begin{macrocode}
\skipdef\ltx@GlobSkipA=1 %
%    \end{macrocode}
%    \end{macro}
%    \begin{macro}{\ltx@GlobSkipB}
%    \begin{macrocode}
\skipdef\ltx@GlobSkipB=3 %
%    \end{macrocode}
%    \end{macro}
%    \begin{macro}{\ltx@GlobSkipC}
%    \begin{macrocode}
\skipdef\ltx@GlobSkipC=5 %
%    \end{macrocode}
%    \end{macro}
%    \begin{macro}{\ltx@GlobSkipD}
%    \begin{macrocode}
\skipdef\ltx@GlobSkipD=7 %
%    \end{macrocode}
%    \end{macro}
%    \begin{macro}{\ltx@GlobSkipE}
%    \begin{macrocode}
\skipdef\ltx@GlobSkipE=9 %
%    \end{macrocode}
%    \end{macro}
%
%
% \subsection{Argument killers}
%
%    \begin{macro}{\ltx@gobble}
%    \begin{macrocode}
\long\def\ltx@gobble#1{}
%    \end{macrocode}
%    \end{macro}
%    \begin{macro}{\ltx@gobbletwo}
%    \begin{macrocode}
\long\def\ltx@gobbletwo#1#2{}
%    \end{macrocode}
%    \end{macro}
%    \begin{macro}{\ltx@gobblethree}
%    \begin{macrocode}
\long\def\ltx@gobblethree#1#2#3{}
%    \end{macrocode}
%    \end{macro}
%    \begin{macro}{\ltx@gobblefour}
%    \begin{macrocode}
\long\def\ltx@gobblefour#1#2#3#4{}
%    \end{macrocode}
%    \end{macro}
%
%    \begin{macro}{\ltx@GobbleNum}
%    \begin{macrocode}
\def\ltx@GobbleNum#1{%
  \romannumeral
  \csname ltx@zero%
  \expandafter\LTXcmds@GobbleNum
  \romannumeral\LTXcmds@num{#1}000{m\endcsname}%
}
%    \end{macrocode}
%    \end{macro}
%    \begin{macro}{\LTXcmds@GobbleNum}
%    \begin{macrocode}
\def\LTXcmds@GobbleNum#1{%
  \csname LTXcmds@G#1\LTXcmds@GobbleNum
}
%    \end{macrocode}
%    \end{macro}
%    \begin{macro}{\LTXcmds@Gm}
%    \begin{macrocode}
\long\def\LTXcmds@Gm#1{%
  \endcsname
}
%    \end{macrocode}
%    \end{macro}
%
% \subsection{Argument grabbers}
%
%    \begin{macro}{\ltx@firstofone}
%    \begin{macrocode}
\long\def\ltx@firstofone#1{#1}
%    \end{macrocode}
%    \end{macro}
%    \begin{macro}{\ltx@firstoftwo}
%    \begin{macrocode}
\long\def\ltx@firstoftwo#1#2{#1}
%    \end{macrocode}
%    \end{macro}
%    \begin{macro}{\ltx@secondoftwo}
%    \begin{macrocode}
\long\def\ltx@secondoftwo#1#2{#2}
%    \end{macrocode}
%    \end{macro}
%
%    \begin{macro}{\ltx@firstofthree}
%    \begin{macrocode}
\long\def\ltx@firstofthree#1#2#3{#1}
%    \end{macrocode}
%    \end{macro}
%    \begin{macro}{\ltx@secondofthree}
%    \begin{macrocode}
\long\def\ltx@secondofthree#1#2#3{#2}
%    \end{macrocode}
%    \end{macro}
%    \begin{macro}{\ltx@thirdofthree}
%    \begin{macrocode}
\long\def\ltx@thirdofthree#1#2#3{#3}%
%    \end{macrocode}
%    \end{macro}
%
%    \begin{macro}{\ltx@firstoffour}
%    \begin{macrocode}
\long\def\ltx@firstoffour#1#2#3#4{#1}
%    \end{macrocode}
%    \end{macro}
%    \begin{macro}{\ltx@secondoffour}
%    \begin{macrocode}
\long\def\ltx@secondoffour#1#2#3#4{#2}
%    \end{macrocode}
%    \end{macro}
%    \begin{macro}{\ltx@thirdoffour}
%    \begin{macrocode}
\long\def\ltx@thirdoffour#1#2#3#4{#3}%
%    \end{macrocode}
%    \end{macro}
%    \begin{macro}{\ltx@fourthoffour}
%    \begin{macrocode}
\long\def\ltx@fourthoffour#1#2#3#4{#4}%
%    \end{macrocode}
%    \end{macro}
%
% \subsection{List helpers}
%
%    \begin{macro}{\ltx@carzero}
%    \begin{macrocode}
\long\def\ltx@carzero#1\@nil{}%
%    \end{macrocode}
%    \end{macro}
%    \begin{macro}{\LTXcmds@cdrzero}
%    \begin{macrocode}
\long\def\LTXcmds@cdrzero#1\@nil{#1}
%    \end{macrocode}
%    \end{macro}
%    \begin{macro}{\ltx@cdrzero}
%    \begin{macrocode}
\def\ltx@cdrzero{%
  \romannumeral\LTXcmds@cdrzero\ltx@zero
}
%    \end{macrocode}
%    \end{macro}
%
%    \begin{macro}{\ltx@car}
%    \begin{macrocode}
\long\def\ltx@car#1#2\@nil{#1}
%    \end{macrocode}
%    \end{macro}
%    \begin{macro}{\ltx@cdr}
%    \begin{macrocode}
\long\def\ltx@cdr#1{%
  \romannumeral\LTXcmds@cdrzero\ltx@zero
}
%    \end{macrocode}
%    \end{macro}
%
%    \begin{macro}{\ltx@cartwo}
%    \begin{macrocode}
\long\def\ltx@cartwo#1#2#3\@nil{#1#2}
%    \end{macrocode}
%    \end{macro}
%    \begin{macro}{\ltx@carsecond}
%    \begin{macrocode}
\long\def\ltx@carsecond#1#2#3\@nil{#2}
%    \end{macrocode}
%    \end{macro}
%    \begin{macro}{\ltx@cdrtwo}
%    \begin{macrocode}
\long\def\ltx@cdrtwo#1#2{%
  \romannumeral\LTXcmds@cdrzero\ltx@zero
}
%    \end{macrocode}
%    \end{macro}
%
%    \begin{macro}{\ltx@carthree}
%    \begin{macrocode}
\long\def\ltx@carthree#1#2#3#4\@nil{#1#2#3}
%    \end{macrocode}
%    \end{macro}
%    \begin{macro}{\ltx@carthird}
%    \begin{macrocode}
\long\def\ltx@carthird#1#2#3#4\@nil{#3}
%    \end{macrocode}
%    \end{macro}
%    \begin{macro}{\ltx@cdrthree}
%    \begin{macrocode}
\long\def\ltx@cdrthree#1#2#3{%
  \romannumeral\LTXcmds@cdrzero\ltx@zero
}
%    \end{macrocode}
%    \end{macro}
%
%    \begin{macro}{\ltx@carfour}
%    \begin{macrocode}
\long\def\ltx@carfour#1#2#3#4#5\@nil{#1#2#3#4}
%    \end{macrocode}
%    \end{macro}
%    \begin{macro}{\ltx@carfourth}
%    \begin{macrocode}
\long\def\ltx@carfourth#1#2#3#4#5\@nil{#4}
%    \end{macrocode}
%    \end{macro}
%    \begin{macro}{\ltx@cdrfour}
%    \begin{macrocode}
\long\def\ltx@cdrfour#1#2#3#4{%
  \romannumeral\LTXcmds@cdrzero\ltx@zero
}
%    \end{macrocode}
%    \end{macro}
%
%    \begin{macro}{\ltx@CarNum}
%    \begin{macrocode}
\def\ltx@CarNum#1{%
  \romannumeral
  \csname LTXcmds@CarNumFinish%
  \expandafter\LTXcmds@CarNum
  \romannumeral\LTXcmds@num{#1}000{x\endcsname}%
}
%    \end{macrocode}
%    \end{macro}
%    \begin{macro}{\LTXcmds@CarNum}
%    \begin{macrocode}
\def\LTXcmds@CarNum#1{%
  \csname LTXcmds@C#1\LTXcmds@CarNum
}
%    \end{macrocode}
%    \end{macro}
%    \begin{macro}{\LTXcmds@Cm}
%    \begin{macrocode}
\long\def\LTXcmds@Cm#1#2{%
  \endcsname{#1#2}%
}
%    \end{macrocode}
%    \end{macro}
%    \begin{macro}{\LTXcmds@Cx}
%    \begin{macrocode}
\def\LTXcmds@Cx#1{%
  \endcsname{}%
}
%    \end{macrocode}
%    \end{macro}
%    \begin{macro}{\LTXcmds@CarNumFinish}
%    \begin{macrocode}
\long\def\LTXcmds@CarNumFinish#1#2\@nil{%
  \ltx@zero
  #1%
}
%    \end{macrocode}
%    \end{macro}
%
%    \begin{macro}{\ltx@CarNumth}
%    \begin{macrocode}
\def\ltx@CarNumth#1{%
  \romannumeral
  \expandafter\expandafter\expandafter
  \LTXcmds@CarNumth
  \ltx@GobbleNum{#1}{}%
}
%    \end{macrocode}
%    \end{macro}
%    \begin{macro}{\LTXcmds@CarNumth}
%    \begin{macrocode}
\long\def\LTXcmds@CarNumth#1#2\@nil{%
  \ltx@zero
  #1%
}
%    \end{macrocode}
%    \end{macro}
%
%    \begin{macro}{\ltx@CdrNum}
%    \begin{macrocode}
\def\ltx@CdrNum#1{%
  \romannumeral%
  \expandafter\expandafter\expandafter\ltx@cdrzero
  \expandafter\expandafter\expandafter\ltx@zero
  \ltx@GobbleNum{#1}%
}
%    \end{macrocode}
%    \end{macro}
%
% \subsection{Tail recursion}
%
%    \begin{macro}{\ltx@ReturnAfterFi}
%    \begin{macrocode}
\long\def\ltx@ReturnAfterFi#1\fi{\fi#1}
%    \end{macrocode}
%    \end{macro}
%    \begin{macro}{\ltx@ReturnAfterElseFi}
%    \begin{macrocode}
\long\def\ltx@ReturnAfterElseFi#1\else#2\fi{\fi#1}
%    \end{macrocode}
%    \end{macro}
%
% \subsection{Empty macro}
%
%    \begin{macro}{\ltx@empty}
%    \begin{macrocode}
\def\ltx@empty{}
%    \end{macrocode}
%    \end{macro}
%
% \subsection{Characters}
%
%    \begin{macro}{\ltx@space}
%    \begin{macrocode}
\def\ltx@space{ }
%    \end{macrocode}
%    \end{macro}
%    \begin{macro}{\ltx@percentchar}
%    \begin{macrocode}
\begingroup
  \lccode`0=`\%\relax
\lowercase{\endgroup
  \def\ltx@percentchar{0}%
}
%    \end{macrocode}
%    \end{macro}
%    \begin{macro}{\ltx@backslashchar}
%    \begin{macrocode}
\begingroup
  \lccode`0=`\\\relax
\lowercase{\endgroup
  \def\ltx@backslashchar{0}%
}
%    \end{macrocode}
%    \end{macro}
%    \begin{macro}{\ltx@hashchar}
%    \begin{macrocode}
\begingroup
  \lccode`0=`\#\relax
\lowercase{\endgroup
  \def\ltx@hashchar{0}%
}
%    \end{macrocode}
%    \end{macro}
%    \begin{macro}{\ltx@leftbracechar}
%    \begin{macrocode}
\begingroup
  \lccode`0=`\{\relax
\lowercase{\endgroup
  \def\ltx@leftbracechar{0}%
}
%    \end{macrocode}
%    \end{macro}
%    \begin{macro}{\ltx@rightbracechar}
%    \begin{macrocode}
\begingroup
  \lccode`0=`\}\relax
\lowercase{\endgroup
  \def\ltx@rightbracechar{0}%
}
%    \end{macrocode}
%    \end{macro}
%
% \subsection{Boolean switch}
%
%    \begin{macro}{\ltx@newif}
%    \begin{macrocode}
\def\ltx@newif#1{%
  \begingroup
    \escapechar=-1 %
  \expandafter\endgroup
  \expandafter\LTXcmds@newif\string#1\@nil
}
%    \end{macrocode}
%    \end{macro}
%    \begin{macro}{\LTXcmds@newif}
%    \begin{macrocode}
\begingroup
  \escapechar=-1 %
\expandafter\endgroup
\expandafter\def\expandafter\LTXcmds@newif\string\if#1\@nil{%
  \expandafter\edef\csname#1true\endcsname{%
    \let
    \expandafter\noexpand\csname if#1\endcsname
    \noexpand\iftrue
  }%
  \expandafter\edef\csname#1false\endcsname{%
    \let
    \expandafter\noexpand\csname if#1\endcsname
    \noexpand\iffalse
  }%
  \csname#1false\endcsname
}
%    \end{macrocode}
%    \end{macro}
%
%    \begin{macro}{\ltx@newglobalif}
%    \begin{macrocode}
\def\ltx@newglobalif#1{%
  \begingroup
    \escapechar=-1 %
  \expandafter\endgroup
  \expandafter\LTXcmds@newglobalif\string#1\@nil
}
%    \end{macrocode}
%    \end{macro}
%    \begin{macro}{\LTXcmds@newglobalif}
%    \begin{macrocode}
\begingroup
  \escapechar=-1 %
\expandafter\endgroup
\expandafter
\def\expandafter\LTXcmds@newglobalif\string\if#1\@nil{%
  \expandafter\edef\csname#1true\endcsname{%
    \global\let
    \expandafter\noexpand\csname if#1\endcsname
    \noexpand\iftrue
  }%
  \expandafter\edef\csname#1false\endcsname{%
    \global\let
    \expandafter\noexpand\csname if#1\endcsname
    \noexpand\iffalse
  }%
  \csname#1false\endcsname
}
%    \end{macrocode}
%    \end{macro}
%
% \subsection{Command definitions}
%
%    \begin{macro}{\ltx@LocalExpandAfter}
%    \begin{macrocode}
\def\ltx@LocalExpandAfter{%
  \begingroup
    \expandafter\expandafter\expandafter
  \endgroup
  \expandafter
}
%    \end{macrocode}
%    \end{macro}
%
%    \begin{macrocode}
\ltx@LocalExpandAfter
\ifx\csname ifcsname\endcsname\relax
%    \end{macrocode}
%    \begin{macro}{\ltx@ifundefined}
%    \begin{macrocode}
  \def\ltx@ifundefined#1{%
    \expandafter\ifx\csname #1\endcsname\relax
      \expandafter\ltx@firstoftwo
    \else
      \expandafter\ltx@secondoftwo
    \fi
  }%
%    \end{macrocode}
%    \end{macro}
%    \begin{macro}{\ltx@IfUndefined}
%    \begin{macrocode}
  \def\ltx@IfUndefined#1{%
    \begingroup\expandafter\expandafter\expandafter\endgroup
    \expandafter\ifx\csname #1\endcsname\relax
      \expandafter\ltx@firstoftwo
    \else
      \expandafter\ltx@secondoftwo
    \fi
  }%
%    \end{macrocode}
%    \end{macro}
%    \begin{macrocode}
  \expandafter\ltx@gobble
\else
  \expandafter\ltx@firstofone
\fi
{%
%    \end{macrocode}
%    \begin{macro}{\ltx@ifundefined}
%    \begin{macrocode}
  \def\ltx@ifundefined#1{%
    \ifcsname #1\endcsname
      \expandafter\ifx\csname #1\endcsname\relax
        \expandafter\expandafter\expandafter\ltx@firstoftwo
      \else
        \expandafter\expandafter\expandafter\ltx@secondoftwo
      \fi
    \else
      \expandafter\ltx@firstoftwo
    \fi
  }%
%    \end{macrocode}
%    \end{macro}
%    \begin{macro}{\ltx@IfUndefined}
%    \begin{macrocode}
  \let\ltx@IfUndefined\ltx@ifundefined
%    \end{macrocode}
%    \end{macro}
%    \begin{macrocode}
}
%    \end{macrocode}
%
% \subsection{Stripping}
%
%    \begin{macro}{\ltx@RemovePrefix}
%    \begin{macrocode}
\def\ltx@RemovePrefix#1>{}
%    \end{macrocode}
%    \end{macro}
%
%    \begin{macro}{\ltx@StripPrefix}
%    \begin{macrocode}
\def\ltx@StripPrefix{%
  \expandafter\ltx@RemovePrefix
}
%    \end{macrocode}
%    \end{macro}
%
%    \begin{macro}{\ltx@onelevel@sanitize}
%    \begin{macrocode}
\def\ltx@onelevel@sanitize#1{%
  \edef#1{%
    \expandafter
    \ltx@RemovePrefix\meaning#1%
  }%
}
%    \end{macrocode}
%    \end{macro}
%
% \subsection{File management}
%
% \subsubsection{File extensions}
%
%    \begin{macro}{\ltx@clsextension}
%    \begin{macrocode}
\def\ltx@clsextension{cls}
%    \end{macrocode}
%    \end{macro}
%    \begin{macro}{\ltx@pkgextension}
%    \begin{macrocode}
\def\ltx@pkgextension{sty}
%    \end{macrocode}
%    \end{macro}
%
% \subsubsection{Load check}
%
%    \begin{macro}{\ltx@iffileloaded}
%    \begin{macrocode}
\def\ltx@iffileloaded#1{%
  \ltx@ifundefined{ver@#1}\ltx@secondoftwo\ltx@firstoftwo
}
%    \end{macrocode}
%    \end{macro}
%    \begin{macro}{\ltx@ifclassloaded}
%    \begin{macrocode}
\def\ltx@ifclassloaded#1{%
  \ltx@iffileloaded{#1.\ltx@clsextension}%
}
%    \end{macrocode}
%    \end{macro}
%    \begin{macro}{\ltx@ifpackageloaded}
%    \begin{macrocode}
\def\ltx@ifpackageloaded#1{%
  \ltx@iffileloaded{#1.\ltx@pkgextension}%
}
%    \end{macrocode}
%    \end{macro}
%
% \subsubsection{Version date check}
%
%    \begin{macro}{\ltx@iffilelater}
%    \begin{macrocode}
\def\ltx@iffilelater#1#2{%
  \ltx@iffileloaded{#1}{%
    \expandafter\LTXcmds@IfLater\expandafter{%
      \number
      \expandafter\expandafter\expandafter\LTXcmds@ParseVersion
      \expandafter\expandafter\expandafter{%
        \csname ver@#1\endcsname
      }%
    \expandafter}\expandafter{%
      \number
      \expandafter\LTXcmds@ParseVersion\expandafter{#2}%
    }%
  }\ltx@secondoftwo
}
%    \end{macrocode}
%    \end{macro}
%    \begin{macro}{\LTXcmds@IfLater}
%    \begin{macrocode}
\def\LTXcmds@IfLater#1#2{%
  \ifcase 0%
      \ifnum#1<19940101 %
      \else
        \ifnum#2<19940101 %
        \else
          \ifnum#2>#1 %
          \else
            1%
          \fi
        \fi
      \fi
      \ltx@space
    \expandafter\ltx@secondoftwo
  \else
    \expandafter\ltx@firstoftwo
  \fi
}
%    \end{macrocode}
%    \end{macro}
%    \begin{macro}{\ltx@ifclasslater}
%    \begin{macrocode}
\def\ltx@ifclasslater#1{%
  \ltx@iffilelater{#1.\ltx@clsextension}%
}
%    \end{macrocode}
%    \end{macro}
%    \begin{macro}{\ltx@ifpackagelater}
%    \begin{macrocode}
\def\ltx@ifpackagelater#1{%
  \ltx@iffilelater{#1.\ltx@pkgextension}%
}
%    \end{macrocode}
%    \end{macro}
%
%    \begin{macrocode}
\ltx@IfUndefined{pdfmatch}{%
%    \end{macrocode}
%    \begin{macro}{\LTXcmds@ParseVersion}
%    \begin{macrocode}
  \def\LTXcmds@ParseVersion#1{%
    \LTXcmds@@ParseVersion#10000/00/00\@nil
  }%
%    \end{macrocode}
%    \end{macro}
%    \begin{macro}{\LTXcmds@@ParseVersion}
%    \begin{macrocode}
  \def\LTXcmds@@ParseVersion#1#2#3#4/#5#6/#7#8#9\@nil{%
    #1#2#3#4#5#6#7#8%
  }%
%    \end{macrocode}
%    \end{macro}
%    \begin{macrocode}
}{%
%    \end{macrocode}
%    \begin{macro}{\LTXcmds@ParseVersion}
%    \begin{macrocode}
  \def\LTXcmds@ParseVersion#1{%
    \ifnum\pdfmatch{%
      ^%
      (199[4-9]|[2-9][0-9][0-9][0-9])/%
      (0[1-9]|1[0-2])/%
      (0[1-9]|[1-2][0-9]|3[0-1])%
    }{#1}=1 %
      \ltx@StripPrefix\pdflastmatch1 %
      \ltx@StripPrefix\pdflastmatch2 %
      \ltx@StripPrefix\pdflastmatch3 %
    \else
      0%
    \fi
  }%
%    \end{macrocode}
%    \end{macro}
%    \begin{macrocode}
}
%    \end{macrocode}
%
% \subsection{Macro additions}
%
%    \begin{macro}{\ltx@GlobalAppendToMacro}
%    \begin{macrocode}
\long\def\ltx@GlobalAppendToMacro#1#2{%
  \ifx\ltx@undefined#1%
    \let#1\ltx@empty
  \else
    \ifx\relax#1%
      \let#1\ltx@empty
    \fi
  \fi
  \begingroup
    \ltx@LocToksA\expandafter{#1#2}%
    \xdef#1{\the\ltx@LocToksA}%
  \endgroup
}
%    \end{macrocode}
%    \end{macro}
%    \begin{macro}{\ltx@LocalAppendToMacro}
%    \begin{macrocode}
\long\def\ltx@LocalAppendToMacro#1#2{%
  \global\let\LTXcmds@gtemp#1%
  \ifx\ltx@undefined\LTXcmds@gtemp
    \global\let\LTXcmds@gtemp\ltx@empty
  \else
    \ifx\relax\LTXcmds@gtemp
      \global\letLTXcmds@gtemp\ltx@empty
    \fi
  \fi
  \begingroup
    \ltx@LocToksA\expandafter{\LTXcmds@gtemp#2}%
    \xdef\LTXcmds@gtemp{\the\ltx@LocToksA}%
  \endgroup
  \let#1\LTXcmds@gtemp
}
%    \end{macrocode}
%    \end{macro}
%
%    \begin{macro}{\ltx@GlobalPrependToMacro}
%    \begin{macrocode}
\long\def\ltx@GlobalPrependToMacro#1#2{%
  \ifx\ltx@undefined#1%
    \let#1\ltx@empty
  \else
    \ifx\relax#1%
      \let#1\ltx@empty
    \fi
  \fi
  \begingroup
    \ltx@LocToksA{#2}%
    \ltx@LocToksB\expandafter{#1}%
    \xdef#1{\the\ltx@LocToksA\the\ltx@LocToksB}%
  \endgroup
}
%    \end{macrocode}
%    \end{macro}
%    \begin{macro}{\ltx@LocalPrependToMacro}
%    \begin{macrocode}
\long\def\ltx@LocalPrependToMacro#1#2{%
  \global\let\LTXcmds@gtemp#1%
  \ifx\ltx@undefined\LTXcmds@gtemp
    \global\let\LTXcmds@gtemp\ltx@empty
  \else
    \ifx\relax\LTXcmds@gtemp
      \global\letLTXcmds@gtemp\ltx@empty
    \fi
  \fi
  \begingroup
    \ltx@LocToksA{#2}%
    \ltx@LocToksB\expandafter{\LTXcmds@gtemp}%
    \xdef\LTXcmds@gtemp{\the\ltx@LocToksA\the\ltx@LocToksB}%
  \endgroup
  \let#1\LTXcmds@gtemp
}
%    \end{macrocode}
%    \end{macro}
%
% \subsection{Next character detection}
%
%    \begin{macro}{\ltx@ifnextchar}
%    \begin{macrocode}
\long\def\ltx@ifnextchar#1#2#3{%
  \begingroup
  \let\LTXcmds@CharToken= #1\relax
  \ltx@LocToksA{\endgroup#2}%
  \ltx@LocToksB{\endgroup#3}%
  \futurelet\LTXcmds@LetToken\LTXcmds@ifnextchar
}
%    \end{macrocode}
%    \end{macro}
%    \begin{macro}{\LTXcmds@ifnextchar}
%    \begin{macrocode}
\def\LTXcmds@ifnextchar{%
  \ifx\LTXcmds@LetToken\LTXcmds@CharToken
    \the\expandafter\ltx@LocToksA
  \else
    \expandafter
      \ifx\csname LTXcmds@LetToken\endcsname\LTXcmds@SpaceToken
      \expandafter\expandafter\expandafter\LTXcmds@@ifnextchar
    \else
      \the\expandafter\expandafter\expandafter\ltx@LocToksB
    \fi
  \fi
}
%    \end{macrocode}
%    \end{macro}
%    \begin{macro}{\LTXcmds@@ifnextchar}
%    \cs{futurelet} does not distinguish between a
%    character and a command that is a character
%    (defined by using \cs{let} or \cs{futurelet}).
%    Therefore the space is catched by \cs{romannumeral}
%    with negative character constant that gobbles one
%    optional space.
%    \begin{macrocode}
\def\LTXcmds@@ifnextchar{%
  \expandafter\futurelet
  \expandafter\LTXcmds@LetToken
  \expandafter\LTXcmds@ifnextchar
  \romannumeral-`\.%
}
%    \end{macrocode}
%    \end{macro}
%    \begin{macro}{\LTXcmds@SpaceToken}
%    \begin{macrocode}
\ltx@firstofone{\let\LTXcmds@SpaceToken= } %
%    \end{macrocode}
%    \end{macro}
%
%    \begin{macro}{\ltx@ifnextchar@nospace}
%    \begin{macrocode}
\long\def\ltx@ifnextchar@nospace#1#2#3{%
  \begingroup
  \let\LTXcmds@CharToken= #1\relax
  \ltx@LocToksA{\endgroup#2}%
  \ltx@LocToksB{\endgroup#3}%
  \futurelet\LTXcmds@LetToken\LTXcmds@ifnextchar@nospace
}
%    \end{macrocode}
%    \end{macro}
%    \begin{macro}{\LTXcmds@ifnextchar@nospace}
%    \begin{macrocode}
\def\LTXcmds@ifnextchar@nospace{%
  \the
  \ifx\LTXcmds@LetToken\LTXcmds@CharToken
    \expandafter\ltx@LocToksA
  \else
    \expandafter\ltx@LocToksB
  \fi
}
%    \end{macrocode}
%    \end{macro}
% \subsection{\cs{ltx@leavevmode}, \cs{ltx@mbox}}
%
%    \begin{macro}{\ltx@leavevmode}
%    \begin{macrocode}
\ltx@IfUndefined{quitvmode}{%
  \ltx@IfUndefined{leavevmode}{%
    \ltx@IfUndefined{voidb@x}{%
      \ltx@IfUndefined{newbox}{%
        \def\ltx@leavevmode{%
          \begingroup
            \setbox\ltx@zero=\hbox{}%
            \begingroup
              \setbox\ltx@zero=\hbox{\box\ltx@zero}%
            \endgroup
            \unhbox\ltx@zero
          \endgroup
        }%
      }{%
        \csname newbox\endcsname\LTXcmds@VoidBox
        \ifvoid\LTXcmds@VoidBox
        \else
          \setbox\LTXcmds@VoidBox=\hbox{}%
          \begingroup
            \setbox\LTXcmds@VoidBox=\hbox{\box\LTXcmds@VoidBox}%
          \endgroup
        \fi
        \def\ltx@leavevmode{\unhbox\LTXcmds@VoidBox}%
      }%
    }{%
      \def\ltx@leavevmode{\unhbox\voidb@x}%
    }%
  }{%
    \let\ltx@leavevmode\leavevmode
  }%
}{%
  \let\ltx@leavevmode\quitvmode
}
%    \end{macrocode}
%    \end{macro}
%
%    \begin{macro}{\ltx@mbox}
%    \begin{macrocode}
\def\ltx@mbox{%
  \ltx@leavevmode
  \hbox
}
%    \end{macrocode}
%    \end{macro}
%
% \subsection{Help macros}
%
%    \begin{macro}{\LTXcmds@num}
%    \begin{macrocode}
\ltx@IfUndefined{numexpr}{%
  \def\LTXcmds@num#1{%
    \expandafter\ltx@firstofone\expandafter{%
      \number#1%
    }%
  }%
}{%
  \def\LTXcmds@num#1{%
    \expandafter\ltx@firstofone\expandafter{%
      \the\numexpr#1%
    }%
  }%
}
%    \end{macrocode}
%    \end{macro}
%
% \subsection{Expandable test for emptiness}
%
%    \begin{macrocode}
\ltx@IfUndefined{detokenize}{%
%    \end{macrocode}
%
% \subsubsection{Vanilla \hologo{TeX}}
%
%    \begin{macro}{\ltx@ifempty}
%    The macro is based on \cs{@ifempty} of Robert R. Schneck
%    \cite{ifempty:schneck}
%    and \cs{@ifnull} of Ulrich Diez \cite{ifnull:diez}.
%    There are three cases to consider:
%    \begin{compactenum}
%    \item |#1| is empty,
%    \item |#1| is not empty and the first token
%      is not a begingroup character,
%    \item |#1| starts with a begingroup character (catcode 1).
%    \end{compactenum}
%    \begin{macrocode}
  \def\LTXcmds@temp#1{%
    \long\def\ltx@ifempty##1{%
      \romannumeral0%
      \iffalse{\fi
        \expandafter\ltx@gobble\expandafter{%
          \expandafter{\string##1}%
          \expandafter\ltx@gobble\string
        }%
        \expandafter\ltx@firstofthree\expandafter
        {\iffalse}\fi
        \expandafter#1\ltx@secondoftwo
      }%
      \expandafter#1\ltx@firstoftwo
    }%
%    \end{macrocode}
%    \end{macro}
%    \begin{macro}{\ltx@ifblank}
%    \begin{macrocode}
    \long\def\ltx@ifblank##1{%
      \romannumeral0%
      \iffalse{\fi
        \expandafter\expandafter\expandafter\ltx@gobble
        \expandafter\expandafter\expandafter{%
          \expandafter\expandafter\expandafter{%
            \expandafter\string\ltx@gobble##1.%
          }%
          \expandafter\ltx@gobble\string
        }%
        \expandafter\ltx@firstofthree\expandafter
        {\iffalse}\fi
        \expandafter#1\ltx@secondoftwo
      }%
      \expandafter#1\ltx@firstoftwo
    }%
  }%
  \LTXcmds@temp{ }%
%    \end{macrocode}
%    \end{macro}
%    \begin{macrocode}
}{%
%    \end{macrocode}
%
% \subsubsection{With \cs{detokenize}}
%
%    Ahmed Musa provided \cs{ifstrempty} using \cs{detokenize}
%    and \cs{pdfstrcmp} \cite{ifstrempty:musa}. Ulrich Diez, GL,
%    Heiko Oberdiek
%    improved it further by removing \cs{pdfstrcmp} and
%    taking three arguments
%    \cite{ifstrempty:diez1,ifstrempty:diez2,ifstrempty:diez3,^^A
%      ifstrempty:gl,ifstrempty:ho}.
%
%    \begin{macro}{\ltx@ifempty}
%    \begin{macrocode}
  \long\def\ltx@ifempty#1{%
    \romannumeral%
    \csname
      LTXcmds@ifempty%
      \ifcat$\detokenize{#1}$%
        @%
      \fi
    \endcsname
  }%
%    \end{macrocode}
%    \end{macro}
%    \begin{macro}{\LTXcmds@ifempty@}
%    \begin{macrocode}
  \long\def\LTXcmds@ifempty@#1#2{0 #1}%
%    \end{macrocode}
%    \end{macro}
%    \begin{macro}{\LTXcmds@ifempty}
%    \begin{macrocode}
  \long\def\LTXcmds@ifempty#1#2{0 #2}%
%    \end{macrocode}
%    \end{macro}
%
% \subsubsection{\cs{ltx@ifblank}}
%
%    \begin{macro}{\ltx@ifblank}
%    \begin{macrocode}
  \long\def\ltx@ifblank#1{%
    \romannumeral%
    \csname
      LTXcmds@ifempty%
      \ifcat$\detokenize\expandafter{\ltx@gobble#1.}$%
        @%
      \fi
    \endcsname
  }%
%    \end{macrocode}
%    \end{macro}
%
%    \begin{macrocode}
}
%    \end{macrocode}
%
% \subsection{\cs{ltx@zapspace}}
%
%    \begin{macro}{\ltx@zapspace}
%    \begin{macrocode}
\long\def\ltx@zapspace#1{%
  \romannumeral
  \LTXcmds@zapspace\ltx@zero#1 \@nil
}
%    \end{macrocode}
%    \end{macro}
%    \begin{macro}{\LTXcmds@zapspace}
%    \begin{macrocode}
\long\def\LTXcmds@zapspace#1 #2\@nil{%
  \ltx@ifempty{#2}{%
    #1%
  }{%
    \LTXcmds@zapspace#1#2\@nil
  }%
}
%    \end{macrocode}
%    \end{macro}
%
% \subsection{\cs{ltx@IfBoxEmpty}}
%
%    In case of \hologo{eTeX} the test for an empty box
%    is done via \cs{lastnodetype} as suggested
%    by David Kastrup \cite{ifboxempty:dak}.
%    \begin{macrocode}
\ltx@IfUndefined{lastnodetype}{%
  \catcode`\$=9 %
  \catcode`\&=14 %
}{%
  \catcode`\$=14 %
  \catcode`\&=9 %
}
%    \end{macrocode}
%    \begin{macro}{\ltx@IfBoxEmpty}
%    \begin{macrocode}
\def\ltx@IfBoxEmpty#1{%
  \ifvoid#1\relax
    \expandafter\ltx@secondoftwo
  \else
%    \end{macrocode}
%    Implementation using \hologo{eTeX}'s \cs{lastnodetype}.
%    \begin{macrocode}
&   \begingroup
&     \setbox\ltx@zero=\ifhbox#1\hbox\else\vbox\fi{%
&       \ifhmode\unhcopy\else\unvcopy\fi#1\relax
&       \expandafter
&     }%
&   \expandafter\endgroup
&   \ifnum\lastnodetype<\ltx@zero
&     \expandafter\expandafter\expandafter\ltx@firstoftwo
&   \else
&     \expandafter\expandafter\expandafter\ltx@secondoftwo
&   \fi
%    \end{macrocode}
%    Implementation without \hologo{eTeX} using a signature
%    at the beginning of the test box.
%    \begin{macrocode}
$   \begingroup
$     \setbox\ltx@zero=\ifhbox#1\hbox\else\vbox\fi{%
$       \penalty\ltx@one
$       \ifhmode\unhcopy\else\unvcopy\fi#1\relax
$       \expandafter
$     }%
$     \ifnum\lastpenalty=\ltx@one
%    \end{macrocode}
%    Box 0 has been changed and is restored by
%    closing the group.
%    \begin{macrocode}
$       \endgroup
$       \begingroup
$       \setbox\ltx@zero=\ifhbox#1\hbox\else\vbox\fi{%
$         \penalty\ltx@two
$         \ifhmode\unhcopy\else\unvcopy\fi#1\relax
$         \expandafter
$       }%
$       \ifnum\lastpenalty=\ltx@two
$         \def\next{\endgroup\expandafter\ltx@firstoftwo}%
$       \else
$         \def\next{\endgroup\expandafter\ltx@secondoftwo}%
$       \fi
$     \else
$       \def\next{\endgroup\expandafter\ltx@secondoftwo}%
$     \fi
$   \next
  \fi
}
%    \end{macrocode}
%    \end{macro}
%
%    \begin{macro}{\ltx@IfBoxVoidOrEmpty}
%    \begin{macrocode}
\def\ltx@IfBoxVoidOrEmpty#1{%
  \ifvoid#1\relax
    \expandafter\ltx@thirdoffour
  \fi
  \ltx@IfBoxEmpty{#1}%
}
%    \end{macrocode}
%    \end{macro}
%
%    \begin{macrocode}
\LTXcmds@AtEnd%
%</package>
%    \end{macrocode}
%% \section{Installation}
%
% \subsection{Download}
%
% \paragraph{Package.} This package is available on
% CTAN\footnote{\CTANpkg{ltxcmds}}:
% \begin{description}
% \item[\CTAN{macros/latex/contrib/ltxcmds/ltxcmds.dtx}] The source file.
% \item[\CTAN{macros/latex/contrib/ltxcmds/ltxcmds.pdf}] Documentation.
% \end{description}
%
%
% \paragraph{Bundle.} All the packages of the bundle `ltxcmds'
% are also available in a TDS compliant ZIP archive. There
% the packages are already unpacked and the documentation files
% are generated. The files and directories obey the TDS standard.
% \begin{description}
% \item[\CTANinstall{install/macros/latex/contrib/ltxcmds.tds.zip}]
% \end{description}
% \emph{TDS} refers to the standard ``A Directory Structure
% for \TeX\ Files'' (\CTANpkg{tds}). Directories
% with \xfile{texmf} in their name are usually organized this way.
%
% \subsection{Bundle installation}
%
% \paragraph{Unpacking.} Unpack the \xfile{ltxcmds.tds.zip} in the
% TDS tree (also known as \xfile{texmf} tree) of your choice.
% Example (linux):
% \begin{quote}
%   |unzip ltxcmds.tds.zip -d ~/texmf|
% \end{quote}
%
% \subsection{Package installation}
%
% \paragraph{Unpacking.} The \xfile{.dtx} file is a self-extracting
% \docstrip\ archive. The files are extracted by running the
% \xfile{.dtx} through \plainTeX:
% \begin{quote}
%   \verb|tex ltxcmds.dtx|
% \end{quote}
%
% \paragraph{TDS.} Now the different files must be moved into
% the different directories in your installation TDS tree
% (also known as \xfile{texmf} tree):
% \begin{quote}
% \def\t{^^A
% \begin{tabular}{@{}>{\ttfamily}l@{ $\rightarrow$ }>{\ttfamily}l@{}}
%   ltxcmds.sty & tex/generic/ltxcmds/ltxcmds.sty\\
%   ltxcmds.pdf & doc/latex/ltxcmds/ltxcmds.pdf\\
%   ltxcmds.dtx & source/latex/ltxcmds/ltxcmds.dtx\\
% \end{tabular}^^A
% }^^A
% \sbox0{\t}^^A
% \ifdim\wd0>\linewidth
%   \begingroup
%     \advance\linewidth by\leftmargin
%     \advance\linewidth by\rightmargin
%   \edef\x{\endgroup
%     \def\noexpand\lw{\the\linewidth}^^A
%   }\x
%   \def\lwbox{^^A
%     \leavevmode
%     \hbox to \linewidth{^^A
%       \kern-\leftmargin\relax
%       \hss
%       \usebox0
%       \hss
%       \kern-\rightmargin\relax
%     }^^A
%   }^^A
%   \ifdim\wd0>\lw
%     \sbox0{\small\t}^^A
%     \ifdim\wd0>\linewidth
%       \ifdim\wd0>\lw
%         \sbox0{\footnotesize\t}^^A
%         \ifdim\wd0>\linewidth
%           \ifdim\wd0>\lw
%             \sbox0{\scriptsize\t}^^A
%             \ifdim\wd0>\linewidth
%               \ifdim\wd0>\lw
%                 \sbox0{\tiny\t}^^A
%                 \ifdim\wd0>\linewidth
%                   \lwbox
%                 \else
%                   \usebox0
%                 \fi
%               \else
%                 \lwbox
%               \fi
%             \else
%               \usebox0
%             \fi
%           \else
%             \lwbox
%           \fi
%         \else
%           \usebox0
%         \fi
%       \else
%         \lwbox
%       \fi
%     \else
%       \usebox0
%     \fi
%   \else
%     \lwbox
%   \fi
% \else
%   \usebox0
% \fi
% \end{quote}
% If you have a \xfile{docstrip.cfg} that configures and enables \docstrip's
% TDS installing feature, then some files can already be in the right
% place, see the documentation of \docstrip.
%
% \subsection{Refresh file name databases}
%
% If your \TeX~distribution
% (\TeX\,Live, \mikTeX, \dots) relies on file name databases, you must refresh
% these. For example, \TeX\,Live\ users run \verb|texhash| or
% \verb|mktexlsr|.
%
% \subsection{Some details for the interested}
%
% \paragraph{Unpacking with \LaTeX.}
% The \xfile{.dtx} chooses its action depending on the format:
% \begin{description}
% \item[\plainTeX:] Run \docstrip\ and extract the files.
% \item[\LaTeX:] Generate the documentation.
% \end{description}
% If you insist on using \LaTeX\ for \docstrip\ (really,
% \docstrip\ does not need \LaTeX), then inform the autodetect routine
% about your intention:
% \begin{quote}
%   \verb|latex \let\install=y\input{ltxcmds.dtx}|
% \end{quote}
% Do not forget to quote the argument according to the demands
% of your shell.
%
% \paragraph{Generating the documentation.}
% You can use both the \xfile{.dtx} or the \xfile{.drv} to generate
% the documentation. The process can be configured by the
% configuration file \xfile{ltxdoc.cfg}. For instance, put this
% line into this file, if you want to have A4 as paper format:
% \begin{quote}
%   \verb|\PassOptionsToClass{a4paper}{article}|
% \end{quote}
% An example follows how to generate the
% documentation with pdf\LaTeX:
% \begin{quote}
%\begin{verbatim}
%pdflatex ltxcmds.dtx
%makeindex -s gind.ist ltxcmds.idx
%pdflatex ltxcmds.dtx
%makeindex -s gind.ist ltxcmds.idx
%pdflatex ltxcmds.dtx
%\end{verbatim}
% \end{quote}
%
% \begin{thebibliography}{9}
% \raggedright
%
% \bibitem{ifempty:schneck}
% Robert R. Schneck:
% \textit{Re: \cs{ifempty} solution %
%   (was Macro puzzle: maximally general \cs{ifempty})};
% newsgroup \xnewsgroup{comp.text.tex},
% \url{news:3eef1ada_6@corp.newsgroups.com},
% 2003-06-17.\\
% {\small
% \url{https://groups.google.com/group/comp.text.tex/msg/be03a159ec374895}}
%
% \bibitem{ifnull:diez}
% Ulrich Diez:
% \textit{Re: TeX refuses to strip outer braces in argument};
% newsgroup \xnewsgroup{comp.text.tex},
% \url{news:ibk3t8$ee7$1@news.albasani.net},
% 2010-11-12.\\
% {\small
% \url{https://groups.google.com/group/comp.text.tex/msg/803bd57221a04996}}
%
% \bibitem{ifstrempty:musa}
% Ahmed Musa:
% \textit{Re: TeX refuses to strip outer braces in argument};
% newsgroup \xnewsgroup{comp.text.tex},
% \url{news:f5496afe-40ed-42bd-b629-a2419ecf7c0d@o14g2000prn.googlegroups.com},
% 2010-12-03.\\
% {\small
% \url{https://groups.google.com/group/comp.text.tex/msg/fbf7d61a0c3a807d}}
%
% \bibitem{ifstrempty:diez1}
% Ulrich Diez:
% \textit{Re: TeX refuses to strip outer braces in argument};
% newsgroup \xnewsgroup{comp.text.tex},
% \url{news:idbo94$uka$1@four.albasani.net},
% 2010-12-03.\\
% {\small
% \url{https://groups.google.com/group/comp.text.tex/msg/0c230ee479487962}}
%
% \bibitem{ifstrempty:diez2}
% Ulrich Diez:
% \textit{Re: TeX refuses to strip outer braces in argument};
% newsgroup \xnewsgroup{comp.text.tex},
% \url{news:idbpu4$cg1$1@news.albasani.net},
% 2010-12-03.\\
% {\small
% \url{https://groups.google.com/group/comp.text.tex/msg/bbef4263390d647b}}
%
% \bibitem{ifstrempty:diez3}
% Ulrich Diez:
% \textit{Re: TeX refuses to strip outer braces in argument};
% newsgroup \xnewsgroup{comp.text.tex},
% \url{news:idd4ga$r83$1@four.albasani.net},
% 2010-12-04.\\
% {\small
% \url{https://groups.google.com/group/comp.text.tex/msg/00dfd1ec103cd272}}
%
% \bibitem{ifstrempty:gl}
% GL:
% \textit{Re: TeX refuses to strip outer braces in argument};
% newsgroup \xnewsgroup{comp.text.tex},
% \url{news:4cfa2e27$0$7389$426a74cc@news.free.fr},
% 2010-12-04.\\
% {\small
% \url{https://groups.google.com/group/comp.text.tex/msg/d3a75995c1cf267e}}
%
% \bibitem{ifstrempty:ho}
% Heiko Oberdiek:
% \textit{Re: TeX refuses to strip outer braces in argument};
% newsgroup \xnewsgroup{comp.text.tex},
% \url{news:iddhq1$3kj$1@news.eternal-september.org},
% 2010-12-04.\\
% {\small
% \url{https://groups.google.com/group/comp.text.tex/msg/5f7a23e3ab70e347}}
%
% \bibitem{ifboxempty:dak}
% David Kastrup:
% \textit{How to detect if \cs{vbox} is empty};
% newsgroup \xnewsgroup{comp.text.tex},
% 2011-02-04.\\
% {\small
% \url{https://groups.google.com/group/comp.text.tex/msg/8d3cb89496a4d86d}}
%
% \end{thebibliography}
%
% \begin{History}
%   \begin{Version}{2009/08/05 v1.0}
%   \item
%     First version.
%   \end{Version}
%   \begin{Version}{2009/12/12 v1.1}
%   \item
%     Short title shortened.
%   \item
%     \cs{ltx@IfUndefined} added.
%   \end{Version}
%   \begin{Version}{2010/01/28 v1.2}
%   \item
%     \cs{ltx@RemovePrefix} and \cs{ltx@StripPrefix} added.
%   \item
%     \cs{ltx@ifclassloaded}, \cs{ltx@ifpackageloaded},
%     \cs{ltx@iffileloaded},
%     \cs{ltx@ifclasslater}, \cs{ltx@ifpackagelater},
%     \cs{ltx@iffilelater},
%     \cs{ltx@clsextension}, \cs{ltx@pkgextension} added.
%   \item
%     \cs{ltx@GlobalAppendToMacro}, \cs{ltx@LocalAppendToMacro} added.
%   \end{Version}
%   \begin{Version}{2010/03/01 v1.3}
%   \item
%     \cs{ltx@newif} added.
%   \item
%     \cs{ltx@ifnextchar} added.
%   \item
%     Numbers \cs{ltx@zero}, \cs{ltx@one}, \cs{ltx@two}, \cs{ltx@cclv} added.
%   \end{Version}
%   \begin{Version}{2010/03/09 v1.4}
%   \item
%     \cs{ltx@pkgextension} and \cs{ltx@clsextension} are hardcoded to
%     avoid trouble with \cs{@onlypreamble}.
%   \end{Version}
%   \begin{Version}{2010/04/08 v1.5}
%   \item
%     \cs{ltx@cartwo}, \cs{ltx@cdrtwo},
%     \cs{ltx@carthree}, \cs{ltx@cdrthree},
%     \cs{ltx@carfour}, \cs{ltx@cdrfour} added.
%   \item
%     \cs{ltx@ReturnAfterFi} and \cs{ltx@ReturnAfterElseFi} fixed.
%   \end{Version}
%   \begin{Version}{2010/04/16 v1.6}
%   \item
%     \cs{ltx@leavevmode}, \cs{ltx@mbox} added.
%   \end{Version}
%   \begin{Version}{2010/04/26 v1.7}
%   \item
%     \cs{ltx@GobbleNum}, \cs{ltx@CdrNum}, \cs{ltx@CarNum} added.
%   \item
%     \cs{ltx@carzero}, \cs{ltx@cdrzero} added.
%   \item
%     \cs{ltx@hashchar} added.
%   \end{Version}
%   \begin{Version}{2010/09/11 v1.8}
%   \item
%     \cs{ltx@leftbracechar}, \cs{ltx@rightbracechar} added.
%   \end{Version}
%   \begin{Version}{2010/10/25 v1.9}
%   \item
%     \cs{ltx@LocalAppendToMacro} and \cs{ltx@GlobalAppendToMacro} are
%     now \cs{long}.
%   \end{Version}
%   \begin{Version}{2010/10/31 v1.10}
%   \item
%     \cs{ltx@newglobalif} added.
%   \end{Version}
%   \begin{Version}{2010/11/12 v1.11}
%   \item
%     \cs{ltx@ifempty} added.
%   \item
%     \cs{ltx@firstofthree}, \cs{ltx@secondofthree}, \cs{ltx@thirdofthree}
%     added.
%   \end{Version}
%   \begin{Version}{2010/12/02 v1.12}
%   \item
%     \cs{ltx@onelevel@sanitize} added.
%   \item
%     \cs{LTXcmds@num} fixed for the case with \cs{numexpr}
%     (bug found by GL).
%   \end{Version}
%   \begin{Version}{2010/12/04 v1.13}
%   \item
%     \cs{ltx@ifblank} added.
%   \item
%     Optimization for \cs{ltx@ifempty}.
%   \end{Version}
%   \begin{Version}{2010/12/07 v1.14}
%   \item
%     \cs{ltx@zapspace} added.
%   \end{Version}
%   \begin{Version}{2010/12/12 v1.15}
%   \item
%     \cs{ltx@minusone} added.
%   \end{Version}
%   \begin{Version}{2011/02/04 v1.16}
%   \item
%     \cs{ltx@IfBoxEmpty} and \cs{ltx@IfBoxVoidOrEmpty} added.
%   \item
%     \cs{ltx@firstoffour}, \dots, \cs{ltx@fourthoffour} added.
%   \end{Version}
%   \begin{Version}{2011/02/05 v1.17}
%   \item
%     \cs{ltx@IfBoxEmpty}: an empty box may have non-zero dimensions.
%   \end{Version}
%   \begin{Version}{2011/03/16 v1.18}
%   \item
%     \cs{ltx@ifclasslater} fixed.
%   \end{Version}
%   \begin{Version}{2011/04/14 v1.19}
%   \item
%     \cs{ltx@ifnextchar}: detection of optional spaces modified.
%   \item
%     \cs{ltx(Loc,Glob)(Toks,Dimen,Skip)(A,B,C,D,E)} added.
%   \end{Version}
%   \begin{Version}{2011/04/18 v1.20}
%   \item
%     \cs{ltx@ifnextchar} with conditional support (thanks GL for
%     bug report).
%   \end{Version}
%   \begin{Version}{2011/08/22 v1.21}
%   \item
%     \cs{ltx@GlobalPrependToMacro}, \cs{ltx@LocalPrependToMacro} added
%     (feature request of Martin M\"unch).
%   \end{Version}
%   \begin{Version}{2011/11/09 v1.22}
%   \item
%     \cs{ltx@carsecond}, \cs{ltx@carthird}, \cs{ltx@carfourth},
%     \cs{ltx@CarNumth} added.
%   \item
%     \cs{ltx@cdrzero}, \cs{ltx@cdr}, \cs{ltx@cdrtwo}, cs{ltx@cdrthree},
%     \cs{ltx@cdrfour}, \cs{ltx@CdrNum} modified to retain braces and spaces.
%     They are expandable in two expansion steps.
%   \end{Version}
%   \begin{Version}{2016/05/16 v1.23}
%   \item
%     Documentation updates.
%   \end{Version}
%   \begin{Version}{2019/12/15 v1.24}
%   \item
%     Documentation updates.
%   \end{Version}
% \end{History}
%
% \PrintIndex
%
% \Finale
\endinput

%        (quote the arguments according to the demands of your shell)
%
% Documentation:
%    (a) If ltxcmds.drv is present:
%           latex ltxcmds.drv
%    (b) Without ltxcmds.drv:
%           latex ltxcmds.dtx; ...
%    The class ltxdoc loads the configuration file ltxdoc.cfg
%    if available. Here you can specify further options, e.g.
%    use A4 as paper format:
%       \PassOptionsToClass{a4paper}{article}
%
%    Programm calls to get the documentation (example):
%       pdflatex ltxcmds.dtx
%       makeindex -s gind.ist ltxcmds.idx
%       pdflatex ltxcmds.dtx
%       makeindex -s gind.ist ltxcmds.idx
%       pdflatex ltxcmds.dtx
%
% Installation:
%    TDS:tex/generic/ltxcmds/ltxcmds.sty
%    TDS:doc/latex/ltxcmds/ltxcmds.pdf
%    TDS:source/latex/ltxcmds/ltxcmds.dtx
%
%<*ignore>
\begingroup
  \catcode123=1 %
  \catcode125=2 %
  \def\x{LaTeX2e}%
\expandafter\endgroup
\ifcase 0\ifx\install y1\fi\expandafter
         \ifx\csname processbatchFile\endcsname\relax\else1\fi
         \ifx\fmtname\x\else 1\fi\relax
\else\csname fi\endcsname
%</ignore>
%<*install>
\input docstrip.tex
\Msg{************************************************************************}
\Msg{* Installation}
\Msg{* Package: ltxcmds 2019/12/15 v1.24 LaTeX kernel commands for general use (HO)}
\Msg{************************************************************************}

\keepsilent
\askforoverwritefalse

\let\MetaPrefix\relax
\preamble

This is a generated file.

Project: ltxcmds
Version: 2019/12/15 v1.24

Copyright (C)
   2009-2011 Heiko Oberdiek
   2016-2019 Oberdiek Package Support Group

This work may be distributed and/or modified under the
conditions of the LaTeX Project Public License, either
version 1.3c of this license or (at your option) any later
version. This version of this license is in
   https://www.latex-project.org/lppl/lppl-1-3c.txt
and the latest version of this license is in
   https://www.latex-project.org/lppl.txt
and version 1.3 or later is part of all distributions of
LaTeX version 2005/12/01 or later.

This work has the LPPL maintenance status "maintained".

The Current Maintainers of this work are
Heiko Oberdiek and the Oberdiek Package Support Group
https://github.com/ho-tex/ltxcmds/issues


The Base Interpreter refers to any `TeX-Format',
because some files are installed in TDS:tex/generic//.

This work consists of the main source file ltxcmds.dtx
and the derived files
   ltxcmds.sty, ltxcmds.pdf, ltxcmds.ins, ltxcmds.drv,
   ltxcmds-test1.tex, ltxcmds-test-gobble.tex,
   ltxcmds-test-ifempty.tex, ltxcmds-test-zapspace.tex,
   ltxcmds-test-ifboxempty.tex, ltxcmds-test-nextchar.tex,
   ltxcmds-test-carcdr.tex.

\endpreamble
\let\MetaPrefix\DoubleperCent

\generate{%
  \file{ltxcmds.ins}{\from{ltxcmds.dtx}{install}}%
  \file{ltxcmds.drv}{\from{ltxcmds.dtx}{driver}}%
  \usedir{tex/generic/ltxcmds}%
  \file{ltxcmds.sty}{\from{ltxcmds.dtx}{package}}%
}

\catcode32=13\relax% active space
\let =\space%
\Msg{************************************************************************}
\Msg{*}
\Msg{* To finish the installation you have to move the following}
\Msg{* file into a directory searched by TeX:}
\Msg{*}
\Msg{*     ltxcmds.sty}
\Msg{*}
\Msg{* To produce the documentation run the file `ltxcmds.drv'}
\Msg{* through LaTeX.}
\Msg{*}
\Msg{* Happy TeXing!}
\Msg{*}
\Msg{************************************************************************}

\endbatchfile
%</install>
%<*ignore>
\fi
%</ignore>
%<*driver>
\NeedsTeXFormat{LaTeX2e}
\ProvidesFile{ltxcmds.drv}%
  [2019/12/15 v1.24 LaTeX kernel commands for general use (HO)]%
\documentclass{ltxdoc}
\usepackage{holtxdoc}[2011/11/22]
\usepackage{zref-savepos}
\usepackage{paralist}
\hfuzz=1pt
\begin{document}
  \DocInput{ltxcmds.dtx}%
\end{document}
%</driver>
% \fi
%
%
%
% \GetFileInfo{ltxcmds.drv}
%
% \title{The \xpackage{ltxcmds} package}
% \date{2019/12/15 v1.24}
% \author{Heiko Oberdiek\thanks
% {Please report any issues at \url{https://github.com/ho-tex/ltxcmds/issues}}}
%
% \maketitle
%
% \begin{abstract}
% The package \xpackage{ltxcmds} exports some utility macros
% from the \LaTeX\ kernel into a separate namespace and
% also provides them for other formats such as plain-\TeX.
% \end{abstract}
%
% \tableofcontents
%
% \newcounter{to}
% \renewcommand*{\theto}{to\arabic{to}}
% \newlength\tolength
% \tolength=\dimexpr .5\linewidth+\oddsidemargin+1in+15mm\relax
% \newcommand*{\TO}{^^A
%   \quad
%   \stepcounter{to}^^A
%   \zsavepos\theto
%   \ifdim\zposx\theto sp>0pt
%     \ifdim\zposx\theto sp<\tolength
%       \kern\dimexpr\tolength-\zposx\theto sp\relax
%     \fi
%   \fi
%   $\rightarrow$\quad
%   \ignorespaces
% }
%
% \section{Documentation}
%
% \subsection{Introduction}
%
% Many of my packages also support other formats such as plain-\TeX.
% Because I am rather familiar with the utility macros from
% \LaTeX's kernel (e.g. \cs{@gobble}, \cs{@firstoftwo}), I found
% myself rewriting them again and again, because they are lacking
% in plain-\TeX.
%
% Therefore this package provides often used macros and similar
% ones with the name prefix \cs{ltx@}. This avoids also faulty
% redefinitions. I remember an example where a package redefined
% \cs{@firstoftwo} with forgetting \cs{long}.
%
% \subsection{Numbers}
%
% \begin{declcs}{ltx@zero} \TO 0\\
%   \cs{ltx@one} \TO 1\\
%   \cs{ltx@two} \TO 2\\
%   \cs{ltx@cclv} \TO 255\\
%   \cs{ltx@minusone} \TO -1
% \end{declcs}
% These commands are numbers 0, 1, 2, 255 and -1. They are not digits
% and a space is not gobbled afterwards. Macro \cs{ltx@minusone}
% is available since version 2010/12/12 v1.15.
%
% \subsection{Scratch registers}
%
% Following the conventions of \hologo{plainTeX} and \hologo{LaTeX}
% the first ten registers are free to use. Even numbered registers
% are for local, odd numbered for global use.
%
% \begin{declcs}{ltx@(Loc,Glob)(Toks,Dimen,Skip)(A,B,C,D,E)}
% \end{declcs}
% The name consists of the prefix \cs{ltx@}, then
% \texttt{Loc} or \texttt{Glob} for local or global usage follows.
% The register type is given by \texttt{Toks} for token register,
% \texttt{Dimen} for dimen register and \texttt{Skip} for skip
% register. As last part the registers are numbered from \texttt{A}
% to \texttt{E}. Example: \cs{ltx@LocToksA}.
%
% Since 2011/04/14 v1.19.
%
% \subsection{Argument killers}
%
% \begin{declcs}{ltx@gobble} \M{1} \TO\\
%   \cs{ltx@gobbletwo} \M{1} \M{2} \TO\\
%   \cs{ltx@gobblethree} \M{1} \M{2} \M{3} \TO\\
%   \cs{ltx@gobblefour} \M{1} \M{2} \M{3} \M{4} \TO
% \end{declcs}
%
% \begin{declcs}{ltx@GobbleNum} \M{num} \M{1} \M{2} \dots\space
%   \M{\meta{num}} \TO
% \end{declcs}
% The first argument \meta{num} of macro \cs{ltx@GobbleNum} specifies,
% how many following arguments are eaten. Macro \cs{ltx@GobbleNum}
% is expandable in exact two expansion steps.
%
% \subsection{Argument grabbers}
%
% \begin{declcs}{ltx@firstofone} \M{1} \TO \meta{1}\\[1ex]
%  \cs{ltx@firstoftwo} \M{1} \M{2} \TO \meta{1}\\
%  \cs{ltx@secondoftwo} \M{1} \M{2} \TO \meta{2}\\[1ex]
%  \cs{ltx@firstofthree} \M{1} \M{2} \M{3} \TO \meta{1}\\
%  \cs{ltx@secondofthree} \M{1} \M{2} \M{3} \TO \meta{2}\\
%  \cs{ltx@thirdofthree} \M{1} \M{2} \M{3} \TO \meta{3}\\
%  \cs{ltx@firstoffour} \M{1} \M{2} \M{3} \M{4} \TO \meta{1}\\
%  \cs{ltx@secondoffour} \M{1} \M{2} \M{3} \M{4} \TO \meta{2}\\
%  \cs{ltx@thirdoffour} \M{1} \M{2} \M{3} \M{4} \TO \meta{3}\\
%  \cs{ltx@fourthoffour} \M{1} \M{2} \M{3} \M{4} \TO \meta{4}
% \end{declcs}
% Macros \cs{ltx@firstofthree}, \cs{ltx@secondofthree}
% and \cs{ltx@thirdofthree} were added in version
% 2010/11/12 v1.11.
% Macros \cs{ltx@firstoffour}, \dots, \cs{ltx@fourthoffour}
% were added in version 2011/02/04 v1.16.
%
% \subsection{List helpers}
%
% \begin{declcs}{ltx@carzero} \dots\ \cs{@nil} \TO\\
%   \cs{ltx@cdrzero} \dots\ \cs{@nil} \TO \dots
% \end{declcs}
%
% \begin{declcs}{ltx@car} \M{1} \dots\ \cs{@nil} \TO \meta{1}\\
%   \cs{ltx@cdr} \M{1} \dots\ \cs{@nil} \TO \dots
% \end{declcs}
%
% \begin{declcs}{ltx@cartwo} \M{1} \M{2} \dots\ \cs{@nil}
%   \TO \meta{1}\meta{2}\\
%   \cs{ltx@carsecond} \M{1} \M{2} \dots\ \cs{@nil} \TO \meta{2}\\
%   \cs{ltx@cdrtwo} \M{1} \M{2} \dots\ \cs{@nil} \TO \dots
% \end{declcs}
%
% \begin{declcs}{ltx@carthree} \M{1} \M{2} \M{3} \dots\ \cs{@nil}
%       \TO \meta{1}\meta{2}\meta{3}\\
%   \cs{ltx@carthird} \M{1} \M{2} \M{3} \dots\ \cs{@nil}
%       \TO \meta{3}\\
%   \cs{ltx@cdrthree} \M{1} \M{2} \M{3} \dots\ \cs{@nil} \TO \dots
% \end{declcs}
%
% \begin{declcs}{ltx@carfour} \M{1} \M{2} \M{3} \M{4} \dots\ \cs{@nil}
%       \TO \meta{1}\meta{2}\meta{3}\meta{4}\\
%   \cs{ltx@carfourth} \M{1} \M{2} \M{3} \M{4} \dots\ \cs{@nil}
%       \TO \meta{4}\\
%   \cs{ltx@cdrfour} \M{1} \M{2} \M{3} \M{4} \dots\ \cs{@nil} \TO \dots
% \end{declcs}
%
% \begin{declcs}{ltx@CarNum} \M{num} \M{1} \dots\ \M{\meta{num}}
%   \M{\meta{num}+1} \dots\ \cs{@nil}\\
%   \TO \M{1} \dots\ \M{\meta{num}} \dots\\
%   \cs{ltx@CarNumth} \M{num} \M{1} \dots\ \M{\meta{num}}
%   \M{\meta{num}+1} \dots\ \cs{@nil}\\
%   \TO \M{\meta{num}} \dots\\
%   \cs{ltx@CdrNum} \M{num} \M{1} \dots\ \M{\meta{num}}
%   \M{\meta{num}+1} \dots\ \cs{@nil}\\
%   \TO \M{\meta{num}+1} \dots
% \end{declcs}
%
% Macros with uppercase letters are expandable in two expansion steps.
% Changes in version 2019/12/15 v1.24:
% \begin{itemize}
% \item
%   Macros \cs{ltx@carsecond}, \cs{ltx@carthird}, \cs{ltx@carfourth},
%   \cs{ltx@CarNumth} added.
% \item
%   Macros \cs{ltx@cdr}, \cs{ltx@cdrtwo}, \cs{ltx@cdrthree}, \cs{ltx@cdrfour},
%   \cs{ltx@CdrNum} are expandable in two expansion steps and retain
%   spaces and braces after the first gobbled arguments.
% \end{itemize}
%
% \subsection{Tail recursion}
%
% \begin{declcs}{ltx@ReturnAfterFi} \M{1} \cs{fi} \TO \cs{fi} \meta{1}\\
%   \cs{ltx@ReturnAfterElseFi} \M{1} \cs{else} \M{2} \cs{fi}
%     \TO \cs{fi} \meta{1}
% \end{declcs}
%
% \subsection{Empty macro}
%
% \begin{declcs}{ltx@empty} \TO
% \end{declcs}
%
% \subsection{Characters}
%
% \begin{declcs}{ltx@space} \TO \verb*| |\\
%   \cs{ltx@percentchar} \TO \texttt{\%}\\
%   \cs{ltx@backslashchar} \TO
%      \makeatletter\texttt{\@backslashchar}\\
%   \cs{ltx@hashchar} \TO \texttt{\#}\qquad(since v1.7)\\
%   \cs{ltx@leftbracechar} \TO \texttt{\textbraceleft}\qquad(since v1.8)\\
%   \cs{ltx@rightbracechar} \TO \texttt{\textbraceright}\qquad(since v1.8)
% \end{declcs}
%
% \subsection{Boolean switch}
%
% \begin{declcs}{ltx@newif} \M{cmd}
% \end{declcs}
% \cs{ltx@newif} defines a new boolean switch \meta{cmd} like \cs{newif}.
% Unlike \plainTeX's \cs{newif}, \cs{ltx@newif} is not \cs{outer}.
% The command \meta{cmd} must start with the two characters |if|.
%
% \begin{declcs}{ltx@newglobalif} \M{cmd}
% \end{declcs}
% \cs{ltx@newglobalif} defines a new boolean switch \meta{cmd}
% like \cs{ltx@newif}. However the switch setting commands,
% \meta{cmd} without the prefix |if| and followed by |true| or |false|
% are acting globally.
%
% \subsection{Command definitions}
%
% \begin{declcs}{ltx@ifundefined} \M{cmd} \M{yes} \M{no}
% \end{declcs}
% If \eTeX\ is available, \cs{ifcsname} is used that does
% not have the side effect of defining undefined commands
% with meaning of \cs{relax}. This command is always expandable.
% Change in version 1.1: Also the meaning \cs{relax} is
% always considered ``undefined''.
%
% \begin{declcs}{ltx@IfUndefined} \M{cmd} \M{yes} \M{no}
% \end{declcs}
% If \eTeX\ is available, \cs{ifcsname} is used that does
% not have the side effect of defining undefined commands
% with meaning of \cs{relax}. Also it always checks for
% the meaning of \cs{relax} and considers this as undefined.
% This macro is not expandable without \eTeX.
%
% \begin{declcs}{ltx@LocalExpandAfter}
% \end{declcs}
% It expands the token after the next token but in a local
% context. That is the difference to \cs{expandafter}.
% The local context discards the side effect of \cs{csname}
% and let the command undefined after the expansion step.
%
% \subsection{Stripping}
%
% \begin{declcs}{ltx@RemovePrefix}\\
%   \cs{ltx@StripPrefix}
% \end{declcs}
% All tokens up to and including the next available character
% `\textgreater' are thrown away. Usually it is used to strip
% the first part of the output of the commands \cs{meaning}
% or \cs{pdflastmatch}. Macro \cs{ltx@RemovePrefix}
% has the same meaning as \LaTeX's \cs{strip@prefix}, whereas
% macro \cs{ltx@StripPrefix} expands the next token
% once before stripping the prefix.
%
% \begin{declcs}{ltx@onelevel@sanitize} \M{macro}
% \end{declcs}
% Macro \cs{ltx@onelevel@sanitize} provides \hologo{LaTeX}'s
% \cs{@onelevel@sanitize}. The macro is expanded once and
% the contents is converted to characters with catcode 12 (other)
% and space tokens with catcode 10 (space). Then then sanitized
% contents is stored into the macro again. Since version 1.12.
%
% \subsection{File management}
%
% All macros in this section are expandable like
% the counterparts of the \LaTeX\ kernel. Also they
% can be used after the preamble.
%
% \subsubsection{File extensions}
%
% \begin{declcs}{ltx@clsextension}\\
%   \cs{ltx@pkgextension}
% \end{declcs}
% Macros \cs{ltx@clsextension} and \cs{ltx@styextension}
% stores the strings |cls| and |sty|. In opposite to \hologo{LaTeX}'s
% \cs{@clsextension} and \cs{@styextension} they can also be
% used after |\begin{document}|.
%
% \subsubsection{Load check}
%
% \begin{declcs}{ltx@ifclassloaded} \M{class} \M{yes} \M{no}\\
%   \cs{ltx@ifpackageloaded} \M{package} \M{yes} \M{no}
% \end{declcs}
% Macros \cs{ltx@ifclassloaded}\slash\cs{ltx@ifpackageloaded}
% execute \meta{yes}, if the \meta{class} or \meta{package} is loaded,
% otherwise \meta{no} is called.
% Both \meta{class} and \meta{package} are specified without
% extension. The macros can also be used after |\begin{document}|.
%
% \begin{declcs}{ltx@iffileloaded} \M{file} \M{yes} \M{no}
% \end{declcs}
% If \LaTeX's \cs{ProvidesFile} macro was called before
% using \meta{file} as argument, then \cs{ltx@iffileloaded}
% calls \meta{yes}, otherwise \meta{no}.
% Therefore it is possible
% that the \meta{file} is loaded, but \meta{no} is executed because
% of a missing \cs{ProvidesFile}. The \LaTeX\ kernel does not have
% a counterpart of \cs{ltx@iffileloaded}.
%
% Note that the file name used in \cs{ProvidesFile} and
% \cs{ltx@iffileloaded} must match. For example, if \TeX's
% default extension \xext{tex} was given in the first command,
% then it must also specified in the latter command and vice versa.
%
% \subsubsection{Version date check}
%
% \begin{declcs}{ltx@ifclasslater} \M{class} \M{date} \M{yes} \M{no}\\
%   \cs{ltx@ifpackagelater} \M{package} \M{date} \M{yes} \M{no}\\
%   \cs{ltx@iffilelater} \M{file} \M{date} \M{yes} \M{no}
% \end{declcs}
% If a \cs{ProvidesClass}\slash\cs{ProvidesPackage}\slash
% \cs{ProvidesFile} command with exact the same class\slash package\slash
% file was executed before with an optional
% argument that starts with a \LaTeX\ version date, then
% this version date is compared with the argument \meta{date}.
% If they are equal or if the version date is the later date,
% then \meta{yes} is called. In all other cases \meta{no} is executed.
%
% A \LaTeX\ date has the format |YYYY/MM/DD| with |YYYY| as year
% with four digits, |MM| as month with two digits and |DD| as day
% with two digits. If \pdfTeX's \cs{pdfmatch} is available, then
% it is used to detect the version date, to reject invalid
% date formats and to reject some invalid dates.
% Dates before 1994/01/01 are always invalid, because version
% dates are introduced with \LaTeXe\ in 1994.
%
% \subsection{Macro additions}
%
% \begin{declcs}{ltx@GlobalAppendToMacro} \M{cmd} \M{addition}\\
%   \cs{ltx@LocalAppendToMacro} \M{cmd} \M{addition}
% \end{declcs}
% The \meta{addition} is appended to the parameterless macro \meta{cmd}.
% If \meta{cmd} is undefined or has the meaning \cs{relax}, then
% it will be initialized as empty macro beforehand. Due to a bug \meta{addition}
% must not contain \cs{par} before version 2010/10/25 v1.9.
%
% \begin{declcs}{ltx@GlobalPrependToMacro} \M{cmd} \M{addition}\\
%   \cs{ltx@LocalPrependToMacro} \M{cmd} \M{addition}
% \end{declcs}
% The \meta{addition} is prepended to the parameterless macro \meta{cmd}.
% If \meta{cmd} is undefined or has the meaning \cs{relax}, then
% it will be initialized as empty macro beforehand. The macros
% were added in version 2011/08/22 v1.21.
%
% \subsection{Next character detection}
%
% \begin{declcs}{ltx@ifnextchar} \M{char} \M{yes} \M{no}
% \end{declcs}
% If next character is \meta{char} then \meta{yes} is called,
% otherwise \meta{no}. The character is not removed.
% Spaces are silently removed when looking for \meta{char}
% as \hologo{LaTeX}'s version \cs{kernel@ifnextchar} does.
% But there are also small differences:
% \begin{itemize}
% \item The space can be used as \meta{char}. In this case
%   optional spaces before \meta{char} are not supported of course.
% \item If the optional space is a command that is a character
%   (defined by \cs{let} or \cs{futurelet}), then
%   \cs{kernel@ifnextchar} breaks with an \hologo{TeX} error.
%   \cs{ltx@ifnextchar} silently removes this token as
%   optional space.
% \end{itemize}
% Since 2010/03/01 v1.3.
%
% \begin{declcs}{ltx@ifnextchar@nospace} \M{char} \M{yes} \M{no}
% \end{declcs}
% Macro \cs{ltx@ifnextchar@nospace} behaves like
% macro \cs{ltx@ifnextchar} with the exception that
% optional spaces are not supported before \meta{char}.
% Since 2011/04/14 v1.19.
%
% \subsection{\cs{ltx@leavevmode}, \cs{ltx@mbox}}
%
% \begin{declcs}{ltx@leavevmode}
% \end{declcs}
% Macro \cs{ltx@leavevmode} calls \hologo{pdfTeX}'s \cs{quitvmode}.
% Otherwise \cs{leavevmode} is used and defined if it is necessary.
%
% \begin{declcs}{ltx@mbox}
% \end{declcs}
% Macro \cs{ltx@mbox} reimplements \cs{mbox} with two changes.
% Instead of \cs{leavevmode} it uses \cs{ltx@leavevmode} and
% stops right after \cs{hbox}. Especially it does not grab
% the argument and allows the extended syntax of \cs{hbox}.
%
% \subsection{Expandable test for emptiness}
%
% \begin{declcs}{ltx@ifempty} \M{stuff} \M{yes} \M{no}
% \end{declcs}
% Macro \cs{ltx@ifempty} checks in exact two expansion steps
% whether \meta{stuff}
% is empty or contains token. Depending on the result
% \meta{yes} or \meta{no} is executed. The token in \meta{stuff}
% may contain \cs{par} and unmatched conditionals (\cs{if}, \cs{else},
% \cs{fi}, \dots). Since version 2010/11/12 v1.11.
%
% \begin{declcs}{ltx@ifblank} \M{stuff} \M{yes} \M{no}
% \end{declcs}
% Macro \cs{ltx@ifblank} tests in exact two expansion steps
% if \meta{stuff} is empty or contain only blank spaces.
% In this case argument \meta{yes} is called. If \meta{stuff}
% contains other tokens than spaces then \meta{no} is executed.
% Since version 2010/12/04 v1.13.
%
% \subsection{Stripping spaces}
%
% \begin{declcs}{ltx@zapspace} \M{stuff}
% \end{declcs}
% Macro \cs{ltx@zapspace} strips spaces from \meta{stuff}
% that are not hidden inside curly braces. Like \hologo{LaTeX}'s
% \cs{zap@space} it is expandable. Differences:
% \begin{itemize}
% \item
% Syntax: \cs{zap@space} also expects a space token
% and \cs{@empty} after \meta{stuff}.
% \item
% Macro \cs{ltx@zapspace} is expandable in exact two
% expansion steps.
% \item
% Macro \cs{ltx@zapspace} always retains curly braces.
% \item
% Macro \cs{zap@space} has a bug. It stops stripping spaces
% after a token group in curly braces if the first two tokens
% inside the group are equal.
% \item
% Macro \cs{ltx@zapspace} also works with \cs{par}
% and conditionals (\cs{if}, \cs{else}, \cs{fi}, \dots).
% \end{itemize}
% Macro \cs{ltx@zapspace} is available since version
% 2010/12/07 v1.14.
%
% \subsection{Check for emptiness of boxes}
%
% \begin{declcs}{ltx@IfBoxEmpty} \M{box register number}
%   \M{yes} \M{no}
% \end{declcs}
% Macro \cs{ltx@IfBoxEmpty} calls \meta{yes} if
% the box exists (\cs{ifvoid} returns false) and the box
% does not contain any content. Otherwise if the box
% is void or contains something, then \meta{no} is executed.
% Thus being empty means that the box exists and is either
% an \cs{hbox} or a \cs{vbox} and may even have dimensions other than
% 0.0\,pt, but the box does not contain anything.
% Macro \cs{ltx@IfBoxEmpty} is available since 2010/02/04 v1.16.
%
% \begin{declcs}{ltx@IfBoxVoidOrEmpty} \M{box register number}
%   \M{yes} \M{no}
% \end{declcs}
% Macro \cs{ltx@IfBoxVoidOrEmpty} calls \meta{yes} if
% the box is either void or does not contain any content.
% Otherwise \meta{no} is executed.
% Macro \cs{ltx@IfBoxVoidOrEmpty} is available since 2010/02/04 v1.16.
%
% \StopEventually{
% }
%
% \section{Implementation}
%
% \subsection{Identification}
%
%    \begin{macrocode}
%<*package>
%    \end{macrocode}
%    Reload check, especially if the package is not used with \LaTeX.
%    \begin{macrocode}
\begingroup\catcode61\catcode48\catcode32=10\relax%
  \catcode13=5 % ^^M
  \endlinechar=13 %
  \catcode35=6 % #
  \catcode39=12 % '
  \catcode44=12 % ,
  \catcode45=12 % -
  \catcode46=12 % .
  \catcode58=12 % :
  \catcode64=11 % @
  \catcode123=1 % {
  \catcode125=2 % }
  \expandafter\let\expandafter\x\csname ver@ltxcmds.sty\endcsname
  \ifx\x\relax % plain-TeX, first loading
  \else
    \def\empty{}%
    \ifx\x\empty % LaTeX, first loading,
      % variable is initialized, but \ProvidesPackage not yet seen
    \else
      \expandafter\ifx\csname PackageInfo\endcsname\relax
        \def\x#1#2{%
          \immediate\write-1{Package #1 Info: #2.}%
        }%
      \else
        \def\x#1#2{\PackageInfo{#1}{#2, stopped}}%
      \fi
      \x{ltxcmds}{The package is already loaded}%
      \aftergroup\endinput
    \fi
  \fi
\endgroup%
%    \end{macrocode}
%    Package identification:
%    \begin{macrocode}
\begingroup\catcode61\catcode48\catcode32=10\relax%
  \catcode13=5 % ^^M
  \endlinechar=13 %
  \catcode35=6 % #
  \catcode39=12 % '
  \catcode40=12 % (
  \catcode41=12 % )
  \catcode44=12 % ,
  \catcode45=12 % -
  \catcode46=12 % .
  \catcode47=12 % /
  \catcode58=12 % :
  \catcode64=11 % @
  \catcode91=12 % [
  \catcode93=12 % ]
  \catcode123=1 % {
  \catcode125=2 % }
  \expandafter\ifx\csname ProvidesPackage\endcsname\relax
    \def\x#1#2#3[#4]{\endgroup
      \immediate\write-1{Package: #3 #4}%
      \xdef#1{#4}%
    }%
  \else
    \def\x#1#2[#3]{\endgroup
      #2[{#3}]%
      \ifx#1\@undefined
        \xdef#1{#3}%
      \fi
      \ifx#1\relax
        \xdef#1{#3}%
      \fi
    }%
  \fi
\expandafter\x\csname ver@ltxcmds.sty\endcsname
\ProvidesPackage{ltxcmds}%
  [2019/12/15 v1.24 LaTeX kernel commands for general use (HO)]%
%    \end{macrocode}
%
%    \begin{macrocode}
\begingroup\catcode61\catcode48\catcode32=10\relax%
  \catcode13=5 % ^^M
  \endlinechar=13 %
  \catcode123=1 % {
  \catcode125=2 % }
  \catcode64=11 % @
  \def\x{\endgroup
    \expandafter\edef\csname LTXcmds@AtEnd\endcsname{%
      \endlinechar=\the\endlinechar\relax
      \catcode13=\the\catcode13\relax
      \catcode32=\the\catcode32\relax
      \catcode35=\the\catcode35\relax
      \catcode61=\the\catcode61\relax
      \catcode64=\the\catcode64\relax
      \catcode123=\the\catcode123\relax
      \catcode125=\the\catcode125\relax
    }%
  }%
\x\catcode61\catcode48\catcode32=10\relax%
\catcode13=5 % ^^M
\endlinechar=13 %
\catcode35=6 % #
\catcode64=11 % @
\catcode123=1 % {
\catcode125=2 % }
\def\TMP@EnsureCode#1#2{%
  \edef\LTXcmds@AtEnd{%
    \LTXcmds@AtEnd
    \catcode#1=\the\catcode#1\relax
  }%
  \catcode#1=#2\relax
}
\TMP@EnsureCode{36}{3}% $
\TMP@EnsureCode{38}{4}% &
\TMP@EnsureCode{40}{12}% (
\TMP@EnsureCode{41}{12}% )
\TMP@EnsureCode{45}{12}% -
\TMP@EnsureCode{46}{12}% .
\TMP@EnsureCode{47}{12}% /
\TMP@EnsureCode{60}{12}% <
\TMP@EnsureCode{62}{12}% >
\TMP@EnsureCode{91}{12}% [
\TMP@EnsureCode{96}{12}% `
\TMP@EnsureCode{93}{12}% ]
\TMP@EnsureCode{94}{12}% ^ (superscript) (!)
\TMP@EnsureCode{124}{12}% |
\edef\LTXcmds@AtEnd{\LTXcmds@AtEnd\noexpand\endinput}
%    \end{macrocode}
%
% \subsection{Numbers}
%
%    \begin{macro}{\ltx@zero}
%    \begin{macrocode}
\chardef\ltx@zero=0 %
%    \end{macrocode}
%    \end{macro}
%    \begin{macro}{\ltx@one}
%    \begin{macrocode}
\chardef\ltx@one=1 %
%    \end{macrocode}
%    \end{macro}
%    \begin{macro}{\ltx@two}
%    \begin{macrocode}
\chardef\ltx@two=2 %
%    \end{macrocode}
%    \end{macro}
%    \begin{macro}{\ltx@active}
%    \begin{macrocode}
\chardef\ltx@active=13 %
%    \end{macrocode}
%    \end{macro}
%    \begin{macro}{\ltx@cclv}
%    \begin{macrocode}
\chardef\ltx@cclv=255 %
%    \end{macrocode}
%    \end{macro}
%
%    \begin{macro}{\ltx@minusone}
%    \begin{macrocode}
\def\ltx@minusone{%
  -\ltx@one
}
%    \end{macrocode}
%    \end{macro}
%
% \subsection{Scratch registers}
%
%    \begin{macro}{\ltx@LocToksA}
%    \begin{macrocode}
\toksdef\ltx@LocToksA=0 %
%    \end{macrocode}
%    \end{macro}
%    \begin{macro}{\ltx@LocToksB}
%    \begin{macrocode}
\toksdef\ltx@LocToksB=2 %
%    \end{macrocode}
%    \end{macro}
%    \begin{macro}{\ltx@LocToksC}
%    \begin{macrocode}
\toksdef\ltx@LocToksC=4 %
%    \end{macrocode}
%    \end{macro}
%    \begin{macro}{\ltx@LocToksD}
%    \begin{macrocode}
\toksdef\ltx@LocToksD=6 %
%    \end{macrocode}
%    \end{macro}
%    \begin{macro}{\ltx@LocToksE}
%    \begin{macrocode}
\toksdef\ltx@LocToksE=8 %
%    \end{macrocode}
%    \end{macro}
%
%    \begin{macro}{\ltx@GlobToksA}
%    \begin{macrocode}
\toksdef\ltx@GlobToksA=1 %
%    \end{macrocode}
%    \end{macro}
%    \begin{macro}{\ltx@GlobToksB}
%    \begin{macrocode}
\toksdef\ltx@GlobToksB=3 %
%    \end{macrocode}
%    \end{macro}
%    \begin{macro}{\ltx@GlobToksC}
%    \begin{macrocode}
\toksdef\ltx@GlobToksC=5 %
%    \end{macrocode}
%    \end{macro}
%    \begin{macro}{\ltx@GlobToksD}
%    \begin{macrocode}
\toksdef\ltx@GlobToksD=7 %
%    \end{macrocode}
%    \end{macro}
%    \begin{macro}{\ltx@GlobToksE}
%    \begin{macrocode}
\toksdef\ltx@GlobToksE=9 %
%    \end{macrocode}
%    \end{macro}
%
%    \begin{macro}{\ltx@LocDimenA}
%    \begin{macrocode}
\dimendef\ltx@LocDimenA=0 %
%    \end{macrocode}
%    \end{macro}
%    \begin{macro}{\ltx@LocDimenB}
%    \begin{macrocode}
\dimendef\ltx@LocDimenB=2 %
%    \end{macrocode}
%    \end{macro}
%    \begin{macro}{\ltx@LocDimenC}
%    \begin{macrocode}
\dimendef\ltx@LocDimenC=4 %
%    \end{macrocode}
%    \end{macro}
%    \begin{macro}{\ltx@LocDimenD}
%    \begin{macrocode}
\dimendef\ltx@LocDimenD=6 %
%    \end{macrocode}
%    \end{macro}
%    \begin{macro}{\ltx@LocDimenE}
%    \begin{macrocode}
\dimendef\ltx@LocDimenE=8 %
%    \end{macrocode}
%    \end{macro}
%
%    \begin{macro}{\ltx@GlobDimenA}
%    \begin{macrocode}
\dimendef\ltx@GlobDimenA=1 %
%    \end{macrocode}
%    \end{macro}
%    \begin{macro}{\ltx@GlobDimenB}
%    \begin{macrocode}
\dimendef\ltx@GlobDimenB=3 %
%    \end{macrocode}
%    \end{macro}
%    \begin{macro}{\ltx@GlobDimenC}
%    \begin{macrocode}
\dimendef\ltx@GlobDimenC=5 %
%    \end{macrocode}
%    \end{macro}
%    \begin{macro}{\ltx@GlobDimenD}
%    \begin{macrocode}
\dimendef\ltx@GlobDimenD=7 %
%    \end{macrocode}
%    \end{macro}
%    \begin{macro}{\ltx@GlobDimenE}
%    \begin{macrocode}
\dimendef\ltx@GlobDimenE=9 %
%    \end{macrocode}
%    \end{macro}
%
%    \begin{macro}{\ltx@LocSkipA}
%    \begin{macrocode}
\skipdef\ltx@LocSkipA=0 %
%    \end{macrocode}
%    \end{macro}
%    \begin{macro}{\ltx@LocSkipB}
%    \begin{macrocode}
\skipdef\ltx@LocSkipB=2 %
%    \end{macrocode}
%    \end{macro}
%    \begin{macro}{\ltx@LocSkipC}
%    \begin{macrocode}
\skipdef\ltx@LocSkipC=4 %
%    \end{macrocode}
%    \end{macro}
%    \begin{macro}{\ltx@LocSkipD}
%    \begin{macrocode}
\skipdef\ltx@LocSkipD=6 %
%    \end{macrocode}
%    \end{macro}
%    \begin{macro}{\ltx@LocSkipE}
%    \begin{macrocode}
\skipdef\ltx@LocSkipE=8 %
%    \end{macrocode}
%    \end{macro}
%
%    \begin{macro}{\ltx@GlobSkipA}
%    \begin{macrocode}
\skipdef\ltx@GlobSkipA=1 %
%    \end{macrocode}
%    \end{macro}
%    \begin{macro}{\ltx@GlobSkipB}
%    \begin{macrocode}
\skipdef\ltx@GlobSkipB=3 %
%    \end{macrocode}
%    \end{macro}
%    \begin{macro}{\ltx@GlobSkipC}
%    \begin{macrocode}
\skipdef\ltx@GlobSkipC=5 %
%    \end{macrocode}
%    \end{macro}
%    \begin{macro}{\ltx@GlobSkipD}
%    \begin{macrocode}
\skipdef\ltx@GlobSkipD=7 %
%    \end{macrocode}
%    \end{macro}
%    \begin{macro}{\ltx@GlobSkipE}
%    \begin{macrocode}
\skipdef\ltx@GlobSkipE=9 %
%    \end{macrocode}
%    \end{macro}
%
%
% \subsection{Argument killers}
%
%    \begin{macro}{\ltx@gobble}
%    \begin{macrocode}
\long\def\ltx@gobble#1{}
%    \end{macrocode}
%    \end{macro}
%    \begin{macro}{\ltx@gobbletwo}
%    \begin{macrocode}
\long\def\ltx@gobbletwo#1#2{}
%    \end{macrocode}
%    \end{macro}
%    \begin{macro}{\ltx@gobblethree}
%    \begin{macrocode}
\long\def\ltx@gobblethree#1#2#3{}
%    \end{macrocode}
%    \end{macro}
%    \begin{macro}{\ltx@gobblefour}
%    \begin{macrocode}
\long\def\ltx@gobblefour#1#2#3#4{}
%    \end{macrocode}
%    \end{macro}
%
%    \begin{macro}{\ltx@GobbleNum}
%    \begin{macrocode}
\def\ltx@GobbleNum#1{%
  \romannumeral
  \csname ltx@zero%
  \expandafter\LTXcmds@GobbleNum
  \romannumeral\LTXcmds@num{#1}000{m\endcsname}%
}
%    \end{macrocode}
%    \end{macro}
%    \begin{macro}{\LTXcmds@GobbleNum}
%    \begin{macrocode}
\def\LTXcmds@GobbleNum#1{%
  \csname LTXcmds@G#1\LTXcmds@GobbleNum
}
%    \end{macrocode}
%    \end{macro}
%    \begin{macro}{\LTXcmds@Gm}
%    \begin{macrocode}
\long\def\LTXcmds@Gm#1{%
  \endcsname
}
%    \end{macrocode}
%    \end{macro}
%
% \subsection{Argument grabbers}
%
%    \begin{macro}{\ltx@firstofone}
%    \begin{macrocode}
\long\def\ltx@firstofone#1{#1}
%    \end{macrocode}
%    \end{macro}
%    \begin{macro}{\ltx@firstoftwo}
%    \begin{macrocode}
\long\def\ltx@firstoftwo#1#2{#1}
%    \end{macrocode}
%    \end{macro}
%    \begin{macro}{\ltx@secondoftwo}
%    \begin{macrocode}
\long\def\ltx@secondoftwo#1#2{#2}
%    \end{macrocode}
%    \end{macro}
%
%    \begin{macro}{\ltx@firstofthree}
%    \begin{macrocode}
\long\def\ltx@firstofthree#1#2#3{#1}
%    \end{macrocode}
%    \end{macro}
%    \begin{macro}{\ltx@secondofthree}
%    \begin{macrocode}
\long\def\ltx@secondofthree#1#2#3{#2}
%    \end{macrocode}
%    \end{macro}
%    \begin{macro}{\ltx@thirdofthree}
%    \begin{macrocode}
\long\def\ltx@thirdofthree#1#2#3{#3}%
%    \end{macrocode}
%    \end{macro}
%
%    \begin{macro}{\ltx@firstoffour}
%    \begin{macrocode}
\long\def\ltx@firstoffour#1#2#3#4{#1}
%    \end{macrocode}
%    \end{macro}
%    \begin{macro}{\ltx@secondoffour}
%    \begin{macrocode}
\long\def\ltx@secondoffour#1#2#3#4{#2}
%    \end{macrocode}
%    \end{macro}
%    \begin{macro}{\ltx@thirdoffour}
%    \begin{macrocode}
\long\def\ltx@thirdoffour#1#2#3#4{#3}%
%    \end{macrocode}
%    \end{macro}
%    \begin{macro}{\ltx@fourthoffour}
%    \begin{macrocode}
\long\def\ltx@fourthoffour#1#2#3#4{#4}%
%    \end{macrocode}
%    \end{macro}
%
% \subsection{List helpers}
%
%    \begin{macro}{\ltx@carzero}
%    \begin{macrocode}
\long\def\ltx@carzero#1\@nil{}%
%    \end{macrocode}
%    \end{macro}
%    \begin{macro}{\LTXcmds@cdrzero}
%    \begin{macrocode}
\long\def\LTXcmds@cdrzero#1\@nil{#1}
%    \end{macrocode}
%    \end{macro}
%    \begin{macro}{\ltx@cdrzero}
%    \begin{macrocode}
\def\ltx@cdrzero{%
  \romannumeral\LTXcmds@cdrzero\ltx@zero
}
%    \end{macrocode}
%    \end{macro}
%
%    \begin{macro}{\ltx@car}
%    \begin{macrocode}
\long\def\ltx@car#1#2\@nil{#1}
%    \end{macrocode}
%    \end{macro}
%    \begin{macro}{\ltx@cdr}
%    \begin{macrocode}
\long\def\ltx@cdr#1{%
  \romannumeral\LTXcmds@cdrzero\ltx@zero
}
%    \end{macrocode}
%    \end{macro}
%
%    \begin{macro}{\ltx@cartwo}
%    \begin{macrocode}
\long\def\ltx@cartwo#1#2#3\@nil{#1#2}
%    \end{macrocode}
%    \end{macro}
%    \begin{macro}{\ltx@carsecond}
%    \begin{macrocode}
\long\def\ltx@carsecond#1#2#3\@nil{#2}
%    \end{macrocode}
%    \end{macro}
%    \begin{macro}{\ltx@cdrtwo}
%    \begin{macrocode}
\long\def\ltx@cdrtwo#1#2{%
  \romannumeral\LTXcmds@cdrzero\ltx@zero
}
%    \end{macrocode}
%    \end{macro}
%
%    \begin{macro}{\ltx@carthree}
%    \begin{macrocode}
\long\def\ltx@carthree#1#2#3#4\@nil{#1#2#3}
%    \end{macrocode}
%    \end{macro}
%    \begin{macro}{\ltx@carthird}
%    \begin{macrocode}
\long\def\ltx@carthird#1#2#3#4\@nil{#3}
%    \end{macrocode}
%    \end{macro}
%    \begin{macro}{\ltx@cdrthree}
%    \begin{macrocode}
\long\def\ltx@cdrthree#1#2#3{%
  \romannumeral\LTXcmds@cdrzero\ltx@zero
}
%    \end{macrocode}
%    \end{macro}
%
%    \begin{macro}{\ltx@carfour}
%    \begin{macrocode}
\long\def\ltx@carfour#1#2#3#4#5\@nil{#1#2#3#4}
%    \end{macrocode}
%    \end{macro}
%    \begin{macro}{\ltx@carfourth}
%    \begin{macrocode}
\long\def\ltx@carfourth#1#2#3#4#5\@nil{#4}
%    \end{macrocode}
%    \end{macro}
%    \begin{macro}{\ltx@cdrfour}
%    \begin{macrocode}
\long\def\ltx@cdrfour#1#2#3#4{%
  \romannumeral\LTXcmds@cdrzero\ltx@zero
}
%    \end{macrocode}
%    \end{macro}
%
%    \begin{macro}{\ltx@CarNum}
%    \begin{macrocode}
\def\ltx@CarNum#1{%
  \romannumeral
  \csname LTXcmds@CarNumFinish%
  \expandafter\LTXcmds@CarNum
  \romannumeral\LTXcmds@num{#1}000{x\endcsname}%
}
%    \end{macrocode}
%    \end{macro}
%    \begin{macro}{\LTXcmds@CarNum}
%    \begin{macrocode}
\def\LTXcmds@CarNum#1{%
  \csname LTXcmds@C#1\LTXcmds@CarNum
}
%    \end{macrocode}
%    \end{macro}
%    \begin{macro}{\LTXcmds@Cm}
%    \begin{macrocode}
\long\def\LTXcmds@Cm#1#2{%
  \endcsname{#1#2}%
}
%    \end{macrocode}
%    \end{macro}
%    \begin{macro}{\LTXcmds@Cx}
%    \begin{macrocode}
\def\LTXcmds@Cx#1{%
  \endcsname{}%
}
%    \end{macrocode}
%    \end{macro}
%    \begin{macro}{\LTXcmds@CarNumFinish}
%    \begin{macrocode}
\long\def\LTXcmds@CarNumFinish#1#2\@nil{%
  \ltx@zero
  #1%
}
%    \end{macrocode}
%    \end{macro}
%
%    \begin{macro}{\ltx@CarNumth}
%    \begin{macrocode}
\def\ltx@CarNumth#1{%
  \romannumeral
  \expandafter\expandafter\expandafter
  \LTXcmds@CarNumth
  \ltx@GobbleNum{#1}{}%
}
%    \end{macrocode}
%    \end{macro}
%    \begin{macro}{\LTXcmds@CarNumth}
%    \begin{macrocode}
\long\def\LTXcmds@CarNumth#1#2\@nil{%
  \ltx@zero
  #1%
}
%    \end{macrocode}
%    \end{macro}
%
%    \begin{macro}{\ltx@CdrNum}
%    \begin{macrocode}
\def\ltx@CdrNum#1{%
  \romannumeral%
  \expandafter\expandafter\expandafter\ltx@cdrzero
  \expandafter\expandafter\expandafter\ltx@zero
  \ltx@GobbleNum{#1}%
}
%    \end{macrocode}
%    \end{macro}
%
% \subsection{Tail recursion}
%
%    \begin{macro}{\ltx@ReturnAfterFi}
%    \begin{macrocode}
\long\def\ltx@ReturnAfterFi#1\fi{\fi#1}
%    \end{macrocode}
%    \end{macro}
%    \begin{macro}{\ltx@ReturnAfterElseFi}
%    \begin{macrocode}
\long\def\ltx@ReturnAfterElseFi#1\else#2\fi{\fi#1}
%    \end{macrocode}
%    \end{macro}
%
% \subsection{Empty macro}
%
%    \begin{macro}{\ltx@empty}
%    \begin{macrocode}
\def\ltx@empty{}
%    \end{macrocode}
%    \end{macro}
%
% \subsection{Characters}
%
%    \begin{macro}{\ltx@space}
%    \begin{macrocode}
\def\ltx@space{ }
%    \end{macrocode}
%    \end{macro}
%    \begin{macro}{\ltx@percentchar}
%    \begin{macrocode}
\begingroup
  \lccode`0=`\%\relax
\lowercase{\endgroup
  \def\ltx@percentchar{0}%
}
%    \end{macrocode}
%    \end{macro}
%    \begin{macro}{\ltx@backslashchar}
%    \begin{macrocode}
\begingroup
  \lccode`0=`\\\relax
\lowercase{\endgroup
  \def\ltx@backslashchar{0}%
}
%    \end{macrocode}
%    \end{macro}
%    \begin{macro}{\ltx@hashchar}
%    \begin{macrocode}
\begingroup
  \lccode`0=`\#\relax
\lowercase{\endgroup
  \def\ltx@hashchar{0}%
}
%    \end{macrocode}
%    \end{macro}
%    \begin{macro}{\ltx@leftbracechar}
%    \begin{macrocode}
\begingroup
  \lccode`0=`\{\relax
\lowercase{\endgroup
  \def\ltx@leftbracechar{0}%
}
%    \end{macrocode}
%    \end{macro}
%    \begin{macro}{\ltx@rightbracechar}
%    \begin{macrocode}
\begingroup
  \lccode`0=`\}\relax
\lowercase{\endgroup
  \def\ltx@rightbracechar{0}%
}
%    \end{macrocode}
%    \end{macro}
%
% \subsection{Boolean switch}
%
%    \begin{macro}{\ltx@newif}
%    \begin{macrocode}
\def\ltx@newif#1{%
  \begingroup
    \escapechar=-1 %
  \expandafter\endgroup
  \expandafter\LTXcmds@newif\string#1\@nil
}
%    \end{macrocode}
%    \end{macro}
%    \begin{macro}{\LTXcmds@newif}
%    \begin{macrocode}
\begingroup
  \escapechar=-1 %
\expandafter\endgroup
\expandafter\def\expandafter\LTXcmds@newif\string\if#1\@nil{%
  \expandafter\edef\csname#1true\endcsname{%
    \let
    \expandafter\noexpand\csname if#1\endcsname
    \noexpand\iftrue
  }%
  \expandafter\edef\csname#1false\endcsname{%
    \let
    \expandafter\noexpand\csname if#1\endcsname
    \noexpand\iffalse
  }%
  \csname#1false\endcsname
}
%    \end{macrocode}
%    \end{macro}
%
%    \begin{macro}{\ltx@newglobalif}
%    \begin{macrocode}
\def\ltx@newglobalif#1{%
  \begingroup
    \escapechar=-1 %
  \expandafter\endgroup
  \expandafter\LTXcmds@newglobalif\string#1\@nil
}
%    \end{macrocode}
%    \end{macro}
%    \begin{macro}{\LTXcmds@newglobalif}
%    \begin{macrocode}
\begingroup
  \escapechar=-1 %
\expandafter\endgroup
\expandafter
\def\expandafter\LTXcmds@newglobalif\string\if#1\@nil{%
  \expandafter\edef\csname#1true\endcsname{%
    \global\let
    \expandafter\noexpand\csname if#1\endcsname
    \noexpand\iftrue
  }%
  \expandafter\edef\csname#1false\endcsname{%
    \global\let
    \expandafter\noexpand\csname if#1\endcsname
    \noexpand\iffalse
  }%
  \csname#1false\endcsname
}
%    \end{macrocode}
%    \end{macro}
%
% \subsection{Command definitions}
%
%    \begin{macro}{\ltx@LocalExpandAfter}
%    \begin{macrocode}
\def\ltx@LocalExpandAfter{%
  \begingroup
    \expandafter\expandafter\expandafter
  \endgroup
  \expandafter
}
%    \end{macrocode}
%    \end{macro}
%
%    \begin{macrocode}
\ltx@LocalExpandAfter
\ifx\csname ifcsname\endcsname\relax
%    \end{macrocode}
%    \begin{macro}{\ltx@ifundefined}
%    \begin{macrocode}
  \def\ltx@ifundefined#1{%
    \expandafter\ifx\csname #1\endcsname\relax
      \expandafter\ltx@firstoftwo
    \else
      \expandafter\ltx@secondoftwo
    \fi
  }%
%    \end{macrocode}
%    \end{macro}
%    \begin{macro}{\ltx@IfUndefined}
%    \begin{macrocode}
  \def\ltx@IfUndefined#1{%
    \begingroup\expandafter\expandafter\expandafter\endgroup
    \expandafter\ifx\csname #1\endcsname\relax
      \expandafter\ltx@firstoftwo
    \else
      \expandafter\ltx@secondoftwo
    \fi
  }%
%    \end{macrocode}
%    \end{macro}
%    \begin{macrocode}
  \expandafter\ltx@gobble
\else
  \expandafter\ltx@firstofone
\fi
{%
%    \end{macrocode}
%    \begin{macro}{\ltx@ifundefined}
%    \begin{macrocode}
  \def\ltx@ifundefined#1{%
    \ifcsname #1\endcsname
      \expandafter\ifx\csname #1\endcsname\relax
        \expandafter\expandafter\expandafter\ltx@firstoftwo
      \else
        \expandafter\expandafter\expandafter\ltx@secondoftwo
      \fi
    \else
      \expandafter\ltx@firstoftwo
    \fi
  }%
%    \end{macrocode}
%    \end{macro}
%    \begin{macro}{\ltx@IfUndefined}
%    \begin{macrocode}
  \let\ltx@IfUndefined\ltx@ifundefined
%    \end{macrocode}
%    \end{macro}
%    \begin{macrocode}
}
%    \end{macrocode}
%
% \subsection{Stripping}
%
%    \begin{macro}{\ltx@RemovePrefix}
%    \begin{macrocode}
\def\ltx@RemovePrefix#1>{}
%    \end{macrocode}
%    \end{macro}
%
%    \begin{macro}{\ltx@StripPrefix}
%    \begin{macrocode}
\def\ltx@StripPrefix{%
  \expandafter\ltx@RemovePrefix
}
%    \end{macrocode}
%    \end{macro}
%
%    \begin{macro}{\ltx@onelevel@sanitize}
%    \begin{macrocode}
\def\ltx@onelevel@sanitize#1{%
  \edef#1{%
    \expandafter
    \ltx@RemovePrefix\meaning#1%
  }%
}
%    \end{macrocode}
%    \end{macro}
%
% \subsection{File management}
%
% \subsubsection{File extensions}
%
%    \begin{macro}{\ltx@clsextension}
%    \begin{macrocode}
\def\ltx@clsextension{cls}
%    \end{macrocode}
%    \end{macro}
%    \begin{macro}{\ltx@pkgextension}
%    \begin{macrocode}
\def\ltx@pkgextension{sty}
%    \end{macrocode}
%    \end{macro}
%
% \subsubsection{Load check}
%
%    \begin{macro}{\ltx@iffileloaded}
%    \begin{macrocode}
\def\ltx@iffileloaded#1{%
  \ltx@ifundefined{ver@#1}\ltx@secondoftwo\ltx@firstoftwo
}
%    \end{macrocode}
%    \end{macro}
%    \begin{macro}{\ltx@ifclassloaded}
%    \begin{macrocode}
\def\ltx@ifclassloaded#1{%
  \ltx@iffileloaded{#1.\ltx@clsextension}%
}
%    \end{macrocode}
%    \end{macro}
%    \begin{macro}{\ltx@ifpackageloaded}
%    \begin{macrocode}
\def\ltx@ifpackageloaded#1{%
  \ltx@iffileloaded{#1.\ltx@pkgextension}%
}
%    \end{macrocode}
%    \end{macro}
%
% \subsubsection{Version date check}
%
%    \begin{macro}{\ltx@iffilelater}
%    \begin{macrocode}
\def\ltx@iffilelater#1#2{%
  \ltx@iffileloaded{#1}{%
    \expandafter\LTXcmds@IfLater\expandafter{%
      \number
      \expandafter\expandafter\expandafter\LTXcmds@ParseVersion
      \expandafter\expandafter\expandafter{%
        \csname ver@#1\endcsname
      }%
    \expandafter}\expandafter{%
      \number
      \expandafter\LTXcmds@ParseVersion\expandafter{#2}%
    }%
  }\ltx@secondoftwo
}
%    \end{macrocode}
%    \end{macro}
%    \begin{macro}{\LTXcmds@IfLater}
%    \begin{macrocode}
\def\LTXcmds@IfLater#1#2{%
  \ifcase 0%
      \ifnum#1<19940101 %
      \else
        \ifnum#2<19940101 %
        \else
          \ifnum#2>#1 %
          \else
            1%
          \fi
        \fi
      \fi
      \ltx@space
    \expandafter\ltx@secondoftwo
  \else
    \expandafter\ltx@firstoftwo
  \fi
}
%    \end{macrocode}
%    \end{macro}
%    \begin{macro}{\ltx@ifclasslater}
%    \begin{macrocode}
\def\ltx@ifclasslater#1{%
  \ltx@iffilelater{#1.\ltx@clsextension}%
}
%    \end{macrocode}
%    \end{macro}
%    \begin{macro}{\ltx@ifpackagelater}
%    \begin{macrocode}
\def\ltx@ifpackagelater#1{%
  \ltx@iffilelater{#1.\ltx@pkgextension}%
}
%    \end{macrocode}
%    \end{macro}
%
%    \begin{macrocode}
\ltx@IfUndefined{pdfmatch}{%
%    \end{macrocode}
%    \begin{macro}{\LTXcmds@ParseVersion}
%    \begin{macrocode}
  \def\LTXcmds@ParseVersion#1{%
    \LTXcmds@@ParseVersion#10000/00/00\@nil
  }%
%    \end{macrocode}
%    \end{macro}
%    \begin{macro}{\LTXcmds@@ParseVersion}
%    \begin{macrocode}
  \def\LTXcmds@@ParseVersion#1#2#3#4/#5#6/#7#8#9\@nil{%
    #1#2#3#4#5#6#7#8%
  }%
%    \end{macrocode}
%    \end{macro}
%    \begin{macrocode}
}{%
%    \end{macrocode}
%    \begin{macro}{\LTXcmds@ParseVersion}
%    \begin{macrocode}
  \def\LTXcmds@ParseVersion#1{%
    \ifnum\pdfmatch{%
      ^%
      (199[4-9]|[2-9][0-9][0-9][0-9])/%
      (0[1-9]|1[0-2])/%
      (0[1-9]|[1-2][0-9]|3[0-1])%
    }{#1}=1 %
      \ltx@StripPrefix\pdflastmatch1 %
      \ltx@StripPrefix\pdflastmatch2 %
      \ltx@StripPrefix\pdflastmatch3 %
    \else
      0%
    \fi
  }%
%    \end{macrocode}
%    \end{macro}
%    \begin{macrocode}
}
%    \end{macrocode}
%
% \subsection{Macro additions}
%
%    \begin{macro}{\ltx@GlobalAppendToMacro}
%    \begin{macrocode}
\long\def\ltx@GlobalAppendToMacro#1#2{%
  \ifx\ltx@undefined#1%
    \let#1\ltx@empty
  \else
    \ifx\relax#1%
      \let#1\ltx@empty
    \fi
  \fi
  \begingroup
    \ltx@LocToksA\expandafter{#1#2}%
    \xdef#1{\the\ltx@LocToksA}%
  \endgroup
}
%    \end{macrocode}
%    \end{macro}
%    \begin{macro}{\ltx@LocalAppendToMacro}
%    \begin{macrocode}
\long\def\ltx@LocalAppendToMacro#1#2{%
  \global\let\LTXcmds@gtemp#1%
  \ifx\ltx@undefined\LTXcmds@gtemp
    \global\let\LTXcmds@gtemp\ltx@empty
  \else
    \ifx\relax\LTXcmds@gtemp
      \global\letLTXcmds@gtemp\ltx@empty
    \fi
  \fi
  \begingroup
    \ltx@LocToksA\expandafter{\LTXcmds@gtemp#2}%
    \xdef\LTXcmds@gtemp{\the\ltx@LocToksA}%
  \endgroup
  \let#1\LTXcmds@gtemp
}
%    \end{macrocode}
%    \end{macro}
%
%    \begin{macro}{\ltx@GlobalPrependToMacro}
%    \begin{macrocode}
\long\def\ltx@GlobalPrependToMacro#1#2{%
  \ifx\ltx@undefined#1%
    \let#1\ltx@empty
  \else
    \ifx\relax#1%
      \let#1\ltx@empty
    \fi
  \fi
  \begingroup
    \ltx@LocToksA{#2}%
    \ltx@LocToksB\expandafter{#1}%
    \xdef#1{\the\ltx@LocToksA\the\ltx@LocToksB}%
  \endgroup
}
%    \end{macrocode}
%    \end{macro}
%    \begin{macro}{\ltx@LocalPrependToMacro}
%    \begin{macrocode}
\long\def\ltx@LocalPrependToMacro#1#2{%
  \global\let\LTXcmds@gtemp#1%
  \ifx\ltx@undefined\LTXcmds@gtemp
    \global\let\LTXcmds@gtemp\ltx@empty
  \else
    \ifx\relax\LTXcmds@gtemp
      \global\letLTXcmds@gtemp\ltx@empty
    \fi
  \fi
  \begingroup
    \ltx@LocToksA{#2}%
    \ltx@LocToksB\expandafter{\LTXcmds@gtemp}%
    \xdef\LTXcmds@gtemp{\the\ltx@LocToksA\the\ltx@LocToksB}%
  \endgroup
  \let#1\LTXcmds@gtemp
}
%    \end{macrocode}
%    \end{macro}
%
% \subsection{Next character detection}
%
%    \begin{macro}{\ltx@ifnextchar}
%    \begin{macrocode}
\long\def\ltx@ifnextchar#1#2#3{%
  \begingroup
  \let\LTXcmds@CharToken= #1\relax
  \ltx@LocToksA{\endgroup#2}%
  \ltx@LocToksB{\endgroup#3}%
  \futurelet\LTXcmds@LetToken\LTXcmds@ifnextchar
}
%    \end{macrocode}
%    \end{macro}
%    \begin{macro}{\LTXcmds@ifnextchar}
%    \begin{macrocode}
\def\LTXcmds@ifnextchar{%
  \ifx\LTXcmds@LetToken\LTXcmds@CharToken
    \the\expandafter\ltx@LocToksA
  \else
    \expandafter
      \ifx\csname LTXcmds@LetToken\endcsname\LTXcmds@SpaceToken
      \expandafter\expandafter\expandafter\LTXcmds@@ifnextchar
    \else
      \the\expandafter\expandafter\expandafter\ltx@LocToksB
    \fi
  \fi
}
%    \end{macrocode}
%    \end{macro}
%    \begin{macro}{\LTXcmds@@ifnextchar}
%    \cs{futurelet} does not distinguish between a
%    character and a command that is a character
%    (defined by using \cs{let} or \cs{futurelet}).
%    Therefore the space is catched by \cs{romannumeral}
%    with negative character constant that gobbles one
%    optional space.
%    \begin{macrocode}
\def\LTXcmds@@ifnextchar{%
  \expandafter\futurelet
  \expandafter\LTXcmds@LetToken
  \expandafter\LTXcmds@ifnextchar
  \romannumeral-`\.%
}
%    \end{macrocode}
%    \end{macro}
%    \begin{macro}{\LTXcmds@SpaceToken}
%    \begin{macrocode}
\ltx@firstofone{\let\LTXcmds@SpaceToken= } %
%    \end{macrocode}
%    \end{macro}
%
%    \begin{macro}{\ltx@ifnextchar@nospace}
%    \begin{macrocode}
\long\def\ltx@ifnextchar@nospace#1#2#3{%
  \begingroup
  \let\LTXcmds@CharToken= #1\relax
  \ltx@LocToksA{\endgroup#2}%
  \ltx@LocToksB{\endgroup#3}%
  \futurelet\LTXcmds@LetToken\LTXcmds@ifnextchar@nospace
}
%    \end{macrocode}
%    \end{macro}
%    \begin{macro}{\LTXcmds@ifnextchar@nospace}
%    \begin{macrocode}
\def\LTXcmds@ifnextchar@nospace{%
  \the
  \ifx\LTXcmds@LetToken\LTXcmds@CharToken
    \expandafter\ltx@LocToksA
  \else
    \expandafter\ltx@LocToksB
  \fi
}
%    \end{macrocode}
%    \end{macro}
% \subsection{\cs{ltx@leavevmode}, \cs{ltx@mbox}}
%
%    \begin{macro}{\ltx@leavevmode}
%    \begin{macrocode}
\ltx@IfUndefined{quitvmode}{%
  \ltx@IfUndefined{leavevmode}{%
    \ltx@IfUndefined{voidb@x}{%
      \ltx@IfUndefined{newbox}{%
        \def\ltx@leavevmode{%
          \begingroup
            \setbox\ltx@zero=\hbox{}%
            \begingroup
              \setbox\ltx@zero=\hbox{\box\ltx@zero}%
            \endgroup
            \unhbox\ltx@zero
          \endgroup
        }%
      }{%
        \csname newbox\endcsname\LTXcmds@VoidBox
        \ifvoid\LTXcmds@VoidBox
        \else
          \setbox\LTXcmds@VoidBox=\hbox{}%
          \begingroup
            \setbox\LTXcmds@VoidBox=\hbox{\box\LTXcmds@VoidBox}%
          \endgroup
        \fi
        \def\ltx@leavevmode{\unhbox\LTXcmds@VoidBox}%
      }%
    }{%
      \def\ltx@leavevmode{\unhbox\voidb@x}%
    }%
  }{%
    \let\ltx@leavevmode\leavevmode
  }%
}{%
  \let\ltx@leavevmode\quitvmode
}
%    \end{macrocode}
%    \end{macro}
%
%    \begin{macro}{\ltx@mbox}
%    \begin{macrocode}
\def\ltx@mbox{%
  \ltx@leavevmode
  \hbox
}
%    \end{macrocode}
%    \end{macro}
%
% \subsection{Help macros}
%
%    \begin{macro}{\LTXcmds@num}
%    \begin{macrocode}
\ltx@IfUndefined{numexpr}{%
  \def\LTXcmds@num#1{%
    \expandafter\ltx@firstofone\expandafter{%
      \number#1%
    }%
  }%
}{%
  \def\LTXcmds@num#1{%
    \expandafter\ltx@firstofone\expandafter{%
      \the\numexpr#1%
    }%
  }%
}
%    \end{macrocode}
%    \end{macro}
%
% \subsection{Expandable test for emptiness}
%
%    \begin{macrocode}
\ltx@IfUndefined{detokenize}{%
%    \end{macrocode}
%
% \subsubsection{Vanilla \hologo{TeX}}
%
%    \begin{macro}{\ltx@ifempty}
%    The macro is based on \cs{@ifempty} of Robert R. Schneck
%    \cite{ifempty:schneck}
%    and \cs{@ifnull} of Ulrich Diez \cite{ifnull:diez}.
%    There are three cases to consider:
%    \begin{compactenum}
%    \item |#1| is empty,
%    \item |#1| is not empty and the first token
%      is not a begingroup character,
%    \item |#1| starts with a begingroup character (catcode 1).
%    \end{compactenum}
%    \begin{macrocode}
  \def\LTXcmds@temp#1{%
    \long\def\ltx@ifempty##1{%
      \romannumeral0%
      \iffalse{\fi
        \expandafter\ltx@gobble\expandafter{%
          \expandafter{\string##1}%
          \expandafter\ltx@gobble\string
        }%
        \expandafter\ltx@firstofthree\expandafter
        {\iffalse}\fi
        \expandafter#1\ltx@secondoftwo
      }%
      \expandafter#1\ltx@firstoftwo
    }%
%    \end{macrocode}
%    \end{macro}
%    \begin{macro}{\ltx@ifblank}
%    \begin{macrocode}
    \long\def\ltx@ifblank##1{%
      \romannumeral0%
      \iffalse{\fi
        \expandafter\expandafter\expandafter\ltx@gobble
        \expandafter\expandafter\expandafter{%
          \expandafter\expandafter\expandafter{%
            \expandafter\string\ltx@gobble##1.%
          }%
          \expandafter\ltx@gobble\string
        }%
        \expandafter\ltx@firstofthree\expandafter
        {\iffalse}\fi
        \expandafter#1\ltx@secondoftwo
      }%
      \expandafter#1\ltx@firstoftwo
    }%
  }%
  \LTXcmds@temp{ }%
%    \end{macrocode}
%    \end{macro}
%    \begin{macrocode}
}{%
%    \end{macrocode}
%
% \subsubsection{With \cs{detokenize}}
%
%    Ahmed Musa provided \cs{ifstrempty} using \cs{detokenize}
%    and \cs{pdfstrcmp} \cite{ifstrempty:musa}. Ulrich Diez, GL,
%    Heiko Oberdiek
%    improved it further by removing \cs{pdfstrcmp} and
%    taking three arguments
%    \cite{ifstrempty:diez1,ifstrempty:diez2,ifstrempty:diez3,^^A
%      ifstrempty:gl,ifstrempty:ho}.
%
%    \begin{macro}{\ltx@ifempty}
%    \begin{macrocode}
  \long\def\ltx@ifempty#1{%
    \romannumeral%
    \csname
      LTXcmds@ifempty%
      \ifcat$\detokenize{#1}$%
        @%
      \fi
    \endcsname
  }%
%    \end{macrocode}
%    \end{macro}
%    \begin{macro}{\LTXcmds@ifempty@}
%    \begin{macrocode}
  \long\def\LTXcmds@ifempty@#1#2{0 #1}%
%    \end{macrocode}
%    \end{macro}
%    \begin{macro}{\LTXcmds@ifempty}
%    \begin{macrocode}
  \long\def\LTXcmds@ifempty#1#2{0 #2}%
%    \end{macrocode}
%    \end{macro}
%
% \subsubsection{\cs{ltx@ifblank}}
%
%    \begin{macro}{\ltx@ifblank}
%    \begin{macrocode}
  \long\def\ltx@ifblank#1{%
    \romannumeral%
    \csname
      LTXcmds@ifempty%
      \ifcat$\detokenize\expandafter{\ltx@gobble#1.}$%
        @%
      \fi
    \endcsname
  }%
%    \end{macrocode}
%    \end{macro}
%
%    \begin{macrocode}
}
%    \end{macrocode}
%
% \subsection{\cs{ltx@zapspace}}
%
%    \begin{macro}{\ltx@zapspace}
%    \begin{macrocode}
\long\def\ltx@zapspace#1{%
  \romannumeral
  \LTXcmds@zapspace\ltx@zero#1 \@nil
}
%    \end{macrocode}
%    \end{macro}
%    \begin{macro}{\LTXcmds@zapspace}
%    \begin{macrocode}
\long\def\LTXcmds@zapspace#1 #2\@nil{%
  \ltx@ifempty{#2}{%
    #1%
  }{%
    \LTXcmds@zapspace#1#2\@nil
  }%
}
%    \end{macrocode}
%    \end{macro}
%
% \subsection{\cs{ltx@IfBoxEmpty}}
%
%    In case of \hologo{eTeX} the test for an empty box
%    is done via \cs{lastnodetype} as suggested
%    by David Kastrup \cite{ifboxempty:dak}.
%    \begin{macrocode}
\ltx@IfUndefined{lastnodetype}{%
  \catcode`\$=9 %
  \catcode`\&=14 %
}{%
  \catcode`\$=14 %
  \catcode`\&=9 %
}
%    \end{macrocode}
%    \begin{macro}{\ltx@IfBoxEmpty}
%    \begin{macrocode}
\def\ltx@IfBoxEmpty#1{%
  \ifvoid#1\relax
    \expandafter\ltx@secondoftwo
  \else
%    \end{macrocode}
%    Implementation using \hologo{eTeX}'s \cs{lastnodetype}.
%    \begin{macrocode}
&   \begingroup
&     \setbox\ltx@zero=\ifhbox#1\hbox\else\vbox\fi{%
&       \ifhmode\unhcopy\else\unvcopy\fi#1\relax
&       \expandafter
&     }%
&   \expandafter\endgroup
&   \ifnum\lastnodetype<\ltx@zero
&     \expandafter\expandafter\expandafter\ltx@firstoftwo
&   \else
&     \expandafter\expandafter\expandafter\ltx@secondoftwo
&   \fi
%    \end{macrocode}
%    Implementation without \hologo{eTeX} using a signature
%    at the beginning of the test box.
%    \begin{macrocode}
$   \begingroup
$     \setbox\ltx@zero=\ifhbox#1\hbox\else\vbox\fi{%
$       \penalty\ltx@one
$       \ifhmode\unhcopy\else\unvcopy\fi#1\relax
$       \expandafter
$     }%
$     \ifnum\lastpenalty=\ltx@one
%    \end{macrocode}
%    Box 0 has been changed and is restored by
%    closing the group.
%    \begin{macrocode}
$       \endgroup
$       \begingroup
$       \setbox\ltx@zero=\ifhbox#1\hbox\else\vbox\fi{%
$         \penalty\ltx@two
$         \ifhmode\unhcopy\else\unvcopy\fi#1\relax
$         \expandafter
$       }%
$       \ifnum\lastpenalty=\ltx@two
$         \def\next{\endgroup\expandafter\ltx@firstoftwo}%
$       \else
$         \def\next{\endgroup\expandafter\ltx@secondoftwo}%
$       \fi
$     \else
$       \def\next{\endgroup\expandafter\ltx@secondoftwo}%
$     \fi
$   \next
  \fi
}
%    \end{macrocode}
%    \end{macro}
%
%    \begin{macro}{\ltx@IfBoxVoidOrEmpty}
%    \begin{macrocode}
\def\ltx@IfBoxVoidOrEmpty#1{%
  \ifvoid#1\relax
    \expandafter\ltx@thirdoffour
  \fi
  \ltx@IfBoxEmpty{#1}%
}
%    \end{macrocode}
%    \end{macro}
%
%    \begin{macrocode}
\LTXcmds@AtEnd%
%</package>
%    \end{macrocode}
%% \section{Installation}
%
% \subsection{Download}
%
% \paragraph{Package.} This package is available on
% CTAN\footnote{\CTANpkg{ltxcmds}}:
% \begin{description}
% \item[\CTAN{macros/latex/contrib/ltxcmds/ltxcmds.dtx}] The source file.
% \item[\CTAN{macros/latex/contrib/ltxcmds/ltxcmds.pdf}] Documentation.
% \end{description}
%
%
% \paragraph{Bundle.} All the packages of the bundle `ltxcmds'
% are also available in a TDS compliant ZIP archive. There
% the packages are already unpacked and the documentation files
% are generated. The files and directories obey the TDS standard.
% \begin{description}
% \item[\CTANinstall{install/macros/latex/contrib/ltxcmds.tds.zip}]
% \end{description}
% \emph{TDS} refers to the standard ``A Directory Structure
% for \TeX\ Files'' (\CTANpkg{tds}). Directories
% with \xfile{texmf} in their name are usually organized this way.
%
% \subsection{Bundle installation}
%
% \paragraph{Unpacking.} Unpack the \xfile{ltxcmds.tds.zip} in the
% TDS tree (also known as \xfile{texmf} tree) of your choice.
% Example (linux):
% \begin{quote}
%   |unzip ltxcmds.tds.zip -d ~/texmf|
% \end{quote}
%
% \subsection{Package installation}
%
% \paragraph{Unpacking.} The \xfile{.dtx} file is a self-extracting
% \docstrip\ archive. The files are extracted by running the
% \xfile{.dtx} through \plainTeX:
% \begin{quote}
%   \verb|tex ltxcmds.dtx|
% \end{quote}
%
% \paragraph{TDS.} Now the different files must be moved into
% the different directories in your installation TDS tree
% (also known as \xfile{texmf} tree):
% \begin{quote}
% \def\t{^^A
% \begin{tabular}{@{}>{\ttfamily}l@{ $\rightarrow$ }>{\ttfamily}l@{}}
%   ltxcmds.sty & tex/generic/ltxcmds/ltxcmds.sty\\
%   ltxcmds.pdf & doc/latex/ltxcmds/ltxcmds.pdf\\
%   ltxcmds.dtx & source/latex/ltxcmds/ltxcmds.dtx\\
% \end{tabular}^^A
% }^^A
% \sbox0{\t}^^A
% \ifdim\wd0>\linewidth
%   \begingroup
%     \advance\linewidth by\leftmargin
%     \advance\linewidth by\rightmargin
%   \edef\x{\endgroup
%     \def\noexpand\lw{\the\linewidth}^^A
%   }\x
%   \def\lwbox{^^A
%     \leavevmode
%     \hbox to \linewidth{^^A
%       \kern-\leftmargin\relax
%       \hss
%       \usebox0
%       \hss
%       \kern-\rightmargin\relax
%     }^^A
%   }^^A
%   \ifdim\wd0>\lw
%     \sbox0{\small\t}^^A
%     \ifdim\wd0>\linewidth
%       \ifdim\wd0>\lw
%         \sbox0{\footnotesize\t}^^A
%         \ifdim\wd0>\linewidth
%           \ifdim\wd0>\lw
%             \sbox0{\scriptsize\t}^^A
%             \ifdim\wd0>\linewidth
%               \ifdim\wd0>\lw
%                 \sbox0{\tiny\t}^^A
%                 \ifdim\wd0>\linewidth
%                   \lwbox
%                 \else
%                   \usebox0
%                 \fi
%               \else
%                 \lwbox
%               \fi
%             \else
%               \usebox0
%             \fi
%           \else
%             \lwbox
%           \fi
%         \else
%           \usebox0
%         \fi
%       \else
%         \lwbox
%       \fi
%     \else
%       \usebox0
%     \fi
%   \else
%     \lwbox
%   \fi
% \else
%   \usebox0
% \fi
% \end{quote}
% If you have a \xfile{docstrip.cfg} that configures and enables \docstrip's
% TDS installing feature, then some files can already be in the right
% place, see the documentation of \docstrip.
%
% \subsection{Refresh file name databases}
%
% If your \TeX~distribution
% (\TeX\,Live, \mikTeX, \dots) relies on file name databases, you must refresh
% these. For example, \TeX\,Live\ users run \verb|texhash| or
% \verb|mktexlsr|.
%
% \subsection{Some details for the interested}
%
% \paragraph{Unpacking with \LaTeX.}
% The \xfile{.dtx} chooses its action depending on the format:
% \begin{description}
% \item[\plainTeX:] Run \docstrip\ and extract the files.
% \item[\LaTeX:] Generate the documentation.
% \end{description}
% If you insist on using \LaTeX\ for \docstrip\ (really,
% \docstrip\ does not need \LaTeX), then inform the autodetect routine
% about your intention:
% \begin{quote}
%   \verb|latex \let\install=y% \iffalse meta-comment
%
% File: ltxcmds.dtx
% Version: 2019/12/15 v1.24
% Info: LaTeX kernel commands for general use
%
% Copyright (C)
%    2009-2011 Heiko Oberdiek
%    2016-2019 Oberdiek Package Support Group
%    https://github.com/ho-tex/ltxcmds/issues
%
% This work may be distributed and/or modified under the
% conditions of the LaTeX Project Public License, either
% version 1.3c of this license or (at your option) any later
% version. This version of this license is in
%    https://www.latex-project.org/lppl/lppl-1-3c.txt
% and the latest version of this license is in
%    https://www.latex-project.org/lppl.txt
% and version 1.3 or later is part of all distributions of
% LaTeX version 2005/12/01 or later.
%
% This work has the LPPL maintenance status "maintained".
%
% The Current Maintainers of this work are
% Heiko Oberdiek and the Oberdiek Package Support Group
% https://github.com/ho-tex/ltxcmds/issues
%
% The Base Interpreter refers to any `TeX-Format',
% because some files are installed in TDS:tex/generic//.
%
% This work consists of the main source file ltxcmds.dtx
% and the derived files
%    ltxcmds.sty, ltxcmds.pdf, ltxcmds.ins, ltxcmds.drv,
%    ltxcmds-test1.tex, ltxcmds-test-gobble.tex,
%    ltxcmds-test-ifempty.tex, ltxcmds-test-zapspace.tex,
%    ltxcmds-test-ifboxempty.tex, ltxcmds-test-nextchar.tex,
%    ltxcmds-test-carcdr.tex.
%
% Distribution:
%    CTAN:macros/latex/contrib/ltxcmds/ltxcmds.dtx
%    CTAN:macros/latex/contrib/ltxcmds/ltxcmds.pdf
%
% Unpacking:
%    (a) If ltxcmds.ins is present:
%           tex ltxcmds.ins
%    (b) Without ltxcmds.ins:
%           tex ltxcmds.dtx
%    (c) If you insist on using LaTeX
%           latex \let\install=y\input{ltxcmds.dtx}
%        (quote the arguments according to the demands of your shell)
%
% Documentation:
%    (a) If ltxcmds.drv is present:
%           latex ltxcmds.drv
%    (b) Without ltxcmds.drv:
%           latex ltxcmds.dtx; ...
%    The class ltxdoc loads the configuration file ltxdoc.cfg
%    if available. Here you can specify further options, e.g.
%    use A4 as paper format:
%       \PassOptionsToClass{a4paper}{article}
%
%    Programm calls to get the documentation (example):
%       pdflatex ltxcmds.dtx
%       makeindex -s gind.ist ltxcmds.idx
%       pdflatex ltxcmds.dtx
%       makeindex -s gind.ist ltxcmds.idx
%       pdflatex ltxcmds.dtx
%
% Installation:
%    TDS:tex/generic/ltxcmds/ltxcmds.sty
%    TDS:doc/latex/ltxcmds/ltxcmds.pdf
%    TDS:source/latex/ltxcmds/ltxcmds.dtx
%
%<*ignore>
\begingroup
  \catcode123=1 %
  \catcode125=2 %
  \def\x{LaTeX2e}%
\expandafter\endgroup
\ifcase 0\ifx\install y1\fi\expandafter
         \ifx\csname processbatchFile\endcsname\relax\else1\fi
         \ifx\fmtname\x\else 1\fi\relax
\else\csname fi\endcsname
%</ignore>
%<*install>
\input docstrip.tex
\Msg{************************************************************************}
\Msg{* Installation}
\Msg{* Package: ltxcmds 2019/12/15 v1.24 LaTeX kernel commands for general use (HO)}
\Msg{************************************************************************}

\keepsilent
\askforoverwritefalse

\let\MetaPrefix\relax
\preamble

This is a generated file.

Project: ltxcmds
Version: 2019/12/15 v1.24

Copyright (C)
   2009-2011 Heiko Oberdiek
   2016-2019 Oberdiek Package Support Group

This work may be distributed and/or modified under the
conditions of the LaTeX Project Public License, either
version 1.3c of this license or (at your option) any later
version. This version of this license is in
   https://www.latex-project.org/lppl/lppl-1-3c.txt
and the latest version of this license is in
   https://www.latex-project.org/lppl.txt
and version 1.3 or later is part of all distributions of
LaTeX version 2005/12/01 or later.

This work has the LPPL maintenance status "maintained".

The Current Maintainers of this work are
Heiko Oberdiek and the Oberdiek Package Support Group
https://github.com/ho-tex/ltxcmds/issues


The Base Interpreter refers to any `TeX-Format',
because some files are installed in TDS:tex/generic//.

This work consists of the main source file ltxcmds.dtx
and the derived files
   ltxcmds.sty, ltxcmds.pdf, ltxcmds.ins, ltxcmds.drv,
   ltxcmds-test1.tex, ltxcmds-test-gobble.tex,
   ltxcmds-test-ifempty.tex, ltxcmds-test-zapspace.tex,
   ltxcmds-test-ifboxempty.tex, ltxcmds-test-nextchar.tex,
   ltxcmds-test-carcdr.tex.

\endpreamble
\let\MetaPrefix\DoubleperCent

\generate{%
  \file{ltxcmds.ins}{\from{ltxcmds.dtx}{install}}%
  \file{ltxcmds.drv}{\from{ltxcmds.dtx}{driver}}%
  \usedir{tex/generic/ltxcmds}%
  \file{ltxcmds.sty}{\from{ltxcmds.dtx}{package}}%
}

\catcode32=13\relax% active space
\let =\space%
\Msg{************************************************************************}
\Msg{*}
\Msg{* To finish the installation you have to move the following}
\Msg{* file into a directory searched by TeX:}
\Msg{*}
\Msg{*     ltxcmds.sty}
\Msg{*}
\Msg{* To produce the documentation run the file `ltxcmds.drv'}
\Msg{* through LaTeX.}
\Msg{*}
\Msg{* Happy TeXing!}
\Msg{*}
\Msg{************************************************************************}

\endbatchfile
%</install>
%<*ignore>
\fi
%</ignore>
%<*driver>
\NeedsTeXFormat{LaTeX2e}
\ProvidesFile{ltxcmds.drv}%
  [2019/12/15 v1.24 LaTeX kernel commands for general use (HO)]%
\documentclass{ltxdoc}
\usepackage{holtxdoc}[2011/11/22]
\usepackage{zref-savepos}
\usepackage{paralist}
\hfuzz=1pt
\begin{document}
  \DocInput{ltxcmds.dtx}%
\end{document}
%</driver>
% \fi
%
%
%
% \GetFileInfo{ltxcmds.drv}
%
% \title{The \xpackage{ltxcmds} package}
% \date{2019/12/15 v1.24}
% \author{Heiko Oberdiek\thanks
% {Please report any issues at \url{https://github.com/ho-tex/ltxcmds/issues}}}
%
% \maketitle
%
% \begin{abstract}
% The package \xpackage{ltxcmds} exports some utility macros
% from the \LaTeX\ kernel into a separate namespace and
% also provides them for other formats such as plain-\TeX.
% \end{abstract}
%
% \tableofcontents
%
% \newcounter{to}
% \renewcommand*{\theto}{to\arabic{to}}
% \newlength\tolength
% \tolength=\dimexpr .5\linewidth+\oddsidemargin+1in+15mm\relax
% \newcommand*{\TO}{^^A
%   \quad
%   \stepcounter{to}^^A
%   \zsavepos\theto
%   \ifdim\zposx\theto sp>0pt
%     \ifdim\zposx\theto sp<\tolength
%       \kern\dimexpr\tolength-\zposx\theto sp\relax
%     \fi
%   \fi
%   $\rightarrow$\quad
%   \ignorespaces
% }
%
% \section{Documentation}
%
% \subsection{Introduction}
%
% Many of my packages also support other formats such as plain-\TeX.
% Because I am rather familiar with the utility macros from
% \LaTeX's kernel (e.g. \cs{@gobble}, \cs{@firstoftwo}), I found
% myself rewriting them again and again, because they are lacking
% in plain-\TeX.
%
% Therefore this package provides often used macros and similar
% ones with the name prefix \cs{ltx@}. This avoids also faulty
% redefinitions. I remember an example where a package redefined
% \cs{@firstoftwo} with forgetting \cs{long}.
%
% \subsection{Numbers}
%
% \begin{declcs}{ltx@zero} \TO 0\\
%   \cs{ltx@one} \TO 1\\
%   \cs{ltx@two} \TO 2\\
%   \cs{ltx@cclv} \TO 255\\
%   \cs{ltx@minusone} \TO -1
% \end{declcs}
% These commands are numbers 0, 1, 2, 255 and -1. They are not digits
% and a space is not gobbled afterwards. Macro \cs{ltx@minusone}
% is available since version 2010/12/12 v1.15.
%
% \subsection{Scratch registers}
%
% Following the conventions of \hologo{plainTeX} and \hologo{LaTeX}
% the first ten registers are free to use. Even numbered registers
% are for local, odd numbered for global use.
%
% \begin{declcs}{ltx@(Loc,Glob)(Toks,Dimen,Skip)(A,B,C,D,E)}
% \end{declcs}
% The name consists of the prefix \cs{ltx@}, then
% \texttt{Loc} or \texttt{Glob} for local or global usage follows.
% The register type is given by \texttt{Toks} for token register,
% \texttt{Dimen} for dimen register and \texttt{Skip} for skip
% register. As last part the registers are numbered from \texttt{A}
% to \texttt{E}. Example: \cs{ltx@LocToksA}.
%
% Since 2011/04/14 v1.19.
%
% \subsection{Argument killers}
%
% \begin{declcs}{ltx@gobble} \M{1} \TO\\
%   \cs{ltx@gobbletwo} \M{1} \M{2} \TO\\
%   \cs{ltx@gobblethree} \M{1} \M{2} \M{3} \TO\\
%   \cs{ltx@gobblefour} \M{1} \M{2} \M{3} \M{4} \TO
% \end{declcs}
%
% \begin{declcs}{ltx@GobbleNum} \M{num} \M{1} \M{2} \dots\space
%   \M{\meta{num}} \TO
% \end{declcs}
% The first argument \meta{num} of macro \cs{ltx@GobbleNum} specifies,
% how many following arguments are eaten. Macro \cs{ltx@GobbleNum}
% is expandable in exact two expansion steps.
%
% \subsection{Argument grabbers}
%
% \begin{declcs}{ltx@firstofone} \M{1} \TO \meta{1}\\[1ex]
%  \cs{ltx@firstoftwo} \M{1} \M{2} \TO \meta{1}\\
%  \cs{ltx@secondoftwo} \M{1} \M{2} \TO \meta{2}\\[1ex]
%  \cs{ltx@firstofthree} \M{1} \M{2} \M{3} \TO \meta{1}\\
%  \cs{ltx@secondofthree} \M{1} \M{2} \M{3} \TO \meta{2}\\
%  \cs{ltx@thirdofthree} \M{1} \M{2} \M{3} \TO \meta{3}\\
%  \cs{ltx@firstoffour} \M{1} \M{2} \M{3} \M{4} \TO \meta{1}\\
%  \cs{ltx@secondoffour} \M{1} \M{2} \M{3} \M{4} \TO \meta{2}\\
%  \cs{ltx@thirdoffour} \M{1} \M{2} \M{3} \M{4} \TO \meta{3}\\
%  \cs{ltx@fourthoffour} \M{1} \M{2} \M{3} \M{4} \TO \meta{4}
% \end{declcs}
% Macros \cs{ltx@firstofthree}, \cs{ltx@secondofthree}
% and \cs{ltx@thirdofthree} were added in version
% 2010/11/12 v1.11.
% Macros \cs{ltx@firstoffour}, \dots, \cs{ltx@fourthoffour}
% were added in version 2011/02/04 v1.16.
%
% \subsection{List helpers}
%
% \begin{declcs}{ltx@carzero} \dots\ \cs{@nil} \TO\\
%   \cs{ltx@cdrzero} \dots\ \cs{@nil} \TO \dots
% \end{declcs}
%
% \begin{declcs}{ltx@car} \M{1} \dots\ \cs{@nil} \TO \meta{1}\\
%   \cs{ltx@cdr} \M{1} \dots\ \cs{@nil} \TO \dots
% \end{declcs}
%
% \begin{declcs}{ltx@cartwo} \M{1} \M{2} \dots\ \cs{@nil}
%   \TO \meta{1}\meta{2}\\
%   \cs{ltx@carsecond} \M{1} \M{2} \dots\ \cs{@nil} \TO \meta{2}\\
%   \cs{ltx@cdrtwo} \M{1} \M{2} \dots\ \cs{@nil} \TO \dots
% \end{declcs}
%
% \begin{declcs}{ltx@carthree} \M{1} \M{2} \M{3} \dots\ \cs{@nil}
%       \TO \meta{1}\meta{2}\meta{3}\\
%   \cs{ltx@carthird} \M{1} \M{2} \M{3} \dots\ \cs{@nil}
%       \TO \meta{3}\\
%   \cs{ltx@cdrthree} \M{1} \M{2} \M{3} \dots\ \cs{@nil} \TO \dots
% \end{declcs}
%
% \begin{declcs}{ltx@carfour} \M{1} \M{2} \M{3} \M{4} \dots\ \cs{@nil}
%       \TO \meta{1}\meta{2}\meta{3}\meta{4}\\
%   \cs{ltx@carfourth} \M{1} \M{2} \M{3} \M{4} \dots\ \cs{@nil}
%       \TO \meta{4}\\
%   \cs{ltx@cdrfour} \M{1} \M{2} \M{3} \M{4} \dots\ \cs{@nil} \TO \dots
% \end{declcs}
%
% \begin{declcs}{ltx@CarNum} \M{num} \M{1} \dots\ \M{\meta{num}}
%   \M{\meta{num}+1} \dots\ \cs{@nil}\\
%   \TO \M{1} \dots\ \M{\meta{num}} \dots\\
%   \cs{ltx@CarNumth} \M{num} \M{1} \dots\ \M{\meta{num}}
%   \M{\meta{num}+1} \dots\ \cs{@nil}\\
%   \TO \M{\meta{num}} \dots\\
%   \cs{ltx@CdrNum} \M{num} \M{1} \dots\ \M{\meta{num}}
%   \M{\meta{num}+1} \dots\ \cs{@nil}\\
%   \TO \M{\meta{num}+1} \dots
% \end{declcs}
%
% Macros with uppercase letters are expandable in two expansion steps.
% Changes in version 2019/12/15 v1.24:
% \begin{itemize}
% \item
%   Macros \cs{ltx@carsecond}, \cs{ltx@carthird}, \cs{ltx@carfourth},
%   \cs{ltx@CarNumth} added.
% \item
%   Macros \cs{ltx@cdr}, \cs{ltx@cdrtwo}, \cs{ltx@cdrthree}, \cs{ltx@cdrfour},
%   \cs{ltx@CdrNum} are expandable in two expansion steps and retain
%   spaces and braces after the first gobbled arguments.
% \end{itemize}
%
% \subsection{Tail recursion}
%
% \begin{declcs}{ltx@ReturnAfterFi} \M{1} \cs{fi} \TO \cs{fi} \meta{1}\\
%   \cs{ltx@ReturnAfterElseFi} \M{1} \cs{else} \M{2} \cs{fi}
%     \TO \cs{fi} \meta{1}
% \end{declcs}
%
% \subsection{Empty macro}
%
% \begin{declcs}{ltx@empty} \TO
% \end{declcs}
%
% \subsection{Characters}
%
% \begin{declcs}{ltx@space} \TO \verb*| |\\
%   \cs{ltx@percentchar} \TO \texttt{\%}\\
%   \cs{ltx@backslashchar} \TO
%      \makeatletter\texttt{\@backslashchar}\\
%   \cs{ltx@hashchar} \TO \texttt{\#}\qquad(since v1.7)\\
%   \cs{ltx@leftbracechar} \TO \texttt{\textbraceleft}\qquad(since v1.8)\\
%   \cs{ltx@rightbracechar} \TO \texttt{\textbraceright}\qquad(since v1.8)
% \end{declcs}
%
% \subsection{Boolean switch}
%
% \begin{declcs}{ltx@newif} \M{cmd}
% \end{declcs}
% \cs{ltx@newif} defines a new boolean switch \meta{cmd} like \cs{newif}.
% Unlike \plainTeX's \cs{newif}, \cs{ltx@newif} is not \cs{outer}.
% The command \meta{cmd} must start with the two characters |if|.
%
% \begin{declcs}{ltx@newglobalif} \M{cmd}
% \end{declcs}
% \cs{ltx@newglobalif} defines a new boolean switch \meta{cmd}
% like \cs{ltx@newif}. However the switch setting commands,
% \meta{cmd} without the prefix |if| and followed by |true| or |false|
% are acting globally.
%
% \subsection{Command definitions}
%
% \begin{declcs}{ltx@ifundefined} \M{cmd} \M{yes} \M{no}
% \end{declcs}
% If \eTeX\ is available, \cs{ifcsname} is used that does
% not have the side effect of defining undefined commands
% with meaning of \cs{relax}. This command is always expandable.
% Change in version 1.1: Also the meaning \cs{relax} is
% always considered ``undefined''.
%
% \begin{declcs}{ltx@IfUndefined} \M{cmd} \M{yes} \M{no}
% \end{declcs}
% If \eTeX\ is available, \cs{ifcsname} is used that does
% not have the side effect of defining undefined commands
% with meaning of \cs{relax}. Also it always checks for
% the meaning of \cs{relax} and considers this as undefined.
% This macro is not expandable without \eTeX.
%
% \begin{declcs}{ltx@LocalExpandAfter}
% \end{declcs}
% It expands the token after the next token but in a local
% context. That is the difference to \cs{expandafter}.
% The local context discards the side effect of \cs{csname}
% and let the command undefined after the expansion step.
%
% \subsection{Stripping}
%
% \begin{declcs}{ltx@RemovePrefix}\\
%   \cs{ltx@StripPrefix}
% \end{declcs}
% All tokens up to and including the next available character
% `\textgreater' are thrown away. Usually it is used to strip
% the first part of the output of the commands \cs{meaning}
% or \cs{pdflastmatch}. Macro \cs{ltx@RemovePrefix}
% has the same meaning as \LaTeX's \cs{strip@prefix}, whereas
% macro \cs{ltx@StripPrefix} expands the next token
% once before stripping the prefix.
%
% \begin{declcs}{ltx@onelevel@sanitize} \M{macro}
% \end{declcs}
% Macro \cs{ltx@onelevel@sanitize} provides \hologo{LaTeX}'s
% \cs{@onelevel@sanitize}. The macro is expanded once and
% the contents is converted to characters with catcode 12 (other)
% and space tokens with catcode 10 (space). Then then sanitized
% contents is stored into the macro again. Since version 1.12.
%
% \subsection{File management}
%
% All macros in this section are expandable like
% the counterparts of the \LaTeX\ kernel. Also they
% can be used after the preamble.
%
% \subsubsection{File extensions}
%
% \begin{declcs}{ltx@clsextension}\\
%   \cs{ltx@pkgextension}
% \end{declcs}
% Macros \cs{ltx@clsextension} and \cs{ltx@styextension}
% stores the strings |cls| and |sty|. In opposite to \hologo{LaTeX}'s
% \cs{@clsextension} and \cs{@styextension} they can also be
% used after |\begin{document}|.
%
% \subsubsection{Load check}
%
% \begin{declcs}{ltx@ifclassloaded} \M{class} \M{yes} \M{no}\\
%   \cs{ltx@ifpackageloaded} \M{package} \M{yes} \M{no}
% \end{declcs}
% Macros \cs{ltx@ifclassloaded}\slash\cs{ltx@ifpackageloaded}
% execute \meta{yes}, if the \meta{class} or \meta{package} is loaded,
% otherwise \meta{no} is called.
% Both \meta{class} and \meta{package} are specified without
% extension. The macros can also be used after |\begin{document}|.
%
% \begin{declcs}{ltx@iffileloaded} \M{file} \M{yes} \M{no}
% \end{declcs}
% If \LaTeX's \cs{ProvidesFile} macro was called before
% using \meta{file} as argument, then \cs{ltx@iffileloaded}
% calls \meta{yes}, otherwise \meta{no}.
% Therefore it is possible
% that the \meta{file} is loaded, but \meta{no} is executed because
% of a missing \cs{ProvidesFile}. The \LaTeX\ kernel does not have
% a counterpart of \cs{ltx@iffileloaded}.
%
% Note that the file name used in \cs{ProvidesFile} and
% \cs{ltx@iffileloaded} must match. For example, if \TeX's
% default extension \xext{tex} was given in the first command,
% then it must also specified in the latter command and vice versa.
%
% \subsubsection{Version date check}
%
% \begin{declcs}{ltx@ifclasslater} \M{class} \M{date} \M{yes} \M{no}\\
%   \cs{ltx@ifpackagelater} \M{package} \M{date} \M{yes} \M{no}\\
%   \cs{ltx@iffilelater} \M{file} \M{date} \M{yes} \M{no}
% \end{declcs}
% If a \cs{ProvidesClass}\slash\cs{ProvidesPackage}\slash
% \cs{ProvidesFile} command with exact the same class\slash package\slash
% file was executed before with an optional
% argument that starts with a \LaTeX\ version date, then
% this version date is compared with the argument \meta{date}.
% If they are equal or if the version date is the later date,
% then \meta{yes} is called. In all other cases \meta{no} is executed.
%
% A \LaTeX\ date has the format |YYYY/MM/DD| with |YYYY| as year
% with four digits, |MM| as month with two digits and |DD| as day
% with two digits. If \pdfTeX's \cs{pdfmatch} is available, then
% it is used to detect the version date, to reject invalid
% date formats and to reject some invalid dates.
% Dates before 1994/01/01 are always invalid, because version
% dates are introduced with \LaTeXe\ in 1994.
%
% \subsection{Macro additions}
%
% \begin{declcs}{ltx@GlobalAppendToMacro} \M{cmd} \M{addition}\\
%   \cs{ltx@LocalAppendToMacro} \M{cmd} \M{addition}
% \end{declcs}
% The \meta{addition} is appended to the parameterless macro \meta{cmd}.
% If \meta{cmd} is undefined or has the meaning \cs{relax}, then
% it will be initialized as empty macro beforehand. Due to a bug \meta{addition}
% must not contain \cs{par} before version 2010/10/25 v1.9.
%
% \begin{declcs}{ltx@GlobalPrependToMacro} \M{cmd} \M{addition}\\
%   \cs{ltx@LocalPrependToMacro} \M{cmd} \M{addition}
% \end{declcs}
% The \meta{addition} is prepended to the parameterless macro \meta{cmd}.
% If \meta{cmd} is undefined or has the meaning \cs{relax}, then
% it will be initialized as empty macro beforehand. The macros
% were added in version 2011/08/22 v1.21.
%
% \subsection{Next character detection}
%
% \begin{declcs}{ltx@ifnextchar} \M{char} \M{yes} \M{no}
% \end{declcs}
% If next character is \meta{char} then \meta{yes} is called,
% otherwise \meta{no}. The character is not removed.
% Spaces are silently removed when looking for \meta{char}
% as \hologo{LaTeX}'s version \cs{kernel@ifnextchar} does.
% But there are also small differences:
% \begin{itemize}
% \item The space can be used as \meta{char}. In this case
%   optional spaces before \meta{char} are not supported of course.
% \item If the optional space is a command that is a character
%   (defined by \cs{let} or \cs{futurelet}), then
%   \cs{kernel@ifnextchar} breaks with an \hologo{TeX} error.
%   \cs{ltx@ifnextchar} silently removes this token as
%   optional space.
% \end{itemize}
% Since 2010/03/01 v1.3.
%
% \begin{declcs}{ltx@ifnextchar@nospace} \M{char} \M{yes} \M{no}
% \end{declcs}
% Macro \cs{ltx@ifnextchar@nospace} behaves like
% macro \cs{ltx@ifnextchar} with the exception that
% optional spaces are not supported before \meta{char}.
% Since 2011/04/14 v1.19.
%
% \subsection{\cs{ltx@leavevmode}, \cs{ltx@mbox}}
%
% \begin{declcs}{ltx@leavevmode}
% \end{declcs}
% Macro \cs{ltx@leavevmode} calls \hologo{pdfTeX}'s \cs{quitvmode}.
% Otherwise \cs{leavevmode} is used and defined if it is necessary.
%
% \begin{declcs}{ltx@mbox}
% \end{declcs}
% Macro \cs{ltx@mbox} reimplements \cs{mbox} with two changes.
% Instead of \cs{leavevmode} it uses \cs{ltx@leavevmode} and
% stops right after \cs{hbox}. Especially it does not grab
% the argument and allows the extended syntax of \cs{hbox}.
%
% \subsection{Expandable test for emptiness}
%
% \begin{declcs}{ltx@ifempty} \M{stuff} \M{yes} \M{no}
% \end{declcs}
% Macro \cs{ltx@ifempty} checks in exact two expansion steps
% whether \meta{stuff}
% is empty or contains token. Depending on the result
% \meta{yes} or \meta{no} is executed. The token in \meta{stuff}
% may contain \cs{par} and unmatched conditionals (\cs{if}, \cs{else},
% \cs{fi}, \dots). Since version 2010/11/12 v1.11.
%
% \begin{declcs}{ltx@ifblank} \M{stuff} \M{yes} \M{no}
% \end{declcs}
% Macro \cs{ltx@ifblank} tests in exact two expansion steps
% if \meta{stuff} is empty or contain only blank spaces.
% In this case argument \meta{yes} is called. If \meta{stuff}
% contains other tokens than spaces then \meta{no} is executed.
% Since version 2010/12/04 v1.13.
%
% \subsection{Stripping spaces}
%
% \begin{declcs}{ltx@zapspace} \M{stuff}
% \end{declcs}
% Macro \cs{ltx@zapspace} strips spaces from \meta{stuff}
% that are not hidden inside curly braces. Like \hologo{LaTeX}'s
% \cs{zap@space} it is expandable. Differences:
% \begin{itemize}
% \item
% Syntax: \cs{zap@space} also expects a space token
% and \cs{@empty} after \meta{stuff}.
% \item
% Macro \cs{ltx@zapspace} is expandable in exact two
% expansion steps.
% \item
% Macro \cs{ltx@zapspace} always retains curly braces.
% \item
% Macro \cs{zap@space} has a bug. It stops stripping spaces
% after a token group in curly braces if the first two tokens
% inside the group are equal.
% \item
% Macro \cs{ltx@zapspace} also works with \cs{par}
% and conditionals (\cs{if}, \cs{else}, \cs{fi}, \dots).
% \end{itemize}
% Macro \cs{ltx@zapspace} is available since version
% 2010/12/07 v1.14.
%
% \subsection{Check for emptiness of boxes}
%
% \begin{declcs}{ltx@IfBoxEmpty} \M{box register number}
%   \M{yes} \M{no}
% \end{declcs}
% Macro \cs{ltx@IfBoxEmpty} calls \meta{yes} if
% the box exists (\cs{ifvoid} returns false) and the box
% does not contain any content. Otherwise if the box
% is void or contains something, then \meta{no} is executed.
% Thus being empty means that the box exists and is either
% an \cs{hbox} or a \cs{vbox} and may even have dimensions other than
% 0.0\,pt, but the box does not contain anything.
% Macro \cs{ltx@IfBoxEmpty} is available since 2010/02/04 v1.16.
%
% \begin{declcs}{ltx@IfBoxVoidOrEmpty} \M{box register number}
%   \M{yes} \M{no}
% \end{declcs}
% Macro \cs{ltx@IfBoxVoidOrEmpty} calls \meta{yes} if
% the box is either void or does not contain any content.
% Otherwise \meta{no} is executed.
% Macro \cs{ltx@IfBoxVoidOrEmpty} is available since 2010/02/04 v1.16.
%
% \StopEventually{
% }
%
% \section{Implementation}
%
% \subsection{Identification}
%
%    \begin{macrocode}
%<*package>
%    \end{macrocode}
%    Reload check, especially if the package is not used with \LaTeX.
%    \begin{macrocode}
\begingroup\catcode61\catcode48\catcode32=10\relax%
  \catcode13=5 % ^^M
  \endlinechar=13 %
  \catcode35=6 % #
  \catcode39=12 % '
  \catcode44=12 % ,
  \catcode45=12 % -
  \catcode46=12 % .
  \catcode58=12 % :
  \catcode64=11 % @
  \catcode123=1 % {
  \catcode125=2 % }
  \expandafter\let\expandafter\x\csname ver@ltxcmds.sty\endcsname
  \ifx\x\relax % plain-TeX, first loading
  \else
    \def\empty{}%
    \ifx\x\empty % LaTeX, first loading,
      % variable is initialized, but \ProvidesPackage not yet seen
    \else
      \expandafter\ifx\csname PackageInfo\endcsname\relax
        \def\x#1#2{%
          \immediate\write-1{Package #1 Info: #2.}%
        }%
      \else
        \def\x#1#2{\PackageInfo{#1}{#2, stopped}}%
      \fi
      \x{ltxcmds}{The package is already loaded}%
      \aftergroup\endinput
    \fi
  \fi
\endgroup%
%    \end{macrocode}
%    Package identification:
%    \begin{macrocode}
\begingroup\catcode61\catcode48\catcode32=10\relax%
  \catcode13=5 % ^^M
  \endlinechar=13 %
  \catcode35=6 % #
  \catcode39=12 % '
  \catcode40=12 % (
  \catcode41=12 % )
  \catcode44=12 % ,
  \catcode45=12 % -
  \catcode46=12 % .
  \catcode47=12 % /
  \catcode58=12 % :
  \catcode64=11 % @
  \catcode91=12 % [
  \catcode93=12 % ]
  \catcode123=1 % {
  \catcode125=2 % }
  \expandafter\ifx\csname ProvidesPackage\endcsname\relax
    \def\x#1#2#3[#4]{\endgroup
      \immediate\write-1{Package: #3 #4}%
      \xdef#1{#4}%
    }%
  \else
    \def\x#1#2[#3]{\endgroup
      #2[{#3}]%
      \ifx#1\@undefined
        \xdef#1{#3}%
      \fi
      \ifx#1\relax
        \xdef#1{#3}%
      \fi
    }%
  \fi
\expandafter\x\csname ver@ltxcmds.sty\endcsname
\ProvidesPackage{ltxcmds}%
  [2019/12/15 v1.24 LaTeX kernel commands for general use (HO)]%
%    \end{macrocode}
%
%    \begin{macrocode}
\begingroup\catcode61\catcode48\catcode32=10\relax%
  \catcode13=5 % ^^M
  \endlinechar=13 %
  \catcode123=1 % {
  \catcode125=2 % }
  \catcode64=11 % @
  \def\x{\endgroup
    \expandafter\edef\csname LTXcmds@AtEnd\endcsname{%
      \endlinechar=\the\endlinechar\relax
      \catcode13=\the\catcode13\relax
      \catcode32=\the\catcode32\relax
      \catcode35=\the\catcode35\relax
      \catcode61=\the\catcode61\relax
      \catcode64=\the\catcode64\relax
      \catcode123=\the\catcode123\relax
      \catcode125=\the\catcode125\relax
    }%
  }%
\x\catcode61\catcode48\catcode32=10\relax%
\catcode13=5 % ^^M
\endlinechar=13 %
\catcode35=6 % #
\catcode64=11 % @
\catcode123=1 % {
\catcode125=2 % }
\def\TMP@EnsureCode#1#2{%
  \edef\LTXcmds@AtEnd{%
    \LTXcmds@AtEnd
    \catcode#1=\the\catcode#1\relax
  }%
  \catcode#1=#2\relax
}
\TMP@EnsureCode{36}{3}% $
\TMP@EnsureCode{38}{4}% &
\TMP@EnsureCode{40}{12}% (
\TMP@EnsureCode{41}{12}% )
\TMP@EnsureCode{45}{12}% -
\TMP@EnsureCode{46}{12}% .
\TMP@EnsureCode{47}{12}% /
\TMP@EnsureCode{60}{12}% <
\TMP@EnsureCode{62}{12}% >
\TMP@EnsureCode{91}{12}% [
\TMP@EnsureCode{96}{12}% `
\TMP@EnsureCode{93}{12}% ]
\TMP@EnsureCode{94}{12}% ^ (superscript) (!)
\TMP@EnsureCode{124}{12}% |
\edef\LTXcmds@AtEnd{\LTXcmds@AtEnd\noexpand\endinput}
%    \end{macrocode}
%
% \subsection{Numbers}
%
%    \begin{macro}{\ltx@zero}
%    \begin{macrocode}
\chardef\ltx@zero=0 %
%    \end{macrocode}
%    \end{macro}
%    \begin{macro}{\ltx@one}
%    \begin{macrocode}
\chardef\ltx@one=1 %
%    \end{macrocode}
%    \end{macro}
%    \begin{macro}{\ltx@two}
%    \begin{macrocode}
\chardef\ltx@two=2 %
%    \end{macrocode}
%    \end{macro}
%    \begin{macro}{\ltx@active}
%    \begin{macrocode}
\chardef\ltx@active=13 %
%    \end{macrocode}
%    \end{macro}
%    \begin{macro}{\ltx@cclv}
%    \begin{macrocode}
\chardef\ltx@cclv=255 %
%    \end{macrocode}
%    \end{macro}
%
%    \begin{macro}{\ltx@minusone}
%    \begin{macrocode}
\def\ltx@minusone{%
  -\ltx@one
}
%    \end{macrocode}
%    \end{macro}
%
% \subsection{Scratch registers}
%
%    \begin{macro}{\ltx@LocToksA}
%    \begin{macrocode}
\toksdef\ltx@LocToksA=0 %
%    \end{macrocode}
%    \end{macro}
%    \begin{macro}{\ltx@LocToksB}
%    \begin{macrocode}
\toksdef\ltx@LocToksB=2 %
%    \end{macrocode}
%    \end{macro}
%    \begin{macro}{\ltx@LocToksC}
%    \begin{macrocode}
\toksdef\ltx@LocToksC=4 %
%    \end{macrocode}
%    \end{macro}
%    \begin{macro}{\ltx@LocToksD}
%    \begin{macrocode}
\toksdef\ltx@LocToksD=6 %
%    \end{macrocode}
%    \end{macro}
%    \begin{macro}{\ltx@LocToksE}
%    \begin{macrocode}
\toksdef\ltx@LocToksE=8 %
%    \end{macrocode}
%    \end{macro}
%
%    \begin{macro}{\ltx@GlobToksA}
%    \begin{macrocode}
\toksdef\ltx@GlobToksA=1 %
%    \end{macrocode}
%    \end{macro}
%    \begin{macro}{\ltx@GlobToksB}
%    \begin{macrocode}
\toksdef\ltx@GlobToksB=3 %
%    \end{macrocode}
%    \end{macro}
%    \begin{macro}{\ltx@GlobToksC}
%    \begin{macrocode}
\toksdef\ltx@GlobToksC=5 %
%    \end{macrocode}
%    \end{macro}
%    \begin{macro}{\ltx@GlobToksD}
%    \begin{macrocode}
\toksdef\ltx@GlobToksD=7 %
%    \end{macrocode}
%    \end{macro}
%    \begin{macro}{\ltx@GlobToksE}
%    \begin{macrocode}
\toksdef\ltx@GlobToksE=9 %
%    \end{macrocode}
%    \end{macro}
%
%    \begin{macro}{\ltx@LocDimenA}
%    \begin{macrocode}
\dimendef\ltx@LocDimenA=0 %
%    \end{macrocode}
%    \end{macro}
%    \begin{macro}{\ltx@LocDimenB}
%    \begin{macrocode}
\dimendef\ltx@LocDimenB=2 %
%    \end{macrocode}
%    \end{macro}
%    \begin{macro}{\ltx@LocDimenC}
%    \begin{macrocode}
\dimendef\ltx@LocDimenC=4 %
%    \end{macrocode}
%    \end{macro}
%    \begin{macro}{\ltx@LocDimenD}
%    \begin{macrocode}
\dimendef\ltx@LocDimenD=6 %
%    \end{macrocode}
%    \end{macro}
%    \begin{macro}{\ltx@LocDimenE}
%    \begin{macrocode}
\dimendef\ltx@LocDimenE=8 %
%    \end{macrocode}
%    \end{macro}
%
%    \begin{macro}{\ltx@GlobDimenA}
%    \begin{macrocode}
\dimendef\ltx@GlobDimenA=1 %
%    \end{macrocode}
%    \end{macro}
%    \begin{macro}{\ltx@GlobDimenB}
%    \begin{macrocode}
\dimendef\ltx@GlobDimenB=3 %
%    \end{macrocode}
%    \end{macro}
%    \begin{macro}{\ltx@GlobDimenC}
%    \begin{macrocode}
\dimendef\ltx@GlobDimenC=5 %
%    \end{macrocode}
%    \end{macro}
%    \begin{macro}{\ltx@GlobDimenD}
%    \begin{macrocode}
\dimendef\ltx@GlobDimenD=7 %
%    \end{macrocode}
%    \end{macro}
%    \begin{macro}{\ltx@GlobDimenE}
%    \begin{macrocode}
\dimendef\ltx@GlobDimenE=9 %
%    \end{macrocode}
%    \end{macro}
%
%    \begin{macro}{\ltx@LocSkipA}
%    \begin{macrocode}
\skipdef\ltx@LocSkipA=0 %
%    \end{macrocode}
%    \end{macro}
%    \begin{macro}{\ltx@LocSkipB}
%    \begin{macrocode}
\skipdef\ltx@LocSkipB=2 %
%    \end{macrocode}
%    \end{macro}
%    \begin{macro}{\ltx@LocSkipC}
%    \begin{macrocode}
\skipdef\ltx@LocSkipC=4 %
%    \end{macrocode}
%    \end{macro}
%    \begin{macro}{\ltx@LocSkipD}
%    \begin{macrocode}
\skipdef\ltx@LocSkipD=6 %
%    \end{macrocode}
%    \end{macro}
%    \begin{macro}{\ltx@LocSkipE}
%    \begin{macrocode}
\skipdef\ltx@LocSkipE=8 %
%    \end{macrocode}
%    \end{macro}
%
%    \begin{macro}{\ltx@GlobSkipA}
%    \begin{macrocode}
\skipdef\ltx@GlobSkipA=1 %
%    \end{macrocode}
%    \end{macro}
%    \begin{macro}{\ltx@GlobSkipB}
%    \begin{macrocode}
\skipdef\ltx@GlobSkipB=3 %
%    \end{macrocode}
%    \end{macro}
%    \begin{macro}{\ltx@GlobSkipC}
%    \begin{macrocode}
\skipdef\ltx@GlobSkipC=5 %
%    \end{macrocode}
%    \end{macro}
%    \begin{macro}{\ltx@GlobSkipD}
%    \begin{macrocode}
\skipdef\ltx@GlobSkipD=7 %
%    \end{macrocode}
%    \end{macro}
%    \begin{macro}{\ltx@GlobSkipE}
%    \begin{macrocode}
\skipdef\ltx@GlobSkipE=9 %
%    \end{macrocode}
%    \end{macro}
%
%
% \subsection{Argument killers}
%
%    \begin{macro}{\ltx@gobble}
%    \begin{macrocode}
\long\def\ltx@gobble#1{}
%    \end{macrocode}
%    \end{macro}
%    \begin{macro}{\ltx@gobbletwo}
%    \begin{macrocode}
\long\def\ltx@gobbletwo#1#2{}
%    \end{macrocode}
%    \end{macro}
%    \begin{macro}{\ltx@gobblethree}
%    \begin{macrocode}
\long\def\ltx@gobblethree#1#2#3{}
%    \end{macrocode}
%    \end{macro}
%    \begin{macro}{\ltx@gobblefour}
%    \begin{macrocode}
\long\def\ltx@gobblefour#1#2#3#4{}
%    \end{macrocode}
%    \end{macro}
%
%    \begin{macro}{\ltx@GobbleNum}
%    \begin{macrocode}
\def\ltx@GobbleNum#1{%
  \romannumeral
  \csname ltx@zero%
  \expandafter\LTXcmds@GobbleNum
  \romannumeral\LTXcmds@num{#1}000{m\endcsname}%
}
%    \end{macrocode}
%    \end{macro}
%    \begin{macro}{\LTXcmds@GobbleNum}
%    \begin{macrocode}
\def\LTXcmds@GobbleNum#1{%
  \csname LTXcmds@G#1\LTXcmds@GobbleNum
}
%    \end{macrocode}
%    \end{macro}
%    \begin{macro}{\LTXcmds@Gm}
%    \begin{macrocode}
\long\def\LTXcmds@Gm#1{%
  \endcsname
}
%    \end{macrocode}
%    \end{macro}
%
% \subsection{Argument grabbers}
%
%    \begin{macro}{\ltx@firstofone}
%    \begin{macrocode}
\long\def\ltx@firstofone#1{#1}
%    \end{macrocode}
%    \end{macro}
%    \begin{macro}{\ltx@firstoftwo}
%    \begin{macrocode}
\long\def\ltx@firstoftwo#1#2{#1}
%    \end{macrocode}
%    \end{macro}
%    \begin{macro}{\ltx@secondoftwo}
%    \begin{macrocode}
\long\def\ltx@secondoftwo#1#2{#2}
%    \end{macrocode}
%    \end{macro}
%
%    \begin{macro}{\ltx@firstofthree}
%    \begin{macrocode}
\long\def\ltx@firstofthree#1#2#3{#1}
%    \end{macrocode}
%    \end{macro}
%    \begin{macro}{\ltx@secondofthree}
%    \begin{macrocode}
\long\def\ltx@secondofthree#1#2#3{#2}
%    \end{macrocode}
%    \end{macro}
%    \begin{macro}{\ltx@thirdofthree}
%    \begin{macrocode}
\long\def\ltx@thirdofthree#1#2#3{#3}%
%    \end{macrocode}
%    \end{macro}
%
%    \begin{macro}{\ltx@firstoffour}
%    \begin{macrocode}
\long\def\ltx@firstoffour#1#2#3#4{#1}
%    \end{macrocode}
%    \end{macro}
%    \begin{macro}{\ltx@secondoffour}
%    \begin{macrocode}
\long\def\ltx@secondoffour#1#2#3#4{#2}
%    \end{macrocode}
%    \end{macro}
%    \begin{macro}{\ltx@thirdoffour}
%    \begin{macrocode}
\long\def\ltx@thirdoffour#1#2#3#4{#3}%
%    \end{macrocode}
%    \end{macro}
%    \begin{macro}{\ltx@fourthoffour}
%    \begin{macrocode}
\long\def\ltx@fourthoffour#1#2#3#4{#4}%
%    \end{macrocode}
%    \end{macro}
%
% \subsection{List helpers}
%
%    \begin{macro}{\ltx@carzero}
%    \begin{macrocode}
\long\def\ltx@carzero#1\@nil{}%
%    \end{macrocode}
%    \end{macro}
%    \begin{macro}{\LTXcmds@cdrzero}
%    \begin{macrocode}
\long\def\LTXcmds@cdrzero#1\@nil{#1}
%    \end{macrocode}
%    \end{macro}
%    \begin{macro}{\ltx@cdrzero}
%    \begin{macrocode}
\def\ltx@cdrzero{%
  \romannumeral\LTXcmds@cdrzero\ltx@zero
}
%    \end{macrocode}
%    \end{macro}
%
%    \begin{macro}{\ltx@car}
%    \begin{macrocode}
\long\def\ltx@car#1#2\@nil{#1}
%    \end{macrocode}
%    \end{macro}
%    \begin{macro}{\ltx@cdr}
%    \begin{macrocode}
\long\def\ltx@cdr#1{%
  \romannumeral\LTXcmds@cdrzero\ltx@zero
}
%    \end{macrocode}
%    \end{macro}
%
%    \begin{macro}{\ltx@cartwo}
%    \begin{macrocode}
\long\def\ltx@cartwo#1#2#3\@nil{#1#2}
%    \end{macrocode}
%    \end{macro}
%    \begin{macro}{\ltx@carsecond}
%    \begin{macrocode}
\long\def\ltx@carsecond#1#2#3\@nil{#2}
%    \end{macrocode}
%    \end{macro}
%    \begin{macro}{\ltx@cdrtwo}
%    \begin{macrocode}
\long\def\ltx@cdrtwo#1#2{%
  \romannumeral\LTXcmds@cdrzero\ltx@zero
}
%    \end{macrocode}
%    \end{macro}
%
%    \begin{macro}{\ltx@carthree}
%    \begin{macrocode}
\long\def\ltx@carthree#1#2#3#4\@nil{#1#2#3}
%    \end{macrocode}
%    \end{macro}
%    \begin{macro}{\ltx@carthird}
%    \begin{macrocode}
\long\def\ltx@carthird#1#2#3#4\@nil{#3}
%    \end{macrocode}
%    \end{macro}
%    \begin{macro}{\ltx@cdrthree}
%    \begin{macrocode}
\long\def\ltx@cdrthree#1#2#3{%
  \romannumeral\LTXcmds@cdrzero\ltx@zero
}
%    \end{macrocode}
%    \end{macro}
%
%    \begin{macro}{\ltx@carfour}
%    \begin{macrocode}
\long\def\ltx@carfour#1#2#3#4#5\@nil{#1#2#3#4}
%    \end{macrocode}
%    \end{macro}
%    \begin{macro}{\ltx@carfourth}
%    \begin{macrocode}
\long\def\ltx@carfourth#1#2#3#4#5\@nil{#4}
%    \end{macrocode}
%    \end{macro}
%    \begin{macro}{\ltx@cdrfour}
%    \begin{macrocode}
\long\def\ltx@cdrfour#1#2#3#4{%
  \romannumeral\LTXcmds@cdrzero\ltx@zero
}
%    \end{macrocode}
%    \end{macro}
%
%    \begin{macro}{\ltx@CarNum}
%    \begin{macrocode}
\def\ltx@CarNum#1{%
  \romannumeral
  \csname LTXcmds@CarNumFinish%
  \expandafter\LTXcmds@CarNum
  \romannumeral\LTXcmds@num{#1}000{x\endcsname}%
}
%    \end{macrocode}
%    \end{macro}
%    \begin{macro}{\LTXcmds@CarNum}
%    \begin{macrocode}
\def\LTXcmds@CarNum#1{%
  \csname LTXcmds@C#1\LTXcmds@CarNum
}
%    \end{macrocode}
%    \end{macro}
%    \begin{macro}{\LTXcmds@Cm}
%    \begin{macrocode}
\long\def\LTXcmds@Cm#1#2{%
  \endcsname{#1#2}%
}
%    \end{macrocode}
%    \end{macro}
%    \begin{macro}{\LTXcmds@Cx}
%    \begin{macrocode}
\def\LTXcmds@Cx#1{%
  \endcsname{}%
}
%    \end{macrocode}
%    \end{macro}
%    \begin{macro}{\LTXcmds@CarNumFinish}
%    \begin{macrocode}
\long\def\LTXcmds@CarNumFinish#1#2\@nil{%
  \ltx@zero
  #1%
}
%    \end{macrocode}
%    \end{macro}
%
%    \begin{macro}{\ltx@CarNumth}
%    \begin{macrocode}
\def\ltx@CarNumth#1{%
  \romannumeral
  \expandafter\expandafter\expandafter
  \LTXcmds@CarNumth
  \ltx@GobbleNum{#1}{}%
}
%    \end{macrocode}
%    \end{macro}
%    \begin{macro}{\LTXcmds@CarNumth}
%    \begin{macrocode}
\long\def\LTXcmds@CarNumth#1#2\@nil{%
  \ltx@zero
  #1%
}
%    \end{macrocode}
%    \end{macro}
%
%    \begin{macro}{\ltx@CdrNum}
%    \begin{macrocode}
\def\ltx@CdrNum#1{%
  \romannumeral%
  \expandafter\expandafter\expandafter\ltx@cdrzero
  \expandafter\expandafter\expandafter\ltx@zero
  \ltx@GobbleNum{#1}%
}
%    \end{macrocode}
%    \end{macro}
%
% \subsection{Tail recursion}
%
%    \begin{macro}{\ltx@ReturnAfterFi}
%    \begin{macrocode}
\long\def\ltx@ReturnAfterFi#1\fi{\fi#1}
%    \end{macrocode}
%    \end{macro}
%    \begin{macro}{\ltx@ReturnAfterElseFi}
%    \begin{macrocode}
\long\def\ltx@ReturnAfterElseFi#1\else#2\fi{\fi#1}
%    \end{macrocode}
%    \end{macro}
%
% \subsection{Empty macro}
%
%    \begin{macro}{\ltx@empty}
%    \begin{macrocode}
\def\ltx@empty{}
%    \end{macrocode}
%    \end{macro}
%
% \subsection{Characters}
%
%    \begin{macro}{\ltx@space}
%    \begin{macrocode}
\def\ltx@space{ }
%    \end{macrocode}
%    \end{macro}
%    \begin{macro}{\ltx@percentchar}
%    \begin{macrocode}
\begingroup
  \lccode`0=`\%\relax
\lowercase{\endgroup
  \def\ltx@percentchar{0}%
}
%    \end{macrocode}
%    \end{macro}
%    \begin{macro}{\ltx@backslashchar}
%    \begin{macrocode}
\begingroup
  \lccode`0=`\\\relax
\lowercase{\endgroup
  \def\ltx@backslashchar{0}%
}
%    \end{macrocode}
%    \end{macro}
%    \begin{macro}{\ltx@hashchar}
%    \begin{macrocode}
\begingroup
  \lccode`0=`\#\relax
\lowercase{\endgroup
  \def\ltx@hashchar{0}%
}
%    \end{macrocode}
%    \end{macro}
%    \begin{macro}{\ltx@leftbracechar}
%    \begin{macrocode}
\begingroup
  \lccode`0=`\{\relax
\lowercase{\endgroup
  \def\ltx@leftbracechar{0}%
}
%    \end{macrocode}
%    \end{macro}
%    \begin{macro}{\ltx@rightbracechar}
%    \begin{macrocode}
\begingroup
  \lccode`0=`\}\relax
\lowercase{\endgroup
  \def\ltx@rightbracechar{0}%
}
%    \end{macrocode}
%    \end{macro}
%
% \subsection{Boolean switch}
%
%    \begin{macro}{\ltx@newif}
%    \begin{macrocode}
\def\ltx@newif#1{%
  \begingroup
    \escapechar=-1 %
  \expandafter\endgroup
  \expandafter\LTXcmds@newif\string#1\@nil
}
%    \end{macrocode}
%    \end{macro}
%    \begin{macro}{\LTXcmds@newif}
%    \begin{macrocode}
\begingroup
  \escapechar=-1 %
\expandafter\endgroup
\expandafter\def\expandafter\LTXcmds@newif\string\if#1\@nil{%
  \expandafter\edef\csname#1true\endcsname{%
    \let
    \expandafter\noexpand\csname if#1\endcsname
    \noexpand\iftrue
  }%
  \expandafter\edef\csname#1false\endcsname{%
    \let
    \expandafter\noexpand\csname if#1\endcsname
    \noexpand\iffalse
  }%
  \csname#1false\endcsname
}
%    \end{macrocode}
%    \end{macro}
%
%    \begin{macro}{\ltx@newglobalif}
%    \begin{macrocode}
\def\ltx@newglobalif#1{%
  \begingroup
    \escapechar=-1 %
  \expandafter\endgroup
  \expandafter\LTXcmds@newglobalif\string#1\@nil
}
%    \end{macrocode}
%    \end{macro}
%    \begin{macro}{\LTXcmds@newglobalif}
%    \begin{macrocode}
\begingroup
  \escapechar=-1 %
\expandafter\endgroup
\expandafter
\def\expandafter\LTXcmds@newglobalif\string\if#1\@nil{%
  \expandafter\edef\csname#1true\endcsname{%
    \global\let
    \expandafter\noexpand\csname if#1\endcsname
    \noexpand\iftrue
  }%
  \expandafter\edef\csname#1false\endcsname{%
    \global\let
    \expandafter\noexpand\csname if#1\endcsname
    \noexpand\iffalse
  }%
  \csname#1false\endcsname
}
%    \end{macrocode}
%    \end{macro}
%
% \subsection{Command definitions}
%
%    \begin{macro}{\ltx@LocalExpandAfter}
%    \begin{macrocode}
\def\ltx@LocalExpandAfter{%
  \begingroup
    \expandafter\expandafter\expandafter
  \endgroup
  \expandafter
}
%    \end{macrocode}
%    \end{macro}
%
%    \begin{macrocode}
\ltx@LocalExpandAfter
\ifx\csname ifcsname\endcsname\relax
%    \end{macrocode}
%    \begin{macro}{\ltx@ifundefined}
%    \begin{macrocode}
  \def\ltx@ifundefined#1{%
    \expandafter\ifx\csname #1\endcsname\relax
      \expandafter\ltx@firstoftwo
    \else
      \expandafter\ltx@secondoftwo
    \fi
  }%
%    \end{macrocode}
%    \end{macro}
%    \begin{macro}{\ltx@IfUndefined}
%    \begin{macrocode}
  \def\ltx@IfUndefined#1{%
    \begingroup\expandafter\expandafter\expandafter\endgroup
    \expandafter\ifx\csname #1\endcsname\relax
      \expandafter\ltx@firstoftwo
    \else
      \expandafter\ltx@secondoftwo
    \fi
  }%
%    \end{macrocode}
%    \end{macro}
%    \begin{macrocode}
  \expandafter\ltx@gobble
\else
  \expandafter\ltx@firstofone
\fi
{%
%    \end{macrocode}
%    \begin{macro}{\ltx@ifundefined}
%    \begin{macrocode}
  \def\ltx@ifundefined#1{%
    \ifcsname #1\endcsname
      \expandafter\ifx\csname #1\endcsname\relax
        \expandafter\expandafter\expandafter\ltx@firstoftwo
      \else
        \expandafter\expandafter\expandafter\ltx@secondoftwo
      \fi
    \else
      \expandafter\ltx@firstoftwo
    \fi
  }%
%    \end{macrocode}
%    \end{macro}
%    \begin{macro}{\ltx@IfUndefined}
%    \begin{macrocode}
  \let\ltx@IfUndefined\ltx@ifundefined
%    \end{macrocode}
%    \end{macro}
%    \begin{macrocode}
}
%    \end{macrocode}
%
% \subsection{Stripping}
%
%    \begin{macro}{\ltx@RemovePrefix}
%    \begin{macrocode}
\def\ltx@RemovePrefix#1>{}
%    \end{macrocode}
%    \end{macro}
%
%    \begin{macro}{\ltx@StripPrefix}
%    \begin{macrocode}
\def\ltx@StripPrefix{%
  \expandafter\ltx@RemovePrefix
}
%    \end{macrocode}
%    \end{macro}
%
%    \begin{macro}{\ltx@onelevel@sanitize}
%    \begin{macrocode}
\def\ltx@onelevel@sanitize#1{%
  \edef#1{%
    \expandafter
    \ltx@RemovePrefix\meaning#1%
  }%
}
%    \end{macrocode}
%    \end{macro}
%
% \subsection{File management}
%
% \subsubsection{File extensions}
%
%    \begin{macro}{\ltx@clsextension}
%    \begin{macrocode}
\def\ltx@clsextension{cls}
%    \end{macrocode}
%    \end{macro}
%    \begin{macro}{\ltx@pkgextension}
%    \begin{macrocode}
\def\ltx@pkgextension{sty}
%    \end{macrocode}
%    \end{macro}
%
% \subsubsection{Load check}
%
%    \begin{macro}{\ltx@iffileloaded}
%    \begin{macrocode}
\def\ltx@iffileloaded#1{%
  \ltx@ifundefined{ver@#1}\ltx@secondoftwo\ltx@firstoftwo
}
%    \end{macrocode}
%    \end{macro}
%    \begin{macro}{\ltx@ifclassloaded}
%    \begin{macrocode}
\def\ltx@ifclassloaded#1{%
  \ltx@iffileloaded{#1.\ltx@clsextension}%
}
%    \end{macrocode}
%    \end{macro}
%    \begin{macro}{\ltx@ifpackageloaded}
%    \begin{macrocode}
\def\ltx@ifpackageloaded#1{%
  \ltx@iffileloaded{#1.\ltx@pkgextension}%
}
%    \end{macrocode}
%    \end{macro}
%
% \subsubsection{Version date check}
%
%    \begin{macro}{\ltx@iffilelater}
%    \begin{macrocode}
\def\ltx@iffilelater#1#2{%
  \ltx@iffileloaded{#1}{%
    \expandafter\LTXcmds@IfLater\expandafter{%
      \number
      \expandafter\expandafter\expandafter\LTXcmds@ParseVersion
      \expandafter\expandafter\expandafter{%
        \csname ver@#1\endcsname
      }%
    \expandafter}\expandafter{%
      \number
      \expandafter\LTXcmds@ParseVersion\expandafter{#2}%
    }%
  }\ltx@secondoftwo
}
%    \end{macrocode}
%    \end{macro}
%    \begin{macro}{\LTXcmds@IfLater}
%    \begin{macrocode}
\def\LTXcmds@IfLater#1#2{%
  \ifcase 0%
      \ifnum#1<19940101 %
      \else
        \ifnum#2<19940101 %
        \else
          \ifnum#2>#1 %
          \else
            1%
          \fi
        \fi
      \fi
      \ltx@space
    \expandafter\ltx@secondoftwo
  \else
    \expandafter\ltx@firstoftwo
  \fi
}
%    \end{macrocode}
%    \end{macro}
%    \begin{macro}{\ltx@ifclasslater}
%    \begin{macrocode}
\def\ltx@ifclasslater#1{%
  \ltx@iffilelater{#1.\ltx@clsextension}%
}
%    \end{macrocode}
%    \end{macro}
%    \begin{macro}{\ltx@ifpackagelater}
%    \begin{macrocode}
\def\ltx@ifpackagelater#1{%
  \ltx@iffilelater{#1.\ltx@pkgextension}%
}
%    \end{macrocode}
%    \end{macro}
%
%    \begin{macrocode}
\ltx@IfUndefined{pdfmatch}{%
%    \end{macrocode}
%    \begin{macro}{\LTXcmds@ParseVersion}
%    \begin{macrocode}
  \def\LTXcmds@ParseVersion#1{%
    \LTXcmds@@ParseVersion#10000/00/00\@nil
  }%
%    \end{macrocode}
%    \end{macro}
%    \begin{macro}{\LTXcmds@@ParseVersion}
%    \begin{macrocode}
  \def\LTXcmds@@ParseVersion#1#2#3#4/#5#6/#7#8#9\@nil{%
    #1#2#3#4#5#6#7#8%
  }%
%    \end{macrocode}
%    \end{macro}
%    \begin{macrocode}
}{%
%    \end{macrocode}
%    \begin{macro}{\LTXcmds@ParseVersion}
%    \begin{macrocode}
  \def\LTXcmds@ParseVersion#1{%
    \ifnum\pdfmatch{%
      ^%
      (199[4-9]|[2-9][0-9][0-9][0-9])/%
      (0[1-9]|1[0-2])/%
      (0[1-9]|[1-2][0-9]|3[0-1])%
    }{#1}=1 %
      \ltx@StripPrefix\pdflastmatch1 %
      \ltx@StripPrefix\pdflastmatch2 %
      \ltx@StripPrefix\pdflastmatch3 %
    \else
      0%
    \fi
  }%
%    \end{macrocode}
%    \end{macro}
%    \begin{macrocode}
}
%    \end{macrocode}
%
% \subsection{Macro additions}
%
%    \begin{macro}{\ltx@GlobalAppendToMacro}
%    \begin{macrocode}
\long\def\ltx@GlobalAppendToMacro#1#2{%
  \ifx\ltx@undefined#1%
    \let#1\ltx@empty
  \else
    \ifx\relax#1%
      \let#1\ltx@empty
    \fi
  \fi
  \begingroup
    \ltx@LocToksA\expandafter{#1#2}%
    \xdef#1{\the\ltx@LocToksA}%
  \endgroup
}
%    \end{macrocode}
%    \end{macro}
%    \begin{macro}{\ltx@LocalAppendToMacro}
%    \begin{macrocode}
\long\def\ltx@LocalAppendToMacro#1#2{%
  \global\let\LTXcmds@gtemp#1%
  \ifx\ltx@undefined\LTXcmds@gtemp
    \global\let\LTXcmds@gtemp\ltx@empty
  \else
    \ifx\relax\LTXcmds@gtemp
      \global\letLTXcmds@gtemp\ltx@empty
    \fi
  \fi
  \begingroup
    \ltx@LocToksA\expandafter{\LTXcmds@gtemp#2}%
    \xdef\LTXcmds@gtemp{\the\ltx@LocToksA}%
  \endgroup
  \let#1\LTXcmds@gtemp
}
%    \end{macrocode}
%    \end{macro}
%
%    \begin{macro}{\ltx@GlobalPrependToMacro}
%    \begin{macrocode}
\long\def\ltx@GlobalPrependToMacro#1#2{%
  \ifx\ltx@undefined#1%
    \let#1\ltx@empty
  \else
    \ifx\relax#1%
      \let#1\ltx@empty
    \fi
  \fi
  \begingroup
    \ltx@LocToksA{#2}%
    \ltx@LocToksB\expandafter{#1}%
    \xdef#1{\the\ltx@LocToksA\the\ltx@LocToksB}%
  \endgroup
}
%    \end{macrocode}
%    \end{macro}
%    \begin{macro}{\ltx@LocalPrependToMacro}
%    \begin{macrocode}
\long\def\ltx@LocalPrependToMacro#1#2{%
  \global\let\LTXcmds@gtemp#1%
  \ifx\ltx@undefined\LTXcmds@gtemp
    \global\let\LTXcmds@gtemp\ltx@empty
  \else
    \ifx\relax\LTXcmds@gtemp
      \global\letLTXcmds@gtemp\ltx@empty
    \fi
  \fi
  \begingroup
    \ltx@LocToksA{#2}%
    \ltx@LocToksB\expandafter{\LTXcmds@gtemp}%
    \xdef\LTXcmds@gtemp{\the\ltx@LocToksA\the\ltx@LocToksB}%
  \endgroup
  \let#1\LTXcmds@gtemp
}
%    \end{macrocode}
%    \end{macro}
%
% \subsection{Next character detection}
%
%    \begin{macro}{\ltx@ifnextchar}
%    \begin{macrocode}
\long\def\ltx@ifnextchar#1#2#3{%
  \begingroup
  \let\LTXcmds@CharToken= #1\relax
  \ltx@LocToksA{\endgroup#2}%
  \ltx@LocToksB{\endgroup#3}%
  \futurelet\LTXcmds@LetToken\LTXcmds@ifnextchar
}
%    \end{macrocode}
%    \end{macro}
%    \begin{macro}{\LTXcmds@ifnextchar}
%    \begin{macrocode}
\def\LTXcmds@ifnextchar{%
  \ifx\LTXcmds@LetToken\LTXcmds@CharToken
    \the\expandafter\ltx@LocToksA
  \else
    \expandafter
      \ifx\csname LTXcmds@LetToken\endcsname\LTXcmds@SpaceToken
      \expandafter\expandafter\expandafter\LTXcmds@@ifnextchar
    \else
      \the\expandafter\expandafter\expandafter\ltx@LocToksB
    \fi
  \fi
}
%    \end{macrocode}
%    \end{macro}
%    \begin{macro}{\LTXcmds@@ifnextchar}
%    \cs{futurelet} does not distinguish between a
%    character and a command that is a character
%    (defined by using \cs{let} or \cs{futurelet}).
%    Therefore the space is catched by \cs{romannumeral}
%    with negative character constant that gobbles one
%    optional space.
%    \begin{macrocode}
\def\LTXcmds@@ifnextchar{%
  \expandafter\futurelet
  \expandafter\LTXcmds@LetToken
  \expandafter\LTXcmds@ifnextchar
  \romannumeral-`\.%
}
%    \end{macrocode}
%    \end{macro}
%    \begin{macro}{\LTXcmds@SpaceToken}
%    \begin{macrocode}
\ltx@firstofone{\let\LTXcmds@SpaceToken= } %
%    \end{macrocode}
%    \end{macro}
%
%    \begin{macro}{\ltx@ifnextchar@nospace}
%    \begin{macrocode}
\long\def\ltx@ifnextchar@nospace#1#2#3{%
  \begingroup
  \let\LTXcmds@CharToken= #1\relax
  \ltx@LocToksA{\endgroup#2}%
  \ltx@LocToksB{\endgroup#3}%
  \futurelet\LTXcmds@LetToken\LTXcmds@ifnextchar@nospace
}
%    \end{macrocode}
%    \end{macro}
%    \begin{macro}{\LTXcmds@ifnextchar@nospace}
%    \begin{macrocode}
\def\LTXcmds@ifnextchar@nospace{%
  \the
  \ifx\LTXcmds@LetToken\LTXcmds@CharToken
    \expandafter\ltx@LocToksA
  \else
    \expandafter\ltx@LocToksB
  \fi
}
%    \end{macrocode}
%    \end{macro}
% \subsection{\cs{ltx@leavevmode}, \cs{ltx@mbox}}
%
%    \begin{macro}{\ltx@leavevmode}
%    \begin{macrocode}
\ltx@IfUndefined{quitvmode}{%
  \ltx@IfUndefined{leavevmode}{%
    \ltx@IfUndefined{voidb@x}{%
      \ltx@IfUndefined{newbox}{%
        \def\ltx@leavevmode{%
          \begingroup
            \setbox\ltx@zero=\hbox{}%
            \begingroup
              \setbox\ltx@zero=\hbox{\box\ltx@zero}%
            \endgroup
            \unhbox\ltx@zero
          \endgroup
        }%
      }{%
        \csname newbox\endcsname\LTXcmds@VoidBox
        \ifvoid\LTXcmds@VoidBox
        \else
          \setbox\LTXcmds@VoidBox=\hbox{}%
          \begingroup
            \setbox\LTXcmds@VoidBox=\hbox{\box\LTXcmds@VoidBox}%
          \endgroup
        \fi
        \def\ltx@leavevmode{\unhbox\LTXcmds@VoidBox}%
      }%
    }{%
      \def\ltx@leavevmode{\unhbox\voidb@x}%
    }%
  }{%
    \let\ltx@leavevmode\leavevmode
  }%
}{%
  \let\ltx@leavevmode\quitvmode
}
%    \end{macrocode}
%    \end{macro}
%
%    \begin{macro}{\ltx@mbox}
%    \begin{macrocode}
\def\ltx@mbox{%
  \ltx@leavevmode
  \hbox
}
%    \end{macrocode}
%    \end{macro}
%
% \subsection{Help macros}
%
%    \begin{macro}{\LTXcmds@num}
%    \begin{macrocode}
\ltx@IfUndefined{numexpr}{%
  \def\LTXcmds@num#1{%
    \expandafter\ltx@firstofone\expandafter{%
      \number#1%
    }%
  }%
}{%
  \def\LTXcmds@num#1{%
    \expandafter\ltx@firstofone\expandafter{%
      \the\numexpr#1%
    }%
  }%
}
%    \end{macrocode}
%    \end{macro}
%
% \subsection{Expandable test for emptiness}
%
%    \begin{macrocode}
\ltx@IfUndefined{detokenize}{%
%    \end{macrocode}
%
% \subsubsection{Vanilla \hologo{TeX}}
%
%    \begin{macro}{\ltx@ifempty}
%    The macro is based on \cs{@ifempty} of Robert R. Schneck
%    \cite{ifempty:schneck}
%    and \cs{@ifnull} of Ulrich Diez \cite{ifnull:diez}.
%    There are three cases to consider:
%    \begin{compactenum}
%    \item |#1| is empty,
%    \item |#1| is not empty and the first token
%      is not a begingroup character,
%    \item |#1| starts with a begingroup character (catcode 1).
%    \end{compactenum}
%    \begin{macrocode}
  \def\LTXcmds@temp#1{%
    \long\def\ltx@ifempty##1{%
      \romannumeral0%
      \iffalse{\fi
        \expandafter\ltx@gobble\expandafter{%
          \expandafter{\string##1}%
          \expandafter\ltx@gobble\string
        }%
        \expandafter\ltx@firstofthree\expandafter
        {\iffalse}\fi
        \expandafter#1\ltx@secondoftwo
      }%
      \expandafter#1\ltx@firstoftwo
    }%
%    \end{macrocode}
%    \end{macro}
%    \begin{macro}{\ltx@ifblank}
%    \begin{macrocode}
    \long\def\ltx@ifblank##1{%
      \romannumeral0%
      \iffalse{\fi
        \expandafter\expandafter\expandafter\ltx@gobble
        \expandafter\expandafter\expandafter{%
          \expandafter\expandafter\expandafter{%
            \expandafter\string\ltx@gobble##1.%
          }%
          \expandafter\ltx@gobble\string
        }%
        \expandafter\ltx@firstofthree\expandafter
        {\iffalse}\fi
        \expandafter#1\ltx@secondoftwo
      }%
      \expandafter#1\ltx@firstoftwo
    }%
  }%
  \LTXcmds@temp{ }%
%    \end{macrocode}
%    \end{macro}
%    \begin{macrocode}
}{%
%    \end{macrocode}
%
% \subsubsection{With \cs{detokenize}}
%
%    Ahmed Musa provided \cs{ifstrempty} using \cs{detokenize}
%    and \cs{pdfstrcmp} \cite{ifstrempty:musa}. Ulrich Diez, GL,
%    Heiko Oberdiek
%    improved it further by removing \cs{pdfstrcmp} and
%    taking three arguments
%    \cite{ifstrempty:diez1,ifstrempty:diez2,ifstrempty:diez3,^^A
%      ifstrempty:gl,ifstrempty:ho}.
%
%    \begin{macro}{\ltx@ifempty}
%    \begin{macrocode}
  \long\def\ltx@ifempty#1{%
    \romannumeral%
    \csname
      LTXcmds@ifempty%
      \ifcat$\detokenize{#1}$%
        @%
      \fi
    \endcsname
  }%
%    \end{macrocode}
%    \end{macro}
%    \begin{macro}{\LTXcmds@ifempty@}
%    \begin{macrocode}
  \long\def\LTXcmds@ifempty@#1#2{0 #1}%
%    \end{macrocode}
%    \end{macro}
%    \begin{macro}{\LTXcmds@ifempty}
%    \begin{macrocode}
  \long\def\LTXcmds@ifempty#1#2{0 #2}%
%    \end{macrocode}
%    \end{macro}
%
% \subsubsection{\cs{ltx@ifblank}}
%
%    \begin{macro}{\ltx@ifblank}
%    \begin{macrocode}
  \long\def\ltx@ifblank#1{%
    \romannumeral%
    \csname
      LTXcmds@ifempty%
      \ifcat$\detokenize\expandafter{\ltx@gobble#1.}$%
        @%
      \fi
    \endcsname
  }%
%    \end{macrocode}
%    \end{macro}
%
%    \begin{macrocode}
}
%    \end{macrocode}
%
% \subsection{\cs{ltx@zapspace}}
%
%    \begin{macro}{\ltx@zapspace}
%    \begin{macrocode}
\long\def\ltx@zapspace#1{%
  \romannumeral
  \LTXcmds@zapspace\ltx@zero#1 \@nil
}
%    \end{macrocode}
%    \end{macro}
%    \begin{macro}{\LTXcmds@zapspace}
%    \begin{macrocode}
\long\def\LTXcmds@zapspace#1 #2\@nil{%
  \ltx@ifempty{#2}{%
    #1%
  }{%
    \LTXcmds@zapspace#1#2\@nil
  }%
}
%    \end{macrocode}
%    \end{macro}
%
% \subsection{\cs{ltx@IfBoxEmpty}}
%
%    In case of \hologo{eTeX} the test for an empty box
%    is done via \cs{lastnodetype} as suggested
%    by David Kastrup \cite{ifboxempty:dak}.
%    \begin{macrocode}
\ltx@IfUndefined{lastnodetype}{%
  \catcode`\$=9 %
  \catcode`\&=14 %
}{%
  \catcode`\$=14 %
  \catcode`\&=9 %
}
%    \end{macrocode}
%    \begin{macro}{\ltx@IfBoxEmpty}
%    \begin{macrocode}
\def\ltx@IfBoxEmpty#1{%
  \ifvoid#1\relax
    \expandafter\ltx@secondoftwo
  \else
%    \end{macrocode}
%    Implementation using \hologo{eTeX}'s \cs{lastnodetype}.
%    \begin{macrocode}
&   \begingroup
&     \setbox\ltx@zero=\ifhbox#1\hbox\else\vbox\fi{%
&       \ifhmode\unhcopy\else\unvcopy\fi#1\relax
&       \expandafter
&     }%
&   \expandafter\endgroup
&   \ifnum\lastnodetype<\ltx@zero
&     \expandafter\expandafter\expandafter\ltx@firstoftwo
&   \else
&     \expandafter\expandafter\expandafter\ltx@secondoftwo
&   \fi
%    \end{macrocode}
%    Implementation without \hologo{eTeX} using a signature
%    at the beginning of the test box.
%    \begin{macrocode}
$   \begingroup
$     \setbox\ltx@zero=\ifhbox#1\hbox\else\vbox\fi{%
$       \penalty\ltx@one
$       \ifhmode\unhcopy\else\unvcopy\fi#1\relax
$       \expandafter
$     }%
$     \ifnum\lastpenalty=\ltx@one
%    \end{macrocode}
%    Box 0 has been changed and is restored by
%    closing the group.
%    \begin{macrocode}
$       \endgroup
$       \begingroup
$       \setbox\ltx@zero=\ifhbox#1\hbox\else\vbox\fi{%
$         \penalty\ltx@two
$         \ifhmode\unhcopy\else\unvcopy\fi#1\relax
$         \expandafter
$       }%
$       \ifnum\lastpenalty=\ltx@two
$         \def\next{\endgroup\expandafter\ltx@firstoftwo}%
$       \else
$         \def\next{\endgroup\expandafter\ltx@secondoftwo}%
$       \fi
$     \else
$       \def\next{\endgroup\expandafter\ltx@secondoftwo}%
$     \fi
$   \next
  \fi
}
%    \end{macrocode}
%    \end{macro}
%
%    \begin{macro}{\ltx@IfBoxVoidOrEmpty}
%    \begin{macrocode}
\def\ltx@IfBoxVoidOrEmpty#1{%
  \ifvoid#1\relax
    \expandafter\ltx@thirdoffour
  \fi
  \ltx@IfBoxEmpty{#1}%
}
%    \end{macrocode}
%    \end{macro}
%
%    \begin{macrocode}
\LTXcmds@AtEnd%
%</package>
%    \end{macrocode}
%% \section{Installation}
%
% \subsection{Download}
%
% \paragraph{Package.} This package is available on
% CTAN\footnote{\CTANpkg{ltxcmds}}:
% \begin{description}
% \item[\CTAN{macros/latex/contrib/ltxcmds/ltxcmds.dtx}] The source file.
% \item[\CTAN{macros/latex/contrib/ltxcmds/ltxcmds.pdf}] Documentation.
% \end{description}
%
%
% \paragraph{Bundle.} All the packages of the bundle `ltxcmds'
% are also available in a TDS compliant ZIP archive. There
% the packages are already unpacked and the documentation files
% are generated. The files and directories obey the TDS standard.
% \begin{description}
% \item[\CTANinstall{install/macros/latex/contrib/ltxcmds.tds.zip}]
% \end{description}
% \emph{TDS} refers to the standard ``A Directory Structure
% for \TeX\ Files'' (\CTANpkg{tds}). Directories
% with \xfile{texmf} in their name are usually organized this way.
%
% \subsection{Bundle installation}
%
% \paragraph{Unpacking.} Unpack the \xfile{ltxcmds.tds.zip} in the
% TDS tree (also known as \xfile{texmf} tree) of your choice.
% Example (linux):
% \begin{quote}
%   |unzip ltxcmds.tds.zip -d ~/texmf|
% \end{quote}
%
% \subsection{Package installation}
%
% \paragraph{Unpacking.} The \xfile{.dtx} file is a self-extracting
% \docstrip\ archive. The files are extracted by running the
% \xfile{.dtx} through \plainTeX:
% \begin{quote}
%   \verb|tex ltxcmds.dtx|
% \end{quote}
%
% \paragraph{TDS.} Now the different files must be moved into
% the different directories in your installation TDS tree
% (also known as \xfile{texmf} tree):
% \begin{quote}
% \def\t{^^A
% \begin{tabular}{@{}>{\ttfamily}l@{ $\rightarrow$ }>{\ttfamily}l@{}}
%   ltxcmds.sty & tex/generic/ltxcmds/ltxcmds.sty\\
%   ltxcmds.pdf & doc/latex/ltxcmds/ltxcmds.pdf\\
%   ltxcmds.dtx & source/latex/ltxcmds/ltxcmds.dtx\\
% \end{tabular}^^A
% }^^A
% \sbox0{\t}^^A
% \ifdim\wd0>\linewidth
%   \begingroup
%     \advance\linewidth by\leftmargin
%     \advance\linewidth by\rightmargin
%   \edef\x{\endgroup
%     \def\noexpand\lw{\the\linewidth}^^A
%   }\x
%   \def\lwbox{^^A
%     \leavevmode
%     \hbox to \linewidth{^^A
%       \kern-\leftmargin\relax
%       \hss
%       \usebox0
%       \hss
%       \kern-\rightmargin\relax
%     }^^A
%   }^^A
%   \ifdim\wd0>\lw
%     \sbox0{\small\t}^^A
%     \ifdim\wd0>\linewidth
%       \ifdim\wd0>\lw
%         \sbox0{\footnotesize\t}^^A
%         \ifdim\wd0>\linewidth
%           \ifdim\wd0>\lw
%             \sbox0{\scriptsize\t}^^A
%             \ifdim\wd0>\linewidth
%               \ifdim\wd0>\lw
%                 \sbox0{\tiny\t}^^A
%                 \ifdim\wd0>\linewidth
%                   \lwbox
%                 \else
%                   \usebox0
%                 \fi
%               \else
%                 \lwbox
%               \fi
%             \else
%               \usebox0
%             \fi
%           \else
%             \lwbox
%           \fi
%         \else
%           \usebox0
%         \fi
%       \else
%         \lwbox
%       \fi
%     \else
%       \usebox0
%     \fi
%   \else
%     \lwbox
%   \fi
% \else
%   \usebox0
% \fi
% \end{quote}
% If you have a \xfile{docstrip.cfg} that configures and enables \docstrip's
% TDS installing feature, then some files can already be in the right
% place, see the documentation of \docstrip.
%
% \subsection{Refresh file name databases}
%
% If your \TeX~distribution
% (\TeX\,Live, \mikTeX, \dots) relies on file name databases, you must refresh
% these. For example, \TeX\,Live\ users run \verb|texhash| or
% \verb|mktexlsr|.
%
% \subsection{Some details for the interested}
%
% \paragraph{Unpacking with \LaTeX.}
% The \xfile{.dtx} chooses its action depending on the format:
% \begin{description}
% \item[\plainTeX:] Run \docstrip\ and extract the files.
% \item[\LaTeX:] Generate the documentation.
% \end{description}
% If you insist on using \LaTeX\ for \docstrip\ (really,
% \docstrip\ does not need \LaTeX), then inform the autodetect routine
% about your intention:
% \begin{quote}
%   \verb|latex \let\install=y\input{ltxcmds.dtx}|
% \end{quote}
% Do not forget to quote the argument according to the demands
% of your shell.
%
% \paragraph{Generating the documentation.}
% You can use both the \xfile{.dtx} or the \xfile{.drv} to generate
% the documentation. The process can be configured by the
% configuration file \xfile{ltxdoc.cfg}. For instance, put this
% line into this file, if you want to have A4 as paper format:
% \begin{quote}
%   \verb|\PassOptionsToClass{a4paper}{article}|
% \end{quote}
% An example follows how to generate the
% documentation with pdf\LaTeX:
% \begin{quote}
%\begin{verbatim}
%pdflatex ltxcmds.dtx
%makeindex -s gind.ist ltxcmds.idx
%pdflatex ltxcmds.dtx
%makeindex -s gind.ist ltxcmds.idx
%pdflatex ltxcmds.dtx
%\end{verbatim}
% \end{quote}
%
% \begin{thebibliography}{9}
% \raggedright
%
% \bibitem{ifempty:schneck}
% Robert R. Schneck:
% \textit{Re: \cs{ifempty} solution %
%   (was Macro puzzle: maximally general \cs{ifempty})};
% newsgroup \xnewsgroup{comp.text.tex},
% \url{news:3eef1ada_6@corp.newsgroups.com},
% 2003-06-17.\\
% {\small
% \url{https://groups.google.com/group/comp.text.tex/msg/be03a159ec374895}}
%
% \bibitem{ifnull:diez}
% Ulrich Diez:
% \textit{Re: TeX refuses to strip outer braces in argument};
% newsgroup \xnewsgroup{comp.text.tex},
% \url{news:ibk3t8$ee7$1@news.albasani.net},
% 2010-11-12.\\
% {\small
% \url{https://groups.google.com/group/comp.text.tex/msg/803bd57221a04996}}
%
% \bibitem{ifstrempty:musa}
% Ahmed Musa:
% \textit{Re: TeX refuses to strip outer braces in argument};
% newsgroup \xnewsgroup{comp.text.tex},
% \url{news:f5496afe-40ed-42bd-b629-a2419ecf7c0d@o14g2000prn.googlegroups.com},
% 2010-12-03.\\
% {\small
% \url{https://groups.google.com/group/comp.text.tex/msg/fbf7d61a0c3a807d}}
%
% \bibitem{ifstrempty:diez1}
% Ulrich Diez:
% \textit{Re: TeX refuses to strip outer braces in argument};
% newsgroup \xnewsgroup{comp.text.tex},
% \url{news:idbo94$uka$1@four.albasani.net},
% 2010-12-03.\\
% {\small
% \url{https://groups.google.com/group/comp.text.tex/msg/0c230ee479487962}}
%
% \bibitem{ifstrempty:diez2}
% Ulrich Diez:
% \textit{Re: TeX refuses to strip outer braces in argument};
% newsgroup \xnewsgroup{comp.text.tex},
% \url{news:idbpu4$cg1$1@news.albasani.net},
% 2010-12-03.\\
% {\small
% \url{https://groups.google.com/group/comp.text.tex/msg/bbef4263390d647b}}
%
% \bibitem{ifstrempty:diez3}
% Ulrich Diez:
% \textit{Re: TeX refuses to strip outer braces in argument};
% newsgroup \xnewsgroup{comp.text.tex},
% \url{news:idd4ga$r83$1@four.albasani.net},
% 2010-12-04.\\
% {\small
% \url{https://groups.google.com/group/comp.text.tex/msg/00dfd1ec103cd272}}
%
% \bibitem{ifstrempty:gl}
% GL:
% \textit{Re: TeX refuses to strip outer braces in argument};
% newsgroup \xnewsgroup{comp.text.tex},
% \url{news:4cfa2e27$0$7389$426a74cc@news.free.fr},
% 2010-12-04.\\
% {\small
% \url{https://groups.google.com/group/comp.text.tex/msg/d3a75995c1cf267e}}
%
% \bibitem{ifstrempty:ho}
% Heiko Oberdiek:
% \textit{Re: TeX refuses to strip outer braces in argument};
% newsgroup \xnewsgroup{comp.text.tex},
% \url{news:iddhq1$3kj$1@news.eternal-september.org},
% 2010-12-04.\\
% {\small
% \url{https://groups.google.com/group/comp.text.tex/msg/5f7a23e3ab70e347}}
%
% \bibitem{ifboxempty:dak}
% David Kastrup:
% \textit{How to detect if \cs{vbox} is empty};
% newsgroup \xnewsgroup{comp.text.tex},
% 2011-02-04.\\
% {\small
% \url{https://groups.google.com/group/comp.text.tex/msg/8d3cb89496a4d86d}}
%
% \end{thebibliography}
%
% \begin{History}
%   \begin{Version}{2009/08/05 v1.0}
%   \item
%     First version.
%   \end{Version}
%   \begin{Version}{2009/12/12 v1.1}
%   \item
%     Short title shortened.
%   \item
%     \cs{ltx@IfUndefined} added.
%   \end{Version}
%   \begin{Version}{2010/01/28 v1.2}
%   \item
%     \cs{ltx@RemovePrefix} and \cs{ltx@StripPrefix} added.
%   \item
%     \cs{ltx@ifclassloaded}, \cs{ltx@ifpackageloaded},
%     \cs{ltx@iffileloaded},
%     \cs{ltx@ifclasslater}, \cs{ltx@ifpackagelater},
%     \cs{ltx@iffilelater},
%     \cs{ltx@clsextension}, \cs{ltx@pkgextension} added.
%   \item
%     \cs{ltx@GlobalAppendToMacro}, \cs{ltx@LocalAppendToMacro} added.
%   \end{Version}
%   \begin{Version}{2010/03/01 v1.3}
%   \item
%     \cs{ltx@newif} added.
%   \item
%     \cs{ltx@ifnextchar} added.
%   \item
%     Numbers \cs{ltx@zero}, \cs{ltx@one}, \cs{ltx@two}, \cs{ltx@cclv} added.
%   \end{Version}
%   \begin{Version}{2010/03/09 v1.4}
%   \item
%     \cs{ltx@pkgextension} and \cs{ltx@clsextension} are hardcoded to
%     avoid trouble with \cs{@onlypreamble}.
%   \end{Version}
%   \begin{Version}{2010/04/08 v1.5}
%   \item
%     \cs{ltx@cartwo}, \cs{ltx@cdrtwo},
%     \cs{ltx@carthree}, \cs{ltx@cdrthree},
%     \cs{ltx@carfour}, \cs{ltx@cdrfour} added.
%   \item
%     \cs{ltx@ReturnAfterFi} and \cs{ltx@ReturnAfterElseFi} fixed.
%   \end{Version}
%   \begin{Version}{2010/04/16 v1.6}
%   \item
%     \cs{ltx@leavevmode}, \cs{ltx@mbox} added.
%   \end{Version}
%   \begin{Version}{2010/04/26 v1.7}
%   \item
%     \cs{ltx@GobbleNum}, \cs{ltx@CdrNum}, \cs{ltx@CarNum} added.
%   \item
%     \cs{ltx@carzero}, \cs{ltx@cdrzero} added.
%   \item
%     \cs{ltx@hashchar} added.
%   \end{Version}
%   \begin{Version}{2010/09/11 v1.8}
%   \item
%     \cs{ltx@leftbracechar}, \cs{ltx@rightbracechar} added.
%   \end{Version}
%   \begin{Version}{2010/10/25 v1.9}
%   \item
%     \cs{ltx@LocalAppendToMacro} and \cs{ltx@GlobalAppendToMacro} are
%     now \cs{long}.
%   \end{Version}
%   \begin{Version}{2010/10/31 v1.10}
%   \item
%     \cs{ltx@newglobalif} added.
%   \end{Version}
%   \begin{Version}{2010/11/12 v1.11}
%   \item
%     \cs{ltx@ifempty} added.
%   \item
%     \cs{ltx@firstofthree}, \cs{ltx@secondofthree}, \cs{ltx@thirdofthree}
%     added.
%   \end{Version}
%   \begin{Version}{2010/12/02 v1.12}
%   \item
%     \cs{ltx@onelevel@sanitize} added.
%   \item
%     \cs{LTXcmds@num} fixed for the case with \cs{numexpr}
%     (bug found by GL).
%   \end{Version}
%   \begin{Version}{2010/12/04 v1.13}
%   \item
%     \cs{ltx@ifblank} added.
%   \item
%     Optimization for \cs{ltx@ifempty}.
%   \end{Version}
%   \begin{Version}{2010/12/07 v1.14}
%   \item
%     \cs{ltx@zapspace} added.
%   \end{Version}
%   \begin{Version}{2010/12/12 v1.15}
%   \item
%     \cs{ltx@minusone} added.
%   \end{Version}
%   \begin{Version}{2011/02/04 v1.16}
%   \item
%     \cs{ltx@IfBoxEmpty} and \cs{ltx@IfBoxVoidOrEmpty} added.
%   \item
%     \cs{ltx@firstoffour}, \dots, \cs{ltx@fourthoffour} added.
%   \end{Version}
%   \begin{Version}{2011/02/05 v1.17}
%   \item
%     \cs{ltx@IfBoxEmpty}: an empty box may have non-zero dimensions.
%   \end{Version}
%   \begin{Version}{2011/03/16 v1.18}
%   \item
%     \cs{ltx@ifclasslater} fixed.
%   \end{Version}
%   \begin{Version}{2011/04/14 v1.19}
%   \item
%     \cs{ltx@ifnextchar}: detection of optional spaces modified.
%   \item
%     \cs{ltx(Loc,Glob)(Toks,Dimen,Skip)(A,B,C,D,E)} added.
%   \end{Version}
%   \begin{Version}{2011/04/18 v1.20}
%   \item
%     \cs{ltx@ifnextchar} with conditional support (thanks GL for
%     bug report).
%   \end{Version}
%   \begin{Version}{2011/08/22 v1.21}
%   \item
%     \cs{ltx@GlobalPrependToMacro}, \cs{ltx@LocalPrependToMacro} added
%     (feature request of Martin M\"unch).
%   \end{Version}
%   \begin{Version}{2011/11/09 v1.22}
%   \item
%     \cs{ltx@carsecond}, \cs{ltx@carthird}, \cs{ltx@carfourth},
%     \cs{ltx@CarNumth} added.
%   \item
%     \cs{ltx@cdrzero}, \cs{ltx@cdr}, \cs{ltx@cdrtwo}, cs{ltx@cdrthree},
%     \cs{ltx@cdrfour}, \cs{ltx@CdrNum} modified to retain braces and spaces.
%     They are expandable in two expansion steps.
%   \end{Version}
%   \begin{Version}{2016/05/16 v1.23}
%   \item
%     Documentation updates.
%   \end{Version}
%   \begin{Version}{2019/12/15 v1.24}
%   \item
%     Documentation updates.
%   \end{Version}
% \end{History}
%
% \PrintIndex
%
% \Finale
\endinput
|
% \end{quote}
% Do not forget to quote the argument according to the demands
% of your shell.
%
% \paragraph{Generating the documentation.}
% You can use both the \xfile{.dtx} or the \xfile{.drv} to generate
% the documentation. The process can be configured by the
% configuration file \xfile{ltxdoc.cfg}. For instance, put this
% line into this file, if you want to have A4 as paper format:
% \begin{quote}
%   \verb|\PassOptionsToClass{a4paper}{article}|
% \end{quote}
% An example follows how to generate the
% documentation with pdf\LaTeX:
% \begin{quote}
%\begin{verbatim}
%pdflatex ltxcmds.dtx
%makeindex -s gind.ist ltxcmds.idx
%pdflatex ltxcmds.dtx
%makeindex -s gind.ist ltxcmds.idx
%pdflatex ltxcmds.dtx
%\end{verbatim}
% \end{quote}
%
% \begin{thebibliography}{9}
% \raggedright
%
% \bibitem{ifempty:schneck}
% Robert R. Schneck:
% \textit{Re: \cs{ifempty} solution %
%   (was Macro puzzle: maximally general \cs{ifempty})};
% newsgroup \xnewsgroup{comp.text.tex},
% \url{news:3eef1ada_6@corp.newsgroups.com},
% 2003-06-17.\\
% {\small
% \url{https://groups.google.com/group/comp.text.tex/msg/be03a159ec374895}}
%
% \bibitem{ifnull:diez}
% Ulrich Diez:
% \textit{Re: TeX refuses to strip outer braces in argument};
% newsgroup \xnewsgroup{comp.text.tex},
% \url{news:ibk3t8$ee7$1@news.albasani.net},
% 2010-11-12.\\
% {\small
% \url{https://groups.google.com/group/comp.text.tex/msg/803bd57221a04996}}
%
% \bibitem{ifstrempty:musa}
% Ahmed Musa:
% \textit{Re: TeX refuses to strip outer braces in argument};
% newsgroup \xnewsgroup{comp.text.tex},
% \url{news:f5496afe-40ed-42bd-b629-a2419ecf7c0d@o14g2000prn.googlegroups.com},
% 2010-12-03.\\
% {\small
% \url{https://groups.google.com/group/comp.text.tex/msg/fbf7d61a0c3a807d}}
%
% \bibitem{ifstrempty:diez1}
% Ulrich Diez:
% \textit{Re: TeX refuses to strip outer braces in argument};
% newsgroup \xnewsgroup{comp.text.tex},
% \url{news:idbo94$uka$1@four.albasani.net},
% 2010-12-03.\\
% {\small
% \url{https://groups.google.com/group/comp.text.tex/msg/0c230ee479487962}}
%
% \bibitem{ifstrempty:diez2}
% Ulrich Diez:
% \textit{Re: TeX refuses to strip outer braces in argument};
% newsgroup \xnewsgroup{comp.text.tex},
% \url{news:idbpu4$cg1$1@news.albasani.net},
% 2010-12-03.\\
% {\small
% \url{https://groups.google.com/group/comp.text.tex/msg/bbef4263390d647b}}
%
% \bibitem{ifstrempty:diez3}
% Ulrich Diez:
% \textit{Re: TeX refuses to strip outer braces in argument};
% newsgroup \xnewsgroup{comp.text.tex},
% \url{news:idd4ga$r83$1@four.albasani.net},
% 2010-12-04.\\
% {\small
% \url{https://groups.google.com/group/comp.text.tex/msg/00dfd1ec103cd272}}
%
% \bibitem{ifstrempty:gl}
% GL:
% \textit{Re: TeX refuses to strip outer braces in argument};
% newsgroup \xnewsgroup{comp.text.tex},
% \url{news:4cfa2e27$0$7389$426a74cc@news.free.fr},
% 2010-12-04.\\
% {\small
% \url{https://groups.google.com/group/comp.text.tex/msg/d3a75995c1cf267e}}
%
% \bibitem{ifstrempty:ho}
% Heiko Oberdiek:
% \textit{Re: TeX refuses to strip outer braces in argument};
% newsgroup \xnewsgroup{comp.text.tex},
% \url{news:iddhq1$3kj$1@news.eternal-september.org},
% 2010-12-04.\\
% {\small
% \url{https://groups.google.com/group/comp.text.tex/msg/5f7a23e3ab70e347}}
%
% \bibitem{ifboxempty:dak}
% David Kastrup:
% \textit{How to detect if \cs{vbox} is empty};
% newsgroup \xnewsgroup{comp.text.tex},
% 2011-02-04.\\
% {\small
% \url{https://groups.google.com/group/comp.text.tex/msg/8d3cb89496a4d86d}}
%
% \end{thebibliography}
%
% \begin{History}
%   \begin{Version}{2009/08/05 v1.0}
%   \item
%     First version.
%   \end{Version}
%   \begin{Version}{2009/12/12 v1.1}
%   \item
%     Short title shortened.
%   \item
%     \cs{ltx@IfUndefined} added.
%   \end{Version}
%   \begin{Version}{2010/01/28 v1.2}
%   \item
%     \cs{ltx@RemovePrefix} and \cs{ltx@StripPrefix} added.
%   \item
%     \cs{ltx@ifclassloaded}, \cs{ltx@ifpackageloaded},
%     \cs{ltx@iffileloaded},
%     \cs{ltx@ifclasslater}, \cs{ltx@ifpackagelater},
%     \cs{ltx@iffilelater},
%     \cs{ltx@clsextension}, \cs{ltx@pkgextension} added.
%   \item
%     \cs{ltx@GlobalAppendToMacro}, \cs{ltx@LocalAppendToMacro} added.
%   \end{Version}
%   \begin{Version}{2010/03/01 v1.3}
%   \item
%     \cs{ltx@newif} added.
%   \item
%     \cs{ltx@ifnextchar} added.
%   \item
%     Numbers \cs{ltx@zero}, \cs{ltx@one}, \cs{ltx@two}, \cs{ltx@cclv} added.
%   \end{Version}
%   \begin{Version}{2010/03/09 v1.4}
%   \item
%     \cs{ltx@pkgextension} and \cs{ltx@clsextension} are hardcoded to
%     avoid trouble with \cs{@onlypreamble}.
%   \end{Version}
%   \begin{Version}{2010/04/08 v1.5}
%   \item
%     \cs{ltx@cartwo}, \cs{ltx@cdrtwo},
%     \cs{ltx@carthree}, \cs{ltx@cdrthree},
%     \cs{ltx@carfour}, \cs{ltx@cdrfour} added.
%   \item
%     \cs{ltx@ReturnAfterFi} and \cs{ltx@ReturnAfterElseFi} fixed.
%   \end{Version}
%   \begin{Version}{2010/04/16 v1.6}
%   \item
%     \cs{ltx@leavevmode}, \cs{ltx@mbox} added.
%   \end{Version}
%   \begin{Version}{2010/04/26 v1.7}
%   \item
%     \cs{ltx@GobbleNum}, \cs{ltx@CdrNum}, \cs{ltx@CarNum} added.
%   \item
%     \cs{ltx@carzero}, \cs{ltx@cdrzero} added.
%   \item
%     \cs{ltx@hashchar} added.
%   \end{Version}
%   \begin{Version}{2010/09/11 v1.8}
%   \item
%     \cs{ltx@leftbracechar}, \cs{ltx@rightbracechar} added.
%   \end{Version}
%   \begin{Version}{2010/10/25 v1.9}
%   \item
%     \cs{ltx@LocalAppendToMacro} and \cs{ltx@GlobalAppendToMacro} are
%     now \cs{long}.
%   \end{Version}
%   \begin{Version}{2010/10/31 v1.10}
%   \item
%     \cs{ltx@newglobalif} added.
%   \end{Version}
%   \begin{Version}{2010/11/12 v1.11}
%   \item
%     \cs{ltx@ifempty} added.
%   \item
%     \cs{ltx@firstofthree}, \cs{ltx@secondofthree}, \cs{ltx@thirdofthree}
%     added.
%   \end{Version}
%   \begin{Version}{2010/12/02 v1.12}
%   \item
%     \cs{ltx@onelevel@sanitize} added.
%   \item
%     \cs{LTXcmds@num} fixed for the case with \cs{numexpr}
%     (bug found by GL).
%   \end{Version}
%   \begin{Version}{2010/12/04 v1.13}
%   \item
%     \cs{ltx@ifblank} added.
%   \item
%     Optimization for \cs{ltx@ifempty}.
%   \end{Version}
%   \begin{Version}{2010/12/07 v1.14}
%   \item
%     \cs{ltx@zapspace} added.
%   \end{Version}
%   \begin{Version}{2010/12/12 v1.15}
%   \item
%     \cs{ltx@minusone} added.
%   \end{Version}
%   \begin{Version}{2011/02/04 v1.16}
%   \item
%     \cs{ltx@IfBoxEmpty} and \cs{ltx@IfBoxVoidOrEmpty} added.
%   \item
%     \cs{ltx@firstoffour}, \dots, \cs{ltx@fourthoffour} added.
%   \end{Version}
%   \begin{Version}{2011/02/05 v1.17}
%   \item
%     \cs{ltx@IfBoxEmpty}: an empty box may have non-zero dimensions.
%   \end{Version}
%   \begin{Version}{2011/03/16 v1.18}
%   \item
%     \cs{ltx@ifclasslater} fixed.
%   \end{Version}
%   \begin{Version}{2011/04/14 v1.19}
%   \item
%     \cs{ltx@ifnextchar}: detection of optional spaces modified.
%   \item
%     \cs{ltx(Loc,Glob)(Toks,Dimen,Skip)(A,B,C,D,E)} added.
%   \end{Version}
%   \begin{Version}{2011/04/18 v1.20}
%   \item
%     \cs{ltx@ifnextchar} with conditional support (thanks GL for
%     bug report).
%   \end{Version}
%   \begin{Version}{2011/08/22 v1.21}
%   \item
%     \cs{ltx@GlobalPrependToMacro}, \cs{ltx@LocalPrependToMacro} added
%     (feature request of Martin M\"unch).
%   \end{Version}
%   \begin{Version}{2011/11/09 v1.22}
%   \item
%     \cs{ltx@carsecond}, \cs{ltx@carthird}, \cs{ltx@carfourth},
%     \cs{ltx@CarNumth} added.
%   \item
%     \cs{ltx@cdrzero}, \cs{ltx@cdr}, \cs{ltx@cdrtwo}, cs{ltx@cdrthree},
%     \cs{ltx@cdrfour}, \cs{ltx@CdrNum} modified to retain braces and spaces.
%     They are expandable in two expansion steps.
%   \end{Version}
%   \begin{Version}{2016/05/16 v1.23}
%   \item
%     Documentation updates.
%   \end{Version}
%   \begin{Version}{2019/12/15 v1.24}
%   \item
%     Documentation updates.
%   \end{Version}
% \end{History}
%
% \PrintIndex
%
% \Finale
\endinput

%        (quote the arguments according to the demands of your shell)
%
% Documentation:
%    (a) If ltxcmds.drv is present:
%           latex ltxcmds.drv
%    (b) Without ltxcmds.drv:
%           latex ltxcmds.dtx; ...
%    The class ltxdoc loads the configuration file ltxdoc.cfg
%    if available. Here you can specify further options, e.g.
%    use A4 as paper format:
%       \PassOptionsToClass{a4paper}{article}
%
%    Programm calls to get the documentation (example):
%       pdflatex ltxcmds.dtx
%       makeindex -s gind.ist ltxcmds.idx
%       pdflatex ltxcmds.dtx
%       makeindex -s gind.ist ltxcmds.idx
%       pdflatex ltxcmds.dtx
%
% Installation:
%    TDS:tex/generic/ltxcmds/ltxcmds.sty
%    TDS:doc/latex/ltxcmds/ltxcmds.pdf
%    TDS:source/latex/ltxcmds/ltxcmds.dtx
%
%<*ignore>
\begingroup
  \catcode123=1 %
  \catcode125=2 %
  \def\x{LaTeX2e}%
\expandafter\endgroup
\ifcase 0\ifx\install y1\fi\expandafter
         \ifx\csname processbatchFile\endcsname\relax\else1\fi
         \ifx\fmtname\x\else 1\fi\relax
\else\csname fi\endcsname
%</ignore>
%<*install>
\input docstrip.tex
\Msg{************************************************************************}
\Msg{* Installation}
\Msg{* Package: ltxcmds 2019/12/15 v1.24 LaTeX kernel commands for general use (HO)}
\Msg{************************************************************************}

\keepsilent
\askforoverwritefalse

\let\MetaPrefix\relax
\preamble

This is a generated file.

Project: ltxcmds
Version: 2019/12/15 v1.24

Copyright (C)
   2009-2011 Heiko Oberdiek
   2016-2019 Oberdiek Package Support Group

This work may be distributed and/or modified under the
conditions of the LaTeX Project Public License, either
version 1.3c of this license or (at your option) any later
version. This version of this license is in
   https://www.latex-project.org/lppl/lppl-1-3c.txt
and the latest version of this license is in
   https://www.latex-project.org/lppl.txt
and version 1.3 or later is part of all distributions of
LaTeX version 2005/12/01 or later.

This work has the LPPL maintenance status "maintained".

The Current Maintainers of this work are
Heiko Oberdiek and the Oberdiek Package Support Group
https://github.com/ho-tex/ltxcmds/issues


The Base Interpreter refers to any `TeX-Format',
because some files are installed in TDS:tex/generic//.

This work consists of the main source file ltxcmds.dtx
and the derived files
   ltxcmds.sty, ltxcmds.pdf, ltxcmds.ins, ltxcmds.drv,
   ltxcmds-test1.tex, ltxcmds-test-gobble.tex,
   ltxcmds-test-ifempty.tex, ltxcmds-test-zapspace.tex,
   ltxcmds-test-ifboxempty.tex, ltxcmds-test-nextchar.tex,
   ltxcmds-test-carcdr.tex.

\endpreamble
\let\MetaPrefix\DoubleperCent

\generate{%
  \file{ltxcmds.ins}{\from{ltxcmds.dtx}{install}}%
  \file{ltxcmds.drv}{\from{ltxcmds.dtx}{driver}}%
  \usedir{tex/generic/ltxcmds}%
  \file{ltxcmds.sty}{\from{ltxcmds.dtx}{package}}%
}

\catcode32=13\relax% active space
\let =\space%
\Msg{************************************************************************}
\Msg{*}
\Msg{* To finish the installation you have to move the following}
\Msg{* file into a directory searched by TeX:}
\Msg{*}
\Msg{*     ltxcmds.sty}
\Msg{*}
\Msg{* To produce the documentation run the file `ltxcmds.drv'}
\Msg{* through LaTeX.}
\Msg{*}
\Msg{* Happy TeXing!}
\Msg{*}
\Msg{************************************************************************}

\endbatchfile
%</install>
%<*ignore>
\fi
%</ignore>
%<*driver>
\NeedsTeXFormat{LaTeX2e}
\ProvidesFile{ltxcmds.drv}%
  [2019/12/15 v1.24 LaTeX kernel commands for general use (HO)]%
\documentclass{ltxdoc}
\usepackage{holtxdoc}[2011/11/22]
\usepackage{zref-savepos}
\usepackage{paralist}
\hfuzz=1pt
\begin{document}
  \DocInput{ltxcmds.dtx}%
\end{document}
%</driver>
% \fi
%
%
%
% \GetFileInfo{ltxcmds.drv}
%
% \title{The \xpackage{ltxcmds} package}
% \date{2019/12/15 v1.24}
% \author{Heiko Oberdiek\thanks
% {Please report any issues at \url{https://github.com/ho-tex/ltxcmds/issues}}}
%
% \maketitle
%
% \begin{abstract}
% The package \xpackage{ltxcmds} exports some utility macros
% from the \LaTeX\ kernel into a separate namespace and
% also provides them for other formats such as plain-\TeX.
% \end{abstract}
%
% \tableofcontents
%
% \newcounter{to}
% \renewcommand*{\theto}{to\arabic{to}}
% \newlength\tolength
% \tolength=\dimexpr .5\linewidth+\oddsidemargin+1in+15mm\relax
% \newcommand*{\TO}{^^A
%   \quad
%   \stepcounter{to}^^A
%   \zsavepos\theto
%   \ifdim\zposx\theto sp>0pt
%     \ifdim\zposx\theto sp<\tolength
%       \kern\dimexpr\tolength-\zposx\theto sp\relax
%     \fi
%   \fi
%   $\rightarrow$\quad
%   \ignorespaces
% }
%
% \section{Documentation}
%
% \subsection{Introduction}
%
% Many of my packages also support other formats such as plain-\TeX.
% Because I am rather familiar with the utility macros from
% \LaTeX's kernel (e.g. \cs{@gobble}, \cs{@firstoftwo}), I found
% myself rewriting them again and again, because they are lacking
% in plain-\TeX.
%
% Therefore this package provides often used macros and similar
% ones with the name prefix \cs{ltx@}. This avoids also faulty
% redefinitions. I remember an example where a package redefined
% \cs{@firstoftwo} with forgetting \cs{long}.
%
% \subsection{Numbers}
%
% \begin{declcs}{ltx@zero} \TO 0\\
%   \cs{ltx@one} \TO 1\\
%   \cs{ltx@two} \TO 2\\
%   \cs{ltx@cclv} \TO 255\\
%   \cs{ltx@minusone} \TO -1
% \end{declcs}
% These commands are numbers 0, 1, 2, 255 and -1. They are not digits
% and a space is not gobbled afterwards. Macro \cs{ltx@minusone}
% is available since version 2010/12/12 v1.15.
%
% \subsection{Scratch registers}
%
% Following the conventions of \hologo{plainTeX} and \hologo{LaTeX}
% the first ten registers are free to use. Even numbered registers
% are for local, odd numbered for global use.
%
% \begin{declcs}{ltx@(Loc,Glob)(Toks,Dimen,Skip)(A,B,C,D,E)}
% \end{declcs}
% The name consists of the prefix \cs{ltx@}, then
% \texttt{Loc} or \texttt{Glob} for local or global usage follows.
% The register type is given by \texttt{Toks} for token register,
% \texttt{Dimen} for dimen register and \texttt{Skip} for skip
% register. As last part the registers are numbered from \texttt{A}
% to \texttt{E}. Example: \cs{ltx@LocToksA}.
%
% Since 2011/04/14 v1.19.
%
% \subsection{Argument killers}
%
% \begin{declcs}{ltx@gobble} \M{1} \TO\\
%   \cs{ltx@gobbletwo} \M{1} \M{2} \TO\\
%   \cs{ltx@gobblethree} \M{1} \M{2} \M{3} \TO\\
%   \cs{ltx@gobblefour} \M{1} \M{2} \M{3} \M{4} \TO
% \end{declcs}
%
% \begin{declcs}{ltx@GobbleNum} \M{num} \M{1} \M{2} \dots\space
%   \M{\meta{num}} \TO
% \end{declcs}
% The first argument \meta{num} of macro \cs{ltx@GobbleNum} specifies,
% how many following arguments are eaten. Macro \cs{ltx@GobbleNum}
% is expandable in exact two expansion steps.
%
% \subsection{Argument grabbers}
%
% \begin{declcs}{ltx@firstofone} \M{1} \TO \meta{1}\\[1ex]
%  \cs{ltx@firstoftwo} \M{1} \M{2} \TO \meta{1}\\
%  \cs{ltx@secondoftwo} \M{1} \M{2} \TO \meta{2}\\[1ex]
%  \cs{ltx@firstofthree} \M{1} \M{2} \M{3} \TO \meta{1}\\
%  \cs{ltx@secondofthree} \M{1} \M{2} \M{3} \TO \meta{2}\\
%  \cs{ltx@thirdofthree} \M{1} \M{2} \M{3} \TO \meta{3}\\
%  \cs{ltx@firstoffour} \M{1} \M{2} \M{3} \M{4} \TO \meta{1}\\
%  \cs{ltx@secondoffour} \M{1} \M{2} \M{3} \M{4} \TO \meta{2}\\
%  \cs{ltx@thirdoffour} \M{1} \M{2} \M{3} \M{4} \TO \meta{3}\\
%  \cs{ltx@fourthoffour} \M{1} \M{2} \M{3} \M{4} \TO \meta{4}
% \end{declcs}
% Macros \cs{ltx@firstofthree}, \cs{ltx@secondofthree}
% and \cs{ltx@thirdofthree} were added in version
% 2010/11/12 v1.11.
% Macros \cs{ltx@firstoffour}, \dots, \cs{ltx@fourthoffour}
% were added in version 2011/02/04 v1.16.
%
% \subsection{List helpers}
%
% \begin{declcs}{ltx@carzero} \dots\ \cs{@nil} \TO\\
%   \cs{ltx@cdrzero} \dots\ \cs{@nil} \TO \dots
% \end{declcs}
%
% \begin{declcs}{ltx@car} \M{1} \dots\ \cs{@nil} \TO \meta{1}\\
%   \cs{ltx@cdr} \M{1} \dots\ \cs{@nil} \TO \dots
% \end{declcs}
%
% \begin{declcs}{ltx@cartwo} \M{1} \M{2} \dots\ \cs{@nil}
%   \TO \meta{1}\meta{2}\\
%   \cs{ltx@carsecond} \M{1} \M{2} \dots\ \cs{@nil} \TO \meta{2}\\
%   \cs{ltx@cdrtwo} \M{1} \M{2} \dots\ \cs{@nil} \TO \dots
% \end{declcs}
%
% \begin{declcs}{ltx@carthree} \M{1} \M{2} \M{3} \dots\ \cs{@nil}
%       \TO \meta{1}\meta{2}\meta{3}\\
%   \cs{ltx@carthird} \M{1} \M{2} \M{3} \dots\ \cs{@nil}
%       \TO \meta{3}\\
%   \cs{ltx@cdrthree} \M{1} \M{2} \M{3} \dots\ \cs{@nil} \TO \dots
% \end{declcs}
%
% \begin{declcs}{ltx@carfour} \M{1} \M{2} \M{3} \M{4} \dots\ \cs{@nil}
%       \TO \meta{1}\meta{2}\meta{3}\meta{4}\\
%   \cs{ltx@carfourth} \M{1} \M{2} \M{3} \M{4} \dots\ \cs{@nil}
%       \TO \meta{4}\\
%   \cs{ltx@cdrfour} \M{1} \M{2} \M{3} \M{4} \dots\ \cs{@nil} \TO \dots
% \end{declcs}
%
% \begin{declcs}{ltx@CarNum} \M{num} \M{1} \dots\ \M{\meta{num}}
%   \M{\meta{num}+1} \dots\ \cs{@nil}\\
%   \TO \M{1} \dots\ \M{\meta{num}} \dots\\
%   \cs{ltx@CarNumth} \M{num} \M{1} \dots\ \M{\meta{num}}
%   \M{\meta{num}+1} \dots\ \cs{@nil}\\
%   \TO \M{\meta{num}} \dots\\
%   \cs{ltx@CdrNum} \M{num} \M{1} \dots\ \M{\meta{num}}
%   \M{\meta{num}+1} \dots\ \cs{@nil}\\
%   \TO \M{\meta{num}+1} \dots
% \end{declcs}
%
% Macros with uppercase letters are expandable in two expansion steps.
% Changes in version 2019/12/15 v1.24:
% \begin{itemize}
% \item
%   Macros \cs{ltx@carsecond}, \cs{ltx@carthird}, \cs{ltx@carfourth},
%   \cs{ltx@CarNumth} added.
% \item
%   Macros \cs{ltx@cdr}, \cs{ltx@cdrtwo}, \cs{ltx@cdrthree}, \cs{ltx@cdrfour},
%   \cs{ltx@CdrNum} are expandable in two expansion steps and retain
%   spaces and braces after the first gobbled arguments.
% \end{itemize}
%
% \subsection{Tail recursion}
%
% \begin{declcs}{ltx@ReturnAfterFi} \M{1} \cs{fi} \TO \cs{fi} \meta{1}\\
%   \cs{ltx@ReturnAfterElseFi} \M{1} \cs{else} \M{2} \cs{fi}
%     \TO \cs{fi} \meta{1}
% \end{declcs}
%
% \subsection{Empty macro}
%
% \begin{declcs}{ltx@empty} \TO
% \end{declcs}
%
% \subsection{Characters}
%
% \begin{declcs}{ltx@space} \TO \verb*| |\\
%   \cs{ltx@percentchar} \TO \texttt{\%}\\
%   \cs{ltx@backslashchar} \TO
%      \makeatletter\texttt{\@backslashchar}\\
%   \cs{ltx@hashchar} \TO \texttt{\#}\qquad(since v1.7)\\
%   \cs{ltx@leftbracechar} \TO \texttt{\textbraceleft}\qquad(since v1.8)\\
%   \cs{ltx@rightbracechar} \TO \texttt{\textbraceright}\qquad(since v1.8)
% \end{declcs}
%
% \subsection{Boolean switch}
%
% \begin{declcs}{ltx@newif} \M{cmd}
% \end{declcs}
% \cs{ltx@newif} defines a new boolean switch \meta{cmd} like \cs{newif}.
% Unlike \plainTeX's \cs{newif}, \cs{ltx@newif} is not \cs{outer}.
% The command \meta{cmd} must start with the two characters |if|.
%
% \begin{declcs}{ltx@newglobalif} \M{cmd}
% \end{declcs}
% \cs{ltx@newglobalif} defines a new boolean switch \meta{cmd}
% like \cs{ltx@newif}. However the switch setting commands,
% \meta{cmd} without the prefix |if| and followed by |true| or |false|
% are acting globally.
%
% \subsection{Command definitions}
%
% \begin{declcs}{ltx@ifundefined} \M{cmd} \M{yes} \M{no}
% \end{declcs}
% If \eTeX\ is available, \cs{ifcsname} is used that does
% not have the side effect of defining undefined commands
% with meaning of \cs{relax}. This command is always expandable.
% Change in version 1.1: Also the meaning \cs{relax} is
% always considered ``undefined''.
%
% \begin{declcs}{ltx@IfUndefined} \M{cmd} \M{yes} \M{no}
% \end{declcs}
% If \eTeX\ is available, \cs{ifcsname} is used that does
% not have the side effect of defining undefined commands
% with meaning of \cs{relax}. Also it always checks for
% the meaning of \cs{relax} and considers this as undefined.
% This macro is not expandable without \eTeX.
%
% \begin{declcs}{ltx@LocalExpandAfter}
% \end{declcs}
% It expands the token after the next token but in a local
% context. That is the difference to \cs{expandafter}.
% The local context discards the side effect of \cs{csname}
% and let the command undefined after the expansion step.
%
% \subsection{Stripping}
%
% \begin{declcs}{ltx@RemovePrefix}\\
%   \cs{ltx@StripPrefix}
% \end{declcs}
% All tokens up to and including the next available character
% `\textgreater' are thrown away. Usually it is used to strip
% the first part of the output of the commands \cs{meaning}
% or \cs{pdflastmatch}. Macro \cs{ltx@RemovePrefix}
% has the same meaning as \LaTeX's \cs{strip@prefix}, whereas
% macro \cs{ltx@StripPrefix} expands the next token
% once before stripping the prefix.
%
% \begin{declcs}{ltx@onelevel@sanitize} \M{macro}
% \end{declcs}
% Macro \cs{ltx@onelevel@sanitize} provides \hologo{LaTeX}'s
% \cs{@onelevel@sanitize}. The macro is expanded once and
% the contents is converted to characters with catcode 12 (other)
% and space tokens with catcode 10 (space). Then then sanitized
% contents is stored into the macro again. Since version 1.12.
%
% \subsection{File management}
%
% All macros in this section are expandable like
% the counterparts of the \LaTeX\ kernel. Also they
% can be used after the preamble.
%
% \subsubsection{File extensions}
%
% \begin{declcs}{ltx@clsextension}\\
%   \cs{ltx@pkgextension}
% \end{declcs}
% Macros \cs{ltx@clsextension} and \cs{ltx@styextension}
% stores the strings |cls| and |sty|. In opposite to \hologo{LaTeX}'s
% \cs{@clsextension} and \cs{@styextension} they can also be
% used after |\begin{document}|.
%
% \subsubsection{Load check}
%
% \begin{declcs}{ltx@ifclassloaded} \M{class} \M{yes} \M{no}\\
%   \cs{ltx@ifpackageloaded} \M{package} \M{yes} \M{no}
% \end{declcs}
% Macros \cs{ltx@ifclassloaded}\slash\cs{ltx@ifpackageloaded}
% execute \meta{yes}, if the \meta{class} or \meta{package} is loaded,
% otherwise \meta{no} is called.
% Both \meta{class} and \meta{package} are specified without
% extension. The macros can also be used after |\begin{document}|.
%
% \begin{declcs}{ltx@iffileloaded} \M{file} \M{yes} \M{no}
% \end{declcs}
% If \LaTeX's \cs{ProvidesFile} macro was called before
% using \meta{file} as argument, then \cs{ltx@iffileloaded}
% calls \meta{yes}, otherwise \meta{no}.
% Therefore it is possible
% that the \meta{file} is loaded, but \meta{no} is executed because
% of a missing \cs{ProvidesFile}. The \LaTeX\ kernel does not have
% a counterpart of \cs{ltx@iffileloaded}.
%
% Note that the file name used in \cs{ProvidesFile} and
% \cs{ltx@iffileloaded} must match. For example, if \TeX's
% default extension \xext{tex} was given in the first command,
% then it must also specified in the latter command and vice versa.
%
% \subsubsection{Version date check}
%
% \begin{declcs}{ltx@ifclasslater} \M{class} \M{date} \M{yes} \M{no}\\
%   \cs{ltx@ifpackagelater} \M{package} \M{date} \M{yes} \M{no}\\
%   \cs{ltx@iffilelater} \M{file} \M{date} \M{yes} \M{no}
% \end{declcs}
% If a \cs{ProvidesClass}\slash\cs{ProvidesPackage}\slash
% \cs{ProvidesFile} command with exact the same class\slash package\slash
% file was executed before with an optional
% argument that starts with a \LaTeX\ version date, then
% this version date is compared with the argument \meta{date}.
% If they are equal or if the version date is the later date,
% then \meta{yes} is called. In all other cases \meta{no} is executed.
%
% A \LaTeX\ date has the format |YYYY/MM/DD| with |YYYY| as year
% with four digits, |MM| as month with two digits and |DD| as day
% with two digits. If \pdfTeX's \cs{pdfmatch} is available, then
% it is used to detect the version date, to reject invalid
% date formats and to reject some invalid dates.
% Dates before 1994/01/01 are always invalid, because version
% dates are introduced with \LaTeXe\ in 1994.
%
% \subsection{Macro additions}
%
% \begin{declcs}{ltx@GlobalAppendToMacro} \M{cmd} \M{addition}\\
%   \cs{ltx@LocalAppendToMacro} \M{cmd} \M{addition}
% \end{declcs}
% The \meta{addition} is appended to the parameterless macro \meta{cmd}.
% If \meta{cmd} is undefined or has the meaning \cs{relax}, then
% it will be initialized as empty macro beforehand. Due to a bug \meta{addition}
% must not contain \cs{par} before version 2010/10/25 v1.9.
%
% \begin{declcs}{ltx@GlobalPrependToMacro} \M{cmd} \M{addition}\\
%   \cs{ltx@LocalPrependToMacro} \M{cmd} \M{addition}
% \end{declcs}
% The \meta{addition} is prepended to the parameterless macro \meta{cmd}.
% If \meta{cmd} is undefined or has the meaning \cs{relax}, then
% it will be initialized as empty macro beforehand. The macros
% were added in version 2011/08/22 v1.21.
%
% \subsection{Next character detection}
%
% \begin{declcs}{ltx@ifnextchar} \M{char} \M{yes} \M{no}
% \end{declcs}
% If next character is \meta{char} then \meta{yes} is called,
% otherwise \meta{no}. The character is not removed.
% Spaces are silently removed when looking for \meta{char}
% as \hologo{LaTeX}'s version \cs{kernel@ifnextchar} does.
% But there are also small differences:
% \begin{itemize}
% \item The space can be used as \meta{char}. In this case
%   optional spaces before \meta{char} are not supported of course.
% \item If the optional space is a command that is a character
%   (defined by \cs{let} or \cs{futurelet}), then
%   \cs{kernel@ifnextchar} breaks with an \hologo{TeX} error.
%   \cs{ltx@ifnextchar} silently removes this token as
%   optional space.
% \end{itemize}
% Since 2010/03/01 v1.3.
%
% \begin{declcs}{ltx@ifnextchar@nospace} \M{char} \M{yes} \M{no}
% \end{declcs}
% Macro \cs{ltx@ifnextchar@nospace} behaves like
% macro \cs{ltx@ifnextchar} with the exception that
% optional spaces are not supported before \meta{char}.
% Since 2011/04/14 v1.19.
%
% \subsection{\cs{ltx@leavevmode}, \cs{ltx@mbox}}
%
% \begin{declcs}{ltx@leavevmode}
% \end{declcs}
% Macro \cs{ltx@leavevmode} calls \hologo{pdfTeX}'s \cs{quitvmode}.
% Otherwise \cs{leavevmode} is used and defined if it is necessary.
%
% \begin{declcs}{ltx@mbox}
% \end{declcs}
% Macro \cs{ltx@mbox} reimplements \cs{mbox} with two changes.
% Instead of \cs{leavevmode} it uses \cs{ltx@leavevmode} and
% stops right after \cs{hbox}. Especially it does not grab
% the argument and allows the extended syntax of \cs{hbox}.
%
% \subsection{Expandable test for emptiness}
%
% \begin{declcs}{ltx@ifempty} \M{stuff} \M{yes} \M{no}
% \end{declcs}
% Macro \cs{ltx@ifempty} checks in exact two expansion steps
% whether \meta{stuff}
% is empty or contains token. Depending on the result
% \meta{yes} or \meta{no} is executed. The token in \meta{stuff}
% may contain \cs{par} and unmatched conditionals (\cs{if}, \cs{else},
% \cs{fi}, \dots). Since version 2010/11/12 v1.11.
%
% \begin{declcs}{ltx@ifblank} \M{stuff} \M{yes} \M{no}
% \end{declcs}
% Macro \cs{ltx@ifblank} tests in exact two expansion steps
% if \meta{stuff} is empty or contain only blank spaces.
% In this case argument \meta{yes} is called. If \meta{stuff}
% contains other tokens than spaces then \meta{no} is executed.
% Since version 2010/12/04 v1.13.
%
% \subsection{Stripping spaces}
%
% \begin{declcs}{ltx@zapspace} \M{stuff}
% \end{declcs}
% Macro \cs{ltx@zapspace} strips spaces from \meta{stuff}
% that are not hidden inside curly braces. Like \hologo{LaTeX}'s
% \cs{zap@space} it is expandable. Differences:
% \begin{itemize}
% \item
% Syntax: \cs{zap@space} also expects a space token
% and \cs{@empty} after \meta{stuff}.
% \item
% Macro \cs{ltx@zapspace} is expandable in exact two
% expansion steps.
% \item
% Macro \cs{ltx@zapspace} always retains curly braces.
% \item
% Macro \cs{zap@space} has a bug. It stops stripping spaces
% after a token group in curly braces if the first two tokens
% inside the group are equal.
% \item
% Macro \cs{ltx@zapspace} also works with \cs{par}
% and conditionals (\cs{if}, \cs{else}, \cs{fi}, \dots).
% \end{itemize}
% Macro \cs{ltx@zapspace} is available since version
% 2010/12/07 v1.14.
%
% \subsection{Check for emptiness of boxes}
%
% \begin{declcs}{ltx@IfBoxEmpty} \M{box register number}
%   \M{yes} \M{no}
% \end{declcs}
% Macro \cs{ltx@IfBoxEmpty} calls \meta{yes} if
% the box exists (\cs{ifvoid} returns false) and the box
% does not contain any content. Otherwise if the box
% is void or contains something, then \meta{no} is executed.
% Thus being empty means that the box exists and is either
% an \cs{hbox} or a \cs{vbox} and may even have dimensions other than
% 0.0\,pt, but the box does not contain anything.
% Macro \cs{ltx@IfBoxEmpty} is available since 2010/02/04 v1.16.
%
% \begin{declcs}{ltx@IfBoxVoidOrEmpty} \M{box register number}
%   \M{yes} \M{no}
% \end{declcs}
% Macro \cs{ltx@IfBoxVoidOrEmpty} calls \meta{yes} if
% the box is either void or does not contain any content.
% Otherwise \meta{no} is executed.
% Macro \cs{ltx@IfBoxVoidOrEmpty} is available since 2010/02/04 v1.16.
%
% \StopEventually{
% }
%
% \section{Implementation}
%
% \subsection{Identification}
%
%    \begin{macrocode}
%<*package>
%    \end{macrocode}
%    Reload check, especially if the package is not used with \LaTeX.
%    \begin{macrocode}
\begingroup\catcode61\catcode48\catcode32=10\relax%
  \catcode13=5 % ^^M
  \endlinechar=13 %
  \catcode35=6 % #
  \catcode39=12 % '
  \catcode44=12 % ,
  \catcode45=12 % -
  \catcode46=12 % .
  \catcode58=12 % :
  \catcode64=11 % @
  \catcode123=1 % {
  \catcode125=2 % }
  \expandafter\let\expandafter\x\csname ver@ltxcmds.sty\endcsname
  \ifx\x\relax % plain-TeX, first loading
  \else
    \def\empty{}%
    \ifx\x\empty % LaTeX, first loading,
      % variable is initialized, but \ProvidesPackage not yet seen
    \else
      \expandafter\ifx\csname PackageInfo\endcsname\relax
        \def\x#1#2{%
          \immediate\write-1{Package #1 Info: #2.}%
        }%
      \else
        \def\x#1#2{\PackageInfo{#1}{#2, stopped}}%
      \fi
      \x{ltxcmds}{The package is already loaded}%
      \aftergroup\endinput
    \fi
  \fi
\endgroup%
%    \end{macrocode}
%    Package identification:
%    \begin{macrocode}
\begingroup\catcode61\catcode48\catcode32=10\relax%
  \catcode13=5 % ^^M
  \endlinechar=13 %
  \catcode35=6 % #
  \catcode39=12 % '
  \catcode40=12 % (
  \catcode41=12 % )
  \catcode44=12 % ,
  \catcode45=12 % -
  \catcode46=12 % .
  \catcode47=12 % /
  \catcode58=12 % :
  \catcode64=11 % @
  \catcode91=12 % [
  \catcode93=12 % ]
  \catcode123=1 % {
  \catcode125=2 % }
  \expandafter\ifx\csname ProvidesPackage\endcsname\relax
    \def\x#1#2#3[#4]{\endgroup
      \immediate\write-1{Package: #3 #4}%
      \xdef#1{#4}%
    }%
  \else
    \def\x#1#2[#3]{\endgroup
      #2[{#3}]%
      \ifx#1\@undefined
        \xdef#1{#3}%
      \fi
      \ifx#1\relax
        \xdef#1{#3}%
      \fi
    }%
  \fi
\expandafter\x\csname ver@ltxcmds.sty\endcsname
\ProvidesPackage{ltxcmds}%
  [2019/12/15 v1.24 LaTeX kernel commands for general use (HO)]%
%    \end{macrocode}
%
%    \begin{macrocode}
\begingroup\catcode61\catcode48\catcode32=10\relax%
  \catcode13=5 % ^^M
  \endlinechar=13 %
  \catcode123=1 % {
  \catcode125=2 % }
  \catcode64=11 % @
  \def\x{\endgroup
    \expandafter\edef\csname LTXcmds@AtEnd\endcsname{%
      \endlinechar=\the\endlinechar\relax
      \catcode13=\the\catcode13\relax
      \catcode32=\the\catcode32\relax
      \catcode35=\the\catcode35\relax
      \catcode61=\the\catcode61\relax
      \catcode64=\the\catcode64\relax
      \catcode123=\the\catcode123\relax
      \catcode125=\the\catcode125\relax
    }%
  }%
\x\catcode61\catcode48\catcode32=10\relax%
\catcode13=5 % ^^M
\endlinechar=13 %
\catcode35=6 % #
\catcode64=11 % @
\catcode123=1 % {
\catcode125=2 % }
\def\TMP@EnsureCode#1#2{%
  \edef\LTXcmds@AtEnd{%
    \LTXcmds@AtEnd
    \catcode#1=\the\catcode#1\relax
  }%
  \catcode#1=#2\relax
}
\TMP@EnsureCode{36}{3}% $
\TMP@EnsureCode{38}{4}% &
\TMP@EnsureCode{40}{12}% (
\TMP@EnsureCode{41}{12}% )
\TMP@EnsureCode{45}{12}% -
\TMP@EnsureCode{46}{12}% .
\TMP@EnsureCode{47}{12}% /
\TMP@EnsureCode{60}{12}% <
\TMP@EnsureCode{62}{12}% >
\TMP@EnsureCode{91}{12}% [
\TMP@EnsureCode{96}{12}% `
\TMP@EnsureCode{93}{12}% ]
\TMP@EnsureCode{94}{12}% ^ (superscript) (!)
\TMP@EnsureCode{124}{12}% |
\edef\LTXcmds@AtEnd{\LTXcmds@AtEnd\noexpand\endinput}
%    \end{macrocode}
%
% \subsection{Numbers}
%
%    \begin{macro}{\ltx@zero}
%    \begin{macrocode}
\chardef\ltx@zero=0 %
%    \end{macrocode}
%    \end{macro}
%    \begin{macro}{\ltx@one}
%    \begin{macrocode}
\chardef\ltx@one=1 %
%    \end{macrocode}
%    \end{macro}
%    \begin{macro}{\ltx@two}
%    \begin{macrocode}
\chardef\ltx@two=2 %
%    \end{macrocode}
%    \end{macro}
%    \begin{macro}{\ltx@active}
%    \begin{macrocode}
\chardef\ltx@active=13 %
%    \end{macrocode}
%    \end{macro}
%    \begin{macro}{\ltx@cclv}
%    \begin{macrocode}
\chardef\ltx@cclv=255 %
%    \end{macrocode}
%    \end{macro}
%
%    \begin{macro}{\ltx@minusone}
%    \begin{macrocode}
\def\ltx@minusone{%
  -\ltx@one
}
%    \end{macrocode}
%    \end{macro}
%
% \subsection{Scratch registers}
%
%    \begin{macro}{\ltx@LocToksA}
%    \begin{macrocode}
\toksdef\ltx@LocToksA=0 %
%    \end{macrocode}
%    \end{macro}
%    \begin{macro}{\ltx@LocToksB}
%    \begin{macrocode}
\toksdef\ltx@LocToksB=2 %
%    \end{macrocode}
%    \end{macro}
%    \begin{macro}{\ltx@LocToksC}
%    \begin{macrocode}
\toksdef\ltx@LocToksC=4 %
%    \end{macrocode}
%    \end{macro}
%    \begin{macro}{\ltx@LocToksD}
%    \begin{macrocode}
\toksdef\ltx@LocToksD=6 %
%    \end{macrocode}
%    \end{macro}
%    \begin{macro}{\ltx@LocToksE}
%    \begin{macrocode}
\toksdef\ltx@LocToksE=8 %
%    \end{macrocode}
%    \end{macro}
%
%    \begin{macro}{\ltx@GlobToksA}
%    \begin{macrocode}
\toksdef\ltx@GlobToksA=1 %
%    \end{macrocode}
%    \end{macro}
%    \begin{macro}{\ltx@GlobToksB}
%    \begin{macrocode}
\toksdef\ltx@GlobToksB=3 %
%    \end{macrocode}
%    \end{macro}
%    \begin{macro}{\ltx@GlobToksC}
%    \begin{macrocode}
\toksdef\ltx@GlobToksC=5 %
%    \end{macrocode}
%    \end{macro}
%    \begin{macro}{\ltx@GlobToksD}
%    \begin{macrocode}
\toksdef\ltx@GlobToksD=7 %
%    \end{macrocode}
%    \end{macro}
%    \begin{macro}{\ltx@GlobToksE}
%    \begin{macrocode}
\toksdef\ltx@GlobToksE=9 %
%    \end{macrocode}
%    \end{macro}
%
%    \begin{macro}{\ltx@LocDimenA}
%    \begin{macrocode}
\dimendef\ltx@LocDimenA=0 %
%    \end{macrocode}
%    \end{macro}
%    \begin{macro}{\ltx@LocDimenB}
%    \begin{macrocode}
\dimendef\ltx@LocDimenB=2 %
%    \end{macrocode}
%    \end{macro}
%    \begin{macro}{\ltx@LocDimenC}
%    \begin{macrocode}
\dimendef\ltx@LocDimenC=4 %
%    \end{macrocode}
%    \end{macro}
%    \begin{macro}{\ltx@LocDimenD}
%    \begin{macrocode}
\dimendef\ltx@LocDimenD=6 %
%    \end{macrocode}
%    \end{macro}
%    \begin{macro}{\ltx@LocDimenE}
%    \begin{macrocode}
\dimendef\ltx@LocDimenE=8 %
%    \end{macrocode}
%    \end{macro}
%
%    \begin{macro}{\ltx@GlobDimenA}
%    \begin{macrocode}
\dimendef\ltx@GlobDimenA=1 %
%    \end{macrocode}
%    \end{macro}
%    \begin{macro}{\ltx@GlobDimenB}
%    \begin{macrocode}
\dimendef\ltx@GlobDimenB=3 %
%    \end{macrocode}
%    \end{macro}
%    \begin{macro}{\ltx@GlobDimenC}
%    \begin{macrocode}
\dimendef\ltx@GlobDimenC=5 %
%    \end{macrocode}
%    \end{macro}
%    \begin{macro}{\ltx@GlobDimenD}
%    \begin{macrocode}
\dimendef\ltx@GlobDimenD=7 %
%    \end{macrocode}
%    \end{macro}
%    \begin{macro}{\ltx@GlobDimenE}
%    \begin{macrocode}
\dimendef\ltx@GlobDimenE=9 %
%    \end{macrocode}
%    \end{macro}
%
%    \begin{macro}{\ltx@LocSkipA}
%    \begin{macrocode}
\skipdef\ltx@LocSkipA=0 %
%    \end{macrocode}
%    \end{macro}
%    \begin{macro}{\ltx@LocSkipB}
%    \begin{macrocode}
\skipdef\ltx@LocSkipB=2 %
%    \end{macrocode}
%    \end{macro}
%    \begin{macro}{\ltx@LocSkipC}
%    \begin{macrocode}
\skipdef\ltx@LocSkipC=4 %
%    \end{macrocode}
%    \end{macro}
%    \begin{macro}{\ltx@LocSkipD}
%    \begin{macrocode}
\skipdef\ltx@LocSkipD=6 %
%    \end{macrocode}
%    \end{macro}
%    \begin{macro}{\ltx@LocSkipE}
%    \begin{macrocode}
\skipdef\ltx@LocSkipE=8 %
%    \end{macrocode}
%    \end{macro}
%
%    \begin{macro}{\ltx@GlobSkipA}
%    \begin{macrocode}
\skipdef\ltx@GlobSkipA=1 %
%    \end{macrocode}
%    \end{macro}
%    \begin{macro}{\ltx@GlobSkipB}
%    \begin{macrocode}
\skipdef\ltx@GlobSkipB=3 %
%    \end{macrocode}
%    \end{macro}
%    \begin{macro}{\ltx@GlobSkipC}
%    \begin{macrocode}
\skipdef\ltx@GlobSkipC=5 %
%    \end{macrocode}
%    \end{macro}
%    \begin{macro}{\ltx@GlobSkipD}
%    \begin{macrocode}
\skipdef\ltx@GlobSkipD=7 %
%    \end{macrocode}
%    \end{macro}
%    \begin{macro}{\ltx@GlobSkipE}
%    \begin{macrocode}
\skipdef\ltx@GlobSkipE=9 %
%    \end{macrocode}
%    \end{macro}
%
%
% \subsection{Argument killers}
%
%    \begin{macro}{\ltx@gobble}
%    \begin{macrocode}
\long\def\ltx@gobble#1{}
%    \end{macrocode}
%    \end{macro}
%    \begin{macro}{\ltx@gobbletwo}
%    \begin{macrocode}
\long\def\ltx@gobbletwo#1#2{}
%    \end{macrocode}
%    \end{macro}
%    \begin{macro}{\ltx@gobblethree}
%    \begin{macrocode}
\long\def\ltx@gobblethree#1#2#3{}
%    \end{macrocode}
%    \end{macro}
%    \begin{macro}{\ltx@gobblefour}
%    \begin{macrocode}
\long\def\ltx@gobblefour#1#2#3#4{}
%    \end{macrocode}
%    \end{macro}
%
%    \begin{macro}{\ltx@GobbleNum}
%    \begin{macrocode}
\def\ltx@GobbleNum#1{%
  \romannumeral
  \csname ltx@zero%
  \expandafter\LTXcmds@GobbleNum
  \romannumeral\LTXcmds@num{#1}000{m\endcsname}%
}
%    \end{macrocode}
%    \end{macro}
%    \begin{macro}{\LTXcmds@GobbleNum}
%    \begin{macrocode}
\def\LTXcmds@GobbleNum#1{%
  \csname LTXcmds@G#1\LTXcmds@GobbleNum
}
%    \end{macrocode}
%    \end{macro}
%    \begin{macro}{\LTXcmds@Gm}
%    \begin{macrocode}
\long\def\LTXcmds@Gm#1{%
  \endcsname
}
%    \end{macrocode}
%    \end{macro}
%
% \subsection{Argument grabbers}
%
%    \begin{macro}{\ltx@firstofone}
%    \begin{macrocode}
\long\def\ltx@firstofone#1{#1}
%    \end{macrocode}
%    \end{macro}
%    \begin{macro}{\ltx@firstoftwo}
%    \begin{macrocode}
\long\def\ltx@firstoftwo#1#2{#1}
%    \end{macrocode}
%    \end{macro}
%    \begin{macro}{\ltx@secondoftwo}
%    \begin{macrocode}
\long\def\ltx@secondoftwo#1#2{#2}
%    \end{macrocode}
%    \end{macro}
%
%    \begin{macro}{\ltx@firstofthree}
%    \begin{macrocode}
\long\def\ltx@firstofthree#1#2#3{#1}
%    \end{macrocode}
%    \end{macro}
%    \begin{macro}{\ltx@secondofthree}
%    \begin{macrocode}
\long\def\ltx@secondofthree#1#2#3{#2}
%    \end{macrocode}
%    \end{macro}
%    \begin{macro}{\ltx@thirdofthree}
%    \begin{macrocode}
\long\def\ltx@thirdofthree#1#2#3{#3}%
%    \end{macrocode}
%    \end{macro}
%
%    \begin{macro}{\ltx@firstoffour}
%    \begin{macrocode}
\long\def\ltx@firstoffour#1#2#3#4{#1}
%    \end{macrocode}
%    \end{macro}
%    \begin{macro}{\ltx@secondoffour}
%    \begin{macrocode}
\long\def\ltx@secondoffour#1#2#3#4{#2}
%    \end{macrocode}
%    \end{macro}
%    \begin{macro}{\ltx@thirdoffour}
%    \begin{macrocode}
\long\def\ltx@thirdoffour#1#2#3#4{#3}%
%    \end{macrocode}
%    \end{macro}
%    \begin{macro}{\ltx@fourthoffour}
%    \begin{macrocode}
\long\def\ltx@fourthoffour#1#2#3#4{#4}%
%    \end{macrocode}
%    \end{macro}
%
% \subsection{List helpers}
%
%    \begin{macro}{\ltx@carzero}
%    \begin{macrocode}
\long\def\ltx@carzero#1\@nil{}%
%    \end{macrocode}
%    \end{macro}
%    \begin{macro}{\LTXcmds@cdrzero}
%    \begin{macrocode}
\long\def\LTXcmds@cdrzero#1\@nil{#1}
%    \end{macrocode}
%    \end{macro}
%    \begin{macro}{\ltx@cdrzero}
%    \begin{macrocode}
\def\ltx@cdrzero{%
  \romannumeral\LTXcmds@cdrzero\ltx@zero
}
%    \end{macrocode}
%    \end{macro}
%
%    \begin{macro}{\ltx@car}
%    \begin{macrocode}
\long\def\ltx@car#1#2\@nil{#1}
%    \end{macrocode}
%    \end{macro}
%    \begin{macro}{\ltx@cdr}
%    \begin{macrocode}
\long\def\ltx@cdr#1{%
  \romannumeral\LTXcmds@cdrzero\ltx@zero
}
%    \end{macrocode}
%    \end{macro}
%
%    \begin{macro}{\ltx@cartwo}
%    \begin{macrocode}
\long\def\ltx@cartwo#1#2#3\@nil{#1#2}
%    \end{macrocode}
%    \end{macro}
%    \begin{macro}{\ltx@carsecond}
%    \begin{macrocode}
\long\def\ltx@carsecond#1#2#3\@nil{#2}
%    \end{macrocode}
%    \end{macro}
%    \begin{macro}{\ltx@cdrtwo}
%    \begin{macrocode}
\long\def\ltx@cdrtwo#1#2{%
  \romannumeral\LTXcmds@cdrzero\ltx@zero
}
%    \end{macrocode}
%    \end{macro}
%
%    \begin{macro}{\ltx@carthree}
%    \begin{macrocode}
\long\def\ltx@carthree#1#2#3#4\@nil{#1#2#3}
%    \end{macrocode}
%    \end{macro}
%    \begin{macro}{\ltx@carthird}
%    \begin{macrocode}
\long\def\ltx@carthird#1#2#3#4\@nil{#3}
%    \end{macrocode}
%    \end{macro}
%    \begin{macro}{\ltx@cdrthree}
%    \begin{macrocode}
\long\def\ltx@cdrthree#1#2#3{%
  \romannumeral\LTXcmds@cdrzero\ltx@zero
}
%    \end{macrocode}
%    \end{macro}
%
%    \begin{macro}{\ltx@carfour}
%    \begin{macrocode}
\long\def\ltx@carfour#1#2#3#4#5\@nil{#1#2#3#4}
%    \end{macrocode}
%    \end{macro}
%    \begin{macro}{\ltx@carfourth}
%    \begin{macrocode}
\long\def\ltx@carfourth#1#2#3#4#5\@nil{#4}
%    \end{macrocode}
%    \end{macro}
%    \begin{macro}{\ltx@cdrfour}
%    \begin{macrocode}
\long\def\ltx@cdrfour#1#2#3#4{%
  \romannumeral\LTXcmds@cdrzero\ltx@zero
}
%    \end{macrocode}
%    \end{macro}
%
%    \begin{macro}{\ltx@CarNum}
%    \begin{macrocode}
\def\ltx@CarNum#1{%
  \romannumeral
  \csname LTXcmds@CarNumFinish%
  \expandafter\LTXcmds@CarNum
  \romannumeral\LTXcmds@num{#1}000{x\endcsname}%
}
%    \end{macrocode}
%    \end{macro}
%    \begin{macro}{\LTXcmds@CarNum}
%    \begin{macrocode}
\def\LTXcmds@CarNum#1{%
  \csname LTXcmds@C#1\LTXcmds@CarNum
}
%    \end{macrocode}
%    \end{macro}
%    \begin{macro}{\LTXcmds@Cm}
%    \begin{macrocode}
\long\def\LTXcmds@Cm#1#2{%
  \endcsname{#1#2}%
}
%    \end{macrocode}
%    \end{macro}
%    \begin{macro}{\LTXcmds@Cx}
%    \begin{macrocode}
\def\LTXcmds@Cx#1{%
  \endcsname{}%
}
%    \end{macrocode}
%    \end{macro}
%    \begin{macro}{\LTXcmds@CarNumFinish}
%    \begin{macrocode}
\long\def\LTXcmds@CarNumFinish#1#2\@nil{%
  \ltx@zero
  #1%
}
%    \end{macrocode}
%    \end{macro}
%
%    \begin{macro}{\ltx@CarNumth}
%    \begin{macrocode}
\def\ltx@CarNumth#1{%
  \romannumeral
  \expandafter\expandafter\expandafter
  \LTXcmds@CarNumth
  \ltx@GobbleNum{#1}{}%
}
%    \end{macrocode}
%    \end{macro}
%    \begin{macro}{\LTXcmds@CarNumth}
%    \begin{macrocode}
\long\def\LTXcmds@CarNumth#1#2\@nil{%
  \ltx@zero
  #1%
}
%    \end{macrocode}
%    \end{macro}
%
%    \begin{macro}{\ltx@CdrNum}
%    \begin{macrocode}
\def\ltx@CdrNum#1{%
  \romannumeral%
  \expandafter\expandafter\expandafter\ltx@cdrzero
  \expandafter\expandafter\expandafter\ltx@zero
  \ltx@GobbleNum{#1}%
}
%    \end{macrocode}
%    \end{macro}
%
% \subsection{Tail recursion}
%
%    \begin{macro}{\ltx@ReturnAfterFi}
%    \begin{macrocode}
\long\def\ltx@ReturnAfterFi#1\fi{\fi#1}
%    \end{macrocode}
%    \end{macro}
%    \begin{macro}{\ltx@ReturnAfterElseFi}
%    \begin{macrocode}
\long\def\ltx@ReturnAfterElseFi#1\else#2\fi{\fi#1}
%    \end{macrocode}
%    \end{macro}
%
% \subsection{Empty macro}
%
%    \begin{macro}{\ltx@empty}
%    \begin{macrocode}
\def\ltx@empty{}
%    \end{macrocode}
%    \end{macro}
%
% \subsection{Characters}
%
%    \begin{macro}{\ltx@space}
%    \begin{macrocode}
\def\ltx@space{ }
%    \end{macrocode}
%    \end{macro}
%    \begin{macro}{\ltx@percentchar}
%    \begin{macrocode}
\begingroup
  \lccode`0=`\%\relax
\lowercase{\endgroup
  \def\ltx@percentchar{0}%
}
%    \end{macrocode}
%    \end{macro}
%    \begin{macro}{\ltx@backslashchar}
%    \begin{macrocode}
\begingroup
  \lccode`0=`\\\relax
\lowercase{\endgroup
  \def\ltx@backslashchar{0}%
}
%    \end{macrocode}
%    \end{macro}
%    \begin{macro}{\ltx@hashchar}
%    \begin{macrocode}
\begingroup
  \lccode`0=`\#\relax
\lowercase{\endgroup
  \def\ltx@hashchar{0}%
}
%    \end{macrocode}
%    \end{macro}
%    \begin{macro}{\ltx@leftbracechar}
%    \begin{macrocode}
\begingroup
  \lccode`0=`\{\relax
\lowercase{\endgroup
  \def\ltx@leftbracechar{0}%
}
%    \end{macrocode}
%    \end{macro}
%    \begin{macro}{\ltx@rightbracechar}
%    \begin{macrocode}
\begingroup
  \lccode`0=`\}\relax
\lowercase{\endgroup
  \def\ltx@rightbracechar{0}%
}
%    \end{macrocode}
%    \end{macro}
%
% \subsection{Boolean switch}
%
%    \begin{macro}{\ltx@newif}
%    \begin{macrocode}
\def\ltx@newif#1{%
  \begingroup
    \escapechar=-1 %
  \expandafter\endgroup
  \expandafter\LTXcmds@newif\string#1\@nil
}
%    \end{macrocode}
%    \end{macro}
%    \begin{macro}{\LTXcmds@newif}
%    \begin{macrocode}
\begingroup
  \escapechar=-1 %
\expandafter\endgroup
\expandafter\def\expandafter\LTXcmds@newif\string\if#1\@nil{%
  \expandafter\edef\csname#1true\endcsname{%
    \let
    \expandafter\noexpand\csname if#1\endcsname
    \noexpand\iftrue
  }%
  \expandafter\edef\csname#1false\endcsname{%
    \let
    \expandafter\noexpand\csname if#1\endcsname
    \noexpand\iffalse
  }%
  \csname#1false\endcsname
}
%    \end{macrocode}
%    \end{macro}
%
%    \begin{macro}{\ltx@newglobalif}
%    \begin{macrocode}
\def\ltx@newglobalif#1{%
  \begingroup
    \escapechar=-1 %
  \expandafter\endgroup
  \expandafter\LTXcmds@newglobalif\string#1\@nil
}
%    \end{macrocode}
%    \end{macro}
%    \begin{macro}{\LTXcmds@newglobalif}
%    \begin{macrocode}
\begingroup
  \escapechar=-1 %
\expandafter\endgroup
\expandafter
\def\expandafter\LTXcmds@newglobalif\string\if#1\@nil{%
  \expandafter\edef\csname#1true\endcsname{%
    \global\let
    \expandafter\noexpand\csname if#1\endcsname
    \noexpand\iftrue
  }%
  \expandafter\edef\csname#1false\endcsname{%
    \global\let
    \expandafter\noexpand\csname if#1\endcsname
    \noexpand\iffalse
  }%
  \csname#1false\endcsname
}
%    \end{macrocode}
%    \end{macro}
%
% \subsection{Command definitions}
%
%    \begin{macro}{\ltx@LocalExpandAfter}
%    \begin{macrocode}
\def\ltx@LocalExpandAfter{%
  \begingroup
    \expandafter\expandafter\expandafter
  \endgroup
  \expandafter
}
%    \end{macrocode}
%    \end{macro}
%
%    \begin{macrocode}
\ltx@LocalExpandAfter
\ifx\csname ifcsname\endcsname\relax
%    \end{macrocode}
%    \begin{macro}{\ltx@ifundefined}
%    \begin{macrocode}
  \def\ltx@ifundefined#1{%
    \expandafter\ifx\csname #1\endcsname\relax
      \expandafter\ltx@firstoftwo
    \else
      \expandafter\ltx@secondoftwo
    \fi
  }%
%    \end{macrocode}
%    \end{macro}
%    \begin{macro}{\ltx@IfUndefined}
%    \begin{macrocode}
  \def\ltx@IfUndefined#1{%
    \begingroup\expandafter\expandafter\expandafter\endgroup
    \expandafter\ifx\csname #1\endcsname\relax
      \expandafter\ltx@firstoftwo
    \else
      \expandafter\ltx@secondoftwo
    \fi
  }%
%    \end{macrocode}
%    \end{macro}
%    \begin{macrocode}
  \expandafter\ltx@gobble
\else
  \expandafter\ltx@firstofone
\fi
{%
%    \end{macrocode}
%    \begin{macro}{\ltx@ifundefined}
%    \begin{macrocode}
  \def\ltx@ifundefined#1{%
    \ifcsname #1\endcsname
      \expandafter\ifx\csname #1\endcsname\relax
        \expandafter\expandafter\expandafter\ltx@firstoftwo
      \else
        \expandafter\expandafter\expandafter\ltx@secondoftwo
      \fi
    \else
      \expandafter\ltx@firstoftwo
    \fi
  }%
%    \end{macrocode}
%    \end{macro}
%    \begin{macro}{\ltx@IfUndefined}
%    \begin{macrocode}
  \let\ltx@IfUndefined\ltx@ifundefined
%    \end{macrocode}
%    \end{macro}
%    \begin{macrocode}
}
%    \end{macrocode}
%
% \subsection{Stripping}
%
%    \begin{macro}{\ltx@RemovePrefix}
%    \begin{macrocode}
\def\ltx@RemovePrefix#1>{}
%    \end{macrocode}
%    \end{macro}
%
%    \begin{macro}{\ltx@StripPrefix}
%    \begin{macrocode}
\def\ltx@StripPrefix{%
  \expandafter\ltx@RemovePrefix
}
%    \end{macrocode}
%    \end{macro}
%
%    \begin{macro}{\ltx@onelevel@sanitize}
%    \begin{macrocode}
\def\ltx@onelevel@sanitize#1{%
  \edef#1{%
    \expandafter
    \ltx@RemovePrefix\meaning#1%
  }%
}
%    \end{macrocode}
%    \end{macro}
%
% \subsection{File management}
%
% \subsubsection{File extensions}
%
%    \begin{macro}{\ltx@clsextension}
%    \begin{macrocode}
\def\ltx@clsextension{cls}
%    \end{macrocode}
%    \end{macro}
%    \begin{macro}{\ltx@pkgextension}
%    \begin{macrocode}
\def\ltx@pkgextension{sty}
%    \end{macrocode}
%    \end{macro}
%
% \subsubsection{Load check}
%
%    \begin{macro}{\ltx@iffileloaded}
%    \begin{macrocode}
\def\ltx@iffileloaded#1{%
  \ltx@ifundefined{ver@#1}\ltx@secondoftwo\ltx@firstoftwo
}
%    \end{macrocode}
%    \end{macro}
%    \begin{macro}{\ltx@ifclassloaded}
%    \begin{macrocode}
\def\ltx@ifclassloaded#1{%
  \ltx@iffileloaded{#1.\ltx@clsextension}%
}
%    \end{macrocode}
%    \end{macro}
%    \begin{macro}{\ltx@ifpackageloaded}
%    \begin{macrocode}
\def\ltx@ifpackageloaded#1{%
  \ltx@iffileloaded{#1.\ltx@pkgextension}%
}
%    \end{macrocode}
%    \end{macro}
%
% \subsubsection{Version date check}
%
%    \begin{macro}{\ltx@iffilelater}
%    \begin{macrocode}
\def\ltx@iffilelater#1#2{%
  \ltx@iffileloaded{#1}{%
    \expandafter\LTXcmds@IfLater\expandafter{%
      \number
      \expandafter\expandafter\expandafter\LTXcmds@ParseVersion
      \expandafter\expandafter\expandafter{%
        \csname ver@#1\endcsname
      }%
    \expandafter}\expandafter{%
      \number
      \expandafter\LTXcmds@ParseVersion\expandafter{#2}%
    }%
  }\ltx@secondoftwo
}
%    \end{macrocode}
%    \end{macro}
%    \begin{macro}{\LTXcmds@IfLater}
%    \begin{macrocode}
\def\LTXcmds@IfLater#1#2{%
  \ifcase 0%
      \ifnum#1<19940101 %
      \else
        \ifnum#2<19940101 %
        \else
          \ifnum#2>#1 %
          \else
            1%
          \fi
        \fi
      \fi
      \ltx@space
    \expandafter\ltx@secondoftwo
  \else
    \expandafter\ltx@firstoftwo
  \fi
}
%    \end{macrocode}
%    \end{macro}
%    \begin{macro}{\ltx@ifclasslater}
%    \begin{macrocode}
\def\ltx@ifclasslater#1{%
  \ltx@iffilelater{#1.\ltx@clsextension}%
}
%    \end{macrocode}
%    \end{macro}
%    \begin{macro}{\ltx@ifpackagelater}
%    \begin{macrocode}
\def\ltx@ifpackagelater#1{%
  \ltx@iffilelater{#1.\ltx@pkgextension}%
}
%    \end{macrocode}
%    \end{macro}
%
%    \begin{macrocode}
\ltx@IfUndefined{pdfmatch}{%
%    \end{macrocode}
%    \begin{macro}{\LTXcmds@ParseVersion}
%    \begin{macrocode}
  \def\LTXcmds@ParseVersion#1{%
    \LTXcmds@@ParseVersion#10000/00/00\@nil
  }%
%    \end{macrocode}
%    \end{macro}
%    \begin{macro}{\LTXcmds@@ParseVersion}
%    \begin{macrocode}
  \def\LTXcmds@@ParseVersion#1#2#3#4/#5#6/#7#8#9\@nil{%
    #1#2#3#4#5#6#7#8%
  }%
%    \end{macrocode}
%    \end{macro}
%    \begin{macrocode}
}{%
%    \end{macrocode}
%    \begin{macro}{\LTXcmds@ParseVersion}
%    \begin{macrocode}
  \def\LTXcmds@ParseVersion#1{%
    \ifnum\pdfmatch{%
      ^%
      (199[4-9]|[2-9][0-9][0-9][0-9])/%
      (0[1-9]|1[0-2])/%
      (0[1-9]|[1-2][0-9]|3[0-1])%
    }{#1}=1 %
      \ltx@StripPrefix\pdflastmatch1 %
      \ltx@StripPrefix\pdflastmatch2 %
      \ltx@StripPrefix\pdflastmatch3 %
    \else
      0%
    \fi
  }%
%    \end{macrocode}
%    \end{macro}
%    \begin{macrocode}
}
%    \end{macrocode}
%
% \subsection{Macro additions}
%
%    \begin{macro}{\ltx@GlobalAppendToMacro}
%    \begin{macrocode}
\long\def\ltx@GlobalAppendToMacro#1#2{%
  \ifx\ltx@undefined#1%
    \let#1\ltx@empty
  \else
    \ifx\relax#1%
      \let#1\ltx@empty
    \fi
  \fi
  \begingroup
    \ltx@LocToksA\expandafter{#1#2}%
    \xdef#1{\the\ltx@LocToksA}%
  \endgroup
}
%    \end{macrocode}
%    \end{macro}
%    \begin{macro}{\ltx@LocalAppendToMacro}
%    \begin{macrocode}
\long\def\ltx@LocalAppendToMacro#1#2{%
  \global\let\LTXcmds@gtemp#1%
  \ifx\ltx@undefined\LTXcmds@gtemp
    \global\let\LTXcmds@gtemp\ltx@empty
  \else
    \ifx\relax\LTXcmds@gtemp
      \global\letLTXcmds@gtemp\ltx@empty
    \fi
  \fi
  \begingroup
    \ltx@LocToksA\expandafter{\LTXcmds@gtemp#2}%
    \xdef\LTXcmds@gtemp{\the\ltx@LocToksA}%
  \endgroup
  \let#1\LTXcmds@gtemp
}
%    \end{macrocode}
%    \end{macro}
%
%    \begin{macro}{\ltx@GlobalPrependToMacro}
%    \begin{macrocode}
\long\def\ltx@GlobalPrependToMacro#1#2{%
  \ifx\ltx@undefined#1%
    \let#1\ltx@empty
  \else
    \ifx\relax#1%
      \let#1\ltx@empty
    \fi
  \fi
  \begingroup
    \ltx@LocToksA{#2}%
    \ltx@LocToksB\expandafter{#1}%
    \xdef#1{\the\ltx@LocToksA\the\ltx@LocToksB}%
  \endgroup
}
%    \end{macrocode}
%    \end{macro}
%    \begin{macro}{\ltx@LocalPrependToMacro}
%    \begin{macrocode}
\long\def\ltx@LocalPrependToMacro#1#2{%
  \global\let\LTXcmds@gtemp#1%
  \ifx\ltx@undefined\LTXcmds@gtemp
    \global\let\LTXcmds@gtemp\ltx@empty
  \else
    \ifx\relax\LTXcmds@gtemp
      \global\letLTXcmds@gtemp\ltx@empty
    \fi
  \fi
  \begingroup
    \ltx@LocToksA{#2}%
    \ltx@LocToksB\expandafter{\LTXcmds@gtemp}%
    \xdef\LTXcmds@gtemp{\the\ltx@LocToksA\the\ltx@LocToksB}%
  \endgroup
  \let#1\LTXcmds@gtemp
}
%    \end{macrocode}
%    \end{macro}
%
% \subsection{Next character detection}
%
%    \begin{macro}{\ltx@ifnextchar}
%    \begin{macrocode}
\long\def\ltx@ifnextchar#1#2#3{%
  \begingroup
  \let\LTXcmds@CharToken= #1\relax
  \ltx@LocToksA{\endgroup#2}%
  \ltx@LocToksB{\endgroup#3}%
  \futurelet\LTXcmds@LetToken\LTXcmds@ifnextchar
}
%    \end{macrocode}
%    \end{macro}
%    \begin{macro}{\LTXcmds@ifnextchar}
%    \begin{macrocode}
\def\LTXcmds@ifnextchar{%
  \ifx\LTXcmds@LetToken\LTXcmds@CharToken
    \the\expandafter\ltx@LocToksA
  \else
    \expandafter
      \ifx\csname LTXcmds@LetToken\endcsname\LTXcmds@SpaceToken
      \expandafter\expandafter\expandafter\LTXcmds@@ifnextchar
    \else
      \the\expandafter\expandafter\expandafter\ltx@LocToksB
    \fi
  \fi
}
%    \end{macrocode}
%    \end{macro}
%    \begin{macro}{\LTXcmds@@ifnextchar}
%    \cs{futurelet} does not distinguish between a
%    character and a command that is a character
%    (defined by using \cs{let} or \cs{futurelet}).
%    Therefore the space is catched by \cs{romannumeral}
%    with negative character constant that gobbles one
%    optional space.
%    \begin{macrocode}
\def\LTXcmds@@ifnextchar{%
  \expandafter\futurelet
  \expandafter\LTXcmds@LetToken
  \expandafter\LTXcmds@ifnextchar
  \romannumeral-`\.%
}
%    \end{macrocode}
%    \end{macro}
%    \begin{macro}{\LTXcmds@SpaceToken}
%    \begin{macrocode}
\ltx@firstofone{\let\LTXcmds@SpaceToken= } %
%    \end{macrocode}
%    \end{macro}
%
%    \begin{macro}{\ltx@ifnextchar@nospace}
%    \begin{macrocode}
\long\def\ltx@ifnextchar@nospace#1#2#3{%
  \begingroup
  \let\LTXcmds@CharToken= #1\relax
  \ltx@LocToksA{\endgroup#2}%
  \ltx@LocToksB{\endgroup#3}%
  \futurelet\LTXcmds@LetToken\LTXcmds@ifnextchar@nospace
}
%    \end{macrocode}
%    \end{macro}
%    \begin{macro}{\LTXcmds@ifnextchar@nospace}
%    \begin{macrocode}
\def\LTXcmds@ifnextchar@nospace{%
  \the
  \ifx\LTXcmds@LetToken\LTXcmds@CharToken
    \expandafter\ltx@LocToksA
  \else
    \expandafter\ltx@LocToksB
  \fi
}
%    \end{macrocode}
%    \end{macro}
% \subsection{\cs{ltx@leavevmode}, \cs{ltx@mbox}}
%
%    \begin{macro}{\ltx@leavevmode}
%    \begin{macrocode}
\ltx@IfUndefined{quitvmode}{%
  \ltx@IfUndefined{leavevmode}{%
    \ltx@IfUndefined{voidb@x}{%
      \ltx@IfUndefined{newbox}{%
        \def\ltx@leavevmode{%
          \begingroup
            \setbox\ltx@zero=\hbox{}%
            \begingroup
              \setbox\ltx@zero=\hbox{\box\ltx@zero}%
            \endgroup
            \unhbox\ltx@zero
          \endgroup
        }%
      }{%
        \csname newbox\endcsname\LTXcmds@VoidBox
        \ifvoid\LTXcmds@VoidBox
        \else
          \setbox\LTXcmds@VoidBox=\hbox{}%
          \begingroup
            \setbox\LTXcmds@VoidBox=\hbox{\box\LTXcmds@VoidBox}%
          \endgroup
        \fi
        \def\ltx@leavevmode{\unhbox\LTXcmds@VoidBox}%
      }%
    }{%
      \def\ltx@leavevmode{\unhbox\voidb@x}%
    }%
  }{%
    \let\ltx@leavevmode\leavevmode
  }%
}{%
  \let\ltx@leavevmode\quitvmode
}
%    \end{macrocode}
%    \end{macro}
%
%    \begin{macro}{\ltx@mbox}
%    \begin{macrocode}
\def\ltx@mbox{%
  \ltx@leavevmode
  \hbox
}
%    \end{macrocode}
%    \end{macro}
%
% \subsection{Help macros}
%
%    \begin{macro}{\LTXcmds@num}
%    \begin{macrocode}
\ltx@IfUndefined{numexpr}{%
  \def\LTXcmds@num#1{%
    \expandafter\ltx@firstofone\expandafter{%
      \number#1%
    }%
  }%
}{%
  \def\LTXcmds@num#1{%
    \expandafter\ltx@firstofone\expandafter{%
      \the\numexpr#1%
    }%
  }%
}
%    \end{macrocode}
%    \end{macro}
%
% \subsection{Expandable test for emptiness}
%
%    \begin{macrocode}
\ltx@IfUndefined{detokenize}{%
%    \end{macrocode}
%
% \subsubsection{Vanilla \hologo{TeX}}
%
%    \begin{macro}{\ltx@ifempty}
%    The macro is based on \cs{@ifempty} of Robert R. Schneck
%    \cite{ifempty:schneck}
%    and \cs{@ifnull} of Ulrich Diez \cite{ifnull:diez}.
%    There are three cases to consider:
%    \begin{compactenum}
%    \item |#1| is empty,
%    \item |#1| is not empty and the first token
%      is not a begingroup character,
%    \item |#1| starts with a begingroup character (catcode 1).
%    \end{compactenum}
%    \begin{macrocode}
  \def\LTXcmds@temp#1{%
    \long\def\ltx@ifempty##1{%
      \romannumeral0%
      \iffalse{\fi
        \expandafter\ltx@gobble\expandafter{%
          \expandafter{\string##1}%
          \expandafter\ltx@gobble\string
        }%
        \expandafter\ltx@firstofthree\expandafter
        {\iffalse}\fi
        \expandafter#1\ltx@secondoftwo
      }%
      \expandafter#1\ltx@firstoftwo
    }%
%    \end{macrocode}
%    \end{macro}
%    \begin{macro}{\ltx@ifblank}
%    \begin{macrocode}
    \long\def\ltx@ifblank##1{%
      \romannumeral0%
      \iffalse{\fi
        \expandafter\expandafter\expandafter\ltx@gobble
        \expandafter\expandafter\expandafter{%
          \expandafter\expandafter\expandafter{%
            \expandafter\string\ltx@gobble##1.%
          }%
          \expandafter\ltx@gobble\string
        }%
        \expandafter\ltx@firstofthree\expandafter
        {\iffalse}\fi
        \expandafter#1\ltx@secondoftwo
      }%
      \expandafter#1\ltx@firstoftwo
    }%
  }%
  \LTXcmds@temp{ }%
%    \end{macrocode}
%    \end{macro}
%    \begin{macrocode}
}{%
%    \end{macrocode}
%
% \subsubsection{With \cs{detokenize}}
%
%    Ahmed Musa provided \cs{ifstrempty} using \cs{detokenize}
%    and \cs{pdfstrcmp} \cite{ifstrempty:musa}. Ulrich Diez, GL,
%    Heiko Oberdiek
%    improved it further by removing \cs{pdfstrcmp} and
%    taking three arguments
%    \cite{ifstrempty:diez1,ifstrempty:diez2,ifstrempty:diez3,^^A
%      ifstrempty:gl,ifstrempty:ho}.
%
%    \begin{macro}{\ltx@ifempty}
%    \begin{macrocode}
  \long\def\ltx@ifempty#1{%
    \romannumeral%
    \csname
      LTXcmds@ifempty%
      \ifcat$\detokenize{#1}$%
        @%
      \fi
    \endcsname
  }%
%    \end{macrocode}
%    \end{macro}
%    \begin{macro}{\LTXcmds@ifempty@}
%    \begin{macrocode}
  \long\def\LTXcmds@ifempty@#1#2{0 #1}%
%    \end{macrocode}
%    \end{macro}
%    \begin{macro}{\LTXcmds@ifempty}
%    \begin{macrocode}
  \long\def\LTXcmds@ifempty#1#2{0 #2}%
%    \end{macrocode}
%    \end{macro}
%
% \subsubsection{\cs{ltx@ifblank}}
%
%    \begin{macro}{\ltx@ifblank}
%    \begin{macrocode}
  \long\def\ltx@ifblank#1{%
    \romannumeral%
    \csname
      LTXcmds@ifempty%
      \ifcat$\detokenize\expandafter{\ltx@gobble#1.}$%
        @%
      \fi
    \endcsname
  }%
%    \end{macrocode}
%    \end{macro}
%
%    \begin{macrocode}
}
%    \end{macrocode}
%
% \subsection{\cs{ltx@zapspace}}
%
%    \begin{macro}{\ltx@zapspace}
%    \begin{macrocode}
\long\def\ltx@zapspace#1{%
  \romannumeral
  \LTXcmds@zapspace\ltx@zero#1 \@nil
}
%    \end{macrocode}
%    \end{macro}
%    \begin{macro}{\LTXcmds@zapspace}
%    \begin{macrocode}
\long\def\LTXcmds@zapspace#1 #2\@nil{%
  \ltx@ifempty{#2}{%
    #1%
  }{%
    \LTXcmds@zapspace#1#2\@nil
  }%
}
%    \end{macrocode}
%    \end{macro}
%
% \subsection{\cs{ltx@IfBoxEmpty}}
%
%    In case of \hologo{eTeX} the test for an empty box
%    is done via \cs{lastnodetype} as suggested
%    by David Kastrup \cite{ifboxempty:dak}.
%    \begin{macrocode}
\ltx@IfUndefined{lastnodetype}{%
  \catcode`\$=9 %
  \catcode`\&=14 %
}{%
  \catcode`\$=14 %
  \catcode`\&=9 %
}
%    \end{macrocode}
%    \begin{macro}{\ltx@IfBoxEmpty}
%    \begin{macrocode}
\def\ltx@IfBoxEmpty#1{%
  \ifvoid#1\relax
    \expandafter\ltx@secondoftwo
  \else
%    \end{macrocode}
%    Implementation using \hologo{eTeX}'s \cs{lastnodetype}.
%    \begin{macrocode}
&   \begingroup
&     \setbox\ltx@zero=\ifhbox#1\hbox\else\vbox\fi{%
&       \ifhmode\unhcopy\else\unvcopy\fi#1\relax
&       \expandafter
&     }%
&   \expandafter\endgroup
&   \ifnum\lastnodetype<\ltx@zero
&     \expandafter\expandafter\expandafter\ltx@firstoftwo
&   \else
&     \expandafter\expandafter\expandafter\ltx@secondoftwo
&   \fi
%    \end{macrocode}
%    Implementation without \hologo{eTeX} using a signature
%    at the beginning of the test box.
%    \begin{macrocode}
$   \begingroup
$     \setbox\ltx@zero=\ifhbox#1\hbox\else\vbox\fi{%
$       \penalty\ltx@one
$       \ifhmode\unhcopy\else\unvcopy\fi#1\relax
$       \expandafter
$     }%
$     \ifnum\lastpenalty=\ltx@one
%    \end{macrocode}
%    Box 0 has been changed and is restored by
%    closing the group.
%    \begin{macrocode}
$       \endgroup
$       \begingroup
$       \setbox\ltx@zero=\ifhbox#1\hbox\else\vbox\fi{%
$         \penalty\ltx@two
$         \ifhmode\unhcopy\else\unvcopy\fi#1\relax
$         \expandafter
$       }%
$       \ifnum\lastpenalty=\ltx@two
$         \def\next{\endgroup\expandafter\ltx@firstoftwo}%
$       \else
$         \def\next{\endgroup\expandafter\ltx@secondoftwo}%
$       \fi
$     \else
$       \def\next{\endgroup\expandafter\ltx@secondoftwo}%
$     \fi
$   \next
  \fi
}
%    \end{macrocode}
%    \end{macro}
%
%    \begin{macro}{\ltx@IfBoxVoidOrEmpty}
%    \begin{macrocode}
\def\ltx@IfBoxVoidOrEmpty#1{%
  \ifvoid#1\relax
    \expandafter\ltx@thirdoffour
  \fi
  \ltx@IfBoxEmpty{#1}%
}
%    \end{macrocode}
%    \end{macro}
%
%    \begin{macrocode}
\LTXcmds@AtEnd%
%</package>
%    \end{macrocode}
%% \section{Installation}
%
% \subsection{Download}
%
% \paragraph{Package.} This package is available on
% CTAN\footnote{\CTANpkg{ltxcmds}}:
% \begin{description}
% \item[\CTAN{macros/latex/contrib/ltxcmds/ltxcmds.dtx}] The source file.
% \item[\CTAN{macros/latex/contrib/ltxcmds/ltxcmds.pdf}] Documentation.
% \end{description}
%
%
% \paragraph{Bundle.} All the packages of the bundle `ltxcmds'
% are also available in a TDS compliant ZIP archive. There
% the packages are already unpacked and the documentation files
% are generated. The files and directories obey the TDS standard.
% \begin{description}
% \item[\CTANinstall{install/macros/latex/contrib/ltxcmds.tds.zip}]
% \end{description}
% \emph{TDS} refers to the standard ``A Directory Structure
% for \TeX\ Files'' (\CTANpkg{tds}). Directories
% with \xfile{texmf} in their name are usually organized this way.
%
% \subsection{Bundle installation}
%
% \paragraph{Unpacking.} Unpack the \xfile{ltxcmds.tds.zip} in the
% TDS tree (also known as \xfile{texmf} tree) of your choice.
% Example (linux):
% \begin{quote}
%   |unzip ltxcmds.tds.zip -d ~/texmf|
% \end{quote}
%
% \subsection{Package installation}
%
% \paragraph{Unpacking.} The \xfile{.dtx} file is a self-extracting
% \docstrip\ archive. The files are extracted by running the
% \xfile{.dtx} through \plainTeX:
% \begin{quote}
%   \verb|tex ltxcmds.dtx|
% \end{quote}
%
% \paragraph{TDS.} Now the different files must be moved into
% the different directories in your installation TDS tree
% (also known as \xfile{texmf} tree):
% \begin{quote}
% \def\t{^^A
% \begin{tabular}{@{}>{\ttfamily}l@{ $\rightarrow$ }>{\ttfamily}l@{}}
%   ltxcmds.sty & tex/generic/ltxcmds/ltxcmds.sty\\
%   ltxcmds.pdf & doc/latex/ltxcmds/ltxcmds.pdf\\
%   ltxcmds.dtx & source/latex/ltxcmds/ltxcmds.dtx\\
% \end{tabular}^^A
% }^^A
% \sbox0{\t}^^A
% \ifdim\wd0>\linewidth
%   \begingroup
%     \advance\linewidth by\leftmargin
%     \advance\linewidth by\rightmargin
%   \edef\x{\endgroup
%     \def\noexpand\lw{\the\linewidth}^^A
%   }\x
%   \def\lwbox{^^A
%     \leavevmode
%     \hbox to \linewidth{^^A
%       \kern-\leftmargin\relax
%       \hss
%       \usebox0
%       \hss
%       \kern-\rightmargin\relax
%     }^^A
%   }^^A
%   \ifdim\wd0>\lw
%     \sbox0{\small\t}^^A
%     \ifdim\wd0>\linewidth
%       \ifdim\wd0>\lw
%         \sbox0{\footnotesize\t}^^A
%         \ifdim\wd0>\linewidth
%           \ifdim\wd0>\lw
%             \sbox0{\scriptsize\t}^^A
%             \ifdim\wd0>\linewidth
%               \ifdim\wd0>\lw
%                 \sbox0{\tiny\t}^^A
%                 \ifdim\wd0>\linewidth
%                   \lwbox
%                 \else
%                   \usebox0
%                 \fi
%               \else
%                 \lwbox
%               \fi
%             \else
%               \usebox0
%             \fi
%           \else
%             \lwbox
%           \fi
%         \else
%           \usebox0
%         \fi
%       \else
%         \lwbox
%       \fi
%     \else
%       \usebox0
%     \fi
%   \else
%     \lwbox
%   \fi
% \else
%   \usebox0
% \fi
% \end{quote}
% If you have a \xfile{docstrip.cfg} that configures and enables \docstrip's
% TDS installing feature, then some files can already be in the right
% place, see the documentation of \docstrip.
%
% \subsection{Refresh file name databases}
%
% If your \TeX~distribution
% (\TeX\,Live, \mikTeX, \dots) relies on file name databases, you must refresh
% these. For example, \TeX\,Live\ users run \verb|texhash| or
% \verb|mktexlsr|.
%
% \subsection{Some details for the interested}
%
% \paragraph{Unpacking with \LaTeX.}
% The \xfile{.dtx} chooses its action depending on the format:
% \begin{description}
% \item[\plainTeX:] Run \docstrip\ and extract the files.
% \item[\LaTeX:] Generate the documentation.
% \end{description}
% If you insist on using \LaTeX\ for \docstrip\ (really,
% \docstrip\ does not need \LaTeX), then inform the autodetect routine
% about your intention:
% \begin{quote}
%   \verb|latex \let\install=y% \iffalse meta-comment
%
% File: ltxcmds.dtx
% Version: 2019/12/15 v1.24
% Info: LaTeX kernel commands for general use
%
% Copyright (C)
%    2009-2011 Heiko Oberdiek
%    2016-2019 Oberdiek Package Support Group
%    https://github.com/ho-tex/ltxcmds/issues
%
% This work may be distributed and/or modified under the
% conditions of the LaTeX Project Public License, either
% version 1.3c of this license or (at your option) any later
% version. This version of this license is in
%    https://www.latex-project.org/lppl/lppl-1-3c.txt
% and the latest version of this license is in
%    https://www.latex-project.org/lppl.txt
% and version 1.3 or later is part of all distributions of
% LaTeX version 2005/12/01 or later.
%
% This work has the LPPL maintenance status "maintained".
%
% The Current Maintainers of this work are
% Heiko Oberdiek and the Oberdiek Package Support Group
% https://github.com/ho-tex/ltxcmds/issues
%
% The Base Interpreter refers to any `TeX-Format',
% because some files are installed in TDS:tex/generic//.
%
% This work consists of the main source file ltxcmds.dtx
% and the derived files
%    ltxcmds.sty, ltxcmds.pdf, ltxcmds.ins, ltxcmds.drv,
%    ltxcmds-test1.tex, ltxcmds-test-gobble.tex,
%    ltxcmds-test-ifempty.tex, ltxcmds-test-zapspace.tex,
%    ltxcmds-test-ifboxempty.tex, ltxcmds-test-nextchar.tex,
%    ltxcmds-test-carcdr.tex.
%
% Distribution:
%    CTAN:macros/latex/contrib/ltxcmds/ltxcmds.dtx
%    CTAN:macros/latex/contrib/ltxcmds/ltxcmds.pdf
%
% Unpacking:
%    (a) If ltxcmds.ins is present:
%           tex ltxcmds.ins
%    (b) Without ltxcmds.ins:
%           tex ltxcmds.dtx
%    (c) If you insist on using LaTeX
%           latex \let\install=y% \iffalse meta-comment
%
% File: ltxcmds.dtx
% Version: 2019/12/15 v1.24
% Info: LaTeX kernel commands for general use
%
% Copyright (C)
%    2009-2011 Heiko Oberdiek
%    2016-2019 Oberdiek Package Support Group
%    https://github.com/ho-tex/ltxcmds/issues
%
% This work may be distributed and/or modified under the
% conditions of the LaTeX Project Public License, either
% version 1.3c of this license or (at your option) any later
% version. This version of this license is in
%    https://www.latex-project.org/lppl/lppl-1-3c.txt
% and the latest version of this license is in
%    https://www.latex-project.org/lppl.txt
% and version 1.3 or later is part of all distributions of
% LaTeX version 2005/12/01 or later.
%
% This work has the LPPL maintenance status "maintained".
%
% The Current Maintainers of this work are
% Heiko Oberdiek and the Oberdiek Package Support Group
% https://github.com/ho-tex/ltxcmds/issues
%
% The Base Interpreter refers to any `TeX-Format',
% because some files are installed in TDS:tex/generic//.
%
% This work consists of the main source file ltxcmds.dtx
% and the derived files
%    ltxcmds.sty, ltxcmds.pdf, ltxcmds.ins, ltxcmds.drv,
%    ltxcmds-test1.tex, ltxcmds-test-gobble.tex,
%    ltxcmds-test-ifempty.tex, ltxcmds-test-zapspace.tex,
%    ltxcmds-test-ifboxempty.tex, ltxcmds-test-nextchar.tex,
%    ltxcmds-test-carcdr.tex.
%
% Distribution:
%    CTAN:macros/latex/contrib/ltxcmds/ltxcmds.dtx
%    CTAN:macros/latex/contrib/ltxcmds/ltxcmds.pdf
%
% Unpacking:
%    (a) If ltxcmds.ins is present:
%           tex ltxcmds.ins
%    (b) Without ltxcmds.ins:
%           tex ltxcmds.dtx
%    (c) If you insist on using LaTeX
%           latex \let\install=y\input{ltxcmds.dtx}
%        (quote the arguments according to the demands of your shell)
%
% Documentation:
%    (a) If ltxcmds.drv is present:
%           latex ltxcmds.drv
%    (b) Without ltxcmds.drv:
%           latex ltxcmds.dtx; ...
%    The class ltxdoc loads the configuration file ltxdoc.cfg
%    if available. Here you can specify further options, e.g.
%    use A4 as paper format:
%       \PassOptionsToClass{a4paper}{article}
%
%    Programm calls to get the documentation (example):
%       pdflatex ltxcmds.dtx
%       makeindex -s gind.ist ltxcmds.idx
%       pdflatex ltxcmds.dtx
%       makeindex -s gind.ist ltxcmds.idx
%       pdflatex ltxcmds.dtx
%
% Installation:
%    TDS:tex/generic/ltxcmds/ltxcmds.sty
%    TDS:doc/latex/ltxcmds/ltxcmds.pdf
%    TDS:source/latex/ltxcmds/ltxcmds.dtx
%
%<*ignore>
\begingroup
  \catcode123=1 %
  \catcode125=2 %
  \def\x{LaTeX2e}%
\expandafter\endgroup
\ifcase 0\ifx\install y1\fi\expandafter
         \ifx\csname processbatchFile\endcsname\relax\else1\fi
         \ifx\fmtname\x\else 1\fi\relax
\else\csname fi\endcsname
%</ignore>
%<*install>
\input docstrip.tex
\Msg{************************************************************************}
\Msg{* Installation}
\Msg{* Package: ltxcmds 2019/12/15 v1.24 LaTeX kernel commands for general use (HO)}
\Msg{************************************************************************}

\keepsilent
\askforoverwritefalse

\let\MetaPrefix\relax
\preamble

This is a generated file.

Project: ltxcmds
Version: 2019/12/15 v1.24

Copyright (C)
   2009-2011 Heiko Oberdiek
   2016-2019 Oberdiek Package Support Group

This work may be distributed and/or modified under the
conditions of the LaTeX Project Public License, either
version 1.3c of this license or (at your option) any later
version. This version of this license is in
   https://www.latex-project.org/lppl/lppl-1-3c.txt
and the latest version of this license is in
   https://www.latex-project.org/lppl.txt
and version 1.3 or later is part of all distributions of
LaTeX version 2005/12/01 or later.

This work has the LPPL maintenance status "maintained".

The Current Maintainers of this work are
Heiko Oberdiek and the Oberdiek Package Support Group
https://github.com/ho-tex/ltxcmds/issues


The Base Interpreter refers to any `TeX-Format',
because some files are installed in TDS:tex/generic//.

This work consists of the main source file ltxcmds.dtx
and the derived files
   ltxcmds.sty, ltxcmds.pdf, ltxcmds.ins, ltxcmds.drv,
   ltxcmds-test1.tex, ltxcmds-test-gobble.tex,
   ltxcmds-test-ifempty.tex, ltxcmds-test-zapspace.tex,
   ltxcmds-test-ifboxempty.tex, ltxcmds-test-nextchar.tex,
   ltxcmds-test-carcdr.tex.

\endpreamble
\let\MetaPrefix\DoubleperCent

\generate{%
  \file{ltxcmds.ins}{\from{ltxcmds.dtx}{install}}%
  \file{ltxcmds.drv}{\from{ltxcmds.dtx}{driver}}%
  \usedir{tex/generic/ltxcmds}%
  \file{ltxcmds.sty}{\from{ltxcmds.dtx}{package}}%
}

\catcode32=13\relax% active space
\let =\space%
\Msg{************************************************************************}
\Msg{*}
\Msg{* To finish the installation you have to move the following}
\Msg{* file into a directory searched by TeX:}
\Msg{*}
\Msg{*     ltxcmds.sty}
\Msg{*}
\Msg{* To produce the documentation run the file `ltxcmds.drv'}
\Msg{* through LaTeX.}
\Msg{*}
\Msg{* Happy TeXing!}
\Msg{*}
\Msg{************************************************************************}

\endbatchfile
%</install>
%<*ignore>
\fi
%</ignore>
%<*driver>
\NeedsTeXFormat{LaTeX2e}
\ProvidesFile{ltxcmds.drv}%
  [2019/12/15 v1.24 LaTeX kernel commands for general use (HO)]%
\documentclass{ltxdoc}
\usepackage{holtxdoc}[2011/11/22]
\usepackage{zref-savepos}
\usepackage{paralist}
\hfuzz=1pt
\begin{document}
  \DocInput{ltxcmds.dtx}%
\end{document}
%</driver>
% \fi
%
%
%
% \GetFileInfo{ltxcmds.drv}
%
% \title{The \xpackage{ltxcmds} package}
% \date{2019/12/15 v1.24}
% \author{Heiko Oberdiek\thanks
% {Please report any issues at \url{https://github.com/ho-tex/ltxcmds/issues}}}
%
% \maketitle
%
% \begin{abstract}
% The package \xpackage{ltxcmds} exports some utility macros
% from the \LaTeX\ kernel into a separate namespace and
% also provides them for other formats such as plain-\TeX.
% \end{abstract}
%
% \tableofcontents
%
% \newcounter{to}
% \renewcommand*{\theto}{to\arabic{to}}
% \newlength\tolength
% \tolength=\dimexpr .5\linewidth+\oddsidemargin+1in+15mm\relax
% \newcommand*{\TO}{^^A
%   \quad
%   \stepcounter{to}^^A
%   \zsavepos\theto
%   \ifdim\zposx\theto sp>0pt
%     \ifdim\zposx\theto sp<\tolength
%       \kern\dimexpr\tolength-\zposx\theto sp\relax
%     \fi
%   \fi
%   $\rightarrow$\quad
%   \ignorespaces
% }
%
% \section{Documentation}
%
% \subsection{Introduction}
%
% Many of my packages also support other formats such as plain-\TeX.
% Because I am rather familiar with the utility macros from
% \LaTeX's kernel (e.g. \cs{@gobble}, \cs{@firstoftwo}), I found
% myself rewriting them again and again, because they are lacking
% in plain-\TeX.
%
% Therefore this package provides often used macros and similar
% ones with the name prefix \cs{ltx@}. This avoids also faulty
% redefinitions. I remember an example where a package redefined
% \cs{@firstoftwo} with forgetting \cs{long}.
%
% \subsection{Numbers}
%
% \begin{declcs}{ltx@zero} \TO 0\\
%   \cs{ltx@one} \TO 1\\
%   \cs{ltx@two} \TO 2\\
%   \cs{ltx@cclv} \TO 255\\
%   \cs{ltx@minusone} \TO -1
% \end{declcs}
% These commands are numbers 0, 1, 2, 255 and -1. They are not digits
% and a space is not gobbled afterwards. Macro \cs{ltx@minusone}
% is available since version 2010/12/12 v1.15.
%
% \subsection{Scratch registers}
%
% Following the conventions of \hologo{plainTeX} and \hologo{LaTeX}
% the first ten registers are free to use. Even numbered registers
% are for local, odd numbered for global use.
%
% \begin{declcs}{ltx@(Loc,Glob)(Toks,Dimen,Skip)(A,B,C,D,E)}
% \end{declcs}
% The name consists of the prefix \cs{ltx@}, then
% \texttt{Loc} or \texttt{Glob} for local or global usage follows.
% The register type is given by \texttt{Toks} for token register,
% \texttt{Dimen} for dimen register and \texttt{Skip} for skip
% register. As last part the registers are numbered from \texttt{A}
% to \texttt{E}. Example: \cs{ltx@LocToksA}.
%
% Since 2011/04/14 v1.19.
%
% \subsection{Argument killers}
%
% \begin{declcs}{ltx@gobble} \M{1} \TO\\
%   \cs{ltx@gobbletwo} \M{1} \M{2} \TO\\
%   \cs{ltx@gobblethree} \M{1} \M{2} \M{3} \TO\\
%   \cs{ltx@gobblefour} \M{1} \M{2} \M{3} \M{4} \TO
% \end{declcs}
%
% \begin{declcs}{ltx@GobbleNum} \M{num} \M{1} \M{2} \dots\space
%   \M{\meta{num}} \TO
% \end{declcs}
% The first argument \meta{num} of macro \cs{ltx@GobbleNum} specifies,
% how many following arguments are eaten. Macro \cs{ltx@GobbleNum}
% is expandable in exact two expansion steps.
%
% \subsection{Argument grabbers}
%
% \begin{declcs}{ltx@firstofone} \M{1} \TO \meta{1}\\[1ex]
%  \cs{ltx@firstoftwo} \M{1} \M{2} \TO \meta{1}\\
%  \cs{ltx@secondoftwo} \M{1} \M{2} \TO \meta{2}\\[1ex]
%  \cs{ltx@firstofthree} \M{1} \M{2} \M{3} \TO \meta{1}\\
%  \cs{ltx@secondofthree} \M{1} \M{2} \M{3} \TO \meta{2}\\
%  \cs{ltx@thirdofthree} \M{1} \M{2} \M{3} \TO \meta{3}\\
%  \cs{ltx@firstoffour} \M{1} \M{2} \M{3} \M{4} \TO \meta{1}\\
%  \cs{ltx@secondoffour} \M{1} \M{2} \M{3} \M{4} \TO \meta{2}\\
%  \cs{ltx@thirdoffour} \M{1} \M{2} \M{3} \M{4} \TO \meta{3}\\
%  \cs{ltx@fourthoffour} \M{1} \M{2} \M{3} \M{4} \TO \meta{4}
% \end{declcs}
% Macros \cs{ltx@firstofthree}, \cs{ltx@secondofthree}
% and \cs{ltx@thirdofthree} were added in version
% 2010/11/12 v1.11.
% Macros \cs{ltx@firstoffour}, \dots, \cs{ltx@fourthoffour}
% were added in version 2011/02/04 v1.16.
%
% \subsection{List helpers}
%
% \begin{declcs}{ltx@carzero} \dots\ \cs{@nil} \TO\\
%   \cs{ltx@cdrzero} \dots\ \cs{@nil} \TO \dots
% \end{declcs}
%
% \begin{declcs}{ltx@car} \M{1} \dots\ \cs{@nil} \TO \meta{1}\\
%   \cs{ltx@cdr} \M{1} \dots\ \cs{@nil} \TO \dots
% \end{declcs}
%
% \begin{declcs}{ltx@cartwo} \M{1} \M{2} \dots\ \cs{@nil}
%   \TO \meta{1}\meta{2}\\
%   \cs{ltx@carsecond} \M{1} \M{2} \dots\ \cs{@nil} \TO \meta{2}\\
%   \cs{ltx@cdrtwo} \M{1} \M{2} \dots\ \cs{@nil} \TO \dots
% \end{declcs}
%
% \begin{declcs}{ltx@carthree} \M{1} \M{2} \M{3} \dots\ \cs{@nil}
%       \TO \meta{1}\meta{2}\meta{3}\\
%   \cs{ltx@carthird} \M{1} \M{2} \M{3} \dots\ \cs{@nil}
%       \TO \meta{3}\\
%   \cs{ltx@cdrthree} \M{1} \M{2} \M{3} \dots\ \cs{@nil} \TO \dots
% \end{declcs}
%
% \begin{declcs}{ltx@carfour} \M{1} \M{2} \M{3} \M{4} \dots\ \cs{@nil}
%       \TO \meta{1}\meta{2}\meta{3}\meta{4}\\
%   \cs{ltx@carfourth} \M{1} \M{2} \M{3} \M{4} \dots\ \cs{@nil}
%       \TO \meta{4}\\
%   \cs{ltx@cdrfour} \M{1} \M{2} \M{3} \M{4} \dots\ \cs{@nil} \TO \dots
% \end{declcs}
%
% \begin{declcs}{ltx@CarNum} \M{num} \M{1} \dots\ \M{\meta{num}}
%   \M{\meta{num}+1} \dots\ \cs{@nil}\\
%   \TO \M{1} \dots\ \M{\meta{num}} \dots\\
%   \cs{ltx@CarNumth} \M{num} \M{1} \dots\ \M{\meta{num}}
%   \M{\meta{num}+1} \dots\ \cs{@nil}\\
%   \TO \M{\meta{num}} \dots\\
%   \cs{ltx@CdrNum} \M{num} \M{1} \dots\ \M{\meta{num}}
%   \M{\meta{num}+1} \dots\ \cs{@nil}\\
%   \TO \M{\meta{num}+1} \dots
% \end{declcs}
%
% Macros with uppercase letters are expandable in two expansion steps.
% Changes in version 2019/12/15 v1.24:
% \begin{itemize}
% \item
%   Macros \cs{ltx@carsecond}, \cs{ltx@carthird}, \cs{ltx@carfourth},
%   \cs{ltx@CarNumth} added.
% \item
%   Macros \cs{ltx@cdr}, \cs{ltx@cdrtwo}, \cs{ltx@cdrthree}, \cs{ltx@cdrfour},
%   \cs{ltx@CdrNum} are expandable in two expansion steps and retain
%   spaces and braces after the first gobbled arguments.
% \end{itemize}
%
% \subsection{Tail recursion}
%
% \begin{declcs}{ltx@ReturnAfterFi} \M{1} \cs{fi} \TO \cs{fi} \meta{1}\\
%   \cs{ltx@ReturnAfterElseFi} \M{1} \cs{else} \M{2} \cs{fi}
%     \TO \cs{fi} \meta{1}
% \end{declcs}
%
% \subsection{Empty macro}
%
% \begin{declcs}{ltx@empty} \TO
% \end{declcs}
%
% \subsection{Characters}
%
% \begin{declcs}{ltx@space} \TO \verb*| |\\
%   \cs{ltx@percentchar} \TO \texttt{\%}\\
%   \cs{ltx@backslashchar} \TO
%      \makeatletter\texttt{\@backslashchar}\\
%   \cs{ltx@hashchar} \TO \texttt{\#}\qquad(since v1.7)\\
%   \cs{ltx@leftbracechar} \TO \texttt{\textbraceleft}\qquad(since v1.8)\\
%   \cs{ltx@rightbracechar} \TO \texttt{\textbraceright}\qquad(since v1.8)
% \end{declcs}
%
% \subsection{Boolean switch}
%
% \begin{declcs}{ltx@newif} \M{cmd}
% \end{declcs}
% \cs{ltx@newif} defines a new boolean switch \meta{cmd} like \cs{newif}.
% Unlike \plainTeX's \cs{newif}, \cs{ltx@newif} is not \cs{outer}.
% The command \meta{cmd} must start with the two characters |if|.
%
% \begin{declcs}{ltx@newglobalif} \M{cmd}
% \end{declcs}
% \cs{ltx@newglobalif} defines a new boolean switch \meta{cmd}
% like \cs{ltx@newif}. However the switch setting commands,
% \meta{cmd} without the prefix |if| and followed by |true| or |false|
% are acting globally.
%
% \subsection{Command definitions}
%
% \begin{declcs}{ltx@ifundefined} \M{cmd} \M{yes} \M{no}
% \end{declcs}
% If \eTeX\ is available, \cs{ifcsname} is used that does
% not have the side effect of defining undefined commands
% with meaning of \cs{relax}. This command is always expandable.
% Change in version 1.1: Also the meaning \cs{relax} is
% always considered ``undefined''.
%
% \begin{declcs}{ltx@IfUndefined} \M{cmd} \M{yes} \M{no}
% \end{declcs}
% If \eTeX\ is available, \cs{ifcsname} is used that does
% not have the side effect of defining undefined commands
% with meaning of \cs{relax}. Also it always checks for
% the meaning of \cs{relax} and considers this as undefined.
% This macro is not expandable without \eTeX.
%
% \begin{declcs}{ltx@LocalExpandAfter}
% \end{declcs}
% It expands the token after the next token but in a local
% context. That is the difference to \cs{expandafter}.
% The local context discards the side effect of \cs{csname}
% and let the command undefined after the expansion step.
%
% \subsection{Stripping}
%
% \begin{declcs}{ltx@RemovePrefix}\\
%   \cs{ltx@StripPrefix}
% \end{declcs}
% All tokens up to and including the next available character
% `\textgreater' are thrown away. Usually it is used to strip
% the first part of the output of the commands \cs{meaning}
% or \cs{pdflastmatch}. Macro \cs{ltx@RemovePrefix}
% has the same meaning as \LaTeX's \cs{strip@prefix}, whereas
% macro \cs{ltx@StripPrefix} expands the next token
% once before stripping the prefix.
%
% \begin{declcs}{ltx@onelevel@sanitize} \M{macro}
% \end{declcs}
% Macro \cs{ltx@onelevel@sanitize} provides \hologo{LaTeX}'s
% \cs{@onelevel@sanitize}. The macro is expanded once and
% the contents is converted to characters with catcode 12 (other)
% and space tokens with catcode 10 (space). Then then sanitized
% contents is stored into the macro again. Since version 1.12.
%
% \subsection{File management}
%
% All macros in this section are expandable like
% the counterparts of the \LaTeX\ kernel. Also they
% can be used after the preamble.
%
% \subsubsection{File extensions}
%
% \begin{declcs}{ltx@clsextension}\\
%   \cs{ltx@pkgextension}
% \end{declcs}
% Macros \cs{ltx@clsextension} and \cs{ltx@styextension}
% stores the strings |cls| and |sty|. In opposite to \hologo{LaTeX}'s
% \cs{@clsextension} and \cs{@styextension} they can also be
% used after |\begin{document}|.
%
% \subsubsection{Load check}
%
% \begin{declcs}{ltx@ifclassloaded} \M{class} \M{yes} \M{no}\\
%   \cs{ltx@ifpackageloaded} \M{package} \M{yes} \M{no}
% \end{declcs}
% Macros \cs{ltx@ifclassloaded}\slash\cs{ltx@ifpackageloaded}
% execute \meta{yes}, if the \meta{class} or \meta{package} is loaded,
% otherwise \meta{no} is called.
% Both \meta{class} and \meta{package} are specified without
% extension. The macros can also be used after |\begin{document}|.
%
% \begin{declcs}{ltx@iffileloaded} \M{file} \M{yes} \M{no}
% \end{declcs}
% If \LaTeX's \cs{ProvidesFile} macro was called before
% using \meta{file} as argument, then \cs{ltx@iffileloaded}
% calls \meta{yes}, otherwise \meta{no}.
% Therefore it is possible
% that the \meta{file} is loaded, but \meta{no} is executed because
% of a missing \cs{ProvidesFile}. The \LaTeX\ kernel does not have
% a counterpart of \cs{ltx@iffileloaded}.
%
% Note that the file name used in \cs{ProvidesFile} and
% \cs{ltx@iffileloaded} must match. For example, if \TeX's
% default extension \xext{tex} was given in the first command,
% then it must also specified in the latter command and vice versa.
%
% \subsubsection{Version date check}
%
% \begin{declcs}{ltx@ifclasslater} \M{class} \M{date} \M{yes} \M{no}\\
%   \cs{ltx@ifpackagelater} \M{package} \M{date} \M{yes} \M{no}\\
%   \cs{ltx@iffilelater} \M{file} \M{date} \M{yes} \M{no}
% \end{declcs}
% If a \cs{ProvidesClass}\slash\cs{ProvidesPackage}\slash
% \cs{ProvidesFile} command with exact the same class\slash package\slash
% file was executed before with an optional
% argument that starts with a \LaTeX\ version date, then
% this version date is compared with the argument \meta{date}.
% If they are equal or if the version date is the later date,
% then \meta{yes} is called. In all other cases \meta{no} is executed.
%
% A \LaTeX\ date has the format |YYYY/MM/DD| with |YYYY| as year
% with four digits, |MM| as month with two digits and |DD| as day
% with two digits. If \pdfTeX's \cs{pdfmatch} is available, then
% it is used to detect the version date, to reject invalid
% date formats and to reject some invalid dates.
% Dates before 1994/01/01 are always invalid, because version
% dates are introduced with \LaTeXe\ in 1994.
%
% \subsection{Macro additions}
%
% \begin{declcs}{ltx@GlobalAppendToMacro} \M{cmd} \M{addition}\\
%   \cs{ltx@LocalAppendToMacro} \M{cmd} \M{addition}
% \end{declcs}
% The \meta{addition} is appended to the parameterless macro \meta{cmd}.
% If \meta{cmd} is undefined or has the meaning \cs{relax}, then
% it will be initialized as empty macro beforehand. Due to a bug \meta{addition}
% must not contain \cs{par} before version 2010/10/25 v1.9.
%
% \begin{declcs}{ltx@GlobalPrependToMacro} \M{cmd} \M{addition}\\
%   \cs{ltx@LocalPrependToMacro} \M{cmd} \M{addition}
% \end{declcs}
% The \meta{addition} is prepended to the parameterless macro \meta{cmd}.
% If \meta{cmd} is undefined or has the meaning \cs{relax}, then
% it will be initialized as empty macro beforehand. The macros
% were added in version 2011/08/22 v1.21.
%
% \subsection{Next character detection}
%
% \begin{declcs}{ltx@ifnextchar} \M{char} \M{yes} \M{no}
% \end{declcs}
% If next character is \meta{char} then \meta{yes} is called,
% otherwise \meta{no}. The character is not removed.
% Spaces are silently removed when looking for \meta{char}
% as \hologo{LaTeX}'s version \cs{kernel@ifnextchar} does.
% But there are also small differences:
% \begin{itemize}
% \item The space can be used as \meta{char}. In this case
%   optional spaces before \meta{char} are not supported of course.
% \item If the optional space is a command that is a character
%   (defined by \cs{let} or \cs{futurelet}), then
%   \cs{kernel@ifnextchar} breaks with an \hologo{TeX} error.
%   \cs{ltx@ifnextchar} silently removes this token as
%   optional space.
% \end{itemize}
% Since 2010/03/01 v1.3.
%
% \begin{declcs}{ltx@ifnextchar@nospace} \M{char} \M{yes} \M{no}
% \end{declcs}
% Macro \cs{ltx@ifnextchar@nospace} behaves like
% macro \cs{ltx@ifnextchar} with the exception that
% optional spaces are not supported before \meta{char}.
% Since 2011/04/14 v1.19.
%
% \subsection{\cs{ltx@leavevmode}, \cs{ltx@mbox}}
%
% \begin{declcs}{ltx@leavevmode}
% \end{declcs}
% Macro \cs{ltx@leavevmode} calls \hologo{pdfTeX}'s \cs{quitvmode}.
% Otherwise \cs{leavevmode} is used and defined if it is necessary.
%
% \begin{declcs}{ltx@mbox}
% \end{declcs}
% Macro \cs{ltx@mbox} reimplements \cs{mbox} with two changes.
% Instead of \cs{leavevmode} it uses \cs{ltx@leavevmode} and
% stops right after \cs{hbox}. Especially it does not grab
% the argument and allows the extended syntax of \cs{hbox}.
%
% \subsection{Expandable test for emptiness}
%
% \begin{declcs}{ltx@ifempty} \M{stuff} \M{yes} \M{no}
% \end{declcs}
% Macro \cs{ltx@ifempty} checks in exact two expansion steps
% whether \meta{stuff}
% is empty or contains token. Depending on the result
% \meta{yes} or \meta{no} is executed. The token in \meta{stuff}
% may contain \cs{par} and unmatched conditionals (\cs{if}, \cs{else},
% \cs{fi}, \dots). Since version 2010/11/12 v1.11.
%
% \begin{declcs}{ltx@ifblank} \M{stuff} \M{yes} \M{no}
% \end{declcs}
% Macro \cs{ltx@ifblank} tests in exact two expansion steps
% if \meta{stuff} is empty or contain only blank spaces.
% In this case argument \meta{yes} is called. If \meta{stuff}
% contains other tokens than spaces then \meta{no} is executed.
% Since version 2010/12/04 v1.13.
%
% \subsection{Stripping spaces}
%
% \begin{declcs}{ltx@zapspace} \M{stuff}
% \end{declcs}
% Macro \cs{ltx@zapspace} strips spaces from \meta{stuff}
% that are not hidden inside curly braces. Like \hologo{LaTeX}'s
% \cs{zap@space} it is expandable. Differences:
% \begin{itemize}
% \item
% Syntax: \cs{zap@space} also expects a space token
% and \cs{@empty} after \meta{stuff}.
% \item
% Macro \cs{ltx@zapspace} is expandable in exact two
% expansion steps.
% \item
% Macro \cs{ltx@zapspace} always retains curly braces.
% \item
% Macro \cs{zap@space} has a bug. It stops stripping spaces
% after a token group in curly braces if the first two tokens
% inside the group are equal.
% \item
% Macro \cs{ltx@zapspace} also works with \cs{par}
% and conditionals (\cs{if}, \cs{else}, \cs{fi}, \dots).
% \end{itemize}
% Macro \cs{ltx@zapspace} is available since version
% 2010/12/07 v1.14.
%
% \subsection{Check for emptiness of boxes}
%
% \begin{declcs}{ltx@IfBoxEmpty} \M{box register number}
%   \M{yes} \M{no}
% \end{declcs}
% Macro \cs{ltx@IfBoxEmpty} calls \meta{yes} if
% the box exists (\cs{ifvoid} returns false) and the box
% does not contain any content. Otherwise if the box
% is void or contains something, then \meta{no} is executed.
% Thus being empty means that the box exists and is either
% an \cs{hbox} or a \cs{vbox} and may even have dimensions other than
% 0.0\,pt, but the box does not contain anything.
% Macro \cs{ltx@IfBoxEmpty} is available since 2010/02/04 v1.16.
%
% \begin{declcs}{ltx@IfBoxVoidOrEmpty} \M{box register number}
%   \M{yes} \M{no}
% \end{declcs}
% Macro \cs{ltx@IfBoxVoidOrEmpty} calls \meta{yes} if
% the box is either void or does not contain any content.
% Otherwise \meta{no} is executed.
% Macro \cs{ltx@IfBoxVoidOrEmpty} is available since 2010/02/04 v1.16.
%
% \StopEventually{
% }
%
% \section{Implementation}
%
% \subsection{Identification}
%
%    \begin{macrocode}
%<*package>
%    \end{macrocode}
%    Reload check, especially if the package is not used with \LaTeX.
%    \begin{macrocode}
\begingroup\catcode61\catcode48\catcode32=10\relax%
  \catcode13=5 % ^^M
  \endlinechar=13 %
  \catcode35=6 % #
  \catcode39=12 % '
  \catcode44=12 % ,
  \catcode45=12 % -
  \catcode46=12 % .
  \catcode58=12 % :
  \catcode64=11 % @
  \catcode123=1 % {
  \catcode125=2 % }
  \expandafter\let\expandafter\x\csname ver@ltxcmds.sty\endcsname
  \ifx\x\relax % plain-TeX, first loading
  \else
    \def\empty{}%
    \ifx\x\empty % LaTeX, first loading,
      % variable is initialized, but \ProvidesPackage not yet seen
    \else
      \expandafter\ifx\csname PackageInfo\endcsname\relax
        \def\x#1#2{%
          \immediate\write-1{Package #1 Info: #2.}%
        }%
      \else
        \def\x#1#2{\PackageInfo{#1}{#2, stopped}}%
      \fi
      \x{ltxcmds}{The package is already loaded}%
      \aftergroup\endinput
    \fi
  \fi
\endgroup%
%    \end{macrocode}
%    Package identification:
%    \begin{macrocode}
\begingroup\catcode61\catcode48\catcode32=10\relax%
  \catcode13=5 % ^^M
  \endlinechar=13 %
  \catcode35=6 % #
  \catcode39=12 % '
  \catcode40=12 % (
  \catcode41=12 % )
  \catcode44=12 % ,
  \catcode45=12 % -
  \catcode46=12 % .
  \catcode47=12 % /
  \catcode58=12 % :
  \catcode64=11 % @
  \catcode91=12 % [
  \catcode93=12 % ]
  \catcode123=1 % {
  \catcode125=2 % }
  \expandafter\ifx\csname ProvidesPackage\endcsname\relax
    \def\x#1#2#3[#4]{\endgroup
      \immediate\write-1{Package: #3 #4}%
      \xdef#1{#4}%
    }%
  \else
    \def\x#1#2[#3]{\endgroup
      #2[{#3}]%
      \ifx#1\@undefined
        \xdef#1{#3}%
      \fi
      \ifx#1\relax
        \xdef#1{#3}%
      \fi
    }%
  \fi
\expandafter\x\csname ver@ltxcmds.sty\endcsname
\ProvidesPackage{ltxcmds}%
  [2019/12/15 v1.24 LaTeX kernel commands for general use (HO)]%
%    \end{macrocode}
%
%    \begin{macrocode}
\begingroup\catcode61\catcode48\catcode32=10\relax%
  \catcode13=5 % ^^M
  \endlinechar=13 %
  \catcode123=1 % {
  \catcode125=2 % }
  \catcode64=11 % @
  \def\x{\endgroup
    \expandafter\edef\csname LTXcmds@AtEnd\endcsname{%
      \endlinechar=\the\endlinechar\relax
      \catcode13=\the\catcode13\relax
      \catcode32=\the\catcode32\relax
      \catcode35=\the\catcode35\relax
      \catcode61=\the\catcode61\relax
      \catcode64=\the\catcode64\relax
      \catcode123=\the\catcode123\relax
      \catcode125=\the\catcode125\relax
    }%
  }%
\x\catcode61\catcode48\catcode32=10\relax%
\catcode13=5 % ^^M
\endlinechar=13 %
\catcode35=6 % #
\catcode64=11 % @
\catcode123=1 % {
\catcode125=2 % }
\def\TMP@EnsureCode#1#2{%
  \edef\LTXcmds@AtEnd{%
    \LTXcmds@AtEnd
    \catcode#1=\the\catcode#1\relax
  }%
  \catcode#1=#2\relax
}
\TMP@EnsureCode{36}{3}% $
\TMP@EnsureCode{38}{4}% &
\TMP@EnsureCode{40}{12}% (
\TMP@EnsureCode{41}{12}% )
\TMP@EnsureCode{45}{12}% -
\TMP@EnsureCode{46}{12}% .
\TMP@EnsureCode{47}{12}% /
\TMP@EnsureCode{60}{12}% <
\TMP@EnsureCode{62}{12}% >
\TMP@EnsureCode{91}{12}% [
\TMP@EnsureCode{96}{12}% `
\TMP@EnsureCode{93}{12}% ]
\TMP@EnsureCode{94}{12}% ^ (superscript) (!)
\TMP@EnsureCode{124}{12}% |
\edef\LTXcmds@AtEnd{\LTXcmds@AtEnd\noexpand\endinput}
%    \end{macrocode}
%
% \subsection{Numbers}
%
%    \begin{macro}{\ltx@zero}
%    \begin{macrocode}
\chardef\ltx@zero=0 %
%    \end{macrocode}
%    \end{macro}
%    \begin{macro}{\ltx@one}
%    \begin{macrocode}
\chardef\ltx@one=1 %
%    \end{macrocode}
%    \end{macro}
%    \begin{macro}{\ltx@two}
%    \begin{macrocode}
\chardef\ltx@two=2 %
%    \end{macrocode}
%    \end{macro}
%    \begin{macro}{\ltx@active}
%    \begin{macrocode}
\chardef\ltx@active=13 %
%    \end{macrocode}
%    \end{macro}
%    \begin{macro}{\ltx@cclv}
%    \begin{macrocode}
\chardef\ltx@cclv=255 %
%    \end{macrocode}
%    \end{macro}
%
%    \begin{macro}{\ltx@minusone}
%    \begin{macrocode}
\def\ltx@minusone{%
  -\ltx@one
}
%    \end{macrocode}
%    \end{macro}
%
% \subsection{Scratch registers}
%
%    \begin{macro}{\ltx@LocToksA}
%    \begin{macrocode}
\toksdef\ltx@LocToksA=0 %
%    \end{macrocode}
%    \end{macro}
%    \begin{macro}{\ltx@LocToksB}
%    \begin{macrocode}
\toksdef\ltx@LocToksB=2 %
%    \end{macrocode}
%    \end{macro}
%    \begin{macro}{\ltx@LocToksC}
%    \begin{macrocode}
\toksdef\ltx@LocToksC=4 %
%    \end{macrocode}
%    \end{macro}
%    \begin{macro}{\ltx@LocToksD}
%    \begin{macrocode}
\toksdef\ltx@LocToksD=6 %
%    \end{macrocode}
%    \end{macro}
%    \begin{macro}{\ltx@LocToksE}
%    \begin{macrocode}
\toksdef\ltx@LocToksE=8 %
%    \end{macrocode}
%    \end{macro}
%
%    \begin{macro}{\ltx@GlobToksA}
%    \begin{macrocode}
\toksdef\ltx@GlobToksA=1 %
%    \end{macrocode}
%    \end{macro}
%    \begin{macro}{\ltx@GlobToksB}
%    \begin{macrocode}
\toksdef\ltx@GlobToksB=3 %
%    \end{macrocode}
%    \end{macro}
%    \begin{macro}{\ltx@GlobToksC}
%    \begin{macrocode}
\toksdef\ltx@GlobToksC=5 %
%    \end{macrocode}
%    \end{macro}
%    \begin{macro}{\ltx@GlobToksD}
%    \begin{macrocode}
\toksdef\ltx@GlobToksD=7 %
%    \end{macrocode}
%    \end{macro}
%    \begin{macro}{\ltx@GlobToksE}
%    \begin{macrocode}
\toksdef\ltx@GlobToksE=9 %
%    \end{macrocode}
%    \end{macro}
%
%    \begin{macro}{\ltx@LocDimenA}
%    \begin{macrocode}
\dimendef\ltx@LocDimenA=0 %
%    \end{macrocode}
%    \end{macro}
%    \begin{macro}{\ltx@LocDimenB}
%    \begin{macrocode}
\dimendef\ltx@LocDimenB=2 %
%    \end{macrocode}
%    \end{macro}
%    \begin{macro}{\ltx@LocDimenC}
%    \begin{macrocode}
\dimendef\ltx@LocDimenC=4 %
%    \end{macrocode}
%    \end{macro}
%    \begin{macro}{\ltx@LocDimenD}
%    \begin{macrocode}
\dimendef\ltx@LocDimenD=6 %
%    \end{macrocode}
%    \end{macro}
%    \begin{macro}{\ltx@LocDimenE}
%    \begin{macrocode}
\dimendef\ltx@LocDimenE=8 %
%    \end{macrocode}
%    \end{macro}
%
%    \begin{macro}{\ltx@GlobDimenA}
%    \begin{macrocode}
\dimendef\ltx@GlobDimenA=1 %
%    \end{macrocode}
%    \end{macro}
%    \begin{macro}{\ltx@GlobDimenB}
%    \begin{macrocode}
\dimendef\ltx@GlobDimenB=3 %
%    \end{macrocode}
%    \end{macro}
%    \begin{macro}{\ltx@GlobDimenC}
%    \begin{macrocode}
\dimendef\ltx@GlobDimenC=5 %
%    \end{macrocode}
%    \end{macro}
%    \begin{macro}{\ltx@GlobDimenD}
%    \begin{macrocode}
\dimendef\ltx@GlobDimenD=7 %
%    \end{macrocode}
%    \end{macro}
%    \begin{macro}{\ltx@GlobDimenE}
%    \begin{macrocode}
\dimendef\ltx@GlobDimenE=9 %
%    \end{macrocode}
%    \end{macro}
%
%    \begin{macro}{\ltx@LocSkipA}
%    \begin{macrocode}
\skipdef\ltx@LocSkipA=0 %
%    \end{macrocode}
%    \end{macro}
%    \begin{macro}{\ltx@LocSkipB}
%    \begin{macrocode}
\skipdef\ltx@LocSkipB=2 %
%    \end{macrocode}
%    \end{macro}
%    \begin{macro}{\ltx@LocSkipC}
%    \begin{macrocode}
\skipdef\ltx@LocSkipC=4 %
%    \end{macrocode}
%    \end{macro}
%    \begin{macro}{\ltx@LocSkipD}
%    \begin{macrocode}
\skipdef\ltx@LocSkipD=6 %
%    \end{macrocode}
%    \end{macro}
%    \begin{macro}{\ltx@LocSkipE}
%    \begin{macrocode}
\skipdef\ltx@LocSkipE=8 %
%    \end{macrocode}
%    \end{macro}
%
%    \begin{macro}{\ltx@GlobSkipA}
%    \begin{macrocode}
\skipdef\ltx@GlobSkipA=1 %
%    \end{macrocode}
%    \end{macro}
%    \begin{macro}{\ltx@GlobSkipB}
%    \begin{macrocode}
\skipdef\ltx@GlobSkipB=3 %
%    \end{macrocode}
%    \end{macro}
%    \begin{macro}{\ltx@GlobSkipC}
%    \begin{macrocode}
\skipdef\ltx@GlobSkipC=5 %
%    \end{macrocode}
%    \end{macro}
%    \begin{macro}{\ltx@GlobSkipD}
%    \begin{macrocode}
\skipdef\ltx@GlobSkipD=7 %
%    \end{macrocode}
%    \end{macro}
%    \begin{macro}{\ltx@GlobSkipE}
%    \begin{macrocode}
\skipdef\ltx@GlobSkipE=9 %
%    \end{macrocode}
%    \end{macro}
%
%
% \subsection{Argument killers}
%
%    \begin{macro}{\ltx@gobble}
%    \begin{macrocode}
\long\def\ltx@gobble#1{}
%    \end{macrocode}
%    \end{macro}
%    \begin{macro}{\ltx@gobbletwo}
%    \begin{macrocode}
\long\def\ltx@gobbletwo#1#2{}
%    \end{macrocode}
%    \end{macro}
%    \begin{macro}{\ltx@gobblethree}
%    \begin{macrocode}
\long\def\ltx@gobblethree#1#2#3{}
%    \end{macrocode}
%    \end{macro}
%    \begin{macro}{\ltx@gobblefour}
%    \begin{macrocode}
\long\def\ltx@gobblefour#1#2#3#4{}
%    \end{macrocode}
%    \end{macro}
%
%    \begin{macro}{\ltx@GobbleNum}
%    \begin{macrocode}
\def\ltx@GobbleNum#1{%
  \romannumeral
  \csname ltx@zero%
  \expandafter\LTXcmds@GobbleNum
  \romannumeral\LTXcmds@num{#1}000{m\endcsname}%
}
%    \end{macrocode}
%    \end{macro}
%    \begin{macro}{\LTXcmds@GobbleNum}
%    \begin{macrocode}
\def\LTXcmds@GobbleNum#1{%
  \csname LTXcmds@G#1\LTXcmds@GobbleNum
}
%    \end{macrocode}
%    \end{macro}
%    \begin{macro}{\LTXcmds@Gm}
%    \begin{macrocode}
\long\def\LTXcmds@Gm#1{%
  \endcsname
}
%    \end{macrocode}
%    \end{macro}
%
% \subsection{Argument grabbers}
%
%    \begin{macro}{\ltx@firstofone}
%    \begin{macrocode}
\long\def\ltx@firstofone#1{#1}
%    \end{macrocode}
%    \end{macro}
%    \begin{macro}{\ltx@firstoftwo}
%    \begin{macrocode}
\long\def\ltx@firstoftwo#1#2{#1}
%    \end{macrocode}
%    \end{macro}
%    \begin{macro}{\ltx@secondoftwo}
%    \begin{macrocode}
\long\def\ltx@secondoftwo#1#2{#2}
%    \end{macrocode}
%    \end{macro}
%
%    \begin{macro}{\ltx@firstofthree}
%    \begin{macrocode}
\long\def\ltx@firstofthree#1#2#3{#1}
%    \end{macrocode}
%    \end{macro}
%    \begin{macro}{\ltx@secondofthree}
%    \begin{macrocode}
\long\def\ltx@secondofthree#1#2#3{#2}
%    \end{macrocode}
%    \end{macro}
%    \begin{macro}{\ltx@thirdofthree}
%    \begin{macrocode}
\long\def\ltx@thirdofthree#1#2#3{#3}%
%    \end{macrocode}
%    \end{macro}
%
%    \begin{macro}{\ltx@firstoffour}
%    \begin{macrocode}
\long\def\ltx@firstoffour#1#2#3#4{#1}
%    \end{macrocode}
%    \end{macro}
%    \begin{macro}{\ltx@secondoffour}
%    \begin{macrocode}
\long\def\ltx@secondoffour#1#2#3#4{#2}
%    \end{macrocode}
%    \end{macro}
%    \begin{macro}{\ltx@thirdoffour}
%    \begin{macrocode}
\long\def\ltx@thirdoffour#1#2#3#4{#3}%
%    \end{macrocode}
%    \end{macro}
%    \begin{macro}{\ltx@fourthoffour}
%    \begin{macrocode}
\long\def\ltx@fourthoffour#1#2#3#4{#4}%
%    \end{macrocode}
%    \end{macro}
%
% \subsection{List helpers}
%
%    \begin{macro}{\ltx@carzero}
%    \begin{macrocode}
\long\def\ltx@carzero#1\@nil{}%
%    \end{macrocode}
%    \end{macro}
%    \begin{macro}{\LTXcmds@cdrzero}
%    \begin{macrocode}
\long\def\LTXcmds@cdrzero#1\@nil{#1}
%    \end{macrocode}
%    \end{macro}
%    \begin{macro}{\ltx@cdrzero}
%    \begin{macrocode}
\def\ltx@cdrzero{%
  \romannumeral\LTXcmds@cdrzero\ltx@zero
}
%    \end{macrocode}
%    \end{macro}
%
%    \begin{macro}{\ltx@car}
%    \begin{macrocode}
\long\def\ltx@car#1#2\@nil{#1}
%    \end{macrocode}
%    \end{macro}
%    \begin{macro}{\ltx@cdr}
%    \begin{macrocode}
\long\def\ltx@cdr#1{%
  \romannumeral\LTXcmds@cdrzero\ltx@zero
}
%    \end{macrocode}
%    \end{macro}
%
%    \begin{macro}{\ltx@cartwo}
%    \begin{macrocode}
\long\def\ltx@cartwo#1#2#3\@nil{#1#2}
%    \end{macrocode}
%    \end{macro}
%    \begin{macro}{\ltx@carsecond}
%    \begin{macrocode}
\long\def\ltx@carsecond#1#2#3\@nil{#2}
%    \end{macrocode}
%    \end{macro}
%    \begin{macro}{\ltx@cdrtwo}
%    \begin{macrocode}
\long\def\ltx@cdrtwo#1#2{%
  \romannumeral\LTXcmds@cdrzero\ltx@zero
}
%    \end{macrocode}
%    \end{macro}
%
%    \begin{macro}{\ltx@carthree}
%    \begin{macrocode}
\long\def\ltx@carthree#1#2#3#4\@nil{#1#2#3}
%    \end{macrocode}
%    \end{macro}
%    \begin{macro}{\ltx@carthird}
%    \begin{macrocode}
\long\def\ltx@carthird#1#2#3#4\@nil{#3}
%    \end{macrocode}
%    \end{macro}
%    \begin{macro}{\ltx@cdrthree}
%    \begin{macrocode}
\long\def\ltx@cdrthree#1#2#3{%
  \romannumeral\LTXcmds@cdrzero\ltx@zero
}
%    \end{macrocode}
%    \end{macro}
%
%    \begin{macro}{\ltx@carfour}
%    \begin{macrocode}
\long\def\ltx@carfour#1#2#3#4#5\@nil{#1#2#3#4}
%    \end{macrocode}
%    \end{macro}
%    \begin{macro}{\ltx@carfourth}
%    \begin{macrocode}
\long\def\ltx@carfourth#1#2#3#4#5\@nil{#4}
%    \end{macrocode}
%    \end{macro}
%    \begin{macro}{\ltx@cdrfour}
%    \begin{macrocode}
\long\def\ltx@cdrfour#1#2#3#4{%
  \romannumeral\LTXcmds@cdrzero\ltx@zero
}
%    \end{macrocode}
%    \end{macro}
%
%    \begin{macro}{\ltx@CarNum}
%    \begin{macrocode}
\def\ltx@CarNum#1{%
  \romannumeral
  \csname LTXcmds@CarNumFinish%
  \expandafter\LTXcmds@CarNum
  \romannumeral\LTXcmds@num{#1}000{x\endcsname}%
}
%    \end{macrocode}
%    \end{macro}
%    \begin{macro}{\LTXcmds@CarNum}
%    \begin{macrocode}
\def\LTXcmds@CarNum#1{%
  \csname LTXcmds@C#1\LTXcmds@CarNum
}
%    \end{macrocode}
%    \end{macro}
%    \begin{macro}{\LTXcmds@Cm}
%    \begin{macrocode}
\long\def\LTXcmds@Cm#1#2{%
  \endcsname{#1#2}%
}
%    \end{macrocode}
%    \end{macro}
%    \begin{macro}{\LTXcmds@Cx}
%    \begin{macrocode}
\def\LTXcmds@Cx#1{%
  \endcsname{}%
}
%    \end{macrocode}
%    \end{macro}
%    \begin{macro}{\LTXcmds@CarNumFinish}
%    \begin{macrocode}
\long\def\LTXcmds@CarNumFinish#1#2\@nil{%
  \ltx@zero
  #1%
}
%    \end{macrocode}
%    \end{macro}
%
%    \begin{macro}{\ltx@CarNumth}
%    \begin{macrocode}
\def\ltx@CarNumth#1{%
  \romannumeral
  \expandafter\expandafter\expandafter
  \LTXcmds@CarNumth
  \ltx@GobbleNum{#1}{}%
}
%    \end{macrocode}
%    \end{macro}
%    \begin{macro}{\LTXcmds@CarNumth}
%    \begin{macrocode}
\long\def\LTXcmds@CarNumth#1#2\@nil{%
  \ltx@zero
  #1%
}
%    \end{macrocode}
%    \end{macro}
%
%    \begin{macro}{\ltx@CdrNum}
%    \begin{macrocode}
\def\ltx@CdrNum#1{%
  \romannumeral%
  \expandafter\expandafter\expandafter\ltx@cdrzero
  \expandafter\expandafter\expandafter\ltx@zero
  \ltx@GobbleNum{#1}%
}
%    \end{macrocode}
%    \end{macro}
%
% \subsection{Tail recursion}
%
%    \begin{macro}{\ltx@ReturnAfterFi}
%    \begin{macrocode}
\long\def\ltx@ReturnAfterFi#1\fi{\fi#1}
%    \end{macrocode}
%    \end{macro}
%    \begin{macro}{\ltx@ReturnAfterElseFi}
%    \begin{macrocode}
\long\def\ltx@ReturnAfterElseFi#1\else#2\fi{\fi#1}
%    \end{macrocode}
%    \end{macro}
%
% \subsection{Empty macro}
%
%    \begin{macro}{\ltx@empty}
%    \begin{macrocode}
\def\ltx@empty{}
%    \end{macrocode}
%    \end{macro}
%
% \subsection{Characters}
%
%    \begin{macro}{\ltx@space}
%    \begin{macrocode}
\def\ltx@space{ }
%    \end{macrocode}
%    \end{macro}
%    \begin{macro}{\ltx@percentchar}
%    \begin{macrocode}
\begingroup
  \lccode`0=`\%\relax
\lowercase{\endgroup
  \def\ltx@percentchar{0}%
}
%    \end{macrocode}
%    \end{macro}
%    \begin{macro}{\ltx@backslashchar}
%    \begin{macrocode}
\begingroup
  \lccode`0=`\\\relax
\lowercase{\endgroup
  \def\ltx@backslashchar{0}%
}
%    \end{macrocode}
%    \end{macro}
%    \begin{macro}{\ltx@hashchar}
%    \begin{macrocode}
\begingroup
  \lccode`0=`\#\relax
\lowercase{\endgroup
  \def\ltx@hashchar{0}%
}
%    \end{macrocode}
%    \end{macro}
%    \begin{macro}{\ltx@leftbracechar}
%    \begin{macrocode}
\begingroup
  \lccode`0=`\{\relax
\lowercase{\endgroup
  \def\ltx@leftbracechar{0}%
}
%    \end{macrocode}
%    \end{macro}
%    \begin{macro}{\ltx@rightbracechar}
%    \begin{macrocode}
\begingroup
  \lccode`0=`\}\relax
\lowercase{\endgroup
  \def\ltx@rightbracechar{0}%
}
%    \end{macrocode}
%    \end{macro}
%
% \subsection{Boolean switch}
%
%    \begin{macro}{\ltx@newif}
%    \begin{macrocode}
\def\ltx@newif#1{%
  \begingroup
    \escapechar=-1 %
  \expandafter\endgroup
  \expandafter\LTXcmds@newif\string#1\@nil
}
%    \end{macrocode}
%    \end{macro}
%    \begin{macro}{\LTXcmds@newif}
%    \begin{macrocode}
\begingroup
  \escapechar=-1 %
\expandafter\endgroup
\expandafter\def\expandafter\LTXcmds@newif\string\if#1\@nil{%
  \expandafter\edef\csname#1true\endcsname{%
    \let
    \expandafter\noexpand\csname if#1\endcsname
    \noexpand\iftrue
  }%
  \expandafter\edef\csname#1false\endcsname{%
    \let
    \expandafter\noexpand\csname if#1\endcsname
    \noexpand\iffalse
  }%
  \csname#1false\endcsname
}
%    \end{macrocode}
%    \end{macro}
%
%    \begin{macro}{\ltx@newglobalif}
%    \begin{macrocode}
\def\ltx@newglobalif#1{%
  \begingroup
    \escapechar=-1 %
  \expandafter\endgroup
  \expandafter\LTXcmds@newglobalif\string#1\@nil
}
%    \end{macrocode}
%    \end{macro}
%    \begin{macro}{\LTXcmds@newglobalif}
%    \begin{macrocode}
\begingroup
  \escapechar=-1 %
\expandafter\endgroup
\expandafter
\def\expandafter\LTXcmds@newglobalif\string\if#1\@nil{%
  \expandafter\edef\csname#1true\endcsname{%
    \global\let
    \expandafter\noexpand\csname if#1\endcsname
    \noexpand\iftrue
  }%
  \expandafter\edef\csname#1false\endcsname{%
    \global\let
    \expandafter\noexpand\csname if#1\endcsname
    \noexpand\iffalse
  }%
  \csname#1false\endcsname
}
%    \end{macrocode}
%    \end{macro}
%
% \subsection{Command definitions}
%
%    \begin{macro}{\ltx@LocalExpandAfter}
%    \begin{macrocode}
\def\ltx@LocalExpandAfter{%
  \begingroup
    \expandafter\expandafter\expandafter
  \endgroup
  \expandafter
}
%    \end{macrocode}
%    \end{macro}
%
%    \begin{macrocode}
\ltx@LocalExpandAfter
\ifx\csname ifcsname\endcsname\relax
%    \end{macrocode}
%    \begin{macro}{\ltx@ifundefined}
%    \begin{macrocode}
  \def\ltx@ifundefined#1{%
    \expandafter\ifx\csname #1\endcsname\relax
      \expandafter\ltx@firstoftwo
    \else
      \expandafter\ltx@secondoftwo
    \fi
  }%
%    \end{macrocode}
%    \end{macro}
%    \begin{macro}{\ltx@IfUndefined}
%    \begin{macrocode}
  \def\ltx@IfUndefined#1{%
    \begingroup\expandafter\expandafter\expandafter\endgroup
    \expandafter\ifx\csname #1\endcsname\relax
      \expandafter\ltx@firstoftwo
    \else
      \expandafter\ltx@secondoftwo
    \fi
  }%
%    \end{macrocode}
%    \end{macro}
%    \begin{macrocode}
  \expandafter\ltx@gobble
\else
  \expandafter\ltx@firstofone
\fi
{%
%    \end{macrocode}
%    \begin{macro}{\ltx@ifundefined}
%    \begin{macrocode}
  \def\ltx@ifundefined#1{%
    \ifcsname #1\endcsname
      \expandafter\ifx\csname #1\endcsname\relax
        \expandafter\expandafter\expandafter\ltx@firstoftwo
      \else
        \expandafter\expandafter\expandafter\ltx@secondoftwo
      \fi
    \else
      \expandafter\ltx@firstoftwo
    \fi
  }%
%    \end{macrocode}
%    \end{macro}
%    \begin{macro}{\ltx@IfUndefined}
%    \begin{macrocode}
  \let\ltx@IfUndefined\ltx@ifundefined
%    \end{macrocode}
%    \end{macro}
%    \begin{macrocode}
}
%    \end{macrocode}
%
% \subsection{Stripping}
%
%    \begin{macro}{\ltx@RemovePrefix}
%    \begin{macrocode}
\def\ltx@RemovePrefix#1>{}
%    \end{macrocode}
%    \end{macro}
%
%    \begin{macro}{\ltx@StripPrefix}
%    \begin{macrocode}
\def\ltx@StripPrefix{%
  \expandafter\ltx@RemovePrefix
}
%    \end{macrocode}
%    \end{macro}
%
%    \begin{macro}{\ltx@onelevel@sanitize}
%    \begin{macrocode}
\def\ltx@onelevel@sanitize#1{%
  \edef#1{%
    \expandafter
    \ltx@RemovePrefix\meaning#1%
  }%
}
%    \end{macrocode}
%    \end{macro}
%
% \subsection{File management}
%
% \subsubsection{File extensions}
%
%    \begin{macro}{\ltx@clsextension}
%    \begin{macrocode}
\def\ltx@clsextension{cls}
%    \end{macrocode}
%    \end{macro}
%    \begin{macro}{\ltx@pkgextension}
%    \begin{macrocode}
\def\ltx@pkgextension{sty}
%    \end{macrocode}
%    \end{macro}
%
% \subsubsection{Load check}
%
%    \begin{macro}{\ltx@iffileloaded}
%    \begin{macrocode}
\def\ltx@iffileloaded#1{%
  \ltx@ifundefined{ver@#1}\ltx@secondoftwo\ltx@firstoftwo
}
%    \end{macrocode}
%    \end{macro}
%    \begin{macro}{\ltx@ifclassloaded}
%    \begin{macrocode}
\def\ltx@ifclassloaded#1{%
  \ltx@iffileloaded{#1.\ltx@clsextension}%
}
%    \end{macrocode}
%    \end{macro}
%    \begin{macro}{\ltx@ifpackageloaded}
%    \begin{macrocode}
\def\ltx@ifpackageloaded#1{%
  \ltx@iffileloaded{#1.\ltx@pkgextension}%
}
%    \end{macrocode}
%    \end{macro}
%
% \subsubsection{Version date check}
%
%    \begin{macro}{\ltx@iffilelater}
%    \begin{macrocode}
\def\ltx@iffilelater#1#2{%
  \ltx@iffileloaded{#1}{%
    \expandafter\LTXcmds@IfLater\expandafter{%
      \number
      \expandafter\expandafter\expandafter\LTXcmds@ParseVersion
      \expandafter\expandafter\expandafter{%
        \csname ver@#1\endcsname
      }%
    \expandafter}\expandafter{%
      \number
      \expandafter\LTXcmds@ParseVersion\expandafter{#2}%
    }%
  }\ltx@secondoftwo
}
%    \end{macrocode}
%    \end{macro}
%    \begin{macro}{\LTXcmds@IfLater}
%    \begin{macrocode}
\def\LTXcmds@IfLater#1#2{%
  \ifcase 0%
      \ifnum#1<19940101 %
      \else
        \ifnum#2<19940101 %
        \else
          \ifnum#2>#1 %
          \else
            1%
          \fi
        \fi
      \fi
      \ltx@space
    \expandafter\ltx@secondoftwo
  \else
    \expandafter\ltx@firstoftwo
  \fi
}
%    \end{macrocode}
%    \end{macro}
%    \begin{macro}{\ltx@ifclasslater}
%    \begin{macrocode}
\def\ltx@ifclasslater#1{%
  \ltx@iffilelater{#1.\ltx@clsextension}%
}
%    \end{macrocode}
%    \end{macro}
%    \begin{macro}{\ltx@ifpackagelater}
%    \begin{macrocode}
\def\ltx@ifpackagelater#1{%
  \ltx@iffilelater{#1.\ltx@pkgextension}%
}
%    \end{macrocode}
%    \end{macro}
%
%    \begin{macrocode}
\ltx@IfUndefined{pdfmatch}{%
%    \end{macrocode}
%    \begin{macro}{\LTXcmds@ParseVersion}
%    \begin{macrocode}
  \def\LTXcmds@ParseVersion#1{%
    \LTXcmds@@ParseVersion#10000/00/00\@nil
  }%
%    \end{macrocode}
%    \end{macro}
%    \begin{macro}{\LTXcmds@@ParseVersion}
%    \begin{macrocode}
  \def\LTXcmds@@ParseVersion#1#2#3#4/#5#6/#7#8#9\@nil{%
    #1#2#3#4#5#6#7#8%
  }%
%    \end{macrocode}
%    \end{macro}
%    \begin{macrocode}
}{%
%    \end{macrocode}
%    \begin{macro}{\LTXcmds@ParseVersion}
%    \begin{macrocode}
  \def\LTXcmds@ParseVersion#1{%
    \ifnum\pdfmatch{%
      ^%
      (199[4-9]|[2-9][0-9][0-9][0-9])/%
      (0[1-9]|1[0-2])/%
      (0[1-9]|[1-2][0-9]|3[0-1])%
    }{#1}=1 %
      \ltx@StripPrefix\pdflastmatch1 %
      \ltx@StripPrefix\pdflastmatch2 %
      \ltx@StripPrefix\pdflastmatch3 %
    \else
      0%
    \fi
  }%
%    \end{macrocode}
%    \end{macro}
%    \begin{macrocode}
}
%    \end{macrocode}
%
% \subsection{Macro additions}
%
%    \begin{macro}{\ltx@GlobalAppendToMacro}
%    \begin{macrocode}
\long\def\ltx@GlobalAppendToMacro#1#2{%
  \ifx\ltx@undefined#1%
    \let#1\ltx@empty
  \else
    \ifx\relax#1%
      \let#1\ltx@empty
    \fi
  \fi
  \begingroup
    \ltx@LocToksA\expandafter{#1#2}%
    \xdef#1{\the\ltx@LocToksA}%
  \endgroup
}
%    \end{macrocode}
%    \end{macro}
%    \begin{macro}{\ltx@LocalAppendToMacro}
%    \begin{macrocode}
\long\def\ltx@LocalAppendToMacro#1#2{%
  \global\let\LTXcmds@gtemp#1%
  \ifx\ltx@undefined\LTXcmds@gtemp
    \global\let\LTXcmds@gtemp\ltx@empty
  \else
    \ifx\relax\LTXcmds@gtemp
      \global\letLTXcmds@gtemp\ltx@empty
    \fi
  \fi
  \begingroup
    \ltx@LocToksA\expandafter{\LTXcmds@gtemp#2}%
    \xdef\LTXcmds@gtemp{\the\ltx@LocToksA}%
  \endgroup
  \let#1\LTXcmds@gtemp
}
%    \end{macrocode}
%    \end{macro}
%
%    \begin{macro}{\ltx@GlobalPrependToMacro}
%    \begin{macrocode}
\long\def\ltx@GlobalPrependToMacro#1#2{%
  \ifx\ltx@undefined#1%
    \let#1\ltx@empty
  \else
    \ifx\relax#1%
      \let#1\ltx@empty
    \fi
  \fi
  \begingroup
    \ltx@LocToksA{#2}%
    \ltx@LocToksB\expandafter{#1}%
    \xdef#1{\the\ltx@LocToksA\the\ltx@LocToksB}%
  \endgroup
}
%    \end{macrocode}
%    \end{macro}
%    \begin{macro}{\ltx@LocalPrependToMacro}
%    \begin{macrocode}
\long\def\ltx@LocalPrependToMacro#1#2{%
  \global\let\LTXcmds@gtemp#1%
  \ifx\ltx@undefined\LTXcmds@gtemp
    \global\let\LTXcmds@gtemp\ltx@empty
  \else
    \ifx\relax\LTXcmds@gtemp
      \global\letLTXcmds@gtemp\ltx@empty
    \fi
  \fi
  \begingroup
    \ltx@LocToksA{#2}%
    \ltx@LocToksB\expandafter{\LTXcmds@gtemp}%
    \xdef\LTXcmds@gtemp{\the\ltx@LocToksA\the\ltx@LocToksB}%
  \endgroup
  \let#1\LTXcmds@gtemp
}
%    \end{macrocode}
%    \end{macro}
%
% \subsection{Next character detection}
%
%    \begin{macro}{\ltx@ifnextchar}
%    \begin{macrocode}
\long\def\ltx@ifnextchar#1#2#3{%
  \begingroup
  \let\LTXcmds@CharToken= #1\relax
  \ltx@LocToksA{\endgroup#2}%
  \ltx@LocToksB{\endgroup#3}%
  \futurelet\LTXcmds@LetToken\LTXcmds@ifnextchar
}
%    \end{macrocode}
%    \end{macro}
%    \begin{macro}{\LTXcmds@ifnextchar}
%    \begin{macrocode}
\def\LTXcmds@ifnextchar{%
  \ifx\LTXcmds@LetToken\LTXcmds@CharToken
    \the\expandafter\ltx@LocToksA
  \else
    \expandafter
      \ifx\csname LTXcmds@LetToken\endcsname\LTXcmds@SpaceToken
      \expandafter\expandafter\expandafter\LTXcmds@@ifnextchar
    \else
      \the\expandafter\expandafter\expandafter\ltx@LocToksB
    \fi
  \fi
}
%    \end{macrocode}
%    \end{macro}
%    \begin{macro}{\LTXcmds@@ifnextchar}
%    \cs{futurelet} does not distinguish between a
%    character and a command that is a character
%    (defined by using \cs{let} or \cs{futurelet}).
%    Therefore the space is catched by \cs{romannumeral}
%    with negative character constant that gobbles one
%    optional space.
%    \begin{macrocode}
\def\LTXcmds@@ifnextchar{%
  \expandafter\futurelet
  \expandafter\LTXcmds@LetToken
  \expandafter\LTXcmds@ifnextchar
  \romannumeral-`\.%
}
%    \end{macrocode}
%    \end{macro}
%    \begin{macro}{\LTXcmds@SpaceToken}
%    \begin{macrocode}
\ltx@firstofone{\let\LTXcmds@SpaceToken= } %
%    \end{macrocode}
%    \end{macro}
%
%    \begin{macro}{\ltx@ifnextchar@nospace}
%    \begin{macrocode}
\long\def\ltx@ifnextchar@nospace#1#2#3{%
  \begingroup
  \let\LTXcmds@CharToken= #1\relax
  \ltx@LocToksA{\endgroup#2}%
  \ltx@LocToksB{\endgroup#3}%
  \futurelet\LTXcmds@LetToken\LTXcmds@ifnextchar@nospace
}
%    \end{macrocode}
%    \end{macro}
%    \begin{macro}{\LTXcmds@ifnextchar@nospace}
%    \begin{macrocode}
\def\LTXcmds@ifnextchar@nospace{%
  \the
  \ifx\LTXcmds@LetToken\LTXcmds@CharToken
    \expandafter\ltx@LocToksA
  \else
    \expandafter\ltx@LocToksB
  \fi
}
%    \end{macrocode}
%    \end{macro}
% \subsection{\cs{ltx@leavevmode}, \cs{ltx@mbox}}
%
%    \begin{macro}{\ltx@leavevmode}
%    \begin{macrocode}
\ltx@IfUndefined{quitvmode}{%
  \ltx@IfUndefined{leavevmode}{%
    \ltx@IfUndefined{voidb@x}{%
      \ltx@IfUndefined{newbox}{%
        \def\ltx@leavevmode{%
          \begingroup
            \setbox\ltx@zero=\hbox{}%
            \begingroup
              \setbox\ltx@zero=\hbox{\box\ltx@zero}%
            \endgroup
            \unhbox\ltx@zero
          \endgroup
        }%
      }{%
        \csname newbox\endcsname\LTXcmds@VoidBox
        \ifvoid\LTXcmds@VoidBox
        \else
          \setbox\LTXcmds@VoidBox=\hbox{}%
          \begingroup
            \setbox\LTXcmds@VoidBox=\hbox{\box\LTXcmds@VoidBox}%
          \endgroup
        \fi
        \def\ltx@leavevmode{\unhbox\LTXcmds@VoidBox}%
      }%
    }{%
      \def\ltx@leavevmode{\unhbox\voidb@x}%
    }%
  }{%
    \let\ltx@leavevmode\leavevmode
  }%
}{%
  \let\ltx@leavevmode\quitvmode
}
%    \end{macrocode}
%    \end{macro}
%
%    \begin{macro}{\ltx@mbox}
%    \begin{macrocode}
\def\ltx@mbox{%
  \ltx@leavevmode
  \hbox
}
%    \end{macrocode}
%    \end{macro}
%
% \subsection{Help macros}
%
%    \begin{macro}{\LTXcmds@num}
%    \begin{macrocode}
\ltx@IfUndefined{numexpr}{%
  \def\LTXcmds@num#1{%
    \expandafter\ltx@firstofone\expandafter{%
      \number#1%
    }%
  }%
}{%
  \def\LTXcmds@num#1{%
    \expandafter\ltx@firstofone\expandafter{%
      \the\numexpr#1%
    }%
  }%
}
%    \end{macrocode}
%    \end{macro}
%
% \subsection{Expandable test for emptiness}
%
%    \begin{macrocode}
\ltx@IfUndefined{detokenize}{%
%    \end{macrocode}
%
% \subsubsection{Vanilla \hologo{TeX}}
%
%    \begin{macro}{\ltx@ifempty}
%    The macro is based on \cs{@ifempty} of Robert R. Schneck
%    \cite{ifempty:schneck}
%    and \cs{@ifnull} of Ulrich Diez \cite{ifnull:diez}.
%    There are three cases to consider:
%    \begin{compactenum}
%    \item |#1| is empty,
%    \item |#1| is not empty and the first token
%      is not a begingroup character,
%    \item |#1| starts with a begingroup character (catcode 1).
%    \end{compactenum}
%    \begin{macrocode}
  \def\LTXcmds@temp#1{%
    \long\def\ltx@ifempty##1{%
      \romannumeral0%
      \iffalse{\fi
        \expandafter\ltx@gobble\expandafter{%
          \expandafter{\string##1}%
          \expandafter\ltx@gobble\string
        }%
        \expandafter\ltx@firstofthree\expandafter
        {\iffalse}\fi
        \expandafter#1\ltx@secondoftwo
      }%
      \expandafter#1\ltx@firstoftwo
    }%
%    \end{macrocode}
%    \end{macro}
%    \begin{macro}{\ltx@ifblank}
%    \begin{macrocode}
    \long\def\ltx@ifblank##1{%
      \romannumeral0%
      \iffalse{\fi
        \expandafter\expandafter\expandafter\ltx@gobble
        \expandafter\expandafter\expandafter{%
          \expandafter\expandafter\expandafter{%
            \expandafter\string\ltx@gobble##1.%
          }%
          \expandafter\ltx@gobble\string
        }%
        \expandafter\ltx@firstofthree\expandafter
        {\iffalse}\fi
        \expandafter#1\ltx@secondoftwo
      }%
      \expandafter#1\ltx@firstoftwo
    }%
  }%
  \LTXcmds@temp{ }%
%    \end{macrocode}
%    \end{macro}
%    \begin{macrocode}
}{%
%    \end{macrocode}
%
% \subsubsection{With \cs{detokenize}}
%
%    Ahmed Musa provided \cs{ifstrempty} using \cs{detokenize}
%    and \cs{pdfstrcmp} \cite{ifstrempty:musa}. Ulrich Diez, GL,
%    Heiko Oberdiek
%    improved it further by removing \cs{pdfstrcmp} and
%    taking three arguments
%    \cite{ifstrempty:diez1,ifstrempty:diez2,ifstrempty:diez3,^^A
%      ifstrempty:gl,ifstrempty:ho}.
%
%    \begin{macro}{\ltx@ifempty}
%    \begin{macrocode}
  \long\def\ltx@ifempty#1{%
    \romannumeral%
    \csname
      LTXcmds@ifempty%
      \ifcat$\detokenize{#1}$%
        @%
      \fi
    \endcsname
  }%
%    \end{macrocode}
%    \end{macro}
%    \begin{macro}{\LTXcmds@ifempty@}
%    \begin{macrocode}
  \long\def\LTXcmds@ifempty@#1#2{0 #1}%
%    \end{macrocode}
%    \end{macro}
%    \begin{macro}{\LTXcmds@ifempty}
%    \begin{macrocode}
  \long\def\LTXcmds@ifempty#1#2{0 #2}%
%    \end{macrocode}
%    \end{macro}
%
% \subsubsection{\cs{ltx@ifblank}}
%
%    \begin{macro}{\ltx@ifblank}
%    \begin{macrocode}
  \long\def\ltx@ifblank#1{%
    \romannumeral%
    \csname
      LTXcmds@ifempty%
      \ifcat$\detokenize\expandafter{\ltx@gobble#1.}$%
        @%
      \fi
    \endcsname
  }%
%    \end{macrocode}
%    \end{macro}
%
%    \begin{macrocode}
}
%    \end{macrocode}
%
% \subsection{\cs{ltx@zapspace}}
%
%    \begin{macro}{\ltx@zapspace}
%    \begin{macrocode}
\long\def\ltx@zapspace#1{%
  \romannumeral
  \LTXcmds@zapspace\ltx@zero#1 \@nil
}
%    \end{macrocode}
%    \end{macro}
%    \begin{macro}{\LTXcmds@zapspace}
%    \begin{macrocode}
\long\def\LTXcmds@zapspace#1 #2\@nil{%
  \ltx@ifempty{#2}{%
    #1%
  }{%
    \LTXcmds@zapspace#1#2\@nil
  }%
}
%    \end{macrocode}
%    \end{macro}
%
% \subsection{\cs{ltx@IfBoxEmpty}}
%
%    In case of \hologo{eTeX} the test for an empty box
%    is done via \cs{lastnodetype} as suggested
%    by David Kastrup \cite{ifboxempty:dak}.
%    \begin{macrocode}
\ltx@IfUndefined{lastnodetype}{%
  \catcode`\$=9 %
  \catcode`\&=14 %
}{%
  \catcode`\$=14 %
  \catcode`\&=9 %
}
%    \end{macrocode}
%    \begin{macro}{\ltx@IfBoxEmpty}
%    \begin{macrocode}
\def\ltx@IfBoxEmpty#1{%
  \ifvoid#1\relax
    \expandafter\ltx@secondoftwo
  \else
%    \end{macrocode}
%    Implementation using \hologo{eTeX}'s \cs{lastnodetype}.
%    \begin{macrocode}
&   \begingroup
&     \setbox\ltx@zero=\ifhbox#1\hbox\else\vbox\fi{%
&       \ifhmode\unhcopy\else\unvcopy\fi#1\relax
&       \expandafter
&     }%
&   \expandafter\endgroup
&   \ifnum\lastnodetype<\ltx@zero
&     \expandafter\expandafter\expandafter\ltx@firstoftwo
&   \else
&     \expandafter\expandafter\expandafter\ltx@secondoftwo
&   \fi
%    \end{macrocode}
%    Implementation without \hologo{eTeX} using a signature
%    at the beginning of the test box.
%    \begin{macrocode}
$   \begingroup
$     \setbox\ltx@zero=\ifhbox#1\hbox\else\vbox\fi{%
$       \penalty\ltx@one
$       \ifhmode\unhcopy\else\unvcopy\fi#1\relax
$       \expandafter
$     }%
$     \ifnum\lastpenalty=\ltx@one
%    \end{macrocode}
%    Box 0 has been changed and is restored by
%    closing the group.
%    \begin{macrocode}
$       \endgroup
$       \begingroup
$       \setbox\ltx@zero=\ifhbox#1\hbox\else\vbox\fi{%
$         \penalty\ltx@two
$         \ifhmode\unhcopy\else\unvcopy\fi#1\relax
$         \expandafter
$       }%
$       \ifnum\lastpenalty=\ltx@two
$         \def\next{\endgroup\expandafter\ltx@firstoftwo}%
$       \else
$         \def\next{\endgroup\expandafter\ltx@secondoftwo}%
$       \fi
$     \else
$       \def\next{\endgroup\expandafter\ltx@secondoftwo}%
$     \fi
$   \next
  \fi
}
%    \end{macrocode}
%    \end{macro}
%
%    \begin{macro}{\ltx@IfBoxVoidOrEmpty}
%    \begin{macrocode}
\def\ltx@IfBoxVoidOrEmpty#1{%
  \ifvoid#1\relax
    \expandafter\ltx@thirdoffour
  \fi
  \ltx@IfBoxEmpty{#1}%
}
%    \end{macrocode}
%    \end{macro}
%
%    \begin{macrocode}
\LTXcmds@AtEnd%
%</package>
%    \end{macrocode}
%% \section{Installation}
%
% \subsection{Download}
%
% \paragraph{Package.} This package is available on
% CTAN\footnote{\CTANpkg{ltxcmds}}:
% \begin{description}
% \item[\CTAN{macros/latex/contrib/ltxcmds/ltxcmds.dtx}] The source file.
% \item[\CTAN{macros/latex/contrib/ltxcmds/ltxcmds.pdf}] Documentation.
% \end{description}
%
%
% \paragraph{Bundle.} All the packages of the bundle `ltxcmds'
% are also available in a TDS compliant ZIP archive. There
% the packages are already unpacked and the documentation files
% are generated. The files and directories obey the TDS standard.
% \begin{description}
% \item[\CTANinstall{install/macros/latex/contrib/ltxcmds.tds.zip}]
% \end{description}
% \emph{TDS} refers to the standard ``A Directory Structure
% for \TeX\ Files'' (\CTANpkg{tds}). Directories
% with \xfile{texmf} in their name are usually organized this way.
%
% \subsection{Bundle installation}
%
% \paragraph{Unpacking.} Unpack the \xfile{ltxcmds.tds.zip} in the
% TDS tree (also known as \xfile{texmf} tree) of your choice.
% Example (linux):
% \begin{quote}
%   |unzip ltxcmds.tds.zip -d ~/texmf|
% \end{quote}
%
% \subsection{Package installation}
%
% \paragraph{Unpacking.} The \xfile{.dtx} file is a self-extracting
% \docstrip\ archive. The files are extracted by running the
% \xfile{.dtx} through \plainTeX:
% \begin{quote}
%   \verb|tex ltxcmds.dtx|
% \end{quote}
%
% \paragraph{TDS.} Now the different files must be moved into
% the different directories in your installation TDS tree
% (also known as \xfile{texmf} tree):
% \begin{quote}
% \def\t{^^A
% \begin{tabular}{@{}>{\ttfamily}l@{ $\rightarrow$ }>{\ttfamily}l@{}}
%   ltxcmds.sty & tex/generic/ltxcmds/ltxcmds.sty\\
%   ltxcmds.pdf & doc/latex/ltxcmds/ltxcmds.pdf\\
%   ltxcmds.dtx & source/latex/ltxcmds/ltxcmds.dtx\\
% \end{tabular}^^A
% }^^A
% \sbox0{\t}^^A
% \ifdim\wd0>\linewidth
%   \begingroup
%     \advance\linewidth by\leftmargin
%     \advance\linewidth by\rightmargin
%   \edef\x{\endgroup
%     \def\noexpand\lw{\the\linewidth}^^A
%   }\x
%   \def\lwbox{^^A
%     \leavevmode
%     \hbox to \linewidth{^^A
%       \kern-\leftmargin\relax
%       \hss
%       \usebox0
%       \hss
%       \kern-\rightmargin\relax
%     }^^A
%   }^^A
%   \ifdim\wd0>\lw
%     \sbox0{\small\t}^^A
%     \ifdim\wd0>\linewidth
%       \ifdim\wd0>\lw
%         \sbox0{\footnotesize\t}^^A
%         \ifdim\wd0>\linewidth
%           \ifdim\wd0>\lw
%             \sbox0{\scriptsize\t}^^A
%             \ifdim\wd0>\linewidth
%               \ifdim\wd0>\lw
%                 \sbox0{\tiny\t}^^A
%                 \ifdim\wd0>\linewidth
%                   \lwbox
%                 \else
%                   \usebox0
%                 \fi
%               \else
%                 \lwbox
%               \fi
%             \else
%               \usebox0
%             \fi
%           \else
%             \lwbox
%           \fi
%         \else
%           \usebox0
%         \fi
%       \else
%         \lwbox
%       \fi
%     \else
%       \usebox0
%     \fi
%   \else
%     \lwbox
%   \fi
% \else
%   \usebox0
% \fi
% \end{quote}
% If you have a \xfile{docstrip.cfg} that configures and enables \docstrip's
% TDS installing feature, then some files can already be in the right
% place, see the documentation of \docstrip.
%
% \subsection{Refresh file name databases}
%
% If your \TeX~distribution
% (\TeX\,Live, \mikTeX, \dots) relies on file name databases, you must refresh
% these. For example, \TeX\,Live\ users run \verb|texhash| or
% \verb|mktexlsr|.
%
% \subsection{Some details for the interested}
%
% \paragraph{Unpacking with \LaTeX.}
% The \xfile{.dtx} chooses its action depending on the format:
% \begin{description}
% \item[\plainTeX:] Run \docstrip\ and extract the files.
% \item[\LaTeX:] Generate the documentation.
% \end{description}
% If you insist on using \LaTeX\ for \docstrip\ (really,
% \docstrip\ does not need \LaTeX), then inform the autodetect routine
% about your intention:
% \begin{quote}
%   \verb|latex \let\install=y\input{ltxcmds.dtx}|
% \end{quote}
% Do not forget to quote the argument according to the demands
% of your shell.
%
% \paragraph{Generating the documentation.}
% You can use both the \xfile{.dtx} or the \xfile{.drv} to generate
% the documentation. The process can be configured by the
% configuration file \xfile{ltxdoc.cfg}. For instance, put this
% line into this file, if you want to have A4 as paper format:
% \begin{quote}
%   \verb|\PassOptionsToClass{a4paper}{article}|
% \end{quote}
% An example follows how to generate the
% documentation with pdf\LaTeX:
% \begin{quote}
%\begin{verbatim}
%pdflatex ltxcmds.dtx
%makeindex -s gind.ist ltxcmds.idx
%pdflatex ltxcmds.dtx
%makeindex -s gind.ist ltxcmds.idx
%pdflatex ltxcmds.dtx
%\end{verbatim}
% \end{quote}
%
% \begin{thebibliography}{9}
% \raggedright
%
% \bibitem{ifempty:schneck}
% Robert R. Schneck:
% \textit{Re: \cs{ifempty} solution %
%   (was Macro puzzle: maximally general \cs{ifempty})};
% newsgroup \xnewsgroup{comp.text.tex},
% \url{news:3eef1ada_6@corp.newsgroups.com},
% 2003-06-17.\\
% {\small
% \url{https://groups.google.com/group/comp.text.tex/msg/be03a159ec374895}}
%
% \bibitem{ifnull:diez}
% Ulrich Diez:
% \textit{Re: TeX refuses to strip outer braces in argument};
% newsgroup \xnewsgroup{comp.text.tex},
% \url{news:ibk3t8$ee7$1@news.albasani.net},
% 2010-11-12.\\
% {\small
% \url{https://groups.google.com/group/comp.text.tex/msg/803bd57221a04996}}
%
% \bibitem{ifstrempty:musa}
% Ahmed Musa:
% \textit{Re: TeX refuses to strip outer braces in argument};
% newsgroup \xnewsgroup{comp.text.tex},
% \url{news:f5496afe-40ed-42bd-b629-a2419ecf7c0d@o14g2000prn.googlegroups.com},
% 2010-12-03.\\
% {\small
% \url{https://groups.google.com/group/comp.text.tex/msg/fbf7d61a0c3a807d}}
%
% \bibitem{ifstrempty:diez1}
% Ulrich Diez:
% \textit{Re: TeX refuses to strip outer braces in argument};
% newsgroup \xnewsgroup{comp.text.tex},
% \url{news:idbo94$uka$1@four.albasani.net},
% 2010-12-03.\\
% {\small
% \url{https://groups.google.com/group/comp.text.tex/msg/0c230ee479487962}}
%
% \bibitem{ifstrempty:diez2}
% Ulrich Diez:
% \textit{Re: TeX refuses to strip outer braces in argument};
% newsgroup \xnewsgroup{comp.text.tex},
% \url{news:idbpu4$cg1$1@news.albasani.net},
% 2010-12-03.\\
% {\small
% \url{https://groups.google.com/group/comp.text.tex/msg/bbef4263390d647b}}
%
% \bibitem{ifstrempty:diez3}
% Ulrich Diez:
% \textit{Re: TeX refuses to strip outer braces in argument};
% newsgroup \xnewsgroup{comp.text.tex},
% \url{news:idd4ga$r83$1@four.albasani.net},
% 2010-12-04.\\
% {\small
% \url{https://groups.google.com/group/comp.text.tex/msg/00dfd1ec103cd272}}
%
% \bibitem{ifstrempty:gl}
% GL:
% \textit{Re: TeX refuses to strip outer braces in argument};
% newsgroup \xnewsgroup{comp.text.tex},
% \url{news:4cfa2e27$0$7389$426a74cc@news.free.fr},
% 2010-12-04.\\
% {\small
% \url{https://groups.google.com/group/comp.text.tex/msg/d3a75995c1cf267e}}
%
% \bibitem{ifstrempty:ho}
% Heiko Oberdiek:
% \textit{Re: TeX refuses to strip outer braces in argument};
% newsgroup \xnewsgroup{comp.text.tex},
% \url{news:iddhq1$3kj$1@news.eternal-september.org},
% 2010-12-04.\\
% {\small
% \url{https://groups.google.com/group/comp.text.tex/msg/5f7a23e3ab70e347}}
%
% \bibitem{ifboxempty:dak}
% David Kastrup:
% \textit{How to detect if \cs{vbox} is empty};
% newsgroup \xnewsgroup{comp.text.tex},
% 2011-02-04.\\
% {\small
% \url{https://groups.google.com/group/comp.text.tex/msg/8d3cb89496a4d86d}}
%
% \end{thebibliography}
%
% \begin{History}
%   \begin{Version}{2009/08/05 v1.0}
%   \item
%     First version.
%   \end{Version}
%   \begin{Version}{2009/12/12 v1.1}
%   \item
%     Short title shortened.
%   \item
%     \cs{ltx@IfUndefined} added.
%   \end{Version}
%   \begin{Version}{2010/01/28 v1.2}
%   \item
%     \cs{ltx@RemovePrefix} and \cs{ltx@StripPrefix} added.
%   \item
%     \cs{ltx@ifclassloaded}, \cs{ltx@ifpackageloaded},
%     \cs{ltx@iffileloaded},
%     \cs{ltx@ifclasslater}, \cs{ltx@ifpackagelater},
%     \cs{ltx@iffilelater},
%     \cs{ltx@clsextension}, \cs{ltx@pkgextension} added.
%   \item
%     \cs{ltx@GlobalAppendToMacro}, \cs{ltx@LocalAppendToMacro} added.
%   \end{Version}
%   \begin{Version}{2010/03/01 v1.3}
%   \item
%     \cs{ltx@newif} added.
%   \item
%     \cs{ltx@ifnextchar} added.
%   \item
%     Numbers \cs{ltx@zero}, \cs{ltx@one}, \cs{ltx@two}, \cs{ltx@cclv} added.
%   \end{Version}
%   \begin{Version}{2010/03/09 v1.4}
%   \item
%     \cs{ltx@pkgextension} and \cs{ltx@clsextension} are hardcoded to
%     avoid trouble with \cs{@onlypreamble}.
%   \end{Version}
%   \begin{Version}{2010/04/08 v1.5}
%   \item
%     \cs{ltx@cartwo}, \cs{ltx@cdrtwo},
%     \cs{ltx@carthree}, \cs{ltx@cdrthree},
%     \cs{ltx@carfour}, \cs{ltx@cdrfour} added.
%   \item
%     \cs{ltx@ReturnAfterFi} and \cs{ltx@ReturnAfterElseFi} fixed.
%   \end{Version}
%   \begin{Version}{2010/04/16 v1.6}
%   \item
%     \cs{ltx@leavevmode}, \cs{ltx@mbox} added.
%   \end{Version}
%   \begin{Version}{2010/04/26 v1.7}
%   \item
%     \cs{ltx@GobbleNum}, \cs{ltx@CdrNum}, \cs{ltx@CarNum} added.
%   \item
%     \cs{ltx@carzero}, \cs{ltx@cdrzero} added.
%   \item
%     \cs{ltx@hashchar} added.
%   \end{Version}
%   \begin{Version}{2010/09/11 v1.8}
%   \item
%     \cs{ltx@leftbracechar}, \cs{ltx@rightbracechar} added.
%   \end{Version}
%   \begin{Version}{2010/10/25 v1.9}
%   \item
%     \cs{ltx@LocalAppendToMacro} and \cs{ltx@GlobalAppendToMacro} are
%     now \cs{long}.
%   \end{Version}
%   \begin{Version}{2010/10/31 v1.10}
%   \item
%     \cs{ltx@newglobalif} added.
%   \end{Version}
%   \begin{Version}{2010/11/12 v1.11}
%   \item
%     \cs{ltx@ifempty} added.
%   \item
%     \cs{ltx@firstofthree}, \cs{ltx@secondofthree}, \cs{ltx@thirdofthree}
%     added.
%   \end{Version}
%   \begin{Version}{2010/12/02 v1.12}
%   \item
%     \cs{ltx@onelevel@sanitize} added.
%   \item
%     \cs{LTXcmds@num} fixed for the case with \cs{numexpr}
%     (bug found by GL).
%   \end{Version}
%   \begin{Version}{2010/12/04 v1.13}
%   \item
%     \cs{ltx@ifblank} added.
%   \item
%     Optimization for \cs{ltx@ifempty}.
%   \end{Version}
%   \begin{Version}{2010/12/07 v1.14}
%   \item
%     \cs{ltx@zapspace} added.
%   \end{Version}
%   \begin{Version}{2010/12/12 v1.15}
%   \item
%     \cs{ltx@minusone} added.
%   \end{Version}
%   \begin{Version}{2011/02/04 v1.16}
%   \item
%     \cs{ltx@IfBoxEmpty} and \cs{ltx@IfBoxVoidOrEmpty} added.
%   \item
%     \cs{ltx@firstoffour}, \dots, \cs{ltx@fourthoffour} added.
%   \end{Version}
%   \begin{Version}{2011/02/05 v1.17}
%   \item
%     \cs{ltx@IfBoxEmpty}: an empty box may have non-zero dimensions.
%   \end{Version}
%   \begin{Version}{2011/03/16 v1.18}
%   \item
%     \cs{ltx@ifclasslater} fixed.
%   \end{Version}
%   \begin{Version}{2011/04/14 v1.19}
%   \item
%     \cs{ltx@ifnextchar}: detection of optional spaces modified.
%   \item
%     \cs{ltx(Loc,Glob)(Toks,Dimen,Skip)(A,B,C,D,E)} added.
%   \end{Version}
%   \begin{Version}{2011/04/18 v1.20}
%   \item
%     \cs{ltx@ifnextchar} with conditional support (thanks GL for
%     bug report).
%   \end{Version}
%   \begin{Version}{2011/08/22 v1.21}
%   \item
%     \cs{ltx@GlobalPrependToMacro}, \cs{ltx@LocalPrependToMacro} added
%     (feature request of Martin M\"unch).
%   \end{Version}
%   \begin{Version}{2011/11/09 v1.22}
%   \item
%     \cs{ltx@carsecond}, \cs{ltx@carthird}, \cs{ltx@carfourth},
%     \cs{ltx@CarNumth} added.
%   \item
%     \cs{ltx@cdrzero}, \cs{ltx@cdr}, \cs{ltx@cdrtwo}, cs{ltx@cdrthree},
%     \cs{ltx@cdrfour}, \cs{ltx@CdrNum} modified to retain braces and spaces.
%     They are expandable in two expansion steps.
%   \end{Version}
%   \begin{Version}{2016/05/16 v1.23}
%   \item
%     Documentation updates.
%   \end{Version}
%   \begin{Version}{2019/12/15 v1.24}
%   \item
%     Documentation updates.
%   \end{Version}
% \end{History}
%
% \PrintIndex
%
% \Finale
\endinput

%        (quote the arguments according to the demands of your shell)
%
% Documentation:
%    (a) If ltxcmds.drv is present:
%           latex ltxcmds.drv
%    (b) Without ltxcmds.drv:
%           latex ltxcmds.dtx; ...
%    The class ltxdoc loads the configuration file ltxdoc.cfg
%    if available. Here you can specify further options, e.g.
%    use A4 as paper format:
%       \PassOptionsToClass{a4paper}{article}
%
%    Programm calls to get the documentation (example):
%       pdflatex ltxcmds.dtx
%       makeindex -s gind.ist ltxcmds.idx
%       pdflatex ltxcmds.dtx
%       makeindex -s gind.ist ltxcmds.idx
%       pdflatex ltxcmds.dtx
%
% Installation:
%    TDS:tex/generic/ltxcmds/ltxcmds.sty
%    TDS:doc/latex/ltxcmds/ltxcmds.pdf
%    TDS:source/latex/ltxcmds/ltxcmds.dtx
%
%<*ignore>
\begingroup
  \catcode123=1 %
  \catcode125=2 %
  \def\x{LaTeX2e}%
\expandafter\endgroup
\ifcase 0\ifx\install y1\fi\expandafter
         \ifx\csname processbatchFile\endcsname\relax\else1\fi
         \ifx\fmtname\x\else 1\fi\relax
\else\csname fi\endcsname
%</ignore>
%<*install>
\input docstrip.tex
\Msg{************************************************************************}
\Msg{* Installation}
\Msg{* Package: ltxcmds 2019/12/15 v1.24 LaTeX kernel commands for general use (HO)}
\Msg{************************************************************************}

\keepsilent
\askforoverwritefalse

\let\MetaPrefix\relax
\preamble

This is a generated file.

Project: ltxcmds
Version: 2019/12/15 v1.24

Copyright (C)
   2009-2011 Heiko Oberdiek
   2016-2019 Oberdiek Package Support Group

This work may be distributed and/or modified under the
conditions of the LaTeX Project Public License, either
version 1.3c of this license or (at your option) any later
version. This version of this license is in
   https://www.latex-project.org/lppl/lppl-1-3c.txt
and the latest version of this license is in
   https://www.latex-project.org/lppl.txt
and version 1.3 or later is part of all distributions of
LaTeX version 2005/12/01 or later.

This work has the LPPL maintenance status "maintained".

The Current Maintainers of this work are
Heiko Oberdiek and the Oberdiek Package Support Group
https://github.com/ho-tex/ltxcmds/issues


The Base Interpreter refers to any `TeX-Format',
because some files are installed in TDS:tex/generic//.

This work consists of the main source file ltxcmds.dtx
and the derived files
   ltxcmds.sty, ltxcmds.pdf, ltxcmds.ins, ltxcmds.drv,
   ltxcmds-test1.tex, ltxcmds-test-gobble.tex,
   ltxcmds-test-ifempty.tex, ltxcmds-test-zapspace.tex,
   ltxcmds-test-ifboxempty.tex, ltxcmds-test-nextchar.tex,
   ltxcmds-test-carcdr.tex.

\endpreamble
\let\MetaPrefix\DoubleperCent

\generate{%
  \file{ltxcmds.ins}{\from{ltxcmds.dtx}{install}}%
  \file{ltxcmds.drv}{\from{ltxcmds.dtx}{driver}}%
  \usedir{tex/generic/ltxcmds}%
  \file{ltxcmds.sty}{\from{ltxcmds.dtx}{package}}%
}

\catcode32=13\relax% active space
\let =\space%
\Msg{************************************************************************}
\Msg{*}
\Msg{* To finish the installation you have to move the following}
\Msg{* file into a directory searched by TeX:}
\Msg{*}
\Msg{*     ltxcmds.sty}
\Msg{*}
\Msg{* To produce the documentation run the file `ltxcmds.drv'}
\Msg{* through LaTeX.}
\Msg{*}
\Msg{* Happy TeXing!}
\Msg{*}
\Msg{************************************************************************}

\endbatchfile
%</install>
%<*ignore>
\fi
%</ignore>
%<*driver>
\NeedsTeXFormat{LaTeX2e}
\ProvidesFile{ltxcmds.drv}%
  [2019/12/15 v1.24 LaTeX kernel commands for general use (HO)]%
\documentclass{ltxdoc}
\usepackage{holtxdoc}[2011/11/22]
\usepackage{zref-savepos}
\usepackage{paralist}
\hfuzz=1pt
\begin{document}
  \DocInput{ltxcmds.dtx}%
\end{document}
%</driver>
% \fi
%
%
%
% \GetFileInfo{ltxcmds.drv}
%
% \title{The \xpackage{ltxcmds} package}
% \date{2019/12/15 v1.24}
% \author{Heiko Oberdiek\thanks
% {Please report any issues at \url{https://github.com/ho-tex/ltxcmds/issues}}}
%
% \maketitle
%
% \begin{abstract}
% The package \xpackage{ltxcmds} exports some utility macros
% from the \LaTeX\ kernel into a separate namespace and
% also provides them for other formats such as plain-\TeX.
% \end{abstract}
%
% \tableofcontents
%
% \newcounter{to}
% \renewcommand*{\theto}{to\arabic{to}}
% \newlength\tolength
% \tolength=\dimexpr .5\linewidth+\oddsidemargin+1in+15mm\relax
% \newcommand*{\TO}{^^A
%   \quad
%   \stepcounter{to}^^A
%   \zsavepos\theto
%   \ifdim\zposx\theto sp>0pt
%     \ifdim\zposx\theto sp<\tolength
%       \kern\dimexpr\tolength-\zposx\theto sp\relax
%     \fi
%   \fi
%   $\rightarrow$\quad
%   \ignorespaces
% }
%
% \section{Documentation}
%
% \subsection{Introduction}
%
% Many of my packages also support other formats such as plain-\TeX.
% Because I am rather familiar with the utility macros from
% \LaTeX's kernel (e.g. \cs{@gobble}, \cs{@firstoftwo}), I found
% myself rewriting them again and again, because they are lacking
% in plain-\TeX.
%
% Therefore this package provides often used macros and similar
% ones with the name prefix \cs{ltx@}. This avoids also faulty
% redefinitions. I remember an example where a package redefined
% \cs{@firstoftwo} with forgetting \cs{long}.
%
% \subsection{Numbers}
%
% \begin{declcs}{ltx@zero} \TO 0\\
%   \cs{ltx@one} \TO 1\\
%   \cs{ltx@two} \TO 2\\
%   \cs{ltx@cclv} \TO 255\\
%   \cs{ltx@minusone} \TO -1
% \end{declcs}
% These commands are numbers 0, 1, 2, 255 and -1. They are not digits
% and a space is not gobbled afterwards. Macro \cs{ltx@minusone}
% is available since version 2010/12/12 v1.15.
%
% \subsection{Scratch registers}
%
% Following the conventions of \hologo{plainTeX} and \hologo{LaTeX}
% the first ten registers are free to use. Even numbered registers
% are for local, odd numbered for global use.
%
% \begin{declcs}{ltx@(Loc,Glob)(Toks,Dimen,Skip)(A,B,C,D,E)}
% \end{declcs}
% The name consists of the prefix \cs{ltx@}, then
% \texttt{Loc} or \texttt{Glob} for local or global usage follows.
% The register type is given by \texttt{Toks} for token register,
% \texttt{Dimen} for dimen register and \texttt{Skip} for skip
% register. As last part the registers are numbered from \texttt{A}
% to \texttt{E}. Example: \cs{ltx@LocToksA}.
%
% Since 2011/04/14 v1.19.
%
% \subsection{Argument killers}
%
% \begin{declcs}{ltx@gobble} \M{1} \TO\\
%   \cs{ltx@gobbletwo} \M{1} \M{2} \TO\\
%   \cs{ltx@gobblethree} \M{1} \M{2} \M{3} \TO\\
%   \cs{ltx@gobblefour} \M{1} \M{2} \M{3} \M{4} \TO
% \end{declcs}
%
% \begin{declcs}{ltx@GobbleNum} \M{num} \M{1} \M{2} \dots\space
%   \M{\meta{num}} \TO
% \end{declcs}
% The first argument \meta{num} of macro \cs{ltx@GobbleNum} specifies,
% how many following arguments are eaten. Macro \cs{ltx@GobbleNum}
% is expandable in exact two expansion steps.
%
% \subsection{Argument grabbers}
%
% \begin{declcs}{ltx@firstofone} \M{1} \TO \meta{1}\\[1ex]
%  \cs{ltx@firstoftwo} \M{1} \M{2} \TO \meta{1}\\
%  \cs{ltx@secondoftwo} \M{1} \M{2} \TO \meta{2}\\[1ex]
%  \cs{ltx@firstofthree} \M{1} \M{2} \M{3} \TO \meta{1}\\
%  \cs{ltx@secondofthree} \M{1} \M{2} \M{3} \TO \meta{2}\\
%  \cs{ltx@thirdofthree} \M{1} \M{2} \M{3} \TO \meta{3}\\
%  \cs{ltx@firstoffour} \M{1} \M{2} \M{3} \M{4} \TO \meta{1}\\
%  \cs{ltx@secondoffour} \M{1} \M{2} \M{3} \M{4} \TO \meta{2}\\
%  \cs{ltx@thirdoffour} \M{1} \M{2} \M{3} \M{4} \TO \meta{3}\\
%  \cs{ltx@fourthoffour} \M{1} \M{2} \M{3} \M{4} \TO \meta{4}
% \end{declcs}
% Macros \cs{ltx@firstofthree}, \cs{ltx@secondofthree}
% and \cs{ltx@thirdofthree} were added in version
% 2010/11/12 v1.11.
% Macros \cs{ltx@firstoffour}, \dots, \cs{ltx@fourthoffour}
% were added in version 2011/02/04 v1.16.
%
% \subsection{List helpers}
%
% \begin{declcs}{ltx@carzero} \dots\ \cs{@nil} \TO\\
%   \cs{ltx@cdrzero} \dots\ \cs{@nil} \TO \dots
% \end{declcs}
%
% \begin{declcs}{ltx@car} \M{1} \dots\ \cs{@nil} \TO \meta{1}\\
%   \cs{ltx@cdr} \M{1} \dots\ \cs{@nil} \TO \dots
% \end{declcs}
%
% \begin{declcs}{ltx@cartwo} \M{1} \M{2} \dots\ \cs{@nil}
%   \TO \meta{1}\meta{2}\\
%   \cs{ltx@carsecond} \M{1} \M{2} \dots\ \cs{@nil} \TO \meta{2}\\
%   \cs{ltx@cdrtwo} \M{1} \M{2} \dots\ \cs{@nil} \TO \dots
% \end{declcs}
%
% \begin{declcs}{ltx@carthree} \M{1} \M{2} \M{3} \dots\ \cs{@nil}
%       \TO \meta{1}\meta{2}\meta{3}\\
%   \cs{ltx@carthird} \M{1} \M{2} \M{3} \dots\ \cs{@nil}
%       \TO \meta{3}\\
%   \cs{ltx@cdrthree} \M{1} \M{2} \M{3} \dots\ \cs{@nil} \TO \dots
% \end{declcs}
%
% \begin{declcs}{ltx@carfour} \M{1} \M{2} \M{3} \M{4} \dots\ \cs{@nil}
%       \TO \meta{1}\meta{2}\meta{3}\meta{4}\\
%   \cs{ltx@carfourth} \M{1} \M{2} \M{3} \M{4} \dots\ \cs{@nil}
%       \TO \meta{4}\\
%   \cs{ltx@cdrfour} \M{1} \M{2} \M{3} \M{4} \dots\ \cs{@nil} \TO \dots
% \end{declcs}
%
% \begin{declcs}{ltx@CarNum} \M{num} \M{1} \dots\ \M{\meta{num}}
%   \M{\meta{num}+1} \dots\ \cs{@nil}\\
%   \TO \M{1} \dots\ \M{\meta{num}} \dots\\
%   \cs{ltx@CarNumth} \M{num} \M{1} \dots\ \M{\meta{num}}
%   \M{\meta{num}+1} \dots\ \cs{@nil}\\
%   \TO \M{\meta{num}} \dots\\
%   \cs{ltx@CdrNum} \M{num} \M{1} \dots\ \M{\meta{num}}
%   \M{\meta{num}+1} \dots\ \cs{@nil}\\
%   \TO \M{\meta{num}+1} \dots
% \end{declcs}
%
% Macros with uppercase letters are expandable in two expansion steps.
% Changes in version 2019/12/15 v1.24:
% \begin{itemize}
% \item
%   Macros \cs{ltx@carsecond}, \cs{ltx@carthird}, \cs{ltx@carfourth},
%   \cs{ltx@CarNumth} added.
% \item
%   Macros \cs{ltx@cdr}, \cs{ltx@cdrtwo}, \cs{ltx@cdrthree}, \cs{ltx@cdrfour},
%   \cs{ltx@CdrNum} are expandable in two expansion steps and retain
%   spaces and braces after the first gobbled arguments.
% \end{itemize}
%
% \subsection{Tail recursion}
%
% \begin{declcs}{ltx@ReturnAfterFi} \M{1} \cs{fi} \TO \cs{fi} \meta{1}\\
%   \cs{ltx@ReturnAfterElseFi} \M{1} \cs{else} \M{2} \cs{fi}
%     \TO \cs{fi} \meta{1}
% \end{declcs}
%
% \subsection{Empty macro}
%
% \begin{declcs}{ltx@empty} \TO
% \end{declcs}
%
% \subsection{Characters}
%
% \begin{declcs}{ltx@space} \TO \verb*| |\\
%   \cs{ltx@percentchar} \TO \texttt{\%}\\
%   \cs{ltx@backslashchar} \TO
%      \makeatletter\texttt{\@backslashchar}\\
%   \cs{ltx@hashchar} \TO \texttt{\#}\qquad(since v1.7)\\
%   \cs{ltx@leftbracechar} \TO \texttt{\textbraceleft}\qquad(since v1.8)\\
%   \cs{ltx@rightbracechar} \TO \texttt{\textbraceright}\qquad(since v1.8)
% \end{declcs}
%
% \subsection{Boolean switch}
%
% \begin{declcs}{ltx@newif} \M{cmd}
% \end{declcs}
% \cs{ltx@newif} defines a new boolean switch \meta{cmd} like \cs{newif}.
% Unlike \plainTeX's \cs{newif}, \cs{ltx@newif} is not \cs{outer}.
% The command \meta{cmd} must start with the two characters |if|.
%
% \begin{declcs}{ltx@newglobalif} \M{cmd}
% \end{declcs}
% \cs{ltx@newglobalif} defines a new boolean switch \meta{cmd}
% like \cs{ltx@newif}. However the switch setting commands,
% \meta{cmd} without the prefix |if| and followed by |true| or |false|
% are acting globally.
%
% \subsection{Command definitions}
%
% \begin{declcs}{ltx@ifundefined} \M{cmd} \M{yes} \M{no}
% \end{declcs}
% If \eTeX\ is available, \cs{ifcsname} is used that does
% not have the side effect of defining undefined commands
% with meaning of \cs{relax}. This command is always expandable.
% Change in version 1.1: Also the meaning \cs{relax} is
% always considered ``undefined''.
%
% \begin{declcs}{ltx@IfUndefined} \M{cmd} \M{yes} \M{no}
% \end{declcs}
% If \eTeX\ is available, \cs{ifcsname} is used that does
% not have the side effect of defining undefined commands
% with meaning of \cs{relax}. Also it always checks for
% the meaning of \cs{relax} and considers this as undefined.
% This macro is not expandable without \eTeX.
%
% \begin{declcs}{ltx@LocalExpandAfter}
% \end{declcs}
% It expands the token after the next token but in a local
% context. That is the difference to \cs{expandafter}.
% The local context discards the side effect of \cs{csname}
% and let the command undefined after the expansion step.
%
% \subsection{Stripping}
%
% \begin{declcs}{ltx@RemovePrefix}\\
%   \cs{ltx@StripPrefix}
% \end{declcs}
% All tokens up to and including the next available character
% `\textgreater' are thrown away. Usually it is used to strip
% the first part of the output of the commands \cs{meaning}
% or \cs{pdflastmatch}. Macro \cs{ltx@RemovePrefix}
% has the same meaning as \LaTeX's \cs{strip@prefix}, whereas
% macro \cs{ltx@StripPrefix} expands the next token
% once before stripping the prefix.
%
% \begin{declcs}{ltx@onelevel@sanitize} \M{macro}
% \end{declcs}
% Macro \cs{ltx@onelevel@sanitize} provides \hologo{LaTeX}'s
% \cs{@onelevel@sanitize}. The macro is expanded once and
% the contents is converted to characters with catcode 12 (other)
% and space tokens with catcode 10 (space). Then then sanitized
% contents is stored into the macro again. Since version 1.12.
%
% \subsection{File management}
%
% All macros in this section are expandable like
% the counterparts of the \LaTeX\ kernel. Also they
% can be used after the preamble.
%
% \subsubsection{File extensions}
%
% \begin{declcs}{ltx@clsextension}\\
%   \cs{ltx@pkgextension}
% \end{declcs}
% Macros \cs{ltx@clsextension} and \cs{ltx@styextension}
% stores the strings |cls| and |sty|. In opposite to \hologo{LaTeX}'s
% \cs{@clsextension} and \cs{@styextension} they can also be
% used after |\begin{document}|.
%
% \subsubsection{Load check}
%
% \begin{declcs}{ltx@ifclassloaded} \M{class} \M{yes} \M{no}\\
%   \cs{ltx@ifpackageloaded} \M{package} \M{yes} \M{no}
% \end{declcs}
% Macros \cs{ltx@ifclassloaded}\slash\cs{ltx@ifpackageloaded}
% execute \meta{yes}, if the \meta{class} or \meta{package} is loaded,
% otherwise \meta{no} is called.
% Both \meta{class} and \meta{package} are specified without
% extension. The macros can also be used after |\begin{document}|.
%
% \begin{declcs}{ltx@iffileloaded} \M{file} \M{yes} \M{no}
% \end{declcs}
% If \LaTeX's \cs{ProvidesFile} macro was called before
% using \meta{file} as argument, then \cs{ltx@iffileloaded}
% calls \meta{yes}, otherwise \meta{no}.
% Therefore it is possible
% that the \meta{file} is loaded, but \meta{no} is executed because
% of a missing \cs{ProvidesFile}. The \LaTeX\ kernel does not have
% a counterpart of \cs{ltx@iffileloaded}.
%
% Note that the file name used in \cs{ProvidesFile} and
% \cs{ltx@iffileloaded} must match. For example, if \TeX's
% default extension \xext{tex} was given in the first command,
% then it must also specified in the latter command and vice versa.
%
% \subsubsection{Version date check}
%
% \begin{declcs}{ltx@ifclasslater} \M{class} \M{date} \M{yes} \M{no}\\
%   \cs{ltx@ifpackagelater} \M{package} \M{date} \M{yes} \M{no}\\
%   \cs{ltx@iffilelater} \M{file} \M{date} \M{yes} \M{no}
% \end{declcs}
% If a \cs{ProvidesClass}\slash\cs{ProvidesPackage}\slash
% \cs{ProvidesFile} command with exact the same class\slash package\slash
% file was executed before with an optional
% argument that starts with a \LaTeX\ version date, then
% this version date is compared with the argument \meta{date}.
% If they are equal or if the version date is the later date,
% then \meta{yes} is called. In all other cases \meta{no} is executed.
%
% A \LaTeX\ date has the format |YYYY/MM/DD| with |YYYY| as year
% with four digits, |MM| as month with two digits and |DD| as day
% with two digits. If \pdfTeX's \cs{pdfmatch} is available, then
% it is used to detect the version date, to reject invalid
% date formats and to reject some invalid dates.
% Dates before 1994/01/01 are always invalid, because version
% dates are introduced with \LaTeXe\ in 1994.
%
% \subsection{Macro additions}
%
% \begin{declcs}{ltx@GlobalAppendToMacro} \M{cmd} \M{addition}\\
%   \cs{ltx@LocalAppendToMacro} \M{cmd} \M{addition}
% \end{declcs}
% The \meta{addition} is appended to the parameterless macro \meta{cmd}.
% If \meta{cmd} is undefined or has the meaning \cs{relax}, then
% it will be initialized as empty macro beforehand. Due to a bug \meta{addition}
% must not contain \cs{par} before version 2010/10/25 v1.9.
%
% \begin{declcs}{ltx@GlobalPrependToMacro} \M{cmd} \M{addition}\\
%   \cs{ltx@LocalPrependToMacro} \M{cmd} \M{addition}
% \end{declcs}
% The \meta{addition} is prepended to the parameterless macro \meta{cmd}.
% If \meta{cmd} is undefined or has the meaning \cs{relax}, then
% it will be initialized as empty macro beforehand. The macros
% were added in version 2011/08/22 v1.21.
%
% \subsection{Next character detection}
%
% \begin{declcs}{ltx@ifnextchar} \M{char} \M{yes} \M{no}
% \end{declcs}
% If next character is \meta{char} then \meta{yes} is called,
% otherwise \meta{no}. The character is not removed.
% Spaces are silently removed when looking for \meta{char}
% as \hologo{LaTeX}'s version \cs{kernel@ifnextchar} does.
% But there are also small differences:
% \begin{itemize}
% \item The space can be used as \meta{char}. In this case
%   optional spaces before \meta{char} are not supported of course.
% \item If the optional space is a command that is a character
%   (defined by \cs{let} or \cs{futurelet}), then
%   \cs{kernel@ifnextchar} breaks with an \hologo{TeX} error.
%   \cs{ltx@ifnextchar} silently removes this token as
%   optional space.
% \end{itemize}
% Since 2010/03/01 v1.3.
%
% \begin{declcs}{ltx@ifnextchar@nospace} \M{char} \M{yes} \M{no}
% \end{declcs}
% Macro \cs{ltx@ifnextchar@nospace} behaves like
% macro \cs{ltx@ifnextchar} with the exception that
% optional spaces are not supported before \meta{char}.
% Since 2011/04/14 v1.19.
%
% \subsection{\cs{ltx@leavevmode}, \cs{ltx@mbox}}
%
% \begin{declcs}{ltx@leavevmode}
% \end{declcs}
% Macro \cs{ltx@leavevmode} calls \hologo{pdfTeX}'s \cs{quitvmode}.
% Otherwise \cs{leavevmode} is used and defined if it is necessary.
%
% \begin{declcs}{ltx@mbox}
% \end{declcs}
% Macro \cs{ltx@mbox} reimplements \cs{mbox} with two changes.
% Instead of \cs{leavevmode} it uses \cs{ltx@leavevmode} and
% stops right after \cs{hbox}. Especially it does not grab
% the argument and allows the extended syntax of \cs{hbox}.
%
% \subsection{Expandable test for emptiness}
%
% \begin{declcs}{ltx@ifempty} \M{stuff} \M{yes} \M{no}
% \end{declcs}
% Macro \cs{ltx@ifempty} checks in exact two expansion steps
% whether \meta{stuff}
% is empty or contains token. Depending on the result
% \meta{yes} or \meta{no} is executed. The token in \meta{stuff}
% may contain \cs{par} and unmatched conditionals (\cs{if}, \cs{else},
% \cs{fi}, \dots). Since version 2010/11/12 v1.11.
%
% \begin{declcs}{ltx@ifblank} \M{stuff} \M{yes} \M{no}
% \end{declcs}
% Macro \cs{ltx@ifblank} tests in exact two expansion steps
% if \meta{stuff} is empty or contain only blank spaces.
% In this case argument \meta{yes} is called. If \meta{stuff}
% contains other tokens than spaces then \meta{no} is executed.
% Since version 2010/12/04 v1.13.
%
% \subsection{Stripping spaces}
%
% \begin{declcs}{ltx@zapspace} \M{stuff}
% \end{declcs}
% Macro \cs{ltx@zapspace} strips spaces from \meta{stuff}
% that are not hidden inside curly braces. Like \hologo{LaTeX}'s
% \cs{zap@space} it is expandable. Differences:
% \begin{itemize}
% \item
% Syntax: \cs{zap@space} also expects a space token
% and \cs{@empty} after \meta{stuff}.
% \item
% Macro \cs{ltx@zapspace} is expandable in exact two
% expansion steps.
% \item
% Macro \cs{ltx@zapspace} always retains curly braces.
% \item
% Macro \cs{zap@space} has a bug. It stops stripping spaces
% after a token group in curly braces if the first two tokens
% inside the group are equal.
% \item
% Macro \cs{ltx@zapspace} also works with \cs{par}
% and conditionals (\cs{if}, \cs{else}, \cs{fi}, \dots).
% \end{itemize}
% Macro \cs{ltx@zapspace} is available since version
% 2010/12/07 v1.14.
%
% \subsection{Check for emptiness of boxes}
%
% \begin{declcs}{ltx@IfBoxEmpty} \M{box register number}
%   \M{yes} \M{no}
% \end{declcs}
% Macro \cs{ltx@IfBoxEmpty} calls \meta{yes} if
% the box exists (\cs{ifvoid} returns false) and the box
% does not contain any content. Otherwise if the box
% is void or contains something, then \meta{no} is executed.
% Thus being empty means that the box exists and is either
% an \cs{hbox} or a \cs{vbox} and may even have dimensions other than
% 0.0\,pt, but the box does not contain anything.
% Macro \cs{ltx@IfBoxEmpty} is available since 2010/02/04 v1.16.
%
% \begin{declcs}{ltx@IfBoxVoidOrEmpty} \M{box register number}
%   \M{yes} \M{no}
% \end{declcs}
% Macro \cs{ltx@IfBoxVoidOrEmpty} calls \meta{yes} if
% the box is either void or does not contain any content.
% Otherwise \meta{no} is executed.
% Macro \cs{ltx@IfBoxVoidOrEmpty} is available since 2010/02/04 v1.16.
%
% \StopEventually{
% }
%
% \section{Implementation}
%
% \subsection{Identification}
%
%    \begin{macrocode}
%<*package>
%    \end{macrocode}
%    Reload check, especially if the package is not used with \LaTeX.
%    \begin{macrocode}
\begingroup\catcode61\catcode48\catcode32=10\relax%
  \catcode13=5 % ^^M
  \endlinechar=13 %
  \catcode35=6 % #
  \catcode39=12 % '
  \catcode44=12 % ,
  \catcode45=12 % -
  \catcode46=12 % .
  \catcode58=12 % :
  \catcode64=11 % @
  \catcode123=1 % {
  \catcode125=2 % }
  \expandafter\let\expandafter\x\csname ver@ltxcmds.sty\endcsname
  \ifx\x\relax % plain-TeX, first loading
  \else
    \def\empty{}%
    \ifx\x\empty % LaTeX, first loading,
      % variable is initialized, but \ProvidesPackage not yet seen
    \else
      \expandafter\ifx\csname PackageInfo\endcsname\relax
        \def\x#1#2{%
          \immediate\write-1{Package #1 Info: #2.}%
        }%
      \else
        \def\x#1#2{\PackageInfo{#1}{#2, stopped}}%
      \fi
      \x{ltxcmds}{The package is already loaded}%
      \aftergroup\endinput
    \fi
  \fi
\endgroup%
%    \end{macrocode}
%    Package identification:
%    \begin{macrocode}
\begingroup\catcode61\catcode48\catcode32=10\relax%
  \catcode13=5 % ^^M
  \endlinechar=13 %
  \catcode35=6 % #
  \catcode39=12 % '
  \catcode40=12 % (
  \catcode41=12 % )
  \catcode44=12 % ,
  \catcode45=12 % -
  \catcode46=12 % .
  \catcode47=12 % /
  \catcode58=12 % :
  \catcode64=11 % @
  \catcode91=12 % [
  \catcode93=12 % ]
  \catcode123=1 % {
  \catcode125=2 % }
  \expandafter\ifx\csname ProvidesPackage\endcsname\relax
    \def\x#1#2#3[#4]{\endgroup
      \immediate\write-1{Package: #3 #4}%
      \xdef#1{#4}%
    }%
  \else
    \def\x#1#2[#3]{\endgroup
      #2[{#3}]%
      \ifx#1\@undefined
        \xdef#1{#3}%
      \fi
      \ifx#1\relax
        \xdef#1{#3}%
      \fi
    }%
  \fi
\expandafter\x\csname ver@ltxcmds.sty\endcsname
\ProvidesPackage{ltxcmds}%
  [2019/12/15 v1.24 LaTeX kernel commands for general use (HO)]%
%    \end{macrocode}
%
%    \begin{macrocode}
\begingroup\catcode61\catcode48\catcode32=10\relax%
  \catcode13=5 % ^^M
  \endlinechar=13 %
  \catcode123=1 % {
  \catcode125=2 % }
  \catcode64=11 % @
  \def\x{\endgroup
    \expandafter\edef\csname LTXcmds@AtEnd\endcsname{%
      \endlinechar=\the\endlinechar\relax
      \catcode13=\the\catcode13\relax
      \catcode32=\the\catcode32\relax
      \catcode35=\the\catcode35\relax
      \catcode61=\the\catcode61\relax
      \catcode64=\the\catcode64\relax
      \catcode123=\the\catcode123\relax
      \catcode125=\the\catcode125\relax
    }%
  }%
\x\catcode61\catcode48\catcode32=10\relax%
\catcode13=5 % ^^M
\endlinechar=13 %
\catcode35=6 % #
\catcode64=11 % @
\catcode123=1 % {
\catcode125=2 % }
\def\TMP@EnsureCode#1#2{%
  \edef\LTXcmds@AtEnd{%
    \LTXcmds@AtEnd
    \catcode#1=\the\catcode#1\relax
  }%
  \catcode#1=#2\relax
}
\TMP@EnsureCode{36}{3}% $
\TMP@EnsureCode{38}{4}% &
\TMP@EnsureCode{40}{12}% (
\TMP@EnsureCode{41}{12}% )
\TMP@EnsureCode{45}{12}% -
\TMP@EnsureCode{46}{12}% .
\TMP@EnsureCode{47}{12}% /
\TMP@EnsureCode{60}{12}% <
\TMP@EnsureCode{62}{12}% >
\TMP@EnsureCode{91}{12}% [
\TMP@EnsureCode{96}{12}% `
\TMP@EnsureCode{93}{12}% ]
\TMP@EnsureCode{94}{12}% ^ (superscript) (!)
\TMP@EnsureCode{124}{12}% |
\edef\LTXcmds@AtEnd{\LTXcmds@AtEnd\noexpand\endinput}
%    \end{macrocode}
%
% \subsection{Numbers}
%
%    \begin{macro}{\ltx@zero}
%    \begin{macrocode}
\chardef\ltx@zero=0 %
%    \end{macrocode}
%    \end{macro}
%    \begin{macro}{\ltx@one}
%    \begin{macrocode}
\chardef\ltx@one=1 %
%    \end{macrocode}
%    \end{macro}
%    \begin{macro}{\ltx@two}
%    \begin{macrocode}
\chardef\ltx@two=2 %
%    \end{macrocode}
%    \end{macro}
%    \begin{macro}{\ltx@active}
%    \begin{macrocode}
\chardef\ltx@active=13 %
%    \end{macrocode}
%    \end{macro}
%    \begin{macro}{\ltx@cclv}
%    \begin{macrocode}
\chardef\ltx@cclv=255 %
%    \end{macrocode}
%    \end{macro}
%
%    \begin{macro}{\ltx@minusone}
%    \begin{macrocode}
\def\ltx@minusone{%
  -\ltx@one
}
%    \end{macrocode}
%    \end{macro}
%
% \subsection{Scratch registers}
%
%    \begin{macro}{\ltx@LocToksA}
%    \begin{macrocode}
\toksdef\ltx@LocToksA=0 %
%    \end{macrocode}
%    \end{macro}
%    \begin{macro}{\ltx@LocToksB}
%    \begin{macrocode}
\toksdef\ltx@LocToksB=2 %
%    \end{macrocode}
%    \end{macro}
%    \begin{macro}{\ltx@LocToksC}
%    \begin{macrocode}
\toksdef\ltx@LocToksC=4 %
%    \end{macrocode}
%    \end{macro}
%    \begin{macro}{\ltx@LocToksD}
%    \begin{macrocode}
\toksdef\ltx@LocToksD=6 %
%    \end{macrocode}
%    \end{macro}
%    \begin{macro}{\ltx@LocToksE}
%    \begin{macrocode}
\toksdef\ltx@LocToksE=8 %
%    \end{macrocode}
%    \end{macro}
%
%    \begin{macro}{\ltx@GlobToksA}
%    \begin{macrocode}
\toksdef\ltx@GlobToksA=1 %
%    \end{macrocode}
%    \end{macro}
%    \begin{macro}{\ltx@GlobToksB}
%    \begin{macrocode}
\toksdef\ltx@GlobToksB=3 %
%    \end{macrocode}
%    \end{macro}
%    \begin{macro}{\ltx@GlobToksC}
%    \begin{macrocode}
\toksdef\ltx@GlobToksC=5 %
%    \end{macrocode}
%    \end{macro}
%    \begin{macro}{\ltx@GlobToksD}
%    \begin{macrocode}
\toksdef\ltx@GlobToksD=7 %
%    \end{macrocode}
%    \end{macro}
%    \begin{macro}{\ltx@GlobToksE}
%    \begin{macrocode}
\toksdef\ltx@GlobToksE=9 %
%    \end{macrocode}
%    \end{macro}
%
%    \begin{macro}{\ltx@LocDimenA}
%    \begin{macrocode}
\dimendef\ltx@LocDimenA=0 %
%    \end{macrocode}
%    \end{macro}
%    \begin{macro}{\ltx@LocDimenB}
%    \begin{macrocode}
\dimendef\ltx@LocDimenB=2 %
%    \end{macrocode}
%    \end{macro}
%    \begin{macro}{\ltx@LocDimenC}
%    \begin{macrocode}
\dimendef\ltx@LocDimenC=4 %
%    \end{macrocode}
%    \end{macro}
%    \begin{macro}{\ltx@LocDimenD}
%    \begin{macrocode}
\dimendef\ltx@LocDimenD=6 %
%    \end{macrocode}
%    \end{macro}
%    \begin{macro}{\ltx@LocDimenE}
%    \begin{macrocode}
\dimendef\ltx@LocDimenE=8 %
%    \end{macrocode}
%    \end{macro}
%
%    \begin{macro}{\ltx@GlobDimenA}
%    \begin{macrocode}
\dimendef\ltx@GlobDimenA=1 %
%    \end{macrocode}
%    \end{macro}
%    \begin{macro}{\ltx@GlobDimenB}
%    \begin{macrocode}
\dimendef\ltx@GlobDimenB=3 %
%    \end{macrocode}
%    \end{macro}
%    \begin{macro}{\ltx@GlobDimenC}
%    \begin{macrocode}
\dimendef\ltx@GlobDimenC=5 %
%    \end{macrocode}
%    \end{macro}
%    \begin{macro}{\ltx@GlobDimenD}
%    \begin{macrocode}
\dimendef\ltx@GlobDimenD=7 %
%    \end{macrocode}
%    \end{macro}
%    \begin{macro}{\ltx@GlobDimenE}
%    \begin{macrocode}
\dimendef\ltx@GlobDimenE=9 %
%    \end{macrocode}
%    \end{macro}
%
%    \begin{macro}{\ltx@LocSkipA}
%    \begin{macrocode}
\skipdef\ltx@LocSkipA=0 %
%    \end{macrocode}
%    \end{macro}
%    \begin{macro}{\ltx@LocSkipB}
%    \begin{macrocode}
\skipdef\ltx@LocSkipB=2 %
%    \end{macrocode}
%    \end{macro}
%    \begin{macro}{\ltx@LocSkipC}
%    \begin{macrocode}
\skipdef\ltx@LocSkipC=4 %
%    \end{macrocode}
%    \end{macro}
%    \begin{macro}{\ltx@LocSkipD}
%    \begin{macrocode}
\skipdef\ltx@LocSkipD=6 %
%    \end{macrocode}
%    \end{macro}
%    \begin{macro}{\ltx@LocSkipE}
%    \begin{macrocode}
\skipdef\ltx@LocSkipE=8 %
%    \end{macrocode}
%    \end{macro}
%
%    \begin{macro}{\ltx@GlobSkipA}
%    \begin{macrocode}
\skipdef\ltx@GlobSkipA=1 %
%    \end{macrocode}
%    \end{macro}
%    \begin{macro}{\ltx@GlobSkipB}
%    \begin{macrocode}
\skipdef\ltx@GlobSkipB=3 %
%    \end{macrocode}
%    \end{macro}
%    \begin{macro}{\ltx@GlobSkipC}
%    \begin{macrocode}
\skipdef\ltx@GlobSkipC=5 %
%    \end{macrocode}
%    \end{macro}
%    \begin{macro}{\ltx@GlobSkipD}
%    \begin{macrocode}
\skipdef\ltx@GlobSkipD=7 %
%    \end{macrocode}
%    \end{macro}
%    \begin{macro}{\ltx@GlobSkipE}
%    \begin{macrocode}
\skipdef\ltx@GlobSkipE=9 %
%    \end{macrocode}
%    \end{macro}
%
%
% \subsection{Argument killers}
%
%    \begin{macro}{\ltx@gobble}
%    \begin{macrocode}
\long\def\ltx@gobble#1{}
%    \end{macrocode}
%    \end{macro}
%    \begin{macro}{\ltx@gobbletwo}
%    \begin{macrocode}
\long\def\ltx@gobbletwo#1#2{}
%    \end{macrocode}
%    \end{macro}
%    \begin{macro}{\ltx@gobblethree}
%    \begin{macrocode}
\long\def\ltx@gobblethree#1#2#3{}
%    \end{macrocode}
%    \end{macro}
%    \begin{macro}{\ltx@gobblefour}
%    \begin{macrocode}
\long\def\ltx@gobblefour#1#2#3#4{}
%    \end{macrocode}
%    \end{macro}
%
%    \begin{macro}{\ltx@GobbleNum}
%    \begin{macrocode}
\def\ltx@GobbleNum#1{%
  \romannumeral
  \csname ltx@zero%
  \expandafter\LTXcmds@GobbleNum
  \romannumeral\LTXcmds@num{#1}000{m\endcsname}%
}
%    \end{macrocode}
%    \end{macro}
%    \begin{macro}{\LTXcmds@GobbleNum}
%    \begin{macrocode}
\def\LTXcmds@GobbleNum#1{%
  \csname LTXcmds@G#1\LTXcmds@GobbleNum
}
%    \end{macrocode}
%    \end{macro}
%    \begin{macro}{\LTXcmds@Gm}
%    \begin{macrocode}
\long\def\LTXcmds@Gm#1{%
  \endcsname
}
%    \end{macrocode}
%    \end{macro}
%
% \subsection{Argument grabbers}
%
%    \begin{macro}{\ltx@firstofone}
%    \begin{macrocode}
\long\def\ltx@firstofone#1{#1}
%    \end{macrocode}
%    \end{macro}
%    \begin{macro}{\ltx@firstoftwo}
%    \begin{macrocode}
\long\def\ltx@firstoftwo#1#2{#1}
%    \end{macrocode}
%    \end{macro}
%    \begin{macro}{\ltx@secondoftwo}
%    \begin{macrocode}
\long\def\ltx@secondoftwo#1#2{#2}
%    \end{macrocode}
%    \end{macro}
%
%    \begin{macro}{\ltx@firstofthree}
%    \begin{macrocode}
\long\def\ltx@firstofthree#1#2#3{#1}
%    \end{macrocode}
%    \end{macro}
%    \begin{macro}{\ltx@secondofthree}
%    \begin{macrocode}
\long\def\ltx@secondofthree#1#2#3{#2}
%    \end{macrocode}
%    \end{macro}
%    \begin{macro}{\ltx@thirdofthree}
%    \begin{macrocode}
\long\def\ltx@thirdofthree#1#2#3{#3}%
%    \end{macrocode}
%    \end{macro}
%
%    \begin{macro}{\ltx@firstoffour}
%    \begin{macrocode}
\long\def\ltx@firstoffour#1#2#3#4{#1}
%    \end{macrocode}
%    \end{macro}
%    \begin{macro}{\ltx@secondoffour}
%    \begin{macrocode}
\long\def\ltx@secondoffour#1#2#3#4{#2}
%    \end{macrocode}
%    \end{macro}
%    \begin{macro}{\ltx@thirdoffour}
%    \begin{macrocode}
\long\def\ltx@thirdoffour#1#2#3#4{#3}%
%    \end{macrocode}
%    \end{macro}
%    \begin{macro}{\ltx@fourthoffour}
%    \begin{macrocode}
\long\def\ltx@fourthoffour#1#2#3#4{#4}%
%    \end{macrocode}
%    \end{macro}
%
% \subsection{List helpers}
%
%    \begin{macro}{\ltx@carzero}
%    \begin{macrocode}
\long\def\ltx@carzero#1\@nil{}%
%    \end{macrocode}
%    \end{macro}
%    \begin{macro}{\LTXcmds@cdrzero}
%    \begin{macrocode}
\long\def\LTXcmds@cdrzero#1\@nil{#1}
%    \end{macrocode}
%    \end{macro}
%    \begin{macro}{\ltx@cdrzero}
%    \begin{macrocode}
\def\ltx@cdrzero{%
  \romannumeral\LTXcmds@cdrzero\ltx@zero
}
%    \end{macrocode}
%    \end{macro}
%
%    \begin{macro}{\ltx@car}
%    \begin{macrocode}
\long\def\ltx@car#1#2\@nil{#1}
%    \end{macrocode}
%    \end{macro}
%    \begin{macro}{\ltx@cdr}
%    \begin{macrocode}
\long\def\ltx@cdr#1{%
  \romannumeral\LTXcmds@cdrzero\ltx@zero
}
%    \end{macrocode}
%    \end{macro}
%
%    \begin{macro}{\ltx@cartwo}
%    \begin{macrocode}
\long\def\ltx@cartwo#1#2#3\@nil{#1#2}
%    \end{macrocode}
%    \end{macro}
%    \begin{macro}{\ltx@carsecond}
%    \begin{macrocode}
\long\def\ltx@carsecond#1#2#3\@nil{#2}
%    \end{macrocode}
%    \end{macro}
%    \begin{macro}{\ltx@cdrtwo}
%    \begin{macrocode}
\long\def\ltx@cdrtwo#1#2{%
  \romannumeral\LTXcmds@cdrzero\ltx@zero
}
%    \end{macrocode}
%    \end{macro}
%
%    \begin{macro}{\ltx@carthree}
%    \begin{macrocode}
\long\def\ltx@carthree#1#2#3#4\@nil{#1#2#3}
%    \end{macrocode}
%    \end{macro}
%    \begin{macro}{\ltx@carthird}
%    \begin{macrocode}
\long\def\ltx@carthird#1#2#3#4\@nil{#3}
%    \end{macrocode}
%    \end{macro}
%    \begin{macro}{\ltx@cdrthree}
%    \begin{macrocode}
\long\def\ltx@cdrthree#1#2#3{%
  \romannumeral\LTXcmds@cdrzero\ltx@zero
}
%    \end{macrocode}
%    \end{macro}
%
%    \begin{macro}{\ltx@carfour}
%    \begin{macrocode}
\long\def\ltx@carfour#1#2#3#4#5\@nil{#1#2#3#4}
%    \end{macrocode}
%    \end{macro}
%    \begin{macro}{\ltx@carfourth}
%    \begin{macrocode}
\long\def\ltx@carfourth#1#2#3#4#5\@nil{#4}
%    \end{macrocode}
%    \end{macro}
%    \begin{macro}{\ltx@cdrfour}
%    \begin{macrocode}
\long\def\ltx@cdrfour#1#2#3#4{%
  \romannumeral\LTXcmds@cdrzero\ltx@zero
}
%    \end{macrocode}
%    \end{macro}
%
%    \begin{macro}{\ltx@CarNum}
%    \begin{macrocode}
\def\ltx@CarNum#1{%
  \romannumeral
  \csname LTXcmds@CarNumFinish%
  \expandafter\LTXcmds@CarNum
  \romannumeral\LTXcmds@num{#1}000{x\endcsname}%
}
%    \end{macrocode}
%    \end{macro}
%    \begin{macro}{\LTXcmds@CarNum}
%    \begin{macrocode}
\def\LTXcmds@CarNum#1{%
  \csname LTXcmds@C#1\LTXcmds@CarNum
}
%    \end{macrocode}
%    \end{macro}
%    \begin{macro}{\LTXcmds@Cm}
%    \begin{macrocode}
\long\def\LTXcmds@Cm#1#2{%
  \endcsname{#1#2}%
}
%    \end{macrocode}
%    \end{macro}
%    \begin{macro}{\LTXcmds@Cx}
%    \begin{macrocode}
\def\LTXcmds@Cx#1{%
  \endcsname{}%
}
%    \end{macrocode}
%    \end{macro}
%    \begin{macro}{\LTXcmds@CarNumFinish}
%    \begin{macrocode}
\long\def\LTXcmds@CarNumFinish#1#2\@nil{%
  \ltx@zero
  #1%
}
%    \end{macrocode}
%    \end{macro}
%
%    \begin{macro}{\ltx@CarNumth}
%    \begin{macrocode}
\def\ltx@CarNumth#1{%
  \romannumeral
  \expandafter\expandafter\expandafter
  \LTXcmds@CarNumth
  \ltx@GobbleNum{#1}{}%
}
%    \end{macrocode}
%    \end{macro}
%    \begin{macro}{\LTXcmds@CarNumth}
%    \begin{macrocode}
\long\def\LTXcmds@CarNumth#1#2\@nil{%
  \ltx@zero
  #1%
}
%    \end{macrocode}
%    \end{macro}
%
%    \begin{macro}{\ltx@CdrNum}
%    \begin{macrocode}
\def\ltx@CdrNum#1{%
  \romannumeral%
  \expandafter\expandafter\expandafter\ltx@cdrzero
  \expandafter\expandafter\expandafter\ltx@zero
  \ltx@GobbleNum{#1}%
}
%    \end{macrocode}
%    \end{macro}
%
% \subsection{Tail recursion}
%
%    \begin{macro}{\ltx@ReturnAfterFi}
%    \begin{macrocode}
\long\def\ltx@ReturnAfterFi#1\fi{\fi#1}
%    \end{macrocode}
%    \end{macro}
%    \begin{macro}{\ltx@ReturnAfterElseFi}
%    \begin{macrocode}
\long\def\ltx@ReturnAfterElseFi#1\else#2\fi{\fi#1}
%    \end{macrocode}
%    \end{macro}
%
% \subsection{Empty macro}
%
%    \begin{macro}{\ltx@empty}
%    \begin{macrocode}
\def\ltx@empty{}
%    \end{macrocode}
%    \end{macro}
%
% \subsection{Characters}
%
%    \begin{macro}{\ltx@space}
%    \begin{macrocode}
\def\ltx@space{ }
%    \end{macrocode}
%    \end{macro}
%    \begin{macro}{\ltx@percentchar}
%    \begin{macrocode}
\begingroup
  \lccode`0=`\%\relax
\lowercase{\endgroup
  \def\ltx@percentchar{0}%
}
%    \end{macrocode}
%    \end{macro}
%    \begin{macro}{\ltx@backslashchar}
%    \begin{macrocode}
\begingroup
  \lccode`0=`\\\relax
\lowercase{\endgroup
  \def\ltx@backslashchar{0}%
}
%    \end{macrocode}
%    \end{macro}
%    \begin{macro}{\ltx@hashchar}
%    \begin{macrocode}
\begingroup
  \lccode`0=`\#\relax
\lowercase{\endgroup
  \def\ltx@hashchar{0}%
}
%    \end{macrocode}
%    \end{macro}
%    \begin{macro}{\ltx@leftbracechar}
%    \begin{macrocode}
\begingroup
  \lccode`0=`\{\relax
\lowercase{\endgroup
  \def\ltx@leftbracechar{0}%
}
%    \end{macrocode}
%    \end{macro}
%    \begin{macro}{\ltx@rightbracechar}
%    \begin{macrocode}
\begingroup
  \lccode`0=`\}\relax
\lowercase{\endgroup
  \def\ltx@rightbracechar{0}%
}
%    \end{macrocode}
%    \end{macro}
%
% \subsection{Boolean switch}
%
%    \begin{macro}{\ltx@newif}
%    \begin{macrocode}
\def\ltx@newif#1{%
  \begingroup
    \escapechar=-1 %
  \expandafter\endgroup
  \expandafter\LTXcmds@newif\string#1\@nil
}
%    \end{macrocode}
%    \end{macro}
%    \begin{macro}{\LTXcmds@newif}
%    \begin{macrocode}
\begingroup
  \escapechar=-1 %
\expandafter\endgroup
\expandafter\def\expandafter\LTXcmds@newif\string\if#1\@nil{%
  \expandafter\edef\csname#1true\endcsname{%
    \let
    \expandafter\noexpand\csname if#1\endcsname
    \noexpand\iftrue
  }%
  \expandafter\edef\csname#1false\endcsname{%
    \let
    \expandafter\noexpand\csname if#1\endcsname
    \noexpand\iffalse
  }%
  \csname#1false\endcsname
}
%    \end{macrocode}
%    \end{macro}
%
%    \begin{macro}{\ltx@newglobalif}
%    \begin{macrocode}
\def\ltx@newglobalif#1{%
  \begingroup
    \escapechar=-1 %
  \expandafter\endgroup
  \expandafter\LTXcmds@newglobalif\string#1\@nil
}
%    \end{macrocode}
%    \end{macro}
%    \begin{macro}{\LTXcmds@newglobalif}
%    \begin{macrocode}
\begingroup
  \escapechar=-1 %
\expandafter\endgroup
\expandafter
\def\expandafter\LTXcmds@newglobalif\string\if#1\@nil{%
  \expandafter\edef\csname#1true\endcsname{%
    \global\let
    \expandafter\noexpand\csname if#1\endcsname
    \noexpand\iftrue
  }%
  \expandafter\edef\csname#1false\endcsname{%
    \global\let
    \expandafter\noexpand\csname if#1\endcsname
    \noexpand\iffalse
  }%
  \csname#1false\endcsname
}
%    \end{macrocode}
%    \end{macro}
%
% \subsection{Command definitions}
%
%    \begin{macro}{\ltx@LocalExpandAfter}
%    \begin{macrocode}
\def\ltx@LocalExpandAfter{%
  \begingroup
    \expandafter\expandafter\expandafter
  \endgroup
  \expandafter
}
%    \end{macrocode}
%    \end{macro}
%
%    \begin{macrocode}
\ltx@LocalExpandAfter
\ifx\csname ifcsname\endcsname\relax
%    \end{macrocode}
%    \begin{macro}{\ltx@ifundefined}
%    \begin{macrocode}
  \def\ltx@ifundefined#1{%
    \expandafter\ifx\csname #1\endcsname\relax
      \expandafter\ltx@firstoftwo
    \else
      \expandafter\ltx@secondoftwo
    \fi
  }%
%    \end{macrocode}
%    \end{macro}
%    \begin{macro}{\ltx@IfUndefined}
%    \begin{macrocode}
  \def\ltx@IfUndefined#1{%
    \begingroup\expandafter\expandafter\expandafter\endgroup
    \expandafter\ifx\csname #1\endcsname\relax
      \expandafter\ltx@firstoftwo
    \else
      \expandafter\ltx@secondoftwo
    \fi
  }%
%    \end{macrocode}
%    \end{macro}
%    \begin{macrocode}
  \expandafter\ltx@gobble
\else
  \expandafter\ltx@firstofone
\fi
{%
%    \end{macrocode}
%    \begin{macro}{\ltx@ifundefined}
%    \begin{macrocode}
  \def\ltx@ifundefined#1{%
    \ifcsname #1\endcsname
      \expandafter\ifx\csname #1\endcsname\relax
        \expandafter\expandafter\expandafter\ltx@firstoftwo
      \else
        \expandafter\expandafter\expandafter\ltx@secondoftwo
      \fi
    \else
      \expandafter\ltx@firstoftwo
    \fi
  }%
%    \end{macrocode}
%    \end{macro}
%    \begin{macro}{\ltx@IfUndefined}
%    \begin{macrocode}
  \let\ltx@IfUndefined\ltx@ifundefined
%    \end{macrocode}
%    \end{macro}
%    \begin{macrocode}
}
%    \end{macrocode}
%
% \subsection{Stripping}
%
%    \begin{macro}{\ltx@RemovePrefix}
%    \begin{macrocode}
\def\ltx@RemovePrefix#1>{}
%    \end{macrocode}
%    \end{macro}
%
%    \begin{macro}{\ltx@StripPrefix}
%    \begin{macrocode}
\def\ltx@StripPrefix{%
  \expandafter\ltx@RemovePrefix
}
%    \end{macrocode}
%    \end{macro}
%
%    \begin{macro}{\ltx@onelevel@sanitize}
%    \begin{macrocode}
\def\ltx@onelevel@sanitize#1{%
  \edef#1{%
    \expandafter
    \ltx@RemovePrefix\meaning#1%
  }%
}
%    \end{macrocode}
%    \end{macro}
%
% \subsection{File management}
%
% \subsubsection{File extensions}
%
%    \begin{macro}{\ltx@clsextension}
%    \begin{macrocode}
\def\ltx@clsextension{cls}
%    \end{macrocode}
%    \end{macro}
%    \begin{macro}{\ltx@pkgextension}
%    \begin{macrocode}
\def\ltx@pkgextension{sty}
%    \end{macrocode}
%    \end{macro}
%
% \subsubsection{Load check}
%
%    \begin{macro}{\ltx@iffileloaded}
%    \begin{macrocode}
\def\ltx@iffileloaded#1{%
  \ltx@ifundefined{ver@#1}\ltx@secondoftwo\ltx@firstoftwo
}
%    \end{macrocode}
%    \end{macro}
%    \begin{macro}{\ltx@ifclassloaded}
%    \begin{macrocode}
\def\ltx@ifclassloaded#1{%
  \ltx@iffileloaded{#1.\ltx@clsextension}%
}
%    \end{macrocode}
%    \end{macro}
%    \begin{macro}{\ltx@ifpackageloaded}
%    \begin{macrocode}
\def\ltx@ifpackageloaded#1{%
  \ltx@iffileloaded{#1.\ltx@pkgextension}%
}
%    \end{macrocode}
%    \end{macro}
%
% \subsubsection{Version date check}
%
%    \begin{macro}{\ltx@iffilelater}
%    \begin{macrocode}
\def\ltx@iffilelater#1#2{%
  \ltx@iffileloaded{#1}{%
    \expandafter\LTXcmds@IfLater\expandafter{%
      \number
      \expandafter\expandafter\expandafter\LTXcmds@ParseVersion
      \expandafter\expandafter\expandafter{%
        \csname ver@#1\endcsname
      }%
    \expandafter}\expandafter{%
      \number
      \expandafter\LTXcmds@ParseVersion\expandafter{#2}%
    }%
  }\ltx@secondoftwo
}
%    \end{macrocode}
%    \end{macro}
%    \begin{macro}{\LTXcmds@IfLater}
%    \begin{macrocode}
\def\LTXcmds@IfLater#1#2{%
  \ifcase 0%
      \ifnum#1<19940101 %
      \else
        \ifnum#2<19940101 %
        \else
          \ifnum#2>#1 %
          \else
            1%
          \fi
        \fi
      \fi
      \ltx@space
    \expandafter\ltx@secondoftwo
  \else
    \expandafter\ltx@firstoftwo
  \fi
}
%    \end{macrocode}
%    \end{macro}
%    \begin{macro}{\ltx@ifclasslater}
%    \begin{macrocode}
\def\ltx@ifclasslater#1{%
  \ltx@iffilelater{#1.\ltx@clsextension}%
}
%    \end{macrocode}
%    \end{macro}
%    \begin{macro}{\ltx@ifpackagelater}
%    \begin{macrocode}
\def\ltx@ifpackagelater#1{%
  \ltx@iffilelater{#1.\ltx@pkgextension}%
}
%    \end{macrocode}
%    \end{macro}
%
%    \begin{macrocode}
\ltx@IfUndefined{pdfmatch}{%
%    \end{macrocode}
%    \begin{macro}{\LTXcmds@ParseVersion}
%    \begin{macrocode}
  \def\LTXcmds@ParseVersion#1{%
    \LTXcmds@@ParseVersion#10000/00/00\@nil
  }%
%    \end{macrocode}
%    \end{macro}
%    \begin{macro}{\LTXcmds@@ParseVersion}
%    \begin{macrocode}
  \def\LTXcmds@@ParseVersion#1#2#3#4/#5#6/#7#8#9\@nil{%
    #1#2#3#4#5#6#7#8%
  }%
%    \end{macrocode}
%    \end{macro}
%    \begin{macrocode}
}{%
%    \end{macrocode}
%    \begin{macro}{\LTXcmds@ParseVersion}
%    \begin{macrocode}
  \def\LTXcmds@ParseVersion#1{%
    \ifnum\pdfmatch{%
      ^%
      (199[4-9]|[2-9][0-9][0-9][0-9])/%
      (0[1-9]|1[0-2])/%
      (0[1-9]|[1-2][0-9]|3[0-1])%
    }{#1}=1 %
      \ltx@StripPrefix\pdflastmatch1 %
      \ltx@StripPrefix\pdflastmatch2 %
      \ltx@StripPrefix\pdflastmatch3 %
    \else
      0%
    \fi
  }%
%    \end{macrocode}
%    \end{macro}
%    \begin{macrocode}
}
%    \end{macrocode}
%
% \subsection{Macro additions}
%
%    \begin{macro}{\ltx@GlobalAppendToMacro}
%    \begin{macrocode}
\long\def\ltx@GlobalAppendToMacro#1#2{%
  \ifx\ltx@undefined#1%
    \let#1\ltx@empty
  \else
    \ifx\relax#1%
      \let#1\ltx@empty
    \fi
  \fi
  \begingroup
    \ltx@LocToksA\expandafter{#1#2}%
    \xdef#1{\the\ltx@LocToksA}%
  \endgroup
}
%    \end{macrocode}
%    \end{macro}
%    \begin{macro}{\ltx@LocalAppendToMacro}
%    \begin{macrocode}
\long\def\ltx@LocalAppendToMacro#1#2{%
  \global\let\LTXcmds@gtemp#1%
  \ifx\ltx@undefined\LTXcmds@gtemp
    \global\let\LTXcmds@gtemp\ltx@empty
  \else
    \ifx\relax\LTXcmds@gtemp
      \global\letLTXcmds@gtemp\ltx@empty
    \fi
  \fi
  \begingroup
    \ltx@LocToksA\expandafter{\LTXcmds@gtemp#2}%
    \xdef\LTXcmds@gtemp{\the\ltx@LocToksA}%
  \endgroup
  \let#1\LTXcmds@gtemp
}
%    \end{macrocode}
%    \end{macro}
%
%    \begin{macro}{\ltx@GlobalPrependToMacro}
%    \begin{macrocode}
\long\def\ltx@GlobalPrependToMacro#1#2{%
  \ifx\ltx@undefined#1%
    \let#1\ltx@empty
  \else
    \ifx\relax#1%
      \let#1\ltx@empty
    \fi
  \fi
  \begingroup
    \ltx@LocToksA{#2}%
    \ltx@LocToksB\expandafter{#1}%
    \xdef#1{\the\ltx@LocToksA\the\ltx@LocToksB}%
  \endgroup
}
%    \end{macrocode}
%    \end{macro}
%    \begin{macro}{\ltx@LocalPrependToMacro}
%    \begin{macrocode}
\long\def\ltx@LocalPrependToMacro#1#2{%
  \global\let\LTXcmds@gtemp#1%
  \ifx\ltx@undefined\LTXcmds@gtemp
    \global\let\LTXcmds@gtemp\ltx@empty
  \else
    \ifx\relax\LTXcmds@gtemp
      \global\letLTXcmds@gtemp\ltx@empty
    \fi
  \fi
  \begingroup
    \ltx@LocToksA{#2}%
    \ltx@LocToksB\expandafter{\LTXcmds@gtemp}%
    \xdef\LTXcmds@gtemp{\the\ltx@LocToksA\the\ltx@LocToksB}%
  \endgroup
  \let#1\LTXcmds@gtemp
}
%    \end{macrocode}
%    \end{macro}
%
% \subsection{Next character detection}
%
%    \begin{macro}{\ltx@ifnextchar}
%    \begin{macrocode}
\long\def\ltx@ifnextchar#1#2#3{%
  \begingroup
  \let\LTXcmds@CharToken= #1\relax
  \ltx@LocToksA{\endgroup#2}%
  \ltx@LocToksB{\endgroup#3}%
  \futurelet\LTXcmds@LetToken\LTXcmds@ifnextchar
}
%    \end{macrocode}
%    \end{macro}
%    \begin{macro}{\LTXcmds@ifnextchar}
%    \begin{macrocode}
\def\LTXcmds@ifnextchar{%
  \ifx\LTXcmds@LetToken\LTXcmds@CharToken
    \the\expandafter\ltx@LocToksA
  \else
    \expandafter
      \ifx\csname LTXcmds@LetToken\endcsname\LTXcmds@SpaceToken
      \expandafter\expandafter\expandafter\LTXcmds@@ifnextchar
    \else
      \the\expandafter\expandafter\expandafter\ltx@LocToksB
    \fi
  \fi
}
%    \end{macrocode}
%    \end{macro}
%    \begin{macro}{\LTXcmds@@ifnextchar}
%    \cs{futurelet} does not distinguish between a
%    character and a command that is a character
%    (defined by using \cs{let} or \cs{futurelet}).
%    Therefore the space is catched by \cs{romannumeral}
%    with negative character constant that gobbles one
%    optional space.
%    \begin{macrocode}
\def\LTXcmds@@ifnextchar{%
  \expandafter\futurelet
  \expandafter\LTXcmds@LetToken
  \expandafter\LTXcmds@ifnextchar
  \romannumeral-`\.%
}
%    \end{macrocode}
%    \end{macro}
%    \begin{macro}{\LTXcmds@SpaceToken}
%    \begin{macrocode}
\ltx@firstofone{\let\LTXcmds@SpaceToken= } %
%    \end{macrocode}
%    \end{macro}
%
%    \begin{macro}{\ltx@ifnextchar@nospace}
%    \begin{macrocode}
\long\def\ltx@ifnextchar@nospace#1#2#3{%
  \begingroup
  \let\LTXcmds@CharToken= #1\relax
  \ltx@LocToksA{\endgroup#2}%
  \ltx@LocToksB{\endgroup#3}%
  \futurelet\LTXcmds@LetToken\LTXcmds@ifnextchar@nospace
}
%    \end{macrocode}
%    \end{macro}
%    \begin{macro}{\LTXcmds@ifnextchar@nospace}
%    \begin{macrocode}
\def\LTXcmds@ifnextchar@nospace{%
  \the
  \ifx\LTXcmds@LetToken\LTXcmds@CharToken
    \expandafter\ltx@LocToksA
  \else
    \expandafter\ltx@LocToksB
  \fi
}
%    \end{macrocode}
%    \end{macro}
% \subsection{\cs{ltx@leavevmode}, \cs{ltx@mbox}}
%
%    \begin{macro}{\ltx@leavevmode}
%    \begin{macrocode}
\ltx@IfUndefined{quitvmode}{%
  \ltx@IfUndefined{leavevmode}{%
    \ltx@IfUndefined{voidb@x}{%
      \ltx@IfUndefined{newbox}{%
        \def\ltx@leavevmode{%
          \begingroup
            \setbox\ltx@zero=\hbox{}%
            \begingroup
              \setbox\ltx@zero=\hbox{\box\ltx@zero}%
            \endgroup
            \unhbox\ltx@zero
          \endgroup
        }%
      }{%
        \csname newbox\endcsname\LTXcmds@VoidBox
        \ifvoid\LTXcmds@VoidBox
        \else
          \setbox\LTXcmds@VoidBox=\hbox{}%
          \begingroup
            \setbox\LTXcmds@VoidBox=\hbox{\box\LTXcmds@VoidBox}%
          \endgroup
        \fi
        \def\ltx@leavevmode{\unhbox\LTXcmds@VoidBox}%
      }%
    }{%
      \def\ltx@leavevmode{\unhbox\voidb@x}%
    }%
  }{%
    \let\ltx@leavevmode\leavevmode
  }%
}{%
  \let\ltx@leavevmode\quitvmode
}
%    \end{macrocode}
%    \end{macro}
%
%    \begin{macro}{\ltx@mbox}
%    \begin{macrocode}
\def\ltx@mbox{%
  \ltx@leavevmode
  \hbox
}
%    \end{macrocode}
%    \end{macro}
%
% \subsection{Help macros}
%
%    \begin{macro}{\LTXcmds@num}
%    \begin{macrocode}
\ltx@IfUndefined{numexpr}{%
  \def\LTXcmds@num#1{%
    \expandafter\ltx@firstofone\expandafter{%
      \number#1%
    }%
  }%
}{%
  \def\LTXcmds@num#1{%
    \expandafter\ltx@firstofone\expandafter{%
      \the\numexpr#1%
    }%
  }%
}
%    \end{macrocode}
%    \end{macro}
%
% \subsection{Expandable test for emptiness}
%
%    \begin{macrocode}
\ltx@IfUndefined{detokenize}{%
%    \end{macrocode}
%
% \subsubsection{Vanilla \hologo{TeX}}
%
%    \begin{macro}{\ltx@ifempty}
%    The macro is based on \cs{@ifempty} of Robert R. Schneck
%    \cite{ifempty:schneck}
%    and \cs{@ifnull} of Ulrich Diez \cite{ifnull:diez}.
%    There are three cases to consider:
%    \begin{compactenum}
%    \item |#1| is empty,
%    \item |#1| is not empty and the first token
%      is not a begingroup character,
%    \item |#1| starts with a begingroup character (catcode 1).
%    \end{compactenum}
%    \begin{macrocode}
  \def\LTXcmds@temp#1{%
    \long\def\ltx@ifempty##1{%
      \romannumeral0%
      \iffalse{\fi
        \expandafter\ltx@gobble\expandafter{%
          \expandafter{\string##1}%
          \expandafter\ltx@gobble\string
        }%
        \expandafter\ltx@firstofthree\expandafter
        {\iffalse}\fi
        \expandafter#1\ltx@secondoftwo
      }%
      \expandafter#1\ltx@firstoftwo
    }%
%    \end{macrocode}
%    \end{macro}
%    \begin{macro}{\ltx@ifblank}
%    \begin{macrocode}
    \long\def\ltx@ifblank##1{%
      \romannumeral0%
      \iffalse{\fi
        \expandafter\expandafter\expandafter\ltx@gobble
        \expandafter\expandafter\expandafter{%
          \expandafter\expandafter\expandafter{%
            \expandafter\string\ltx@gobble##1.%
          }%
          \expandafter\ltx@gobble\string
        }%
        \expandafter\ltx@firstofthree\expandafter
        {\iffalse}\fi
        \expandafter#1\ltx@secondoftwo
      }%
      \expandafter#1\ltx@firstoftwo
    }%
  }%
  \LTXcmds@temp{ }%
%    \end{macrocode}
%    \end{macro}
%    \begin{macrocode}
}{%
%    \end{macrocode}
%
% \subsubsection{With \cs{detokenize}}
%
%    Ahmed Musa provided \cs{ifstrempty} using \cs{detokenize}
%    and \cs{pdfstrcmp} \cite{ifstrempty:musa}. Ulrich Diez, GL,
%    Heiko Oberdiek
%    improved it further by removing \cs{pdfstrcmp} and
%    taking three arguments
%    \cite{ifstrempty:diez1,ifstrempty:diez2,ifstrempty:diez3,^^A
%      ifstrempty:gl,ifstrempty:ho}.
%
%    \begin{macro}{\ltx@ifempty}
%    \begin{macrocode}
  \long\def\ltx@ifempty#1{%
    \romannumeral%
    \csname
      LTXcmds@ifempty%
      \ifcat$\detokenize{#1}$%
        @%
      \fi
    \endcsname
  }%
%    \end{macrocode}
%    \end{macro}
%    \begin{macro}{\LTXcmds@ifempty@}
%    \begin{macrocode}
  \long\def\LTXcmds@ifempty@#1#2{0 #1}%
%    \end{macrocode}
%    \end{macro}
%    \begin{macro}{\LTXcmds@ifempty}
%    \begin{macrocode}
  \long\def\LTXcmds@ifempty#1#2{0 #2}%
%    \end{macrocode}
%    \end{macro}
%
% \subsubsection{\cs{ltx@ifblank}}
%
%    \begin{macro}{\ltx@ifblank}
%    \begin{macrocode}
  \long\def\ltx@ifblank#1{%
    \romannumeral%
    \csname
      LTXcmds@ifempty%
      \ifcat$\detokenize\expandafter{\ltx@gobble#1.}$%
        @%
      \fi
    \endcsname
  }%
%    \end{macrocode}
%    \end{macro}
%
%    \begin{macrocode}
}
%    \end{macrocode}
%
% \subsection{\cs{ltx@zapspace}}
%
%    \begin{macro}{\ltx@zapspace}
%    \begin{macrocode}
\long\def\ltx@zapspace#1{%
  \romannumeral
  \LTXcmds@zapspace\ltx@zero#1 \@nil
}
%    \end{macrocode}
%    \end{macro}
%    \begin{macro}{\LTXcmds@zapspace}
%    \begin{macrocode}
\long\def\LTXcmds@zapspace#1 #2\@nil{%
  \ltx@ifempty{#2}{%
    #1%
  }{%
    \LTXcmds@zapspace#1#2\@nil
  }%
}
%    \end{macrocode}
%    \end{macro}
%
% \subsection{\cs{ltx@IfBoxEmpty}}
%
%    In case of \hologo{eTeX} the test for an empty box
%    is done via \cs{lastnodetype} as suggested
%    by David Kastrup \cite{ifboxempty:dak}.
%    \begin{macrocode}
\ltx@IfUndefined{lastnodetype}{%
  \catcode`\$=9 %
  \catcode`\&=14 %
}{%
  \catcode`\$=14 %
  \catcode`\&=9 %
}
%    \end{macrocode}
%    \begin{macro}{\ltx@IfBoxEmpty}
%    \begin{macrocode}
\def\ltx@IfBoxEmpty#1{%
  \ifvoid#1\relax
    \expandafter\ltx@secondoftwo
  \else
%    \end{macrocode}
%    Implementation using \hologo{eTeX}'s \cs{lastnodetype}.
%    \begin{macrocode}
&   \begingroup
&     \setbox\ltx@zero=\ifhbox#1\hbox\else\vbox\fi{%
&       \ifhmode\unhcopy\else\unvcopy\fi#1\relax
&       \expandafter
&     }%
&   \expandafter\endgroup
&   \ifnum\lastnodetype<\ltx@zero
&     \expandafter\expandafter\expandafter\ltx@firstoftwo
&   \else
&     \expandafter\expandafter\expandafter\ltx@secondoftwo
&   \fi
%    \end{macrocode}
%    Implementation without \hologo{eTeX} using a signature
%    at the beginning of the test box.
%    \begin{macrocode}
$   \begingroup
$     \setbox\ltx@zero=\ifhbox#1\hbox\else\vbox\fi{%
$       \penalty\ltx@one
$       \ifhmode\unhcopy\else\unvcopy\fi#1\relax
$       \expandafter
$     }%
$     \ifnum\lastpenalty=\ltx@one
%    \end{macrocode}
%    Box 0 has been changed and is restored by
%    closing the group.
%    \begin{macrocode}
$       \endgroup
$       \begingroup
$       \setbox\ltx@zero=\ifhbox#1\hbox\else\vbox\fi{%
$         \penalty\ltx@two
$         \ifhmode\unhcopy\else\unvcopy\fi#1\relax
$         \expandafter
$       }%
$       \ifnum\lastpenalty=\ltx@two
$         \def\next{\endgroup\expandafter\ltx@firstoftwo}%
$       \else
$         \def\next{\endgroup\expandafter\ltx@secondoftwo}%
$       \fi
$     \else
$       \def\next{\endgroup\expandafter\ltx@secondoftwo}%
$     \fi
$   \next
  \fi
}
%    \end{macrocode}
%    \end{macro}
%
%    \begin{macro}{\ltx@IfBoxVoidOrEmpty}
%    \begin{macrocode}
\def\ltx@IfBoxVoidOrEmpty#1{%
  \ifvoid#1\relax
    \expandafter\ltx@thirdoffour
  \fi
  \ltx@IfBoxEmpty{#1}%
}
%    \end{macrocode}
%    \end{macro}
%
%    \begin{macrocode}
\LTXcmds@AtEnd%
%</package>
%    \end{macrocode}
%% \section{Installation}
%
% \subsection{Download}
%
% \paragraph{Package.} This package is available on
% CTAN\footnote{\CTANpkg{ltxcmds}}:
% \begin{description}
% \item[\CTAN{macros/latex/contrib/ltxcmds/ltxcmds.dtx}] The source file.
% \item[\CTAN{macros/latex/contrib/ltxcmds/ltxcmds.pdf}] Documentation.
% \end{description}
%
%
% \paragraph{Bundle.} All the packages of the bundle `ltxcmds'
% are also available in a TDS compliant ZIP archive. There
% the packages are already unpacked and the documentation files
% are generated. The files and directories obey the TDS standard.
% \begin{description}
% \item[\CTANinstall{install/macros/latex/contrib/ltxcmds.tds.zip}]
% \end{description}
% \emph{TDS} refers to the standard ``A Directory Structure
% for \TeX\ Files'' (\CTANpkg{tds}). Directories
% with \xfile{texmf} in their name are usually organized this way.
%
% \subsection{Bundle installation}
%
% \paragraph{Unpacking.} Unpack the \xfile{ltxcmds.tds.zip} in the
% TDS tree (also known as \xfile{texmf} tree) of your choice.
% Example (linux):
% \begin{quote}
%   |unzip ltxcmds.tds.zip -d ~/texmf|
% \end{quote}
%
% \subsection{Package installation}
%
% \paragraph{Unpacking.} The \xfile{.dtx} file is a self-extracting
% \docstrip\ archive. The files are extracted by running the
% \xfile{.dtx} through \plainTeX:
% \begin{quote}
%   \verb|tex ltxcmds.dtx|
% \end{quote}
%
% \paragraph{TDS.} Now the different files must be moved into
% the different directories in your installation TDS tree
% (also known as \xfile{texmf} tree):
% \begin{quote}
% \def\t{^^A
% \begin{tabular}{@{}>{\ttfamily}l@{ $\rightarrow$ }>{\ttfamily}l@{}}
%   ltxcmds.sty & tex/generic/ltxcmds/ltxcmds.sty\\
%   ltxcmds.pdf & doc/latex/ltxcmds/ltxcmds.pdf\\
%   ltxcmds.dtx & source/latex/ltxcmds/ltxcmds.dtx\\
% \end{tabular}^^A
% }^^A
% \sbox0{\t}^^A
% \ifdim\wd0>\linewidth
%   \begingroup
%     \advance\linewidth by\leftmargin
%     \advance\linewidth by\rightmargin
%   \edef\x{\endgroup
%     \def\noexpand\lw{\the\linewidth}^^A
%   }\x
%   \def\lwbox{^^A
%     \leavevmode
%     \hbox to \linewidth{^^A
%       \kern-\leftmargin\relax
%       \hss
%       \usebox0
%       \hss
%       \kern-\rightmargin\relax
%     }^^A
%   }^^A
%   \ifdim\wd0>\lw
%     \sbox0{\small\t}^^A
%     \ifdim\wd0>\linewidth
%       \ifdim\wd0>\lw
%         \sbox0{\footnotesize\t}^^A
%         \ifdim\wd0>\linewidth
%           \ifdim\wd0>\lw
%             \sbox0{\scriptsize\t}^^A
%             \ifdim\wd0>\linewidth
%               \ifdim\wd0>\lw
%                 \sbox0{\tiny\t}^^A
%                 \ifdim\wd0>\linewidth
%                   \lwbox
%                 \else
%                   \usebox0
%                 \fi
%               \else
%                 \lwbox
%               \fi
%             \else
%               \usebox0
%             \fi
%           \else
%             \lwbox
%           \fi
%         \else
%           \usebox0
%         \fi
%       \else
%         \lwbox
%       \fi
%     \else
%       \usebox0
%     \fi
%   \else
%     \lwbox
%   \fi
% \else
%   \usebox0
% \fi
% \end{quote}
% If you have a \xfile{docstrip.cfg} that configures and enables \docstrip's
% TDS installing feature, then some files can already be in the right
% place, see the documentation of \docstrip.
%
% \subsection{Refresh file name databases}
%
% If your \TeX~distribution
% (\TeX\,Live, \mikTeX, \dots) relies on file name databases, you must refresh
% these. For example, \TeX\,Live\ users run \verb|texhash| or
% \verb|mktexlsr|.
%
% \subsection{Some details for the interested}
%
% \paragraph{Unpacking with \LaTeX.}
% The \xfile{.dtx} chooses its action depending on the format:
% \begin{description}
% \item[\plainTeX:] Run \docstrip\ and extract the files.
% \item[\LaTeX:] Generate the documentation.
% \end{description}
% If you insist on using \LaTeX\ for \docstrip\ (really,
% \docstrip\ does not need \LaTeX), then inform the autodetect routine
% about your intention:
% \begin{quote}
%   \verb|latex \let\install=y% \iffalse meta-comment
%
% File: ltxcmds.dtx
% Version: 2019/12/15 v1.24
% Info: LaTeX kernel commands for general use
%
% Copyright (C)
%    2009-2011 Heiko Oberdiek
%    2016-2019 Oberdiek Package Support Group
%    https://github.com/ho-tex/ltxcmds/issues
%
% This work may be distributed and/or modified under the
% conditions of the LaTeX Project Public License, either
% version 1.3c of this license or (at your option) any later
% version. This version of this license is in
%    https://www.latex-project.org/lppl/lppl-1-3c.txt
% and the latest version of this license is in
%    https://www.latex-project.org/lppl.txt
% and version 1.3 or later is part of all distributions of
% LaTeX version 2005/12/01 or later.
%
% This work has the LPPL maintenance status "maintained".
%
% The Current Maintainers of this work are
% Heiko Oberdiek and the Oberdiek Package Support Group
% https://github.com/ho-tex/ltxcmds/issues
%
% The Base Interpreter refers to any `TeX-Format',
% because some files are installed in TDS:tex/generic//.
%
% This work consists of the main source file ltxcmds.dtx
% and the derived files
%    ltxcmds.sty, ltxcmds.pdf, ltxcmds.ins, ltxcmds.drv,
%    ltxcmds-test1.tex, ltxcmds-test-gobble.tex,
%    ltxcmds-test-ifempty.tex, ltxcmds-test-zapspace.tex,
%    ltxcmds-test-ifboxempty.tex, ltxcmds-test-nextchar.tex,
%    ltxcmds-test-carcdr.tex.
%
% Distribution:
%    CTAN:macros/latex/contrib/ltxcmds/ltxcmds.dtx
%    CTAN:macros/latex/contrib/ltxcmds/ltxcmds.pdf
%
% Unpacking:
%    (a) If ltxcmds.ins is present:
%           tex ltxcmds.ins
%    (b) Without ltxcmds.ins:
%           tex ltxcmds.dtx
%    (c) If you insist on using LaTeX
%           latex \let\install=y\input{ltxcmds.dtx}
%        (quote the arguments according to the demands of your shell)
%
% Documentation:
%    (a) If ltxcmds.drv is present:
%           latex ltxcmds.drv
%    (b) Without ltxcmds.drv:
%           latex ltxcmds.dtx; ...
%    The class ltxdoc loads the configuration file ltxdoc.cfg
%    if available. Here you can specify further options, e.g.
%    use A4 as paper format:
%       \PassOptionsToClass{a4paper}{article}
%
%    Programm calls to get the documentation (example):
%       pdflatex ltxcmds.dtx
%       makeindex -s gind.ist ltxcmds.idx
%       pdflatex ltxcmds.dtx
%       makeindex -s gind.ist ltxcmds.idx
%       pdflatex ltxcmds.dtx
%
% Installation:
%    TDS:tex/generic/ltxcmds/ltxcmds.sty
%    TDS:doc/latex/ltxcmds/ltxcmds.pdf
%    TDS:source/latex/ltxcmds/ltxcmds.dtx
%
%<*ignore>
\begingroup
  \catcode123=1 %
  \catcode125=2 %
  \def\x{LaTeX2e}%
\expandafter\endgroup
\ifcase 0\ifx\install y1\fi\expandafter
         \ifx\csname processbatchFile\endcsname\relax\else1\fi
         \ifx\fmtname\x\else 1\fi\relax
\else\csname fi\endcsname
%</ignore>
%<*install>
\input docstrip.tex
\Msg{************************************************************************}
\Msg{* Installation}
\Msg{* Package: ltxcmds 2019/12/15 v1.24 LaTeX kernel commands for general use (HO)}
\Msg{************************************************************************}

\keepsilent
\askforoverwritefalse

\let\MetaPrefix\relax
\preamble

This is a generated file.

Project: ltxcmds
Version: 2019/12/15 v1.24

Copyright (C)
   2009-2011 Heiko Oberdiek
   2016-2019 Oberdiek Package Support Group

This work may be distributed and/or modified under the
conditions of the LaTeX Project Public License, either
version 1.3c of this license or (at your option) any later
version. This version of this license is in
   https://www.latex-project.org/lppl/lppl-1-3c.txt
and the latest version of this license is in
   https://www.latex-project.org/lppl.txt
and version 1.3 or later is part of all distributions of
LaTeX version 2005/12/01 or later.

This work has the LPPL maintenance status "maintained".

The Current Maintainers of this work are
Heiko Oberdiek and the Oberdiek Package Support Group
https://github.com/ho-tex/ltxcmds/issues


The Base Interpreter refers to any `TeX-Format',
because some files are installed in TDS:tex/generic//.

This work consists of the main source file ltxcmds.dtx
and the derived files
   ltxcmds.sty, ltxcmds.pdf, ltxcmds.ins, ltxcmds.drv,
   ltxcmds-test1.tex, ltxcmds-test-gobble.tex,
   ltxcmds-test-ifempty.tex, ltxcmds-test-zapspace.tex,
   ltxcmds-test-ifboxempty.tex, ltxcmds-test-nextchar.tex,
   ltxcmds-test-carcdr.tex.

\endpreamble
\let\MetaPrefix\DoubleperCent

\generate{%
  \file{ltxcmds.ins}{\from{ltxcmds.dtx}{install}}%
  \file{ltxcmds.drv}{\from{ltxcmds.dtx}{driver}}%
  \usedir{tex/generic/ltxcmds}%
  \file{ltxcmds.sty}{\from{ltxcmds.dtx}{package}}%
}

\catcode32=13\relax% active space
\let =\space%
\Msg{************************************************************************}
\Msg{*}
\Msg{* To finish the installation you have to move the following}
\Msg{* file into a directory searched by TeX:}
\Msg{*}
\Msg{*     ltxcmds.sty}
\Msg{*}
\Msg{* To produce the documentation run the file `ltxcmds.drv'}
\Msg{* through LaTeX.}
\Msg{*}
\Msg{* Happy TeXing!}
\Msg{*}
\Msg{************************************************************************}

\endbatchfile
%</install>
%<*ignore>
\fi
%</ignore>
%<*driver>
\NeedsTeXFormat{LaTeX2e}
\ProvidesFile{ltxcmds.drv}%
  [2019/12/15 v1.24 LaTeX kernel commands for general use (HO)]%
\documentclass{ltxdoc}
\usepackage{holtxdoc}[2011/11/22]
\usepackage{zref-savepos}
\usepackage{paralist}
\hfuzz=1pt
\begin{document}
  \DocInput{ltxcmds.dtx}%
\end{document}
%</driver>
% \fi
%
%
%
% \GetFileInfo{ltxcmds.drv}
%
% \title{The \xpackage{ltxcmds} package}
% \date{2019/12/15 v1.24}
% \author{Heiko Oberdiek\thanks
% {Please report any issues at \url{https://github.com/ho-tex/ltxcmds/issues}}}
%
% \maketitle
%
% \begin{abstract}
% The package \xpackage{ltxcmds} exports some utility macros
% from the \LaTeX\ kernel into a separate namespace and
% also provides them for other formats such as plain-\TeX.
% \end{abstract}
%
% \tableofcontents
%
% \newcounter{to}
% \renewcommand*{\theto}{to\arabic{to}}
% \newlength\tolength
% \tolength=\dimexpr .5\linewidth+\oddsidemargin+1in+15mm\relax
% \newcommand*{\TO}{^^A
%   \quad
%   \stepcounter{to}^^A
%   \zsavepos\theto
%   \ifdim\zposx\theto sp>0pt
%     \ifdim\zposx\theto sp<\tolength
%       \kern\dimexpr\tolength-\zposx\theto sp\relax
%     \fi
%   \fi
%   $\rightarrow$\quad
%   \ignorespaces
% }
%
% \section{Documentation}
%
% \subsection{Introduction}
%
% Many of my packages also support other formats such as plain-\TeX.
% Because I am rather familiar with the utility macros from
% \LaTeX's kernel (e.g. \cs{@gobble}, \cs{@firstoftwo}), I found
% myself rewriting them again and again, because they are lacking
% in plain-\TeX.
%
% Therefore this package provides often used macros and similar
% ones with the name prefix \cs{ltx@}. This avoids also faulty
% redefinitions. I remember an example where a package redefined
% \cs{@firstoftwo} with forgetting \cs{long}.
%
% \subsection{Numbers}
%
% \begin{declcs}{ltx@zero} \TO 0\\
%   \cs{ltx@one} \TO 1\\
%   \cs{ltx@two} \TO 2\\
%   \cs{ltx@cclv} \TO 255\\
%   \cs{ltx@minusone} \TO -1
% \end{declcs}
% These commands are numbers 0, 1, 2, 255 and -1. They are not digits
% and a space is not gobbled afterwards. Macro \cs{ltx@minusone}
% is available since version 2010/12/12 v1.15.
%
% \subsection{Scratch registers}
%
% Following the conventions of \hologo{plainTeX} and \hologo{LaTeX}
% the first ten registers are free to use. Even numbered registers
% are for local, odd numbered for global use.
%
% \begin{declcs}{ltx@(Loc,Glob)(Toks,Dimen,Skip)(A,B,C,D,E)}
% \end{declcs}
% The name consists of the prefix \cs{ltx@}, then
% \texttt{Loc} or \texttt{Glob} for local or global usage follows.
% The register type is given by \texttt{Toks} for token register,
% \texttt{Dimen} for dimen register and \texttt{Skip} for skip
% register. As last part the registers are numbered from \texttt{A}
% to \texttt{E}. Example: \cs{ltx@LocToksA}.
%
% Since 2011/04/14 v1.19.
%
% \subsection{Argument killers}
%
% \begin{declcs}{ltx@gobble} \M{1} \TO\\
%   \cs{ltx@gobbletwo} \M{1} \M{2} \TO\\
%   \cs{ltx@gobblethree} \M{1} \M{2} \M{3} \TO\\
%   \cs{ltx@gobblefour} \M{1} \M{2} \M{3} \M{4} \TO
% \end{declcs}
%
% \begin{declcs}{ltx@GobbleNum} \M{num} \M{1} \M{2} \dots\space
%   \M{\meta{num}} \TO
% \end{declcs}
% The first argument \meta{num} of macro \cs{ltx@GobbleNum} specifies,
% how many following arguments are eaten. Macro \cs{ltx@GobbleNum}
% is expandable in exact two expansion steps.
%
% \subsection{Argument grabbers}
%
% \begin{declcs}{ltx@firstofone} \M{1} \TO \meta{1}\\[1ex]
%  \cs{ltx@firstoftwo} \M{1} \M{2} \TO \meta{1}\\
%  \cs{ltx@secondoftwo} \M{1} \M{2} \TO \meta{2}\\[1ex]
%  \cs{ltx@firstofthree} \M{1} \M{2} \M{3} \TO \meta{1}\\
%  \cs{ltx@secondofthree} \M{1} \M{2} \M{3} \TO \meta{2}\\
%  \cs{ltx@thirdofthree} \M{1} \M{2} \M{3} \TO \meta{3}\\
%  \cs{ltx@firstoffour} \M{1} \M{2} \M{3} \M{4} \TO \meta{1}\\
%  \cs{ltx@secondoffour} \M{1} \M{2} \M{3} \M{4} \TO \meta{2}\\
%  \cs{ltx@thirdoffour} \M{1} \M{2} \M{3} \M{4} \TO \meta{3}\\
%  \cs{ltx@fourthoffour} \M{1} \M{2} \M{3} \M{4} \TO \meta{4}
% \end{declcs}
% Macros \cs{ltx@firstofthree}, \cs{ltx@secondofthree}
% and \cs{ltx@thirdofthree} were added in version
% 2010/11/12 v1.11.
% Macros \cs{ltx@firstoffour}, \dots, \cs{ltx@fourthoffour}
% were added in version 2011/02/04 v1.16.
%
% \subsection{List helpers}
%
% \begin{declcs}{ltx@carzero} \dots\ \cs{@nil} \TO\\
%   \cs{ltx@cdrzero} \dots\ \cs{@nil} \TO \dots
% \end{declcs}
%
% \begin{declcs}{ltx@car} \M{1} \dots\ \cs{@nil} \TO \meta{1}\\
%   \cs{ltx@cdr} \M{1} \dots\ \cs{@nil} \TO \dots
% \end{declcs}
%
% \begin{declcs}{ltx@cartwo} \M{1} \M{2} \dots\ \cs{@nil}
%   \TO \meta{1}\meta{2}\\
%   \cs{ltx@carsecond} \M{1} \M{2} \dots\ \cs{@nil} \TO \meta{2}\\
%   \cs{ltx@cdrtwo} \M{1} \M{2} \dots\ \cs{@nil} \TO \dots
% \end{declcs}
%
% \begin{declcs}{ltx@carthree} \M{1} \M{2} \M{3} \dots\ \cs{@nil}
%       \TO \meta{1}\meta{2}\meta{3}\\
%   \cs{ltx@carthird} \M{1} \M{2} \M{3} \dots\ \cs{@nil}
%       \TO \meta{3}\\
%   \cs{ltx@cdrthree} \M{1} \M{2} \M{3} \dots\ \cs{@nil} \TO \dots
% \end{declcs}
%
% \begin{declcs}{ltx@carfour} \M{1} \M{2} \M{3} \M{4} \dots\ \cs{@nil}
%       \TO \meta{1}\meta{2}\meta{3}\meta{4}\\
%   \cs{ltx@carfourth} \M{1} \M{2} \M{3} \M{4} \dots\ \cs{@nil}
%       \TO \meta{4}\\
%   \cs{ltx@cdrfour} \M{1} \M{2} \M{3} \M{4} \dots\ \cs{@nil} \TO \dots
% \end{declcs}
%
% \begin{declcs}{ltx@CarNum} \M{num} \M{1} \dots\ \M{\meta{num}}
%   \M{\meta{num}+1} \dots\ \cs{@nil}\\
%   \TO \M{1} \dots\ \M{\meta{num}} \dots\\
%   \cs{ltx@CarNumth} \M{num} \M{1} \dots\ \M{\meta{num}}
%   \M{\meta{num}+1} \dots\ \cs{@nil}\\
%   \TO \M{\meta{num}} \dots\\
%   \cs{ltx@CdrNum} \M{num} \M{1} \dots\ \M{\meta{num}}
%   \M{\meta{num}+1} \dots\ \cs{@nil}\\
%   \TO \M{\meta{num}+1} \dots
% \end{declcs}
%
% Macros with uppercase letters are expandable in two expansion steps.
% Changes in version 2019/12/15 v1.24:
% \begin{itemize}
% \item
%   Macros \cs{ltx@carsecond}, \cs{ltx@carthird}, \cs{ltx@carfourth},
%   \cs{ltx@CarNumth} added.
% \item
%   Macros \cs{ltx@cdr}, \cs{ltx@cdrtwo}, \cs{ltx@cdrthree}, \cs{ltx@cdrfour},
%   \cs{ltx@CdrNum} are expandable in two expansion steps and retain
%   spaces and braces after the first gobbled arguments.
% \end{itemize}
%
% \subsection{Tail recursion}
%
% \begin{declcs}{ltx@ReturnAfterFi} \M{1} \cs{fi} \TO \cs{fi} \meta{1}\\
%   \cs{ltx@ReturnAfterElseFi} \M{1} \cs{else} \M{2} \cs{fi}
%     \TO \cs{fi} \meta{1}
% \end{declcs}
%
% \subsection{Empty macro}
%
% \begin{declcs}{ltx@empty} \TO
% \end{declcs}
%
% \subsection{Characters}
%
% \begin{declcs}{ltx@space} \TO \verb*| |\\
%   \cs{ltx@percentchar} \TO \texttt{\%}\\
%   \cs{ltx@backslashchar} \TO
%      \makeatletter\texttt{\@backslashchar}\\
%   \cs{ltx@hashchar} \TO \texttt{\#}\qquad(since v1.7)\\
%   \cs{ltx@leftbracechar} \TO \texttt{\textbraceleft}\qquad(since v1.8)\\
%   \cs{ltx@rightbracechar} \TO \texttt{\textbraceright}\qquad(since v1.8)
% \end{declcs}
%
% \subsection{Boolean switch}
%
% \begin{declcs}{ltx@newif} \M{cmd}
% \end{declcs}
% \cs{ltx@newif} defines a new boolean switch \meta{cmd} like \cs{newif}.
% Unlike \plainTeX's \cs{newif}, \cs{ltx@newif} is not \cs{outer}.
% The command \meta{cmd} must start with the two characters |if|.
%
% \begin{declcs}{ltx@newglobalif} \M{cmd}
% \end{declcs}
% \cs{ltx@newglobalif} defines a new boolean switch \meta{cmd}
% like \cs{ltx@newif}. However the switch setting commands,
% \meta{cmd} without the prefix |if| and followed by |true| or |false|
% are acting globally.
%
% \subsection{Command definitions}
%
% \begin{declcs}{ltx@ifundefined} \M{cmd} \M{yes} \M{no}
% \end{declcs}
% If \eTeX\ is available, \cs{ifcsname} is used that does
% not have the side effect of defining undefined commands
% with meaning of \cs{relax}. This command is always expandable.
% Change in version 1.1: Also the meaning \cs{relax} is
% always considered ``undefined''.
%
% \begin{declcs}{ltx@IfUndefined} \M{cmd} \M{yes} \M{no}
% \end{declcs}
% If \eTeX\ is available, \cs{ifcsname} is used that does
% not have the side effect of defining undefined commands
% with meaning of \cs{relax}. Also it always checks for
% the meaning of \cs{relax} and considers this as undefined.
% This macro is not expandable without \eTeX.
%
% \begin{declcs}{ltx@LocalExpandAfter}
% \end{declcs}
% It expands the token after the next token but in a local
% context. That is the difference to \cs{expandafter}.
% The local context discards the side effect of \cs{csname}
% and let the command undefined after the expansion step.
%
% \subsection{Stripping}
%
% \begin{declcs}{ltx@RemovePrefix}\\
%   \cs{ltx@StripPrefix}
% \end{declcs}
% All tokens up to and including the next available character
% `\textgreater' are thrown away. Usually it is used to strip
% the first part of the output of the commands \cs{meaning}
% or \cs{pdflastmatch}. Macro \cs{ltx@RemovePrefix}
% has the same meaning as \LaTeX's \cs{strip@prefix}, whereas
% macro \cs{ltx@StripPrefix} expands the next token
% once before stripping the prefix.
%
% \begin{declcs}{ltx@onelevel@sanitize} \M{macro}
% \end{declcs}
% Macro \cs{ltx@onelevel@sanitize} provides \hologo{LaTeX}'s
% \cs{@onelevel@sanitize}. The macro is expanded once and
% the contents is converted to characters with catcode 12 (other)
% and space tokens with catcode 10 (space). Then then sanitized
% contents is stored into the macro again. Since version 1.12.
%
% \subsection{File management}
%
% All macros in this section are expandable like
% the counterparts of the \LaTeX\ kernel. Also they
% can be used after the preamble.
%
% \subsubsection{File extensions}
%
% \begin{declcs}{ltx@clsextension}\\
%   \cs{ltx@pkgextension}
% \end{declcs}
% Macros \cs{ltx@clsextension} and \cs{ltx@styextension}
% stores the strings |cls| and |sty|. In opposite to \hologo{LaTeX}'s
% \cs{@clsextension} and \cs{@styextension} they can also be
% used after |\begin{document}|.
%
% \subsubsection{Load check}
%
% \begin{declcs}{ltx@ifclassloaded} \M{class} \M{yes} \M{no}\\
%   \cs{ltx@ifpackageloaded} \M{package} \M{yes} \M{no}
% \end{declcs}
% Macros \cs{ltx@ifclassloaded}\slash\cs{ltx@ifpackageloaded}
% execute \meta{yes}, if the \meta{class} or \meta{package} is loaded,
% otherwise \meta{no} is called.
% Both \meta{class} and \meta{package} are specified without
% extension. The macros can also be used after |\begin{document}|.
%
% \begin{declcs}{ltx@iffileloaded} \M{file} \M{yes} \M{no}
% \end{declcs}
% If \LaTeX's \cs{ProvidesFile} macro was called before
% using \meta{file} as argument, then \cs{ltx@iffileloaded}
% calls \meta{yes}, otherwise \meta{no}.
% Therefore it is possible
% that the \meta{file} is loaded, but \meta{no} is executed because
% of a missing \cs{ProvidesFile}. The \LaTeX\ kernel does not have
% a counterpart of \cs{ltx@iffileloaded}.
%
% Note that the file name used in \cs{ProvidesFile} and
% \cs{ltx@iffileloaded} must match. For example, if \TeX's
% default extension \xext{tex} was given in the first command,
% then it must also specified in the latter command and vice versa.
%
% \subsubsection{Version date check}
%
% \begin{declcs}{ltx@ifclasslater} \M{class} \M{date} \M{yes} \M{no}\\
%   \cs{ltx@ifpackagelater} \M{package} \M{date} \M{yes} \M{no}\\
%   \cs{ltx@iffilelater} \M{file} \M{date} \M{yes} \M{no}
% \end{declcs}
% If a \cs{ProvidesClass}\slash\cs{ProvidesPackage}\slash
% \cs{ProvidesFile} command with exact the same class\slash package\slash
% file was executed before with an optional
% argument that starts with a \LaTeX\ version date, then
% this version date is compared with the argument \meta{date}.
% If they are equal or if the version date is the later date,
% then \meta{yes} is called. In all other cases \meta{no} is executed.
%
% A \LaTeX\ date has the format |YYYY/MM/DD| with |YYYY| as year
% with four digits, |MM| as month with two digits and |DD| as day
% with two digits. If \pdfTeX's \cs{pdfmatch} is available, then
% it is used to detect the version date, to reject invalid
% date formats and to reject some invalid dates.
% Dates before 1994/01/01 are always invalid, because version
% dates are introduced with \LaTeXe\ in 1994.
%
% \subsection{Macro additions}
%
% \begin{declcs}{ltx@GlobalAppendToMacro} \M{cmd} \M{addition}\\
%   \cs{ltx@LocalAppendToMacro} \M{cmd} \M{addition}
% \end{declcs}
% The \meta{addition} is appended to the parameterless macro \meta{cmd}.
% If \meta{cmd} is undefined or has the meaning \cs{relax}, then
% it will be initialized as empty macro beforehand. Due to a bug \meta{addition}
% must not contain \cs{par} before version 2010/10/25 v1.9.
%
% \begin{declcs}{ltx@GlobalPrependToMacro} \M{cmd} \M{addition}\\
%   \cs{ltx@LocalPrependToMacro} \M{cmd} \M{addition}
% \end{declcs}
% The \meta{addition} is prepended to the parameterless macro \meta{cmd}.
% If \meta{cmd} is undefined or has the meaning \cs{relax}, then
% it will be initialized as empty macro beforehand. The macros
% were added in version 2011/08/22 v1.21.
%
% \subsection{Next character detection}
%
% \begin{declcs}{ltx@ifnextchar} \M{char} \M{yes} \M{no}
% \end{declcs}
% If next character is \meta{char} then \meta{yes} is called,
% otherwise \meta{no}. The character is not removed.
% Spaces are silently removed when looking for \meta{char}
% as \hologo{LaTeX}'s version \cs{kernel@ifnextchar} does.
% But there are also small differences:
% \begin{itemize}
% \item The space can be used as \meta{char}. In this case
%   optional spaces before \meta{char} are not supported of course.
% \item If the optional space is a command that is a character
%   (defined by \cs{let} or \cs{futurelet}), then
%   \cs{kernel@ifnextchar} breaks with an \hologo{TeX} error.
%   \cs{ltx@ifnextchar} silently removes this token as
%   optional space.
% \end{itemize}
% Since 2010/03/01 v1.3.
%
% \begin{declcs}{ltx@ifnextchar@nospace} \M{char} \M{yes} \M{no}
% \end{declcs}
% Macro \cs{ltx@ifnextchar@nospace} behaves like
% macro \cs{ltx@ifnextchar} with the exception that
% optional spaces are not supported before \meta{char}.
% Since 2011/04/14 v1.19.
%
% \subsection{\cs{ltx@leavevmode}, \cs{ltx@mbox}}
%
% \begin{declcs}{ltx@leavevmode}
% \end{declcs}
% Macro \cs{ltx@leavevmode} calls \hologo{pdfTeX}'s \cs{quitvmode}.
% Otherwise \cs{leavevmode} is used and defined if it is necessary.
%
% \begin{declcs}{ltx@mbox}
% \end{declcs}
% Macro \cs{ltx@mbox} reimplements \cs{mbox} with two changes.
% Instead of \cs{leavevmode} it uses \cs{ltx@leavevmode} and
% stops right after \cs{hbox}. Especially it does not grab
% the argument and allows the extended syntax of \cs{hbox}.
%
% \subsection{Expandable test for emptiness}
%
% \begin{declcs}{ltx@ifempty} \M{stuff} \M{yes} \M{no}
% \end{declcs}
% Macro \cs{ltx@ifempty} checks in exact two expansion steps
% whether \meta{stuff}
% is empty or contains token. Depending on the result
% \meta{yes} or \meta{no} is executed. The token in \meta{stuff}
% may contain \cs{par} and unmatched conditionals (\cs{if}, \cs{else},
% \cs{fi}, \dots). Since version 2010/11/12 v1.11.
%
% \begin{declcs}{ltx@ifblank} \M{stuff} \M{yes} \M{no}
% \end{declcs}
% Macro \cs{ltx@ifblank} tests in exact two expansion steps
% if \meta{stuff} is empty or contain only blank spaces.
% In this case argument \meta{yes} is called. If \meta{stuff}
% contains other tokens than spaces then \meta{no} is executed.
% Since version 2010/12/04 v1.13.
%
% \subsection{Stripping spaces}
%
% \begin{declcs}{ltx@zapspace} \M{stuff}
% \end{declcs}
% Macro \cs{ltx@zapspace} strips spaces from \meta{stuff}
% that are not hidden inside curly braces. Like \hologo{LaTeX}'s
% \cs{zap@space} it is expandable. Differences:
% \begin{itemize}
% \item
% Syntax: \cs{zap@space} also expects a space token
% and \cs{@empty} after \meta{stuff}.
% \item
% Macro \cs{ltx@zapspace} is expandable in exact two
% expansion steps.
% \item
% Macro \cs{ltx@zapspace} always retains curly braces.
% \item
% Macro \cs{zap@space} has a bug. It stops stripping spaces
% after a token group in curly braces if the first two tokens
% inside the group are equal.
% \item
% Macro \cs{ltx@zapspace} also works with \cs{par}
% and conditionals (\cs{if}, \cs{else}, \cs{fi}, \dots).
% \end{itemize}
% Macro \cs{ltx@zapspace} is available since version
% 2010/12/07 v1.14.
%
% \subsection{Check for emptiness of boxes}
%
% \begin{declcs}{ltx@IfBoxEmpty} \M{box register number}
%   \M{yes} \M{no}
% \end{declcs}
% Macro \cs{ltx@IfBoxEmpty} calls \meta{yes} if
% the box exists (\cs{ifvoid} returns false) and the box
% does not contain any content. Otherwise if the box
% is void or contains something, then \meta{no} is executed.
% Thus being empty means that the box exists and is either
% an \cs{hbox} or a \cs{vbox} and may even have dimensions other than
% 0.0\,pt, but the box does not contain anything.
% Macro \cs{ltx@IfBoxEmpty} is available since 2010/02/04 v1.16.
%
% \begin{declcs}{ltx@IfBoxVoidOrEmpty} \M{box register number}
%   \M{yes} \M{no}
% \end{declcs}
% Macro \cs{ltx@IfBoxVoidOrEmpty} calls \meta{yes} if
% the box is either void or does not contain any content.
% Otherwise \meta{no} is executed.
% Macro \cs{ltx@IfBoxVoidOrEmpty} is available since 2010/02/04 v1.16.
%
% \StopEventually{
% }
%
% \section{Implementation}
%
% \subsection{Identification}
%
%    \begin{macrocode}
%<*package>
%    \end{macrocode}
%    Reload check, especially if the package is not used with \LaTeX.
%    \begin{macrocode}
\begingroup\catcode61\catcode48\catcode32=10\relax%
  \catcode13=5 % ^^M
  \endlinechar=13 %
  \catcode35=6 % #
  \catcode39=12 % '
  \catcode44=12 % ,
  \catcode45=12 % -
  \catcode46=12 % .
  \catcode58=12 % :
  \catcode64=11 % @
  \catcode123=1 % {
  \catcode125=2 % }
  \expandafter\let\expandafter\x\csname ver@ltxcmds.sty\endcsname
  \ifx\x\relax % plain-TeX, first loading
  \else
    \def\empty{}%
    \ifx\x\empty % LaTeX, first loading,
      % variable is initialized, but \ProvidesPackage not yet seen
    \else
      \expandafter\ifx\csname PackageInfo\endcsname\relax
        \def\x#1#2{%
          \immediate\write-1{Package #1 Info: #2.}%
        }%
      \else
        \def\x#1#2{\PackageInfo{#1}{#2, stopped}}%
      \fi
      \x{ltxcmds}{The package is already loaded}%
      \aftergroup\endinput
    \fi
  \fi
\endgroup%
%    \end{macrocode}
%    Package identification:
%    \begin{macrocode}
\begingroup\catcode61\catcode48\catcode32=10\relax%
  \catcode13=5 % ^^M
  \endlinechar=13 %
  \catcode35=6 % #
  \catcode39=12 % '
  \catcode40=12 % (
  \catcode41=12 % )
  \catcode44=12 % ,
  \catcode45=12 % -
  \catcode46=12 % .
  \catcode47=12 % /
  \catcode58=12 % :
  \catcode64=11 % @
  \catcode91=12 % [
  \catcode93=12 % ]
  \catcode123=1 % {
  \catcode125=2 % }
  \expandafter\ifx\csname ProvidesPackage\endcsname\relax
    \def\x#1#2#3[#4]{\endgroup
      \immediate\write-1{Package: #3 #4}%
      \xdef#1{#4}%
    }%
  \else
    \def\x#1#2[#3]{\endgroup
      #2[{#3}]%
      \ifx#1\@undefined
        \xdef#1{#3}%
      \fi
      \ifx#1\relax
        \xdef#1{#3}%
      \fi
    }%
  \fi
\expandafter\x\csname ver@ltxcmds.sty\endcsname
\ProvidesPackage{ltxcmds}%
  [2019/12/15 v1.24 LaTeX kernel commands for general use (HO)]%
%    \end{macrocode}
%
%    \begin{macrocode}
\begingroup\catcode61\catcode48\catcode32=10\relax%
  \catcode13=5 % ^^M
  \endlinechar=13 %
  \catcode123=1 % {
  \catcode125=2 % }
  \catcode64=11 % @
  \def\x{\endgroup
    \expandafter\edef\csname LTXcmds@AtEnd\endcsname{%
      \endlinechar=\the\endlinechar\relax
      \catcode13=\the\catcode13\relax
      \catcode32=\the\catcode32\relax
      \catcode35=\the\catcode35\relax
      \catcode61=\the\catcode61\relax
      \catcode64=\the\catcode64\relax
      \catcode123=\the\catcode123\relax
      \catcode125=\the\catcode125\relax
    }%
  }%
\x\catcode61\catcode48\catcode32=10\relax%
\catcode13=5 % ^^M
\endlinechar=13 %
\catcode35=6 % #
\catcode64=11 % @
\catcode123=1 % {
\catcode125=2 % }
\def\TMP@EnsureCode#1#2{%
  \edef\LTXcmds@AtEnd{%
    \LTXcmds@AtEnd
    \catcode#1=\the\catcode#1\relax
  }%
  \catcode#1=#2\relax
}
\TMP@EnsureCode{36}{3}% $
\TMP@EnsureCode{38}{4}% &
\TMP@EnsureCode{40}{12}% (
\TMP@EnsureCode{41}{12}% )
\TMP@EnsureCode{45}{12}% -
\TMP@EnsureCode{46}{12}% .
\TMP@EnsureCode{47}{12}% /
\TMP@EnsureCode{60}{12}% <
\TMP@EnsureCode{62}{12}% >
\TMP@EnsureCode{91}{12}% [
\TMP@EnsureCode{96}{12}% `
\TMP@EnsureCode{93}{12}% ]
\TMP@EnsureCode{94}{12}% ^ (superscript) (!)
\TMP@EnsureCode{124}{12}% |
\edef\LTXcmds@AtEnd{\LTXcmds@AtEnd\noexpand\endinput}
%    \end{macrocode}
%
% \subsection{Numbers}
%
%    \begin{macro}{\ltx@zero}
%    \begin{macrocode}
\chardef\ltx@zero=0 %
%    \end{macrocode}
%    \end{macro}
%    \begin{macro}{\ltx@one}
%    \begin{macrocode}
\chardef\ltx@one=1 %
%    \end{macrocode}
%    \end{macro}
%    \begin{macro}{\ltx@two}
%    \begin{macrocode}
\chardef\ltx@two=2 %
%    \end{macrocode}
%    \end{macro}
%    \begin{macro}{\ltx@active}
%    \begin{macrocode}
\chardef\ltx@active=13 %
%    \end{macrocode}
%    \end{macro}
%    \begin{macro}{\ltx@cclv}
%    \begin{macrocode}
\chardef\ltx@cclv=255 %
%    \end{macrocode}
%    \end{macro}
%
%    \begin{macro}{\ltx@minusone}
%    \begin{macrocode}
\def\ltx@minusone{%
  -\ltx@one
}
%    \end{macrocode}
%    \end{macro}
%
% \subsection{Scratch registers}
%
%    \begin{macro}{\ltx@LocToksA}
%    \begin{macrocode}
\toksdef\ltx@LocToksA=0 %
%    \end{macrocode}
%    \end{macro}
%    \begin{macro}{\ltx@LocToksB}
%    \begin{macrocode}
\toksdef\ltx@LocToksB=2 %
%    \end{macrocode}
%    \end{macro}
%    \begin{macro}{\ltx@LocToksC}
%    \begin{macrocode}
\toksdef\ltx@LocToksC=4 %
%    \end{macrocode}
%    \end{macro}
%    \begin{macro}{\ltx@LocToksD}
%    \begin{macrocode}
\toksdef\ltx@LocToksD=6 %
%    \end{macrocode}
%    \end{macro}
%    \begin{macro}{\ltx@LocToksE}
%    \begin{macrocode}
\toksdef\ltx@LocToksE=8 %
%    \end{macrocode}
%    \end{macro}
%
%    \begin{macro}{\ltx@GlobToksA}
%    \begin{macrocode}
\toksdef\ltx@GlobToksA=1 %
%    \end{macrocode}
%    \end{macro}
%    \begin{macro}{\ltx@GlobToksB}
%    \begin{macrocode}
\toksdef\ltx@GlobToksB=3 %
%    \end{macrocode}
%    \end{macro}
%    \begin{macro}{\ltx@GlobToksC}
%    \begin{macrocode}
\toksdef\ltx@GlobToksC=5 %
%    \end{macrocode}
%    \end{macro}
%    \begin{macro}{\ltx@GlobToksD}
%    \begin{macrocode}
\toksdef\ltx@GlobToksD=7 %
%    \end{macrocode}
%    \end{macro}
%    \begin{macro}{\ltx@GlobToksE}
%    \begin{macrocode}
\toksdef\ltx@GlobToksE=9 %
%    \end{macrocode}
%    \end{macro}
%
%    \begin{macro}{\ltx@LocDimenA}
%    \begin{macrocode}
\dimendef\ltx@LocDimenA=0 %
%    \end{macrocode}
%    \end{macro}
%    \begin{macro}{\ltx@LocDimenB}
%    \begin{macrocode}
\dimendef\ltx@LocDimenB=2 %
%    \end{macrocode}
%    \end{macro}
%    \begin{macro}{\ltx@LocDimenC}
%    \begin{macrocode}
\dimendef\ltx@LocDimenC=4 %
%    \end{macrocode}
%    \end{macro}
%    \begin{macro}{\ltx@LocDimenD}
%    \begin{macrocode}
\dimendef\ltx@LocDimenD=6 %
%    \end{macrocode}
%    \end{macro}
%    \begin{macro}{\ltx@LocDimenE}
%    \begin{macrocode}
\dimendef\ltx@LocDimenE=8 %
%    \end{macrocode}
%    \end{macro}
%
%    \begin{macro}{\ltx@GlobDimenA}
%    \begin{macrocode}
\dimendef\ltx@GlobDimenA=1 %
%    \end{macrocode}
%    \end{macro}
%    \begin{macro}{\ltx@GlobDimenB}
%    \begin{macrocode}
\dimendef\ltx@GlobDimenB=3 %
%    \end{macrocode}
%    \end{macro}
%    \begin{macro}{\ltx@GlobDimenC}
%    \begin{macrocode}
\dimendef\ltx@GlobDimenC=5 %
%    \end{macrocode}
%    \end{macro}
%    \begin{macro}{\ltx@GlobDimenD}
%    \begin{macrocode}
\dimendef\ltx@GlobDimenD=7 %
%    \end{macrocode}
%    \end{macro}
%    \begin{macro}{\ltx@GlobDimenE}
%    \begin{macrocode}
\dimendef\ltx@GlobDimenE=9 %
%    \end{macrocode}
%    \end{macro}
%
%    \begin{macro}{\ltx@LocSkipA}
%    \begin{macrocode}
\skipdef\ltx@LocSkipA=0 %
%    \end{macrocode}
%    \end{macro}
%    \begin{macro}{\ltx@LocSkipB}
%    \begin{macrocode}
\skipdef\ltx@LocSkipB=2 %
%    \end{macrocode}
%    \end{macro}
%    \begin{macro}{\ltx@LocSkipC}
%    \begin{macrocode}
\skipdef\ltx@LocSkipC=4 %
%    \end{macrocode}
%    \end{macro}
%    \begin{macro}{\ltx@LocSkipD}
%    \begin{macrocode}
\skipdef\ltx@LocSkipD=6 %
%    \end{macrocode}
%    \end{macro}
%    \begin{macro}{\ltx@LocSkipE}
%    \begin{macrocode}
\skipdef\ltx@LocSkipE=8 %
%    \end{macrocode}
%    \end{macro}
%
%    \begin{macro}{\ltx@GlobSkipA}
%    \begin{macrocode}
\skipdef\ltx@GlobSkipA=1 %
%    \end{macrocode}
%    \end{macro}
%    \begin{macro}{\ltx@GlobSkipB}
%    \begin{macrocode}
\skipdef\ltx@GlobSkipB=3 %
%    \end{macrocode}
%    \end{macro}
%    \begin{macro}{\ltx@GlobSkipC}
%    \begin{macrocode}
\skipdef\ltx@GlobSkipC=5 %
%    \end{macrocode}
%    \end{macro}
%    \begin{macro}{\ltx@GlobSkipD}
%    \begin{macrocode}
\skipdef\ltx@GlobSkipD=7 %
%    \end{macrocode}
%    \end{macro}
%    \begin{macro}{\ltx@GlobSkipE}
%    \begin{macrocode}
\skipdef\ltx@GlobSkipE=9 %
%    \end{macrocode}
%    \end{macro}
%
%
% \subsection{Argument killers}
%
%    \begin{macro}{\ltx@gobble}
%    \begin{macrocode}
\long\def\ltx@gobble#1{}
%    \end{macrocode}
%    \end{macro}
%    \begin{macro}{\ltx@gobbletwo}
%    \begin{macrocode}
\long\def\ltx@gobbletwo#1#2{}
%    \end{macrocode}
%    \end{macro}
%    \begin{macro}{\ltx@gobblethree}
%    \begin{macrocode}
\long\def\ltx@gobblethree#1#2#3{}
%    \end{macrocode}
%    \end{macro}
%    \begin{macro}{\ltx@gobblefour}
%    \begin{macrocode}
\long\def\ltx@gobblefour#1#2#3#4{}
%    \end{macrocode}
%    \end{macro}
%
%    \begin{macro}{\ltx@GobbleNum}
%    \begin{macrocode}
\def\ltx@GobbleNum#1{%
  \romannumeral
  \csname ltx@zero%
  \expandafter\LTXcmds@GobbleNum
  \romannumeral\LTXcmds@num{#1}000{m\endcsname}%
}
%    \end{macrocode}
%    \end{macro}
%    \begin{macro}{\LTXcmds@GobbleNum}
%    \begin{macrocode}
\def\LTXcmds@GobbleNum#1{%
  \csname LTXcmds@G#1\LTXcmds@GobbleNum
}
%    \end{macrocode}
%    \end{macro}
%    \begin{macro}{\LTXcmds@Gm}
%    \begin{macrocode}
\long\def\LTXcmds@Gm#1{%
  \endcsname
}
%    \end{macrocode}
%    \end{macro}
%
% \subsection{Argument grabbers}
%
%    \begin{macro}{\ltx@firstofone}
%    \begin{macrocode}
\long\def\ltx@firstofone#1{#1}
%    \end{macrocode}
%    \end{macro}
%    \begin{macro}{\ltx@firstoftwo}
%    \begin{macrocode}
\long\def\ltx@firstoftwo#1#2{#1}
%    \end{macrocode}
%    \end{macro}
%    \begin{macro}{\ltx@secondoftwo}
%    \begin{macrocode}
\long\def\ltx@secondoftwo#1#2{#2}
%    \end{macrocode}
%    \end{macro}
%
%    \begin{macro}{\ltx@firstofthree}
%    \begin{macrocode}
\long\def\ltx@firstofthree#1#2#3{#1}
%    \end{macrocode}
%    \end{macro}
%    \begin{macro}{\ltx@secondofthree}
%    \begin{macrocode}
\long\def\ltx@secondofthree#1#2#3{#2}
%    \end{macrocode}
%    \end{macro}
%    \begin{macro}{\ltx@thirdofthree}
%    \begin{macrocode}
\long\def\ltx@thirdofthree#1#2#3{#3}%
%    \end{macrocode}
%    \end{macro}
%
%    \begin{macro}{\ltx@firstoffour}
%    \begin{macrocode}
\long\def\ltx@firstoffour#1#2#3#4{#1}
%    \end{macrocode}
%    \end{macro}
%    \begin{macro}{\ltx@secondoffour}
%    \begin{macrocode}
\long\def\ltx@secondoffour#1#2#3#4{#2}
%    \end{macrocode}
%    \end{macro}
%    \begin{macro}{\ltx@thirdoffour}
%    \begin{macrocode}
\long\def\ltx@thirdoffour#1#2#3#4{#3}%
%    \end{macrocode}
%    \end{macro}
%    \begin{macro}{\ltx@fourthoffour}
%    \begin{macrocode}
\long\def\ltx@fourthoffour#1#2#3#4{#4}%
%    \end{macrocode}
%    \end{macro}
%
% \subsection{List helpers}
%
%    \begin{macro}{\ltx@carzero}
%    \begin{macrocode}
\long\def\ltx@carzero#1\@nil{}%
%    \end{macrocode}
%    \end{macro}
%    \begin{macro}{\LTXcmds@cdrzero}
%    \begin{macrocode}
\long\def\LTXcmds@cdrzero#1\@nil{#1}
%    \end{macrocode}
%    \end{macro}
%    \begin{macro}{\ltx@cdrzero}
%    \begin{macrocode}
\def\ltx@cdrzero{%
  \romannumeral\LTXcmds@cdrzero\ltx@zero
}
%    \end{macrocode}
%    \end{macro}
%
%    \begin{macro}{\ltx@car}
%    \begin{macrocode}
\long\def\ltx@car#1#2\@nil{#1}
%    \end{macrocode}
%    \end{macro}
%    \begin{macro}{\ltx@cdr}
%    \begin{macrocode}
\long\def\ltx@cdr#1{%
  \romannumeral\LTXcmds@cdrzero\ltx@zero
}
%    \end{macrocode}
%    \end{macro}
%
%    \begin{macro}{\ltx@cartwo}
%    \begin{macrocode}
\long\def\ltx@cartwo#1#2#3\@nil{#1#2}
%    \end{macrocode}
%    \end{macro}
%    \begin{macro}{\ltx@carsecond}
%    \begin{macrocode}
\long\def\ltx@carsecond#1#2#3\@nil{#2}
%    \end{macrocode}
%    \end{macro}
%    \begin{macro}{\ltx@cdrtwo}
%    \begin{macrocode}
\long\def\ltx@cdrtwo#1#2{%
  \romannumeral\LTXcmds@cdrzero\ltx@zero
}
%    \end{macrocode}
%    \end{macro}
%
%    \begin{macro}{\ltx@carthree}
%    \begin{macrocode}
\long\def\ltx@carthree#1#2#3#4\@nil{#1#2#3}
%    \end{macrocode}
%    \end{macro}
%    \begin{macro}{\ltx@carthird}
%    \begin{macrocode}
\long\def\ltx@carthird#1#2#3#4\@nil{#3}
%    \end{macrocode}
%    \end{macro}
%    \begin{macro}{\ltx@cdrthree}
%    \begin{macrocode}
\long\def\ltx@cdrthree#1#2#3{%
  \romannumeral\LTXcmds@cdrzero\ltx@zero
}
%    \end{macrocode}
%    \end{macro}
%
%    \begin{macro}{\ltx@carfour}
%    \begin{macrocode}
\long\def\ltx@carfour#1#2#3#4#5\@nil{#1#2#3#4}
%    \end{macrocode}
%    \end{macro}
%    \begin{macro}{\ltx@carfourth}
%    \begin{macrocode}
\long\def\ltx@carfourth#1#2#3#4#5\@nil{#4}
%    \end{macrocode}
%    \end{macro}
%    \begin{macro}{\ltx@cdrfour}
%    \begin{macrocode}
\long\def\ltx@cdrfour#1#2#3#4{%
  \romannumeral\LTXcmds@cdrzero\ltx@zero
}
%    \end{macrocode}
%    \end{macro}
%
%    \begin{macro}{\ltx@CarNum}
%    \begin{macrocode}
\def\ltx@CarNum#1{%
  \romannumeral
  \csname LTXcmds@CarNumFinish%
  \expandafter\LTXcmds@CarNum
  \romannumeral\LTXcmds@num{#1}000{x\endcsname}%
}
%    \end{macrocode}
%    \end{macro}
%    \begin{macro}{\LTXcmds@CarNum}
%    \begin{macrocode}
\def\LTXcmds@CarNum#1{%
  \csname LTXcmds@C#1\LTXcmds@CarNum
}
%    \end{macrocode}
%    \end{macro}
%    \begin{macro}{\LTXcmds@Cm}
%    \begin{macrocode}
\long\def\LTXcmds@Cm#1#2{%
  \endcsname{#1#2}%
}
%    \end{macrocode}
%    \end{macro}
%    \begin{macro}{\LTXcmds@Cx}
%    \begin{macrocode}
\def\LTXcmds@Cx#1{%
  \endcsname{}%
}
%    \end{macrocode}
%    \end{macro}
%    \begin{macro}{\LTXcmds@CarNumFinish}
%    \begin{macrocode}
\long\def\LTXcmds@CarNumFinish#1#2\@nil{%
  \ltx@zero
  #1%
}
%    \end{macrocode}
%    \end{macro}
%
%    \begin{macro}{\ltx@CarNumth}
%    \begin{macrocode}
\def\ltx@CarNumth#1{%
  \romannumeral
  \expandafter\expandafter\expandafter
  \LTXcmds@CarNumth
  \ltx@GobbleNum{#1}{}%
}
%    \end{macrocode}
%    \end{macro}
%    \begin{macro}{\LTXcmds@CarNumth}
%    \begin{macrocode}
\long\def\LTXcmds@CarNumth#1#2\@nil{%
  \ltx@zero
  #1%
}
%    \end{macrocode}
%    \end{macro}
%
%    \begin{macro}{\ltx@CdrNum}
%    \begin{macrocode}
\def\ltx@CdrNum#1{%
  \romannumeral%
  \expandafter\expandafter\expandafter\ltx@cdrzero
  \expandafter\expandafter\expandafter\ltx@zero
  \ltx@GobbleNum{#1}%
}
%    \end{macrocode}
%    \end{macro}
%
% \subsection{Tail recursion}
%
%    \begin{macro}{\ltx@ReturnAfterFi}
%    \begin{macrocode}
\long\def\ltx@ReturnAfterFi#1\fi{\fi#1}
%    \end{macrocode}
%    \end{macro}
%    \begin{macro}{\ltx@ReturnAfterElseFi}
%    \begin{macrocode}
\long\def\ltx@ReturnAfterElseFi#1\else#2\fi{\fi#1}
%    \end{macrocode}
%    \end{macro}
%
% \subsection{Empty macro}
%
%    \begin{macro}{\ltx@empty}
%    \begin{macrocode}
\def\ltx@empty{}
%    \end{macrocode}
%    \end{macro}
%
% \subsection{Characters}
%
%    \begin{macro}{\ltx@space}
%    \begin{macrocode}
\def\ltx@space{ }
%    \end{macrocode}
%    \end{macro}
%    \begin{macro}{\ltx@percentchar}
%    \begin{macrocode}
\begingroup
  \lccode`0=`\%\relax
\lowercase{\endgroup
  \def\ltx@percentchar{0}%
}
%    \end{macrocode}
%    \end{macro}
%    \begin{macro}{\ltx@backslashchar}
%    \begin{macrocode}
\begingroup
  \lccode`0=`\\\relax
\lowercase{\endgroup
  \def\ltx@backslashchar{0}%
}
%    \end{macrocode}
%    \end{macro}
%    \begin{macro}{\ltx@hashchar}
%    \begin{macrocode}
\begingroup
  \lccode`0=`\#\relax
\lowercase{\endgroup
  \def\ltx@hashchar{0}%
}
%    \end{macrocode}
%    \end{macro}
%    \begin{macro}{\ltx@leftbracechar}
%    \begin{macrocode}
\begingroup
  \lccode`0=`\{\relax
\lowercase{\endgroup
  \def\ltx@leftbracechar{0}%
}
%    \end{macrocode}
%    \end{macro}
%    \begin{macro}{\ltx@rightbracechar}
%    \begin{macrocode}
\begingroup
  \lccode`0=`\}\relax
\lowercase{\endgroup
  \def\ltx@rightbracechar{0}%
}
%    \end{macrocode}
%    \end{macro}
%
% \subsection{Boolean switch}
%
%    \begin{macro}{\ltx@newif}
%    \begin{macrocode}
\def\ltx@newif#1{%
  \begingroup
    \escapechar=-1 %
  \expandafter\endgroup
  \expandafter\LTXcmds@newif\string#1\@nil
}
%    \end{macrocode}
%    \end{macro}
%    \begin{macro}{\LTXcmds@newif}
%    \begin{macrocode}
\begingroup
  \escapechar=-1 %
\expandafter\endgroup
\expandafter\def\expandafter\LTXcmds@newif\string\if#1\@nil{%
  \expandafter\edef\csname#1true\endcsname{%
    \let
    \expandafter\noexpand\csname if#1\endcsname
    \noexpand\iftrue
  }%
  \expandafter\edef\csname#1false\endcsname{%
    \let
    \expandafter\noexpand\csname if#1\endcsname
    \noexpand\iffalse
  }%
  \csname#1false\endcsname
}
%    \end{macrocode}
%    \end{macro}
%
%    \begin{macro}{\ltx@newglobalif}
%    \begin{macrocode}
\def\ltx@newglobalif#1{%
  \begingroup
    \escapechar=-1 %
  \expandafter\endgroup
  \expandafter\LTXcmds@newglobalif\string#1\@nil
}
%    \end{macrocode}
%    \end{macro}
%    \begin{macro}{\LTXcmds@newglobalif}
%    \begin{macrocode}
\begingroup
  \escapechar=-1 %
\expandafter\endgroup
\expandafter
\def\expandafter\LTXcmds@newglobalif\string\if#1\@nil{%
  \expandafter\edef\csname#1true\endcsname{%
    \global\let
    \expandafter\noexpand\csname if#1\endcsname
    \noexpand\iftrue
  }%
  \expandafter\edef\csname#1false\endcsname{%
    \global\let
    \expandafter\noexpand\csname if#1\endcsname
    \noexpand\iffalse
  }%
  \csname#1false\endcsname
}
%    \end{macrocode}
%    \end{macro}
%
% \subsection{Command definitions}
%
%    \begin{macro}{\ltx@LocalExpandAfter}
%    \begin{macrocode}
\def\ltx@LocalExpandAfter{%
  \begingroup
    \expandafter\expandafter\expandafter
  \endgroup
  \expandafter
}
%    \end{macrocode}
%    \end{macro}
%
%    \begin{macrocode}
\ltx@LocalExpandAfter
\ifx\csname ifcsname\endcsname\relax
%    \end{macrocode}
%    \begin{macro}{\ltx@ifundefined}
%    \begin{macrocode}
  \def\ltx@ifundefined#1{%
    \expandafter\ifx\csname #1\endcsname\relax
      \expandafter\ltx@firstoftwo
    \else
      \expandafter\ltx@secondoftwo
    \fi
  }%
%    \end{macrocode}
%    \end{macro}
%    \begin{macro}{\ltx@IfUndefined}
%    \begin{macrocode}
  \def\ltx@IfUndefined#1{%
    \begingroup\expandafter\expandafter\expandafter\endgroup
    \expandafter\ifx\csname #1\endcsname\relax
      \expandafter\ltx@firstoftwo
    \else
      \expandafter\ltx@secondoftwo
    \fi
  }%
%    \end{macrocode}
%    \end{macro}
%    \begin{macrocode}
  \expandafter\ltx@gobble
\else
  \expandafter\ltx@firstofone
\fi
{%
%    \end{macrocode}
%    \begin{macro}{\ltx@ifundefined}
%    \begin{macrocode}
  \def\ltx@ifundefined#1{%
    \ifcsname #1\endcsname
      \expandafter\ifx\csname #1\endcsname\relax
        \expandafter\expandafter\expandafter\ltx@firstoftwo
      \else
        \expandafter\expandafter\expandafter\ltx@secondoftwo
      \fi
    \else
      \expandafter\ltx@firstoftwo
    \fi
  }%
%    \end{macrocode}
%    \end{macro}
%    \begin{macro}{\ltx@IfUndefined}
%    \begin{macrocode}
  \let\ltx@IfUndefined\ltx@ifundefined
%    \end{macrocode}
%    \end{macro}
%    \begin{macrocode}
}
%    \end{macrocode}
%
% \subsection{Stripping}
%
%    \begin{macro}{\ltx@RemovePrefix}
%    \begin{macrocode}
\def\ltx@RemovePrefix#1>{}
%    \end{macrocode}
%    \end{macro}
%
%    \begin{macro}{\ltx@StripPrefix}
%    \begin{macrocode}
\def\ltx@StripPrefix{%
  \expandafter\ltx@RemovePrefix
}
%    \end{macrocode}
%    \end{macro}
%
%    \begin{macro}{\ltx@onelevel@sanitize}
%    \begin{macrocode}
\def\ltx@onelevel@sanitize#1{%
  \edef#1{%
    \expandafter
    \ltx@RemovePrefix\meaning#1%
  }%
}
%    \end{macrocode}
%    \end{macro}
%
% \subsection{File management}
%
% \subsubsection{File extensions}
%
%    \begin{macro}{\ltx@clsextension}
%    \begin{macrocode}
\def\ltx@clsextension{cls}
%    \end{macrocode}
%    \end{macro}
%    \begin{macro}{\ltx@pkgextension}
%    \begin{macrocode}
\def\ltx@pkgextension{sty}
%    \end{macrocode}
%    \end{macro}
%
% \subsubsection{Load check}
%
%    \begin{macro}{\ltx@iffileloaded}
%    \begin{macrocode}
\def\ltx@iffileloaded#1{%
  \ltx@ifundefined{ver@#1}\ltx@secondoftwo\ltx@firstoftwo
}
%    \end{macrocode}
%    \end{macro}
%    \begin{macro}{\ltx@ifclassloaded}
%    \begin{macrocode}
\def\ltx@ifclassloaded#1{%
  \ltx@iffileloaded{#1.\ltx@clsextension}%
}
%    \end{macrocode}
%    \end{macro}
%    \begin{macro}{\ltx@ifpackageloaded}
%    \begin{macrocode}
\def\ltx@ifpackageloaded#1{%
  \ltx@iffileloaded{#1.\ltx@pkgextension}%
}
%    \end{macrocode}
%    \end{macro}
%
% \subsubsection{Version date check}
%
%    \begin{macro}{\ltx@iffilelater}
%    \begin{macrocode}
\def\ltx@iffilelater#1#2{%
  \ltx@iffileloaded{#1}{%
    \expandafter\LTXcmds@IfLater\expandafter{%
      \number
      \expandafter\expandafter\expandafter\LTXcmds@ParseVersion
      \expandafter\expandafter\expandafter{%
        \csname ver@#1\endcsname
      }%
    \expandafter}\expandafter{%
      \number
      \expandafter\LTXcmds@ParseVersion\expandafter{#2}%
    }%
  }\ltx@secondoftwo
}
%    \end{macrocode}
%    \end{macro}
%    \begin{macro}{\LTXcmds@IfLater}
%    \begin{macrocode}
\def\LTXcmds@IfLater#1#2{%
  \ifcase 0%
      \ifnum#1<19940101 %
      \else
        \ifnum#2<19940101 %
        \else
          \ifnum#2>#1 %
          \else
            1%
          \fi
        \fi
      \fi
      \ltx@space
    \expandafter\ltx@secondoftwo
  \else
    \expandafter\ltx@firstoftwo
  \fi
}
%    \end{macrocode}
%    \end{macro}
%    \begin{macro}{\ltx@ifclasslater}
%    \begin{macrocode}
\def\ltx@ifclasslater#1{%
  \ltx@iffilelater{#1.\ltx@clsextension}%
}
%    \end{macrocode}
%    \end{macro}
%    \begin{macro}{\ltx@ifpackagelater}
%    \begin{macrocode}
\def\ltx@ifpackagelater#1{%
  \ltx@iffilelater{#1.\ltx@pkgextension}%
}
%    \end{macrocode}
%    \end{macro}
%
%    \begin{macrocode}
\ltx@IfUndefined{pdfmatch}{%
%    \end{macrocode}
%    \begin{macro}{\LTXcmds@ParseVersion}
%    \begin{macrocode}
  \def\LTXcmds@ParseVersion#1{%
    \LTXcmds@@ParseVersion#10000/00/00\@nil
  }%
%    \end{macrocode}
%    \end{macro}
%    \begin{macro}{\LTXcmds@@ParseVersion}
%    \begin{macrocode}
  \def\LTXcmds@@ParseVersion#1#2#3#4/#5#6/#7#8#9\@nil{%
    #1#2#3#4#5#6#7#8%
  }%
%    \end{macrocode}
%    \end{macro}
%    \begin{macrocode}
}{%
%    \end{macrocode}
%    \begin{macro}{\LTXcmds@ParseVersion}
%    \begin{macrocode}
  \def\LTXcmds@ParseVersion#1{%
    \ifnum\pdfmatch{%
      ^%
      (199[4-9]|[2-9][0-9][0-9][0-9])/%
      (0[1-9]|1[0-2])/%
      (0[1-9]|[1-2][0-9]|3[0-1])%
    }{#1}=1 %
      \ltx@StripPrefix\pdflastmatch1 %
      \ltx@StripPrefix\pdflastmatch2 %
      \ltx@StripPrefix\pdflastmatch3 %
    \else
      0%
    \fi
  }%
%    \end{macrocode}
%    \end{macro}
%    \begin{macrocode}
}
%    \end{macrocode}
%
% \subsection{Macro additions}
%
%    \begin{macro}{\ltx@GlobalAppendToMacro}
%    \begin{macrocode}
\long\def\ltx@GlobalAppendToMacro#1#2{%
  \ifx\ltx@undefined#1%
    \let#1\ltx@empty
  \else
    \ifx\relax#1%
      \let#1\ltx@empty
    \fi
  \fi
  \begingroup
    \ltx@LocToksA\expandafter{#1#2}%
    \xdef#1{\the\ltx@LocToksA}%
  \endgroup
}
%    \end{macrocode}
%    \end{macro}
%    \begin{macro}{\ltx@LocalAppendToMacro}
%    \begin{macrocode}
\long\def\ltx@LocalAppendToMacro#1#2{%
  \global\let\LTXcmds@gtemp#1%
  \ifx\ltx@undefined\LTXcmds@gtemp
    \global\let\LTXcmds@gtemp\ltx@empty
  \else
    \ifx\relax\LTXcmds@gtemp
      \global\letLTXcmds@gtemp\ltx@empty
    \fi
  \fi
  \begingroup
    \ltx@LocToksA\expandafter{\LTXcmds@gtemp#2}%
    \xdef\LTXcmds@gtemp{\the\ltx@LocToksA}%
  \endgroup
  \let#1\LTXcmds@gtemp
}
%    \end{macrocode}
%    \end{macro}
%
%    \begin{macro}{\ltx@GlobalPrependToMacro}
%    \begin{macrocode}
\long\def\ltx@GlobalPrependToMacro#1#2{%
  \ifx\ltx@undefined#1%
    \let#1\ltx@empty
  \else
    \ifx\relax#1%
      \let#1\ltx@empty
    \fi
  \fi
  \begingroup
    \ltx@LocToksA{#2}%
    \ltx@LocToksB\expandafter{#1}%
    \xdef#1{\the\ltx@LocToksA\the\ltx@LocToksB}%
  \endgroup
}
%    \end{macrocode}
%    \end{macro}
%    \begin{macro}{\ltx@LocalPrependToMacro}
%    \begin{macrocode}
\long\def\ltx@LocalPrependToMacro#1#2{%
  \global\let\LTXcmds@gtemp#1%
  \ifx\ltx@undefined\LTXcmds@gtemp
    \global\let\LTXcmds@gtemp\ltx@empty
  \else
    \ifx\relax\LTXcmds@gtemp
      \global\letLTXcmds@gtemp\ltx@empty
    \fi
  \fi
  \begingroup
    \ltx@LocToksA{#2}%
    \ltx@LocToksB\expandafter{\LTXcmds@gtemp}%
    \xdef\LTXcmds@gtemp{\the\ltx@LocToksA\the\ltx@LocToksB}%
  \endgroup
  \let#1\LTXcmds@gtemp
}
%    \end{macrocode}
%    \end{macro}
%
% \subsection{Next character detection}
%
%    \begin{macro}{\ltx@ifnextchar}
%    \begin{macrocode}
\long\def\ltx@ifnextchar#1#2#3{%
  \begingroup
  \let\LTXcmds@CharToken= #1\relax
  \ltx@LocToksA{\endgroup#2}%
  \ltx@LocToksB{\endgroup#3}%
  \futurelet\LTXcmds@LetToken\LTXcmds@ifnextchar
}
%    \end{macrocode}
%    \end{macro}
%    \begin{macro}{\LTXcmds@ifnextchar}
%    \begin{macrocode}
\def\LTXcmds@ifnextchar{%
  \ifx\LTXcmds@LetToken\LTXcmds@CharToken
    \the\expandafter\ltx@LocToksA
  \else
    \expandafter
      \ifx\csname LTXcmds@LetToken\endcsname\LTXcmds@SpaceToken
      \expandafter\expandafter\expandafter\LTXcmds@@ifnextchar
    \else
      \the\expandafter\expandafter\expandafter\ltx@LocToksB
    \fi
  \fi
}
%    \end{macrocode}
%    \end{macro}
%    \begin{macro}{\LTXcmds@@ifnextchar}
%    \cs{futurelet} does not distinguish between a
%    character and a command that is a character
%    (defined by using \cs{let} or \cs{futurelet}).
%    Therefore the space is catched by \cs{romannumeral}
%    with negative character constant that gobbles one
%    optional space.
%    \begin{macrocode}
\def\LTXcmds@@ifnextchar{%
  \expandafter\futurelet
  \expandafter\LTXcmds@LetToken
  \expandafter\LTXcmds@ifnextchar
  \romannumeral-`\.%
}
%    \end{macrocode}
%    \end{macro}
%    \begin{macro}{\LTXcmds@SpaceToken}
%    \begin{macrocode}
\ltx@firstofone{\let\LTXcmds@SpaceToken= } %
%    \end{macrocode}
%    \end{macro}
%
%    \begin{macro}{\ltx@ifnextchar@nospace}
%    \begin{macrocode}
\long\def\ltx@ifnextchar@nospace#1#2#3{%
  \begingroup
  \let\LTXcmds@CharToken= #1\relax
  \ltx@LocToksA{\endgroup#2}%
  \ltx@LocToksB{\endgroup#3}%
  \futurelet\LTXcmds@LetToken\LTXcmds@ifnextchar@nospace
}
%    \end{macrocode}
%    \end{macro}
%    \begin{macro}{\LTXcmds@ifnextchar@nospace}
%    \begin{macrocode}
\def\LTXcmds@ifnextchar@nospace{%
  \the
  \ifx\LTXcmds@LetToken\LTXcmds@CharToken
    \expandafter\ltx@LocToksA
  \else
    \expandafter\ltx@LocToksB
  \fi
}
%    \end{macrocode}
%    \end{macro}
% \subsection{\cs{ltx@leavevmode}, \cs{ltx@mbox}}
%
%    \begin{macro}{\ltx@leavevmode}
%    \begin{macrocode}
\ltx@IfUndefined{quitvmode}{%
  \ltx@IfUndefined{leavevmode}{%
    \ltx@IfUndefined{voidb@x}{%
      \ltx@IfUndefined{newbox}{%
        \def\ltx@leavevmode{%
          \begingroup
            \setbox\ltx@zero=\hbox{}%
            \begingroup
              \setbox\ltx@zero=\hbox{\box\ltx@zero}%
            \endgroup
            \unhbox\ltx@zero
          \endgroup
        }%
      }{%
        \csname newbox\endcsname\LTXcmds@VoidBox
        \ifvoid\LTXcmds@VoidBox
        \else
          \setbox\LTXcmds@VoidBox=\hbox{}%
          \begingroup
            \setbox\LTXcmds@VoidBox=\hbox{\box\LTXcmds@VoidBox}%
          \endgroup
        \fi
        \def\ltx@leavevmode{\unhbox\LTXcmds@VoidBox}%
      }%
    }{%
      \def\ltx@leavevmode{\unhbox\voidb@x}%
    }%
  }{%
    \let\ltx@leavevmode\leavevmode
  }%
}{%
  \let\ltx@leavevmode\quitvmode
}
%    \end{macrocode}
%    \end{macro}
%
%    \begin{macro}{\ltx@mbox}
%    \begin{macrocode}
\def\ltx@mbox{%
  \ltx@leavevmode
  \hbox
}
%    \end{macrocode}
%    \end{macro}
%
% \subsection{Help macros}
%
%    \begin{macro}{\LTXcmds@num}
%    \begin{macrocode}
\ltx@IfUndefined{numexpr}{%
  \def\LTXcmds@num#1{%
    \expandafter\ltx@firstofone\expandafter{%
      \number#1%
    }%
  }%
}{%
  \def\LTXcmds@num#1{%
    \expandafter\ltx@firstofone\expandafter{%
      \the\numexpr#1%
    }%
  }%
}
%    \end{macrocode}
%    \end{macro}
%
% \subsection{Expandable test for emptiness}
%
%    \begin{macrocode}
\ltx@IfUndefined{detokenize}{%
%    \end{macrocode}
%
% \subsubsection{Vanilla \hologo{TeX}}
%
%    \begin{macro}{\ltx@ifempty}
%    The macro is based on \cs{@ifempty} of Robert R. Schneck
%    \cite{ifempty:schneck}
%    and \cs{@ifnull} of Ulrich Diez \cite{ifnull:diez}.
%    There are three cases to consider:
%    \begin{compactenum}
%    \item |#1| is empty,
%    \item |#1| is not empty and the first token
%      is not a begingroup character,
%    \item |#1| starts with a begingroup character (catcode 1).
%    \end{compactenum}
%    \begin{macrocode}
  \def\LTXcmds@temp#1{%
    \long\def\ltx@ifempty##1{%
      \romannumeral0%
      \iffalse{\fi
        \expandafter\ltx@gobble\expandafter{%
          \expandafter{\string##1}%
          \expandafter\ltx@gobble\string
        }%
        \expandafter\ltx@firstofthree\expandafter
        {\iffalse}\fi
        \expandafter#1\ltx@secondoftwo
      }%
      \expandafter#1\ltx@firstoftwo
    }%
%    \end{macrocode}
%    \end{macro}
%    \begin{macro}{\ltx@ifblank}
%    \begin{macrocode}
    \long\def\ltx@ifblank##1{%
      \romannumeral0%
      \iffalse{\fi
        \expandafter\expandafter\expandafter\ltx@gobble
        \expandafter\expandafter\expandafter{%
          \expandafter\expandafter\expandafter{%
            \expandafter\string\ltx@gobble##1.%
          }%
          \expandafter\ltx@gobble\string
        }%
        \expandafter\ltx@firstofthree\expandafter
        {\iffalse}\fi
        \expandafter#1\ltx@secondoftwo
      }%
      \expandafter#1\ltx@firstoftwo
    }%
  }%
  \LTXcmds@temp{ }%
%    \end{macrocode}
%    \end{macro}
%    \begin{macrocode}
}{%
%    \end{macrocode}
%
% \subsubsection{With \cs{detokenize}}
%
%    Ahmed Musa provided \cs{ifstrempty} using \cs{detokenize}
%    and \cs{pdfstrcmp} \cite{ifstrempty:musa}. Ulrich Diez, GL,
%    Heiko Oberdiek
%    improved it further by removing \cs{pdfstrcmp} and
%    taking three arguments
%    \cite{ifstrempty:diez1,ifstrempty:diez2,ifstrempty:diez3,^^A
%      ifstrempty:gl,ifstrempty:ho}.
%
%    \begin{macro}{\ltx@ifempty}
%    \begin{macrocode}
  \long\def\ltx@ifempty#1{%
    \romannumeral%
    \csname
      LTXcmds@ifempty%
      \ifcat$\detokenize{#1}$%
        @%
      \fi
    \endcsname
  }%
%    \end{macrocode}
%    \end{macro}
%    \begin{macro}{\LTXcmds@ifempty@}
%    \begin{macrocode}
  \long\def\LTXcmds@ifempty@#1#2{0 #1}%
%    \end{macrocode}
%    \end{macro}
%    \begin{macro}{\LTXcmds@ifempty}
%    \begin{macrocode}
  \long\def\LTXcmds@ifempty#1#2{0 #2}%
%    \end{macrocode}
%    \end{macro}
%
% \subsubsection{\cs{ltx@ifblank}}
%
%    \begin{macro}{\ltx@ifblank}
%    \begin{macrocode}
  \long\def\ltx@ifblank#1{%
    \romannumeral%
    \csname
      LTXcmds@ifempty%
      \ifcat$\detokenize\expandafter{\ltx@gobble#1.}$%
        @%
      \fi
    \endcsname
  }%
%    \end{macrocode}
%    \end{macro}
%
%    \begin{macrocode}
}
%    \end{macrocode}
%
% \subsection{\cs{ltx@zapspace}}
%
%    \begin{macro}{\ltx@zapspace}
%    \begin{macrocode}
\long\def\ltx@zapspace#1{%
  \romannumeral
  \LTXcmds@zapspace\ltx@zero#1 \@nil
}
%    \end{macrocode}
%    \end{macro}
%    \begin{macro}{\LTXcmds@zapspace}
%    \begin{macrocode}
\long\def\LTXcmds@zapspace#1 #2\@nil{%
  \ltx@ifempty{#2}{%
    #1%
  }{%
    \LTXcmds@zapspace#1#2\@nil
  }%
}
%    \end{macrocode}
%    \end{macro}
%
% \subsection{\cs{ltx@IfBoxEmpty}}
%
%    In case of \hologo{eTeX} the test for an empty box
%    is done via \cs{lastnodetype} as suggested
%    by David Kastrup \cite{ifboxempty:dak}.
%    \begin{macrocode}
\ltx@IfUndefined{lastnodetype}{%
  \catcode`\$=9 %
  \catcode`\&=14 %
}{%
  \catcode`\$=14 %
  \catcode`\&=9 %
}
%    \end{macrocode}
%    \begin{macro}{\ltx@IfBoxEmpty}
%    \begin{macrocode}
\def\ltx@IfBoxEmpty#1{%
  \ifvoid#1\relax
    \expandafter\ltx@secondoftwo
  \else
%    \end{macrocode}
%    Implementation using \hologo{eTeX}'s \cs{lastnodetype}.
%    \begin{macrocode}
&   \begingroup
&     \setbox\ltx@zero=\ifhbox#1\hbox\else\vbox\fi{%
&       \ifhmode\unhcopy\else\unvcopy\fi#1\relax
&       \expandafter
&     }%
&   \expandafter\endgroup
&   \ifnum\lastnodetype<\ltx@zero
&     \expandafter\expandafter\expandafter\ltx@firstoftwo
&   \else
&     \expandafter\expandafter\expandafter\ltx@secondoftwo
&   \fi
%    \end{macrocode}
%    Implementation without \hologo{eTeX} using a signature
%    at the beginning of the test box.
%    \begin{macrocode}
$   \begingroup
$     \setbox\ltx@zero=\ifhbox#1\hbox\else\vbox\fi{%
$       \penalty\ltx@one
$       \ifhmode\unhcopy\else\unvcopy\fi#1\relax
$       \expandafter
$     }%
$     \ifnum\lastpenalty=\ltx@one
%    \end{macrocode}
%    Box 0 has been changed and is restored by
%    closing the group.
%    \begin{macrocode}
$       \endgroup
$       \begingroup
$       \setbox\ltx@zero=\ifhbox#1\hbox\else\vbox\fi{%
$         \penalty\ltx@two
$         \ifhmode\unhcopy\else\unvcopy\fi#1\relax
$         \expandafter
$       }%
$       \ifnum\lastpenalty=\ltx@two
$         \def\next{\endgroup\expandafter\ltx@firstoftwo}%
$       \else
$         \def\next{\endgroup\expandafter\ltx@secondoftwo}%
$       \fi
$     \else
$       \def\next{\endgroup\expandafter\ltx@secondoftwo}%
$     \fi
$   \next
  \fi
}
%    \end{macrocode}
%    \end{macro}
%
%    \begin{macro}{\ltx@IfBoxVoidOrEmpty}
%    \begin{macrocode}
\def\ltx@IfBoxVoidOrEmpty#1{%
  \ifvoid#1\relax
    \expandafter\ltx@thirdoffour
  \fi
  \ltx@IfBoxEmpty{#1}%
}
%    \end{macrocode}
%    \end{macro}
%
%    \begin{macrocode}
\LTXcmds@AtEnd%
%</package>
%    \end{macrocode}
%% \section{Installation}
%
% \subsection{Download}
%
% \paragraph{Package.} This package is available on
% CTAN\footnote{\CTANpkg{ltxcmds}}:
% \begin{description}
% \item[\CTAN{macros/latex/contrib/ltxcmds/ltxcmds.dtx}] The source file.
% \item[\CTAN{macros/latex/contrib/ltxcmds/ltxcmds.pdf}] Documentation.
% \end{description}
%
%
% \paragraph{Bundle.} All the packages of the bundle `ltxcmds'
% are also available in a TDS compliant ZIP archive. There
% the packages are already unpacked and the documentation files
% are generated. The files and directories obey the TDS standard.
% \begin{description}
% \item[\CTANinstall{install/macros/latex/contrib/ltxcmds.tds.zip}]
% \end{description}
% \emph{TDS} refers to the standard ``A Directory Structure
% for \TeX\ Files'' (\CTANpkg{tds}). Directories
% with \xfile{texmf} in their name are usually organized this way.
%
% \subsection{Bundle installation}
%
% \paragraph{Unpacking.} Unpack the \xfile{ltxcmds.tds.zip} in the
% TDS tree (also known as \xfile{texmf} tree) of your choice.
% Example (linux):
% \begin{quote}
%   |unzip ltxcmds.tds.zip -d ~/texmf|
% \end{quote}
%
% \subsection{Package installation}
%
% \paragraph{Unpacking.} The \xfile{.dtx} file is a self-extracting
% \docstrip\ archive. The files are extracted by running the
% \xfile{.dtx} through \plainTeX:
% \begin{quote}
%   \verb|tex ltxcmds.dtx|
% \end{quote}
%
% \paragraph{TDS.} Now the different files must be moved into
% the different directories in your installation TDS tree
% (also known as \xfile{texmf} tree):
% \begin{quote}
% \def\t{^^A
% \begin{tabular}{@{}>{\ttfamily}l@{ $\rightarrow$ }>{\ttfamily}l@{}}
%   ltxcmds.sty & tex/generic/ltxcmds/ltxcmds.sty\\
%   ltxcmds.pdf & doc/latex/ltxcmds/ltxcmds.pdf\\
%   ltxcmds.dtx & source/latex/ltxcmds/ltxcmds.dtx\\
% \end{tabular}^^A
% }^^A
% \sbox0{\t}^^A
% \ifdim\wd0>\linewidth
%   \begingroup
%     \advance\linewidth by\leftmargin
%     \advance\linewidth by\rightmargin
%   \edef\x{\endgroup
%     \def\noexpand\lw{\the\linewidth}^^A
%   }\x
%   \def\lwbox{^^A
%     \leavevmode
%     \hbox to \linewidth{^^A
%       \kern-\leftmargin\relax
%       \hss
%       \usebox0
%       \hss
%       \kern-\rightmargin\relax
%     }^^A
%   }^^A
%   \ifdim\wd0>\lw
%     \sbox0{\small\t}^^A
%     \ifdim\wd0>\linewidth
%       \ifdim\wd0>\lw
%         \sbox0{\footnotesize\t}^^A
%         \ifdim\wd0>\linewidth
%           \ifdim\wd0>\lw
%             \sbox0{\scriptsize\t}^^A
%             \ifdim\wd0>\linewidth
%               \ifdim\wd0>\lw
%                 \sbox0{\tiny\t}^^A
%                 \ifdim\wd0>\linewidth
%                   \lwbox
%                 \else
%                   \usebox0
%                 \fi
%               \else
%                 \lwbox
%               \fi
%             \else
%               \usebox0
%             \fi
%           \else
%             \lwbox
%           \fi
%         \else
%           \usebox0
%         \fi
%       \else
%         \lwbox
%       \fi
%     \else
%       \usebox0
%     \fi
%   \else
%     \lwbox
%   \fi
% \else
%   \usebox0
% \fi
% \end{quote}
% If you have a \xfile{docstrip.cfg} that configures and enables \docstrip's
% TDS installing feature, then some files can already be in the right
% place, see the documentation of \docstrip.
%
% \subsection{Refresh file name databases}
%
% If your \TeX~distribution
% (\TeX\,Live, \mikTeX, \dots) relies on file name databases, you must refresh
% these. For example, \TeX\,Live\ users run \verb|texhash| or
% \verb|mktexlsr|.
%
% \subsection{Some details for the interested}
%
% \paragraph{Unpacking with \LaTeX.}
% The \xfile{.dtx} chooses its action depending on the format:
% \begin{description}
% \item[\plainTeX:] Run \docstrip\ and extract the files.
% \item[\LaTeX:] Generate the documentation.
% \end{description}
% If you insist on using \LaTeX\ for \docstrip\ (really,
% \docstrip\ does not need \LaTeX), then inform the autodetect routine
% about your intention:
% \begin{quote}
%   \verb|latex \let\install=y\input{ltxcmds.dtx}|
% \end{quote}
% Do not forget to quote the argument according to the demands
% of your shell.
%
% \paragraph{Generating the documentation.}
% You can use both the \xfile{.dtx} or the \xfile{.drv} to generate
% the documentation. The process can be configured by the
% configuration file \xfile{ltxdoc.cfg}. For instance, put this
% line into this file, if you want to have A4 as paper format:
% \begin{quote}
%   \verb|\PassOptionsToClass{a4paper}{article}|
% \end{quote}
% An example follows how to generate the
% documentation with pdf\LaTeX:
% \begin{quote}
%\begin{verbatim}
%pdflatex ltxcmds.dtx
%makeindex -s gind.ist ltxcmds.idx
%pdflatex ltxcmds.dtx
%makeindex -s gind.ist ltxcmds.idx
%pdflatex ltxcmds.dtx
%\end{verbatim}
% \end{quote}
%
% \begin{thebibliography}{9}
% \raggedright
%
% \bibitem{ifempty:schneck}
% Robert R. Schneck:
% \textit{Re: \cs{ifempty} solution %
%   (was Macro puzzle: maximally general \cs{ifempty})};
% newsgroup \xnewsgroup{comp.text.tex},
% \url{news:3eef1ada_6@corp.newsgroups.com},
% 2003-06-17.\\
% {\small
% \url{https://groups.google.com/group/comp.text.tex/msg/be03a159ec374895}}
%
% \bibitem{ifnull:diez}
% Ulrich Diez:
% \textit{Re: TeX refuses to strip outer braces in argument};
% newsgroup \xnewsgroup{comp.text.tex},
% \url{news:ibk3t8$ee7$1@news.albasani.net},
% 2010-11-12.\\
% {\small
% \url{https://groups.google.com/group/comp.text.tex/msg/803bd57221a04996}}
%
% \bibitem{ifstrempty:musa}
% Ahmed Musa:
% \textit{Re: TeX refuses to strip outer braces in argument};
% newsgroup \xnewsgroup{comp.text.tex},
% \url{news:f5496afe-40ed-42bd-b629-a2419ecf7c0d@o14g2000prn.googlegroups.com},
% 2010-12-03.\\
% {\small
% \url{https://groups.google.com/group/comp.text.tex/msg/fbf7d61a0c3a807d}}
%
% \bibitem{ifstrempty:diez1}
% Ulrich Diez:
% \textit{Re: TeX refuses to strip outer braces in argument};
% newsgroup \xnewsgroup{comp.text.tex},
% \url{news:idbo94$uka$1@four.albasani.net},
% 2010-12-03.\\
% {\small
% \url{https://groups.google.com/group/comp.text.tex/msg/0c230ee479487962}}
%
% \bibitem{ifstrempty:diez2}
% Ulrich Diez:
% \textit{Re: TeX refuses to strip outer braces in argument};
% newsgroup \xnewsgroup{comp.text.tex},
% \url{news:idbpu4$cg1$1@news.albasani.net},
% 2010-12-03.\\
% {\small
% \url{https://groups.google.com/group/comp.text.tex/msg/bbef4263390d647b}}
%
% \bibitem{ifstrempty:diez3}
% Ulrich Diez:
% \textit{Re: TeX refuses to strip outer braces in argument};
% newsgroup \xnewsgroup{comp.text.tex},
% \url{news:idd4ga$r83$1@four.albasani.net},
% 2010-12-04.\\
% {\small
% \url{https://groups.google.com/group/comp.text.tex/msg/00dfd1ec103cd272}}
%
% \bibitem{ifstrempty:gl}
% GL:
% \textit{Re: TeX refuses to strip outer braces in argument};
% newsgroup \xnewsgroup{comp.text.tex},
% \url{news:4cfa2e27$0$7389$426a74cc@news.free.fr},
% 2010-12-04.\\
% {\small
% \url{https://groups.google.com/group/comp.text.tex/msg/d3a75995c1cf267e}}
%
% \bibitem{ifstrempty:ho}
% Heiko Oberdiek:
% \textit{Re: TeX refuses to strip outer braces in argument};
% newsgroup \xnewsgroup{comp.text.tex},
% \url{news:iddhq1$3kj$1@news.eternal-september.org},
% 2010-12-04.\\
% {\small
% \url{https://groups.google.com/group/comp.text.tex/msg/5f7a23e3ab70e347}}
%
% \bibitem{ifboxempty:dak}
% David Kastrup:
% \textit{How to detect if \cs{vbox} is empty};
% newsgroup \xnewsgroup{comp.text.tex},
% 2011-02-04.\\
% {\small
% \url{https://groups.google.com/group/comp.text.tex/msg/8d3cb89496a4d86d}}
%
% \end{thebibliography}
%
% \begin{History}
%   \begin{Version}{2009/08/05 v1.0}
%   \item
%     First version.
%   \end{Version}
%   \begin{Version}{2009/12/12 v1.1}
%   \item
%     Short title shortened.
%   \item
%     \cs{ltx@IfUndefined} added.
%   \end{Version}
%   \begin{Version}{2010/01/28 v1.2}
%   \item
%     \cs{ltx@RemovePrefix} and \cs{ltx@StripPrefix} added.
%   \item
%     \cs{ltx@ifclassloaded}, \cs{ltx@ifpackageloaded},
%     \cs{ltx@iffileloaded},
%     \cs{ltx@ifclasslater}, \cs{ltx@ifpackagelater},
%     \cs{ltx@iffilelater},
%     \cs{ltx@clsextension}, \cs{ltx@pkgextension} added.
%   \item
%     \cs{ltx@GlobalAppendToMacro}, \cs{ltx@LocalAppendToMacro} added.
%   \end{Version}
%   \begin{Version}{2010/03/01 v1.3}
%   \item
%     \cs{ltx@newif} added.
%   \item
%     \cs{ltx@ifnextchar} added.
%   \item
%     Numbers \cs{ltx@zero}, \cs{ltx@one}, \cs{ltx@two}, \cs{ltx@cclv} added.
%   \end{Version}
%   \begin{Version}{2010/03/09 v1.4}
%   \item
%     \cs{ltx@pkgextension} and \cs{ltx@clsextension} are hardcoded to
%     avoid trouble with \cs{@onlypreamble}.
%   \end{Version}
%   \begin{Version}{2010/04/08 v1.5}
%   \item
%     \cs{ltx@cartwo}, \cs{ltx@cdrtwo},
%     \cs{ltx@carthree}, \cs{ltx@cdrthree},
%     \cs{ltx@carfour}, \cs{ltx@cdrfour} added.
%   \item
%     \cs{ltx@ReturnAfterFi} and \cs{ltx@ReturnAfterElseFi} fixed.
%   \end{Version}
%   \begin{Version}{2010/04/16 v1.6}
%   \item
%     \cs{ltx@leavevmode}, \cs{ltx@mbox} added.
%   \end{Version}
%   \begin{Version}{2010/04/26 v1.7}
%   \item
%     \cs{ltx@GobbleNum}, \cs{ltx@CdrNum}, \cs{ltx@CarNum} added.
%   \item
%     \cs{ltx@carzero}, \cs{ltx@cdrzero} added.
%   \item
%     \cs{ltx@hashchar} added.
%   \end{Version}
%   \begin{Version}{2010/09/11 v1.8}
%   \item
%     \cs{ltx@leftbracechar}, \cs{ltx@rightbracechar} added.
%   \end{Version}
%   \begin{Version}{2010/10/25 v1.9}
%   \item
%     \cs{ltx@LocalAppendToMacro} and \cs{ltx@GlobalAppendToMacro} are
%     now \cs{long}.
%   \end{Version}
%   \begin{Version}{2010/10/31 v1.10}
%   \item
%     \cs{ltx@newglobalif} added.
%   \end{Version}
%   \begin{Version}{2010/11/12 v1.11}
%   \item
%     \cs{ltx@ifempty} added.
%   \item
%     \cs{ltx@firstofthree}, \cs{ltx@secondofthree}, \cs{ltx@thirdofthree}
%     added.
%   \end{Version}
%   \begin{Version}{2010/12/02 v1.12}
%   \item
%     \cs{ltx@onelevel@sanitize} added.
%   \item
%     \cs{LTXcmds@num} fixed for the case with \cs{numexpr}
%     (bug found by GL).
%   \end{Version}
%   \begin{Version}{2010/12/04 v1.13}
%   \item
%     \cs{ltx@ifblank} added.
%   \item
%     Optimization for \cs{ltx@ifempty}.
%   \end{Version}
%   \begin{Version}{2010/12/07 v1.14}
%   \item
%     \cs{ltx@zapspace} added.
%   \end{Version}
%   \begin{Version}{2010/12/12 v1.15}
%   \item
%     \cs{ltx@minusone} added.
%   \end{Version}
%   \begin{Version}{2011/02/04 v1.16}
%   \item
%     \cs{ltx@IfBoxEmpty} and \cs{ltx@IfBoxVoidOrEmpty} added.
%   \item
%     \cs{ltx@firstoffour}, \dots, \cs{ltx@fourthoffour} added.
%   \end{Version}
%   \begin{Version}{2011/02/05 v1.17}
%   \item
%     \cs{ltx@IfBoxEmpty}: an empty box may have non-zero dimensions.
%   \end{Version}
%   \begin{Version}{2011/03/16 v1.18}
%   \item
%     \cs{ltx@ifclasslater} fixed.
%   \end{Version}
%   \begin{Version}{2011/04/14 v1.19}
%   \item
%     \cs{ltx@ifnextchar}: detection of optional spaces modified.
%   \item
%     \cs{ltx(Loc,Glob)(Toks,Dimen,Skip)(A,B,C,D,E)} added.
%   \end{Version}
%   \begin{Version}{2011/04/18 v1.20}
%   \item
%     \cs{ltx@ifnextchar} with conditional support (thanks GL for
%     bug report).
%   \end{Version}
%   \begin{Version}{2011/08/22 v1.21}
%   \item
%     \cs{ltx@GlobalPrependToMacro}, \cs{ltx@LocalPrependToMacro} added
%     (feature request of Martin M\"unch).
%   \end{Version}
%   \begin{Version}{2011/11/09 v1.22}
%   \item
%     \cs{ltx@carsecond}, \cs{ltx@carthird}, \cs{ltx@carfourth},
%     \cs{ltx@CarNumth} added.
%   \item
%     \cs{ltx@cdrzero}, \cs{ltx@cdr}, \cs{ltx@cdrtwo}, cs{ltx@cdrthree},
%     \cs{ltx@cdrfour}, \cs{ltx@CdrNum} modified to retain braces and spaces.
%     They are expandable in two expansion steps.
%   \end{Version}
%   \begin{Version}{2016/05/16 v1.23}
%   \item
%     Documentation updates.
%   \end{Version}
%   \begin{Version}{2019/12/15 v1.24}
%   \item
%     Documentation updates.
%   \end{Version}
% \end{History}
%
% \PrintIndex
%
% \Finale
\endinput
|
% \end{quote}
% Do not forget to quote the argument according to the demands
% of your shell.
%
% \paragraph{Generating the documentation.}
% You can use both the \xfile{.dtx} or the \xfile{.drv} to generate
% the documentation. The process can be configured by the
% configuration file \xfile{ltxdoc.cfg}. For instance, put this
% line into this file, if you want to have A4 as paper format:
% \begin{quote}
%   \verb|\PassOptionsToClass{a4paper}{article}|
% \end{quote}
% An example follows how to generate the
% documentation with pdf\LaTeX:
% \begin{quote}
%\begin{verbatim}
%pdflatex ltxcmds.dtx
%makeindex -s gind.ist ltxcmds.idx
%pdflatex ltxcmds.dtx
%makeindex -s gind.ist ltxcmds.idx
%pdflatex ltxcmds.dtx
%\end{verbatim}
% \end{quote}
%
% \begin{thebibliography}{9}
% \raggedright
%
% \bibitem{ifempty:schneck}
% Robert R. Schneck:
% \textit{Re: \cs{ifempty} solution %
%   (was Macro puzzle: maximally general \cs{ifempty})};
% newsgroup \xnewsgroup{comp.text.tex},
% \url{news:3eef1ada_6@corp.newsgroups.com},
% 2003-06-17.\\
% {\small
% \url{https://groups.google.com/group/comp.text.tex/msg/be03a159ec374895}}
%
% \bibitem{ifnull:diez}
% Ulrich Diez:
% \textit{Re: TeX refuses to strip outer braces in argument};
% newsgroup \xnewsgroup{comp.text.tex},
% \url{news:ibk3t8$ee7$1@news.albasani.net},
% 2010-11-12.\\
% {\small
% \url{https://groups.google.com/group/comp.text.tex/msg/803bd57221a04996}}
%
% \bibitem{ifstrempty:musa}
% Ahmed Musa:
% \textit{Re: TeX refuses to strip outer braces in argument};
% newsgroup \xnewsgroup{comp.text.tex},
% \url{news:f5496afe-40ed-42bd-b629-a2419ecf7c0d@o14g2000prn.googlegroups.com},
% 2010-12-03.\\
% {\small
% \url{https://groups.google.com/group/comp.text.tex/msg/fbf7d61a0c3a807d}}
%
% \bibitem{ifstrempty:diez1}
% Ulrich Diez:
% \textit{Re: TeX refuses to strip outer braces in argument};
% newsgroup \xnewsgroup{comp.text.tex},
% \url{news:idbo94$uka$1@four.albasani.net},
% 2010-12-03.\\
% {\small
% \url{https://groups.google.com/group/comp.text.tex/msg/0c230ee479487962}}
%
% \bibitem{ifstrempty:diez2}
% Ulrich Diez:
% \textit{Re: TeX refuses to strip outer braces in argument};
% newsgroup \xnewsgroup{comp.text.tex},
% \url{news:idbpu4$cg1$1@news.albasani.net},
% 2010-12-03.\\
% {\small
% \url{https://groups.google.com/group/comp.text.tex/msg/bbef4263390d647b}}
%
% \bibitem{ifstrempty:diez3}
% Ulrich Diez:
% \textit{Re: TeX refuses to strip outer braces in argument};
% newsgroup \xnewsgroup{comp.text.tex},
% \url{news:idd4ga$r83$1@four.albasani.net},
% 2010-12-04.\\
% {\small
% \url{https://groups.google.com/group/comp.text.tex/msg/00dfd1ec103cd272}}
%
% \bibitem{ifstrempty:gl}
% GL:
% \textit{Re: TeX refuses to strip outer braces in argument};
% newsgroup \xnewsgroup{comp.text.tex},
% \url{news:4cfa2e27$0$7389$426a74cc@news.free.fr},
% 2010-12-04.\\
% {\small
% \url{https://groups.google.com/group/comp.text.tex/msg/d3a75995c1cf267e}}
%
% \bibitem{ifstrempty:ho}
% Heiko Oberdiek:
% \textit{Re: TeX refuses to strip outer braces in argument};
% newsgroup \xnewsgroup{comp.text.tex},
% \url{news:iddhq1$3kj$1@news.eternal-september.org},
% 2010-12-04.\\
% {\small
% \url{https://groups.google.com/group/comp.text.tex/msg/5f7a23e3ab70e347}}
%
% \bibitem{ifboxempty:dak}
% David Kastrup:
% \textit{How to detect if \cs{vbox} is empty};
% newsgroup \xnewsgroup{comp.text.tex},
% 2011-02-04.\\
% {\small
% \url{https://groups.google.com/group/comp.text.tex/msg/8d3cb89496a4d86d}}
%
% \end{thebibliography}
%
% \begin{History}
%   \begin{Version}{2009/08/05 v1.0}
%   \item
%     First version.
%   \end{Version}
%   \begin{Version}{2009/12/12 v1.1}
%   \item
%     Short title shortened.
%   \item
%     \cs{ltx@IfUndefined} added.
%   \end{Version}
%   \begin{Version}{2010/01/28 v1.2}
%   \item
%     \cs{ltx@RemovePrefix} and \cs{ltx@StripPrefix} added.
%   \item
%     \cs{ltx@ifclassloaded}, \cs{ltx@ifpackageloaded},
%     \cs{ltx@iffileloaded},
%     \cs{ltx@ifclasslater}, \cs{ltx@ifpackagelater},
%     \cs{ltx@iffilelater},
%     \cs{ltx@clsextension}, \cs{ltx@pkgextension} added.
%   \item
%     \cs{ltx@GlobalAppendToMacro}, \cs{ltx@LocalAppendToMacro} added.
%   \end{Version}
%   \begin{Version}{2010/03/01 v1.3}
%   \item
%     \cs{ltx@newif} added.
%   \item
%     \cs{ltx@ifnextchar} added.
%   \item
%     Numbers \cs{ltx@zero}, \cs{ltx@one}, \cs{ltx@two}, \cs{ltx@cclv} added.
%   \end{Version}
%   \begin{Version}{2010/03/09 v1.4}
%   \item
%     \cs{ltx@pkgextension} and \cs{ltx@clsextension} are hardcoded to
%     avoid trouble with \cs{@onlypreamble}.
%   \end{Version}
%   \begin{Version}{2010/04/08 v1.5}
%   \item
%     \cs{ltx@cartwo}, \cs{ltx@cdrtwo},
%     \cs{ltx@carthree}, \cs{ltx@cdrthree},
%     \cs{ltx@carfour}, \cs{ltx@cdrfour} added.
%   \item
%     \cs{ltx@ReturnAfterFi} and \cs{ltx@ReturnAfterElseFi} fixed.
%   \end{Version}
%   \begin{Version}{2010/04/16 v1.6}
%   \item
%     \cs{ltx@leavevmode}, \cs{ltx@mbox} added.
%   \end{Version}
%   \begin{Version}{2010/04/26 v1.7}
%   \item
%     \cs{ltx@GobbleNum}, \cs{ltx@CdrNum}, \cs{ltx@CarNum} added.
%   \item
%     \cs{ltx@carzero}, \cs{ltx@cdrzero} added.
%   \item
%     \cs{ltx@hashchar} added.
%   \end{Version}
%   \begin{Version}{2010/09/11 v1.8}
%   \item
%     \cs{ltx@leftbracechar}, \cs{ltx@rightbracechar} added.
%   \end{Version}
%   \begin{Version}{2010/10/25 v1.9}
%   \item
%     \cs{ltx@LocalAppendToMacro} and \cs{ltx@GlobalAppendToMacro} are
%     now \cs{long}.
%   \end{Version}
%   \begin{Version}{2010/10/31 v1.10}
%   \item
%     \cs{ltx@newglobalif} added.
%   \end{Version}
%   \begin{Version}{2010/11/12 v1.11}
%   \item
%     \cs{ltx@ifempty} added.
%   \item
%     \cs{ltx@firstofthree}, \cs{ltx@secondofthree}, \cs{ltx@thirdofthree}
%     added.
%   \end{Version}
%   \begin{Version}{2010/12/02 v1.12}
%   \item
%     \cs{ltx@onelevel@sanitize} added.
%   \item
%     \cs{LTXcmds@num} fixed for the case with \cs{numexpr}
%     (bug found by GL).
%   \end{Version}
%   \begin{Version}{2010/12/04 v1.13}
%   \item
%     \cs{ltx@ifblank} added.
%   \item
%     Optimization for \cs{ltx@ifempty}.
%   \end{Version}
%   \begin{Version}{2010/12/07 v1.14}
%   \item
%     \cs{ltx@zapspace} added.
%   \end{Version}
%   \begin{Version}{2010/12/12 v1.15}
%   \item
%     \cs{ltx@minusone} added.
%   \end{Version}
%   \begin{Version}{2011/02/04 v1.16}
%   \item
%     \cs{ltx@IfBoxEmpty} and \cs{ltx@IfBoxVoidOrEmpty} added.
%   \item
%     \cs{ltx@firstoffour}, \dots, \cs{ltx@fourthoffour} added.
%   \end{Version}
%   \begin{Version}{2011/02/05 v1.17}
%   \item
%     \cs{ltx@IfBoxEmpty}: an empty box may have non-zero dimensions.
%   \end{Version}
%   \begin{Version}{2011/03/16 v1.18}
%   \item
%     \cs{ltx@ifclasslater} fixed.
%   \end{Version}
%   \begin{Version}{2011/04/14 v1.19}
%   \item
%     \cs{ltx@ifnextchar}: detection of optional spaces modified.
%   \item
%     \cs{ltx(Loc,Glob)(Toks,Dimen,Skip)(A,B,C,D,E)} added.
%   \end{Version}
%   \begin{Version}{2011/04/18 v1.20}
%   \item
%     \cs{ltx@ifnextchar} with conditional support (thanks GL for
%     bug report).
%   \end{Version}
%   \begin{Version}{2011/08/22 v1.21}
%   \item
%     \cs{ltx@GlobalPrependToMacro}, \cs{ltx@LocalPrependToMacro} added
%     (feature request of Martin M\"unch).
%   \end{Version}
%   \begin{Version}{2011/11/09 v1.22}
%   \item
%     \cs{ltx@carsecond}, \cs{ltx@carthird}, \cs{ltx@carfourth},
%     \cs{ltx@CarNumth} added.
%   \item
%     \cs{ltx@cdrzero}, \cs{ltx@cdr}, \cs{ltx@cdrtwo}, cs{ltx@cdrthree},
%     \cs{ltx@cdrfour}, \cs{ltx@CdrNum} modified to retain braces and spaces.
%     They are expandable in two expansion steps.
%   \end{Version}
%   \begin{Version}{2016/05/16 v1.23}
%   \item
%     Documentation updates.
%   \end{Version}
%   \begin{Version}{2019/12/15 v1.24}
%   \item
%     Documentation updates.
%   \end{Version}
% \end{History}
%
% \PrintIndex
%
% \Finale
\endinput
|
% \end{quote}
% Do not forget to quote the argument according to the demands
% of your shell.
%
% \paragraph{Generating the documentation.}
% You can use both the \xfile{.dtx} or the \xfile{.drv} to generate
% the documentation. The process can be configured by the
% configuration file \xfile{ltxdoc.cfg}. For instance, put this
% line into this file, if you want to have A4 as paper format:
% \begin{quote}
%   \verb|\PassOptionsToClass{a4paper}{article}|
% \end{quote}
% An example follows how to generate the
% documentation with pdf\LaTeX:
% \begin{quote}
%\begin{verbatim}
%pdflatex ltxcmds.dtx
%makeindex -s gind.ist ltxcmds.idx
%pdflatex ltxcmds.dtx
%makeindex -s gind.ist ltxcmds.idx
%pdflatex ltxcmds.dtx
%\end{verbatim}
% \end{quote}
%
% \begin{thebibliography}{9}
% \raggedright
%
% \bibitem{ifempty:schneck}
% Robert R. Schneck:
% \textit{Re: \cs{ifempty} solution %
%   (was Macro puzzle: maximally general \cs{ifempty})};
% newsgroup \xnewsgroup{comp.text.tex},
% \url{news:3eef1ada_6@corp.newsgroups.com},
% 2003-06-17.\\
% {\small
% \url{https://groups.google.com/group/comp.text.tex/msg/be03a159ec374895}}
%
% \bibitem{ifnull:diez}
% Ulrich Diez:
% \textit{Re: TeX refuses to strip outer braces in argument};
% newsgroup \xnewsgroup{comp.text.tex},
% \url{news:ibk3t8$ee7$1@news.albasani.net},
% 2010-11-12.\\
% {\small
% \url{https://groups.google.com/group/comp.text.tex/msg/803bd57221a04996}}
%
% \bibitem{ifstrempty:musa}
% Ahmed Musa:
% \textit{Re: TeX refuses to strip outer braces in argument};
% newsgroup \xnewsgroup{comp.text.tex},
% \url{news:f5496afe-40ed-42bd-b629-a2419ecf7c0d@o14g2000prn.googlegroups.com},
% 2010-12-03.\\
% {\small
% \url{https://groups.google.com/group/comp.text.tex/msg/fbf7d61a0c3a807d}}
%
% \bibitem{ifstrempty:diez1}
% Ulrich Diez:
% \textit{Re: TeX refuses to strip outer braces in argument};
% newsgroup \xnewsgroup{comp.text.tex},
% \url{news:idbo94$uka$1@four.albasani.net},
% 2010-12-03.\\
% {\small
% \url{https://groups.google.com/group/comp.text.tex/msg/0c230ee479487962}}
%
% \bibitem{ifstrempty:diez2}
% Ulrich Diez:
% \textit{Re: TeX refuses to strip outer braces in argument};
% newsgroup \xnewsgroup{comp.text.tex},
% \url{news:idbpu4$cg1$1@news.albasani.net},
% 2010-12-03.\\
% {\small
% \url{https://groups.google.com/group/comp.text.tex/msg/bbef4263390d647b}}
%
% \bibitem{ifstrempty:diez3}
% Ulrich Diez:
% \textit{Re: TeX refuses to strip outer braces in argument};
% newsgroup \xnewsgroup{comp.text.tex},
% \url{news:idd4ga$r83$1@four.albasani.net},
% 2010-12-04.\\
% {\small
% \url{https://groups.google.com/group/comp.text.tex/msg/00dfd1ec103cd272}}
%
% \bibitem{ifstrempty:gl}
% GL:
% \textit{Re: TeX refuses to strip outer braces in argument};
% newsgroup \xnewsgroup{comp.text.tex},
% \url{news:4cfa2e27$0$7389$426a74cc@news.free.fr},
% 2010-12-04.\\
% {\small
% \url{https://groups.google.com/group/comp.text.tex/msg/d3a75995c1cf267e}}
%
% \bibitem{ifstrempty:ho}
% Heiko Oberdiek:
% \textit{Re: TeX refuses to strip outer braces in argument};
% newsgroup \xnewsgroup{comp.text.tex},
% \url{news:iddhq1$3kj$1@news.eternal-september.org},
% 2010-12-04.\\
% {\small
% \url{https://groups.google.com/group/comp.text.tex/msg/5f7a23e3ab70e347}}
%
% \bibitem{ifboxempty:dak}
% David Kastrup:
% \textit{How to detect if \cs{vbox} is empty};
% newsgroup \xnewsgroup{comp.text.tex},
% 2011-02-04.\\
% {\small
% \url{https://groups.google.com/group/comp.text.tex/msg/8d3cb89496a4d86d}}
%
% \end{thebibliography}
%
% \begin{History}
%   \begin{Version}{2009/08/05 v1.0}
%   \item
%     First version.
%   \end{Version}
%   \begin{Version}{2009/12/12 v1.1}
%   \item
%     Short title shortened.
%   \item
%     \cs{ltx@IfUndefined} added.
%   \end{Version}
%   \begin{Version}{2010/01/28 v1.2}
%   \item
%     \cs{ltx@RemovePrefix} and \cs{ltx@StripPrefix} added.
%   \item
%     \cs{ltx@ifclassloaded}, \cs{ltx@ifpackageloaded},
%     \cs{ltx@iffileloaded},
%     \cs{ltx@ifclasslater}, \cs{ltx@ifpackagelater},
%     \cs{ltx@iffilelater},
%     \cs{ltx@clsextension}, \cs{ltx@pkgextension} added.
%   \item
%     \cs{ltx@GlobalAppendToMacro}, \cs{ltx@LocalAppendToMacro} added.
%   \end{Version}
%   \begin{Version}{2010/03/01 v1.3}
%   \item
%     \cs{ltx@newif} added.
%   \item
%     \cs{ltx@ifnextchar} added.
%   \item
%     Numbers \cs{ltx@zero}, \cs{ltx@one}, \cs{ltx@two}, \cs{ltx@cclv} added.
%   \end{Version}
%   \begin{Version}{2010/03/09 v1.4}
%   \item
%     \cs{ltx@pkgextension} and \cs{ltx@clsextension} are hardcoded to
%     avoid trouble with \cs{@onlypreamble}.
%   \end{Version}
%   \begin{Version}{2010/04/08 v1.5}
%   \item
%     \cs{ltx@cartwo}, \cs{ltx@cdrtwo},
%     \cs{ltx@carthree}, \cs{ltx@cdrthree},
%     \cs{ltx@carfour}, \cs{ltx@cdrfour} added.
%   \item
%     \cs{ltx@ReturnAfterFi} and \cs{ltx@ReturnAfterElseFi} fixed.
%   \end{Version}
%   \begin{Version}{2010/04/16 v1.6}
%   \item
%     \cs{ltx@leavevmode}, \cs{ltx@mbox} added.
%   \end{Version}
%   \begin{Version}{2010/04/26 v1.7}
%   \item
%     \cs{ltx@GobbleNum}, \cs{ltx@CdrNum}, \cs{ltx@CarNum} added.
%   \item
%     \cs{ltx@carzero}, \cs{ltx@cdrzero} added.
%   \item
%     \cs{ltx@hashchar} added.
%   \end{Version}
%   \begin{Version}{2010/09/11 v1.8}
%   \item
%     \cs{ltx@leftbracechar}, \cs{ltx@rightbracechar} added.
%   \end{Version}
%   \begin{Version}{2010/10/25 v1.9}
%   \item
%     \cs{ltx@LocalAppendToMacro} and \cs{ltx@GlobalAppendToMacro} are
%     now \cs{long}.
%   \end{Version}
%   \begin{Version}{2010/10/31 v1.10}
%   \item
%     \cs{ltx@newglobalif} added.
%   \end{Version}
%   \begin{Version}{2010/11/12 v1.11}
%   \item
%     \cs{ltx@ifempty} added.
%   \item
%     \cs{ltx@firstofthree}, \cs{ltx@secondofthree}, \cs{ltx@thirdofthree}
%     added.
%   \end{Version}
%   \begin{Version}{2010/12/02 v1.12}
%   \item
%     \cs{ltx@onelevel@sanitize} added.
%   \item
%     \cs{LTXcmds@num} fixed for the case with \cs{numexpr}
%     (bug found by GL).
%   \end{Version}
%   \begin{Version}{2010/12/04 v1.13}
%   \item
%     \cs{ltx@ifblank} added.
%   \item
%     Optimization for \cs{ltx@ifempty}.
%   \end{Version}
%   \begin{Version}{2010/12/07 v1.14}
%   \item
%     \cs{ltx@zapspace} added.
%   \end{Version}
%   \begin{Version}{2010/12/12 v1.15}
%   \item
%     \cs{ltx@minusone} added.
%   \end{Version}
%   \begin{Version}{2011/02/04 v1.16}
%   \item
%     \cs{ltx@IfBoxEmpty} and \cs{ltx@IfBoxVoidOrEmpty} added.
%   \item
%     \cs{ltx@firstoffour}, \dots, \cs{ltx@fourthoffour} added.
%   \end{Version}
%   \begin{Version}{2011/02/05 v1.17}
%   \item
%     \cs{ltx@IfBoxEmpty}: an empty box may have non-zero dimensions.
%   \end{Version}
%   \begin{Version}{2011/03/16 v1.18}
%   \item
%     \cs{ltx@ifclasslater} fixed.
%   \end{Version}
%   \begin{Version}{2011/04/14 v1.19}
%   \item
%     \cs{ltx@ifnextchar}: detection of optional spaces modified.
%   \item
%     \cs{ltx(Loc,Glob)(Toks,Dimen,Skip)(A,B,C,D,E)} added.
%   \end{Version}
%   \begin{Version}{2011/04/18 v1.20}
%   \item
%     \cs{ltx@ifnextchar} with conditional support (thanks GL for
%     bug report).
%   \end{Version}
%   \begin{Version}{2011/08/22 v1.21}
%   \item
%     \cs{ltx@GlobalPrependToMacro}, \cs{ltx@LocalPrependToMacro} added
%     (feature request of Martin M\"unch).
%   \end{Version}
%   \begin{Version}{2011/11/09 v1.22}
%   \item
%     \cs{ltx@carsecond}, \cs{ltx@carthird}, \cs{ltx@carfourth},
%     \cs{ltx@CarNumth} added.
%   \item
%     \cs{ltx@cdrzero}, \cs{ltx@cdr}, \cs{ltx@cdrtwo}, cs{ltx@cdrthree},
%     \cs{ltx@cdrfour}, \cs{ltx@CdrNum} modified to retain braces and spaces.
%     They are expandable in two expansion steps.
%   \end{Version}
%   \begin{Version}{2016/05/16 v1.23}
%   \item
%     Documentation updates.
%   \end{Version}
%   \begin{Version}{2019/12/15 v1.24}
%   \item
%     Documentation updates.
%   \end{Version}
% \end{History}
%
% \PrintIndex
%
% \Finale
\endinput

%        (quote the arguments according to the demands of your shell)
%
% Documentation:
%    (a) If ltxcmds.drv is present:
%           latex ltxcmds.drv
%    (b) Without ltxcmds.drv:
%           latex ltxcmds.dtx; ...
%    The class ltxdoc loads the configuration file ltxdoc.cfg
%    if available. Here you can specify further options, e.g.
%    use A4 as paper format:
%       \PassOptionsToClass{a4paper}{article}
%
%    Programm calls to get the documentation (example):
%       pdflatex ltxcmds.dtx
%       makeindex -s gind.ist ltxcmds.idx
%       pdflatex ltxcmds.dtx
%       makeindex -s gind.ist ltxcmds.idx
%       pdflatex ltxcmds.dtx
%
% Installation:
%    TDS:tex/generic/ltxcmds/ltxcmds.sty
%    TDS:doc/latex/ltxcmds/ltxcmds.pdf
%    TDS:source/latex/ltxcmds/ltxcmds.dtx
%
%<*ignore>
\begingroup
  \catcode123=1 %
  \catcode125=2 %
  \def\x{LaTeX2e}%
\expandafter\endgroup
\ifcase 0\ifx\install y1\fi\expandafter
         \ifx\csname processbatchFile\endcsname\relax\else1\fi
         \ifx\fmtname\x\else 1\fi\relax
\else\csname fi\endcsname
%</ignore>
%<*install>
\input docstrip.tex
\Msg{************************************************************************}
\Msg{* Installation}
\Msg{* Package: ltxcmds 2019/12/15 v1.24 LaTeX kernel commands for general use (HO)}
\Msg{************************************************************************}

\keepsilent
\askforoverwritefalse

\let\MetaPrefix\relax
\preamble

This is a generated file.

Project: ltxcmds
Version: 2019/12/15 v1.24

Copyright (C)
   2009-2011 Heiko Oberdiek
   2016-2019 Oberdiek Package Support Group

This work may be distributed and/or modified under the
conditions of the LaTeX Project Public License, either
version 1.3c of this license or (at your option) any later
version. This version of this license is in
   https://www.latex-project.org/lppl/lppl-1-3c.txt
and the latest version of this license is in
   https://www.latex-project.org/lppl.txt
and version 1.3 or later is part of all distributions of
LaTeX version 2005/12/01 or later.

This work has the LPPL maintenance status "maintained".

The Current Maintainers of this work are
Heiko Oberdiek and the Oberdiek Package Support Group
https://github.com/ho-tex/ltxcmds/issues


The Base Interpreter refers to any `TeX-Format',
because some files are installed in TDS:tex/generic//.

This work consists of the main source file ltxcmds.dtx
and the derived files
   ltxcmds.sty, ltxcmds.pdf, ltxcmds.ins, ltxcmds.drv,
   ltxcmds-test1.tex, ltxcmds-test-gobble.tex,
   ltxcmds-test-ifempty.tex, ltxcmds-test-zapspace.tex,
   ltxcmds-test-ifboxempty.tex, ltxcmds-test-nextchar.tex,
   ltxcmds-test-carcdr.tex.

\endpreamble
\let\MetaPrefix\DoubleperCent

\generate{%
  \file{ltxcmds.ins}{\from{ltxcmds.dtx}{install}}%
  \file{ltxcmds.drv}{\from{ltxcmds.dtx}{driver}}%
  \usedir{tex/generic/ltxcmds}%
  \file{ltxcmds.sty}{\from{ltxcmds.dtx}{package}}%
}

\catcode32=13\relax% active space
\let =\space%
\Msg{************************************************************************}
\Msg{*}
\Msg{* To finish the installation you have to move the following}
\Msg{* file into a directory searched by TeX:}
\Msg{*}
\Msg{*     ltxcmds.sty}
\Msg{*}
\Msg{* To produce the documentation run the file `ltxcmds.drv'}
\Msg{* through LaTeX.}
\Msg{*}
\Msg{* Happy TeXing!}
\Msg{*}
\Msg{************************************************************************}

\endbatchfile
%</install>
%<*ignore>
\fi
%</ignore>
%<*driver>
\NeedsTeXFormat{LaTeX2e}
\ProvidesFile{ltxcmds.drv}%
  [2019/12/15 v1.24 LaTeX kernel commands for general use (HO)]%
\documentclass{ltxdoc}
\usepackage{holtxdoc}[2011/11/22]
\usepackage{zref-savepos}
\usepackage{paralist}
\hfuzz=1pt
\begin{document}
  \DocInput{ltxcmds.dtx}%
\end{document}
%</driver>
% \fi
%
%
%
% \GetFileInfo{ltxcmds.drv}
%
% \title{The \xpackage{ltxcmds} package}
% \date{2019/12/15 v1.24}
% \author{Heiko Oberdiek\thanks
% {Please report any issues at \url{https://github.com/ho-tex/ltxcmds/issues}}}
%
% \maketitle
%
% \begin{abstract}
% The package \xpackage{ltxcmds} exports some utility macros
% from the \LaTeX\ kernel into a separate namespace and
% also provides them for other formats such as plain-\TeX.
% \end{abstract}
%
% \tableofcontents
%
% \newcounter{to}
% \renewcommand*{\theto}{to\arabic{to}}
% \newlength\tolength
% \tolength=\dimexpr .5\linewidth+\oddsidemargin+1in+15mm\relax
% \newcommand*{\TO}{^^A
%   \quad
%   \stepcounter{to}^^A
%   \zsavepos\theto
%   \ifdim\zposx\theto sp>0pt
%     \ifdim\zposx\theto sp<\tolength
%       \kern\dimexpr\tolength-\zposx\theto sp\relax
%     \fi
%   \fi
%   $\rightarrow$\quad
%   \ignorespaces
% }
%
% \section{Documentation}
%
% \subsection{Introduction}
%
% Many of my packages also support other formats such as plain-\TeX.
% Because I am rather familiar with the utility macros from
% \LaTeX's kernel (e.g. \cs{@gobble}, \cs{@firstoftwo}), I found
% myself rewriting them again and again, because they are lacking
% in plain-\TeX.
%
% Therefore this package provides often used macros and similar
% ones with the name prefix \cs{ltx@}. This avoids also faulty
% redefinitions. I remember an example where a package redefined
% \cs{@firstoftwo} with forgetting \cs{long}.
%
% \subsection{Numbers}
%
% \begin{declcs}{ltx@zero} \TO 0\\
%   \cs{ltx@one} \TO 1\\
%   \cs{ltx@two} \TO 2\\
%   \cs{ltx@cclv} \TO 255\\
%   \cs{ltx@minusone} \TO -1
% \end{declcs}
% These commands are numbers 0, 1, 2, 255 and -1. They are not digits
% and a space is not gobbled afterwards. Macro \cs{ltx@minusone}
% is available since version 2010/12/12 v1.15.
%
% \subsection{Scratch registers}
%
% Following the conventions of \hologo{plainTeX} and \hologo{LaTeX}
% the first ten registers are free to use. Even numbered registers
% are for local, odd numbered for global use.
%
% \begin{declcs}{ltx@(Loc,Glob)(Toks,Dimen,Skip)(A,B,C,D,E)}
% \end{declcs}
% The name consists of the prefix \cs{ltx@}, then
% \texttt{Loc} or \texttt{Glob} for local or global usage follows.
% The register type is given by \texttt{Toks} for token register,
% \texttt{Dimen} for dimen register and \texttt{Skip} for skip
% register. As last part the registers are numbered from \texttt{A}
% to \texttt{E}. Example: \cs{ltx@LocToksA}.
%
% Since 2011/04/14 v1.19.
%
% \subsection{Argument killers}
%
% \begin{declcs}{ltx@gobble} \M{1} \TO\\
%   \cs{ltx@gobbletwo} \M{1} \M{2} \TO\\
%   \cs{ltx@gobblethree} \M{1} \M{2} \M{3} \TO\\
%   \cs{ltx@gobblefour} \M{1} \M{2} \M{3} \M{4} \TO
% \end{declcs}
%
% \begin{declcs}{ltx@GobbleNum} \M{num} \M{1} \M{2} \dots\space
%   \M{\meta{num}} \TO
% \end{declcs}
% The first argument \meta{num} of macro \cs{ltx@GobbleNum} specifies,
% how many following arguments are eaten. Macro \cs{ltx@GobbleNum}
% is expandable in exact two expansion steps.
%
% \subsection{Argument grabbers}
%
% \begin{declcs}{ltx@firstofone} \M{1} \TO \meta{1}\\[1ex]
%  \cs{ltx@firstoftwo} \M{1} \M{2} \TO \meta{1}\\
%  \cs{ltx@secondoftwo} \M{1} \M{2} \TO \meta{2}\\[1ex]
%  \cs{ltx@firstofthree} \M{1} \M{2} \M{3} \TO \meta{1}\\
%  \cs{ltx@secondofthree} \M{1} \M{2} \M{3} \TO \meta{2}\\
%  \cs{ltx@thirdofthree} \M{1} \M{2} \M{3} \TO \meta{3}\\
%  \cs{ltx@firstoffour} \M{1} \M{2} \M{3} \M{4} \TO \meta{1}\\
%  \cs{ltx@secondoffour} \M{1} \M{2} \M{3} \M{4} \TO \meta{2}\\
%  \cs{ltx@thirdoffour} \M{1} \M{2} \M{3} \M{4} \TO \meta{3}\\
%  \cs{ltx@fourthoffour} \M{1} \M{2} \M{3} \M{4} \TO \meta{4}
% \end{declcs}
% Macros \cs{ltx@firstofthree}, \cs{ltx@secondofthree}
% and \cs{ltx@thirdofthree} were added in version
% 2010/11/12 v1.11.
% Macros \cs{ltx@firstoffour}, \dots, \cs{ltx@fourthoffour}
% were added in version 2011/02/04 v1.16.
%
% \subsection{List helpers}
%
% \begin{declcs}{ltx@carzero} \dots\ \cs{@nil} \TO\\
%   \cs{ltx@cdrzero} \dots\ \cs{@nil} \TO \dots
% \end{declcs}
%
% \begin{declcs}{ltx@car} \M{1} \dots\ \cs{@nil} \TO \meta{1}\\
%   \cs{ltx@cdr} \M{1} \dots\ \cs{@nil} \TO \dots
% \end{declcs}
%
% \begin{declcs}{ltx@cartwo} \M{1} \M{2} \dots\ \cs{@nil}
%   \TO \meta{1}\meta{2}\\
%   \cs{ltx@carsecond} \M{1} \M{2} \dots\ \cs{@nil} \TO \meta{2}\\
%   \cs{ltx@cdrtwo} \M{1} \M{2} \dots\ \cs{@nil} \TO \dots
% \end{declcs}
%
% \begin{declcs}{ltx@carthree} \M{1} \M{2} \M{3} \dots\ \cs{@nil}
%       \TO \meta{1}\meta{2}\meta{3}\\
%   \cs{ltx@carthird} \M{1} \M{2} \M{3} \dots\ \cs{@nil}
%       \TO \meta{3}\\
%   \cs{ltx@cdrthree} \M{1} \M{2} \M{3} \dots\ \cs{@nil} \TO \dots
% \end{declcs}
%
% \begin{declcs}{ltx@carfour} \M{1} \M{2} \M{3} \M{4} \dots\ \cs{@nil}
%       \TO \meta{1}\meta{2}\meta{3}\meta{4}\\
%   \cs{ltx@carfourth} \M{1} \M{2} \M{3} \M{4} \dots\ \cs{@nil}
%       \TO \meta{4}\\
%   \cs{ltx@cdrfour} \M{1} \M{2} \M{3} \M{4} \dots\ \cs{@nil} \TO \dots
% \end{declcs}
%
% \begin{declcs}{ltx@CarNum} \M{num} \M{1} \dots\ \M{\meta{num}}
%   \M{\meta{num}+1} \dots\ \cs{@nil}\\
%   \TO \M{1} \dots\ \M{\meta{num}} \dots\\
%   \cs{ltx@CarNumth} \M{num} \M{1} \dots\ \M{\meta{num}}
%   \M{\meta{num}+1} \dots\ \cs{@nil}\\
%   \TO \M{\meta{num}} \dots\\
%   \cs{ltx@CdrNum} \M{num} \M{1} \dots\ \M{\meta{num}}
%   \M{\meta{num}+1} \dots\ \cs{@nil}\\
%   \TO \M{\meta{num}+1} \dots
% \end{declcs}
%
% Macros with uppercase letters are expandable in two expansion steps.
% Changes in version 2019/12/15 v1.24:
% \begin{itemize}
% \item
%   Macros \cs{ltx@carsecond}, \cs{ltx@carthird}, \cs{ltx@carfourth},
%   \cs{ltx@CarNumth} added.
% \item
%   Macros \cs{ltx@cdr}, \cs{ltx@cdrtwo}, \cs{ltx@cdrthree}, \cs{ltx@cdrfour},
%   \cs{ltx@CdrNum} are expandable in two expansion steps and retain
%   spaces and braces after the first gobbled arguments.
% \end{itemize}
%
% \subsection{Tail recursion}
%
% \begin{declcs}{ltx@ReturnAfterFi} \M{1} \cs{fi} \TO \cs{fi} \meta{1}\\
%   \cs{ltx@ReturnAfterElseFi} \M{1} \cs{else} \M{2} \cs{fi}
%     \TO \cs{fi} \meta{1}
% \end{declcs}
%
% \subsection{Empty macro}
%
% \begin{declcs}{ltx@empty} \TO
% \end{declcs}
%
% \subsection{Characters}
%
% \begin{declcs}{ltx@space} \TO \verb*| |\\
%   \cs{ltx@percentchar} \TO \texttt{\%}\\
%   \cs{ltx@backslashchar} \TO
%      \makeatletter\texttt{\@backslashchar}\\
%   \cs{ltx@hashchar} \TO \texttt{\#}\qquad(since v1.7)\\
%   \cs{ltx@leftbracechar} \TO \texttt{\textbraceleft}\qquad(since v1.8)\\
%   \cs{ltx@rightbracechar} \TO \texttt{\textbraceright}\qquad(since v1.8)
% \end{declcs}
%
% \subsection{Boolean switch}
%
% \begin{declcs}{ltx@newif} \M{cmd}
% \end{declcs}
% \cs{ltx@newif} defines a new boolean switch \meta{cmd} like \cs{newif}.
% Unlike \plainTeX's \cs{newif}, \cs{ltx@newif} is not \cs{outer}.
% The command \meta{cmd} must start with the two characters |if|.
%
% \begin{declcs}{ltx@newglobalif} \M{cmd}
% \end{declcs}
% \cs{ltx@newglobalif} defines a new boolean switch \meta{cmd}
% like \cs{ltx@newif}. However the switch setting commands,
% \meta{cmd} without the prefix |if| and followed by |true| or |false|
% are acting globally.
%
% \subsection{Command definitions}
%
% \begin{declcs}{ltx@ifundefined} \M{cmd} \M{yes} \M{no}
% \end{declcs}
% If \eTeX\ is available, \cs{ifcsname} is used that does
% not have the side effect of defining undefined commands
% with meaning of \cs{relax}. This command is always expandable.
% Change in version 1.1: Also the meaning \cs{relax} is
% always considered ``undefined''.
%
% \begin{declcs}{ltx@IfUndefined} \M{cmd} \M{yes} \M{no}
% \end{declcs}
% If \eTeX\ is available, \cs{ifcsname} is used that does
% not have the side effect of defining undefined commands
% with meaning of \cs{relax}. Also it always checks for
% the meaning of \cs{relax} and considers this as undefined.
% This macro is not expandable without \eTeX.
%
% \begin{declcs}{ltx@LocalExpandAfter}
% \end{declcs}
% It expands the token after the next token but in a local
% context. That is the difference to \cs{expandafter}.
% The local context discards the side effect of \cs{csname}
% and let the command undefined after the expansion step.
%
% \subsection{Stripping}
%
% \begin{declcs}{ltx@RemovePrefix}\\
%   \cs{ltx@StripPrefix}
% \end{declcs}
% All tokens up to and including the next available character
% `\textgreater' are thrown away. Usually it is used to strip
% the first part of the output of the commands \cs{meaning}
% or \cs{pdflastmatch}. Macro \cs{ltx@RemovePrefix}
% has the same meaning as \LaTeX's \cs{strip@prefix}, whereas
% macro \cs{ltx@StripPrefix} expands the next token
% once before stripping the prefix.
%
% \begin{declcs}{ltx@onelevel@sanitize} \M{macro}
% \end{declcs}
% Macro \cs{ltx@onelevel@sanitize} provides \hologo{LaTeX}'s
% \cs{@onelevel@sanitize}. The macro is expanded once and
% the contents is converted to characters with catcode 12 (other)
% and space tokens with catcode 10 (space). Then then sanitized
% contents is stored into the macro again. Since version 1.12.
%
% \subsection{File management}
%
% All macros in this section are expandable like
% the counterparts of the \LaTeX\ kernel. Also they
% can be used after the preamble.
%
% \subsubsection{File extensions}
%
% \begin{declcs}{ltx@clsextension}\\
%   \cs{ltx@pkgextension}
% \end{declcs}
% Macros \cs{ltx@clsextension} and \cs{ltx@styextension}
% stores the strings |cls| and |sty|. In opposite to \hologo{LaTeX}'s
% \cs{@clsextension} and \cs{@styextension} they can also be
% used after |\begin{document}|.
%
% \subsubsection{Load check}
%
% \begin{declcs}{ltx@ifclassloaded} \M{class} \M{yes} \M{no}\\
%   \cs{ltx@ifpackageloaded} \M{package} \M{yes} \M{no}
% \end{declcs}
% Macros \cs{ltx@ifclassloaded}\slash\cs{ltx@ifpackageloaded}
% execute \meta{yes}, if the \meta{class} or \meta{package} is loaded,
% otherwise \meta{no} is called.
% Both \meta{class} and \meta{package} are specified without
% extension. The macros can also be used after |\begin{document}|.
%
% \begin{declcs}{ltx@iffileloaded} \M{file} \M{yes} \M{no}
% \end{declcs}
% If \LaTeX's \cs{ProvidesFile} macro was called before
% using \meta{file} as argument, then \cs{ltx@iffileloaded}
% calls \meta{yes}, otherwise \meta{no}.
% Therefore it is possible
% that the \meta{file} is loaded, but \meta{no} is executed because
% of a missing \cs{ProvidesFile}. The \LaTeX\ kernel does not have
% a counterpart of \cs{ltx@iffileloaded}.
%
% Note that the file name used in \cs{ProvidesFile} and
% \cs{ltx@iffileloaded} must match. For example, if \TeX's
% default extension \xext{tex} was given in the first command,
% then it must also specified in the latter command and vice versa.
%
% \subsubsection{Version date check}
%
% \begin{declcs}{ltx@ifclasslater} \M{class} \M{date} \M{yes} \M{no}\\
%   \cs{ltx@ifpackagelater} \M{package} \M{date} \M{yes} \M{no}\\
%   \cs{ltx@iffilelater} \M{file} \M{date} \M{yes} \M{no}
% \end{declcs}
% If a \cs{ProvidesClass}\slash\cs{ProvidesPackage}\slash
% \cs{ProvidesFile} command with exact the same class\slash package\slash
% file was executed before with an optional
% argument that starts with a \LaTeX\ version date, then
% this version date is compared with the argument \meta{date}.
% If they are equal or if the version date is the later date,
% then \meta{yes} is called. In all other cases \meta{no} is executed.
%
% A \LaTeX\ date has the format |YYYY/MM/DD| with |YYYY| as year
% with four digits, |MM| as month with two digits and |DD| as day
% with two digits. If \pdfTeX's \cs{pdfmatch} is available, then
% it is used to detect the version date, to reject invalid
% date formats and to reject some invalid dates.
% Dates before 1994/01/01 are always invalid, because version
% dates are introduced with \LaTeXe\ in 1994.
%
% \subsection{Macro additions}
%
% \begin{declcs}{ltx@GlobalAppendToMacro} \M{cmd} \M{addition}\\
%   \cs{ltx@LocalAppendToMacro} \M{cmd} \M{addition}
% \end{declcs}
% The \meta{addition} is appended to the parameterless macro \meta{cmd}.
% If \meta{cmd} is undefined or has the meaning \cs{relax}, then
% it will be initialized as empty macro beforehand. Due to a bug \meta{addition}
% must not contain \cs{par} before version 2010/10/25 v1.9.
%
% \begin{declcs}{ltx@GlobalPrependToMacro} \M{cmd} \M{addition}\\
%   \cs{ltx@LocalPrependToMacro} \M{cmd} \M{addition}
% \end{declcs}
% The \meta{addition} is prepended to the parameterless macro \meta{cmd}.
% If \meta{cmd} is undefined or has the meaning \cs{relax}, then
% it will be initialized as empty macro beforehand. The macros
% were added in version 2011/08/22 v1.21.
%
% \subsection{Next character detection}
%
% \begin{declcs}{ltx@ifnextchar} \M{char} \M{yes} \M{no}
% \end{declcs}
% If next character is \meta{char} then \meta{yes} is called,
% otherwise \meta{no}. The character is not removed.
% Spaces are silently removed when looking for \meta{char}
% as \hologo{LaTeX}'s version \cs{kernel@ifnextchar} does.
% But there are also small differences:
% \begin{itemize}
% \item The space can be used as \meta{char}. In this case
%   optional spaces before \meta{char} are not supported of course.
% \item If the optional space is a command that is a character
%   (defined by \cs{let} or \cs{futurelet}), then
%   \cs{kernel@ifnextchar} breaks with an \hologo{TeX} error.
%   \cs{ltx@ifnextchar} silently removes this token as
%   optional space.
% \end{itemize}
% Since 2010/03/01 v1.3.
%
% \begin{declcs}{ltx@ifnextchar@nospace} \M{char} \M{yes} \M{no}
% \end{declcs}
% Macro \cs{ltx@ifnextchar@nospace} behaves like
% macro \cs{ltx@ifnextchar} with the exception that
% optional spaces are not supported before \meta{char}.
% Since 2011/04/14 v1.19.
%
% \subsection{\cs{ltx@leavevmode}, \cs{ltx@mbox}}
%
% \begin{declcs}{ltx@leavevmode}
% \end{declcs}
% Macro \cs{ltx@leavevmode} calls \hologo{pdfTeX}'s \cs{quitvmode}.
% Otherwise \cs{leavevmode} is used and defined if it is necessary.
%
% \begin{declcs}{ltx@mbox}
% \end{declcs}
% Macro \cs{ltx@mbox} reimplements \cs{mbox} with two changes.
% Instead of \cs{leavevmode} it uses \cs{ltx@leavevmode} and
% stops right after \cs{hbox}. Especially it does not grab
% the argument and allows the extended syntax of \cs{hbox}.
%
% \subsection{Expandable test for emptiness}
%
% \begin{declcs}{ltx@ifempty} \M{stuff} \M{yes} \M{no}
% \end{declcs}
% Macro \cs{ltx@ifempty} checks in exact two expansion steps
% whether \meta{stuff}
% is empty or contains token. Depending on the result
% \meta{yes} or \meta{no} is executed. The token in \meta{stuff}
% may contain \cs{par} and unmatched conditionals (\cs{if}, \cs{else},
% \cs{fi}, \dots). Since version 2010/11/12 v1.11.
%
% \begin{declcs}{ltx@ifblank} \M{stuff} \M{yes} \M{no}
% \end{declcs}
% Macro \cs{ltx@ifblank} tests in exact two expansion steps
% if \meta{stuff} is empty or contain only blank spaces.
% In this case argument \meta{yes} is called. If \meta{stuff}
% contains other tokens than spaces then \meta{no} is executed.
% Since version 2010/12/04 v1.13.
%
% \subsection{Stripping spaces}
%
% \begin{declcs}{ltx@zapspace} \M{stuff}
% \end{declcs}
% Macro \cs{ltx@zapspace} strips spaces from \meta{stuff}
% that are not hidden inside curly braces. Like \hologo{LaTeX}'s
% \cs{zap@space} it is expandable. Differences:
% \begin{itemize}
% \item
% Syntax: \cs{zap@space} also expects a space token
% and \cs{@empty} after \meta{stuff}.
% \item
% Macro \cs{ltx@zapspace} is expandable in exact two
% expansion steps.
% \item
% Macro \cs{ltx@zapspace} always retains curly braces.
% \item
% Macro \cs{zap@space} has a bug. It stops stripping spaces
% after a token group in curly braces if the first two tokens
% inside the group are equal.
% \item
% Macro \cs{ltx@zapspace} also works with \cs{par}
% and conditionals (\cs{if}, \cs{else}, \cs{fi}, \dots).
% \end{itemize}
% Macro \cs{ltx@zapspace} is available since version
% 2010/12/07 v1.14.
%
% \subsection{Check for emptiness of boxes}
%
% \begin{declcs}{ltx@IfBoxEmpty} \M{box register number}
%   \M{yes} \M{no}
% \end{declcs}
% Macro \cs{ltx@IfBoxEmpty} calls \meta{yes} if
% the box exists (\cs{ifvoid} returns false) and the box
% does not contain any content. Otherwise if the box
% is void or contains something, then \meta{no} is executed.
% Thus being empty means that the box exists and is either
% an \cs{hbox} or a \cs{vbox} and may even have dimensions other than
% 0.0\,pt, but the box does not contain anything.
% Macro \cs{ltx@IfBoxEmpty} is available since 2010/02/04 v1.16.
%
% \begin{declcs}{ltx@IfBoxVoidOrEmpty} \M{box register number}
%   \M{yes} \M{no}
% \end{declcs}
% Macro \cs{ltx@IfBoxVoidOrEmpty} calls \meta{yes} if
% the box is either void or does not contain any content.
% Otherwise \meta{no} is executed.
% Macro \cs{ltx@IfBoxVoidOrEmpty} is available since 2010/02/04 v1.16.
%
% \StopEventually{
% }
%
% \section{Implementation}
%
% \subsection{Identification}
%
%    \begin{macrocode}
%<*package>
%    \end{macrocode}
%    Reload check, especially if the package is not used with \LaTeX.
%    \begin{macrocode}
\begingroup\catcode61\catcode48\catcode32=10\relax%
  \catcode13=5 % ^^M
  \endlinechar=13 %
  \catcode35=6 % #
  \catcode39=12 % '
  \catcode44=12 % ,
  \catcode45=12 % -
  \catcode46=12 % .
  \catcode58=12 % :
  \catcode64=11 % @
  \catcode123=1 % {
  \catcode125=2 % }
  \expandafter\let\expandafter\x\csname ver@ltxcmds.sty\endcsname
  \ifx\x\relax % plain-TeX, first loading
  \else
    \def\empty{}%
    \ifx\x\empty % LaTeX, first loading,
      % variable is initialized, but \ProvidesPackage not yet seen
    \else
      \expandafter\ifx\csname PackageInfo\endcsname\relax
        \def\x#1#2{%
          \immediate\write-1{Package #1 Info: #2.}%
        }%
      \else
        \def\x#1#2{\PackageInfo{#1}{#2, stopped}}%
      \fi
      \x{ltxcmds}{The package is already loaded}%
      \aftergroup\endinput
    \fi
  \fi
\endgroup%
%    \end{macrocode}
%    Package identification:
%    \begin{macrocode}
\begingroup\catcode61\catcode48\catcode32=10\relax%
  \catcode13=5 % ^^M
  \endlinechar=13 %
  \catcode35=6 % #
  \catcode39=12 % '
  \catcode40=12 % (
  \catcode41=12 % )
  \catcode44=12 % ,
  \catcode45=12 % -
  \catcode46=12 % .
  \catcode47=12 % /
  \catcode58=12 % :
  \catcode64=11 % @
  \catcode91=12 % [
  \catcode93=12 % ]
  \catcode123=1 % {
  \catcode125=2 % }
  \expandafter\ifx\csname ProvidesPackage\endcsname\relax
    \def\x#1#2#3[#4]{\endgroup
      \immediate\write-1{Package: #3 #4}%
      \xdef#1{#4}%
    }%
  \else
    \def\x#1#2[#3]{\endgroup
      #2[{#3}]%
      \ifx#1\@undefined
        \xdef#1{#3}%
      \fi
      \ifx#1\relax
        \xdef#1{#3}%
      \fi
    }%
  \fi
\expandafter\x\csname ver@ltxcmds.sty\endcsname
\ProvidesPackage{ltxcmds}%
  [2019/12/15 v1.24 LaTeX kernel commands for general use (HO)]%
%    \end{macrocode}
%
%    \begin{macrocode}
\begingroup\catcode61\catcode48\catcode32=10\relax%
  \catcode13=5 % ^^M
  \endlinechar=13 %
  \catcode123=1 % {
  \catcode125=2 % }
  \catcode64=11 % @
  \def\x{\endgroup
    \expandafter\edef\csname LTXcmds@AtEnd\endcsname{%
      \endlinechar=\the\endlinechar\relax
      \catcode13=\the\catcode13\relax
      \catcode32=\the\catcode32\relax
      \catcode35=\the\catcode35\relax
      \catcode61=\the\catcode61\relax
      \catcode64=\the\catcode64\relax
      \catcode123=\the\catcode123\relax
      \catcode125=\the\catcode125\relax
    }%
  }%
\x\catcode61\catcode48\catcode32=10\relax%
\catcode13=5 % ^^M
\endlinechar=13 %
\catcode35=6 % #
\catcode64=11 % @
\catcode123=1 % {
\catcode125=2 % }
\def\TMP@EnsureCode#1#2{%
  \edef\LTXcmds@AtEnd{%
    \LTXcmds@AtEnd
    \catcode#1=\the\catcode#1\relax
  }%
  \catcode#1=#2\relax
}
\TMP@EnsureCode{36}{3}% $
\TMP@EnsureCode{38}{4}% &
\TMP@EnsureCode{40}{12}% (
\TMP@EnsureCode{41}{12}% )
\TMP@EnsureCode{45}{12}% -
\TMP@EnsureCode{46}{12}% .
\TMP@EnsureCode{47}{12}% /
\TMP@EnsureCode{60}{12}% <
\TMP@EnsureCode{62}{12}% >
\TMP@EnsureCode{91}{12}% [
\TMP@EnsureCode{96}{12}% `
\TMP@EnsureCode{93}{12}% ]
\TMP@EnsureCode{94}{12}% ^ (superscript) (!)
\TMP@EnsureCode{124}{12}% |
\edef\LTXcmds@AtEnd{\LTXcmds@AtEnd\noexpand\endinput}
%    \end{macrocode}
%
% \subsection{Numbers}
%
%    \begin{macro}{\ltx@zero}
%    \begin{macrocode}
\chardef\ltx@zero=0 %
%    \end{macrocode}
%    \end{macro}
%    \begin{macro}{\ltx@one}
%    \begin{macrocode}
\chardef\ltx@one=1 %
%    \end{macrocode}
%    \end{macro}
%    \begin{macro}{\ltx@two}
%    \begin{macrocode}
\chardef\ltx@two=2 %
%    \end{macrocode}
%    \end{macro}
%    \begin{macro}{\ltx@active}
%    \begin{macrocode}
\chardef\ltx@active=13 %
%    \end{macrocode}
%    \end{macro}
%    \begin{macro}{\ltx@cclv}
%    \begin{macrocode}
\chardef\ltx@cclv=255 %
%    \end{macrocode}
%    \end{macro}
%
%    \begin{macro}{\ltx@minusone}
%    \begin{macrocode}
\def\ltx@minusone{%
  -\ltx@one
}
%    \end{macrocode}
%    \end{macro}
%
% \subsection{Scratch registers}
%
%    \begin{macro}{\ltx@LocToksA}
%    \begin{macrocode}
\toksdef\ltx@LocToksA=0 %
%    \end{macrocode}
%    \end{macro}
%    \begin{macro}{\ltx@LocToksB}
%    \begin{macrocode}
\toksdef\ltx@LocToksB=2 %
%    \end{macrocode}
%    \end{macro}
%    \begin{macro}{\ltx@LocToksC}
%    \begin{macrocode}
\toksdef\ltx@LocToksC=4 %
%    \end{macrocode}
%    \end{macro}
%    \begin{macro}{\ltx@LocToksD}
%    \begin{macrocode}
\toksdef\ltx@LocToksD=6 %
%    \end{macrocode}
%    \end{macro}
%    \begin{macro}{\ltx@LocToksE}
%    \begin{macrocode}
\toksdef\ltx@LocToksE=8 %
%    \end{macrocode}
%    \end{macro}
%
%    \begin{macro}{\ltx@GlobToksA}
%    \begin{macrocode}
\toksdef\ltx@GlobToksA=1 %
%    \end{macrocode}
%    \end{macro}
%    \begin{macro}{\ltx@GlobToksB}
%    \begin{macrocode}
\toksdef\ltx@GlobToksB=3 %
%    \end{macrocode}
%    \end{macro}
%    \begin{macro}{\ltx@GlobToksC}
%    \begin{macrocode}
\toksdef\ltx@GlobToksC=5 %
%    \end{macrocode}
%    \end{macro}
%    \begin{macro}{\ltx@GlobToksD}
%    \begin{macrocode}
\toksdef\ltx@GlobToksD=7 %
%    \end{macrocode}
%    \end{macro}
%    \begin{macro}{\ltx@GlobToksE}
%    \begin{macrocode}
\toksdef\ltx@GlobToksE=9 %
%    \end{macrocode}
%    \end{macro}
%
%    \begin{macro}{\ltx@LocDimenA}
%    \begin{macrocode}
\dimendef\ltx@LocDimenA=0 %
%    \end{macrocode}
%    \end{macro}
%    \begin{macro}{\ltx@LocDimenB}
%    \begin{macrocode}
\dimendef\ltx@LocDimenB=2 %
%    \end{macrocode}
%    \end{macro}
%    \begin{macro}{\ltx@LocDimenC}
%    \begin{macrocode}
\dimendef\ltx@LocDimenC=4 %
%    \end{macrocode}
%    \end{macro}
%    \begin{macro}{\ltx@LocDimenD}
%    \begin{macrocode}
\dimendef\ltx@LocDimenD=6 %
%    \end{macrocode}
%    \end{macro}
%    \begin{macro}{\ltx@LocDimenE}
%    \begin{macrocode}
\dimendef\ltx@LocDimenE=8 %
%    \end{macrocode}
%    \end{macro}
%
%    \begin{macro}{\ltx@GlobDimenA}
%    \begin{macrocode}
\dimendef\ltx@GlobDimenA=1 %
%    \end{macrocode}
%    \end{macro}
%    \begin{macro}{\ltx@GlobDimenB}
%    \begin{macrocode}
\dimendef\ltx@GlobDimenB=3 %
%    \end{macrocode}
%    \end{macro}
%    \begin{macro}{\ltx@GlobDimenC}
%    \begin{macrocode}
\dimendef\ltx@GlobDimenC=5 %
%    \end{macrocode}
%    \end{macro}
%    \begin{macro}{\ltx@GlobDimenD}
%    \begin{macrocode}
\dimendef\ltx@GlobDimenD=7 %
%    \end{macrocode}
%    \end{macro}
%    \begin{macro}{\ltx@GlobDimenE}
%    \begin{macrocode}
\dimendef\ltx@GlobDimenE=9 %
%    \end{macrocode}
%    \end{macro}
%
%    \begin{macro}{\ltx@LocSkipA}
%    \begin{macrocode}
\skipdef\ltx@LocSkipA=0 %
%    \end{macrocode}
%    \end{macro}
%    \begin{macro}{\ltx@LocSkipB}
%    \begin{macrocode}
\skipdef\ltx@LocSkipB=2 %
%    \end{macrocode}
%    \end{macro}
%    \begin{macro}{\ltx@LocSkipC}
%    \begin{macrocode}
\skipdef\ltx@LocSkipC=4 %
%    \end{macrocode}
%    \end{macro}
%    \begin{macro}{\ltx@LocSkipD}
%    \begin{macrocode}
\skipdef\ltx@LocSkipD=6 %
%    \end{macrocode}
%    \end{macro}
%    \begin{macro}{\ltx@LocSkipE}
%    \begin{macrocode}
\skipdef\ltx@LocSkipE=8 %
%    \end{macrocode}
%    \end{macro}
%
%    \begin{macro}{\ltx@GlobSkipA}
%    \begin{macrocode}
\skipdef\ltx@GlobSkipA=1 %
%    \end{macrocode}
%    \end{macro}
%    \begin{macro}{\ltx@GlobSkipB}
%    \begin{macrocode}
\skipdef\ltx@GlobSkipB=3 %
%    \end{macrocode}
%    \end{macro}
%    \begin{macro}{\ltx@GlobSkipC}
%    \begin{macrocode}
\skipdef\ltx@GlobSkipC=5 %
%    \end{macrocode}
%    \end{macro}
%    \begin{macro}{\ltx@GlobSkipD}
%    \begin{macrocode}
\skipdef\ltx@GlobSkipD=7 %
%    \end{macrocode}
%    \end{macro}
%    \begin{macro}{\ltx@GlobSkipE}
%    \begin{macrocode}
\skipdef\ltx@GlobSkipE=9 %
%    \end{macrocode}
%    \end{macro}
%
%
% \subsection{Argument killers}
%
%    \begin{macro}{\ltx@gobble}
%    \begin{macrocode}
\long\def\ltx@gobble#1{}
%    \end{macrocode}
%    \end{macro}
%    \begin{macro}{\ltx@gobbletwo}
%    \begin{macrocode}
\long\def\ltx@gobbletwo#1#2{}
%    \end{macrocode}
%    \end{macro}
%    \begin{macro}{\ltx@gobblethree}
%    \begin{macrocode}
\long\def\ltx@gobblethree#1#2#3{}
%    \end{macrocode}
%    \end{macro}
%    \begin{macro}{\ltx@gobblefour}
%    \begin{macrocode}
\long\def\ltx@gobblefour#1#2#3#4{}
%    \end{macrocode}
%    \end{macro}
%
%    \begin{macro}{\ltx@GobbleNum}
%    \begin{macrocode}
\def\ltx@GobbleNum#1{%
  \romannumeral
  \csname ltx@zero%
  \expandafter\LTXcmds@GobbleNum
  \romannumeral\LTXcmds@num{#1}000{m\endcsname}%
}
%    \end{macrocode}
%    \end{macro}
%    \begin{macro}{\LTXcmds@GobbleNum}
%    \begin{macrocode}
\def\LTXcmds@GobbleNum#1{%
  \csname LTXcmds@G#1\LTXcmds@GobbleNum
}
%    \end{macrocode}
%    \end{macro}
%    \begin{macro}{\LTXcmds@Gm}
%    \begin{macrocode}
\long\def\LTXcmds@Gm#1{%
  \endcsname
}
%    \end{macrocode}
%    \end{macro}
%
% \subsection{Argument grabbers}
%
%    \begin{macro}{\ltx@firstofone}
%    \begin{macrocode}
\long\def\ltx@firstofone#1{#1}
%    \end{macrocode}
%    \end{macro}
%    \begin{macro}{\ltx@firstoftwo}
%    \begin{macrocode}
\long\def\ltx@firstoftwo#1#2{#1}
%    \end{macrocode}
%    \end{macro}
%    \begin{macro}{\ltx@secondoftwo}
%    \begin{macrocode}
\long\def\ltx@secondoftwo#1#2{#2}
%    \end{macrocode}
%    \end{macro}
%
%    \begin{macro}{\ltx@firstofthree}
%    \begin{macrocode}
\long\def\ltx@firstofthree#1#2#3{#1}
%    \end{macrocode}
%    \end{macro}
%    \begin{macro}{\ltx@secondofthree}
%    \begin{macrocode}
\long\def\ltx@secondofthree#1#2#3{#2}
%    \end{macrocode}
%    \end{macro}
%    \begin{macro}{\ltx@thirdofthree}
%    \begin{macrocode}
\long\def\ltx@thirdofthree#1#2#3{#3}%
%    \end{macrocode}
%    \end{macro}
%
%    \begin{macro}{\ltx@firstoffour}
%    \begin{macrocode}
\long\def\ltx@firstoffour#1#2#3#4{#1}
%    \end{macrocode}
%    \end{macro}
%    \begin{macro}{\ltx@secondoffour}
%    \begin{macrocode}
\long\def\ltx@secondoffour#1#2#3#4{#2}
%    \end{macrocode}
%    \end{macro}
%    \begin{macro}{\ltx@thirdoffour}
%    \begin{macrocode}
\long\def\ltx@thirdoffour#1#2#3#4{#3}%
%    \end{macrocode}
%    \end{macro}
%    \begin{macro}{\ltx@fourthoffour}
%    \begin{macrocode}
\long\def\ltx@fourthoffour#1#2#3#4{#4}%
%    \end{macrocode}
%    \end{macro}
%
% \subsection{List helpers}
%
%    \begin{macro}{\ltx@carzero}
%    \begin{macrocode}
\long\def\ltx@carzero#1\@nil{}%
%    \end{macrocode}
%    \end{macro}
%    \begin{macro}{\LTXcmds@cdrzero}
%    \begin{macrocode}
\long\def\LTXcmds@cdrzero#1\@nil{#1}
%    \end{macrocode}
%    \end{macro}
%    \begin{macro}{\ltx@cdrzero}
%    \begin{macrocode}
\def\ltx@cdrzero{%
  \romannumeral\LTXcmds@cdrzero\ltx@zero
}
%    \end{macrocode}
%    \end{macro}
%
%    \begin{macro}{\ltx@car}
%    \begin{macrocode}
\long\def\ltx@car#1#2\@nil{#1}
%    \end{macrocode}
%    \end{macro}
%    \begin{macro}{\ltx@cdr}
%    \begin{macrocode}
\long\def\ltx@cdr#1{%
  \romannumeral\LTXcmds@cdrzero\ltx@zero
}
%    \end{macrocode}
%    \end{macro}
%
%    \begin{macro}{\ltx@cartwo}
%    \begin{macrocode}
\long\def\ltx@cartwo#1#2#3\@nil{#1#2}
%    \end{macrocode}
%    \end{macro}
%    \begin{macro}{\ltx@carsecond}
%    \begin{macrocode}
\long\def\ltx@carsecond#1#2#3\@nil{#2}
%    \end{macrocode}
%    \end{macro}
%    \begin{macro}{\ltx@cdrtwo}
%    \begin{macrocode}
\long\def\ltx@cdrtwo#1#2{%
  \romannumeral\LTXcmds@cdrzero\ltx@zero
}
%    \end{macrocode}
%    \end{macro}
%
%    \begin{macro}{\ltx@carthree}
%    \begin{macrocode}
\long\def\ltx@carthree#1#2#3#4\@nil{#1#2#3}
%    \end{macrocode}
%    \end{macro}
%    \begin{macro}{\ltx@carthird}
%    \begin{macrocode}
\long\def\ltx@carthird#1#2#3#4\@nil{#3}
%    \end{macrocode}
%    \end{macro}
%    \begin{macro}{\ltx@cdrthree}
%    \begin{macrocode}
\long\def\ltx@cdrthree#1#2#3{%
  \romannumeral\LTXcmds@cdrzero\ltx@zero
}
%    \end{macrocode}
%    \end{macro}
%
%    \begin{macro}{\ltx@carfour}
%    \begin{macrocode}
\long\def\ltx@carfour#1#2#3#4#5\@nil{#1#2#3#4}
%    \end{macrocode}
%    \end{macro}
%    \begin{macro}{\ltx@carfourth}
%    \begin{macrocode}
\long\def\ltx@carfourth#1#2#3#4#5\@nil{#4}
%    \end{macrocode}
%    \end{macro}
%    \begin{macro}{\ltx@cdrfour}
%    \begin{macrocode}
\long\def\ltx@cdrfour#1#2#3#4{%
  \romannumeral\LTXcmds@cdrzero\ltx@zero
}
%    \end{macrocode}
%    \end{macro}
%
%    \begin{macro}{\ltx@CarNum}
%    \begin{macrocode}
\def\ltx@CarNum#1{%
  \romannumeral
  \csname LTXcmds@CarNumFinish%
  \expandafter\LTXcmds@CarNum
  \romannumeral\LTXcmds@num{#1}000{x\endcsname}%
}
%    \end{macrocode}
%    \end{macro}
%    \begin{macro}{\LTXcmds@CarNum}
%    \begin{macrocode}
\def\LTXcmds@CarNum#1{%
  \csname LTXcmds@C#1\LTXcmds@CarNum
}
%    \end{macrocode}
%    \end{macro}
%    \begin{macro}{\LTXcmds@Cm}
%    \begin{macrocode}
\long\def\LTXcmds@Cm#1#2{%
  \endcsname{#1#2}%
}
%    \end{macrocode}
%    \end{macro}
%    \begin{macro}{\LTXcmds@Cx}
%    \begin{macrocode}
\def\LTXcmds@Cx#1{%
  \endcsname{}%
}
%    \end{macrocode}
%    \end{macro}
%    \begin{macro}{\LTXcmds@CarNumFinish}
%    \begin{macrocode}
\long\def\LTXcmds@CarNumFinish#1#2\@nil{%
  \ltx@zero
  #1%
}
%    \end{macrocode}
%    \end{macro}
%
%    \begin{macro}{\ltx@CarNumth}
%    \begin{macrocode}
\def\ltx@CarNumth#1{%
  \romannumeral
  \expandafter\expandafter\expandafter
  \LTXcmds@CarNumth
  \ltx@GobbleNum{#1}{}%
}
%    \end{macrocode}
%    \end{macro}
%    \begin{macro}{\LTXcmds@CarNumth}
%    \begin{macrocode}
\long\def\LTXcmds@CarNumth#1#2\@nil{%
  \ltx@zero
  #1%
}
%    \end{macrocode}
%    \end{macro}
%
%    \begin{macro}{\ltx@CdrNum}
%    \begin{macrocode}
\def\ltx@CdrNum#1{%
  \romannumeral%
  \expandafter\expandafter\expandafter\ltx@cdrzero
  \expandafter\expandafter\expandafter\ltx@zero
  \ltx@GobbleNum{#1}%
}
%    \end{macrocode}
%    \end{macro}
%
% \subsection{Tail recursion}
%
%    \begin{macro}{\ltx@ReturnAfterFi}
%    \begin{macrocode}
\long\def\ltx@ReturnAfterFi#1\fi{\fi#1}
%    \end{macrocode}
%    \end{macro}
%    \begin{macro}{\ltx@ReturnAfterElseFi}
%    \begin{macrocode}
\long\def\ltx@ReturnAfterElseFi#1\else#2\fi{\fi#1}
%    \end{macrocode}
%    \end{macro}
%
% \subsection{Empty macro}
%
%    \begin{macro}{\ltx@empty}
%    \begin{macrocode}
\def\ltx@empty{}
%    \end{macrocode}
%    \end{macro}
%
% \subsection{Characters}
%
%    \begin{macro}{\ltx@space}
%    \begin{macrocode}
\def\ltx@space{ }
%    \end{macrocode}
%    \end{macro}
%    \begin{macro}{\ltx@percentchar}
%    \begin{macrocode}
\begingroup
  \lccode`0=`\%\relax
\lowercase{\endgroup
  \def\ltx@percentchar{0}%
}
%    \end{macrocode}
%    \end{macro}
%    \begin{macro}{\ltx@backslashchar}
%    \begin{macrocode}
\begingroup
  \lccode`0=`\\\relax
\lowercase{\endgroup
  \def\ltx@backslashchar{0}%
}
%    \end{macrocode}
%    \end{macro}
%    \begin{macro}{\ltx@hashchar}
%    \begin{macrocode}
\begingroup
  \lccode`0=`\#\relax
\lowercase{\endgroup
  \def\ltx@hashchar{0}%
}
%    \end{macrocode}
%    \end{macro}
%    \begin{macro}{\ltx@leftbracechar}
%    \begin{macrocode}
\begingroup
  \lccode`0=`\{\relax
\lowercase{\endgroup
  \def\ltx@leftbracechar{0}%
}
%    \end{macrocode}
%    \end{macro}
%    \begin{macro}{\ltx@rightbracechar}
%    \begin{macrocode}
\begingroup
  \lccode`0=`\}\relax
\lowercase{\endgroup
  \def\ltx@rightbracechar{0}%
}
%    \end{macrocode}
%    \end{macro}
%
% \subsection{Boolean switch}
%
%    \begin{macro}{\ltx@newif}
%    \begin{macrocode}
\def\ltx@newif#1{%
  \begingroup
    \escapechar=-1 %
  \expandafter\endgroup
  \expandafter\LTXcmds@newif\string#1\@nil
}
%    \end{macrocode}
%    \end{macro}
%    \begin{macro}{\LTXcmds@newif}
%    \begin{macrocode}
\begingroup
  \escapechar=-1 %
\expandafter\endgroup
\expandafter\def\expandafter\LTXcmds@newif\string\if#1\@nil{%
  \expandafter\edef\csname#1true\endcsname{%
    \let
    \expandafter\noexpand\csname if#1\endcsname
    \noexpand\iftrue
  }%
  \expandafter\edef\csname#1false\endcsname{%
    \let
    \expandafter\noexpand\csname if#1\endcsname
    \noexpand\iffalse
  }%
  \csname#1false\endcsname
}
%    \end{macrocode}
%    \end{macro}
%
%    \begin{macro}{\ltx@newglobalif}
%    \begin{macrocode}
\def\ltx@newglobalif#1{%
  \begingroup
    \escapechar=-1 %
  \expandafter\endgroup
  \expandafter\LTXcmds@newglobalif\string#1\@nil
}
%    \end{macrocode}
%    \end{macro}
%    \begin{macro}{\LTXcmds@newglobalif}
%    \begin{macrocode}
\begingroup
  \escapechar=-1 %
\expandafter\endgroup
\expandafter
\def\expandafter\LTXcmds@newglobalif\string\if#1\@nil{%
  \expandafter\edef\csname#1true\endcsname{%
    \global\let
    \expandafter\noexpand\csname if#1\endcsname
    \noexpand\iftrue
  }%
  \expandafter\edef\csname#1false\endcsname{%
    \global\let
    \expandafter\noexpand\csname if#1\endcsname
    \noexpand\iffalse
  }%
  \csname#1false\endcsname
}
%    \end{macrocode}
%    \end{macro}
%
% \subsection{Command definitions}
%
%    \begin{macro}{\ltx@LocalExpandAfter}
%    \begin{macrocode}
\def\ltx@LocalExpandAfter{%
  \begingroup
    \expandafter\expandafter\expandafter
  \endgroup
  \expandafter
}
%    \end{macrocode}
%    \end{macro}
%
%    \begin{macrocode}
\ltx@LocalExpandAfter
\ifx\csname ifcsname\endcsname\relax
%    \end{macrocode}
%    \begin{macro}{\ltx@ifundefined}
%    \begin{macrocode}
  \def\ltx@ifundefined#1{%
    \expandafter\ifx\csname #1\endcsname\relax
      \expandafter\ltx@firstoftwo
    \else
      \expandafter\ltx@secondoftwo
    \fi
  }%
%    \end{macrocode}
%    \end{macro}
%    \begin{macro}{\ltx@IfUndefined}
%    \begin{macrocode}
  \def\ltx@IfUndefined#1{%
    \begingroup\expandafter\expandafter\expandafter\endgroup
    \expandafter\ifx\csname #1\endcsname\relax
      \expandafter\ltx@firstoftwo
    \else
      \expandafter\ltx@secondoftwo
    \fi
  }%
%    \end{macrocode}
%    \end{macro}
%    \begin{macrocode}
  \expandafter\ltx@gobble
\else
  \expandafter\ltx@firstofone
\fi
{%
%    \end{macrocode}
%    \begin{macro}{\ltx@ifundefined}
%    \begin{macrocode}
  \def\ltx@ifundefined#1{%
    \ifcsname #1\endcsname
      \expandafter\ifx\csname #1\endcsname\relax
        \expandafter\expandafter\expandafter\ltx@firstoftwo
      \else
        \expandafter\expandafter\expandafter\ltx@secondoftwo
      \fi
    \else
      \expandafter\ltx@firstoftwo
    \fi
  }%
%    \end{macrocode}
%    \end{macro}
%    \begin{macro}{\ltx@IfUndefined}
%    \begin{macrocode}
  \let\ltx@IfUndefined\ltx@ifundefined
%    \end{macrocode}
%    \end{macro}
%    \begin{macrocode}
}
%    \end{macrocode}
%
% \subsection{Stripping}
%
%    \begin{macro}{\ltx@RemovePrefix}
%    \begin{macrocode}
\def\ltx@RemovePrefix#1>{}
%    \end{macrocode}
%    \end{macro}
%
%    \begin{macro}{\ltx@StripPrefix}
%    \begin{macrocode}
\def\ltx@StripPrefix{%
  \expandafter\ltx@RemovePrefix
}
%    \end{macrocode}
%    \end{macro}
%
%    \begin{macro}{\ltx@onelevel@sanitize}
%    \begin{macrocode}
\def\ltx@onelevel@sanitize#1{%
  \edef#1{%
    \expandafter
    \ltx@RemovePrefix\meaning#1%
  }%
}
%    \end{macrocode}
%    \end{macro}
%
% \subsection{File management}
%
% \subsubsection{File extensions}
%
%    \begin{macro}{\ltx@clsextension}
%    \begin{macrocode}
\def\ltx@clsextension{cls}
%    \end{macrocode}
%    \end{macro}
%    \begin{macro}{\ltx@pkgextension}
%    \begin{macrocode}
\def\ltx@pkgextension{sty}
%    \end{macrocode}
%    \end{macro}
%
% \subsubsection{Load check}
%
%    \begin{macro}{\ltx@iffileloaded}
%    \begin{macrocode}
\def\ltx@iffileloaded#1{%
  \ltx@ifundefined{ver@#1}\ltx@secondoftwo\ltx@firstoftwo
}
%    \end{macrocode}
%    \end{macro}
%    \begin{macro}{\ltx@ifclassloaded}
%    \begin{macrocode}
\def\ltx@ifclassloaded#1{%
  \ltx@iffileloaded{#1.\ltx@clsextension}%
}
%    \end{macrocode}
%    \end{macro}
%    \begin{macro}{\ltx@ifpackageloaded}
%    \begin{macrocode}
\def\ltx@ifpackageloaded#1{%
  \ltx@iffileloaded{#1.\ltx@pkgextension}%
}
%    \end{macrocode}
%    \end{macro}
%
% \subsubsection{Version date check}
%
%    \begin{macro}{\ltx@iffilelater}
%    \begin{macrocode}
\def\ltx@iffilelater#1#2{%
  \ltx@iffileloaded{#1}{%
    \expandafter\LTXcmds@IfLater\expandafter{%
      \number
      \expandafter\expandafter\expandafter\LTXcmds@ParseVersion
      \expandafter\expandafter\expandafter{%
        \csname ver@#1\endcsname
      }%
    \expandafter}\expandafter{%
      \number
      \expandafter\LTXcmds@ParseVersion\expandafter{#2}%
    }%
  }\ltx@secondoftwo
}
%    \end{macrocode}
%    \end{macro}
%    \begin{macro}{\LTXcmds@IfLater}
%    \begin{macrocode}
\def\LTXcmds@IfLater#1#2{%
  \ifcase 0%
      \ifnum#1<19940101 %
      \else
        \ifnum#2<19940101 %
        \else
          \ifnum#2>#1 %
          \else
            1%
          \fi
        \fi
      \fi
      \ltx@space
    \expandafter\ltx@secondoftwo
  \else
    \expandafter\ltx@firstoftwo
  \fi
}
%    \end{macrocode}
%    \end{macro}
%    \begin{macro}{\ltx@ifclasslater}
%    \begin{macrocode}
\def\ltx@ifclasslater#1{%
  \ltx@iffilelater{#1.\ltx@clsextension}%
}
%    \end{macrocode}
%    \end{macro}
%    \begin{macro}{\ltx@ifpackagelater}
%    \begin{macrocode}
\def\ltx@ifpackagelater#1{%
  \ltx@iffilelater{#1.\ltx@pkgextension}%
}
%    \end{macrocode}
%    \end{macro}
%
%    \begin{macrocode}
\ltx@IfUndefined{pdfmatch}{%
%    \end{macrocode}
%    \begin{macro}{\LTXcmds@ParseVersion}
%    \begin{macrocode}
  \def\LTXcmds@ParseVersion#1{%
    \LTXcmds@@ParseVersion#10000/00/00\@nil
  }%
%    \end{macrocode}
%    \end{macro}
%    \begin{macro}{\LTXcmds@@ParseVersion}
%    \begin{macrocode}
  \def\LTXcmds@@ParseVersion#1#2#3#4/#5#6/#7#8#9\@nil{%
    #1#2#3#4#5#6#7#8%
  }%
%    \end{macrocode}
%    \end{macro}
%    \begin{macrocode}
}{%
%    \end{macrocode}
%    \begin{macro}{\LTXcmds@ParseVersion}
%    \begin{macrocode}
  \def\LTXcmds@ParseVersion#1{%
    \ifnum\pdfmatch{%
      ^%
      (199[4-9]|[2-9][0-9][0-9][0-9])/%
      (0[1-9]|1[0-2])/%
      (0[1-9]|[1-2][0-9]|3[0-1])%
    }{#1}=1 %
      \ltx@StripPrefix\pdflastmatch1 %
      \ltx@StripPrefix\pdflastmatch2 %
      \ltx@StripPrefix\pdflastmatch3 %
    \else
      0%
    \fi
  }%
%    \end{macrocode}
%    \end{macro}
%    \begin{macrocode}
}
%    \end{macrocode}
%
% \subsection{Macro additions}
%
%    \begin{macro}{\ltx@GlobalAppendToMacro}
%    \begin{macrocode}
\long\def\ltx@GlobalAppendToMacro#1#2{%
  \ifx\ltx@undefined#1%
    \let#1\ltx@empty
  \else
    \ifx\relax#1%
      \let#1\ltx@empty
    \fi
  \fi
  \begingroup
    \ltx@LocToksA\expandafter{#1#2}%
    \xdef#1{\the\ltx@LocToksA}%
  \endgroup
}
%    \end{macrocode}
%    \end{macro}
%    \begin{macro}{\ltx@LocalAppendToMacro}
%    \begin{macrocode}
\long\def\ltx@LocalAppendToMacro#1#2{%
  \global\let\LTXcmds@gtemp#1%
  \ifx\ltx@undefined\LTXcmds@gtemp
    \global\let\LTXcmds@gtemp\ltx@empty
  \else
    \ifx\relax\LTXcmds@gtemp
      \global\letLTXcmds@gtemp\ltx@empty
    \fi
  \fi
  \begingroup
    \ltx@LocToksA\expandafter{\LTXcmds@gtemp#2}%
    \xdef\LTXcmds@gtemp{\the\ltx@LocToksA}%
  \endgroup
  \let#1\LTXcmds@gtemp
}
%    \end{macrocode}
%    \end{macro}
%
%    \begin{macro}{\ltx@GlobalPrependToMacro}
%    \begin{macrocode}
\long\def\ltx@GlobalPrependToMacro#1#2{%
  \ifx\ltx@undefined#1%
    \let#1\ltx@empty
  \else
    \ifx\relax#1%
      \let#1\ltx@empty
    \fi
  \fi
  \begingroup
    \ltx@LocToksA{#2}%
    \ltx@LocToksB\expandafter{#1}%
    \xdef#1{\the\ltx@LocToksA\the\ltx@LocToksB}%
  \endgroup
}
%    \end{macrocode}
%    \end{macro}
%    \begin{macro}{\ltx@LocalPrependToMacro}
%    \begin{macrocode}
\long\def\ltx@LocalPrependToMacro#1#2{%
  \global\let\LTXcmds@gtemp#1%
  \ifx\ltx@undefined\LTXcmds@gtemp
    \global\let\LTXcmds@gtemp\ltx@empty
  \else
    \ifx\relax\LTXcmds@gtemp
      \global\letLTXcmds@gtemp\ltx@empty
    \fi
  \fi
  \begingroup
    \ltx@LocToksA{#2}%
    \ltx@LocToksB\expandafter{\LTXcmds@gtemp}%
    \xdef\LTXcmds@gtemp{\the\ltx@LocToksA\the\ltx@LocToksB}%
  \endgroup
  \let#1\LTXcmds@gtemp
}
%    \end{macrocode}
%    \end{macro}
%
% \subsection{Next character detection}
%
%    \begin{macro}{\ltx@ifnextchar}
%    \begin{macrocode}
\long\def\ltx@ifnextchar#1#2#3{%
  \begingroup
  \let\LTXcmds@CharToken= #1\relax
  \ltx@LocToksA{\endgroup#2}%
  \ltx@LocToksB{\endgroup#3}%
  \futurelet\LTXcmds@LetToken\LTXcmds@ifnextchar
}
%    \end{macrocode}
%    \end{macro}
%    \begin{macro}{\LTXcmds@ifnextchar}
%    \begin{macrocode}
\def\LTXcmds@ifnextchar{%
  \ifx\LTXcmds@LetToken\LTXcmds@CharToken
    \the\expandafter\ltx@LocToksA
  \else
    \expandafter
      \ifx\csname LTXcmds@LetToken\endcsname\LTXcmds@SpaceToken
      \expandafter\expandafter\expandafter\LTXcmds@@ifnextchar
    \else
      \the\expandafter\expandafter\expandafter\ltx@LocToksB
    \fi
  \fi
}
%    \end{macrocode}
%    \end{macro}
%    \begin{macro}{\LTXcmds@@ifnextchar}
%    \cs{futurelet} does not distinguish between a
%    character and a command that is a character
%    (defined by using \cs{let} or \cs{futurelet}).
%    Therefore the space is catched by \cs{romannumeral}
%    with negative character constant that gobbles one
%    optional space.
%    \begin{macrocode}
\def\LTXcmds@@ifnextchar{%
  \expandafter\futurelet
  \expandafter\LTXcmds@LetToken
  \expandafter\LTXcmds@ifnextchar
  \romannumeral-`\.%
}
%    \end{macrocode}
%    \end{macro}
%    \begin{macro}{\LTXcmds@SpaceToken}
%    \begin{macrocode}
\ltx@firstofone{\let\LTXcmds@SpaceToken= } %
%    \end{macrocode}
%    \end{macro}
%
%    \begin{macro}{\ltx@ifnextchar@nospace}
%    \begin{macrocode}
\long\def\ltx@ifnextchar@nospace#1#2#3{%
  \begingroup
  \let\LTXcmds@CharToken= #1\relax
  \ltx@LocToksA{\endgroup#2}%
  \ltx@LocToksB{\endgroup#3}%
  \futurelet\LTXcmds@LetToken\LTXcmds@ifnextchar@nospace
}
%    \end{macrocode}
%    \end{macro}
%    \begin{macro}{\LTXcmds@ifnextchar@nospace}
%    \begin{macrocode}
\def\LTXcmds@ifnextchar@nospace{%
  \the
  \ifx\LTXcmds@LetToken\LTXcmds@CharToken
    \expandafter\ltx@LocToksA
  \else
    \expandafter\ltx@LocToksB
  \fi
}
%    \end{macrocode}
%    \end{macro}
% \subsection{\cs{ltx@leavevmode}, \cs{ltx@mbox}}
%
%    \begin{macro}{\ltx@leavevmode}
%    \begin{macrocode}
\ltx@IfUndefined{quitvmode}{%
  \ltx@IfUndefined{leavevmode}{%
    \ltx@IfUndefined{voidb@x}{%
      \ltx@IfUndefined{newbox}{%
        \def\ltx@leavevmode{%
          \begingroup
            \setbox\ltx@zero=\hbox{}%
            \begingroup
              \setbox\ltx@zero=\hbox{\box\ltx@zero}%
            \endgroup
            \unhbox\ltx@zero
          \endgroup
        }%
      }{%
        \csname newbox\endcsname\LTXcmds@VoidBox
        \ifvoid\LTXcmds@VoidBox
        \else
          \setbox\LTXcmds@VoidBox=\hbox{}%
          \begingroup
            \setbox\LTXcmds@VoidBox=\hbox{\box\LTXcmds@VoidBox}%
          \endgroup
        \fi
        \def\ltx@leavevmode{\unhbox\LTXcmds@VoidBox}%
      }%
    }{%
      \def\ltx@leavevmode{\unhbox\voidb@x}%
    }%
  }{%
    \let\ltx@leavevmode\leavevmode
  }%
}{%
  \let\ltx@leavevmode\quitvmode
}
%    \end{macrocode}
%    \end{macro}
%
%    \begin{macro}{\ltx@mbox}
%    \begin{macrocode}
\def\ltx@mbox{%
  \ltx@leavevmode
  \hbox
}
%    \end{macrocode}
%    \end{macro}
%
% \subsection{Help macros}
%
%    \begin{macro}{\LTXcmds@num}
%    \begin{macrocode}
\ltx@IfUndefined{numexpr}{%
  \def\LTXcmds@num#1{%
    \expandafter\ltx@firstofone\expandafter{%
      \number#1%
    }%
  }%
}{%
  \def\LTXcmds@num#1{%
    \expandafter\ltx@firstofone\expandafter{%
      \the\numexpr#1%
    }%
  }%
}
%    \end{macrocode}
%    \end{macro}
%
% \subsection{Expandable test for emptiness}
%
%    \begin{macrocode}
\ltx@IfUndefined{detokenize}{%
%    \end{macrocode}
%
% \subsubsection{Vanilla \hologo{TeX}}
%
%    \begin{macro}{\ltx@ifempty}
%    The macro is based on \cs{@ifempty} of Robert R. Schneck
%    \cite{ifempty:schneck}
%    and \cs{@ifnull} of Ulrich Diez \cite{ifnull:diez}.
%    There are three cases to consider:
%    \begin{compactenum}
%    \item |#1| is empty,
%    \item |#1| is not empty and the first token
%      is not a begingroup character,
%    \item |#1| starts with a begingroup character (catcode 1).
%    \end{compactenum}
%    \begin{macrocode}
  \def\LTXcmds@temp#1{%
    \long\def\ltx@ifempty##1{%
      \romannumeral0%
      \iffalse{\fi
        \expandafter\ltx@gobble\expandafter{%
          \expandafter{\string##1}%
          \expandafter\ltx@gobble\string
        }%
        \expandafter\ltx@firstofthree\expandafter
        {\iffalse}\fi
        \expandafter#1\ltx@secondoftwo
      }%
      \expandafter#1\ltx@firstoftwo
    }%
%    \end{macrocode}
%    \end{macro}
%    \begin{macro}{\ltx@ifblank}
%    \begin{macrocode}
    \long\def\ltx@ifblank##1{%
      \romannumeral0%
      \iffalse{\fi
        \expandafter\expandafter\expandafter\ltx@gobble
        \expandafter\expandafter\expandafter{%
          \expandafter\expandafter\expandafter{%
            \expandafter\string\ltx@gobble##1.%
          }%
          \expandafter\ltx@gobble\string
        }%
        \expandafter\ltx@firstofthree\expandafter
        {\iffalse}\fi
        \expandafter#1\ltx@secondoftwo
      }%
      \expandafter#1\ltx@firstoftwo
    }%
  }%
  \LTXcmds@temp{ }%
%    \end{macrocode}
%    \end{macro}
%    \begin{macrocode}
}{%
%    \end{macrocode}
%
% \subsubsection{With \cs{detokenize}}
%
%    Ahmed Musa provided \cs{ifstrempty} using \cs{detokenize}
%    and \cs{pdfstrcmp} \cite{ifstrempty:musa}. Ulrich Diez, GL,
%    Heiko Oberdiek
%    improved it further by removing \cs{pdfstrcmp} and
%    taking three arguments
%    \cite{ifstrempty:diez1,ifstrempty:diez2,ifstrempty:diez3,^^A
%      ifstrempty:gl,ifstrempty:ho}.
%
%    \begin{macro}{\ltx@ifempty}
%    \begin{macrocode}
  \long\def\ltx@ifempty#1{%
    \romannumeral%
    \csname
      LTXcmds@ifempty%
      \ifcat$\detokenize{#1}$%
        @%
      \fi
    \endcsname
  }%
%    \end{macrocode}
%    \end{macro}
%    \begin{macro}{\LTXcmds@ifempty@}
%    \begin{macrocode}
  \long\def\LTXcmds@ifempty@#1#2{0 #1}%
%    \end{macrocode}
%    \end{macro}
%    \begin{macro}{\LTXcmds@ifempty}
%    \begin{macrocode}
  \long\def\LTXcmds@ifempty#1#2{0 #2}%
%    \end{macrocode}
%    \end{macro}
%
% \subsubsection{\cs{ltx@ifblank}}
%
%    \begin{macro}{\ltx@ifblank}
%    \begin{macrocode}
  \long\def\ltx@ifblank#1{%
    \romannumeral%
    \csname
      LTXcmds@ifempty%
      \ifcat$\detokenize\expandafter{\ltx@gobble#1.}$%
        @%
      \fi
    \endcsname
  }%
%    \end{macrocode}
%    \end{macro}
%
%    \begin{macrocode}
}
%    \end{macrocode}
%
% \subsection{\cs{ltx@zapspace}}
%
%    \begin{macro}{\ltx@zapspace}
%    \begin{macrocode}
\long\def\ltx@zapspace#1{%
  \romannumeral
  \LTXcmds@zapspace\ltx@zero#1 \@nil
}
%    \end{macrocode}
%    \end{macro}
%    \begin{macro}{\LTXcmds@zapspace}
%    \begin{macrocode}
\long\def\LTXcmds@zapspace#1 #2\@nil{%
  \ltx@ifempty{#2}{%
    #1%
  }{%
    \LTXcmds@zapspace#1#2\@nil
  }%
}
%    \end{macrocode}
%    \end{macro}
%
% \subsection{\cs{ltx@IfBoxEmpty}}
%
%    In case of \hologo{eTeX} the test for an empty box
%    is done via \cs{lastnodetype} as suggested
%    by David Kastrup \cite{ifboxempty:dak}.
%    \begin{macrocode}
\ltx@IfUndefined{lastnodetype}{%
  \catcode`\$=9 %
  \catcode`\&=14 %
}{%
  \catcode`\$=14 %
  \catcode`\&=9 %
}
%    \end{macrocode}
%    \begin{macro}{\ltx@IfBoxEmpty}
%    \begin{macrocode}
\def\ltx@IfBoxEmpty#1{%
  \ifvoid#1\relax
    \expandafter\ltx@secondoftwo
  \else
%    \end{macrocode}
%    Implementation using \hologo{eTeX}'s \cs{lastnodetype}.
%    \begin{macrocode}
&   \begingroup
&     \setbox\ltx@zero=\ifhbox#1\hbox\else\vbox\fi{%
&       \ifhmode\unhcopy\else\unvcopy\fi#1\relax
&       \expandafter
&     }%
&   \expandafter\endgroup
&   \ifnum\lastnodetype<\ltx@zero
&     \expandafter\expandafter\expandafter\ltx@firstoftwo
&   \else
&     \expandafter\expandafter\expandafter\ltx@secondoftwo
&   \fi
%    \end{macrocode}
%    Implementation without \hologo{eTeX} using a signature
%    at the beginning of the test box.
%    \begin{macrocode}
$   \begingroup
$     \setbox\ltx@zero=\ifhbox#1\hbox\else\vbox\fi{%
$       \penalty\ltx@one
$       \ifhmode\unhcopy\else\unvcopy\fi#1\relax
$       \expandafter
$     }%
$     \ifnum\lastpenalty=\ltx@one
%    \end{macrocode}
%    Box 0 has been changed and is restored by
%    closing the group.
%    \begin{macrocode}
$       \endgroup
$       \begingroup
$       \setbox\ltx@zero=\ifhbox#1\hbox\else\vbox\fi{%
$         \penalty\ltx@two
$         \ifhmode\unhcopy\else\unvcopy\fi#1\relax
$         \expandafter
$       }%
$       \ifnum\lastpenalty=\ltx@two
$         \def\next{\endgroup\expandafter\ltx@firstoftwo}%
$       \else
$         \def\next{\endgroup\expandafter\ltx@secondoftwo}%
$       \fi
$     \else
$       \def\next{\endgroup\expandafter\ltx@secondoftwo}%
$     \fi
$   \next
  \fi
}
%    \end{macrocode}
%    \end{macro}
%
%    \begin{macro}{\ltx@IfBoxVoidOrEmpty}
%    \begin{macrocode}
\def\ltx@IfBoxVoidOrEmpty#1{%
  \ifvoid#1\relax
    \expandafter\ltx@thirdoffour
  \fi
  \ltx@IfBoxEmpty{#1}%
}
%    \end{macrocode}
%    \end{macro}
%
%    \begin{macrocode}
\LTXcmds@AtEnd%
%</package>
%    \end{macrocode}
%% \section{Installation}
%
% \subsection{Download}
%
% \paragraph{Package.} This package is available on
% CTAN\footnote{\CTANpkg{ltxcmds}}:
% \begin{description}
% \item[\CTAN{macros/latex/contrib/ltxcmds/ltxcmds.dtx}] The source file.
% \item[\CTAN{macros/latex/contrib/ltxcmds/ltxcmds.pdf}] Documentation.
% \end{description}
%
%
% \paragraph{Bundle.} All the packages of the bundle `ltxcmds'
% are also available in a TDS compliant ZIP archive. There
% the packages are already unpacked and the documentation files
% are generated. The files and directories obey the TDS standard.
% \begin{description}
% \item[\CTANinstall{install/macros/latex/contrib/ltxcmds.tds.zip}]
% \end{description}
% \emph{TDS} refers to the standard ``A Directory Structure
% for \TeX\ Files'' (\CTANpkg{tds}). Directories
% with \xfile{texmf} in their name are usually organized this way.
%
% \subsection{Bundle installation}
%
% \paragraph{Unpacking.} Unpack the \xfile{ltxcmds.tds.zip} in the
% TDS tree (also known as \xfile{texmf} tree) of your choice.
% Example (linux):
% \begin{quote}
%   |unzip ltxcmds.tds.zip -d ~/texmf|
% \end{quote}
%
% \subsection{Package installation}
%
% \paragraph{Unpacking.} The \xfile{.dtx} file is a self-extracting
% \docstrip\ archive. The files are extracted by running the
% \xfile{.dtx} through \plainTeX:
% \begin{quote}
%   \verb|tex ltxcmds.dtx|
% \end{quote}
%
% \paragraph{TDS.} Now the different files must be moved into
% the different directories in your installation TDS tree
% (also known as \xfile{texmf} tree):
% \begin{quote}
% \def\t{^^A
% \begin{tabular}{@{}>{\ttfamily}l@{ $\rightarrow$ }>{\ttfamily}l@{}}
%   ltxcmds.sty & tex/generic/ltxcmds/ltxcmds.sty\\
%   ltxcmds.pdf & doc/latex/ltxcmds/ltxcmds.pdf\\
%   ltxcmds.dtx & source/latex/ltxcmds/ltxcmds.dtx\\
% \end{tabular}^^A
% }^^A
% \sbox0{\t}^^A
% \ifdim\wd0>\linewidth
%   \begingroup
%     \advance\linewidth by\leftmargin
%     \advance\linewidth by\rightmargin
%   \edef\x{\endgroup
%     \def\noexpand\lw{\the\linewidth}^^A
%   }\x
%   \def\lwbox{^^A
%     \leavevmode
%     \hbox to \linewidth{^^A
%       \kern-\leftmargin\relax
%       \hss
%       \usebox0
%       \hss
%       \kern-\rightmargin\relax
%     }^^A
%   }^^A
%   \ifdim\wd0>\lw
%     \sbox0{\small\t}^^A
%     \ifdim\wd0>\linewidth
%       \ifdim\wd0>\lw
%         \sbox0{\footnotesize\t}^^A
%         \ifdim\wd0>\linewidth
%           \ifdim\wd0>\lw
%             \sbox0{\scriptsize\t}^^A
%             \ifdim\wd0>\linewidth
%               \ifdim\wd0>\lw
%                 \sbox0{\tiny\t}^^A
%                 \ifdim\wd0>\linewidth
%                   \lwbox
%                 \else
%                   \usebox0
%                 \fi
%               \else
%                 \lwbox
%               \fi
%             \else
%               \usebox0
%             \fi
%           \else
%             \lwbox
%           \fi
%         \else
%           \usebox0
%         \fi
%       \else
%         \lwbox
%       \fi
%     \else
%       \usebox0
%     \fi
%   \else
%     \lwbox
%   \fi
% \else
%   \usebox0
% \fi
% \end{quote}
% If you have a \xfile{docstrip.cfg} that configures and enables \docstrip's
% TDS installing feature, then some files can already be in the right
% place, see the documentation of \docstrip.
%
% \subsection{Refresh file name databases}
%
% If your \TeX~distribution
% (\TeX\,Live, \mikTeX, \dots) relies on file name databases, you must refresh
% these. For example, \TeX\,Live\ users run \verb|texhash| or
% \verb|mktexlsr|.
%
% \subsection{Some details for the interested}
%
% \paragraph{Unpacking with \LaTeX.}
% The \xfile{.dtx} chooses its action depending on the format:
% \begin{description}
% \item[\plainTeX:] Run \docstrip\ and extract the files.
% \item[\LaTeX:] Generate the documentation.
% \end{description}
% If you insist on using \LaTeX\ for \docstrip\ (really,
% \docstrip\ does not need \LaTeX), then inform the autodetect routine
% about your intention:
% \begin{quote}
%   \verb|latex \let\install=y% \iffalse meta-comment
%
% File: ltxcmds.dtx
% Version: 2019/12/15 v1.24
% Info: LaTeX kernel commands for general use
%
% Copyright (C)
%    2009-2011 Heiko Oberdiek
%    2016-2019 Oberdiek Package Support Group
%    https://github.com/ho-tex/ltxcmds/issues
%
% This work may be distributed and/or modified under the
% conditions of the LaTeX Project Public License, either
% version 1.3c of this license or (at your option) any later
% version. This version of this license is in
%    https://www.latex-project.org/lppl/lppl-1-3c.txt
% and the latest version of this license is in
%    https://www.latex-project.org/lppl.txt
% and version 1.3 or later is part of all distributions of
% LaTeX version 2005/12/01 or later.
%
% This work has the LPPL maintenance status "maintained".
%
% The Current Maintainers of this work are
% Heiko Oberdiek and the Oberdiek Package Support Group
% https://github.com/ho-tex/ltxcmds/issues
%
% The Base Interpreter refers to any `TeX-Format',
% because some files are installed in TDS:tex/generic//.
%
% This work consists of the main source file ltxcmds.dtx
% and the derived files
%    ltxcmds.sty, ltxcmds.pdf, ltxcmds.ins, ltxcmds.drv,
%    ltxcmds-test1.tex, ltxcmds-test-gobble.tex,
%    ltxcmds-test-ifempty.tex, ltxcmds-test-zapspace.tex,
%    ltxcmds-test-ifboxempty.tex, ltxcmds-test-nextchar.tex,
%    ltxcmds-test-carcdr.tex.
%
% Distribution:
%    CTAN:macros/latex/contrib/ltxcmds/ltxcmds.dtx
%    CTAN:macros/latex/contrib/ltxcmds/ltxcmds.pdf
%
% Unpacking:
%    (a) If ltxcmds.ins is present:
%           tex ltxcmds.ins
%    (b) Without ltxcmds.ins:
%           tex ltxcmds.dtx
%    (c) If you insist on using LaTeX
%           latex \let\install=y% \iffalse meta-comment
%
% File: ltxcmds.dtx
% Version: 2019/12/15 v1.24
% Info: LaTeX kernel commands for general use
%
% Copyright (C)
%    2009-2011 Heiko Oberdiek
%    2016-2019 Oberdiek Package Support Group
%    https://github.com/ho-tex/ltxcmds/issues
%
% This work may be distributed and/or modified under the
% conditions of the LaTeX Project Public License, either
% version 1.3c of this license or (at your option) any later
% version. This version of this license is in
%    https://www.latex-project.org/lppl/lppl-1-3c.txt
% and the latest version of this license is in
%    https://www.latex-project.org/lppl.txt
% and version 1.3 or later is part of all distributions of
% LaTeX version 2005/12/01 or later.
%
% This work has the LPPL maintenance status "maintained".
%
% The Current Maintainers of this work are
% Heiko Oberdiek and the Oberdiek Package Support Group
% https://github.com/ho-tex/ltxcmds/issues
%
% The Base Interpreter refers to any `TeX-Format',
% because some files are installed in TDS:tex/generic//.
%
% This work consists of the main source file ltxcmds.dtx
% and the derived files
%    ltxcmds.sty, ltxcmds.pdf, ltxcmds.ins, ltxcmds.drv,
%    ltxcmds-test1.tex, ltxcmds-test-gobble.tex,
%    ltxcmds-test-ifempty.tex, ltxcmds-test-zapspace.tex,
%    ltxcmds-test-ifboxempty.tex, ltxcmds-test-nextchar.tex,
%    ltxcmds-test-carcdr.tex.
%
% Distribution:
%    CTAN:macros/latex/contrib/ltxcmds/ltxcmds.dtx
%    CTAN:macros/latex/contrib/ltxcmds/ltxcmds.pdf
%
% Unpacking:
%    (a) If ltxcmds.ins is present:
%           tex ltxcmds.ins
%    (b) Without ltxcmds.ins:
%           tex ltxcmds.dtx
%    (c) If you insist on using LaTeX
%           latex \let\install=y% \iffalse meta-comment
%
% File: ltxcmds.dtx
% Version: 2019/12/15 v1.24
% Info: LaTeX kernel commands for general use
%
% Copyright (C)
%    2009-2011 Heiko Oberdiek
%    2016-2019 Oberdiek Package Support Group
%    https://github.com/ho-tex/ltxcmds/issues
%
% This work may be distributed and/or modified under the
% conditions of the LaTeX Project Public License, either
% version 1.3c of this license or (at your option) any later
% version. This version of this license is in
%    https://www.latex-project.org/lppl/lppl-1-3c.txt
% and the latest version of this license is in
%    https://www.latex-project.org/lppl.txt
% and version 1.3 or later is part of all distributions of
% LaTeX version 2005/12/01 or later.
%
% This work has the LPPL maintenance status "maintained".
%
% The Current Maintainers of this work are
% Heiko Oberdiek and the Oberdiek Package Support Group
% https://github.com/ho-tex/ltxcmds/issues
%
% The Base Interpreter refers to any `TeX-Format',
% because some files are installed in TDS:tex/generic//.
%
% This work consists of the main source file ltxcmds.dtx
% and the derived files
%    ltxcmds.sty, ltxcmds.pdf, ltxcmds.ins, ltxcmds.drv,
%    ltxcmds-test1.tex, ltxcmds-test-gobble.tex,
%    ltxcmds-test-ifempty.tex, ltxcmds-test-zapspace.tex,
%    ltxcmds-test-ifboxempty.tex, ltxcmds-test-nextchar.tex,
%    ltxcmds-test-carcdr.tex.
%
% Distribution:
%    CTAN:macros/latex/contrib/ltxcmds/ltxcmds.dtx
%    CTAN:macros/latex/contrib/ltxcmds/ltxcmds.pdf
%
% Unpacking:
%    (a) If ltxcmds.ins is present:
%           tex ltxcmds.ins
%    (b) Without ltxcmds.ins:
%           tex ltxcmds.dtx
%    (c) If you insist on using LaTeX
%           latex \let\install=y\input{ltxcmds.dtx}
%        (quote the arguments according to the demands of your shell)
%
% Documentation:
%    (a) If ltxcmds.drv is present:
%           latex ltxcmds.drv
%    (b) Without ltxcmds.drv:
%           latex ltxcmds.dtx; ...
%    The class ltxdoc loads the configuration file ltxdoc.cfg
%    if available. Here you can specify further options, e.g.
%    use A4 as paper format:
%       \PassOptionsToClass{a4paper}{article}
%
%    Programm calls to get the documentation (example):
%       pdflatex ltxcmds.dtx
%       makeindex -s gind.ist ltxcmds.idx
%       pdflatex ltxcmds.dtx
%       makeindex -s gind.ist ltxcmds.idx
%       pdflatex ltxcmds.dtx
%
% Installation:
%    TDS:tex/generic/ltxcmds/ltxcmds.sty
%    TDS:doc/latex/ltxcmds/ltxcmds.pdf
%    TDS:source/latex/ltxcmds/ltxcmds.dtx
%
%<*ignore>
\begingroup
  \catcode123=1 %
  \catcode125=2 %
  \def\x{LaTeX2e}%
\expandafter\endgroup
\ifcase 0\ifx\install y1\fi\expandafter
         \ifx\csname processbatchFile\endcsname\relax\else1\fi
         \ifx\fmtname\x\else 1\fi\relax
\else\csname fi\endcsname
%</ignore>
%<*install>
\input docstrip.tex
\Msg{************************************************************************}
\Msg{* Installation}
\Msg{* Package: ltxcmds 2019/12/15 v1.24 LaTeX kernel commands for general use (HO)}
\Msg{************************************************************************}

\keepsilent
\askforoverwritefalse

\let\MetaPrefix\relax
\preamble

This is a generated file.

Project: ltxcmds
Version: 2019/12/15 v1.24

Copyright (C)
   2009-2011 Heiko Oberdiek
   2016-2019 Oberdiek Package Support Group

This work may be distributed and/or modified under the
conditions of the LaTeX Project Public License, either
version 1.3c of this license or (at your option) any later
version. This version of this license is in
   https://www.latex-project.org/lppl/lppl-1-3c.txt
and the latest version of this license is in
   https://www.latex-project.org/lppl.txt
and version 1.3 or later is part of all distributions of
LaTeX version 2005/12/01 or later.

This work has the LPPL maintenance status "maintained".

The Current Maintainers of this work are
Heiko Oberdiek and the Oberdiek Package Support Group
https://github.com/ho-tex/ltxcmds/issues


The Base Interpreter refers to any `TeX-Format',
because some files are installed in TDS:tex/generic//.

This work consists of the main source file ltxcmds.dtx
and the derived files
   ltxcmds.sty, ltxcmds.pdf, ltxcmds.ins, ltxcmds.drv,
   ltxcmds-test1.tex, ltxcmds-test-gobble.tex,
   ltxcmds-test-ifempty.tex, ltxcmds-test-zapspace.tex,
   ltxcmds-test-ifboxempty.tex, ltxcmds-test-nextchar.tex,
   ltxcmds-test-carcdr.tex.

\endpreamble
\let\MetaPrefix\DoubleperCent

\generate{%
  \file{ltxcmds.ins}{\from{ltxcmds.dtx}{install}}%
  \file{ltxcmds.drv}{\from{ltxcmds.dtx}{driver}}%
  \usedir{tex/generic/ltxcmds}%
  \file{ltxcmds.sty}{\from{ltxcmds.dtx}{package}}%
}

\catcode32=13\relax% active space
\let =\space%
\Msg{************************************************************************}
\Msg{*}
\Msg{* To finish the installation you have to move the following}
\Msg{* file into a directory searched by TeX:}
\Msg{*}
\Msg{*     ltxcmds.sty}
\Msg{*}
\Msg{* To produce the documentation run the file `ltxcmds.drv'}
\Msg{* through LaTeX.}
\Msg{*}
\Msg{* Happy TeXing!}
\Msg{*}
\Msg{************************************************************************}

\endbatchfile
%</install>
%<*ignore>
\fi
%</ignore>
%<*driver>
\NeedsTeXFormat{LaTeX2e}
\ProvidesFile{ltxcmds.drv}%
  [2019/12/15 v1.24 LaTeX kernel commands for general use (HO)]%
\documentclass{ltxdoc}
\usepackage{holtxdoc}[2011/11/22]
\usepackage{zref-savepos}
\usepackage{paralist}
\hfuzz=1pt
\begin{document}
  \DocInput{ltxcmds.dtx}%
\end{document}
%</driver>
% \fi
%
%
%
% \GetFileInfo{ltxcmds.drv}
%
% \title{The \xpackage{ltxcmds} package}
% \date{2019/12/15 v1.24}
% \author{Heiko Oberdiek\thanks
% {Please report any issues at \url{https://github.com/ho-tex/ltxcmds/issues}}}
%
% \maketitle
%
% \begin{abstract}
% The package \xpackage{ltxcmds} exports some utility macros
% from the \LaTeX\ kernel into a separate namespace and
% also provides them for other formats such as plain-\TeX.
% \end{abstract}
%
% \tableofcontents
%
% \newcounter{to}
% \renewcommand*{\theto}{to\arabic{to}}
% \newlength\tolength
% \tolength=\dimexpr .5\linewidth+\oddsidemargin+1in+15mm\relax
% \newcommand*{\TO}{^^A
%   \quad
%   \stepcounter{to}^^A
%   \zsavepos\theto
%   \ifdim\zposx\theto sp>0pt
%     \ifdim\zposx\theto sp<\tolength
%       \kern\dimexpr\tolength-\zposx\theto sp\relax
%     \fi
%   \fi
%   $\rightarrow$\quad
%   \ignorespaces
% }
%
% \section{Documentation}
%
% \subsection{Introduction}
%
% Many of my packages also support other formats such as plain-\TeX.
% Because I am rather familiar with the utility macros from
% \LaTeX's kernel (e.g. \cs{@gobble}, \cs{@firstoftwo}), I found
% myself rewriting them again and again, because they are lacking
% in plain-\TeX.
%
% Therefore this package provides often used macros and similar
% ones with the name prefix \cs{ltx@}. This avoids also faulty
% redefinitions. I remember an example where a package redefined
% \cs{@firstoftwo} with forgetting \cs{long}.
%
% \subsection{Numbers}
%
% \begin{declcs}{ltx@zero} \TO 0\\
%   \cs{ltx@one} \TO 1\\
%   \cs{ltx@two} \TO 2\\
%   \cs{ltx@cclv} \TO 255\\
%   \cs{ltx@minusone} \TO -1
% \end{declcs}
% These commands are numbers 0, 1, 2, 255 and -1. They are not digits
% and a space is not gobbled afterwards. Macro \cs{ltx@minusone}
% is available since version 2010/12/12 v1.15.
%
% \subsection{Scratch registers}
%
% Following the conventions of \hologo{plainTeX} and \hologo{LaTeX}
% the first ten registers are free to use. Even numbered registers
% are for local, odd numbered for global use.
%
% \begin{declcs}{ltx@(Loc,Glob)(Toks,Dimen,Skip)(A,B,C,D,E)}
% \end{declcs}
% The name consists of the prefix \cs{ltx@}, then
% \texttt{Loc} or \texttt{Glob} for local or global usage follows.
% The register type is given by \texttt{Toks} for token register,
% \texttt{Dimen} for dimen register and \texttt{Skip} for skip
% register. As last part the registers are numbered from \texttt{A}
% to \texttt{E}. Example: \cs{ltx@LocToksA}.
%
% Since 2011/04/14 v1.19.
%
% \subsection{Argument killers}
%
% \begin{declcs}{ltx@gobble} \M{1} \TO\\
%   \cs{ltx@gobbletwo} \M{1} \M{2} \TO\\
%   \cs{ltx@gobblethree} \M{1} \M{2} \M{3} \TO\\
%   \cs{ltx@gobblefour} \M{1} \M{2} \M{3} \M{4} \TO
% \end{declcs}
%
% \begin{declcs}{ltx@GobbleNum} \M{num} \M{1} \M{2} \dots\space
%   \M{\meta{num}} \TO
% \end{declcs}
% The first argument \meta{num} of macro \cs{ltx@GobbleNum} specifies,
% how many following arguments are eaten. Macro \cs{ltx@GobbleNum}
% is expandable in exact two expansion steps.
%
% \subsection{Argument grabbers}
%
% \begin{declcs}{ltx@firstofone} \M{1} \TO \meta{1}\\[1ex]
%  \cs{ltx@firstoftwo} \M{1} \M{2} \TO \meta{1}\\
%  \cs{ltx@secondoftwo} \M{1} \M{2} \TO \meta{2}\\[1ex]
%  \cs{ltx@firstofthree} \M{1} \M{2} \M{3} \TO \meta{1}\\
%  \cs{ltx@secondofthree} \M{1} \M{2} \M{3} \TO \meta{2}\\
%  \cs{ltx@thirdofthree} \M{1} \M{2} \M{3} \TO \meta{3}\\
%  \cs{ltx@firstoffour} \M{1} \M{2} \M{3} \M{4} \TO \meta{1}\\
%  \cs{ltx@secondoffour} \M{1} \M{2} \M{3} \M{4} \TO \meta{2}\\
%  \cs{ltx@thirdoffour} \M{1} \M{2} \M{3} \M{4} \TO \meta{3}\\
%  \cs{ltx@fourthoffour} \M{1} \M{2} \M{3} \M{4} \TO \meta{4}
% \end{declcs}
% Macros \cs{ltx@firstofthree}, \cs{ltx@secondofthree}
% and \cs{ltx@thirdofthree} were added in version
% 2010/11/12 v1.11.
% Macros \cs{ltx@firstoffour}, \dots, \cs{ltx@fourthoffour}
% were added in version 2011/02/04 v1.16.
%
% \subsection{List helpers}
%
% \begin{declcs}{ltx@carzero} \dots\ \cs{@nil} \TO\\
%   \cs{ltx@cdrzero} \dots\ \cs{@nil} \TO \dots
% \end{declcs}
%
% \begin{declcs}{ltx@car} \M{1} \dots\ \cs{@nil} \TO \meta{1}\\
%   \cs{ltx@cdr} \M{1} \dots\ \cs{@nil} \TO \dots
% \end{declcs}
%
% \begin{declcs}{ltx@cartwo} \M{1} \M{2} \dots\ \cs{@nil}
%   \TO \meta{1}\meta{2}\\
%   \cs{ltx@carsecond} \M{1} \M{2} \dots\ \cs{@nil} \TO \meta{2}\\
%   \cs{ltx@cdrtwo} \M{1} \M{2} \dots\ \cs{@nil} \TO \dots
% \end{declcs}
%
% \begin{declcs}{ltx@carthree} \M{1} \M{2} \M{3} \dots\ \cs{@nil}
%       \TO \meta{1}\meta{2}\meta{3}\\
%   \cs{ltx@carthird} \M{1} \M{2} \M{3} \dots\ \cs{@nil}
%       \TO \meta{3}\\
%   \cs{ltx@cdrthree} \M{1} \M{2} \M{3} \dots\ \cs{@nil} \TO \dots
% \end{declcs}
%
% \begin{declcs}{ltx@carfour} \M{1} \M{2} \M{3} \M{4} \dots\ \cs{@nil}
%       \TO \meta{1}\meta{2}\meta{3}\meta{4}\\
%   \cs{ltx@carfourth} \M{1} \M{2} \M{3} \M{4} \dots\ \cs{@nil}
%       \TO \meta{4}\\
%   \cs{ltx@cdrfour} \M{1} \M{2} \M{3} \M{4} \dots\ \cs{@nil} \TO \dots
% \end{declcs}
%
% \begin{declcs}{ltx@CarNum} \M{num} \M{1} \dots\ \M{\meta{num}}
%   \M{\meta{num}+1} \dots\ \cs{@nil}\\
%   \TO \M{1} \dots\ \M{\meta{num}} \dots\\
%   \cs{ltx@CarNumth} \M{num} \M{1} \dots\ \M{\meta{num}}
%   \M{\meta{num}+1} \dots\ \cs{@nil}\\
%   \TO \M{\meta{num}} \dots\\
%   \cs{ltx@CdrNum} \M{num} \M{1} \dots\ \M{\meta{num}}
%   \M{\meta{num}+1} \dots\ \cs{@nil}\\
%   \TO \M{\meta{num}+1} \dots
% \end{declcs}
%
% Macros with uppercase letters are expandable in two expansion steps.
% Changes in version 2019/12/15 v1.24:
% \begin{itemize}
% \item
%   Macros \cs{ltx@carsecond}, \cs{ltx@carthird}, \cs{ltx@carfourth},
%   \cs{ltx@CarNumth} added.
% \item
%   Macros \cs{ltx@cdr}, \cs{ltx@cdrtwo}, \cs{ltx@cdrthree}, \cs{ltx@cdrfour},
%   \cs{ltx@CdrNum} are expandable in two expansion steps and retain
%   spaces and braces after the first gobbled arguments.
% \end{itemize}
%
% \subsection{Tail recursion}
%
% \begin{declcs}{ltx@ReturnAfterFi} \M{1} \cs{fi} \TO \cs{fi} \meta{1}\\
%   \cs{ltx@ReturnAfterElseFi} \M{1} \cs{else} \M{2} \cs{fi}
%     \TO \cs{fi} \meta{1}
% \end{declcs}
%
% \subsection{Empty macro}
%
% \begin{declcs}{ltx@empty} \TO
% \end{declcs}
%
% \subsection{Characters}
%
% \begin{declcs}{ltx@space} \TO \verb*| |\\
%   \cs{ltx@percentchar} \TO \texttt{\%}\\
%   \cs{ltx@backslashchar} \TO
%      \makeatletter\texttt{\@backslashchar}\\
%   \cs{ltx@hashchar} \TO \texttt{\#}\qquad(since v1.7)\\
%   \cs{ltx@leftbracechar} \TO \texttt{\textbraceleft}\qquad(since v1.8)\\
%   \cs{ltx@rightbracechar} \TO \texttt{\textbraceright}\qquad(since v1.8)
% \end{declcs}
%
% \subsection{Boolean switch}
%
% \begin{declcs}{ltx@newif} \M{cmd}
% \end{declcs}
% \cs{ltx@newif} defines a new boolean switch \meta{cmd} like \cs{newif}.
% Unlike \plainTeX's \cs{newif}, \cs{ltx@newif} is not \cs{outer}.
% The command \meta{cmd} must start with the two characters |if|.
%
% \begin{declcs}{ltx@newglobalif} \M{cmd}
% \end{declcs}
% \cs{ltx@newglobalif} defines a new boolean switch \meta{cmd}
% like \cs{ltx@newif}. However the switch setting commands,
% \meta{cmd} without the prefix |if| and followed by |true| or |false|
% are acting globally.
%
% \subsection{Command definitions}
%
% \begin{declcs}{ltx@ifundefined} \M{cmd} \M{yes} \M{no}
% \end{declcs}
% If \eTeX\ is available, \cs{ifcsname} is used that does
% not have the side effect of defining undefined commands
% with meaning of \cs{relax}. This command is always expandable.
% Change in version 1.1: Also the meaning \cs{relax} is
% always considered ``undefined''.
%
% \begin{declcs}{ltx@IfUndefined} \M{cmd} \M{yes} \M{no}
% \end{declcs}
% If \eTeX\ is available, \cs{ifcsname} is used that does
% not have the side effect of defining undefined commands
% with meaning of \cs{relax}. Also it always checks for
% the meaning of \cs{relax} and considers this as undefined.
% This macro is not expandable without \eTeX.
%
% \begin{declcs}{ltx@LocalExpandAfter}
% \end{declcs}
% It expands the token after the next token but in a local
% context. That is the difference to \cs{expandafter}.
% The local context discards the side effect of \cs{csname}
% and let the command undefined after the expansion step.
%
% \subsection{Stripping}
%
% \begin{declcs}{ltx@RemovePrefix}\\
%   \cs{ltx@StripPrefix}
% \end{declcs}
% All tokens up to and including the next available character
% `\textgreater' are thrown away. Usually it is used to strip
% the first part of the output of the commands \cs{meaning}
% or \cs{pdflastmatch}. Macro \cs{ltx@RemovePrefix}
% has the same meaning as \LaTeX's \cs{strip@prefix}, whereas
% macro \cs{ltx@StripPrefix} expands the next token
% once before stripping the prefix.
%
% \begin{declcs}{ltx@onelevel@sanitize} \M{macro}
% \end{declcs}
% Macro \cs{ltx@onelevel@sanitize} provides \hologo{LaTeX}'s
% \cs{@onelevel@sanitize}. The macro is expanded once and
% the contents is converted to characters with catcode 12 (other)
% and space tokens with catcode 10 (space). Then then sanitized
% contents is stored into the macro again. Since version 1.12.
%
% \subsection{File management}
%
% All macros in this section are expandable like
% the counterparts of the \LaTeX\ kernel. Also they
% can be used after the preamble.
%
% \subsubsection{File extensions}
%
% \begin{declcs}{ltx@clsextension}\\
%   \cs{ltx@pkgextension}
% \end{declcs}
% Macros \cs{ltx@clsextension} and \cs{ltx@styextension}
% stores the strings |cls| and |sty|. In opposite to \hologo{LaTeX}'s
% \cs{@clsextension} and \cs{@styextension} they can also be
% used after |\begin{document}|.
%
% \subsubsection{Load check}
%
% \begin{declcs}{ltx@ifclassloaded} \M{class} \M{yes} \M{no}\\
%   \cs{ltx@ifpackageloaded} \M{package} \M{yes} \M{no}
% \end{declcs}
% Macros \cs{ltx@ifclassloaded}\slash\cs{ltx@ifpackageloaded}
% execute \meta{yes}, if the \meta{class} or \meta{package} is loaded,
% otherwise \meta{no} is called.
% Both \meta{class} and \meta{package} are specified without
% extension. The macros can also be used after |\begin{document}|.
%
% \begin{declcs}{ltx@iffileloaded} \M{file} \M{yes} \M{no}
% \end{declcs}
% If \LaTeX's \cs{ProvidesFile} macro was called before
% using \meta{file} as argument, then \cs{ltx@iffileloaded}
% calls \meta{yes}, otherwise \meta{no}.
% Therefore it is possible
% that the \meta{file} is loaded, but \meta{no} is executed because
% of a missing \cs{ProvidesFile}. The \LaTeX\ kernel does not have
% a counterpart of \cs{ltx@iffileloaded}.
%
% Note that the file name used in \cs{ProvidesFile} and
% \cs{ltx@iffileloaded} must match. For example, if \TeX's
% default extension \xext{tex} was given in the first command,
% then it must also specified in the latter command and vice versa.
%
% \subsubsection{Version date check}
%
% \begin{declcs}{ltx@ifclasslater} \M{class} \M{date} \M{yes} \M{no}\\
%   \cs{ltx@ifpackagelater} \M{package} \M{date} \M{yes} \M{no}\\
%   \cs{ltx@iffilelater} \M{file} \M{date} \M{yes} \M{no}
% \end{declcs}
% If a \cs{ProvidesClass}\slash\cs{ProvidesPackage}\slash
% \cs{ProvidesFile} command with exact the same class\slash package\slash
% file was executed before with an optional
% argument that starts with a \LaTeX\ version date, then
% this version date is compared with the argument \meta{date}.
% If they are equal or if the version date is the later date,
% then \meta{yes} is called. In all other cases \meta{no} is executed.
%
% A \LaTeX\ date has the format |YYYY/MM/DD| with |YYYY| as year
% with four digits, |MM| as month with two digits and |DD| as day
% with two digits. If \pdfTeX's \cs{pdfmatch} is available, then
% it is used to detect the version date, to reject invalid
% date formats and to reject some invalid dates.
% Dates before 1994/01/01 are always invalid, because version
% dates are introduced with \LaTeXe\ in 1994.
%
% \subsection{Macro additions}
%
% \begin{declcs}{ltx@GlobalAppendToMacro} \M{cmd} \M{addition}\\
%   \cs{ltx@LocalAppendToMacro} \M{cmd} \M{addition}
% \end{declcs}
% The \meta{addition} is appended to the parameterless macro \meta{cmd}.
% If \meta{cmd} is undefined or has the meaning \cs{relax}, then
% it will be initialized as empty macro beforehand. Due to a bug \meta{addition}
% must not contain \cs{par} before version 2010/10/25 v1.9.
%
% \begin{declcs}{ltx@GlobalPrependToMacro} \M{cmd} \M{addition}\\
%   \cs{ltx@LocalPrependToMacro} \M{cmd} \M{addition}
% \end{declcs}
% The \meta{addition} is prepended to the parameterless macro \meta{cmd}.
% If \meta{cmd} is undefined or has the meaning \cs{relax}, then
% it will be initialized as empty macro beforehand. The macros
% were added in version 2011/08/22 v1.21.
%
% \subsection{Next character detection}
%
% \begin{declcs}{ltx@ifnextchar} \M{char} \M{yes} \M{no}
% \end{declcs}
% If next character is \meta{char} then \meta{yes} is called,
% otherwise \meta{no}. The character is not removed.
% Spaces are silently removed when looking for \meta{char}
% as \hologo{LaTeX}'s version \cs{kernel@ifnextchar} does.
% But there are also small differences:
% \begin{itemize}
% \item The space can be used as \meta{char}. In this case
%   optional spaces before \meta{char} are not supported of course.
% \item If the optional space is a command that is a character
%   (defined by \cs{let} or \cs{futurelet}), then
%   \cs{kernel@ifnextchar} breaks with an \hologo{TeX} error.
%   \cs{ltx@ifnextchar} silently removes this token as
%   optional space.
% \end{itemize}
% Since 2010/03/01 v1.3.
%
% \begin{declcs}{ltx@ifnextchar@nospace} \M{char} \M{yes} \M{no}
% \end{declcs}
% Macro \cs{ltx@ifnextchar@nospace} behaves like
% macro \cs{ltx@ifnextchar} with the exception that
% optional spaces are not supported before \meta{char}.
% Since 2011/04/14 v1.19.
%
% \subsection{\cs{ltx@leavevmode}, \cs{ltx@mbox}}
%
% \begin{declcs}{ltx@leavevmode}
% \end{declcs}
% Macro \cs{ltx@leavevmode} calls \hologo{pdfTeX}'s \cs{quitvmode}.
% Otherwise \cs{leavevmode} is used and defined if it is necessary.
%
% \begin{declcs}{ltx@mbox}
% \end{declcs}
% Macro \cs{ltx@mbox} reimplements \cs{mbox} with two changes.
% Instead of \cs{leavevmode} it uses \cs{ltx@leavevmode} and
% stops right after \cs{hbox}. Especially it does not grab
% the argument and allows the extended syntax of \cs{hbox}.
%
% \subsection{Expandable test for emptiness}
%
% \begin{declcs}{ltx@ifempty} \M{stuff} \M{yes} \M{no}
% \end{declcs}
% Macro \cs{ltx@ifempty} checks in exact two expansion steps
% whether \meta{stuff}
% is empty or contains token. Depending on the result
% \meta{yes} or \meta{no} is executed. The token in \meta{stuff}
% may contain \cs{par} and unmatched conditionals (\cs{if}, \cs{else},
% \cs{fi}, \dots). Since version 2010/11/12 v1.11.
%
% \begin{declcs}{ltx@ifblank} \M{stuff} \M{yes} \M{no}
% \end{declcs}
% Macro \cs{ltx@ifblank} tests in exact two expansion steps
% if \meta{stuff} is empty or contain only blank spaces.
% In this case argument \meta{yes} is called. If \meta{stuff}
% contains other tokens than spaces then \meta{no} is executed.
% Since version 2010/12/04 v1.13.
%
% \subsection{Stripping spaces}
%
% \begin{declcs}{ltx@zapspace} \M{stuff}
% \end{declcs}
% Macro \cs{ltx@zapspace} strips spaces from \meta{stuff}
% that are not hidden inside curly braces. Like \hologo{LaTeX}'s
% \cs{zap@space} it is expandable. Differences:
% \begin{itemize}
% \item
% Syntax: \cs{zap@space} also expects a space token
% and \cs{@empty} after \meta{stuff}.
% \item
% Macro \cs{ltx@zapspace} is expandable in exact two
% expansion steps.
% \item
% Macro \cs{ltx@zapspace} always retains curly braces.
% \item
% Macro \cs{zap@space} has a bug. It stops stripping spaces
% after a token group in curly braces if the first two tokens
% inside the group are equal.
% \item
% Macro \cs{ltx@zapspace} also works with \cs{par}
% and conditionals (\cs{if}, \cs{else}, \cs{fi}, \dots).
% \end{itemize}
% Macro \cs{ltx@zapspace} is available since version
% 2010/12/07 v1.14.
%
% \subsection{Check for emptiness of boxes}
%
% \begin{declcs}{ltx@IfBoxEmpty} \M{box register number}
%   \M{yes} \M{no}
% \end{declcs}
% Macro \cs{ltx@IfBoxEmpty} calls \meta{yes} if
% the box exists (\cs{ifvoid} returns false) and the box
% does not contain any content. Otherwise if the box
% is void or contains something, then \meta{no} is executed.
% Thus being empty means that the box exists and is either
% an \cs{hbox} or a \cs{vbox} and may even have dimensions other than
% 0.0\,pt, but the box does not contain anything.
% Macro \cs{ltx@IfBoxEmpty} is available since 2010/02/04 v1.16.
%
% \begin{declcs}{ltx@IfBoxVoidOrEmpty} \M{box register number}
%   \M{yes} \M{no}
% \end{declcs}
% Macro \cs{ltx@IfBoxVoidOrEmpty} calls \meta{yes} if
% the box is either void or does not contain any content.
% Otherwise \meta{no} is executed.
% Macro \cs{ltx@IfBoxVoidOrEmpty} is available since 2010/02/04 v1.16.
%
% \StopEventually{
% }
%
% \section{Implementation}
%
% \subsection{Identification}
%
%    \begin{macrocode}
%<*package>
%    \end{macrocode}
%    Reload check, especially if the package is not used with \LaTeX.
%    \begin{macrocode}
\begingroup\catcode61\catcode48\catcode32=10\relax%
  \catcode13=5 % ^^M
  \endlinechar=13 %
  \catcode35=6 % #
  \catcode39=12 % '
  \catcode44=12 % ,
  \catcode45=12 % -
  \catcode46=12 % .
  \catcode58=12 % :
  \catcode64=11 % @
  \catcode123=1 % {
  \catcode125=2 % }
  \expandafter\let\expandafter\x\csname ver@ltxcmds.sty\endcsname
  \ifx\x\relax % plain-TeX, first loading
  \else
    \def\empty{}%
    \ifx\x\empty % LaTeX, first loading,
      % variable is initialized, but \ProvidesPackage not yet seen
    \else
      \expandafter\ifx\csname PackageInfo\endcsname\relax
        \def\x#1#2{%
          \immediate\write-1{Package #1 Info: #2.}%
        }%
      \else
        \def\x#1#2{\PackageInfo{#1}{#2, stopped}}%
      \fi
      \x{ltxcmds}{The package is already loaded}%
      \aftergroup\endinput
    \fi
  \fi
\endgroup%
%    \end{macrocode}
%    Package identification:
%    \begin{macrocode}
\begingroup\catcode61\catcode48\catcode32=10\relax%
  \catcode13=5 % ^^M
  \endlinechar=13 %
  \catcode35=6 % #
  \catcode39=12 % '
  \catcode40=12 % (
  \catcode41=12 % )
  \catcode44=12 % ,
  \catcode45=12 % -
  \catcode46=12 % .
  \catcode47=12 % /
  \catcode58=12 % :
  \catcode64=11 % @
  \catcode91=12 % [
  \catcode93=12 % ]
  \catcode123=1 % {
  \catcode125=2 % }
  \expandafter\ifx\csname ProvidesPackage\endcsname\relax
    \def\x#1#2#3[#4]{\endgroup
      \immediate\write-1{Package: #3 #4}%
      \xdef#1{#4}%
    }%
  \else
    \def\x#1#2[#3]{\endgroup
      #2[{#3}]%
      \ifx#1\@undefined
        \xdef#1{#3}%
      \fi
      \ifx#1\relax
        \xdef#1{#3}%
      \fi
    }%
  \fi
\expandafter\x\csname ver@ltxcmds.sty\endcsname
\ProvidesPackage{ltxcmds}%
  [2019/12/15 v1.24 LaTeX kernel commands for general use (HO)]%
%    \end{macrocode}
%
%    \begin{macrocode}
\begingroup\catcode61\catcode48\catcode32=10\relax%
  \catcode13=5 % ^^M
  \endlinechar=13 %
  \catcode123=1 % {
  \catcode125=2 % }
  \catcode64=11 % @
  \def\x{\endgroup
    \expandafter\edef\csname LTXcmds@AtEnd\endcsname{%
      \endlinechar=\the\endlinechar\relax
      \catcode13=\the\catcode13\relax
      \catcode32=\the\catcode32\relax
      \catcode35=\the\catcode35\relax
      \catcode61=\the\catcode61\relax
      \catcode64=\the\catcode64\relax
      \catcode123=\the\catcode123\relax
      \catcode125=\the\catcode125\relax
    }%
  }%
\x\catcode61\catcode48\catcode32=10\relax%
\catcode13=5 % ^^M
\endlinechar=13 %
\catcode35=6 % #
\catcode64=11 % @
\catcode123=1 % {
\catcode125=2 % }
\def\TMP@EnsureCode#1#2{%
  \edef\LTXcmds@AtEnd{%
    \LTXcmds@AtEnd
    \catcode#1=\the\catcode#1\relax
  }%
  \catcode#1=#2\relax
}
\TMP@EnsureCode{36}{3}% $
\TMP@EnsureCode{38}{4}% &
\TMP@EnsureCode{40}{12}% (
\TMP@EnsureCode{41}{12}% )
\TMP@EnsureCode{45}{12}% -
\TMP@EnsureCode{46}{12}% .
\TMP@EnsureCode{47}{12}% /
\TMP@EnsureCode{60}{12}% <
\TMP@EnsureCode{62}{12}% >
\TMP@EnsureCode{91}{12}% [
\TMP@EnsureCode{96}{12}% `
\TMP@EnsureCode{93}{12}% ]
\TMP@EnsureCode{94}{12}% ^ (superscript) (!)
\TMP@EnsureCode{124}{12}% |
\edef\LTXcmds@AtEnd{\LTXcmds@AtEnd\noexpand\endinput}
%    \end{macrocode}
%
% \subsection{Numbers}
%
%    \begin{macro}{\ltx@zero}
%    \begin{macrocode}
\chardef\ltx@zero=0 %
%    \end{macrocode}
%    \end{macro}
%    \begin{macro}{\ltx@one}
%    \begin{macrocode}
\chardef\ltx@one=1 %
%    \end{macrocode}
%    \end{macro}
%    \begin{macro}{\ltx@two}
%    \begin{macrocode}
\chardef\ltx@two=2 %
%    \end{macrocode}
%    \end{macro}
%    \begin{macro}{\ltx@active}
%    \begin{macrocode}
\chardef\ltx@active=13 %
%    \end{macrocode}
%    \end{macro}
%    \begin{macro}{\ltx@cclv}
%    \begin{macrocode}
\chardef\ltx@cclv=255 %
%    \end{macrocode}
%    \end{macro}
%
%    \begin{macro}{\ltx@minusone}
%    \begin{macrocode}
\def\ltx@minusone{%
  -\ltx@one
}
%    \end{macrocode}
%    \end{macro}
%
% \subsection{Scratch registers}
%
%    \begin{macro}{\ltx@LocToksA}
%    \begin{macrocode}
\toksdef\ltx@LocToksA=0 %
%    \end{macrocode}
%    \end{macro}
%    \begin{macro}{\ltx@LocToksB}
%    \begin{macrocode}
\toksdef\ltx@LocToksB=2 %
%    \end{macrocode}
%    \end{macro}
%    \begin{macro}{\ltx@LocToksC}
%    \begin{macrocode}
\toksdef\ltx@LocToksC=4 %
%    \end{macrocode}
%    \end{macro}
%    \begin{macro}{\ltx@LocToksD}
%    \begin{macrocode}
\toksdef\ltx@LocToksD=6 %
%    \end{macrocode}
%    \end{macro}
%    \begin{macro}{\ltx@LocToksE}
%    \begin{macrocode}
\toksdef\ltx@LocToksE=8 %
%    \end{macrocode}
%    \end{macro}
%
%    \begin{macro}{\ltx@GlobToksA}
%    \begin{macrocode}
\toksdef\ltx@GlobToksA=1 %
%    \end{macrocode}
%    \end{macro}
%    \begin{macro}{\ltx@GlobToksB}
%    \begin{macrocode}
\toksdef\ltx@GlobToksB=3 %
%    \end{macrocode}
%    \end{macro}
%    \begin{macro}{\ltx@GlobToksC}
%    \begin{macrocode}
\toksdef\ltx@GlobToksC=5 %
%    \end{macrocode}
%    \end{macro}
%    \begin{macro}{\ltx@GlobToksD}
%    \begin{macrocode}
\toksdef\ltx@GlobToksD=7 %
%    \end{macrocode}
%    \end{macro}
%    \begin{macro}{\ltx@GlobToksE}
%    \begin{macrocode}
\toksdef\ltx@GlobToksE=9 %
%    \end{macrocode}
%    \end{macro}
%
%    \begin{macro}{\ltx@LocDimenA}
%    \begin{macrocode}
\dimendef\ltx@LocDimenA=0 %
%    \end{macrocode}
%    \end{macro}
%    \begin{macro}{\ltx@LocDimenB}
%    \begin{macrocode}
\dimendef\ltx@LocDimenB=2 %
%    \end{macrocode}
%    \end{macro}
%    \begin{macro}{\ltx@LocDimenC}
%    \begin{macrocode}
\dimendef\ltx@LocDimenC=4 %
%    \end{macrocode}
%    \end{macro}
%    \begin{macro}{\ltx@LocDimenD}
%    \begin{macrocode}
\dimendef\ltx@LocDimenD=6 %
%    \end{macrocode}
%    \end{macro}
%    \begin{macro}{\ltx@LocDimenE}
%    \begin{macrocode}
\dimendef\ltx@LocDimenE=8 %
%    \end{macrocode}
%    \end{macro}
%
%    \begin{macro}{\ltx@GlobDimenA}
%    \begin{macrocode}
\dimendef\ltx@GlobDimenA=1 %
%    \end{macrocode}
%    \end{macro}
%    \begin{macro}{\ltx@GlobDimenB}
%    \begin{macrocode}
\dimendef\ltx@GlobDimenB=3 %
%    \end{macrocode}
%    \end{macro}
%    \begin{macro}{\ltx@GlobDimenC}
%    \begin{macrocode}
\dimendef\ltx@GlobDimenC=5 %
%    \end{macrocode}
%    \end{macro}
%    \begin{macro}{\ltx@GlobDimenD}
%    \begin{macrocode}
\dimendef\ltx@GlobDimenD=7 %
%    \end{macrocode}
%    \end{macro}
%    \begin{macro}{\ltx@GlobDimenE}
%    \begin{macrocode}
\dimendef\ltx@GlobDimenE=9 %
%    \end{macrocode}
%    \end{macro}
%
%    \begin{macro}{\ltx@LocSkipA}
%    \begin{macrocode}
\skipdef\ltx@LocSkipA=0 %
%    \end{macrocode}
%    \end{macro}
%    \begin{macro}{\ltx@LocSkipB}
%    \begin{macrocode}
\skipdef\ltx@LocSkipB=2 %
%    \end{macrocode}
%    \end{macro}
%    \begin{macro}{\ltx@LocSkipC}
%    \begin{macrocode}
\skipdef\ltx@LocSkipC=4 %
%    \end{macrocode}
%    \end{macro}
%    \begin{macro}{\ltx@LocSkipD}
%    \begin{macrocode}
\skipdef\ltx@LocSkipD=6 %
%    \end{macrocode}
%    \end{macro}
%    \begin{macro}{\ltx@LocSkipE}
%    \begin{macrocode}
\skipdef\ltx@LocSkipE=8 %
%    \end{macrocode}
%    \end{macro}
%
%    \begin{macro}{\ltx@GlobSkipA}
%    \begin{macrocode}
\skipdef\ltx@GlobSkipA=1 %
%    \end{macrocode}
%    \end{macro}
%    \begin{macro}{\ltx@GlobSkipB}
%    \begin{macrocode}
\skipdef\ltx@GlobSkipB=3 %
%    \end{macrocode}
%    \end{macro}
%    \begin{macro}{\ltx@GlobSkipC}
%    \begin{macrocode}
\skipdef\ltx@GlobSkipC=5 %
%    \end{macrocode}
%    \end{macro}
%    \begin{macro}{\ltx@GlobSkipD}
%    \begin{macrocode}
\skipdef\ltx@GlobSkipD=7 %
%    \end{macrocode}
%    \end{macro}
%    \begin{macro}{\ltx@GlobSkipE}
%    \begin{macrocode}
\skipdef\ltx@GlobSkipE=9 %
%    \end{macrocode}
%    \end{macro}
%
%
% \subsection{Argument killers}
%
%    \begin{macro}{\ltx@gobble}
%    \begin{macrocode}
\long\def\ltx@gobble#1{}
%    \end{macrocode}
%    \end{macro}
%    \begin{macro}{\ltx@gobbletwo}
%    \begin{macrocode}
\long\def\ltx@gobbletwo#1#2{}
%    \end{macrocode}
%    \end{macro}
%    \begin{macro}{\ltx@gobblethree}
%    \begin{macrocode}
\long\def\ltx@gobblethree#1#2#3{}
%    \end{macrocode}
%    \end{macro}
%    \begin{macro}{\ltx@gobblefour}
%    \begin{macrocode}
\long\def\ltx@gobblefour#1#2#3#4{}
%    \end{macrocode}
%    \end{macro}
%
%    \begin{macro}{\ltx@GobbleNum}
%    \begin{macrocode}
\def\ltx@GobbleNum#1{%
  \romannumeral
  \csname ltx@zero%
  \expandafter\LTXcmds@GobbleNum
  \romannumeral\LTXcmds@num{#1}000{m\endcsname}%
}
%    \end{macrocode}
%    \end{macro}
%    \begin{macro}{\LTXcmds@GobbleNum}
%    \begin{macrocode}
\def\LTXcmds@GobbleNum#1{%
  \csname LTXcmds@G#1\LTXcmds@GobbleNum
}
%    \end{macrocode}
%    \end{macro}
%    \begin{macro}{\LTXcmds@Gm}
%    \begin{macrocode}
\long\def\LTXcmds@Gm#1{%
  \endcsname
}
%    \end{macrocode}
%    \end{macro}
%
% \subsection{Argument grabbers}
%
%    \begin{macro}{\ltx@firstofone}
%    \begin{macrocode}
\long\def\ltx@firstofone#1{#1}
%    \end{macrocode}
%    \end{macro}
%    \begin{macro}{\ltx@firstoftwo}
%    \begin{macrocode}
\long\def\ltx@firstoftwo#1#2{#1}
%    \end{macrocode}
%    \end{macro}
%    \begin{macro}{\ltx@secondoftwo}
%    \begin{macrocode}
\long\def\ltx@secondoftwo#1#2{#2}
%    \end{macrocode}
%    \end{macro}
%
%    \begin{macro}{\ltx@firstofthree}
%    \begin{macrocode}
\long\def\ltx@firstofthree#1#2#3{#1}
%    \end{macrocode}
%    \end{macro}
%    \begin{macro}{\ltx@secondofthree}
%    \begin{macrocode}
\long\def\ltx@secondofthree#1#2#3{#2}
%    \end{macrocode}
%    \end{macro}
%    \begin{macro}{\ltx@thirdofthree}
%    \begin{macrocode}
\long\def\ltx@thirdofthree#1#2#3{#3}%
%    \end{macrocode}
%    \end{macro}
%
%    \begin{macro}{\ltx@firstoffour}
%    \begin{macrocode}
\long\def\ltx@firstoffour#1#2#3#4{#1}
%    \end{macrocode}
%    \end{macro}
%    \begin{macro}{\ltx@secondoffour}
%    \begin{macrocode}
\long\def\ltx@secondoffour#1#2#3#4{#2}
%    \end{macrocode}
%    \end{macro}
%    \begin{macro}{\ltx@thirdoffour}
%    \begin{macrocode}
\long\def\ltx@thirdoffour#1#2#3#4{#3}%
%    \end{macrocode}
%    \end{macro}
%    \begin{macro}{\ltx@fourthoffour}
%    \begin{macrocode}
\long\def\ltx@fourthoffour#1#2#3#4{#4}%
%    \end{macrocode}
%    \end{macro}
%
% \subsection{List helpers}
%
%    \begin{macro}{\ltx@carzero}
%    \begin{macrocode}
\long\def\ltx@carzero#1\@nil{}%
%    \end{macrocode}
%    \end{macro}
%    \begin{macro}{\LTXcmds@cdrzero}
%    \begin{macrocode}
\long\def\LTXcmds@cdrzero#1\@nil{#1}
%    \end{macrocode}
%    \end{macro}
%    \begin{macro}{\ltx@cdrzero}
%    \begin{macrocode}
\def\ltx@cdrzero{%
  \romannumeral\LTXcmds@cdrzero\ltx@zero
}
%    \end{macrocode}
%    \end{macro}
%
%    \begin{macro}{\ltx@car}
%    \begin{macrocode}
\long\def\ltx@car#1#2\@nil{#1}
%    \end{macrocode}
%    \end{macro}
%    \begin{macro}{\ltx@cdr}
%    \begin{macrocode}
\long\def\ltx@cdr#1{%
  \romannumeral\LTXcmds@cdrzero\ltx@zero
}
%    \end{macrocode}
%    \end{macro}
%
%    \begin{macro}{\ltx@cartwo}
%    \begin{macrocode}
\long\def\ltx@cartwo#1#2#3\@nil{#1#2}
%    \end{macrocode}
%    \end{macro}
%    \begin{macro}{\ltx@carsecond}
%    \begin{macrocode}
\long\def\ltx@carsecond#1#2#3\@nil{#2}
%    \end{macrocode}
%    \end{macro}
%    \begin{macro}{\ltx@cdrtwo}
%    \begin{macrocode}
\long\def\ltx@cdrtwo#1#2{%
  \romannumeral\LTXcmds@cdrzero\ltx@zero
}
%    \end{macrocode}
%    \end{macro}
%
%    \begin{macro}{\ltx@carthree}
%    \begin{macrocode}
\long\def\ltx@carthree#1#2#3#4\@nil{#1#2#3}
%    \end{macrocode}
%    \end{macro}
%    \begin{macro}{\ltx@carthird}
%    \begin{macrocode}
\long\def\ltx@carthird#1#2#3#4\@nil{#3}
%    \end{macrocode}
%    \end{macro}
%    \begin{macro}{\ltx@cdrthree}
%    \begin{macrocode}
\long\def\ltx@cdrthree#1#2#3{%
  \romannumeral\LTXcmds@cdrzero\ltx@zero
}
%    \end{macrocode}
%    \end{macro}
%
%    \begin{macro}{\ltx@carfour}
%    \begin{macrocode}
\long\def\ltx@carfour#1#2#3#4#5\@nil{#1#2#3#4}
%    \end{macrocode}
%    \end{macro}
%    \begin{macro}{\ltx@carfourth}
%    \begin{macrocode}
\long\def\ltx@carfourth#1#2#3#4#5\@nil{#4}
%    \end{macrocode}
%    \end{macro}
%    \begin{macro}{\ltx@cdrfour}
%    \begin{macrocode}
\long\def\ltx@cdrfour#1#2#3#4{%
  \romannumeral\LTXcmds@cdrzero\ltx@zero
}
%    \end{macrocode}
%    \end{macro}
%
%    \begin{macro}{\ltx@CarNum}
%    \begin{macrocode}
\def\ltx@CarNum#1{%
  \romannumeral
  \csname LTXcmds@CarNumFinish%
  \expandafter\LTXcmds@CarNum
  \romannumeral\LTXcmds@num{#1}000{x\endcsname}%
}
%    \end{macrocode}
%    \end{macro}
%    \begin{macro}{\LTXcmds@CarNum}
%    \begin{macrocode}
\def\LTXcmds@CarNum#1{%
  \csname LTXcmds@C#1\LTXcmds@CarNum
}
%    \end{macrocode}
%    \end{macro}
%    \begin{macro}{\LTXcmds@Cm}
%    \begin{macrocode}
\long\def\LTXcmds@Cm#1#2{%
  \endcsname{#1#2}%
}
%    \end{macrocode}
%    \end{macro}
%    \begin{macro}{\LTXcmds@Cx}
%    \begin{macrocode}
\def\LTXcmds@Cx#1{%
  \endcsname{}%
}
%    \end{macrocode}
%    \end{macro}
%    \begin{macro}{\LTXcmds@CarNumFinish}
%    \begin{macrocode}
\long\def\LTXcmds@CarNumFinish#1#2\@nil{%
  \ltx@zero
  #1%
}
%    \end{macrocode}
%    \end{macro}
%
%    \begin{macro}{\ltx@CarNumth}
%    \begin{macrocode}
\def\ltx@CarNumth#1{%
  \romannumeral
  \expandafter\expandafter\expandafter
  \LTXcmds@CarNumth
  \ltx@GobbleNum{#1}{}%
}
%    \end{macrocode}
%    \end{macro}
%    \begin{macro}{\LTXcmds@CarNumth}
%    \begin{macrocode}
\long\def\LTXcmds@CarNumth#1#2\@nil{%
  \ltx@zero
  #1%
}
%    \end{macrocode}
%    \end{macro}
%
%    \begin{macro}{\ltx@CdrNum}
%    \begin{macrocode}
\def\ltx@CdrNum#1{%
  \romannumeral%
  \expandafter\expandafter\expandafter\ltx@cdrzero
  \expandafter\expandafter\expandafter\ltx@zero
  \ltx@GobbleNum{#1}%
}
%    \end{macrocode}
%    \end{macro}
%
% \subsection{Tail recursion}
%
%    \begin{macro}{\ltx@ReturnAfterFi}
%    \begin{macrocode}
\long\def\ltx@ReturnAfterFi#1\fi{\fi#1}
%    \end{macrocode}
%    \end{macro}
%    \begin{macro}{\ltx@ReturnAfterElseFi}
%    \begin{macrocode}
\long\def\ltx@ReturnAfterElseFi#1\else#2\fi{\fi#1}
%    \end{macrocode}
%    \end{macro}
%
% \subsection{Empty macro}
%
%    \begin{macro}{\ltx@empty}
%    \begin{macrocode}
\def\ltx@empty{}
%    \end{macrocode}
%    \end{macro}
%
% \subsection{Characters}
%
%    \begin{macro}{\ltx@space}
%    \begin{macrocode}
\def\ltx@space{ }
%    \end{macrocode}
%    \end{macro}
%    \begin{macro}{\ltx@percentchar}
%    \begin{macrocode}
\begingroup
  \lccode`0=`\%\relax
\lowercase{\endgroup
  \def\ltx@percentchar{0}%
}
%    \end{macrocode}
%    \end{macro}
%    \begin{macro}{\ltx@backslashchar}
%    \begin{macrocode}
\begingroup
  \lccode`0=`\\\relax
\lowercase{\endgroup
  \def\ltx@backslashchar{0}%
}
%    \end{macrocode}
%    \end{macro}
%    \begin{macro}{\ltx@hashchar}
%    \begin{macrocode}
\begingroup
  \lccode`0=`\#\relax
\lowercase{\endgroup
  \def\ltx@hashchar{0}%
}
%    \end{macrocode}
%    \end{macro}
%    \begin{macro}{\ltx@leftbracechar}
%    \begin{macrocode}
\begingroup
  \lccode`0=`\{\relax
\lowercase{\endgroup
  \def\ltx@leftbracechar{0}%
}
%    \end{macrocode}
%    \end{macro}
%    \begin{macro}{\ltx@rightbracechar}
%    \begin{macrocode}
\begingroup
  \lccode`0=`\}\relax
\lowercase{\endgroup
  \def\ltx@rightbracechar{0}%
}
%    \end{macrocode}
%    \end{macro}
%
% \subsection{Boolean switch}
%
%    \begin{macro}{\ltx@newif}
%    \begin{macrocode}
\def\ltx@newif#1{%
  \begingroup
    \escapechar=-1 %
  \expandafter\endgroup
  \expandafter\LTXcmds@newif\string#1\@nil
}
%    \end{macrocode}
%    \end{macro}
%    \begin{macro}{\LTXcmds@newif}
%    \begin{macrocode}
\begingroup
  \escapechar=-1 %
\expandafter\endgroup
\expandafter\def\expandafter\LTXcmds@newif\string\if#1\@nil{%
  \expandafter\edef\csname#1true\endcsname{%
    \let
    \expandafter\noexpand\csname if#1\endcsname
    \noexpand\iftrue
  }%
  \expandafter\edef\csname#1false\endcsname{%
    \let
    \expandafter\noexpand\csname if#1\endcsname
    \noexpand\iffalse
  }%
  \csname#1false\endcsname
}
%    \end{macrocode}
%    \end{macro}
%
%    \begin{macro}{\ltx@newglobalif}
%    \begin{macrocode}
\def\ltx@newglobalif#1{%
  \begingroup
    \escapechar=-1 %
  \expandafter\endgroup
  \expandafter\LTXcmds@newglobalif\string#1\@nil
}
%    \end{macrocode}
%    \end{macro}
%    \begin{macro}{\LTXcmds@newglobalif}
%    \begin{macrocode}
\begingroup
  \escapechar=-1 %
\expandafter\endgroup
\expandafter
\def\expandafter\LTXcmds@newglobalif\string\if#1\@nil{%
  \expandafter\edef\csname#1true\endcsname{%
    \global\let
    \expandafter\noexpand\csname if#1\endcsname
    \noexpand\iftrue
  }%
  \expandafter\edef\csname#1false\endcsname{%
    \global\let
    \expandafter\noexpand\csname if#1\endcsname
    \noexpand\iffalse
  }%
  \csname#1false\endcsname
}
%    \end{macrocode}
%    \end{macro}
%
% \subsection{Command definitions}
%
%    \begin{macro}{\ltx@LocalExpandAfter}
%    \begin{macrocode}
\def\ltx@LocalExpandAfter{%
  \begingroup
    \expandafter\expandafter\expandafter
  \endgroup
  \expandafter
}
%    \end{macrocode}
%    \end{macro}
%
%    \begin{macrocode}
\ltx@LocalExpandAfter
\ifx\csname ifcsname\endcsname\relax
%    \end{macrocode}
%    \begin{macro}{\ltx@ifundefined}
%    \begin{macrocode}
  \def\ltx@ifundefined#1{%
    \expandafter\ifx\csname #1\endcsname\relax
      \expandafter\ltx@firstoftwo
    \else
      \expandafter\ltx@secondoftwo
    \fi
  }%
%    \end{macrocode}
%    \end{macro}
%    \begin{macro}{\ltx@IfUndefined}
%    \begin{macrocode}
  \def\ltx@IfUndefined#1{%
    \begingroup\expandafter\expandafter\expandafter\endgroup
    \expandafter\ifx\csname #1\endcsname\relax
      \expandafter\ltx@firstoftwo
    \else
      \expandafter\ltx@secondoftwo
    \fi
  }%
%    \end{macrocode}
%    \end{macro}
%    \begin{macrocode}
  \expandafter\ltx@gobble
\else
  \expandafter\ltx@firstofone
\fi
{%
%    \end{macrocode}
%    \begin{macro}{\ltx@ifundefined}
%    \begin{macrocode}
  \def\ltx@ifundefined#1{%
    \ifcsname #1\endcsname
      \expandafter\ifx\csname #1\endcsname\relax
        \expandafter\expandafter\expandafter\ltx@firstoftwo
      \else
        \expandafter\expandafter\expandafter\ltx@secondoftwo
      \fi
    \else
      \expandafter\ltx@firstoftwo
    \fi
  }%
%    \end{macrocode}
%    \end{macro}
%    \begin{macro}{\ltx@IfUndefined}
%    \begin{macrocode}
  \let\ltx@IfUndefined\ltx@ifundefined
%    \end{macrocode}
%    \end{macro}
%    \begin{macrocode}
}
%    \end{macrocode}
%
% \subsection{Stripping}
%
%    \begin{macro}{\ltx@RemovePrefix}
%    \begin{macrocode}
\def\ltx@RemovePrefix#1>{}
%    \end{macrocode}
%    \end{macro}
%
%    \begin{macro}{\ltx@StripPrefix}
%    \begin{macrocode}
\def\ltx@StripPrefix{%
  \expandafter\ltx@RemovePrefix
}
%    \end{macrocode}
%    \end{macro}
%
%    \begin{macro}{\ltx@onelevel@sanitize}
%    \begin{macrocode}
\def\ltx@onelevel@sanitize#1{%
  \edef#1{%
    \expandafter
    \ltx@RemovePrefix\meaning#1%
  }%
}
%    \end{macrocode}
%    \end{macro}
%
% \subsection{File management}
%
% \subsubsection{File extensions}
%
%    \begin{macro}{\ltx@clsextension}
%    \begin{macrocode}
\def\ltx@clsextension{cls}
%    \end{macrocode}
%    \end{macro}
%    \begin{macro}{\ltx@pkgextension}
%    \begin{macrocode}
\def\ltx@pkgextension{sty}
%    \end{macrocode}
%    \end{macro}
%
% \subsubsection{Load check}
%
%    \begin{macro}{\ltx@iffileloaded}
%    \begin{macrocode}
\def\ltx@iffileloaded#1{%
  \ltx@ifundefined{ver@#1}\ltx@secondoftwo\ltx@firstoftwo
}
%    \end{macrocode}
%    \end{macro}
%    \begin{macro}{\ltx@ifclassloaded}
%    \begin{macrocode}
\def\ltx@ifclassloaded#1{%
  \ltx@iffileloaded{#1.\ltx@clsextension}%
}
%    \end{macrocode}
%    \end{macro}
%    \begin{macro}{\ltx@ifpackageloaded}
%    \begin{macrocode}
\def\ltx@ifpackageloaded#1{%
  \ltx@iffileloaded{#1.\ltx@pkgextension}%
}
%    \end{macrocode}
%    \end{macro}
%
% \subsubsection{Version date check}
%
%    \begin{macro}{\ltx@iffilelater}
%    \begin{macrocode}
\def\ltx@iffilelater#1#2{%
  \ltx@iffileloaded{#1}{%
    \expandafter\LTXcmds@IfLater\expandafter{%
      \number
      \expandafter\expandafter\expandafter\LTXcmds@ParseVersion
      \expandafter\expandafter\expandafter{%
        \csname ver@#1\endcsname
      }%
    \expandafter}\expandafter{%
      \number
      \expandafter\LTXcmds@ParseVersion\expandafter{#2}%
    }%
  }\ltx@secondoftwo
}
%    \end{macrocode}
%    \end{macro}
%    \begin{macro}{\LTXcmds@IfLater}
%    \begin{macrocode}
\def\LTXcmds@IfLater#1#2{%
  \ifcase 0%
      \ifnum#1<19940101 %
      \else
        \ifnum#2<19940101 %
        \else
          \ifnum#2>#1 %
          \else
            1%
          \fi
        \fi
      \fi
      \ltx@space
    \expandafter\ltx@secondoftwo
  \else
    \expandafter\ltx@firstoftwo
  \fi
}
%    \end{macrocode}
%    \end{macro}
%    \begin{macro}{\ltx@ifclasslater}
%    \begin{macrocode}
\def\ltx@ifclasslater#1{%
  \ltx@iffilelater{#1.\ltx@clsextension}%
}
%    \end{macrocode}
%    \end{macro}
%    \begin{macro}{\ltx@ifpackagelater}
%    \begin{macrocode}
\def\ltx@ifpackagelater#1{%
  \ltx@iffilelater{#1.\ltx@pkgextension}%
}
%    \end{macrocode}
%    \end{macro}
%
%    \begin{macrocode}
\ltx@IfUndefined{pdfmatch}{%
%    \end{macrocode}
%    \begin{macro}{\LTXcmds@ParseVersion}
%    \begin{macrocode}
  \def\LTXcmds@ParseVersion#1{%
    \LTXcmds@@ParseVersion#10000/00/00\@nil
  }%
%    \end{macrocode}
%    \end{macro}
%    \begin{macro}{\LTXcmds@@ParseVersion}
%    \begin{macrocode}
  \def\LTXcmds@@ParseVersion#1#2#3#4/#5#6/#7#8#9\@nil{%
    #1#2#3#4#5#6#7#8%
  }%
%    \end{macrocode}
%    \end{macro}
%    \begin{macrocode}
}{%
%    \end{macrocode}
%    \begin{macro}{\LTXcmds@ParseVersion}
%    \begin{macrocode}
  \def\LTXcmds@ParseVersion#1{%
    \ifnum\pdfmatch{%
      ^%
      (199[4-9]|[2-9][0-9][0-9][0-9])/%
      (0[1-9]|1[0-2])/%
      (0[1-9]|[1-2][0-9]|3[0-1])%
    }{#1}=1 %
      \ltx@StripPrefix\pdflastmatch1 %
      \ltx@StripPrefix\pdflastmatch2 %
      \ltx@StripPrefix\pdflastmatch3 %
    \else
      0%
    \fi
  }%
%    \end{macrocode}
%    \end{macro}
%    \begin{macrocode}
}
%    \end{macrocode}
%
% \subsection{Macro additions}
%
%    \begin{macro}{\ltx@GlobalAppendToMacro}
%    \begin{macrocode}
\long\def\ltx@GlobalAppendToMacro#1#2{%
  \ifx\ltx@undefined#1%
    \let#1\ltx@empty
  \else
    \ifx\relax#1%
      \let#1\ltx@empty
    \fi
  \fi
  \begingroup
    \ltx@LocToksA\expandafter{#1#2}%
    \xdef#1{\the\ltx@LocToksA}%
  \endgroup
}
%    \end{macrocode}
%    \end{macro}
%    \begin{macro}{\ltx@LocalAppendToMacro}
%    \begin{macrocode}
\long\def\ltx@LocalAppendToMacro#1#2{%
  \global\let\LTXcmds@gtemp#1%
  \ifx\ltx@undefined\LTXcmds@gtemp
    \global\let\LTXcmds@gtemp\ltx@empty
  \else
    \ifx\relax\LTXcmds@gtemp
      \global\letLTXcmds@gtemp\ltx@empty
    \fi
  \fi
  \begingroup
    \ltx@LocToksA\expandafter{\LTXcmds@gtemp#2}%
    \xdef\LTXcmds@gtemp{\the\ltx@LocToksA}%
  \endgroup
  \let#1\LTXcmds@gtemp
}
%    \end{macrocode}
%    \end{macro}
%
%    \begin{macro}{\ltx@GlobalPrependToMacro}
%    \begin{macrocode}
\long\def\ltx@GlobalPrependToMacro#1#2{%
  \ifx\ltx@undefined#1%
    \let#1\ltx@empty
  \else
    \ifx\relax#1%
      \let#1\ltx@empty
    \fi
  \fi
  \begingroup
    \ltx@LocToksA{#2}%
    \ltx@LocToksB\expandafter{#1}%
    \xdef#1{\the\ltx@LocToksA\the\ltx@LocToksB}%
  \endgroup
}
%    \end{macrocode}
%    \end{macro}
%    \begin{macro}{\ltx@LocalPrependToMacro}
%    \begin{macrocode}
\long\def\ltx@LocalPrependToMacro#1#2{%
  \global\let\LTXcmds@gtemp#1%
  \ifx\ltx@undefined\LTXcmds@gtemp
    \global\let\LTXcmds@gtemp\ltx@empty
  \else
    \ifx\relax\LTXcmds@gtemp
      \global\letLTXcmds@gtemp\ltx@empty
    \fi
  \fi
  \begingroup
    \ltx@LocToksA{#2}%
    \ltx@LocToksB\expandafter{\LTXcmds@gtemp}%
    \xdef\LTXcmds@gtemp{\the\ltx@LocToksA\the\ltx@LocToksB}%
  \endgroup
  \let#1\LTXcmds@gtemp
}
%    \end{macrocode}
%    \end{macro}
%
% \subsection{Next character detection}
%
%    \begin{macro}{\ltx@ifnextchar}
%    \begin{macrocode}
\long\def\ltx@ifnextchar#1#2#3{%
  \begingroup
  \let\LTXcmds@CharToken= #1\relax
  \ltx@LocToksA{\endgroup#2}%
  \ltx@LocToksB{\endgroup#3}%
  \futurelet\LTXcmds@LetToken\LTXcmds@ifnextchar
}
%    \end{macrocode}
%    \end{macro}
%    \begin{macro}{\LTXcmds@ifnextchar}
%    \begin{macrocode}
\def\LTXcmds@ifnextchar{%
  \ifx\LTXcmds@LetToken\LTXcmds@CharToken
    \the\expandafter\ltx@LocToksA
  \else
    \expandafter
      \ifx\csname LTXcmds@LetToken\endcsname\LTXcmds@SpaceToken
      \expandafter\expandafter\expandafter\LTXcmds@@ifnextchar
    \else
      \the\expandafter\expandafter\expandafter\ltx@LocToksB
    \fi
  \fi
}
%    \end{macrocode}
%    \end{macro}
%    \begin{macro}{\LTXcmds@@ifnextchar}
%    \cs{futurelet} does not distinguish between a
%    character and a command that is a character
%    (defined by using \cs{let} or \cs{futurelet}).
%    Therefore the space is catched by \cs{romannumeral}
%    with negative character constant that gobbles one
%    optional space.
%    \begin{macrocode}
\def\LTXcmds@@ifnextchar{%
  \expandafter\futurelet
  \expandafter\LTXcmds@LetToken
  \expandafter\LTXcmds@ifnextchar
  \romannumeral-`\.%
}
%    \end{macrocode}
%    \end{macro}
%    \begin{macro}{\LTXcmds@SpaceToken}
%    \begin{macrocode}
\ltx@firstofone{\let\LTXcmds@SpaceToken= } %
%    \end{macrocode}
%    \end{macro}
%
%    \begin{macro}{\ltx@ifnextchar@nospace}
%    \begin{macrocode}
\long\def\ltx@ifnextchar@nospace#1#2#3{%
  \begingroup
  \let\LTXcmds@CharToken= #1\relax
  \ltx@LocToksA{\endgroup#2}%
  \ltx@LocToksB{\endgroup#3}%
  \futurelet\LTXcmds@LetToken\LTXcmds@ifnextchar@nospace
}
%    \end{macrocode}
%    \end{macro}
%    \begin{macro}{\LTXcmds@ifnextchar@nospace}
%    \begin{macrocode}
\def\LTXcmds@ifnextchar@nospace{%
  \the
  \ifx\LTXcmds@LetToken\LTXcmds@CharToken
    \expandafter\ltx@LocToksA
  \else
    \expandafter\ltx@LocToksB
  \fi
}
%    \end{macrocode}
%    \end{macro}
% \subsection{\cs{ltx@leavevmode}, \cs{ltx@mbox}}
%
%    \begin{macro}{\ltx@leavevmode}
%    \begin{macrocode}
\ltx@IfUndefined{quitvmode}{%
  \ltx@IfUndefined{leavevmode}{%
    \ltx@IfUndefined{voidb@x}{%
      \ltx@IfUndefined{newbox}{%
        \def\ltx@leavevmode{%
          \begingroup
            \setbox\ltx@zero=\hbox{}%
            \begingroup
              \setbox\ltx@zero=\hbox{\box\ltx@zero}%
            \endgroup
            \unhbox\ltx@zero
          \endgroup
        }%
      }{%
        \csname newbox\endcsname\LTXcmds@VoidBox
        \ifvoid\LTXcmds@VoidBox
        \else
          \setbox\LTXcmds@VoidBox=\hbox{}%
          \begingroup
            \setbox\LTXcmds@VoidBox=\hbox{\box\LTXcmds@VoidBox}%
          \endgroup
        \fi
        \def\ltx@leavevmode{\unhbox\LTXcmds@VoidBox}%
      }%
    }{%
      \def\ltx@leavevmode{\unhbox\voidb@x}%
    }%
  }{%
    \let\ltx@leavevmode\leavevmode
  }%
}{%
  \let\ltx@leavevmode\quitvmode
}
%    \end{macrocode}
%    \end{macro}
%
%    \begin{macro}{\ltx@mbox}
%    \begin{macrocode}
\def\ltx@mbox{%
  \ltx@leavevmode
  \hbox
}
%    \end{macrocode}
%    \end{macro}
%
% \subsection{Help macros}
%
%    \begin{macro}{\LTXcmds@num}
%    \begin{macrocode}
\ltx@IfUndefined{numexpr}{%
  \def\LTXcmds@num#1{%
    \expandafter\ltx@firstofone\expandafter{%
      \number#1%
    }%
  }%
}{%
  \def\LTXcmds@num#1{%
    \expandafter\ltx@firstofone\expandafter{%
      \the\numexpr#1%
    }%
  }%
}
%    \end{macrocode}
%    \end{macro}
%
% \subsection{Expandable test for emptiness}
%
%    \begin{macrocode}
\ltx@IfUndefined{detokenize}{%
%    \end{macrocode}
%
% \subsubsection{Vanilla \hologo{TeX}}
%
%    \begin{macro}{\ltx@ifempty}
%    The macro is based on \cs{@ifempty} of Robert R. Schneck
%    \cite{ifempty:schneck}
%    and \cs{@ifnull} of Ulrich Diez \cite{ifnull:diez}.
%    There are three cases to consider:
%    \begin{compactenum}
%    \item |#1| is empty,
%    \item |#1| is not empty and the first token
%      is not a begingroup character,
%    \item |#1| starts with a begingroup character (catcode 1).
%    \end{compactenum}
%    \begin{macrocode}
  \def\LTXcmds@temp#1{%
    \long\def\ltx@ifempty##1{%
      \romannumeral0%
      \iffalse{\fi
        \expandafter\ltx@gobble\expandafter{%
          \expandafter{\string##1}%
          \expandafter\ltx@gobble\string
        }%
        \expandafter\ltx@firstofthree\expandafter
        {\iffalse}\fi
        \expandafter#1\ltx@secondoftwo
      }%
      \expandafter#1\ltx@firstoftwo
    }%
%    \end{macrocode}
%    \end{macro}
%    \begin{macro}{\ltx@ifblank}
%    \begin{macrocode}
    \long\def\ltx@ifblank##1{%
      \romannumeral0%
      \iffalse{\fi
        \expandafter\expandafter\expandafter\ltx@gobble
        \expandafter\expandafter\expandafter{%
          \expandafter\expandafter\expandafter{%
            \expandafter\string\ltx@gobble##1.%
          }%
          \expandafter\ltx@gobble\string
        }%
        \expandafter\ltx@firstofthree\expandafter
        {\iffalse}\fi
        \expandafter#1\ltx@secondoftwo
      }%
      \expandafter#1\ltx@firstoftwo
    }%
  }%
  \LTXcmds@temp{ }%
%    \end{macrocode}
%    \end{macro}
%    \begin{macrocode}
}{%
%    \end{macrocode}
%
% \subsubsection{With \cs{detokenize}}
%
%    Ahmed Musa provided \cs{ifstrempty} using \cs{detokenize}
%    and \cs{pdfstrcmp} \cite{ifstrempty:musa}. Ulrich Diez, GL,
%    Heiko Oberdiek
%    improved it further by removing \cs{pdfstrcmp} and
%    taking three arguments
%    \cite{ifstrempty:diez1,ifstrempty:diez2,ifstrempty:diez3,^^A
%      ifstrempty:gl,ifstrempty:ho}.
%
%    \begin{macro}{\ltx@ifempty}
%    \begin{macrocode}
  \long\def\ltx@ifempty#1{%
    \romannumeral%
    \csname
      LTXcmds@ifempty%
      \ifcat$\detokenize{#1}$%
        @%
      \fi
    \endcsname
  }%
%    \end{macrocode}
%    \end{macro}
%    \begin{macro}{\LTXcmds@ifempty@}
%    \begin{macrocode}
  \long\def\LTXcmds@ifempty@#1#2{0 #1}%
%    \end{macrocode}
%    \end{macro}
%    \begin{macro}{\LTXcmds@ifempty}
%    \begin{macrocode}
  \long\def\LTXcmds@ifempty#1#2{0 #2}%
%    \end{macrocode}
%    \end{macro}
%
% \subsubsection{\cs{ltx@ifblank}}
%
%    \begin{macro}{\ltx@ifblank}
%    \begin{macrocode}
  \long\def\ltx@ifblank#1{%
    \romannumeral%
    \csname
      LTXcmds@ifempty%
      \ifcat$\detokenize\expandafter{\ltx@gobble#1.}$%
        @%
      \fi
    \endcsname
  }%
%    \end{macrocode}
%    \end{macro}
%
%    \begin{macrocode}
}
%    \end{macrocode}
%
% \subsection{\cs{ltx@zapspace}}
%
%    \begin{macro}{\ltx@zapspace}
%    \begin{macrocode}
\long\def\ltx@zapspace#1{%
  \romannumeral
  \LTXcmds@zapspace\ltx@zero#1 \@nil
}
%    \end{macrocode}
%    \end{macro}
%    \begin{macro}{\LTXcmds@zapspace}
%    \begin{macrocode}
\long\def\LTXcmds@zapspace#1 #2\@nil{%
  \ltx@ifempty{#2}{%
    #1%
  }{%
    \LTXcmds@zapspace#1#2\@nil
  }%
}
%    \end{macrocode}
%    \end{macro}
%
% \subsection{\cs{ltx@IfBoxEmpty}}
%
%    In case of \hologo{eTeX} the test for an empty box
%    is done via \cs{lastnodetype} as suggested
%    by David Kastrup \cite{ifboxempty:dak}.
%    \begin{macrocode}
\ltx@IfUndefined{lastnodetype}{%
  \catcode`\$=9 %
  \catcode`\&=14 %
}{%
  \catcode`\$=14 %
  \catcode`\&=9 %
}
%    \end{macrocode}
%    \begin{macro}{\ltx@IfBoxEmpty}
%    \begin{macrocode}
\def\ltx@IfBoxEmpty#1{%
  \ifvoid#1\relax
    \expandafter\ltx@secondoftwo
  \else
%    \end{macrocode}
%    Implementation using \hologo{eTeX}'s \cs{lastnodetype}.
%    \begin{macrocode}
&   \begingroup
&     \setbox\ltx@zero=\ifhbox#1\hbox\else\vbox\fi{%
&       \ifhmode\unhcopy\else\unvcopy\fi#1\relax
&       \expandafter
&     }%
&   \expandafter\endgroup
&   \ifnum\lastnodetype<\ltx@zero
&     \expandafter\expandafter\expandafter\ltx@firstoftwo
&   \else
&     \expandafter\expandafter\expandafter\ltx@secondoftwo
&   \fi
%    \end{macrocode}
%    Implementation without \hologo{eTeX} using a signature
%    at the beginning of the test box.
%    \begin{macrocode}
$   \begingroup
$     \setbox\ltx@zero=\ifhbox#1\hbox\else\vbox\fi{%
$       \penalty\ltx@one
$       \ifhmode\unhcopy\else\unvcopy\fi#1\relax
$       \expandafter
$     }%
$     \ifnum\lastpenalty=\ltx@one
%    \end{macrocode}
%    Box 0 has been changed and is restored by
%    closing the group.
%    \begin{macrocode}
$       \endgroup
$       \begingroup
$       \setbox\ltx@zero=\ifhbox#1\hbox\else\vbox\fi{%
$         \penalty\ltx@two
$         \ifhmode\unhcopy\else\unvcopy\fi#1\relax
$         \expandafter
$       }%
$       \ifnum\lastpenalty=\ltx@two
$         \def\next{\endgroup\expandafter\ltx@firstoftwo}%
$       \else
$         \def\next{\endgroup\expandafter\ltx@secondoftwo}%
$       \fi
$     \else
$       \def\next{\endgroup\expandafter\ltx@secondoftwo}%
$     \fi
$   \next
  \fi
}
%    \end{macrocode}
%    \end{macro}
%
%    \begin{macro}{\ltx@IfBoxVoidOrEmpty}
%    \begin{macrocode}
\def\ltx@IfBoxVoidOrEmpty#1{%
  \ifvoid#1\relax
    \expandafter\ltx@thirdoffour
  \fi
  \ltx@IfBoxEmpty{#1}%
}
%    \end{macrocode}
%    \end{macro}
%
%    \begin{macrocode}
\LTXcmds@AtEnd%
%</package>
%    \end{macrocode}
%% \section{Installation}
%
% \subsection{Download}
%
% \paragraph{Package.} This package is available on
% CTAN\footnote{\CTANpkg{ltxcmds}}:
% \begin{description}
% \item[\CTAN{macros/latex/contrib/ltxcmds/ltxcmds.dtx}] The source file.
% \item[\CTAN{macros/latex/contrib/ltxcmds/ltxcmds.pdf}] Documentation.
% \end{description}
%
%
% \paragraph{Bundle.} All the packages of the bundle `ltxcmds'
% are also available in a TDS compliant ZIP archive. There
% the packages are already unpacked and the documentation files
% are generated. The files and directories obey the TDS standard.
% \begin{description}
% \item[\CTANinstall{install/macros/latex/contrib/ltxcmds.tds.zip}]
% \end{description}
% \emph{TDS} refers to the standard ``A Directory Structure
% for \TeX\ Files'' (\CTANpkg{tds}). Directories
% with \xfile{texmf} in their name are usually organized this way.
%
% \subsection{Bundle installation}
%
% \paragraph{Unpacking.} Unpack the \xfile{ltxcmds.tds.zip} in the
% TDS tree (also known as \xfile{texmf} tree) of your choice.
% Example (linux):
% \begin{quote}
%   |unzip ltxcmds.tds.zip -d ~/texmf|
% \end{quote}
%
% \subsection{Package installation}
%
% \paragraph{Unpacking.} The \xfile{.dtx} file is a self-extracting
% \docstrip\ archive. The files are extracted by running the
% \xfile{.dtx} through \plainTeX:
% \begin{quote}
%   \verb|tex ltxcmds.dtx|
% \end{quote}
%
% \paragraph{TDS.} Now the different files must be moved into
% the different directories in your installation TDS tree
% (also known as \xfile{texmf} tree):
% \begin{quote}
% \def\t{^^A
% \begin{tabular}{@{}>{\ttfamily}l@{ $\rightarrow$ }>{\ttfamily}l@{}}
%   ltxcmds.sty & tex/generic/ltxcmds/ltxcmds.sty\\
%   ltxcmds.pdf & doc/latex/ltxcmds/ltxcmds.pdf\\
%   ltxcmds.dtx & source/latex/ltxcmds/ltxcmds.dtx\\
% \end{tabular}^^A
% }^^A
% \sbox0{\t}^^A
% \ifdim\wd0>\linewidth
%   \begingroup
%     \advance\linewidth by\leftmargin
%     \advance\linewidth by\rightmargin
%   \edef\x{\endgroup
%     \def\noexpand\lw{\the\linewidth}^^A
%   }\x
%   \def\lwbox{^^A
%     \leavevmode
%     \hbox to \linewidth{^^A
%       \kern-\leftmargin\relax
%       \hss
%       \usebox0
%       \hss
%       \kern-\rightmargin\relax
%     }^^A
%   }^^A
%   \ifdim\wd0>\lw
%     \sbox0{\small\t}^^A
%     \ifdim\wd0>\linewidth
%       \ifdim\wd0>\lw
%         \sbox0{\footnotesize\t}^^A
%         \ifdim\wd0>\linewidth
%           \ifdim\wd0>\lw
%             \sbox0{\scriptsize\t}^^A
%             \ifdim\wd0>\linewidth
%               \ifdim\wd0>\lw
%                 \sbox0{\tiny\t}^^A
%                 \ifdim\wd0>\linewidth
%                   \lwbox
%                 \else
%                   \usebox0
%                 \fi
%               \else
%                 \lwbox
%               \fi
%             \else
%               \usebox0
%             \fi
%           \else
%             \lwbox
%           \fi
%         \else
%           \usebox0
%         \fi
%       \else
%         \lwbox
%       \fi
%     \else
%       \usebox0
%     \fi
%   \else
%     \lwbox
%   \fi
% \else
%   \usebox0
% \fi
% \end{quote}
% If you have a \xfile{docstrip.cfg} that configures and enables \docstrip's
% TDS installing feature, then some files can already be in the right
% place, see the documentation of \docstrip.
%
% \subsection{Refresh file name databases}
%
% If your \TeX~distribution
% (\TeX\,Live, \mikTeX, \dots) relies on file name databases, you must refresh
% these. For example, \TeX\,Live\ users run \verb|texhash| or
% \verb|mktexlsr|.
%
% \subsection{Some details for the interested}
%
% \paragraph{Unpacking with \LaTeX.}
% The \xfile{.dtx} chooses its action depending on the format:
% \begin{description}
% \item[\plainTeX:] Run \docstrip\ and extract the files.
% \item[\LaTeX:] Generate the documentation.
% \end{description}
% If you insist on using \LaTeX\ for \docstrip\ (really,
% \docstrip\ does not need \LaTeX), then inform the autodetect routine
% about your intention:
% \begin{quote}
%   \verb|latex \let\install=y\input{ltxcmds.dtx}|
% \end{quote}
% Do not forget to quote the argument according to the demands
% of your shell.
%
% \paragraph{Generating the documentation.}
% You can use both the \xfile{.dtx} or the \xfile{.drv} to generate
% the documentation. The process can be configured by the
% configuration file \xfile{ltxdoc.cfg}. For instance, put this
% line into this file, if you want to have A4 as paper format:
% \begin{quote}
%   \verb|\PassOptionsToClass{a4paper}{article}|
% \end{quote}
% An example follows how to generate the
% documentation with pdf\LaTeX:
% \begin{quote}
%\begin{verbatim}
%pdflatex ltxcmds.dtx
%makeindex -s gind.ist ltxcmds.idx
%pdflatex ltxcmds.dtx
%makeindex -s gind.ist ltxcmds.idx
%pdflatex ltxcmds.dtx
%\end{verbatim}
% \end{quote}
%
% \begin{thebibliography}{9}
% \raggedright
%
% \bibitem{ifempty:schneck}
% Robert R. Schneck:
% \textit{Re: \cs{ifempty} solution %
%   (was Macro puzzle: maximally general \cs{ifempty})};
% newsgroup \xnewsgroup{comp.text.tex},
% \url{news:3eef1ada_6@corp.newsgroups.com},
% 2003-06-17.\\
% {\small
% \url{https://groups.google.com/group/comp.text.tex/msg/be03a159ec374895}}
%
% \bibitem{ifnull:diez}
% Ulrich Diez:
% \textit{Re: TeX refuses to strip outer braces in argument};
% newsgroup \xnewsgroup{comp.text.tex},
% \url{news:ibk3t8$ee7$1@news.albasani.net},
% 2010-11-12.\\
% {\small
% \url{https://groups.google.com/group/comp.text.tex/msg/803bd57221a04996}}
%
% \bibitem{ifstrempty:musa}
% Ahmed Musa:
% \textit{Re: TeX refuses to strip outer braces in argument};
% newsgroup \xnewsgroup{comp.text.tex},
% \url{news:f5496afe-40ed-42bd-b629-a2419ecf7c0d@o14g2000prn.googlegroups.com},
% 2010-12-03.\\
% {\small
% \url{https://groups.google.com/group/comp.text.tex/msg/fbf7d61a0c3a807d}}
%
% \bibitem{ifstrempty:diez1}
% Ulrich Diez:
% \textit{Re: TeX refuses to strip outer braces in argument};
% newsgroup \xnewsgroup{comp.text.tex},
% \url{news:idbo94$uka$1@four.albasani.net},
% 2010-12-03.\\
% {\small
% \url{https://groups.google.com/group/comp.text.tex/msg/0c230ee479487962}}
%
% \bibitem{ifstrempty:diez2}
% Ulrich Diez:
% \textit{Re: TeX refuses to strip outer braces in argument};
% newsgroup \xnewsgroup{comp.text.tex},
% \url{news:idbpu4$cg1$1@news.albasani.net},
% 2010-12-03.\\
% {\small
% \url{https://groups.google.com/group/comp.text.tex/msg/bbef4263390d647b}}
%
% \bibitem{ifstrempty:diez3}
% Ulrich Diez:
% \textit{Re: TeX refuses to strip outer braces in argument};
% newsgroup \xnewsgroup{comp.text.tex},
% \url{news:idd4ga$r83$1@four.albasani.net},
% 2010-12-04.\\
% {\small
% \url{https://groups.google.com/group/comp.text.tex/msg/00dfd1ec103cd272}}
%
% \bibitem{ifstrempty:gl}
% GL:
% \textit{Re: TeX refuses to strip outer braces in argument};
% newsgroup \xnewsgroup{comp.text.tex},
% \url{news:4cfa2e27$0$7389$426a74cc@news.free.fr},
% 2010-12-04.\\
% {\small
% \url{https://groups.google.com/group/comp.text.tex/msg/d3a75995c1cf267e}}
%
% \bibitem{ifstrempty:ho}
% Heiko Oberdiek:
% \textit{Re: TeX refuses to strip outer braces in argument};
% newsgroup \xnewsgroup{comp.text.tex},
% \url{news:iddhq1$3kj$1@news.eternal-september.org},
% 2010-12-04.\\
% {\small
% \url{https://groups.google.com/group/comp.text.tex/msg/5f7a23e3ab70e347}}
%
% \bibitem{ifboxempty:dak}
% David Kastrup:
% \textit{How to detect if \cs{vbox} is empty};
% newsgroup \xnewsgroup{comp.text.tex},
% 2011-02-04.\\
% {\small
% \url{https://groups.google.com/group/comp.text.tex/msg/8d3cb89496a4d86d}}
%
% \end{thebibliography}
%
% \begin{History}
%   \begin{Version}{2009/08/05 v1.0}
%   \item
%     First version.
%   \end{Version}
%   \begin{Version}{2009/12/12 v1.1}
%   \item
%     Short title shortened.
%   \item
%     \cs{ltx@IfUndefined} added.
%   \end{Version}
%   \begin{Version}{2010/01/28 v1.2}
%   \item
%     \cs{ltx@RemovePrefix} and \cs{ltx@StripPrefix} added.
%   \item
%     \cs{ltx@ifclassloaded}, \cs{ltx@ifpackageloaded},
%     \cs{ltx@iffileloaded},
%     \cs{ltx@ifclasslater}, \cs{ltx@ifpackagelater},
%     \cs{ltx@iffilelater},
%     \cs{ltx@clsextension}, \cs{ltx@pkgextension} added.
%   \item
%     \cs{ltx@GlobalAppendToMacro}, \cs{ltx@LocalAppendToMacro} added.
%   \end{Version}
%   \begin{Version}{2010/03/01 v1.3}
%   \item
%     \cs{ltx@newif} added.
%   \item
%     \cs{ltx@ifnextchar} added.
%   \item
%     Numbers \cs{ltx@zero}, \cs{ltx@one}, \cs{ltx@two}, \cs{ltx@cclv} added.
%   \end{Version}
%   \begin{Version}{2010/03/09 v1.4}
%   \item
%     \cs{ltx@pkgextension} and \cs{ltx@clsextension} are hardcoded to
%     avoid trouble with \cs{@onlypreamble}.
%   \end{Version}
%   \begin{Version}{2010/04/08 v1.5}
%   \item
%     \cs{ltx@cartwo}, \cs{ltx@cdrtwo},
%     \cs{ltx@carthree}, \cs{ltx@cdrthree},
%     \cs{ltx@carfour}, \cs{ltx@cdrfour} added.
%   \item
%     \cs{ltx@ReturnAfterFi} and \cs{ltx@ReturnAfterElseFi} fixed.
%   \end{Version}
%   \begin{Version}{2010/04/16 v1.6}
%   \item
%     \cs{ltx@leavevmode}, \cs{ltx@mbox} added.
%   \end{Version}
%   \begin{Version}{2010/04/26 v1.7}
%   \item
%     \cs{ltx@GobbleNum}, \cs{ltx@CdrNum}, \cs{ltx@CarNum} added.
%   \item
%     \cs{ltx@carzero}, \cs{ltx@cdrzero} added.
%   \item
%     \cs{ltx@hashchar} added.
%   \end{Version}
%   \begin{Version}{2010/09/11 v1.8}
%   \item
%     \cs{ltx@leftbracechar}, \cs{ltx@rightbracechar} added.
%   \end{Version}
%   \begin{Version}{2010/10/25 v1.9}
%   \item
%     \cs{ltx@LocalAppendToMacro} and \cs{ltx@GlobalAppendToMacro} are
%     now \cs{long}.
%   \end{Version}
%   \begin{Version}{2010/10/31 v1.10}
%   \item
%     \cs{ltx@newglobalif} added.
%   \end{Version}
%   \begin{Version}{2010/11/12 v1.11}
%   \item
%     \cs{ltx@ifempty} added.
%   \item
%     \cs{ltx@firstofthree}, \cs{ltx@secondofthree}, \cs{ltx@thirdofthree}
%     added.
%   \end{Version}
%   \begin{Version}{2010/12/02 v1.12}
%   \item
%     \cs{ltx@onelevel@sanitize} added.
%   \item
%     \cs{LTXcmds@num} fixed for the case with \cs{numexpr}
%     (bug found by GL).
%   \end{Version}
%   \begin{Version}{2010/12/04 v1.13}
%   \item
%     \cs{ltx@ifblank} added.
%   \item
%     Optimization for \cs{ltx@ifempty}.
%   \end{Version}
%   \begin{Version}{2010/12/07 v1.14}
%   \item
%     \cs{ltx@zapspace} added.
%   \end{Version}
%   \begin{Version}{2010/12/12 v1.15}
%   \item
%     \cs{ltx@minusone} added.
%   \end{Version}
%   \begin{Version}{2011/02/04 v1.16}
%   \item
%     \cs{ltx@IfBoxEmpty} and \cs{ltx@IfBoxVoidOrEmpty} added.
%   \item
%     \cs{ltx@firstoffour}, \dots, \cs{ltx@fourthoffour} added.
%   \end{Version}
%   \begin{Version}{2011/02/05 v1.17}
%   \item
%     \cs{ltx@IfBoxEmpty}: an empty box may have non-zero dimensions.
%   \end{Version}
%   \begin{Version}{2011/03/16 v1.18}
%   \item
%     \cs{ltx@ifclasslater} fixed.
%   \end{Version}
%   \begin{Version}{2011/04/14 v1.19}
%   \item
%     \cs{ltx@ifnextchar}: detection of optional spaces modified.
%   \item
%     \cs{ltx(Loc,Glob)(Toks,Dimen,Skip)(A,B,C,D,E)} added.
%   \end{Version}
%   \begin{Version}{2011/04/18 v1.20}
%   \item
%     \cs{ltx@ifnextchar} with conditional support (thanks GL for
%     bug report).
%   \end{Version}
%   \begin{Version}{2011/08/22 v1.21}
%   \item
%     \cs{ltx@GlobalPrependToMacro}, \cs{ltx@LocalPrependToMacro} added
%     (feature request of Martin M\"unch).
%   \end{Version}
%   \begin{Version}{2011/11/09 v1.22}
%   \item
%     \cs{ltx@carsecond}, \cs{ltx@carthird}, \cs{ltx@carfourth},
%     \cs{ltx@CarNumth} added.
%   \item
%     \cs{ltx@cdrzero}, \cs{ltx@cdr}, \cs{ltx@cdrtwo}, cs{ltx@cdrthree},
%     \cs{ltx@cdrfour}, \cs{ltx@CdrNum} modified to retain braces and spaces.
%     They are expandable in two expansion steps.
%   \end{Version}
%   \begin{Version}{2016/05/16 v1.23}
%   \item
%     Documentation updates.
%   \end{Version}
%   \begin{Version}{2019/12/15 v1.24}
%   \item
%     Documentation updates.
%   \end{Version}
% \end{History}
%
% \PrintIndex
%
% \Finale
\endinput

%        (quote the arguments according to the demands of your shell)
%
% Documentation:
%    (a) If ltxcmds.drv is present:
%           latex ltxcmds.drv
%    (b) Without ltxcmds.drv:
%           latex ltxcmds.dtx; ...
%    The class ltxdoc loads the configuration file ltxdoc.cfg
%    if available. Here you can specify further options, e.g.
%    use A4 as paper format:
%       \PassOptionsToClass{a4paper}{article}
%
%    Programm calls to get the documentation (example):
%       pdflatex ltxcmds.dtx
%       makeindex -s gind.ist ltxcmds.idx
%       pdflatex ltxcmds.dtx
%       makeindex -s gind.ist ltxcmds.idx
%       pdflatex ltxcmds.dtx
%
% Installation:
%    TDS:tex/generic/ltxcmds/ltxcmds.sty
%    TDS:doc/latex/ltxcmds/ltxcmds.pdf
%    TDS:source/latex/ltxcmds/ltxcmds.dtx
%
%<*ignore>
\begingroup
  \catcode123=1 %
  \catcode125=2 %
  \def\x{LaTeX2e}%
\expandafter\endgroup
\ifcase 0\ifx\install y1\fi\expandafter
         \ifx\csname processbatchFile\endcsname\relax\else1\fi
         \ifx\fmtname\x\else 1\fi\relax
\else\csname fi\endcsname
%</ignore>
%<*install>
\input docstrip.tex
\Msg{************************************************************************}
\Msg{* Installation}
\Msg{* Package: ltxcmds 2019/12/15 v1.24 LaTeX kernel commands for general use (HO)}
\Msg{************************************************************************}

\keepsilent
\askforoverwritefalse

\let\MetaPrefix\relax
\preamble

This is a generated file.

Project: ltxcmds
Version: 2019/12/15 v1.24

Copyright (C)
   2009-2011 Heiko Oberdiek
   2016-2019 Oberdiek Package Support Group

This work may be distributed and/or modified under the
conditions of the LaTeX Project Public License, either
version 1.3c of this license or (at your option) any later
version. This version of this license is in
   https://www.latex-project.org/lppl/lppl-1-3c.txt
and the latest version of this license is in
   https://www.latex-project.org/lppl.txt
and version 1.3 or later is part of all distributions of
LaTeX version 2005/12/01 or later.

This work has the LPPL maintenance status "maintained".

The Current Maintainers of this work are
Heiko Oberdiek and the Oberdiek Package Support Group
https://github.com/ho-tex/ltxcmds/issues


The Base Interpreter refers to any `TeX-Format',
because some files are installed in TDS:tex/generic//.

This work consists of the main source file ltxcmds.dtx
and the derived files
   ltxcmds.sty, ltxcmds.pdf, ltxcmds.ins, ltxcmds.drv,
   ltxcmds-test1.tex, ltxcmds-test-gobble.tex,
   ltxcmds-test-ifempty.tex, ltxcmds-test-zapspace.tex,
   ltxcmds-test-ifboxempty.tex, ltxcmds-test-nextchar.tex,
   ltxcmds-test-carcdr.tex.

\endpreamble
\let\MetaPrefix\DoubleperCent

\generate{%
  \file{ltxcmds.ins}{\from{ltxcmds.dtx}{install}}%
  \file{ltxcmds.drv}{\from{ltxcmds.dtx}{driver}}%
  \usedir{tex/generic/ltxcmds}%
  \file{ltxcmds.sty}{\from{ltxcmds.dtx}{package}}%
}

\catcode32=13\relax% active space
\let =\space%
\Msg{************************************************************************}
\Msg{*}
\Msg{* To finish the installation you have to move the following}
\Msg{* file into a directory searched by TeX:}
\Msg{*}
\Msg{*     ltxcmds.sty}
\Msg{*}
\Msg{* To produce the documentation run the file `ltxcmds.drv'}
\Msg{* through LaTeX.}
\Msg{*}
\Msg{* Happy TeXing!}
\Msg{*}
\Msg{************************************************************************}

\endbatchfile
%</install>
%<*ignore>
\fi
%</ignore>
%<*driver>
\NeedsTeXFormat{LaTeX2e}
\ProvidesFile{ltxcmds.drv}%
  [2019/12/15 v1.24 LaTeX kernel commands for general use (HO)]%
\documentclass{ltxdoc}
\usepackage{holtxdoc}[2011/11/22]
\usepackage{zref-savepos}
\usepackage{paralist}
\hfuzz=1pt
\begin{document}
  \DocInput{ltxcmds.dtx}%
\end{document}
%</driver>
% \fi
%
%
%
% \GetFileInfo{ltxcmds.drv}
%
% \title{The \xpackage{ltxcmds} package}
% \date{2019/12/15 v1.24}
% \author{Heiko Oberdiek\thanks
% {Please report any issues at \url{https://github.com/ho-tex/ltxcmds/issues}}}
%
% \maketitle
%
% \begin{abstract}
% The package \xpackage{ltxcmds} exports some utility macros
% from the \LaTeX\ kernel into a separate namespace and
% also provides them for other formats such as plain-\TeX.
% \end{abstract}
%
% \tableofcontents
%
% \newcounter{to}
% \renewcommand*{\theto}{to\arabic{to}}
% \newlength\tolength
% \tolength=\dimexpr .5\linewidth+\oddsidemargin+1in+15mm\relax
% \newcommand*{\TO}{^^A
%   \quad
%   \stepcounter{to}^^A
%   \zsavepos\theto
%   \ifdim\zposx\theto sp>0pt
%     \ifdim\zposx\theto sp<\tolength
%       \kern\dimexpr\tolength-\zposx\theto sp\relax
%     \fi
%   \fi
%   $\rightarrow$\quad
%   \ignorespaces
% }
%
% \section{Documentation}
%
% \subsection{Introduction}
%
% Many of my packages also support other formats such as plain-\TeX.
% Because I am rather familiar with the utility macros from
% \LaTeX's kernel (e.g. \cs{@gobble}, \cs{@firstoftwo}), I found
% myself rewriting them again and again, because they are lacking
% in plain-\TeX.
%
% Therefore this package provides often used macros and similar
% ones with the name prefix \cs{ltx@}. This avoids also faulty
% redefinitions. I remember an example where a package redefined
% \cs{@firstoftwo} with forgetting \cs{long}.
%
% \subsection{Numbers}
%
% \begin{declcs}{ltx@zero} \TO 0\\
%   \cs{ltx@one} \TO 1\\
%   \cs{ltx@two} \TO 2\\
%   \cs{ltx@cclv} \TO 255\\
%   \cs{ltx@minusone} \TO -1
% \end{declcs}
% These commands are numbers 0, 1, 2, 255 and -1. They are not digits
% and a space is not gobbled afterwards. Macro \cs{ltx@minusone}
% is available since version 2010/12/12 v1.15.
%
% \subsection{Scratch registers}
%
% Following the conventions of \hologo{plainTeX} and \hologo{LaTeX}
% the first ten registers are free to use. Even numbered registers
% are for local, odd numbered for global use.
%
% \begin{declcs}{ltx@(Loc,Glob)(Toks,Dimen,Skip)(A,B,C,D,E)}
% \end{declcs}
% The name consists of the prefix \cs{ltx@}, then
% \texttt{Loc} or \texttt{Glob} for local or global usage follows.
% The register type is given by \texttt{Toks} for token register,
% \texttt{Dimen} for dimen register and \texttt{Skip} for skip
% register. As last part the registers are numbered from \texttt{A}
% to \texttt{E}. Example: \cs{ltx@LocToksA}.
%
% Since 2011/04/14 v1.19.
%
% \subsection{Argument killers}
%
% \begin{declcs}{ltx@gobble} \M{1} \TO\\
%   \cs{ltx@gobbletwo} \M{1} \M{2} \TO\\
%   \cs{ltx@gobblethree} \M{1} \M{2} \M{3} \TO\\
%   \cs{ltx@gobblefour} \M{1} \M{2} \M{3} \M{4} \TO
% \end{declcs}
%
% \begin{declcs}{ltx@GobbleNum} \M{num} \M{1} \M{2} \dots\space
%   \M{\meta{num}} \TO
% \end{declcs}
% The first argument \meta{num} of macro \cs{ltx@GobbleNum} specifies,
% how many following arguments are eaten. Macro \cs{ltx@GobbleNum}
% is expandable in exact two expansion steps.
%
% \subsection{Argument grabbers}
%
% \begin{declcs}{ltx@firstofone} \M{1} \TO \meta{1}\\[1ex]
%  \cs{ltx@firstoftwo} \M{1} \M{2} \TO \meta{1}\\
%  \cs{ltx@secondoftwo} \M{1} \M{2} \TO \meta{2}\\[1ex]
%  \cs{ltx@firstofthree} \M{1} \M{2} \M{3} \TO \meta{1}\\
%  \cs{ltx@secondofthree} \M{1} \M{2} \M{3} \TO \meta{2}\\
%  \cs{ltx@thirdofthree} \M{1} \M{2} \M{3} \TO \meta{3}\\
%  \cs{ltx@firstoffour} \M{1} \M{2} \M{3} \M{4} \TO \meta{1}\\
%  \cs{ltx@secondoffour} \M{1} \M{2} \M{3} \M{4} \TO \meta{2}\\
%  \cs{ltx@thirdoffour} \M{1} \M{2} \M{3} \M{4} \TO \meta{3}\\
%  \cs{ltx@fourthoffour} \M{1} \M{2} \M{3} \M{4} \TO \meta{4}
% \end{declcs}
% Macros \cs{ltx@firstofthree}, \cs{ltx@secondofthree}
% and \cs{ltx@thirdofthree} were added in version
% 2010/11/12 v1.11.
% Macros \cs{ltx@firstoffour}, \dots, \cs{ltx@fourthoffour}
% were added in version 2011/02/04 v1.16.
%
% \subsection{List helpers}
%
% \begin{declcs}{ltx@carzero} \dots\ \cs{@nil} \TO\\
%   \cs{ltx@cdrzero} \dots\ \cs{@nil} \TO \dots
% \end{declcs}
%
% \begin{declcs}{ltx@car} \M{1} \dots\ \cs{@nil} \TO \meta{1}\\
%   \cs{ltx@cdr} \M{1} \dots\ \cs{@nil} \TO \dots
% \end{declcs}
%
% \begin{declcs}{ltx@cartwo} \M{1} \M{2} \dots\ \cs{@nil}
%   \TO \meta{1}\meta{2}\\
%   \cs{ltx@carsecond} \M{1} \M{2} \dots\ \cs{@nil} \TO \meta{2}\\
%   \cs{ltx@cdrtwo} \M{1} \M{2} \dots\ \cs{@nil} \TO \dots
% \end{declcs}
%
% \begin{declcs}{ltx@carthree} \M{1} \M{2} \M{3} \dots\ \cs{@nil}
%       \TO \meta{1}\meta{2}\meta{3}\\
%   \cs{ltx@carthird} \M{1} \M{2} \M{3} \dots\ \cs{@nil}
%       \TO \meta{3}\\
%   \cs{ltx@cdrthree} \M{1} \M{2} \M{3} \dots\ \cs{@nil} \TO \dots
% \end{declcs}
%
% \begin{declcs}{ltx@carfour} \M{1} \M{2} \M{3} \M{4} \dots\ \cs{@nil}
%       \TO \meta{1}\meta{2}\meta{3}\meta{4}\\
%   \cs{ltx@carfourth} \M{1} \M{2} \M{3} \M{4} \dots\ \cs{@nil}
%       \TO \meta{4}\\
%   \cs{ltx@cdrfour} \M{1} \M{2} \M{3} \M{4} \dots\ \cs{@nil} \TO \dots
% \end{declcs}
%
% \begin{declcs}{ltx@CarNum} \M{num} \M{1} \dots\ \M{\meta{num}}
%   \M{\meta{num}+1} \dots\ \cs{@nil}\\
%   \TO \M{1} \dots\ \M{\meta{num}} \dots\\
%   \cs{ltx@CarNumth} \M{num} \M{1} \dots\ \M{\meta{num}}
%   \M{\meta{num}+1} \dots\ \cs{@nil}\\
%   \TO \M{\meta{num}} \dots\\
%   \cs{ltx@CdrNum} \M{num} \M{1} \dots\ \M{\meta{num}}
%   \M{\meta{num}+1} \dots\ \cs{@nil}\\
%   \TO \M{\meta{num}+1} \dots
% \end{declcs}
%
% Macros with uppercase letters are expandable in two expansion steps.
% Changes in version 2019/12/15 v1.24:
% \begin{itemize}
% \item
%   Macros \cs{ltx@carsecond}, \cs{ltx@carthird}, \cs{ltx@carfourth},
%   \cs{ltx@CarNumth} added.
% \item
%   Macros \cs{ltx@cdr}, \cs{ltx@cdrtwo}, \cs{ltx@cdrthree}, \cs{ltx@cdrfour},
%   \cs{ltx@CdrNum} are expandable in two expansion steps and retain
%   spaces and braces after the first gobbled arguments.
% \end{itemize}
%
% \subsection{Tail recursion}
%
% \begin{declcs}{ltx@ReturnAfterFi} \M{1} \cs{fi} \TO \cs{fi} \meta{1}\\
%   \cs{ltx@ReturnAfterElseFi} \M{1} \cs{else} \M{2} \cs{fi}
%     \TO \cs{fi} \meta{1}
% \end{declcs}
%
% \subsection{Empty macro}
%
% \begin{declcs}{ltx@empty} \TO
% \end{declcs}
%
% \subsection{Characters}
%
% \begin{declcs}{ltx@space} \TO \verb*| |\\
%   \cs{ltx@percentchar} \TO \texttt{\%}\\
%   \cs{ltx@backslashchar} \TO
%      \makeatletter\texttt{\@backslashchar}\\
%   \cs{ltx@hashchar} \TO \texttt{\#}\qquad(since v1.7)\\
%   \cs{ltx@leftbracechar} \TO \texttt{\textbraceleft}\qquad(since v1.8)\\
%   \cs{ltx@rightbracechar} \TO \texttt{\textbraceright}\qquad(since v1.8)
% \end{declcs}
%
% \subsection{Boolean switch}
%
% \begin{declcs}{ltx@newif} \M{cmd}
% \end{declcs}
% \cs{ltx@newif} defines a new boolean switch \meta{cmd} like \cs{newif}.
% Unlike \plainTeX's \cs{newif}, \cs{ltx@newif} is not \cs{outer}.
% The command \meta{cmd} must start with the two characters |if|.
%
% \begin{declcs}{ltx@newglobalif} \M{cmd}
% \end{declcs}
% \cs{ltx@newglobalif} defines a new boolean switch \meta{cmd}
% like \cs{ltx@newif}. However the switch setting commands,
% \meta{cmd} without the prefix |if| and followed by |true| or |false|
% are acting globally.
%
% \subsection{Command definitions}
%
% \begin{declcs}{ltx@ifundefined} \M{cmd} \M{yes} \M{no}
% \end{declcs}
% If \eTeX\ is available, \cs{ifcsname} is used that does
% not have the side effect of defining undefined commands
% with meaning of \cs{relax}. This command is always expandable.
% Change in version 1.1: Also the meaning \cs{relax} is
% always considered ``undefined''.
%
% \begin{declcs}{ltx@IfUndefined} \M{cmd} \M{yes} \M{no}
% \end{declcs}
% If \eTeX\ is available, \cs{ifcsname} is used that does
% not have the side effect of defining undefined commands
% with meaning of \cs{relax}. Also it always checks for
% the meaning of \cs{relax} and considers this as undefined.
% This macro is not expandable without \eTeX.
%
% \begin{declcs}{ltx@LocalExpandAfter}
% \end{declcs}
% It expands the token after the next token but in a local
% context. That is the difference to \cs{expandafter}.
% The local context discards the side effect of \cs{csname}
% and let the command undefined after the expansion step.
%
% \subsection{Stripping}
%
% \begin{declcs}{ltx@RemovePrefix}\\
%   \cs{ltx@StripPrefix}
% \end{declcs}
% All tokens up to and including the next available character
% `\textgreater' are thrown away. Usually it is used to strip
% the first part of the output of the commands \cs{meaning}
% or \cs{pdflastmatch}. Macro \cs{ltx@RemovePrefix}
% has the same meaning as \LaTeX's \cs{strip@prefix}, whereas
% macro \cs{ltx@StripPrefix} expands the next token
% once before stripping the prefix.
%
% \begin{declcs}{ltx@onelevel@sanitize} \M{macro}
% \end{declcs}
% Macro \cs{ltx@onelevel@sanitize} provides \hologo{LaTeX}'s
% \cs{@onelevel@sanitize}. The macro is expanded once and
% the contents is converted to characters with catcode 12 (other)
% and space tokens with catcode 10 (space). Then then sanitized
% contents is stored into the macro again. Since version 1.12.
%
% \subsection{File management}
%
% All macros in this section are expandable like
% the counterparts of the \LaTeX\ kernel. Also they
% can be used after the preamble.
%
% \subsubsection{File extensions}
%
% \begin{declcs}{ltx@clsextension}\\
%   \cs{ltx@pkgextension}
% \end{declcs}
% Macros \cs{ltx@clsextension} and \cs{ltx@styextension}
% stores the strings |cls| and |sty|. In opposite to \hologo{LaTeX}'s
% \cs{@clsextension} and \cs{@styextension} they can also be
% used after |\begin{document}|.
%
% \subsubsection{Load check}
%
% \begin{declcs}{ltx@ifclassloaded} \M{class} \M{yes} \M{no}\\
%   \cs{ltx@ifpackageloaded} \M{package} \M{yes} \M{no}
% \end{declcs}
% Macros \cs{ltx@ifclassloaded}\slash\cs{ltx@ifpackageloaded}
% execute \meta{yes}, if the \meta{class} or \meta{package} is loaded,
% otherwise \meta{no} is called.
% Both \meta{class} and \meta{package} are specified without
% extension. The macros can also be used after |\begin{document}|.
%
% \begin{declcs}{ltx@iffileloaded} \M{file} \M{yes} \M{no}
% \end{declcs}
% If \LaTeX's \cs{ProvidesFile} macro was called before
% using \meta{file} as argument, then \cs{ltx@iffileloaded}
% calls \meta{yes}, otherwise \meta{no}.
% Therefore it is possible
% that the \meta{file} is loaded, but \meta{no} is executed because
% of a missing \cs{ProvidesFile}. The \LaTeX\ kernel does not have
% a counterpart of \cs{ltx@iffileloaded}.
%
% Note that the file name used in \cs{ProvidesFile} and
% \cs{ltx@iffileloaded} must match. For example, if \TeX's
% default extension \xext{tex} was given in the first command,
% then it must also specified in the latter command and vice versa.
%
% \subsubsection{Version date check}
%
% \begin{declcs}{ltx@ifclasslater} \M{class} \M{date} \M{yes} \M{no}\\
%   \cs{ltx@ifpackagelater} \M{package} \M{date} \M{yes} \M{no}\\
%   \cs{ltx@iffilelater} \M{file} \M{date} \M{yes} \M{no}
% \end{declcs}
% If a \cs{ProvidesClass}\slash\cs{ProvidesPackage}\slash
% \cs{ProvidesFile} command with exact the same class\slash package\slash
% file was executed before with an optional
% argument that starts with a \LaTeX\ version date, then
% this version date is compared with the argument \meta{date}.
% If they are equal or if the version date is the later date,
% then \meta{yes} is called. In all other cases \meta{no} is executed.
%
% A \LaTeX\ date has the format |YYYY/MM/DD| with |YYYY| as year
% with four digits, |MM| as month with two digits and |DD| as day
% with two digits. If \pdfTeX's \cs{pdfmatch} is available, then
% it is used to detect the version date, to reject invalid
% date formats and to reject some invalid dates.
% Dates before 1994/01/01 are always invalid, because version
% dates are introduced with \LaTeXe\ in 1994.
%
% \subsection{Macro additions}
%
% \begin{declcs}{ltx@GlobalAppendToMacro} \M{cmd} \M{addition}\\
%   \cs{ltx@LocalAppendToMacro} \M{cmd} \M{addition}
% \end{declcs}
% The \meta{addition} is appended to the parameterless macro \meta{cmd}.
% If \meta{cmd} is undefined or has the meaning \cs{relax}, then
% it will be initialized as empty macro beforehand. Due to a bug \meta{addition}
% must not contain \cs{par} before version 2010/10/25 v1.9.
%
% \begin{declcs}{ltx@GlobalPrependToMacro} \M{cmd} \M{addition}\\
%   \cs{ltx@LocalPrependToMacro} \M{cmd} \M{addition}
% \end{declcs}
% The \meta{addition} is prepended to the parameterless macro \meta{cmd}.
% If \meta{cmd} is undefined or has the meaning \cs{relax}, then
% it will be initialized as empty macro beforehand. The macros
% were added in version 2011/08/22 v1.21.
%
% \subsection{Next character detection}
%
% \begin{declcs}{ltx@ifnextchar} \M{char} \M{yes} \M{no}
% \end{declcs}
% If next character is \meta{char} then \meta{yes} is called,
% otherwise \meta{no}. The character is not removed.
% Spaces are silently removed when looking for \meta{char}
% as \hologo{LaTeX}'s version \cs{kernel@ifnextchar} does.
% But there are also small differences:
% \begin{itemize}
% \item The space can be used as \meta{char}. In this case
%   optional spaces before \meta{char} are not supported of course.
% \item If the optional space is a command that is a character
%   (defined by \cs{let} or \cs{futurelet}), then
%   \cs{kernel@ifnextchar} breaks with an \hologo{TeX} error.
%   \cs{ltx@ifnextchar} silently removes this token as
%   optional space.
% \end{itemize}
% Since 2010/03/01 v1.3.
%
% \begin{declcs}{ltx@ifnextchar@nospace} \M{char} \M{yes} \M{no}
% \end{declcs}
% Macro \cs{ltx@ifnextchar@nospace} behaves like
% macro \cs{ltx@ifnextchar} with the exception that
% optional spaces are not supported before \meta{char}.
% Since 2011/04/14 v1.19.
%
% \subsection{\cs{ltx@leavevmode}, \cs{ltx@mbox}}
%
% \begin{declcs}{ltx@leavevmode}
% \end{declcs}
% Macro \cs{ltx@leavevmode} calls \hologo{pdfTeX}'s \cs{quitvmode}.
% Otherwise \cs{leavevmode} is used and defined if it is necessary.
%
% \begin{declcs}{ltx@mbox}
% \end{declcs}
% Macro \cs{ltx@mbox} reimplements \cs{mbox} with two changes.
% Instead of \cs{leavevmode} it uses \cs{ltx@leavevmode} and
% stops right after \cs{hbox}. Especially it does not grab
% the argument and allows the extended syntax of \cs{hbox}.
%
% \subsection{Expandable test for emptiness}
%
% \begin{declcs}{ltx@ifempty} \M{stuff} \M{yes} \M{no}
% \end{declcs}
% Macro \cs{ltx@ifempty} checks in exact two expansion steps
% whether \meta{stuff}
% is empty or contains token. Depending on the result
% \meta{yes} or \meta{no} is executed. The token in \meta{stuff}
% may contain \cs{par} and unmatched conditionals (\cs{if}, \cs{else},
% \cs{fi}, \dots). Since version 2010/11/12 v1.11.
%
% \begin{declcs}{ltx@ifblank} \M{stuff} \M{yes} \M{no}
% \end{declcs}
% Macro \cs{ltx@ifblank} tests in exact two expansion steps
% if \meta{stuff} is empty or contain only blank spaces.
% In this case argument \meta{yes} is called. If \meta{stuff}
% contains other tokens than spaces then \meta{no} is executed.
% Since version 2010/12/04 v1.13.
%
% \subsection{Stripping spaces}
%
% \begin{declcs}{ltx@zapspace} \M{stuff}
% \end{declcs}
% Macro \cs{ltx@zapspace} strips spaces from \meta{stuff}
% that are not hidden inside curly braces. Like \hologo{LaTeX}'s
% \cs{zap@space} it is expandable. Differences:
% \begin{itemize}
% \item
% Syntax: \cs{zap@space} also expects a space token
% and \cs{@empty} after \meta{stuff}.
% \item
% Macro \cs{ltx@zapspace} is expandable in exact two
% expansion steps.
% \item
% Macro \cs{ltx@zapspace} always retains curly braces.
% \item
% Macro \cs{zap@space} has a bug. It stops stripping spaces
% after a token group in curly braces if the first two tokens
% inside the group are equal.
% \item
% Macro \cs{ltx@zapspace} also works with \cs{par}
% and conditionals (\cs{if}, \cs{else}, \cs{fi}, \dots).
% \end{itemize}
% Macro \cs{ltx@zapspace} is available since version
% 2010/12/07 v1.14.
%
% \subsection{Check for emptiness of boxes}
%
% \begin{declcs}{ltx@IfBoxEmpty} \M{box register number}
%   \M{yes} \M{no}
% \end{declcs}
% Macro \cs{ltx@IfBoxEmpty} calls \meta{yes} if
% the box exists (\cs{ifvoid} returns false) and the box
% does not contain any content. Otherwise if the box
% is void or contains something, then \meta{no} is executed.
% Thus being empty means that the box exists and is either
% an \cs{hbox} or a \cs{vbox} and may even have dimensions other than
% 0.0\,pt, but the box does not contain anything.
% Macro \cs{ltx@IfBoxEmpty} is available since 2010/02/04 v1.16.
%
% \begin{declcs}{ltx@IfBoxVoidOrEmpty} \M{box register number}
%   \M{yes} \M{no}
% \end{declcs}
% Macro \cs{ltx@IfBoxVoidOrEmpty} calls \meta{yes} if
% the box is either void or does not contain any content.
% Otherwise \meta{no} is executed.
% Macro \cs{ltx@IfBoxVoidOrEmpty} is available since 2010/02/04 v1.16.
%
% \StopEventually{
% }
%
% \section{Implementation}
%
% \subsection{Identification}
%
%    \begin{macrocode}
%<*package>
%    \end{macrocode}
%    Reload check, especially if the package is not used with \LaTeX.
%    \begin{macrocode}
\begingroup\catcode61\catcode48\catcode32=10\relax%
  \catcode13=5 % ^^M
  \endlinechar=13 %
  \catcode35=6 % #
  \catcode39=12 % '
  \catcode44=12 % ,
  \catcode45=12 % -
  \catcode46=12 % .
  \catcode58=12 % :
  \catcode64=11 % @
  \catcode123=1 % {
  \catcode125=2 % }
  \expandafter\let\expandafter\x\csname ver@ltxcmds.sty\endcsname
  \ifx\x\relax % plain-TeX, first loading
  \else
    \def\empty{}%
    \ifx\x\empty % LaTeX, first loading,
      % variable is initialized, but \ProvidesPackage not yet seen
    \else
      \expandafter\ifx\csname PackageInfo\endcsname\relax
        \def\x#1#2{%
          \immediate\write-1{Package #1 Info: #2.}%
        }%
      \else
        \def\x#1#2{\PackageInfo{#1}{#2, stopped}}%
      \fi
      \x{ltxcmds}{The package is already loaded}%
      \aftergroup\endinput
    \fi
  \fi
\endgroup%
%    \end{macrocode}
%    Package identification:
%    \begin{macrocode}
\begingroup\catcode61\catcode48\catcode32=10\relax%
  \catcode13=5 % ^^M
  \endlinechar=13 %
  \catcode35=6 % #
  \catcode39=12 % '
  \catcode40=12 % (
  \catcode41=12 % )
  \catcode44=12 % ,
  \catcode45=12 % -
  \catcode46=12 % .
  \catcode47=12 % /
  \catcode58=12 % :
  \catcode64=11 % @
  \catcode91=12 % [
  \catcode93=12 % ]
  \catcode123=1 % {
  \catcode125=2 % }
  \expandafter\ifx\csname ProvidesPackage\endcsname\relax
    \def\x#1#2#3[#4]{\endgroup
      \immediate\write-1{Package: #3 #4}%
      \xdef#1{#4}%
    }%
  \else
    \def\x#1#2[#3]{\endgroup
      #2[{#3}]%
      \ifx#1\@undefined
        \xdef#1{#3}%
      \fi
      \ifx#1\relax
        \xdef#1{#3}%
      \fi
    }%
  \fi
\expandafter\x\csname ver@ltxcmds.sty\endcsname
\ProvidesPackage{ltxcmds}%
  [2019/12/15 v1.24 LaTeX kernel commands for general use (HO)]%
%    \end{macrocode}
%
%    \begin{macrocode}
\begingroup\catcode61\catcode48\catcode32=10\relax%
  \catcode13=5 % ^^M
  \endlinechar=13 %
  \catcode123=1 % {
  \catcode125=2 % }
  \catcode64=11 % @
  \def\x{\endgroup
    \expandafter\edef\csname LTXcmds@AtEnd\endcsname{%
      \endlinechar=\the\endlinechar\relax
      \catcode13=\the\catcode13\relax
      \catcode32=\the\catcode32\relax
      \catcode35=\the\catcode35\relax
      \catcode61=\the\catcode61\relax
      \catcode64=\the\catcode64\relax
      \catcode123=\the\catcode123\relax
      \catcode125=\the\catcode125\relax
    }%
  }%
\x\catcode61\catcode48\catcode32=10\relax%
\catcode13=5 % ^^M
\endlinechar=13 %
\catcode35=6 % #
\catcode64=11 % @
\catcode123=1 % {
\catcode125=2 % }
\def\TMP@EnsureCode#1#2{%
  \edef\LTXcmds@AtEnd{%
    \LTXcmds@AtEnd
    \catcode#1=\the\catcode#1\relax
  }%
  \catcode#1=#2\relax
}
\TMP@EnsureCode{36}{3}% $
\TMP@EnsureCode{38}{4}% &
\TMP@EnsureCode{40}{12}% (
\TMP@EnsureCode{41}{12}% )
\TMP@EnsureCode{45}{12}% -
\TMP@EnsureCode{46}{12}% .
\TMP@EnsureCode{47}{12}% /
\TMP@EnsureCode{60}{12}% <
\TMP@EnsureCode{62}{12}% >
\TMP@EnsureCode{91}{12}% [
\TMP@EnsureCode{96}{12}% `
\TMP@EnsureCode{93}{12}% ]
\TMP@EnsureCode{94}{12}% ^ (superscript) (!)
\TMP@EnsureCode{124}{12}% |
\edef\LTXcmds@AtEnd{\LTXcmds@AtEnd\noexpand\endinput}
%    \end{macrocode}
%
% \subsection{Numbers}
%
%    \begin{macro}{\ltx@zero}
%    \begin{macrocode}
\chardef\ltx@zero=0 %
%    \end{macrocode}
%    \end{macro}
%    \begin{macro}{\ltx@one}
%    \begin{macrocode}
\chardef\ltx@one=1 %
%    \end{macrocode}
%    \end{macro}
%    \begin{macro}{\ltx@two}
%    \begin{macrocode}
\chardef\ltx@two=2 %
%    \end{macrocode}
%    \end{macro}
%    \begin{macro}{\ltx@active}
%    \begin{macrocode}
\chardef\ltx@active=13 %
%    \end{macrocode}
%    \end{macro}
%    \begin{macro}{\ltx@cclv}
%    \begin{macrocode}
\chardef\ltx@cclv=255 %
%    \end{macrocode}
%    \end{macro}
%
%    \begin{macro}{\ltx@minusone}
%    \begin{macrocode}
\def\ltx@minusone{%
  -\ltx@one
}
%    \end{macrocode}
%    \end{macro}
%
% \subsection{Scratch registers}
%
%    \begin{macro}{\ltx@LocToksA}
%    \begin{macrocode}
\toksdef\ltx@LocToksA=0 %
%    \end{macrocode}
%    \end{macro}
%    \begin{macro}{\ltx@LocToksB}
%    \begin{macrocode}
\toksdef\ltx@LocToksB=2 %
%    \end{macrocode}
%    \end{macro}
%    \begin{macro}{\ltx@LocToksC}
%    \begin{macrocode}
\toksdef\ltx@LocToksC=4 %
%    \end{macrocode}
%    \end{macro}
%    \begin{macro}{\ltx@LocToksD}
%    \begin{macrocode}
\toksdef\ltx@LocToksD=6 %
%    \end{macrocode}
%    \end{macro}
%    \begin{macro}{\ltx@LocToksE}
%    \begin{macrocode}
\toksdef\ltx@LocToksE=8 %
%    \end{macrocode}
%    \end{macro}
%
%    \begin{macro}{\ltx@GlobToksA}
%    \begin{macrocode}
\toksdef\ltx@GlobToksA=1 %
%    \end{macrocode}
%    \end{macro}
%    \begin{macro}{\ltx@GlobToksB}
%    \begin{macrocode}
\toksdef\ltx@GlobToksB=3 %
%    \end{macrocode}
%    \end{macro}
%    \begin{macro}{\ltx@GlobToksC}
%    \begin{macrocode}
\toksdef\ltx@GlobToksC=5 %
%    \end{macrocode}
%    \end{macro}
%    \begin{macro}{\ltx@GlobToksD}
%    \begin{macrocode}
\toksdef\ltx@GlobToksD=7 %
%    \end{macrocode}
%    \end{macro}
%    \begin{macro}{\ltx@GlobToksE}
%    \begin{macrocode}
\toksdef\ltx@GlobToksE=9 %
%    \end{macrocode}
%    \end{macro}
%
%    \begin{macro}{\ltx@LocDimenA}
%    \begin{macrocode}
\dimendef\ltx@LocDimenA=0 %
%    \end{macrocode}
%    \end{macro}
%    \begin{macro}{\ltx@LocDimenB}
%    \begin{macrocode}
\dimendef\ltx@LocDimenB=2 %
%    \end{macrocode}
%    \end{macro}
%    \begin{macro}{\ltx@LocDimenC}
%    \begin{macrocode}
\dimendef\ltx@LocDimenC=4 %
%    \end{macrocode}
%    \end{macro}
%    \begin{macro}{\ltx@LocDimenD}
%    \begin{macrocode}
\dimendef\ltx@LocDimenD=6 %
%    \end{macrocode}
%    \end{macro}
%    \begin{macro}{\ltx@LocDimenE}
%    \begin{macrocode}
\dimendef\ltx@LocDimenE=8 %
%    \end{macrocode}
%    \end{macro}
%
%    \begin{macro}{\ltx@GlobDimenA}
%    \begin{macrocode}
\dimendef\ltx@GlobDimenA=1 %
%    \end{macrocode}
%    \end{macro}
%    \begin{macro}{\ltx@GlobDimenB}
%    \begin{macrocode}
\dimendef\ltx@GlobDimenB=3 %
%    \end{macrocode}
%    \end{macro}
%    \begin{macro}{\ltx@GlobDimenC}
%    \begin{macrocode}
\dimendef\ltx@GlobDimenC=5 %
%    \end{macrocode}
%    \end{macro}
%    \begin{macro}{\ltx@GlobDimenD}
%    \begin{macrocode}
\dimendef\ltx@GlobDimenD=7 %
%    \end{macrocode}
%    \end{macro}
%    \begin{macro}{\ltx@GlobDimenE}
%    \begin{macrocode}
\dimendef\ltx@GlobDimenE=9 %
%    \end{macrocode}
%    \end{macro}
%
%    \begin{macro}{\ltx@LocSkipA}
%    \begin{macrocode}
\skipdef\ltx@LocSkipA=0 %
%    \end{macrocode}
%    \end{macro}
%    \begin{macro}{\ltx@LocSkipB}
%    \begin{macrocode}
\skipdef\ltx@LocSkipB=2 %
%    \end{macrocode}
%    \end{macro}
%    \begin{macro}{\ltx@LocSkipC}
%    \begin{macrocode}
\skipdef\ltx@LocSkipC=4 %
%    \end{macrocode}
%    \end{macro}
%    \begin{macro}{\ltx@LocSkipD}
%    \begin{macrocode}
\skipdef\ltx@LocSkipD=6 %
%    \end{macrocode}
%    \end{macro}
%    \begin{macro}{\ltx@LocSkipE}
%    \begin{macrocode}
\skipdef\ltx@LocSkipE=8 %
%    \end{macrocode}
%    \end{macro}
%
%    \begin{macro}{\ltx@GlobSkipA}
%    \begin{macrocode}
\skipdef\ltx@GlobSkipA=1 %
%    \end{macrocode}
%    \end{macro}
%    \begin{macro}{\ltx@GlobSkipB}
%    \begin{macrocode}
\skipdef\ltx@GlobSkipB=3 %
%    \end{macrocode}
%    \end{macro}
%    \begin{macro}{\ltx@GlobSkipC}
%    \begin{macrocode}
\skipdef\ltx@GlobSkipC=5 %
%    \end{macrocode}
%    \end{macro}
%    \begin{macro}{\ltx@GlobSkipD}
%    \begin{macrocode}
\skipdef\ltx@GlobSkipD=7 %
%    \end{macrocode}
%    \end{macro}
%    \begin{macro}{\ltx@GlobSkipE}
%    \begin{macrocode}
\skipdef\ltx@GlobSkipE=9 %
%    \end{macrocode}
%    \end{macro}
%
%
% \subsection{Argument killers}
%
%    \begin{macro}{\ltx@gobble}
%    \begin{macrocode}
\long\def\ltx@gobble#1{}
%    \end{macrocode}
%    \end{macro}
%    \begin{macro}{\ltx@gobbletwo}
%    \begin{macrocode}
\long\def\ltx@gobbletwo#1#2{}
%    \end{macrocode}
%    \end{macro}
%    \begin{macro}{\ltx@gobblethree}
%    \begin{macrocode}
\long\def\ltx@gobblethree#1#2#3{}
%    \end{macrocode}
%    \end{macro}
%    \begin{macro}{\ltx@gobblefour}
%    \begin{macrocode}
\long\def\ltx@gobblefour#1#2#3#4{}
%    \end{macrocode}
%    \end{macro}
%
%    \begin{macro}{\ltx@GobbleNum}
%    \begin{macrocode}
\def\ltx@GobbleNum#1{%
  \romannumeral
  \csname ltx@zero%
  \expandafter\LTXcmds@GobbleNum
  \romannumeral\LTXcmds@num{#1}000{m\endcsname}%
}
%    \end{macrocode}
%    \end{macro}
%    \begin{macro}{\LTXcmds@GobbleNum}
%    \begin{macrocode}
\def\LTXcmds@GobbleNum#1{%
  \csname LTXcmds@G#1\LTXcmds@GobbleNum
}
%    \end{macrocode}
%    \end{macro}
%    \begin{macro}{\LTXcmds@Gm}
%    \begin{macrocode}
\long\def\LTXcmds@Gm#1{%
  \endcsname
}
%    \end{macrocode}
%    \end{macro}
%
% \subsection{Argument grabbers}
%
%    \begin{macro}{\ltx@firstofone}
%    \begin{macrocode}
\long\def\ltx@firstofone#1{#1}
%    \end{macrocode}
%    \end{macro}
%    \begin{macro}{\ltx@firstoftwo}
%    \begin{macrocode}
\long\def\ltx@firstoftwo#1#2{#1}
%    \end{macrocode}
%    \end{macro}
%    \begin{macro}{\ltx@secondoftwo}
%    \begin{macrocode}
\long\def\ltx@secondoftwo#1#2{#2}
%    \end{macrocode}
%    \end{macro}
%
%    \begin{macro}{\ltx@firstofthree}
%    \begin{macrocode}
\long\def\ltx@firstofthree#1#2#3{#1}
%    \end{macrocode}
%    \end{macro}
%    \begin{macro}{\ltx@secondofthree}
%    \begin{macrocode}
\long\def\ltx@secondofthree#1#2#3{#2}
%    \end{macrocode}
%    \end{macro}
%    \begin{macro}{\ltx@thirdofthree}
%    \begin{macrocode}
\long\def\ltx@thirdofthree#1#2#3{#3}%
%    \end{macrocode}
%    \end{macro}
%
%    \begin{macro}{\ltx@firstoffour}
%    \begin{macrocode}
\long\def\ltx@firstoffour#1#2#3#4{#1}
%    \end{macrocode}
%    \end{macro}
%    \begin{macro}{\ltx@secondoffour}
%    \begin{macrocode}
\long\def\ltx@secondoffour#1#2#3#4{#2}
%    \end{macrocode}
%    \end{macro}
%    \begin{macro}{\ltx@thirdoffour}
%    \begin{macrocode}
\long\def\ltx@thirdoffour#1#2#3#4{#3}%
%    \end{macrocode}
%    \end{macro}
%    \begin{macro}{\ltx@fourthoffour}
%    \begin{macrocode}
\long\def\ltx@fourthoffour#1#2#3#4{#4}%
%    \end{macrocode}
%    \end{macro}
%
% \subsection{List helpers}
%
%    \begin{macro}{\ltx@carzero}
%    \begin{macrocode}
\long\def\ltx@carzero#1\@nil{}%
%    \end{macrocode}
%    \end{macro}
%    \begin{macro}{\LTXcmds@cdrzero}
%    \begin{macrocode}
\long\def\LTXcmds@cdrzero#1\@nil{#1}
%    \end{macrocode}
%    \end{macro}
%    \begin{macro}{\ltx@cdrzero}
%    \begin{macrocode}
\def\ltx@cdrzero{%
  \romannumeral\LTXcmds@cdrzero\ltx@zero
}
%    \end{macrocode}
%    \end{macro}
%
%    \begin{macro}{\ltx@car}
%    \begin{macrocode}
\long\def\ltx@car#1#2\@nil{#1}
%    \end{macrocode}
%    \end{macro}
%    \begin{macro}{\ltx@cdr}
%    \begin{macrocode}
\long\def\ltx@cdr#1{%
  \romannumeral\LTXcmds@cdrzero\ltx@zero
}
%    \end{macrocode}
%    \end{macro}
%
%    \begin{macro}{\ltx@cartwo}
%    \begin{macrocode}
\long\def\ltx@cartwo#1#2#3\@nil{#1#2}
%    \end{macrocode}
%    \end{macro}
%    \begin{macro}{\ltx@carsecond}
%    \begin{macrocode}
\long\def\ltx@carsecond#1#2#3\@nil{#2}
%    \end{macrocode}
%    \end{macro}
%    \begin{macro}{\ltx@cdrtwo}
%    \begin{macrocode}
\long\def\ltx@cdrtwo#1#2{%
  \romannumeral\LTXcmds@cdrzero\ltx@zero
}
%    \end{macrocode}
%    \end{macro}
%
%    \begin{macro}{\ltx@carthree}
%    \begin{macrocode}
\long\def\ltx@carthree#1#2#3#4\@nil{#1#2#3}
%    \end{macrocode}
%    \end{macro}
%    \begin{macro}{\ltx@carthird}
%    \begin{macrocode}
\long\def\ltx@carthird#1#2#3#4\@nil{#3}
%    \end{macrocode}
%    \end{macro}
%    \begin{macro}{\ltx@cdrthree}
%    \begin{macrocode}
\long\def\ltx@cdrthree#1#2#3{%
  \romannumeral\LTXcmds@cdrzero\ltx@zero
}
%    \end{macrocode}
%    \end{macro}
%
%    \begin{macro}{\ltx@carfour}
%    \begin{macrocode}
\long\def\ltx@carfour#1#2#3#4#5\@nil{#1#2#3#4}
%    \end{macrocode}
%    \end{macro}
%    \begin{macro}{\ltx@carfourth}
%    \begin{macrocode}
\long\def\ltx@carfourth#1#2#3#4#5\@nil{#4}
%    \end{macrocode}
%    \end{macro}
%    \begin{macro}{\ltx@cdrfour}
%    \begin{macrocode}
\long\def\ltx@cdrfour#1#2#3#4{%
  \romannumeral\LTXcmds@cdrzero\ltx@zero
}
%    \end{macrocode}
%    \end{macro}
%
%    \begin{macro}{\ltx@CarNum}
%    \begin{macrocode}
\def\ltx@CarNum#1{%
  \romannumeral
  \csname LTXcmds@CarNumFinish%
  \expandafter\LTXcmds@CarNum
  \romannumeral\LTXcmds@num{#1}000{x\endcsname}%
}
%    \end{macrocode}
%    \end{macro}
%    \begin{macro}{\LTXcmds@CarNum}
%    \begin{macrocode}
\def\LTXcmds@CarNum#1{%
  \csname LTXcmds@C#1\LTXcmds@CarNum
}
%    \end{macrocode}
%    \end{macro}
%    \begin{macro}{\LTXcmds@Cm}
%    \begin{macrocode}
\long\def\LTXcmds@Cm#1#2{%
  \endcsname{#1#2}%
}
%    \end{macrocode}
%    \end{macro}
%    \begin{macro}{\LTXcmds@Cx}
%    \begin{macrocode}
\def\LTXcmds@Cx#1{%
  \endcsname{}%
}
%    \end{macrocode}
%    \end{macro}
%    \begin{macro}{\LTXcmds@CarNumFinish}
%    \begin{macrocode}
\long\def\LTXcmds@CarNumFinish#1#2\@nil{%
  \ltx@zero
  #1%
}
%    \end{macrocode}
%    \end{macro}
%
%    \begin{macro}{\ltx@CarNumth}
%    \begin{macrocode}
\def\ltx@CarNumth#1{%
  \romannumeral
  \expandafter\expandafter\expandafter
  \LTXcmds@CarNumth
  \ltx@GobbleNum{#1}{}%
}
%    \end{macrocode}
%    \end{macro}
%    \begin{macro}{\LTXcmds@CarNumth}
%    \begin{macrocode}
\long\def\LTXcmds@CarNumth#1#2\@nil{%
  \ltx@zero
  #1%
}
%    \end{macrocode}
%    \end{macro}
%
%    \begin{macro}{\ltx@CdrNum}
%    \begin{macrocode}
\def\ltx@CdrNum#1{%
  \romannumeral%
  \expandafter\expandafter\expandafter\ltx@cdrzero
  \expandafter\expandafter\expandafter\ltx@zero
  \ltx@GobbleNum{#1}%
}
%    \end{macrocode}
%    \end{macro}
%
% \subsection{Tail recursion}
%
%    \begin{macro}{\ltx@ReturnAfterFi}
%    \begin{macrocode}
\long\def\ltx@ReturnAfterFi#1\fi{\fi#1}
%    \end{macrocode}
%    \end{macro}
%    \begin{macro}{\ltx@ReturnAfterElseFi}
%    \begin{macrocode}
\long\def\ltx@ReturnAfterElseFi#1\else#2\fi{\fi#1}
%    \end{macrocode}
%    \end{macro}
%
% \subsection{Empty macro}
%
%    \begin{macro}{\ltx@empty}
%    \begin{macrocode}
\def\ltx@empty{}
%    \end{macrocode}
%    \end{macro}
%
% \subsection{Characters}
%
%    \begin{macro}{\ltx@space}
%    \begin{macrocode}
\def\ltx@space{ }
%    \end{macrocode}
%    \end{macro}
%    \begin{macro}{\ltx@percentchar}
%    \begin{macrocode}
\begingroup
  \lccode`0=`\%\relax
\lowercase{\endgroup
  \def\ltx@percentchar{0}%
}
%    \end{macrocode}
%    \end{macro}
%    \begin{macro}{\ltx@backslashchar}
%    \begin{macrocode}
\begingroup
  \lccode`0=`\\\relax
\lowercase{\endgroup
  \def\ltx@backslashchar{0}%
}
%    \end{macrocode}
%    \end{macro}
%    \begin{macro}{\ltx@hashchar}
%    \begin{macrocode}
\begingroup
  \lccode`0=`\#\relax
\lowercase{\endgroup
  \def\ltx@hashchar{0}%
}
%    \end{macrocode}
%    \end{macro}
%    \begin{macro}{\ltx@leftbracechar}
%    \begin{macrocode}
\begingroup
  \lccode`0=`\{\relax
\lowercase{\endgroup
  \def\ltx@leftbracechar{0}%
}
%    \end{macrocode}
%    \end{macro}
%    \begin{macro}{\ltx@rightbracechar}
%    \begin{macrocode}
\begingroup
  \lccode`0=`\}\relax
\lowercase{\endgroup
  \def\ltx@rightbracechar{0}%
}
%    \end{macrocode}
%    \end{macro}
%
% \subsection{Boolean switch}
%
%    \begin{macro}{\ltx@newif}
%    \begin{macrocode}
\def\ltx@newif#1{%
  \begingroup
    \escapechar=-1 %
  \expandafter\endgroup
  \expandafter\LTXcmds@newif\string#1\@nil
}
%    \end{macrocode}
%    \end{macro}
%    \begin{macro}{\LTXcmds@newif}
%    \begin{macrocode}
\begingroup
  \escapechar=-1 %
\expandafter\endgroup
\expandafter\def\expandafter\LTXcmds@newif\string\if#1\@nil{%
  \expandafter\edef\csname#1true\endcsname{%
    \let
    \expandafter\noexpand\csname if#1\endcsname
    \noexpand\iftrue
  }%
  \expandafter\edef\csname#1false\endcsname{%
    \let
    \expandafter\noexpand\csname if#1\endcsname
    \noexpand\iffalse
  }%
  \csname#1false\endcsname
}
%    \end{macrocode}
%    \end{macro}
%
%    \begin{macro}{\ltx@newglobalif}
%    \begin{macrocode}
\def\ltx@newglobalif#1{%
  \begingroup
    \escapechar=-1 %
  \expandafter\endgroup
  \expandafter\LTXcmds@newglobalif\string#1\@nil
}
%    \end{macrocode}
%    \end{macro}
%    \begin{macro}{\LTXcmds@newglobalif}
%    \begin{macrocode}
\begingroup
  \escapechar=-1 %
\expandafter\endgroup
\expandafter
\def\expandafter\LTXcmds@newglobalif\string\if#1\@nil{%
  \expandafter\edef\csname#1true\endcsname{%
    \global\let
    \expandafter\noexpand\csname if#1\endcsname
    \noexpand\iftrue
  }%
  \expandafter\edef\csname#1false\endcsname{%
    \global\let
    \expandafter\noexpand\csname if#1\endcsname
    \noexpand\iffalse
  }%
  \csname#1false\endcsname
}
%    \end{macrocode}
%    \end{macro}
%
% \subsection{Command definitions}
%
%    \begin{macro}{\ltx@LocalExpandAfter}
%    \begin{macrocode}
\def\ltx@LocalExpandAfter{%
  \begingroup
    \expandafter\expandafter\expandafter
  \endgroup
  \expandafter
}
%    \end{macrocode}
%    \end{macro}
%
%    \begin{macrocode}
\ltx@LocalExpandAfter
\ifx\csname ifcsname\endcsname\relax
%    \end{macrocode}
%    \begin{macro}{\ltx@ifundefined}
%    \begin{macrocode}
  \def\ltx@ifundefined#1{%
    \expandafter\ifx\csname #1\endcsname\relax
      \expandafter\ltx@firstoftwo
    \else
      \expandafter\ltx@secondoftwo
    \fi
  }%
%    \end{macrocode}
%    \end{macro}
%    \begin{macro}{\ltx@IfUndefined}
%    \begin{macrocode}
  \def\ltx@IfUndefined#1{%
    \begingroup\expandafter\expandafter\expandafter\endgroup
    \expandafter\ifx\csname #1\endcsname\relax
      \expandafter\ltx@firstoftwo
    \else
      \expandafter\ltx@secondoftwo
    \fi
  }%
%    \end{macrocode}
%    \end{macro}
%    \begin{macrocode}
  \expandafter\ltx@gobble
\else
  \expandafter\ltx@firstofone
\fi
{%
%    \end{macrocode}
%    \begin{macro}{\ltx@ifundefined}
%    \begin{macrocode}
  \def\ltx@ifundefined#1{%
    \ifcsname #1\endcsname
      \expandafter\ifx\csname #1\endcsname\relax
        \expandafter\expandafter\expandafter\ltx@firstoftwo
      \else
        \expandafter\expandafter\expandafter\ltx@secondoftwo
      \fi
    \else
      \expandafter\ltx@firstoftwo
    \fi
  }%
%    \end{macrocode}
%    \end{macro}
%    \begin{macro}{\ltx@IfUndefined}
%    \begin{macrocode}
  \let\ltx@IfUndefined\ltx@ifundefined
%    \end{macrocode}
%    \end{macro}
%    \begin{macrocode}
}
%    \end{macrocode}
%
% \subsection{Stripping}
%
%    \begin{macro}{\ltx@RemovePrefix}
%    \begin{macrocode}
\def\ltx@RemovePrefix#1>{}
%    \end{macrocode}
%    \end{macro}
%
%    \begin{macro}{\ltx@StripPrefix}
%    \begin{macrocode}
\def\ltx@StripPrefix{%
  \expandafter\ltx@RemovePrefix
}
%    \end{macrocode}
%    \end{macro}
%
%    \begin{macro}{\ltx@onelevel@sanitize}
%    \begin{macrocode}
\def\ltx@onelevel@sanitize#1{%
  \edef#1{%
    \expandafter
    \ltx@RemovePrefix\meaning#1%
  }%
}
%    \end{macrocode}
%    \end{macro}
%
% \subsection{File management}
%
% \subsubsection{File extensions}
%
%    \begin{macro}{\ltx@clsextension}
%    \begin{macrocode}
\def\ltx@clsextension{cls}
%    \end{macrocode}
%    \end{macro}
%    \begin{macro}{\ltx@pkgextension}
%    \begin{macrocode}
\def\ltx@pkgextension{sty}
%    \end{macrocode}
%    \end{macro}
%
% \subsubsection{Load check}
%
%    \begin{macro}{\ltx@iffileloaded}
%    \begin{macrocode}
\def\ltx@iffileloaded#1{%
  \ltx@ifundefined{ver@#1}\ltx@secondoftwo\ltx@firstoftwo
}
%    \end{macrocode}
%    \end{macro}
%    \begin{macro}{\ltx@ifclassloaded}
%    \begin{macrocode}
\def\ltx@ifclassloaded#1{%
  \ltx@iffileloaded{#1.\ltx@clsextension}%
}
%    \end{macrocode}
%    \end{macro}
%    \begin{macro}{\ltx@ifpackageloaded}
%    \begin{macrocode}
\def\ltx@ifpackageloaded#1{%
  \ltx@iffileloaded{#1.\ltx@pkgextension}%
}
%    \end{macrocode}
%    \end{macro}
%
% \subsubsection{Version date check}
%
%    \begin{macro}{\ltx@iffilelater}
%    \begin{macrocode}
\def\ltx@iffilelater#1#2{%
  \ltx@iffileloaded{#1}{%
    \expandafter\LTXcmds@IfLater\expandafter{%
      \number
      \expandafter\expandafter\expandafter\LTXcmds@ParseVersion
      \expandafter\expandafter\expandafter{%
        \csname ver@#1\endcsname
      }%
    \expandafter}\expandafter{%
      \number
      \expandafter\LTXcmds@ParseVersion\expandafter{#2}%
    }%
  }\ltx@secondoftwo
}
%    \end{macrocode}
%    \end{macro}
%    \begin{macro}{\LTXcmds@IfLater}
%    \begin{macrocode}
\def\LTXcmds@IfLater#1#2{%
  \ifcase 0%
      \ifnum#1<19940101 %
      \else
        \ifnum#2<19940101 %
        \else
          \ifnum#2>#1 %
          \else
            1%
          \fi
        \fi
      \fi
      \ltx@space
    \expandafter\ltx@secondoftwo
  \else
    \expandafter\ltx@firstoftwo
  \fi
}
%    \end{macrocode}
%    \end{macro}
%    \begin{macro}{\ltx@ifclasslater}
%    \begin{macrocode}
\def\ltx@ifclasslater#1{%
  \ltx@iffilelater{#1.\ltx@clsextension}%
}
%    \end{macrocode}
%    \end{macro}
%    \begin{macro}{\ltx@ifpackagelater}
%    \begin{macrocode}
\def\ltx@ifpackagelater#1{%
  \ltx@iffilelater{#1.\ltx@pkgextension}%
}
%    \end{macrocode}
%    \end{macro}
%
%    \begin{macrocode}
\ltx@IfUndefined{pdfmatch}{%
%    \end{macrocode}
%    \begin{macro}{\LTXcmds@ParseVersion}
%    \begin{macrocode}
  \def\LTXcmds@ParseVersion#1{%
    \LTXcmds@@ParseVersion#10000/00/00\@nil
  }%
%    \end{macrocode}
%    \end{macro}
%    \begin{macro}{\LTXcmds@@ParseVersion}
%    \begin{macrocode}
  \def\LTXcmds@@ParseVersion#1#2#3#4/#5#6/#7#8#9\@nil{%
    #1#2#3#4#5#6#7#8%
  }%
%    \end{macrocode}
%    \end{macro}
%    \begin{macrocode}
}{%
%    \end{macrocode}
%    \begin{macro}{\LTXcmds@ParseVersion}
%    \begin{macrocode}
  \def\LTXcmds@ParseVersion#1{%
    \ifnum\pdfmatch{%
      ^%
      (199[4-9]|[2-9][0-9][0-9][0-9])/%
      (0[1-9]|1[0-2])/%
      (0[1-9]|[1-2][0-9]|3[0-1])%
    }{#1}=1 %
      \ltx@StripPrefix\pdflastmatch1 %
      \ltx@StripPrefix\pdflastmatch2 %
      \ltx@StripPrefix\pdflastmatch3 %
    \else
      0%
    \fi
  }%
%    \end{macrocode}
%    \end{macro}
%    \begin{macrocode}
}
%    \end{macrocode}
%
% \subsection{Macro additions}
%
%    \begin{macro}{\ltx@GlobalAppendToMacro}
%    \begin{macrocode}
\long\def\ltx@GlobalAppendToMacro#1#2{%
  \ifx\ltx@undefined#1%
    \let#1\ltx@empty
  \else
    \ifx\relax#1%
      \let#1\ltx@empty
    \fi
  \fi
  \begingroup
    \ltx@LocToksA\expandafter{#1#2}%
    \xdef#1{\the\ltx@LocToksA}%
  \endgroup
}
%    \end{macrocode}
%    \end{macro}
%    \begin{macro}{\ltx@LocalAppendToMacro}
%    \begin{macrocode}
\long\def\ltx@LocalAppendToMacro#1#2{%
  \global\let\LTXcmds@gtemp#1%
  \ifx\ltx@undefined\LTXcmds@gtemp
    \global\let\LTXcmds@gtemp\ltx@empty
  \else
    \ifx\relax\LTXcmds@gtemp
      \global\letLTXcmds@gtemp\ltx@empty
    \fi
  \fi
  \begingroup
    \ltx@LocToksA\expandafter{\LTXcmds@gtemp#2}%
    \xdef\LTXcmds@gtemp{\the\ltx@LocToksA}%
  \endgroup
  \let#1\LTXcmds@gtemp
}
%    \end{macrocode}
%    \end{macro}
%
%    \begin{macro}{\ltx@GlobalPrependToMacro}
%    \begin{macrocode}
\long\def\ltx@GlobalPrependToMacro#1#2{%
  \ifx\ltx@undefined#1%
    \let#1\ltx@empty
  \else
    \ifx\relax#1%
      \let#1\ltx@empty
    \fi
  \fi
  \begingroup
    \ltx@LocToksA{#2}%
    \ltx@LocToksB\expandafter{#1}%
    \xdef#1{\the\ltx@LocToksA\the\ltx@LocToksB}%
  \endgroup
}
%    \end{macrocode}
%    \end{macro}
%    \begin{macro}{\ltx@LocalPrependToMacro}
%    \begin{macrocode}
\long\def\ltx@LocalPrependToMacro#1#2{%
  \global\let\LTXcmds@gtemp#1%
  \ifx\ltx@undefined\LTXcmds@gtemp
    \global\let\LTXcmds@gtemp\ltx@empty
  \else
    \ifx\relax\LTXcmds@gtemp
      \global\letLTXcmds@gtemp\ltx@empty
    \fi
  \fi
  \begingroup
    \ltx@LocToksA{#2}%
    \ltx@LocToksB\expandafter{\LTXcmds@gtemp}%
    \xdef\LTXcmds@gtemp{\the\ltx@LocToksA\the\ltx@LocToksB}%
  \endgroup
  \let#1\LTXcmds@gtemp
}
%    \end{macrocode}
%    \end{macro}
%
% \subsection{Next character detection}
%
%    \begin{macro}{\ltx@ifnextchar}
%    \begin{macrocode}
\long\def\ltx@ifnextchar#1#2#3{%
  \begingroup
  \let\LTXcmds@CharToken= #1\relax
  \ltx@LocToksA{\endgroup#2}%
  \ltx@LocToksB{\endgroup#3}%
  \futurelet\LTXcmds@LetToken\LTXcmds@ifnextchar
}
%    \end{macrocode}
%    \end{macro}
%    \begin{macro}{\LTXcmds@ifnextchar}
%    \begin{macrocode}
\def\LTXcmds@ifnextchar{%
  \ifx\LTXcmds@LetToken\LTXcmds@CharToken
    \the\expandafter\ltx@LocToksA
  \else
    \expandafter
      \ifx\csname LTXcmds@LetToken\endcsname\LTXcmds@SpaceToken
      \expandafter\expandafter\expandafter\LTXcmds@@ifnextchar
    \else
      \the\expandafter\expandafter\expandafter\ltx@LocToksB
    \fi
  \fi
}
%    \end{macrocode}
%    \end{macro}
%    \begin{macro}{\LTXcmds@@ifnextchar}
%    \cs{futurelet} does not distinguish between a
%    character and a command that is a character
%    (defined by using \cs{let} or \cs{futurelet}).
%    Therefore the space is catched by \cs{romannumeral}
%    with negative character constant that gobbles one
%    optional space.
%    \begin{macrocode}
\def\LTXcmds@@ifnextchar{%
  \expandafter\futurelet
  \expandafter\LTXcmds@LetToken
  \expandafter\LTXcmds@ifnextchar
  \romannumeral-`\.%
}
%    \end{macrocode}
%    \end{macro}
%    \begin{macro}{\LTXcmds@SpaceToken}
%    \begin{macrocode}
\ltx@firstofone{\let\LTXcmds@SpaceToken= } %
%    \end{macrocode}
%    \end{macro}
%
%    \begin{macro}{\ltx@ifnextchar@nospace}
%    \begin{macrocode}
\long\def\ltx@ifnextchar@nospace#1#2#3{%
  \begingroup
  \let\LTXcmds@CharToken= #1\relax
  \ltx@LocToksA{\endgroup#2}%
  \ltx@LocToksB{\endgroup#3}%
  \futurelet\LTXcmds@LetToken\LTXcmds@ifnextchar@nospace
}
%    \end{macrocode}
%    \end{macro}
%    \begin{macro}{\LTXcmds@ifnextchar@nospace}
%    \begin{macrocode}
\def\LTXcmds@ifnextchar@nospace{%
  \the
  \ifx\LTXcmds@LetToken\LTXcmds@CharToken
    \expandafter\ltx@LocToksA
  \else
    \expandafter\ltx@LocToksB
  \fi
}
%    \end{macrocode}
%    \end{macro}
% \subsection{\cs{ltx@leavevmode}, \cs{ltx@mbox}}
%
%    \begin{macro}{\ltx@leavevmode}
%    \begin{macrocode}
\ltx@IfUndefined{quitvmode}{%
  \ltx@IfUndefined{leavevmode}{%
    \ltx@IfUndefined{voidb@x}{%
      \ltx@IfUndefined{newbox}{%
        \def\ltx@leavevmode{%
          \begingroup
            \setbox\ltx@zero=\hbox{}%
            \begingroup
              \setbox\ltx@zero=\hbox{\box\ltx@zero}%
            \endgroup
            \unhbox\ltx@zero
          \endgroup
        }%
      }{%
        \csname newbox\endcsname\LTXcmds@VoidBox
        \ifvoid\LTXcmds@VoidBox
        \else
          \setbox\LTXcmds@VoidBox=\hbox{}%
          \begingroup
            \setbox\LTXcmds@VoidBox=\hbox{\box\LTXcmds@VoidBox}%
          \endgroup
        \fi
        \def\ltx@leavevmode{\unhbox\LTXcmds@VoidBox}%
      }%
    }{%
      \def\ltx@leavevmode{\unhbox\voidb@x}%
    }%
  }{%
    \let\ltx@leavevmode\leavevmode
  }%
}{%
  \let\ltx@leavevmode\quitvmode
}
%    \end{macrocode}
%    \end{macro}
%
%    \begin{macro}{\ltx@mbox}
%    \begin{macrocode}
\def\ltx@mbox{%
  \ltx@leavevmode
  \hbox
}
%    \end{macrocode}
%    \end{macro}
%
% \subsection{Help macros}
%
%    \begin{macro}{\LTXcmds@num}
%    \begin{macrocode}
\ltx@IfUndefined{numexpr}{%
  \def\LTXcmds@num#1{%
    \expandafter\ltx@firstofone\expandafter{%
      \number#1%
    }%
  }%
}{%
  \def\LTXcmds@num#1{%
    \expandafter\ltx@firstofone\expandafter{%
      \the\numexpr#1%
    }%
  }%
}
%    \end{macrocode}
%    \end{macro}
%
% \subsection{Expandable test for emptiness}
%
%    \begin{macrocode}
\ltx@IfUndefined{detokenize}{%
%    \end{macrocode}
%
% \subsubsection{Vanilla \hologo{TeX}}
%
%    \begin{macro}{\ltx@ifempty}
%    The macro is based on \cs{@ifempty} of Robert R. Schneck
%    \cite{ifempty:schneck}
%    and \cs{@ifnull} of Ulrich Diez \cite{ifnull:diez}.
%    There are three cases to consider:
%    \begin{compactenum}
%    \item |#1| is empty,
%    \item |#1| is not empty and the first token
%      is not a begingroup character,
%    \item |#1| starts with a begingroup character (catcode 1).
%    \end{compactenum}
%    \begin{macrocode}
  \def\LTXcmds@temp#1{%
    \long\def\ltx@ifempty##1{%
      \romannumeral0%
      \iffalse{\fi
        \expandafter\ltx@gobble\expandafter{%
          \expandafter{\string##1}%
          \expandafter\ltx@gobble\string
        }%
        \expandafter\ltx@firstofthree\expandafter
        {\iffalse}\fi
        \expandafter#1\ltx@secondoftwo
      }%
      \expandafter#1\ltx@firstoftwo
    }%
%    \end{macrocode}
%    \end{macro}
%    \begin{macro}{\ltx@ifblank}
%    \begin{macrocode}
    \long\def\ltx@ifblank##1{%
      \romannumeral0%
      \iffalse{\fi
        \expandafter\expandafter\expandafter\ltx@gobble
        \expandafter\expandafter\expandafter{%
          \expandafter\expandafter\expandafter{%
            \expandafter\string\ltx@gobble##1.%
          }%
          \expandafter\ltx@gobble\string
        }%
        \expandafter\ltx@firstofthree\expandafter
        {\iffalse}\fi
        \expandafter#1\ltx@secondoftwo
      }%
      \expandafter#1\ltx@firstoftwo
    }%
  }%
  \LTXcmds@temp{ }%
%    \end{macrocode}
%    \end{macro}
%    \begin{macrocode}
}{%
%    \end{macrocode}
%
% \subsubsection{With \cs{detokenize}}
%
%    Ahmed Musa provided \cs{ifstrempty} using \cs{detokenize}
%    and \cs{pdfstrcmp} \cite{ifstrempty:musa}. Ulrich Diez, GL,
%    Heiko Oberdiek
%    improved it further by removing \cs{pdfstrcmp} and
%    taking three arguments
%    \cite{ifstrempty:diez1,ifstrempty:diez2,ifstrempty:diez3,^^A
%      ifstrempty:gl,ifstrempty:ho}.
%
%    \begin{macro}{\ltx@ifempty}
%    \begin{macrocode}
  \long\def\ltx@ifempty#1{%
    \romannumeral%
    \csname
      LTXcmds@ifempty%
      \ifcat$\detokenize{#1}$%
        @%
      \fi
    \endcsname
  }%
%    \end{macrocode}
%    \end{macro}
%    \begin{macro}{\LTXcmds@ifempty@}
%    \begin{macrocode}
  \long\def\LTXcmds@ifempty@#1#2{0 #1}%
%    \end{macrocode}
%    \end{macro}
%    \begin{macro}{\LTXcmds@ifempty}
%    \begin{macrocode}
  \long\def\LTXcmds@ifempty#1#2{0 #2}%
%    \end{macrocode}
%    \end{macro}
%
% \subsubsection{\cs{ltx@ifblank}}
%
%    \begin{macro}{\ltx@ifblank}
%    \begin{macrocode}
  \long\def\ltx@ifblank#1{%
    \romannumeral%
    \csname
      LTXcmds@ifempty%
      \ifcat$\detokenize\expandafter{\ltx@gobble#1.}$%
        @%
      \fi
    \endcsname
  }%
%    \end{macrocode}
%    \end{macro}
%
%    \begin{macrocode}
}
%    \end{macrocode}
%
% \subsection{\cs{ltx@zapspace}}
%
%    \begin{macro}{\ltx@zapspace}
%    \begin{macrocode}
\long\def\ltx@zapspace#1{%
  \romannumeral
  \LTXcmds@zapspace\ltx@zero#1 \@nil
}
%    \end{macrocode}
%    \end{macro}
%    \begin{macro}{\LTXcmds@zapspace}
%    \begin{macrocode}
\long\def\LTXcmds@zapspace#1 #2\@nil{%
  \ltx@ifempty{#2}{%
    #1%
  }{%
    \LTXcmds@zapspace#1#2\@nil
  }%
}
%    \end{macrocode}
%    \end{macro}
%
% \subsection{\cs{ltx@IfBoxEmpty}}
%
%    In case of \hologo{eTeX} the test for an empty box
%    is done via \cs{lastnodetype} as suggested
%    by David Kastrup \cite{ifboxempty:dak}.
%    \begin{macrocode}
\ltx@IfUndefined{lastnodetype}{%
  \catcode`\$=9 %
  \catcode`\&=14 %
}{%
  \catcode`\$=14 %
  \catcode`\&=9 %
}
%    \end{macrocode}
%    \begin{macro}{\ltx@IfBoxEmpty}
%    \begin{macrocode}
\def\ltx@IfBoxEmpty#1{%
  \ifvoid#1\relax
    \expandafter\ltx@secondoftwo
  \else
%    \end{macrocode}
%    Implementation using \hologo{eTeX}'s \cs{lastnodetype}.
%    \begin{macrocode}
&   \begingroup
&     \setbox\ltx@zero=\ifhbox#1\hbox\else\vbox\fi{%
&       \ifhmode\unhcopy\else\unvcopy\fi#1\relax
&       \expandafter
&     }%
&   \expandafter\endgroup
&   \ifnum\lastnodetype<\ltx@zero
&     \expandafter\expandafter\expandafter\ltx@firstoftwo
&   \else
&     \expandafter\expandafter\expandafter\ltx@secondoftwo
&   \fi
%    \end{macrocode}
%    Implementation without \hologo{eTeX} using a signature
%    at the beginning of the test box.
%    \begin{macrocode}
$   \begingroup
$     \setbox\ltx@zero=\ifhbox#1\hbox\else\vbox\fi{%
$       \penalty\ltx@one
$       \ifhmode\unhcopy\else\unvcopy\fi#1\relax
$       \expandafter
$     }%
$     \ifnum\lastpenalty=\ltx@one
%    \end{macrocode}
%    Box 0 has been changed and is restored by
%    closing the group.
%    \begin{macrocode}
$       \endgroup
$       \begingroup
$       \setbox\ltx@zero=\ifhbox#1\hbox\else\vbox\fi{%
$         \penalty\ltx@two
$         \ifhmode\unhcopy\else\unvcopy\fi#1\relax
$         \expandafter
$       }%
$       \ifnum\lastpenalty=\ltx@two
$         \def\next{\endgroup\expandafter\ltx@firstoftwo}%
$       \else
$         \def\next{\endgroup\expandafter\ltx@secondoftwo}%
$       \fi
$     \else
$       \def\next{\endgroup\expandafter\ltx@secondoftwo}%
$     \fi
$   \next
  \fi
}
%    \end{macrocode}
%    \end{macro}
%
%    \begin{macro}{\ltx@IfBoxVoidOrEmpty}
%    \begin{macrocode}
\def\ltx@IfBoxVoidOrEmpty#1{%
  \ifvoid#1\relax
    \expandafter\ltx@thirdoffour
  \fi
  \ltx@IfBoxEmpty{#1}%
}
%    \end{macrocode}
%    \end{macro}
%
%    \begin{macrocode}
\LTXcmds@AtEnd%
%</package>
%    \end{macrocode}
%% \section{Installation}
%
% \subsection{Download}
%
% \paragraph{Package.} This package is available on
% CTAN\footnote{\CTANpkg{ltxcmds}}:
% \begin{description}
% \item[\CTAN{macros/latex/contrib/ltxcmds/ltxcmds.dtx}] The source file.
% \item[\CTAN{macros/latex/contrib/ltxcmds/ltxcmds.pdf}] Documentation.
% \end{description}
%
%
% \paragraph{Bundle.} All the packages of the bundle `ltxcmds'
% are also available in a TDS compliant ZIP archive. There
% the packages are already unpacked and the documentation files
% are generated. The files and directories obey the TDS standard.
% \begin{description}
% \item[\CTANinstall{install/macros/latex/contrib/ltxcmds.tds.zip}]
% \end{description}
% \emph{TDS} refers to the standard ``A Directory Structure
% for \TeX\ Files'' (\CTANpkg{tds}). Directories
% with \xfile{texmf} in their name are usually organized this way.
%
% \subsection{Bundle installation}
%
% \paragraph{Unpacking.} Unpack the \xfile{ltxcmds.tds.zip} in the
% TDS tree (also known as \xfile{texmf} tree) of your choice.
% Example (linux):
% \begin{quote}
%   |unzip ltxcmds.tds.zip -d ~/texmf|
% \end{quote}
%
% \subsection{Package installation}
%
% \paragraph{Unpacking.} The \xfile{.dtx} file is a self-extracting
% \docstrip\ archive. The files are extracted by running the
% \xfile{.dtx} through \plainTeX:
% \begin{quote}
%   \verb|tex ltxcmds.dtx|
% \end{quote}
%
% \paragraph{TDS.} Now the different files must be moved into
% the different directories in your installation TDS tree
% (also known as \xfile{texmf} tree):
% \begin{quote}
% \def\t{^^A
% \begin{tabular}{@{}>{\ttfamily}l@{ $\rightarrow$ }>{\ttfamily}l@{}}
%   ltxcmds.sty & tex/generic/ltxcmds/ltxcmds.sty\\
%   ltxcmds.pdf & doc/latex/ltxcmds/ltxcmds.pdf\\
%   ltxcmds.dtx & source/latex/ltxcmds/ltxcmds.dtx\\
% \end{tabular}^^A
% }^^A
% \sbox0{\t}^^A
% \ifdim\wd0>\linewidth
%   \begingroup
%     \advance\linewidth by\leftmargin
%     \advance\linewidth by\rightmargin
%   \edef\x{\endgroup
%     \def\noexpand\lw{\the\linewidth}^^A
%   }\x
%   \def\lwbox{^^A
%     \leavevmode
%     \hbox to \linewidth{^^A
%       \kern-\leftmargin\relax
%       \hss
%       \usebox0
%       \hss
%       \kern-\rightmargin\relax
%     }^^A
%   }^^A
%   \ifdim\wd0>\lw
%     \sbox0{\small\t}^^A
%     \ifdim\wd0>\linewidth
%       \ifdim\wd0>\lw
%         \sbox0{\footnotesize\t}^^A
%         \ifdim\wd0>\linewidth
%           \ifdim\wd0>\lw
%             \sbox0{\scriptsize\t}^^A
%             \ifdim\wd0>\linewidth
%               \ifdim\wd0>\lw
%                 \sbox0{\tiny\t}^^A
%                 \ifdim\wd0>\linewidth
%                   \lwbox
%                 \else
%                   \usebox0
%                 \fi
%               \else
%                 \lwbox
%               \fi
%             \else
%               \usebox0
%             \fi
%           \else
%             \lwbox
%           \fi
%         \else
%           \usebox0
%         \fi
%       \else
%         \lwbox
%       \fi
%     \else
%       \usebox0
%     \fi
%   \else
%     \lwbox
%   \fi
% \else
%   \usebox0
% \fi
% \end{quote}
% If you have a \xfile{docstrip.cfg} that configures and enables \docstrip's
% TDS installing feature, then some files can already be in the right
% place, see the documentation of \docstrip.
%
% \subsection{Refresh file name databases}
%
% If your \TeX~distribution
% (\TeX\,Live, \mikTeX, \dots) relies on file name databases, you must refresh
% these. For example, \TeX\,Live\ users run \verb|texhash| or
% \verb|mktexlsr|.
%
% \subsection{Some details for the interested}
%
% \paragraph{Unpacking with \LaTeX.}
% The \xfile{.dtx} chooses its action depending on the format:
% \begin{description}
% \item[\plainTeX:] Run \docstrip\ and extract the files.
% \item[\LaTeX:] Generate the documentation.
% \end{description}
% If you insist on using \LaTeX\ for \docstrip\ (really,
% \docstrip\ does not need \LaTeX), then inform the autodetect routine
% about your intention:
% \begin{quote}
%   \verb|latex \let\install=y% \iffalse meta-comment
%
% File: ltxcmds.dtx
% Version: 2019/12/15 v1.24
% Info: LaTeX kernel commands for general use
%
% Copyright (C)
%    2009-2011 Heiko Oberdiek
%    2016-2019 Oberdiek Package Support Group
%    https://github.com/ho-tex/ltxcmds/issues
%
% This work may be distributed and/or modified under the
% conditions of the LaTeX Project Public License, either
% version 1.3c of this license or (at your option) any later
% version. This version of this license is in
%    https://www.latex-project.org/lppl/lppl-1-3c.txt
% and the latest version of this license is in
%    https://www.latex-project.org/lppl.txt
% and version 1.3 or later is part of all distributions of
% LaTeX version 2005/12/01 or later.
%
% This work has the LPPL maintenance status "maintained".
%
% The Current Maintainers of this work are
% Heiko Oberdiek and the Oberdiek Package Support Group
% https://github.com/ho-tex/ltxcmds/issues
%
% The Base Interpreter refers to any `TeX-Format',
% because some files are installed in TDS:tex/generic//.
%
% This work consists of the main source file ltxcmds.dtx
% and the derived files
%    ltxcmds.sty, ltxcmds.pdf, ltxcmds.ins, ltxcmds.drv,
%    ltxcmds-test1.tex, ltxcmds-test-gobble.tex,
%    ltxcmds-test-ifempty.tex, ltxcmds-test-zapspace.tex,
%    ltxcmds-test-ifboxempty.tex, ltxcmds-test-nextchar.tex,
%    ltxcmds-test-carcdr.tex.
%
% Distribution:
%    CTAN:macros/latex/contrib/ltxcmds/ltxcmds.dtx
%    CTAN:macros/latex/contrib/ltxcmds/ltxcmds.pdf
%
% Unpacking:
%    (a) If ltxcmds.ins is present:
%           tex ltxcmds.ins
%    (b) Without ltxcmds.ins:
%           tex ltxcmds.dtx
%    (c) If you insist on using LaTeX
%           latex \let\install=y\input{ltxcmds.dtx}
%        (quote the arguments according to the demands of your shell)
%
% Documentation:
%    (a) If ltxcmds.drv is present:
%           latex ltxcmds.drv
%    (b) Without ltxcmds.drv:
%           latex ltxcmds.dtx; ...
%    The class ltxdoc loads the configuration file ltxdoc.cfg
%    if available. Here you can specify further options, e.g.
%    use A4 as paper format:
%       \PassOptionsToClass{a4paper}{article}
%
%    Programm calls to get the documentation (example):
%       pdflatex ltxcmds.dtx
%       makeindex -s gind.ist ltxcmds.idx
%       pdflatex ltxcmds.dtx
%       makeindex -s gind.ist ltxcmds.idx
%       pdflatex ltxcmds.dtx
%
% Installation:
%    TDS:tex/generic/ltxcmds/ltxcmds.sty
%    TDS:doc/latex/ltxcmds/ltxcmds.pdf
%    TDS:source/latex/ltxcmds/ltxcmds.dtx
%
%<*ignore>
\begingroup
  \catcode123=1 %
  \catcode125=2 %
  \def\x{LaTeX2e}%
\expandafter\endgroup
\ifcase 0\ifx\install y1\fi\expandafter
         \ifx\csname processbatchFile\endcsname\relax\else1\fi
         \ifx\fmtname\x\else 1\fi\relax
\else\csname fi\endcsname
%</ignore>
%<*install>
\input docstrip.tex
\Msg{************************************************************************}
\Msg{* Installation}
\Msg{* Package: ltxcmds 2019/12/15 v1.24 LaTeX kernel commands for general use (HO)}
\Msg{************************************************************************}

\keepsilent
\askforoverwritefalse

\let\MetaPrefix\relax
\preamble

This is a generated file.

Project: ltxcmds
Version: 2019/12/15 v1.24

Copyright (C)
   2009-2011 Heiko Oberdiek
   2016-2019 Oberdiek Package Support Group

This work may be distributed and/or modified under the
conditions of the LaTeX Project Public License, either
version 1.3c of this license or (at your option) any later
version. This version of this license is in
   https://www.latex-project.org/lppl/lppl-1-3c.txt
and the latest version of this license is in
   https://www.latex-project.org/lppl.txt
and version 1.3 or later is part of all distributions of
LaTeX version 2005/12/01 or later.

This work has the LPPL maintenance status "maintained".

The Current Maintainers of this work are
Heiko Oberdiek and the Oberdiek Package Support Group
https://github.com/ho-tex/ltxcmds/issues


The Base Interpreter refers to any `TeX-Format',
because some files are installed in TDS:tex/generic//.

This work consists of the main source file ltxcmds.dtx
and the derived files
   ltxcmds.sty, ltxcmds.pdf, ltxcmds.ins, ltxcmds.drv,
   ltxcmds-test1.tex, ltxcmds-test-gobble.tex,
   ltxcmds-test-ifempty.tex, ltxcmds-test-zapspace.tex,
   ltxcmds-test-ifboxempty.tex, ltxcmds-test-nextchar.tex,
   ltxcmds-test-carcdr.tex.

\endpreamble
\let\MetaPrefix\DoubleperCent

\generate{%
  \file{ltxcmds.ins}{\from{ltxcmds.dtx}{install}}%
  \file{ltxcmds.drv}{\from{ltxcmds.dtx}{driver}}%
  \usedir{tex/generic/ltxcmds}%
  \file{ltxcmds.sty}{\from{ltxcmds.dtx}{package}}%
}

\catcode32=13\relax% active space
\let =\space%
\Msg{************************************************************************}
\Msg{*}
\Msg{* To finish the installation you have to move the following}
\Msg{* file into a directory searched by TeX:}
\Msg{*}
\Msg{*     ltxcmds.sty}
\Msg{*}
\Msg{* To produce the documentation run the file `ltxcmds.drv'}
\Msg{* through LaTeX.}
\Msg{*}
\Msg{* Happy TeXing!}
\Msg{*}
\Msg{************************************************************************}

\endbatchfile
%</install>
%<*ignore>
\fi
%</ignore>
%<*driver>
\NeedsTeXFormat{LaTeX2e}
\ProvidesFile{ltxcmds.drv}%
  [2019/12/15 v1.24 LaTeX kernel commands for general use (HO)]%
\documentclass{ltxdoc}
\usepackage{holtxdoc}[2011/11/22]
\usepackage{zref-savepos}
\usepackage{paralist}
\hfuzz=1pt
\begin{document}
  \DocInput{ltxcmds.dtx}%
\end{document}
%</driver>
% \fi
%
%
%
% \GetFileInfo{ltxcmds.drv}
%
% \title{The \xpackage{ltxcmds} package}
% \date{2019/12/15 v1.24}
% \author{Heiko Oberdiek\thanks
% {Please report any issues at \url{https://github.com/ho-tex/ltxcmds/issues}}}
%
% \maketitle
%
% \begin{abstract}
% The package \xpackage{ltxcmds} exports some utility macros
% from the \LaTeX\ kernel into a separate namespace and
% also provides them for other formats such as plain-\TeX.
% \end{abstract}
%
% \tableofcontents
%
% \newcounter{to}
% \renewcommand*{\theto}{to\arabic{to}}
% \newlength\tolength
% \tolength=\dimexpr .5\linewidth+\oddsidemargin+1in+15mm\relax
% \newcommand*{\TO}{^^A
%   \quad
%   \stepcounter{to}^^A
%   \zsavepos\theto
%   \ifdim\zposx\theto sp>0pt
%     \ifdim\zposx\theto sp<\tolength
%       \kern\dimexpr\tolength-\zposx\theto sp\relax
%     \fi
%   \fi
%   $\rightarrow$\quad
%   \ignorespaces
% }
%
% \section{Documentation}
%
% \subsection{Introduction}
%
% Many of my packages also support other formats such as plain-\TeX.
% Because I am rather familiar with the utility macros from
% \LaTeX's kernel (e.g. \cs{@gobble}, \cs{@firstoftwo}), I found
% myself rewriting them again and again, because they are lacking
% in plain-\TeX.
%
% Therefore this package provides often used macros and similar
% ones with the name prefix \cs{ltx@}. This avoids also faulty
% redefinitions. I remember an example where a package redefined
% \cs{@firstoftwo} with forgetting \cs{long}.
%
% \subsection{Numbers}
%
% \begin{declcs}{ltx@zero} \TO 0\\
%   \cs{ltx@one} \TO 1\\
%   \cs{ltx@two} \TO 2\\
%   \cs{ltx@cclv} \TO 255\\
%   \cs{ltx@minusone} \TO -1
% \end{declcs}
% These commands are numbers 0, 1, 2, 255 and -1. They are not digits
% and a space is not gobbled afterwards. Macro \cs{ltx@minusone}
% is available since version 2010/12/12 v1.15.
%
% \subsection{Scratch registers}
%
% Following the conventions of \hologo{plainTeX} and \hologo{LaTeX}
% the first ten registers are free to use. Even numbered registers
% are for local, odd numbered for global use.
%
% \begin{declcs}{ltx@(Loc,Glob)(Toks,Dimen,Skip)(A,B,C,D,E)}
% \end{declcs}
% The name consists of the prefix \cs{ltx@}, then
% \texttt{Loc} or \texttt{Glob} for local or global usage follows.
% The register type is given by \texttt{Toks} for token register,
% \texttt{Dimen} for dimen register and \texttt{Skip} for skip
% register. As last part the registers are numbered from \texttt{A}
% to \texttt{E}. Example: \cs{ltx@LocToksA}.
%
% Since 2011/04/14 v1.19.
%
% \subsection{Argument killers}
%
% \begin{declcs}{ltx@gobble} \M{1} \TO\\
%   \cs{ltx@gobbletwo} \M{1} \M{2} \TO\\
%   \cs{ltx@gobblethree} \M{1} \M{2} \M{3} \TO\\
%   \cs{ltx@gobblefour} \M{1} \M{2} \M{3} \M{4} \TO
% \end{declcs}
%
% \begin{declcs}{ltx@GobbleNum} \M{num} \M{1} \M{2} \dots\space
%   \M{\meta{num}} \TO
% \end{declcs}
% The first argument \meta{num} of macro \cs{ltx@GobbleNum} specifies,
% how many following arguments are eaten. Macro \cs{ltx@GobbleNum}
% is expandable in exact two expansion steps.
%
% \subsection{Argument grabbers}
%
% \begin{declcs}{ltx@firstofone} \M{1} \TO \meta{1}\\[1ex]
%  \cs{ltx@firstoftwo} \M{1} \M{2} \TO \meta{1}\\
%  \cs{ltx@secondoftwo} \M{1} \M{2} \TO \meta{2}\\[1ex]
%  \cs{ltx@firstofthree} \M{1} \M{2} \M{3} \TO \meta{1}\\
%  \cs{ltx@secondofthree} \M{1} \M{2} \M{3} \TO \meta{2}\\
%  \cs{ltx@thirdofthree} \M{1} \M{2} \M{3} \TO \meta{3}\\
%  \cs{ltx@firstoffour} \M{1} \M{2} \M{3} \M{4} \TO \meta{1}\\
%  \cs{ltx@secondoffour} \M{1} \M{2} \M{3} \M{4} \TO \meta{2}\\
%  \cs{ltx@thirdoffour} \M{1} \M{2} \M{3} \M{4} \TO \meta{3}\\
%  \cs{ltx@fourthoffour} \M{1} \M{2} \M{3} \M{4} \TO \meta{4}
% \end{declcs}
% Macros \cs{ltx@firstofthree}, \cs{ltx@secondofthree}
% and \cs{ltx@thirdofthree} were added in version
% 2010/11/12 v1.11.
% Macros \cs{ltx@firstoffour}, \dots, \cs{ltx@fourthoffour}
% were added in version 2011/02/04 v1.16.
%
% \subsection{List helpers}
%
% \begin{declcs}{ltx@carzero} \dots\ \cs{@nil} \TO\\
%   \cs{ltx@cdrzero} \dots\ \cs{@nil} \TO \dots
% \end{declcs}
%
% \begin{declcs}{ltx@car} \M{1} \dots\ \cs{@nil} \TO \meta{1}\\
%   \cs{ltx@cdr} \M{1} \dots\ \cs{@nil} \TO \dots
% \end{declcs}
%
% \begin{declcs}{ltx@cartwo} \M{1} \M{2} \dots\ \cs{@nil}
%   \TO \meta{1}\meta{2}\\
%   \cs{ltx@carsecond} \M{1} \M{2} \dots\ \cs{@nil} \TO \meta{2}\\
%   \cs{ltx@cdrtwo} \M{1} \M{2} \dots\ \cs{@nil} \TO \dots
% \end{declcs}
%
% \begin{declcs}{ltx@carthree} \M{1} \M{2} \M{3} \dots\ \cs{@nil}
%       \TO \meta{1}\meta{2}\meta{3}\\
%   \cs{ltx@carthird} \M{1} \M{2} \M{3} \dots\ \cs{@nil}
%       \TO \meta{3}\\
%   \cs{ltx@cdrthree} \M{1} \M{2} \M{3} \dots\ \cs{@nil} \TO \dots
% \end{declcs}
%
% \begin{declcs}{ltx@carfour} \M{1} \M{2} \M{3} \M{4} \dots\ \cs{@nil}
%       \TO \meta{1}\meta{2}\meta{3}\meta{4}\\
%   \cs{ltx@carfourth} \M{1} \M{2} \M{3} \M{4} \dots\ \cs{@nil}
%       \TO \meta{4}\\
%   \cs{ltx@cdrfour} \M{1} \M{2} \M{3} \M{4} \dots\ \cs{@nil} \TO \dots
% \end{declcs}
%
% \begin{declcs}{ltx@CarNum} \M{num} \M{1} \dots\ \M{\meta{num}}
%   \M{\meta{num}+1} \dots\ \cs{@nil}\\
%   \TO \M{1} \dots\ \M{\meta{num}} \dots\\
%   \cs{ltx@CarNumth} \M{num} \M{1} \dots\ \M{\meta{num}}
%   \M{\meta{num}+1} \dots\ \cs{@nil}\\
%   \TO \M{\meta{num}} \dots\\
%   \cs{ltx@CdrNum} \M{num} \M{1} \dots\ \M{\meta{num}}
%   \M{\meta{num}+1} \dots\ \cs{@nil}\\
%   \TO \M{\meta{num}+1} \dots
% \end{declcs}
%
% Macros with uppercase letters are expandable in two expansion steps.
% Changes in version 2019/12/15 v1.24:
% \begin{itemize}
% \item
%   Macros \cs{ltx@carsecond}, \cs{ltx@carthird}, \cs{ltx@carfourth},
%   \cs{ltx@CarNumth} added.
% \item
%   Macros \cs{ltx@cdr}, \cs{ltx@cdrtwo}, \cs{ltx@cdrthree}, \cs{ltx@cdrfour},
%   \cs{ltx@CdrNum} are expandable in two expansion steps and retain
%   spaces and braces after the first gobbled arguments.
% \end{itemize}
%
% \subsection{Tail recursion}
%
% \begin{declcs}{ltx@ReturnAfterFi} \M{1} \cs{fi} \TO \cs{fi} \meta{1}\\
%   \cs{ltx@ReturnAfterElseFi} \M{1} \cs{else} \M{2} \cs{fi}
%     \TO \cs{fi} \meta{1}
% \end{declcs}
%
% \subsection{Empty macro}
%
% \begin{declcs}{ltx@empty} \TO
% \end{declcs}
%
% \subsection{Characters}
%
% \begin{declcs}{ltx@space} \TO \verb*| |\\
%   \cs{ltx@percentchar} \TO \texttt{\%}\\
%   \cs{ltx@backslashchar} \TO
%      \makeatletter\texttt{\@backslashchar}\\
%   \cs{ltx@hashchar} \TO \texttt{\#}\qquad(since v1.7)\\
%   \cs{ltx@leftbracechar} \TO \texttt{\textbraceleft}\qquad(since v1.8)\\
%   \cs{ltx@rightbracechar} \TO \texttt{\textbraceright}\qquad(since v1.8)
% \end{declcs}
%
% \subsection{Boolean switch}
%
% \begin{declcs}{ltx@newif} \M{cmd}
% \end{declcs}
% \cs{ltx@newif} defines a new boolean switch \meta{cmd} like \cs{newif}.
% Unlike \plainTeX's \cs{newif}, \cs{ltx@newif} is not \cs{outer}.
% The command \meta{cmd} must start with the two characters |if|.
%
% \begin{declcs}{ltx@newglobalif} \M{cmd}
% \end{declcs}
% \cs{ltx@newglobalif} defines a new boolean switch \meta{cmd}
% like \cs{ltx@newif}. However the switch setting commands,
% \meta{cmd} without the prefix |if| and followed by |true| or |false|
% are acting globally.
%
% \subsection{Command definitions}
%
% \begin{declcs}{ltx@ifundefined} \M{cmd} \M{yes} \M{no}
% \end{declcs}
% If \eTeX\ is available, \cs{ifcsname} is used that does
% not have the side effect of defining undefined commands
% with meaning of \cs{relax}. This command is always expandable.
% Change in version 1.1: Also the meaning \cs{relax} is
% always considered ``undefined''.
%
% \begin{declcs}{ltx@IfUndefined} \M{cmd} \M{yes} \M{no}
% \end{declcs}
% If \eTeX\ is available, \cs{ifcsname} is used that does
% not have the side effect of defining undefined commands
% with meaning of \cs{relax}. Also it always checks for
% the meaning of \cs{relax} and considers this as undefined.
% This macro is not expandable without \eTeX.
%
% \begin{declcs}{ltx@LocalExpandAfter}
% \end{declcs}
% It expands the token after the next token but in a local
% context. That is the difference to \cs{expandafter}.
% The local context discards the side effect of \cs{csname}
% and let the command undefined after the expansion step.
%
% \subsection{Stripping}
%
% \begin{declcs}{ltx@RemovePrefix}\\
%   \cs{ltx@StripPrefix}
% \end{declcs}
% All tokens up to and including the next available character
% `\textgreater' are thrown away. Usually it is used to strip
% the first part of the output of the commands \cs{meaning}
% or \cs{pdflastmatch}. Macro \cs{ltx@RemovePrefix}
% has the same meaning as \LaTeX's \cs{strip@prefix}, whereas
% macro \cs{ltx@StripPrefix} expands the next token
% once before stripping the prefix.
%
% \begin{declcs}{ltx@onelevel@sanitize} \M{macro}
% \end{declcs}
% Macro \cs{ltx@onelevel@sanitize} provides \hologo{LaTeX}'s
% \cs{@onelevel@sanitize}. The macro is expanded once and
% the contents is converted to characters with catcode 12 (other)
% and space tokens with catcode 10 (space). Then then sanitized
% contents is stored into the macro again. Since version 1.12.
%
% \subsection{File management}
%
% All macros in this section are expandable like
% the counterparts of the \LaTeX\ kernel. Also they
% can be used after the preamble.
%
% \subsubsection{File extensions}
%
% \begin{declcs}{ltx@clsextension}\\
%   \cs{ltx@pkgextension}
% \end{declcs}
% Macros \cs{ltx@clsextension} and \cs{ltx@styextension}
% stores the strings |cls| and |sty|. In opposite to \hologo{LaTeX}'s
% \cs{@clsextension} and \cs{@styextension} they can also be
% used after |\begin{document}|.
%
% \subsubsection{Load check}
%
% \begin{declcs}{ltx@ifclassloaded} \M{class} \M{yes} \M{no}\\
%   \cs{ltx@ifpackageloaded} \M{package} \M{yes} \M{no}
% \end{declcs}
% Macros \cs{ltx@ifclassloaded}\slash\cs{ltx@ifpackageloaded}
% execute \meta{yes}, if the \meta{class} or \meta{package} is loaded,
% otherwise \meta{no} is called.
% Both \meta{class} and \meta{package} are specified without
% extension. The macros can also be used after |\begin{document}|.
%
% \begin{declcs}{ltx@iffileloaded} \M{file} \M{yes} \M{no}
% \end{declcs}
% If \LaTeX's \cs{ProvidesFile} macro was called before
% using \meta{file} as argument, then \cs{ltx@iffileloaded}
% calls \meta{yes}, otherwise \meta{no}.
% Therefore it is possible
% that the \meta{file} is loaded, but \meta{no} is executed because
% of a missing \cs{ProvidesFile}. The \LaTeX\ kernel does not have
% a counterpart of \cs{ltx@iffileloaded}.
%
% Note that the file name used in \cs{ProvidesFile} and
% \cs{ltx@iffileloaded} must match. For example, if \TeX's
% default extension \xext{tex} was given in the first command,
% then it must also specified in the latter command and vice versa.
%
% \subsubsection{Version date check}
%
% \begin{declcs}{ltx@ifclasslater} \M{class} \M{date} \M{yes} \M{no}\\
%   \cs{ltx@ifpackagelater} \M{package} \M{date} \M{yes} \M{no}\\
%   \cs{ltx@iffilelater} \M{file} \M{date} \M{yes} \M{no}
% \end{declcs}
% If a \cs{ProvidesClass}\slash\cs{ProvidesPackage}\slash
% \cs{ProvidesFile} command with exact the same class\slash package\slash
% file was executed before with an optional
% argument that starts with a \LaTeX\ version date, then
% this version date is compared with the argument \meta{date}.
% If they are equal or if the version date is the later date,
% then \meta{yes} is called. In all other cases \meta{no} is executed.
%
% A \LaTeX\ date has the format |YYYY/MM/DD| with |YYYY| as year
% with four digits, |MM| as month with two digits and |DD| as day
% with two digits. If \pdfTeX's \cs{pdfmatch} is available, then
% it is used to detect the version date, to reject invalid
% date formats and to reject some invalid dates.
% Dates before 1994/01/01 are always invalid, because version
% dates are introduced with \LaTeXe\ in 1994.
%
% \subsection{Macro additions}
%
% \begin{declcs}{ltx@GlobalAppendToMacro} \M{cmd} \M{addition}\\
%   \cs{ltx@LocalAppendToMacro} \M{cmd} \M{addition}
% \end{declcs}
% The \meta{addition} is appended to the parameterless macro \meta{cmd}.
% If \meta{cmd} is undefined or has the meaning \cs{relax}, then
% it will be initialized as empty macro beforehand. Due to a bug \meta{addition}
% must not contain \cs{par} before version 2010/10/25 v1.9.
%
% \begin{declcs}{ltx@GlobalPrependToMacro} \M{cmd} \M{addition}\\
%   \cs{ltx@LocalPrependToMacro} \M{cmd} \M{addition}
% \end{declcs}
% The \meta{addition} is prepended to the parameterless macro \meta{cmd}.
% If \meta{cmd} is undefined or has the meaning \cs{relax}, then
% it will be initialized as empty macro beforehand. The macros
% were added in version 2011/08/22 v1.21.
%
% \subsection{Next character detection}
%
% \begin{declcs}{ltx@ifnextchar} \M{char} \M{yes} \M{no}
% \end{declcs}
% If next character is \meta{char} then \meta{yes} is called,
% otherwise \meta{no}. The character is not removed.
% Spaces are silently removed when looking for \meta{char}
% as \hologo{LaTeX}'s version \cs{kernel@ifnextchar} does.
% But there are also small differences:
% \begin{itemize}
% \item The space can be used as \meta{char}. In this case
%   optional spaces before \meta{char} are not supported of course.
% \item If the optional space is a command that is a character
%   (defined by \cs{let} or \cs{futurelet}), then
%   \cs{kernel@ifnextchar} breaks with an \hologo{TeX} error.
%   \cs{ltx@ifnextchar} silently removes this token as
%   optional space.
% \end{itemize}
% Since 2010/03/01 v1.3.
%
% \begin{declcs}{ltx@ifnextchar@nospace} \M{char} \M{yes} \M{no}
% \end{declcs}
% Macro \cs{ltx@ifnextchar@nospace} behaves like
% macro \cs{ltx@ifnextchar} with the exception that
% optional spaces are not supported before \meta{char}.
% Since 2011/04/14 v1.19.
%
% \subsection{\cs{ltx@leavevmode}, \cs{ltx@mbox}}
%
% \begin{declcs}{ltx@leavevmode}
% \end{declcs}
% Macro \cs{ltx@leavevmode} calls \hologo{pdfTeX}'s \cs{quitvmode}.
% Otherwise \cs{leavevmode} is used and defined if it is necessary.
%
% \begin{declcs}{ltx@mbox}
% \end{declcs}
% Macro \cs{ltx@mbox} reimplements \cs{mbox} with two changes.
% Instead of \cs{leavevmode} it uses \cs{ltx@leavevmode} and
% stops right after \cs{hbox}. Especially it does not grab
% the argument and allows the extended syntax of \cs{hbox}.
%
% \subsection{Expandable test for emptiness}
%
% \begin{declcs}{ltx@ifempty} \M{stuff} \M{yes} \M{no}
% \end{declcs}
% Macro \cs{ltx@ifempty} checks in exact two expansion steps
% whether \meta{stuff}
% is empty or contains token. Depending on the result
% \meta{yes} or \meta{no} is executed. The token in \meta{stuff}
% may contain \cs{par} and unmatched conditionals (\cs{if}, \cs{else},
% \cs{fi}, \dots). Since version 2010/11/12 v1.11.
%
% \begin{declcs}{ltx@ifblank} \M{stuff} \M{yes} \M{no}
% \end{declcs}
% Macro \cs{ltx@ifblank} tests in exact two expansion steps
% if \meta{stuff} is empty or contain only blank spaces.
% In this case argument \meta{yes} is called. If \meta{stuff}
% contains other tokens than spaces then \meta{no} is executed.
% Since version 2010/12/04 v1.13.
%
% \subsection{Stripping spaces}
%
% \begin{declcs}{ltx@zapspace} \M{stuff}
% \end{declcs}
% Macro \cs{ltx@zapspace} strips spaces from \meta{stuff}
% that are not hidden inside curly braces. Like \hologo{LaTeX}'s
% \cs{zap@space} it is expandable. Differences:
% \begin{itemize}
% \item
% Syntax: \cs{zap@space} also expects a space token
% and \cs{@empty} after \meta{stuff}.
% \item
% Macro \cs{ltx@zapspace} is expandable in exact two
% expansion steps.
% \item
% Macro \cs{ltx@zapspace} always retains curly braces.
% \item
% Macro \cs{zap@space} has a bug. It stops stripping spaces
% after a token group in curly braces if the first two tokens
% inside the group are equal.
% \item
% Macro \cs{ltx@zapspace} also works with \cs{par}
% and conditionals (\cs{if}, \cs{else}, \cs{fi}, \dots).
% \end{itemize}
% Macro \cs{ltx@zapspace} is available since version
% 2010/12/07 v1.14.
%
% \subsection{Check for emptiness of boxes}
%
% \begin{declcs}{ltx@IfBoxEmpty} \M{box register number}
%   \M{yes} \M{no}
% \end{declcs}
% Macro \cs{ltx@IfBoxEmpty} calls \meta{yes} if
% the box exists (\cs{ifvoid} returns false) and the box
% does not contain any content. Otherwise if the box
% is void or contains something, then \meta{no} is executed.
% Thus being empty means that the box exists and is either
% an \cs{hbox} or a \cs{vbox} and may even have dimensions other than
% 0.0\,pt, but the box does not contain anything.
% Macro \cs{ltx@IfBoxEmpty} is available since 2010/02/04 v1.16.
%
% \begin{declcs}{ltx@IfBoxVoidOrEmpty} \M{box register number}
%   \M{yes} \M{no}
% \end{declcs}
% Macro \cs{ltx@IfBoxVoidOrEmpty} calls \meta{yes} if
% the box is either void or does not contain any content.
% Otherwise \meta{no} is executed.
% Macro \cs{ltx@IfBoxVoidOrEmpty} is available since 2010/02/04 v1.16.
%
% \StopEventually{
% }
%
% \section{Implementation}
%
% \subsection{Identification}
%
%    \begin{macrocode}
%<*package>
%    \end{macrocode}
%    Reload check, especially if the package is not used with \LaTeX.
%    \begin{macrocode}
\begingroup\catcode61\catcode48\catcode32=10\relax%
  \catcode13=5 % ^^M
  \endlinechar=13 %
  \catcode35=6 % #
  \catcode39=12 % '
  \catcode44=12 % ,
  \catcode45=12 % -
  \catcode46=12 % .
  \catcode58=12 % :
  \catcode64=11 % @
  \catcode123=1 % {
  \catcode125=2 % }
  \expandafter\let\expandafter\x\csname ver@ltxcmds.sty\endcsname
  \ifx\x\relax % plain-TeX, first loading
  \else
    \def\empty{}%
    \ifx\x\empty % LaTeX, first loading,
      % variable is initialized, but \ProvidesPackage not yet seen
    \else
      \expandafter\ifx\csname PackageInfo\endcsname\relax
        \def\x#1#2{%
          \immediate\write-1{Package #1 Info: #2.}%
        }%
      \else
        \def\x#1#2{\PackageInfo{#1}{#2, stopped}}%
      \fi
      \x{ltxcmds}{The package is already loaded}%
      \aftergroup\endinput
    \fi
  \fi
\endgroup%
%    \end{macrocode}
%    Package identification:
%    \begin{macrocode}
\begingroup\catcode61\catcode48\catcode32=10\relax%
  \catcode13=5 % ^^M
  \endlinechar=13 %
  \catcode35=6 % #
  \catcode39=12 % '
  \catcode40=12 % (
  \catcode41=12 % )
  \catcode44=12 % ,
  \catcode45=12 % -
  \catcode46=12 % .
  \catcode47=12 % /
  \catcode58=12 % :
  \catcode64=11 % @
  \catcode91=12 % [
  \catcode93=12 % ]
  \catcode123=1 % {
  \catcode125=2 % }
  \expandafter\ifx\csname ProvidesPackage\endcsname\relax
    \def\x#1#2#3[#4]{\endgroup
      \immediate\write-1{Package: #3 #4}%
      \xdef#1{#4}%
    }%
  \else
    \def\x#1#2[#3]{\endgroup
      #2[{#3}]%
      \ifx#1\@undefined
        \xdef#1{#3}%
      \fi
      \ifx#1\relax
        \xdef#1{#3}%
      \fi
    }%
  \fi
\expandafter\x\csname ver@ltxcmds.sty\endcsname
\ProvidesPackage{ltxcmds}%
  [2019/12/15 v1.24 LaTeX kernel commands for general use (HO)]%
%    \end{macrocode}
%
%    \begin{macrocode}
\begingroup\catcode61\catcode48\catcode32=10\relax%
  \catcode13=5 % ^^M
  \endlinechar=13 %
  \catcode123=1 % {
  \catcode125=2 % }
  \catcode64=11 % @
  \def\x{\endgroup
    \expandafter\edef\csname LTXcmds@AtEnd\endcsname{%
      \endlinechar=\the\endlinechar\relax
      \catcode13=\the\catcode13\relax
      \catcode32=\the\catcode32\relax
      \catcode35=\the\catcode35\relax
      \catcode61=\the\catcode61\relax
      \catcode64=\the\catcode64\relax
      \catcode123=\the\catcode123\relax
      \catcode125=\the\catcode125\relax
    }%
  }%
\x\catcode61\catcode48\catcode32=10\relax%
\catcode13=5 % ^^M
\endlinechar=13 %
\catcode35=6 % #
\catcode64=11 % @
\catcode123=1 % {
\catcode125=2 % }
\def\TMP@EnsureCode#1#2{%
  \edef\LTXcmds@AtEnd{%
    \LTXcmds@AtEnd
    \catcode#1=\the\catcode#1\relax
  }%
  \catcode#1=#2\relax
}
\TMP@EnsureCode{36}{3}% $
\TMP@EnsureCode{38}{4}% &
\TMP@EnsureCode{40}{12}% (
\TMP@EnsureCode{41}{12}% )
\TMP@EnsureCode{45}{12}% -
\TMP@EnsureCode{46}{12}% .
\TMP@EnsureCode{47}{12}% /
\TMP@EnsureCode{60}{12}% <
\TMP@EnsureCode{62}{12}% >
\TMP@EnsureCode{91}{12}% [
\TMP@EnsureCode{96}{12}% `
\TMP@EnsureCode{93}{12}% ]
\TMP@EnsureCode{94}{12}% ^ (superscript) (!)
\TMP@EnsureCode{124}{12}% |
\edef\LTXcmds@AtEnd{\LTXcmds@AtEnd\noexpand\endinput}
%    \end{macrocode}
%
% \subsection{Numbers}
%
%    \begin{macro}{\ltx@zero}
%    \begin{macrocode}
\chardef\ltx@zero=0 %
%    \end{macrocode}
%    \end{macro}
%    \begin{macro}{\ltx@one}
%    \begin{macrocode}
\chardef\ltx@one=1 %
%    \end{macrocode}
%    \end{macro}
%    \begin{macro}{\ltx@two}
%    \begin{macrocode}
\chardef\ltx@two=2 %
%    \end{macrocode}
%    \end{macro}
%    \begin{macro}{\ltx@active}
%    \begin{macrocode}
\chardef\ltx@active=13 %
%    \end{macrocode}
%    \end{macro}
%    \begin{macro}{\ltx@cclv}
%    \begin{macrocode}
\chardef\ltx@cclv=255 %
%    \end{macrocode}
%    \end{macro}
%
%    \begin{macro}{\ltx@minusone}
%    \begin{macrocode}
\def\ltx@minusone{%
  -\ltx@one
}
%    \end{macrocode}
%    \end{macro}
%
% \subsection{Scratch registers}
%
%    \begin{macro}{\ltx@LocToksA}
%    \begin{macrocode}
\toksdef\ltx@LocToksA=0 %
%    \end{macrocode}
%    \end{macro}
%    \begin{macro}{\ltx@LocToksB}
%    \begin{macrocode}
\toksdef\ltx@LocToksB=2 %
%    \end{macrocode}
%    \end{macro}
%    \begin{macro}{\ltx@LocToksC}
%    \begin{macrocode}
\toksdef\ltx@LocToksC=4 %
%    \end{macrocode}
%    \end{macro}
%    \begin{macro}{\ltx@LocToksD}
%    \begin{macrocode}
\toksdef\ltx@LocToksD=6 %
%    \end{macrocode}
%    \end{macro}
%    \begin{macro}{\ltx@LocToksE}
%    \begin{macrocode}
\toksdef\ltx@LocToksE=8 %
%    \end{macrocode}
%    \end{macro}
%
%    \begin{macro}{\ltx@GlobToksA}
%    \begin{macrocode}
\toksdef\ltx@GlobToksA=1 %
%    \end{macrocode}
%    \end{macro}
%    \begin{macro}{\ltx@GlobToksB}
%    \begin{macrocode}
\toksdef\ltx@GlobToksB=3 %
%    \end{macrocode}
%    \end{macro}
%    \begin{macro}{\ltx@GlobToksC}
%    \begin{macrocode}
\toksdef\ltx@GlobToksC=5 %
%    \end{macrocode}
%    \end{macro}
%    \begin{macro}{\ltx@GlobToksD}
%    \begin{macrocode}
\toksdef\ltx@GlobToksD=7 %
%    \end{macrocode}
%    \end{macro}
%    \begin{macro}{\ltx@GlobToksE}
%    \begin{macrocode}
\toksdef\ltx@GlobToksE=9 %
%    \end{macrocode}
%    \end{macro}
%
%    \begin{macro}{\ltx@LocDimenA}
%    \begin{macrocode}
\dimendef\ltx@LocDimenA=0 %
%    \end{macrocode}
%    \end{macro}
%    \begin{macro}{\ltx@LocDimenB}
%    \begin{macrocode}
\dimendef\ltx@LocDimenB=2 %
%    \end{macrocode}
%    \end{macro}
%    \begin{macro}{\ltx@LocDimenC}
%    \begin{macrocode}
\dimendef\ltx@LocDimenC=4 %
%    \end{macrocode}
%    \end{macro}
%    \begin{macro}{\ltx@LocDimenD}
%    \begin{macrocode}
\dimendef\ltx@LocDimenD=6 %
%    \end{macrocode}
%    \end{macro}
%    \begin{macro}{\ltx@LocDimenE}
%    \begin{macrocode}
\dimendef\ltx@LocDimenE=8 %
%    \end{macrocode}
%    \end{macro}
%
%    \begin{macro}{\ltx@GlobDimenA}
%    \begin{macrocode}
\dimendef\ltx@GlobDimenA=1 %
%    \end{macrocode}
%    \end{macro}
%    \begin{macro}{\ltx@GlobDimenB}
%    \begin{macrocode}
\dimendef\ltx@GlobDimenB=3 %
%    \end{macrocode}
%    \end{macro}
%    \begin{macro}{\ltx@GlobDimenC}
%    \begin{macrocode}
\dimendef\ltx@GlobDimenC=5 %
%    \end{macrocode}
%    \end{macro}
%    \begin{macro}{\ltx@GlobDimenD}
%    \begin{macrocode}
\dimendef\ltx@GlobDimenD=7 %
%    \end{macrocode}
%    \end{macro}
%    \begin{macro}{\ltx@GlobDimenE}
%    \begin{macrocode}
\dimendef\ltx@GlobDimenE=9 %
%    \end{macrocode}
%    \end{macro}
%
%    \begin{macro}{\ltx@LocSkipA}
%    \begin{macrocode}
\skipdef\ltx@LocSkipA=0 %
%    \end{macrocode}
%    \end{macro}
%    \begin{macro}{\ltx@LocSkipB}
%    \begin{macrocode}
\skipdef\ltx@LocSkipB=2 %
%    \end{macrocode}
%    \end{macro}
%    \begin{macro}{\ltx@LocSkipC}
%    \begin{macrocode}
\skipdef\ltx@LocSkipC=4 %
%    \end{macrocode}
%    \end{macro}
%    \begin{macro}{\ltx@LocSkipD}
%    \begin{macrocode}
\skipdef\ltx@LocSkipD=6 %
%    \end{macrocode}
%    \end{macro}
%    \begin{macro}{\ltx@LocSkipE}
%    \begin{macrocode}
\skipdef\ltx@LocSkipE=8 %
%    \end{macrocode}
%    \end{macro}
%
%    \begin{macro}{\ltx@GlobSkipA}
%    \begin{macrocode}
\skipdef\ltx@GlobSkipA=1 %
%    \end{macrocode}
%    \end{macro}
%    \begin{macro}{\ltx@GlobSkipB}
%    \begin{macrocode}
\skipdef\ltx@GlobSkipB=3 %
%    \end{macrocode}
%    \end{macro}
%    \begin{macro}{\ltx@GlobSkipC}
%    \begin{macrocode}
\skipdef\ltx@GlobSkipC=5 %
%    \end{macrocode}
%    \end{macro}
%    \begin{macro}{\ltx@GlobSkipD}
%    \begin{macrocode}
\skipdef\ltx@GlobSkipD=7 %
%    \end{macrocode}
%    \end{macro}
%    \begin{macro}{\ltx@GlobSkipE}
%    \begin{macrocode}
\skipdef\ltx@GlobSkipE=9 %
%    \end{macrocode}
%    \end{macro}
%
%
% \subsection{Argument killers}
%
%    \begin{macro}{\ltx@gobble}
%    \begin{macrocode}
\long\def\ltx@gobble#1{}
%    \end{macrocode}
%    \end{macro}
%    \begin{macro}{\ltx@gobbletwo}
%    \begin{macrocode}
\long\def\ltx@gobbletwo#1#2{}
%    \end{macrocode}
%    \end{macro}
%    \begin{macro}{\ltx@gobblethree}
%    \begin{macrocode}
\long\def\ltx@gobblethree#1#2#3{}
%    \end{macrocode}
%    \end{macro}
%    \begin{macro}{\ltx@gobblefour}
%    \begin{macrocode}
\long\def\ltx@gobblefour#1#2#3#4{}
%    \end{macrocode}
%    \end{macro}
%
%    \begin{macro}{\ltx@GobbleNum}
%    \begin{macrocode}
\def\ltx@GobbleNum#1{%
  \romannumeral
  \csname ltx@zero%
  \expandafter\LTXcmds@GobbleNum
  \romannumeral\LTXcmds@num{#1}000{m\endcsname}%
}
%    \end{macrocode}
%    \end{macro}
%    \begin{macro}{\LTXcmds@GobbleNum}
%    \begin{macrocode}
\def\LTXcmds@GobbleNum#1{%
  \csname LTXcmds@G#1\LTXcmds@GobbleNum
}
%    \end{macrocode}
%    \end{macro}
%    \begin{macro}{\LTXcmds@Gm}
%    \begin{macrocode}
\long\def\LTXcmds@Gm#1{%
  \endcsname
}
%    \end{macrocode}
%    \end{macro}
%
% \subsection{Argument grabbers}
%
%    \begin{macro}{\ltx@firstofone}
%    \begin{macrocode}
\long\def\ltx@firstofone#1{#1}
%    \end{macrocode}
%    \end{macro}
%    \begin{macro}{\ltx@firstoftwo}
%    \begin{macrocode}
\long\def\ltx@firstoftwo#1#2{#1}
%    \end{macrocode}
%    \end{macro}
%    \begin{macro}{\ltx@secondoftwo}
%    \begin{macrocode}
\long\def\ltx@secondoftwo#1#2{#2}
%    \end{macrocode}
%    \end{macro}
%
%    \begin{macro}{\ltx@firstofthree}
%    \begin{macrocode}
\long\def\ltx@firstofthree#1#2#3{#1}
%    \end{macrocode}
%    \end{macro}
%    \begin{macro}{\ltx@secondofthree}
%    \begin{macrocode}
\long\def\ltx@secondofthree#1#2#3{#2}
%    \end{macrocode}
%    \end{macro}
%    \begin{macro}{\ltx@thirdofthree}
%    \begin{macrocode}
\long\def\ltx@thirdofthree#1#2#3{#3}%
%    \end{macrocode}
%    \end{macro}
%
%    \begin{macro}{\ltx@firstoffour}
%    \begin{macrocode}
\long\def\ltx@firstoffour#1#2#3#4{#1}
%    \end{macrocode}
%    \end{macro}
%    \begin{macro}{\ltx@secondoffour}
%    \begin{macrocode}
\long\def\ltx@secondoffour#1#2#3#4{#2}
%    \end{macrocode}
%    \end{macro}
%    \begin{macro}{\ltx@thirdoffour}
%    \begin{macrocode}
\long\def\ltx@thirdoffour#1#2#3#4{#3}%
%    \end{macrocode}
%    \end{macro}
%    \begin{macro}{\ltx@fourthoffour}
%    \begin{macrocode}
\long\def\ltx@fourthoffour#1#2#3#4{#4}%
%    \end{macrocode}
%    \end{macro}
%
% \subsection{List helpers}
%
%    \begin{macro}{\ltx@carzero}
%    \begin{macrocode}
\long\def\ltx@carzero#1\@nil{}%
%    \end{macrocode}
%    \end{macro}
%    \begin{macro}{\LTXcmds@cdrzero}
%    \begin{macrocode}
\long\def\LTXcmds@cdrzero#1\@nil{#1}
%    \end{macrocode}
%    \end{macro}
%    \begin{macro}{\ltx@cdrzero}
%    \begin{macrocode}
\def\ltx@cdrzero{%
  \romannumeral\LTXcmds@cdrzero\ltx@zero
}
%    \end{macrocode}
%    \end{macro}
%
%    \begin{macro}{\ltx@car}
%    \begin{macrocode}
\long\def\ltx@car#1#2\@nil{#1}
%    \end{macrocode}
%    \end{macro}
%    \begin{macro}{\ltx@cdr}
%    \begin{macrocode}
\long\def\ltx@cdr#1{%
  \romannumeral\LTXcmds@cdrzero\ltx@zero
}
%    \end{macrocode}
%    \end{macro}
%
%    \begin{macro}{\ltx@cartwo}
%    \begin{macrocode}
\long\def\ltx@cartwo#1#2#3\@nil{#1#2}
%    \end{macrocode}
%    \end{macro}
%    \begin{macro}{\ltx@carsecond}
%    \begin{macrocode}
\long\def\ltx@carsecond#1#2#3\@nil{#2}
%    \end{macrocode}
%    \end{macro}
%    \begin{macro}{\ltx@cdrtwo}
%    \begin{macrocode}
\long\def\ltx@cdrtwo#1#2{%
  \romannumeral\LTXcmds@cdrzero\ltx@zero
}
%    \end{macrocode}
%    \end{macro}
%
%    \begin{macro}{\ltx@carthree}
%    \begin{macrocode}
\long\def\ltx@carthree#1#2#3#4\@nil{#1#2#3}
%    \end{macrocode}
%    \end{macro}
%    \begin{macro}{\ltx@carthird}
%    \begin{macrocode}
\long\def\ltx@carthird#1#2#3#4\@nil{#3}
%    \end{macrocode}
%    \end{macro}
%    \begin{macro}{\ltx@cdrthree}
%    \begin{macrocode}
\long\def\ltx@cdrthree#1#2#3{%
  \romannumeral\LTXcmds@cdrzero\ltx@zero
}
%    \end{macrocode}
%    \end{macro}
%
%    \begin{macro}{\ltx@carfour}
%    \begin{macrocode}
\long\def\ltx@carfour#1#2#3#4#5\@nil{#1#2#3#4}
%    \end{macrocode}
%    \end{macro}
%    \begin{macro}{\ltx@carfourth}
%    \begin{macrocode}
\long\def\ltx@carfourth#1#2#3#4#5\@nil{#4}
%    \end{macrocode}
%    \end{macro}
%    \begin{macro}{\ltx@cdrfour}
%    \begin{macrocode}
\long\def\ltx@cdrfour#1#2#3#4{%
  \romannumeral\LTXcmds@cdrzero\ltx@zero
}
%    \end{macrocode}
%    \end{macro}
%
%    \begin{macro}{\ltx@CarNum}
%    \begin{macrocode}
\def\ltx@CarNum#1{%
  \romannumeral
  \csname LTXcmds@CarNumFinish%
  \expandafter\LTXcmds@CarNum
  \romannumeral\LTXcmds@num{#1}000{x\endcsname}%
}
%    \end{macrocode}
%    \end{macro}
%    \begin{macro}{\LTXcmds@CarNum}
%    \begin{macrocode}
\def\LTXcmds@CarNum#1{%
  \csname LTXcmds@C#1\LTXcmds@CarNum
}
%    \end{macrocode}
%    \end{macro}
%    \begin{macro}{\LTXcmds@Cm}
%    \begin{macrocode}
\long\def\LTXcmds@Cm#1#2{%
  \endcsname{#1#2}%
}
%    \end{macrocode}
%    \end{macro}
%    \begin{macro}{\LTXcmds@Cx}
%    \begin{macrocode}
\def\LTXcmds@Cx#1{%
  \endcsname{}%
}
%    \end{macrocode}
%    \end{macro}
%    \begin{macro}{\LTXcmds@CarNumFinish}
%    \begin{macrocode}
\long\def\LTXcmds@CarNumFinish#1#2\@nil{%
  \ltx@zero
  #1%
}
%    \end{macrocode}
%    \end{macro}
%
%    \begin{macro}{\ltx@CarNumth}
%    \begin{macrocode}
\def\ltx@CarNumth#1{%
  \romannumeral
  \expandafter\expandafter\expandafter
  \LTXcmds@CarNumth
  \ltx@GobbleNum{#1}{}%
}
%    \end{macrocode}
%    \end{macro}
%    \begin{macro}{\LTXcmds@CarNumth}
%    \begin{macrocode}
\long\def\LTXcmds@CarNumth#1#2\@nil{%
  \ltx@zero
  #1%
}
%    \end{macrocode}
%    \end{macro}
%
%    \begin{macro}{\ltx@CdrNum}
%    \begin{macrocode}
\def\ltx@CdrNum#1{%
  \romannumeral%
  \expandafter\expandafter\expandafter\ltx@cdrzero
  \expandafter\expandafter\expandafter\ltx@zero
  \ltx@GobbleNum{#1}%
}
%    \end{macrocode}
%    \end{macro}
%
% \subsection{Tail recursion}
%
%    \begin{macro}{\ltx@ReturnAfterFi}
%    \begin{macrocode}
\long\def\ltx@ReturnAfterFi#1\fi{\fi#1}
%    \end{macrocode}
%    \end{macro}
%    \begin{macro}{\ltx@ReturnAfterElseFi}
%    \begin{macrocode}
\long\def\ltx@ReturnAfterElseFi#1\else#2\fi{\fi#1}
%    \end{macrocode}
%    \end{macro}
%
% \subsection{Empty macro}
%
%    \begin{macro}{\ltx@empty}
%    \begin{macrocode}
\def\ltx@empty{}
%    \end{macrocode}
%    \end{macro}
%
% \subsection{Characters}
%
%    \begin{macro}{\ltx@space}
%    \begin{macrocode}
\def\ltx@space{ }
%    \end{macrocode}
%    \end{macro}
%    \begin{macro}{\ltx@percentchar}
%    \begin{macrocode}
\begingroup
  \lccode`0=`\%\relax
\lowercase{\endgroup
  \def\ltx@percentchar{0}%
}
%    \end{macrocode}
%    \end{macro}
%    \begin{macro}{\ltx@backslashchar}
%    \begin{macrocode}
\begingroup
  \lccode`0=`\\\relax
\lowercase{\endgroup
  \def\ltx@backslashchar{0}%
}
%    \end{macrocode}
%    \end{macro}
%    \begin{macro}{\ltx@hashchar}
%    \begin{macrocode}
\begingroup
  \lccode`0=`\#\relax
\lowercase{\endgroup
  \def\ltx@hashchar{0}%
}
%    \end{macrocode}
%    \end{macro}
%    \begin{macro}{\ltx@leftbracechar}
%    \begin{macrocode}
\begingroup
  \lccode`0=`\{\relax
\lowercase{\endgroup
  \def\ltx@leftbracechar{0}%
}
%    \end{macrocode}
%    \end{macro}
%    \begin{macro}{\ltx@rightbracechar}
%    \begin{macrocode}
\begingroup
  \lccode`0=`\}\relax
\lowercase{\endgroup
  \def\ltx@rightbracechar{0}%
}
%    \end{macrocode}
%    \end{macro}
%
% \subsection{Boolean switch}
%
%    \begin{macro}{\ltx@newif}
%    \begin{macrocode}
\def\ltx@newif#1{%
  \begingroup
    \escapechar=-1 %
  \expandafter\endgroup
  \expandafter\LTXcmds@newif\string#1\@nil
}
%    \end{macrocode}
%    \end{macro}
%    \begin{macro}{\LTXcmds@newif}
%    \begin{macrocode}
\begingroup
  \escapechar=-1 %
\expandafter\endgroup
\expandafter\def\expandafter\LTXcmds@newif\string\if#1\@nil{%
  \expandafter\edef\csname#1true\endcsname{%
    \let
    \expandafter\noexpand\csname if#1\endcsname
    \noexpand\iftrue
  }%
  \expandafter\edef\csname#1false\endcsname{%
    \let
    \expandafter\noexpand\csname if#1\endcsname
    \noexpand\iffalse
  }%
  \csname#1false\endcsname
}
%    \end{macrocode}
%    \end{macro}
%
%    \begin{macro}{\ltx@newglobalif}
%    \begin{macrocode}
\def\ltx@newglobalif#1{%
  \begingroup
    \escapechar=-1 %
  \expandafter\endgroup
  \expandafter\LTXcmds@newglobalif\string#1\@nil
}
%    \end{macrocode}
%    \end{macro}
%    \begin{macro}{\LTXcmds@newglobalif}
%    \begin{macrocode}
\begingroup
  \escapechar=-1 %
\expandafter\endgroup
\expandafter
\def\expandafter\LTXcmds@newglobalif\string\if#1\@nil{%
  \expandafter\edef\csname#1true\endcsname{%
    \global\let
    \expandafter\noexpand\csname if#1\endcsname
    \noexpand\iftrue
  }%
  \expandafter\edef\csname#1false\endcsname{%
    \global\let
    \expandafter\noexpand\csname if#1\endcsname
    \noexpand\iffalse
  }%
  \csname#1false\endcsname
}
%    \end{macrocode}
%    \end{macro}
%
% \subsection{Command definitions}
%
%    \begin{macro}{\ltx@LocalExpandAfter}
%    \begin{macrocode}
\def\ltx@LocalExpandAfter{%
  \begingroup
    \expandafter\expandafter\expandafter
  \endgroup
  \expandafter
}
%    \end{macrocode}
%    \end{macro}
%
%    \begin{macrocode}
\ltx@LocalExpandAfter
\ifx\csname ifcsname\endcsname\relax
%    \end{macrocode}
%    \begin{macro}{\ltx@ifundefined}
%    \begin{macrocode}
  \def\ltx@ifundefined#1{%
    \expandafter\ifx\csname #1\endcsname\relax
      \expandafter\ltx@firstoftwo
    \else
      \expandafter\ltx@secondoftwo
    \fi
  }%
%    \end{macrocode}
%    \end{macro}
%    \begin{macro}{\ltx@IfUndefined}
%    \begin{macrocode}
  \def\ltx@IfUndefined#1{%
    \begingroup\expandafter\expandafter\expandafter\endgroup
    \expandafter\ifx\csname #1\endcsname\relax
      \expandafter\ltx@firstoftwo
    \else
      \expandafter\ltx@secondoftwo
    \fi
  }%
%    \end{macrocode}
%    \end{macro}
%    \begin{macrocode}
  \expandafter\ltx@gobble
\else
  \expandafter\ltx@firstofone
\fi
{%
%    \end{macrocode}
%    \begin{macro}{\ltx@ifundefined}
%    \begin{macrocode}
  \def\ltx@ifundefined#1{%
    \ifcsname #1\endcsname
      \expandafter\ifx\csname #1\endcsname\relax
        \expandafter\expandafter\expandafter\ltx@firstoftwo
      \else
        \expandafter\expandafter\expandafter\ltx@secondoftwo
      \fi
    \else
      \expandafter\ltx@firstoftwo
    \fi
  }%
%    \end{macrocode}
%    \end{macro}
%    \begin{macro}{\ltx@IfUndefined}
%    \begin{macrocode}
  \let\ltx@IfUndefined\ltx@ifundefined
%    \end{macrocode}
%    \end{macro}
%    \begin{macrocode}
}
%    \end{macrocode}
%
% \subsection{Stripping}
%
%    \begin{macro}{\ltx@RemovePrefix}
%    \begin{macrocode}
\def\ltx@RemovePrefix#1>{}
%    \end{macrocode}
%    \end{macro}
%
%    \begin{macro}{\ltx@StripPrefix}
%    \begin{macrocode}
\def\ltx@StripPrefix{%
  \expandafter\ltx@RemovePrefix
}
%    \end{macrocode}
%    \end{macro}
%
%    \begin{macro}{\ltx@onelevel@sanitize}
%    \begin{macrocode}
\def\ltx@onelevel@sanitize#1{%
  \edef#1{%
    \expandafter
    \ltx@RemovePrefix\meaning#1%
  }%
}
%    \end{macrocode}
%    \end{macro}
%
% \subsection{File management}
%
% \subsubsection{File extensions}
%
%    \begin{macro}{\ltx@clsextension}
%    \begin{macrocode}
\def\ltx@clsextension{cls}
%    \end{macrocode}
%    \end{macro}
%    \begin{macro}{\ltx@pkgextension}
%    \begin{macrocode}
\def\ltx@pkgextension{sty}
%    \end{macrocode}
%    \end{macro}
%
% \subsubsection{Load check}
%
%    \begin{macro}{\ltx@iffileloaded}
%    \begin{macrocode}
\def\ltx@iffileloaded#1{%
  \ltx@ifundefined{ver@#1}\ltx@secondoftwo\ltx@firstoftwo
}
%    \end{macrocode}
%    \end{macro}
%    \begin{macro}{\ltx@ifclassloaded}
%    \begin{macrocode}
\def\ltx@ifclassloaded#1{%
  \ltx@iffileloaded{#1.\ltx@clsextension}%
}
%    \end{macrocode}
%    \end{macro}
%    \begin{macro}{\ltx@ifpackageloaded}
%    \begin{macrocode}
\def\ltx@ifpackageloaded#1{%
  \ltx@iffileloaded{#1.\ltx@pkgextension}%
}
%    \end{macrocode}
%    \end{macro}
%
% \subsubsection{Version date check}
%
%    \begin{macro}{\ltx@iffilelater}
%    \begin{macrocode}
\def\ltx@iffilelater#1#2{%
  \ltx@iffileloaded{#1}{%
    \expandafter\LTXcmds@IfLater\expandafter{%
      \number
      \expandafter\expandafter\expandafter\LTXcmds@ParseVersion
      \expandafter\expandafter\expandafter{%
        \csname ver@#1\endcsname
      }%
    \expandafter}\expandafter{%
      \number
      \expandafter\LTXcmds@ParseVersion\expandafter{#2}%
    }%
  }\ltx@secondoftwo
}
%    \end{macrocode}
%    \end{macro}
%    \begin{macro}{\LTXcmds@IfLater}
%    \begin{macrocode}
\def\LTXcmds@IfLater#1#2{%
  \ifcase 0%
      \ifnum#1<19940101 %
      \else
        \ifnum#2<19940101 %
        \else
          \ifnum#2>#1 %
          \else
            1%
          \fi
        \fi
      \fi
      \ltx@space
    \expandafter\ltx@secondoftwo
  \else
    \expandafter\ltx@firstoftwo
  \fi
}
%    \end{macrocode}
%    \end{macro}
%    \begin{macro}{\ltx@ifclasslater}
%    \begin{macrocode}
\def\ltx@ifclasslater#1{%
  \ltx@iffilelater{#1.\ltx@clsextension}%
}
%    \end{macrocode}
%    \end{macro}
%    \begin{macro}{\ltx@ifpackagelater}
%    \begin{macrocode}
\def\ltx@ifpackagelater#1{%
  \ltx@iffilelater{#1.\ltx@pkgextension}%
}
%    \end{macrocode}
%    \end{macro}
%
%    \begin{macrocode}
\ltx@IfUndefined{pdfmatch}{%
%    \end{macrocode}
%    \begin{macro}{\LTXcmds@ParseVersion}
%    \begin{macrocode}
  \def\LTXcmds@ParseVersion#1{%
    \LTXcmds@@ParseVersion#10000/00/00\@nil
  }%
%    \end{macrocode}
%    \end{macro}
%    \begin{macro}{\LTXcmds@@ParseVersion}
%    \begin{macrocode}
  \def\LTXcmds@@ParseVersion#1#2#3#4/#5#6/#7#8#9\@nil{%
    #1#2#3#4#5#6#7#8%
  }%
%    \end{macrocode}
%    \end{macro}
%    \begin{macrocode}
}{%
%    \end{macrocode}
%    \begin{macro}{\LTXcmds@ParseVersion}
%    \begin{macrocode}
  \def\LTXcmds@ParseVersion#1{%
    \ifnum\pdfmatch{%
      ^%
      (199[4-9]|[2-9][0-9][0-9][0-9])/%
      (0[1-9]|1[0-2])/%
      (0[1-9]|[1-2][0-9]|3[0-1])%
    }{#1}=1 %
      \ltx@StripPrefix\pdflastmatch1 %
      \ltx@StripPrefix\pdflastmatch2 %
      \ltx@StripPrefix\pdflastmatch3 %
    \else
      0%
    \fi
  }%
%    \end{macrocode}
%    \end{macro}
%    \begin{macrocode}
}
%    \end{macrocode}
%
% \subsection{Macro additions}
%
%    \begin{macro}{\ltx@GlobalAppendToMacro}
%    \begin{macrocode}
\long\def\ltx@GlobalAppendToMacro#1#2{%
  \ifx\ltx@undefined#1%
    \let#1\ltx@empty
  \else
    \ifx\relax#1%
      \let#1\ltx@empty
    \fi
  \fi
  \begingroup
    \ltx@LocToksA\expandafter{#1#2}%
    \xdef#1{\the\ltx@LocToksA}%
  \endgroup
}
%    \end{macrocode}
%    \end{macro}
%    \begin{macro}{\ltx@LocalAppendToMacro}
%    \begin{macrocode}
\long\def\ltx@LocalAppendToMacro#1#2{%
  \global\let\LTXcmds@gtemp#1%
  \ifx\ltx@undefined\LTXcmds@gtemp
    \global\let\LTXcmds@gtemp\ltx@empty
  \else
    \ifx\relax\LTXcmds@gtemp
      \global\letLTXcmds@gtemp\ltx@empty
    \fi
  \fi
  \begingroup
    \ltx@LocToksA\expandafter{\LTXcmds@gtemp#2}%
    \xdef\LTXcmds@gtemp{\the\ltx@LocToksA}%
  \endgroup
  \let#1\LTXcmds@gtemp
}
%    \end{macrocode}
%    \end{macro}
%
%    \begin{macro}{\ltx@GlobalPrependToMacro}
%    \begin{macrocode}
\long\def\ltx@GlobalPrependToMacro#1#2{%
  \ifx\ltx@undefined#1%
    \let#1\ltx@empty
  \else
    \ifx\relax#1%
      \let#1\ltx@empty
    \fi
  \fi
  \begingroup
    \ltx@LocToksA{#2}%
    \ltx@LocToksB\expandafter{#1}%
    \xdef#1{\the\ltx@LocToksA\the\ltx@LocToksB}%
  \endgroup
}
%    \end{macrocode}
%    \end{macro}
%    \begin{macro}{\ltx@LocalPrependToMacro}
%    \begin{macrocode}
\long\def\ltx@LocalPrependToMacro#1#2{%
  \global\let\LTXcmds@gtemp#1%
  \ifx\ltx@undefined\LTXcmds@gtemp
    \global\let\LTXcmds@gtemp\ltx@empty
  \else
    \ifx\relax\LTXcmds@gtemp
      \global\letLTXcmds@gtemp\ltx@empty
    \fi
  \fi
  \begingroup
    \ltx@LocToksA{#2}%
    \ltx@LocToksB\expandafter{\LTXcmds@gtemp}%
    \xdef\LTXcmds@gtemp{\the\ltx@LocToksA\the\ltx@LocToksB}%
  \endgroup
  \let#1\LTXcmds@gtemp
}
%    \end{macrocode}
%    \end{macro}
%
% \subsection{Next character detection}
%
%    \begin{macro}{\ltx@ifnextchar}
%    \begin{macrocode}
\long\def\ltx@ifnextchar#1#2#3{%
  \begingroup
  \let\LTXcmds@CharToken= #1\relax
  \ltx@LocToksA{\endgroup#2}%
  \ltx@LocToksB{\endgroup#3}%
  \futurelet\LTXcmds@LetToken\LTXcmds@ifnextchar
}
%    \end{macrocode}
%    \end{macro}
%    \begin{macro}{\LTXcmds@ifnextchar}
%    \begin{macrocode}
\def\LTXcmds@ifnextchar{%
  \ifx\LTXcmds@LetToken\LTXcmds@CharToken
    \the\expandafter\ltx@LocToksA
  \else
    \expandafter
      \ifx\csname LTXcmds@LetToken\endcsname\LTXcmds@SpaceToken
      \expandafter\expandafter\expandafter\LTXcmds@@ifnextchar
    \else
      \the\expandafter\expandafter\expandafter\ltx@LocToksB
    \fi
  \fi
}
%    \end{macrocode}
%    \end{macro}
%    \begin{macro}{\LTXcmds@@ifnextchar}
%    \cs{futurelet} does not distinguish between a
%    character and a command that is a character
%    (defined by using \cs{let} or \cs{futurelet}).
%    Therefore the space is catched by \cs{romannumeral}
%    with negative character constant that gobbles one
%    optional space.
%    \begin{macrocode}
\def\LTXcmds@@ifnextchar{%
  \expandafter\futurelet
  \expandafter\LTXcmds@LetToken
  \expandafter\LTXcmds@ifnextchar
  \romannumeral-`\.%
}
%    \end{macrocode}
%    \end{macro}
%    \begin{macro}{\LTXcmds@SpaceToken}
%    \begin{macrocode}
\ltx@firstofone{\let\LTXcmds@SpaceToken= } %
%    \end{macrocode}
%    \end{macro}
%
%    \begin{macro}{\ltx@ifnextchar@nospace}
%    \begin{macrocode}
\long\def\ltx@ifnextchar@nospace#1#2#3{%
  \begingroup
  \let\LTXcmds@CharToken= #1\relax
  \ltx@LocToksA{\endgroup#2}%
  \ltx@LocToksB{\endgroup#3}%
  \futurelet\LTXcmds@LetToken\LTXcmds@ifnextchar@nospace
}
%    \end{macrocode}
%    \end{macro}
%    \begin{macro}{\LTXcmds@ifnextchar@nospace}
%    \begin{macrocode}
\def\LTXcmds@ifnextchar@nospace{%
  \the
  \ifx\LTXcmds@LetToken\LTXcmds@CharToken
    \expandafter\ltx@LocToksA
  \else
    \expandafter\ltx@LocToksB
  \fi
}
%    \end{macrocode}
%    \end{macro}
% \subsection{\cs{ltx@leavevmode}, \cs{ltx@mbox}}
%
%    \begin{macro}{\ltx@leavevmode}
%    \begin{macrocode}
\ltx@IfUndefined{quitvmode}{%
  \ltx@IfUndefined{leavevmode}{%
    \ltx@IfUndefined{voidb@x}{%
      \ltx@IfUndefined{newbox}{%
        \def\ltx@leavevmode{%
          \begingroup
            \setbox\ltx@zero=\hbox{}%
            \begingroup
              \setbox\ltx@zero=\hbox{\box\ltx@zero}%
            \endgroup
            \unhbox\ltx@zero
          \endgroup
        }%
      }{%
        \csname newbox\endcsname\LTXcmds@VoidBox
        \ifvoid\LTXcmds@VoidBox
        \else
          \setbox\LTXcmds@VoidBox=\hbox{}%
          \begingroup
            \setbox\LTXcmds@VoidBox=\hbox{\box\LTXcmds@VoidBox}%
          \endgroup
        \fi
        \def\ltx@leavevmode{\unhbox\LTXcmds@VoidBox}%
      }%
    }{%
      \def\ltx@leavevmode{\unhbox\voidb@x}%
    }%
  }{%
    \let\ltx@leavevmode\leavevmode
  }%
}{%
  \let\ltx@leavevmode\quitvmode
}
%    \end{macrocode}
%    \end{macro}
%
%    \begin{macro}{\ltx@mbox}
%    \begin{macrocode}
\def\ltx@mbox{%
  \ltx@leavevmode
  \hbox
}
%    \end{macrocode}
%    \end{macro}
%
% \subsection{Help macros}
%
%    \begin{macro}{\LTXcmds@num}
%    \begin{macrocode}
\ltx@IfUndefined{numexpr}{%
  \def\LTXcmds@num#1{%
    \expandafter\ltx@firstofone\expandafter{%
      \number#1%
    }%
  }%
}{%
  \def\LTXcmds@num#1{%
    \expandafter\ltx@firstofone\expandafter{%
      \the\numexpr#1%
    }%
  }%
}
%    \end{macrocode}
%    \end{macro}
%
% \subsection{Expandable test for emptiness}
%
%    \begin{macrocode}
\ltx@IfUndefined{detokenize}{%
%    \end{macrocode}
%
% \subsubsection{Vanilla \hologo{TeX}}
%
%    \begin{macro}{\ltx@ifempty}
%    The macro is based on \cs{@ifempty} of Robert R. Schneck
%    \cite{ifempty:schneck}
%    and \cs{@ifnull} of Ulrich Diez \cite{ifnull:diez}.
%    There are three cases to consider:
%    \begin{compactenum}
%    \item |#1| is empty,
%    \item |#1| is not empty and the first token
%      is not a begingroup character,
%    \item |#1| starts with a begingroup character (catcode 1).
%    \end{compactenum}
%    \begin{macrocode}
  \def\LTXcmds@temp#1{%
    \long\def\ltx@ifempty##1{%
      \romannumeral0%
      \iffalse{\fi
        \expandafter\ltx@gobble\expandafter{%
          \expandafter{\string##1}%
          \expandafter\ltx@gobble\string
        }%
        \expandafter\ltx@firstofthree\expandafter
        {\iffalse}\fi
        \expandafter#1\ltx@secondoftwo
      }%
      \expandafter#1\ltx@firstoftwo
    }%
%    \end{macrocode}
%    \end{macro}
%    \begin{macro}{\ltx@ifblank}
%    \begin{macrocode}
    \long\def\ltx@ifblank##1{%
      \romannumeral0%
      \iffalse{\fi
        \expandafter\expandafter\expandafter\ltx@gobble
        \expandafter\expandafter\expandafter{%
          \expandafter\expandafter\expandafter{%
            \expandafter\string\ltx@gobble##1.%
          }%
          \expandafter\ltx@gobble\string
        }%
        \expandafter\ltx@firstofthree\expandafter
        {\iffalse}\fi
        \expandafter#1\ltx@secondoftwo
      }%
      \expandafter#1\ltx@firstoftwo
    }%
  }%
  \LTXcmds@temp{ }%
%    \end{macrocode}
%    \end{macro}
%    \begin{macrocode}
}{%
%    \end{macrocode}
%
% \subsubsection{With \cs{detokenize}}
%
%    Ahmed Musa provided \cs{ifstrempty} using \cs{detokenize}
%    and \cs{pdfstrcmp} \cite{ifstrempty:musa}. Ulrich Diez, GL,
%    Heiko Oberdiek
%    improved it further by removing \cs{pdfstrcmp} and
%    taking three arguments
%    \cite{ifstrempty:diez1,ifstrempty:diez2,ifstrempty:diez3,^^A
%      ifstrempty:gl,ifstrempty:ho}.
%
%    \begin{macro}{\ltx@ifempty}
%    \begin{macrocode}
  \long\def\ltx@ifempty#1{%
    \romannumeral%
    \csname
      LTXcmds@ifempty%
      \ifcat$\detokenize{#1}$%
        @%
      \fi
    \endcsname
  }%
%    \end{macrocode}
%    \end{macro}
%    \begin{macro}{\LTXcmds@ifempty@}
%    \begin{macrocode}
  \long\def\LTXcmds@ifempty@#1#2{0 #1}%
%    \end{macrocode}
%    \end{macro}
%    \begin{macro}{\LTXcmds@ifempty}
%    \begin{macrocode}
  \long\def\LTXcmds@ifempty#1#2{0 #2}%
%    \end{macrocode}
%    \end{macro}
%
% \subsubsection{\cs{ltx@ifblank}}
%
%    \begin{macro}{\ltx@ifblank}
%    \begin{macrocode}
  \long\def\ltx@ifblank#1{%
    \romannumeral%
    \csname
      LTXcmds@ifempty%
      \ifcat$\detokenize\expandafter{\ltx@gobble#1.}$%
        @%
      \fi
    \endcsname
  }%
%    \end{macrocode}
%    \end{macro}
%
%    \begin{macrocode}
}
%    \end{macrocode}
%
% \subsection{\cs{ltx@zapspace}}
%
%    \begin{macro}{\ltx@zapspace}
%    \begin{macrocode}
\long\def\ltx@zapspace#1{%
  \romannumeral
  \LTXcmds@zapspace\ltx@zero#1 \@nil
}
%    \end{macrocode}
%    \end{macro}
%    \begin{macro}{\LTXcmds@zapspace}
%    \begin{macrocode}
\long\def\LTXcmds@zapspace#1 #2\@nil{%
  \ltx@ifempty{#2}{%
    #1%
  }{%
    \LTXcmds@zapspace#1#2\@nil
  }%
}
%    \end{macrocode}
%    \end{macro}
%
% \subsection{\cs{ltx@IfBoxEmpty}}
%
%    In case of \hologo{eTeX} the test for an empty box
%    is done via \cs{lastnodetype} as suggested
%    by David Kastrup \cite{ifboxempty:dak}.
%    \begin{macrocode}
\ltx@IfUndefined{lastnodetype}{%
  \catcode`\$=9 %
  \catcode`\&=14 %
}{%
  \catcode`\$=14 %
  \catcode`\&=9 %
}
%    \end{macrocode}
%    \begin{macro}{\ltx@IfBoxEmpty}
%    \begin{macrocode}
\def\ltx@IfBoxEmpty#1{%
  \ifvoid#1\relax
    \expandafter\ltx@secondoftwo
  \else
%    \end{macrocode}
%    Implementation using \hologo{eTeX}'s \cs{lastnodetype}.
%    \begin{macrocode}
&   \begingroup
&     \setbox\ltx@zero=\ifhbox#1\hbox\else\vbox\fi{%
&       \ifhmode\unhcopy\else\unvcopy\fi#1\relax
&       \expandafter
&     }%
&   \expandafter\endgroup
&   \ifnum\lastnodetype<\ltx@zero
&     \expandafter\expandafter\expandafter\ltx@firstoftwo
&   \else
&     \expandafter\expandafter\expandafter\ltx@secondoftwo
&   \fi
%    \end{macrocode}
%    Implementation without \hologo{eTeX} using a signature
%    at the beginning of the test box.
%    \begin{macrocode}
$   \begingroup
$     \setbox\ltx@zero=\ifhbox#1\hbox\else\vbox\fi{%
$       \penalty\ltx@one
$       \ifhmode\unhcopy\else\unvcopy\fi#1\relax
$       \expandafter
$     }%
$     \ifnum\lastpenalty=\ltx@one
%    \end{macrocode}
%    Box 0 has been changed and is restored by
%    closing the group.
%    \begin{macrocode}
$       \endgroup
$       \begingroup
$       \setbox\ltx@zero=\ifhbox#1\hbox\else\vbox\fi{%
$         \penalty\ltx@two
$         \ifhmode\unhcopy\else\unvcopy\fi#1\relax
$         \expandafter
$       }%
$       \ifnum\lastpenalty=\ltx@two
$         \def\next{\endgroup\expandafter\ltx@firstoftwo}%
$       \else
$         \def\next{\endgroup\expandafter\ltx@secondoftwo}%
$       \fi
$     \else
$       \def\next{\endgroup\expandafter\ltx@secondoftwo}%
$     \fi
$   \next
  \fi
}
%    \end{macrocode}
%    \end{macro}
%
%    \begin{macro}{\ltx@IfBoxVoidOrEmpty}
%    \begin{macrocode}
\def\ltx@IfBoxVoidOrEmpty#1{%
  \ifvoid#1\relax
    \expandafter\ltx@thirdoffour
  \fi
  \ltx@IfBoxEmpty{#1}%
}
%    \end{macrocode}
%    \end{macro}
%
%    \begin{macrocode}
\LTXcmds@AtEnd%
%</package>
%    \end{macrocode}
%% \section{Installation}
%
% \subsection{Download}
%
% \paragraph{Package.} This package is available on
% CTAN\footnote{\CTANpkg{ltxcmds}}:
% \begin{description}
% \item[\CTAN{macros/latex/contrib/ltxcmds/ltxcmds.dtx}] The source file.
% \item[\CTAN{macros/latex/contrib/ltxcmds/ltxcmds.pdf}] Documentation.
% \end{description}
%
%
% \paragraph{Bundle.} All the packages of the bundle `ltxcmds'
% are also available in a TDS compliant ZIP archive. There
% the packages are already unpacked and the documentation files
% are generated. The files and directories obey the TDS standard.
% \begin{description}
% \item[\CTANinstall{install/macros/latex/contrib/ltxcmds.tds.zip}]
% \end{description}
% \emph{TDS} refers to the standard ``A Directory Structure
% for \TeX\ Files'' (\CTANpkg{tds}). Directories
% with \xfile{texmf} in their name are usually organized this way.
%
% \subsection{Bundle installation}
%
% \paragraph{Unpacking.} Unpack the \xfile{ltxcmds.tds.zip} in the
% TDS tree (also known as \xfile{texmf} tree) of your choice.
% Example (linux):
% \begin{quote}
%   |unzip ltxcmds.tds.zip -d ~/texmf|
% \end{quote}
%
% \subsection{Package installation}
%
% \paragraph{Unpacking.} The \xfile{.dtx} file is a self-extracting
% \docstrip\ archive. The files are extracted by running the
% \xfile{.dtx} through \plainTeX:
% \begin{quote}
%   \verb|tex ltxcmds.dtx|
% \end{quote}
%
% \paragraph{TDS.} Now the different files must be moved into
% the different directories in your installation TDS tree
% (also known as \xfile{texmf} tree):
% \begin{quote}
% \def\t{^^A
% \begin{tabular}{@{}>{\ttfamily}l@{ $\rightarrow$ }>{\ttfamily}l@{}}
%   ltxcmds.sty & tex/generic/ltxcmds/ltxcmds.sty\\
%   ltxcmds.pdf & doc/latex/ltxcmds/ltxcmds.pdf\\
%   ltxcmds.dtx & source/latex/ltxcmds/ltxcmds.dtx\\
% \end{tabular}^^A
% }^^A
% \sbox0{\t}^^A
% \ifdim\wd0>\linewidth
%   \begingroup
%     \advance\linewidth by\leftmargin
%     \advance\linewidth by\rightmargin
%   \edef\x{\endgroup
%     \def\noexpand\lw{\the\linewidth}^^A
%   }\x
%   \def\lwbox{^^A
%     \leavevmode
%     \hbox to \linewidth{^^A
%       \kern-\leftmargin\relax
%       \hss
%       \usebox0
%       \hss
%       \kern-\rightmargin\relax
%     }^^A
%   }^^A
%   \ifdim\wd0>\lw
%     \sbox0{\small\t}^^A
%     \ifdim\wd0>\linewidth
%       \ifdim\wd0>\lw
%         \sbox0{\footnotesize\t}^^A
%         \ifdim\wd0>\linewidth
%           \ifdim\wd0>\lw
%             \sbox0{\scriptsize\t}^^A
%             \ifdim\wd0>\linewidth
%               \ifdim\wd0>\lw
%                 \sbox0{\tiny\t}^^A
%                 \ifdim\wd0>\linewidth
%                   \lwbox
%                 \else
%                   \usebox0
%                 \fi
%               \else
%                 \lwbox
%               \fi
%             \else
%               \usebox0
%             \fi
%           \else
%             \lwbox
%           \fi
%         \else
%           \usebox0
%         \fi
%       \else
%         \lwbox
%       \fi
%     \else
%       \usebox0
%     \fi
%   \else
%     \lwbox
%   \fi
% \else
%   \usebox0
% \fi
% \end{quote}
% If you have a \xfile{docstrip.cfg} that configures and enables \docstrip's
% TDS installing feature, then some files can already be in the right
% place, see the documentation of \docstrip.
%
% \subsection{Refresh file name databases}
%
% If your \TeX~distribution
% (\TeX\,Live, \mikTeX, \dots) relies on file name databases, you must refresh
% these. For example, \TeX\,Live\ users run \verb|texhash| or
% \verb|mktexlsr|.
%
% \subsection{Some details for the interested}
%
% \paragraph{Unpacking with \LaTeX.}
% The \xfile{.dtx} chooses its action depending on the format:
% \begin{description}
% \item[\plainTeX:] Run \docstrip\ and extract the files.
% \item[\LaTeX:] Generate the documentation.
% \end{description}
% If you insist on using \LaTeX\ for \docstrip\ (really,
% \docstrip\ does not need \LaTeX), then inform the autodetect routine
% about your intention:
% \begin{quote}
%   \verb|latex \let\install=y\input{ltxcmds.dtx}|
% \end{quote}
% Do not forget to quote the argument according to the demands
% of your shell.
%
% \paragraph{Generating the documentation.}
% You can use both the \xfile{.dtx} or the \xfile{.drv} to generate
% the documentation. The process can be configured by the
% configuration file \xfile{ltxdoc.cfg}. For instance, put this
% line into this file, if you want to have A4 as paper format:
% \begin{quote}
%   \verb|\PassOptionsToClass{a4paper}{article}|
% \end{quote}
% An example follows how to generate the
% documentation with pdf\LaTeX:
% \begin{quote}
%\begin{verbatim}
%pdflatex ltxcmds.dtx
%makeindex -s gind.ist ltxcmds.idx
%pdflatex ltxcmds.dtx
%makeindex -s gind.ist ltxcmds.idx
%pdflatex ltxcmds.dtx
%\end{verbatim}
% \end{quote}
%
% \begin{thebibliography}{9}
% \raggedright
%
% \bibitem{ifempty:schneck}
% Robert R. Schneck:
% \textit{Re: \cs{ifempty} solution %
%   (was Macro puzzle: maximally general \cs{ifempty})};
% newsgroup \xnewsgroup{comp.text.tex},
% \url{news:3eef1ada_6@corp.newsgroups.com},
% 2003-06-17.\\
% {\small
% \url{https://groups.google.com/group/comp.text.tex/msg/be03a159ec374895}}
%
% \bibitem{ifnull:diez}
% Ulrich Diez:
% \textit{Re: TeX refuses to strip outer braces in argument};
% newsgroup \xnewsgroup{comp.text.tex},
% \url{news:ibk3t8$ee7$1@news.albasani.net},
% 2010-11-12.\\
% {\small
% \url{https://groups.google.com/group/comp.text.tex/msg/803bd57221a04996}}
%
% \bibitem{ifstrempty:musa}
% Ahmed Musa:
% \textit{Re: TeX refuses to strip outer braces in argument};
% newsgroup \xnewsgroup{comp.text.tex},
% \url{news:f5496afe-40ed-42bd-b629-a2419ecf7c0d@o14g2000prn.googlegroups.com},
% 2010-12-03.\\
% {\small
% \url{https://groups.google.com/group/comp.text.tex/msg/fbf7d61a0c3a807d}}
%
% \bibitem{ifstrempty:diez1}
% Ulrich Diez:
% \textit{Re: TeX refuses to strip outer braces in argument};
% newsgroup \xnewsgroup{comp.text.tex},
% \url{news:idbo94$uka$1@four.albasani.net},
% 2010-12-03.\\
% {\small
% \url{https://groups.google.com/group/comp.text.tex/msg/0c230ee479487962}}
%
% \bibitem{ifstrempty:diez2}
% Ulrich Diez:
% \textit{Re: TeX refuses to strip outer braces in argument};
% newsgroup \xnewsgroup{comp.text.tex},
% \url{news:idbpu4$cg1$1@news.albasani.net},
% 2010-12-03.\\
% {\small
% \url{https://groups.google.com/group/comp.text.tex/msg/bbef4263390d647b}}
%
% \bibitem{ifstrempty:diez3}
% Ulrich Diez:
% \textit{Re: TeX refuses to strip outer braces in argument};
% newsgroup \xnewsgroup{comp.text.tex},
% \url{news:idd4ga$r83$1@four.albasani.net},
% 2010-12-04.\\
% {\small
% \url{https://groups.google.com/group/comp.text.tex/msg/00dfd1ec103cd272}}
%
% \bibitem{ifstrempty:gl}
% GL:
% \textit{Re: TeX refuses to strip outer braces in argument};
% newsgroup \xnewsgroup{comp.text.tex},
% \url{news:4cfa2e27$0$7389$426a74cc@news.free.fr},
% 2010-12-04.\\
% {\small
% \url{https://groups.google.com/group/comp.text.tex/msg/d3a75995c1cf267e}}
%
% \bibitem{ifstrempty:ho}
% Heiko Oberdiek:
% \textit{Re: TeX refuses to strip outer braces in argument};
% newsgroup \xnewsgroup{comp.text.tex},
% \url{news:iddhq1$3kj$1@news.eternal-september.org},
% 2010-12-04.\\
% {\small
% \url{https://groups.google.com/group/comp.text.tex/msg/5f7a23e3ab70e347}}
%
% \bibitem{ifboxempty:dak}
% David Kastrup:
% \textit{How to detect if \cs{vbox} is empty};
% newsgroup \xnewsgroup{comp.text.tex},
% 2011-02-04.\\
% {\small
% \url{https://groups.google.com/group/comp.text.tex/msg/8d3cb89496a4d86d}}
%
% \end{thebibliography}
%
% \begin{History}
%   \begin{Version}{2009/08/05 v1.0}
%   \item
%     First version.
%   \end{Version}
%   \begin{Version}{2009/12/12 v1.1}
%   \item
%     Short title shortened.
%   \item
%     \cs{ltx@IfUndefined} added.
%   \end{Version}
%   \begin{Version}{2010/01/28 v1.2}
%   \item
%     \cs{ltx@RemovePrefix} and \cs{ltx@StripPrefix} added.
%   \item
%     \cs{ltx@ifclassloaded}, \cs{ltx@ifpackageloaded},
%     \cs{ltx@iffileloaded},
%     \cs{ltx@ifclasslater}, \cs{ltx@ifpackagelater},
%     \cs{ltx@iffilelater},
%     \cs{ltx@clsextension}, \cs{ltx@pkgextension} added.
%   \item
%     \cs{ltx@GlobalAppendToMacro}, \cs{ltx@LocalAppendToMacro} added.
%   \end{Version}
%   \begin{Version}{2010/03/01 v1.3}
%   \item
%     \cs{ltx@newif} added.
%   \item
%     \cs{ltx@ifnextchar} added.
%   \item
%     Numbers \cs{ltx@zero}, \cs{ltx@one}, \cs{ltx@two}, \cs{ltx@cclv} added.
%   \end{Version}
%   \begin{Version}{2010/03/09 v1.4}
%   \item
%     \cs{ltx@pkgextension} and \cs{ltx@clsextension} are hardcoded to
%     avoid trouble with \cs{@onlypreamble}.
%   \end{Version}
%   \begin{Version}{2010/04/08 v1.5}
%   \item
%     \cs{ltx@cartwo}, \cs{ltx@cdrtwo},
%     \cs{ltx@carthree}, \cs{ltx@cdrthree},
%     \cs{ltx@carfour}, \cs{ltx@cdrfour} added.
%   \item
%     \cs{ltx@ReturnAfterFi} and \cs{ltx@ReturnAfterElseFi} fixed.
%   \end{Version}
%   \begin{Version}{2010/04/16 v1.6}
%   \item
%     \cs{ltx@leavevmode}, \cs{ltx@mbox} added.
%   \end{Version}
%   \begin{Version}{2010/04/26 v1.7}
%   \item
%     \cs{ltx@GobbleNum}, \cs{ltx@CdrNum}, \cs{ltx@CarNum} added.
%   \item
%     \cs{ltx@carzero}, \cs{ltx@cdrzero} added.
%   \item
%     \cs{ltx@hashchar} added.
%   \end{Version}
%   \begin{Version}{2010/09/11 v1.8}
%   \item
%     \cs{ltx@leftbracechar}, \cs{ltx@rightbracechar} added.
%   \end{Version}
%   \begin{Version}{2010/10/25 v1.9}
%   \item
%     \cs{ltx@LocalAppendToMacro} and \cs{ltx@GlobalAppendToMacro} are
%     now \cs{long}.
%   \end{Version}
%   \begin{Version}{2010/10/31 v1.10}
%   \item
%     \cs{ltx@newglobalif} added.
%   \end{Version}
%   \begin{Version}{2010/11/12 v1.11}
%   \item
%     \cs{ltx@ifempty} added.
%   \item
%     \cs{ltx@firstofthree}, \cs{ltx@secondofthree}, \cs{ltx@thirdofthree}
%     added.
%   \end{Version}
%   \begin{Version}{2010/12/02 v1.12}
%   \item
%     \cs{ltx@onelevel@sanitize} added.
%   \item
%     \cs{LTXcmds@num} fixed for the case with \cs{numexpr}
%     (bug found by GL).
%   \end{Version}
%   \begin{Version}{2010/12/04 v1.13}
%   \item
%     \cs{ltx@ifblank} added.
%   \item
%     Optimization for \cs{ltx@ifempty}.
%   \end{Version}
%   \begin{Version}{2010/12/07 v1.14}
%   \item
%     \cs{ltx@zapspace} added.
%   \end{Version}
%   \begin{Version}{2010/12/12 v1.15}
%   \item
%     \cs{ltx@minusone} added.
%   \end{Version}
%   \begin{Version}{2011/02/04 v1.16}
%   \item
%     \cs{ltx@IfBoxEmpty} and \cs{ltx@IfBoxVoidOrEmpty} added.
%   \item
%     \cs{ltx@firstoffour}, \dots, \cs{ltx@fourthoffour} added.
%   \end{Version}
%   \begin{Version}{2011/02/05 v1.17}
%   \item
%     \cs{ltx@IfBoxEmpty}: an empty box may have non-zero dimensions.
%   \end{Version}
%   \begin{Version}{2011/03/16 v1.18}
%   \item
%     \cs{ltx@ifclasslater} fixed.
%   \end{Version}
%   \begin{Version}{2011/04/14 v1.19}
%   \item
%     \cs{ltx@ifnextchar}: detection of optional spaces modified.
%   \item
%     \cs{ltx(Loc,Glob)(Toks,Dimen,Skip)(A,B,C,D,E)} added.
%   \end{Version}
%   \begin{Version}{2011/04/18 v1.20}
%   \item
%     \cs{ltx@ifnextchar} with conditional support (thanks GL for
%     bug report).
%   \end{Version}
%   \begin{Version}{2011/08/22 v1.21}
%   \item
%     \cs{ltx@GlobalPrependToMacro}, \cs{ltx@LocalPrependToMacro} added
%     (feature request of Martin M\"unch).
%   \end{Version}
%   \begin{Version}{2011/11/09 v1.22}
%   \item
%     \cs{ltx@carsecond}, \cs{ltx@carthird}, \cs{ltx@carfourth},
%     \cs{ltx@CarNumth} added.
%   \item
%     \cs{ltx@cdrzero}, \cs{ltx@cdr}, \cs{ltx@cdrtwo}, cs{ltx@cdrthree},
%     \cs{ltx@cdrfour}, \cs{ltx@CdrNum} modified to retain braces and spaces.
%     They are expandable in two expansion steps.
%   \end{Version}
%   \begin{Version}{2016/05/16 v1.23}
%   \item
%     Documentation updates.
%   \end{Version}
%   \begin{Version}{2019/12/15 v1.24}
%   \item
%     Documentation updates.
%   \end{Version}
% \end{History}
%
% \PrintIndex
%
% \Finale
\endinput
|
% \end{quote}
% Do not forget to quote the argument according to the demands
% of your shell.
%
% \paragraph{Generating the documentation.}
% You can use both the \xfile{.dtx} or the \xfile{.drv} to generate
% the documentation. The process can be configured by the
% configuration file \xfile{ltxdoc.cfg}. For instance, put this
% line into this file, if you want to have A4 as paper format:
% \begin{quote}
%   \verb|\PassOptionsToClass{a4paper}{article}|
% \end{quote}
% An example follows how to generate the
% documentation with pdf\LaTeX:
% \begin{quote}
%\begin{verbatim}
%pdflatex ltxcmds.dtx
%makeindex -s gind.ist ltxcmds.idx
%pdflatex ltxcmds.dtx
%makeindex -s gind.ist ltxcmds.idx
%pdflatex ltxcmds.dtx
%\end{verbatim}
% \end{quote}
%
% \begin{thebibliography}{9}
% \raggedright
%
% \bibitem{ifempty:schneck}
% Robert R. Schneck:
% \textit{Re: \cs{ifempty} solution %
%   (was Macro puzzle: maximally general \cs{ifempty})};
% newsgroup \xnewsgroup{comp.text.tex},
% \url{news:3eef1ada_6@corp.newsgroups.com},
% 2003-06-17.\\
% {\small
% \url{https://groups.google.com/group/comp.text.tex/msg/be03a159ec374895}}
%
% \bibitem{ifnull:diez}
% Ulrich Diez:
% \textit{Re: TeX refuses to strip outer braces in argument};
% newsgroup \xnewsgroup{comp.text.tex},
% \url{news:ibk3t8$ee7$1@news.albasani.net},
% 2010-11-12.\\
% {\small
% \url{https://groups.google.com/group/comp.text.tex/msg/803bd57221a04996}}
%
% \bibitem{ifstrempty:musa}
% Ahmed Musa:
% \textit{Re: TeX refuses to strip outer braces in argument};
% newsgroup \xnewsgroup{comp.text.tex},
% \url{news:f5496afe-40ed-42bd-b629-a2419ecf7c0d@o14g2000prn.googlegroups.com},
% 2010-12-03.\\
% {\small
% \url{https://groups.google.com/group/comp.text.tex/msg/fbf7d61a0c3a807d}}
%
% \bibitem{ifstrempty:diez1}
% Ulrich Diez:
% \textit{Re: TeX refuses to strip outer braces in argument};
% newsgroup \xnewsgroup{comp.text.tex},
% \url{news:idbo94$uka$1@four.albasani.net},
% 2010-12-03.\\
% {\small
% \url{https://groups.google.com/group/comp.text.tex/msg/0c230ee479487962}}
%
% \bibitem{ifstrempty:diez2}
% Ulrich Diez:
% \textit{Re: TeX refuses to strip outer braces in argument};
% newsgroup \xnewsgroup{comp.text.tex},
% \url{news:idbpu4$cg1$1@news.albasani.net},
% 2010-12-03.\\
% {\small
% \url{https://groups.google.com/group/comp.text.tex/msg/bbef4263390d647b}}
%
% \bibitem{ifstrempty:diez3}
% Ulrich Diez:
% \textit{Re: TeX refuses to strip outer braces in argument};
% newsgroup \xnewsgroup{comp.text.tex},
% \url{news:idd4ga$r83$1@four.albasani.net},
% 2010-12-04.\\
% {\small
% \url{https://groups.google.com/group/comp.text.tex/msg/00dfd1ec103cd272}}
%
% \bibitem{ifstrempty:gl}
% GL:
% \textit{Re: TeX refuses to strip outer braces in argument};
% newsgroup \xnewsgroup{comp.text.tex},
% \url{news:4cfa2e27$0$7389$426a74cc@news.free.fr},
% 2010-12-04.\\
% {\small
% \url{https://groups.google.com/group/comp.text.tex/msg/d3a75995c1cf267e}}
%
% \bibitem{ifstrempty:ho}
% Heiko Oberdiek:
% \textit{Re: TeX refuses to strip outer braces in argument};
% newsgroup \xnewsgroup{comp.text.tex},
% \url{news:iddhq1$3kj$1@news.eternal-september.org},
% 2010-12-04.\\
% {\small
% \url{https://groups.google.com/group/comp.text.tex/msg/5f7a23e3ab70e347}}
%
% \bibitem{ifboxempty:dak}
% David Kastrup:
% \textit{How to detect if \cs{vbox} is empty};
% newsgroup \xnewsgroup{comp.text.tex},
% 2011-02-04.\\
% {\small
% \url{https://groups.google.com/group/comp.text.tex/msg/8d3cb89496a4d86d}}
%
% \end{thebibliography}
%
% \begin{History}
%   \begin{Version}{2009/08/05 v1.0}
%   \item
%     First version.
%   \end{Version}
%   \begin{Version}{2009/12/12 v1.1}
%   \item
%     Short title shortened.
%   \item
%     \cs{ltx@IfUndefined} added.
%   \end{Version}
%   \begin{Version}{2010/01/28 v1.2}
%   \item
%     \cs{ltx@RemovePrefix} and \cs{ltx@StripPrefix} added.
%   \item
%     \cs{ltx@ifclassloaded}, \cs{ltx@ifpackageloaded},
%     \cs{ltx@iffileloaded},
%     \cs{ltx@ifclasslater}, \cs{ltx@ifpackagelater},
%     \cs{ltx@iffilelater},
%     \cs{ltx@clsextension}, \cs{ltx@pkgextension} added.
%   \item
%     \cs{ltx@GlobalAppendToMacro}, \cs{ltx@LocalAppendToMacro} added.
%   \end{Version}
%   \begin{Version}{2010/03/01 v1.3}
%   \item
%     \cs{ltx@newif} added.
%   \item
%     \cs{ltx@ifnextchar} added.
%   \item
%     Numbers \cs{ltx@zero}, \cs{ltx@one}, \cs{ltx@two}, \cs{ltx@cclv} added.
%   \end{Version}
%   \begin{Version}{2010/03/09 v1.4}
%   \item
%     \cs{ltx@pkgextension} and \cs{ltx@clsextension} are hardcoded to
%     avoid trouble with \cs{@onlypreamble}.
%   \end{Version}
%   \begin{Version}{2010/04/08 v1.5}
%   \item
%     \cs{ltx@cartwo}, \cs{ltx@cdrtwo},
%     \cs{ltx@carthree}, \cs{ltx@cdrthree},
%     \cs{ltx@carfour}, \cs{ltx@cdrfour} added.
%   \item
%     \cs{ltx@ReturnAfterFi} and \cs{ltx@ReturnAfterElseFi} fixed.
%   \end{Version}
%   \begin{Version}{2010/04/16 v1.6}
%   \item
%     \cs{ltx@leavevmode}, \cs{ltx@mbox} added.
%   \end{Version}
%   \begin{Version}{2010/04/26 v1.7}
%   \item
%     \cs{ltx@GobbleNum}, \cs{ltx@CdrNum}, \cs{ltx@CarNum} added.
%   \item
%     \cs{ltx@carzero}, \cs{ltx@cdrzero} added.
%   \item
%     \cs{ltx@hashchar} added.
%   \end{Version}
%   \begin{Version}{2010/09/11 v1.8}
%   \item
%     \cs{ltx@leftbracechar}, \cs{ltx@rightbracechar} added.
%   \end{Version}
%   \begin{Version}{2010/10/25 v1.9}
%   \item
%     \cs{ltx@LocalAppendToMacro} and \cs{ltx@GlobalAppendToMacro} are
%     now \cs{long}.
%   \end{Version}
%   \begin{Version}{2010/10/31 v1.10}
%   \item
%     \cs{ltx@newglobalif} added.
%   \end{Version}
%   \begin{Version}{2010/11/12 v1.11}
%   \item
%     \cs{ltx@ifempty} added.
%   \item
%     \cs{ltx@firstofthree}, \cs{ltx@secondofthree}, \cs{ltx@thirdofthree}
%     added.
%   \end{Version}
%   \begin{Version}{2010/12/02 v1.12}
%   \item
%     \cs{ltx@onelevel@sanitize} added.
%   \item
%     \cs{LTXcmds@num} fixed for the case with \cs{numexpr}
%     (bug found by GL).
%   \end{Version}
%   \begin{Version}{2010/12/04 v1.13}
%   \item
%     \cs{ltx@ifblank} added.
%   \item
%     Optimization for \cs{ltx@ifempty}.
%   \end{Version}
%   \begin{Version}{2010/12/07 v1.14}
%   \item
%     \cs{ltx@zapspace} added.
%   \end{Version}
%   \begin{Version}{2010/12/12 v1.15}
%   \item
%     \cs{ltx@minusone} added.
%   \end{Version}
%   \begin{Version}{2011/02/04 v1.16}
%   \item
%     \cs{ltx@IfBoxEmpty} and \cs{ltx@IfBoxVoidOrEmpty} added.
%   \item
%     \cs{ltx@firstoffour}, \dots, \cs{ltx@fourthoffour} added.
%   \end{Version}
%   \begin{Version}{2011/02/05 v1.17}
%   \item
%     \cs{ltx@IfBoxEmpty}: an empty box may have non-zero dimensions.
%   \end{Version}
%   \begin{Version}{2011/03/16 v1.18}
%   \item
%     \cs{ltx@ifclasslater} fixed.
%   \end{Version}
%   \begin{Version}{2011/04/14 v1.19}
%   \item
%     \cs{ltx@ifnextchar}: detection of optional spaces modified.
%   \item
%     \cs{ltx(Loc,Glob)(Toks,Dimen,Skip)(A,B,C,D,E)} added.
%   \end{Version}
%   \begin{Version}{2011/04/18 v1.20}
%   \item
%     \cs{ltx@ifnextchar} with conditional support (thanks GL for
%     bug report).
%   \end{Version}
%   \begin{Version}{2011/08/22 v1.21}
%   \item
%     \cs{ltx@GlobalPrependToMacro}, \cs{ltx@LocalPrependToMacro} added
%     (feature request of Martin M\"unch).
%   \end{Version}
%   \begin{Version}{2011/11/09 v1.22}
%   \item
%     \cs{ltx@carsecond}, \cs{ltx@carthird}, \cs{ltx@carfourth},
%     \cs{ltx@CarNumth} added.
%   \item
%     \cs{ltx@cdrzero}, \cs{ltx@cdr}, \cs{ltx@cdrtwo}, cs{ltx@cdrthree},
%     \cs{ltx@cdrfour}, \cs{ltx@CdrNum} modified to retain braces and spaces.
%     They are expandable in two expansion steps.
%   \end{Version}
%   \begin{Version}{2016/05/16 v1.23}
%   \item
%     Documentation updates.
%   \end{Version}
%   \begin{Version}{2019/12/15 v1.24}
%   \item
%     Documentation updates.
%   \end{Version}
% \end{History}
%
% \PrintIndex
%
% \Finale
\endinput

%        (quote the arguments according to the demands of your shell)
%
% Documentation:
%    (a) If ltxcmds.drv is present:
%           latex ltxcmds.drv
%    (b) Without ltxcmds.drv:
%           latex ltxcmds.dtx; ...
%    The class ltxdoc loads the configuration file ltxdoc.cfg
%    if available. Here you can specify further options, e.g.
%    use A4 as paper format:
%       \PassOptionsToClass{a4paper}{article}
%
%    Programm calls to get the documentation (example):
%       pdflatex ltxcmds.dtx
%       makeindex -s gind.ist ltxcmds.idx
%       pdflatex ltxcmds.dtx
%       makeindex -s gind.ist ltxcmds.idx
%       pdflatex ltxcmds.dtx
%
% Installation:
%    TDS:tex/generic/ltxcmds/ltxcmds.sty
%    TDS:doc/latex/ltxcmds/ltxcmds.pdf
%    TDS:source/latex/ltxcmds/ltxcmds.dtx
%
%<*ignore>
\begingroup
  \catcode123=1 %
  \catcode125=2 %
  \def\x{LaTeX2e}%
\expandafter\endgroup
\ifcase 0\ifx\install y1\fi\expandafter
         \ifx\csname processbatchFile\endcsname\relax\else1\fi
         \ifx\fmtname\x\else 1\fi\relax
\else\csname fi\endcsname
%</ignore>
%<*install>
\input docstrip.tex
\Msg{************************************************************************}
\Msg{* Installation}
\Msg{* Package: ltxcmds 2019/12/15 v1.24 LaTeX kernel commands for general use (HO)}
\Msg{************************************************************************}

\keepsilent
\askforoverwritefalse

\let\MetaPrefix\relax
\preamble

This is a generated file.

Project: ltxcmds
Version: 2019/12/15 v1.24

Copyright (C)
   2009-2011 Heiko Oberdiek
   2016-2019 Oberdiek Package Support Group

This work may be distributed and/or modified under the
conditions of the LaTeX Project Public License, either
version 1.3c of this license or (at your option) any later
version. This version of this license is in
   https://www.latex-project.org/lppl/lppl-1-3c.txt
and the latest version of this license is in
   https://www.latex-project.org/lppl.txt
and version 1.3 or later is part of all distributions of
LaTeX version 2005/12/01 or later.

This work has the LPPL maintenance status "maintained".

The Current Maintainers of this work are
Heiko Oberdiek and the Oberdiek Package Support Group
https://github.com/ho-tex/ltxcmds/issues


The Base Interpreter refers to any `TeX-Format',
because some files are installed in TDS:tex/generic//.

This work consists of the main source file ltxcmds.dtx
and the derived files
   ltxcmds.sty, ltxcmds.pdf, ltxcmds.ins, ltxcmds.drv,
   ltxcmds-test1.tex, ltxcmds-test-gobble.tex,
   ltxcmds-test-ifempty.tex, ltxcmds-test-zapspace.tex,
   ltxcmds-test-ifboxempty.tex, ltxcmds-test-nextchar.tex,
   ltxcmds-test-carcdr.tex.

\endpreamble
\let\MetaPrefix\DoubleperCent

\generate{%
  \file{ltxcmds.ins}{\from{ltxcmds.dtx}{install}}%
  \file{ltxcmds.drv}{\from{ltxcmds.dtx}{driver}}%
  \usedir{tex/generic/ltxcmds}%
  \file{ltxcmds.sty}{\from{ltxcmds.dtx}{package}}%
}

\catcode32=13\relax% active space
\let =\space%
\Msg{************************************************************************}
\Msg{*}
\Msg{* To finish the installation you have to move the following}
\Msg{* file into a directory searched by TeX:}
\Msg{*}
\Msg{*     ltxcmds.sty}
\Msg{*}
\Msg{* To produce the documentation run the file `ltxcmds.drv'}
\Msg{* through LaTeX.}
\Msg{*}
\Msg{* Happy TeXing!}
\Msg{*}
\Msg{************************************************************************}

\endbatchfile
%</install>
%<*ignore>
\fi
%</ignore>
%<*driver>
\NeedsTeXFormat{LaTeX2e}
\ProvidesFile{ltxcmds.drv}%
  [2019/12/15 v1.24 LaTeX kernel commands for general use (HO)]%
\documentclass{ltxdoc}
\usepackage{holtxdoc}[2011/11/22]
\usepackage{zref-savepos}
\usepackage{paralist}
\hfuzz=1pt
\begin{document}
  \DocInput{ltxcmds.dtx}%
\end{document}
%</driver>
% \fi
%
%
%
% \GetFileInfo{ltxcmds.drv}
%
% \title{The \xpackage{ltxcmds} package}
% \date{2019/12/15 v1.24}
% \author{Heiko Oberdiek\thanks
% {Please report any issues at \url{https://github.com/ho-tex/ltxcmds/issues}}}
%
% \maketitle
%
% \begin{abstract}
% The package \xpackage{ltxcmds} exports some utility macros
% from the \LaTeX\ kernel into a separate namespace and
% also provides them for other formats such as plain-\TeX.
% \end{abstract}
%
% \tableofcontents
%
% \newcounter{to}
% \renewcommand*{\theto}{to\arabic{to}}
% \newlength\tolength
% \tolength=\dimexpr .5\linewidth+\oddsidemargin+1in+15mm\relax
% \newcommand*{\TO}{^^A
%   \quad
%   \stepcounter{to}^^A
%   \zsavepos\theto
%   \ifdim\zposx\theto sp>0pt
%     \ifdim\zposx\theto sp<\tolength
%       \kern\dimexpr\tolength-\zposx\theto sp\relax
%     \fi
%   \fi
%   $\rightarrow$\quad
%   \ignorespaces
% }
%
% \section{Documentation}
%
% \subsection{Introduction}
%
% Many of my packages also support other formats such as plain-\TeX.
% Because I am rather familiar with the utility macros from
% \LaTeX's kernel (e.g. \cs{@gobble}, \cs{@firstoftwo}), I found
% myself rewriting them again and again, because they are lacking
% in plain-\TeX.
%
% Therefore this package provides often used macros and similar
% ones with the name prefix \cs{ltx@}. This avoids also faulty
% redefinitions. I remember an example where a package redefined
% \cs{@firstoftwo} with forgetting \cs{long}.
%
% \subsection{Numbers}
%
% \begin{declcs}{ltx@zero} \TO 0\\
%   \cs{ltx@one} \TO 1\\
%   \cs{ltx@two} \TO 2\\
%   \cs{ltx@cclv} \TO 255\\
%   \cs{ltx@minusone} \TO -1
% \end{declcs}
% These commands are numbers 0, 1, 2, 255 and -1. They are not digits
% and a space is not gobbled afterwards. Macro \cs{ltx@minusone}
% is available since version 2010/12/12 v1.15.
%
% \subsection{Scratch registers}
%
% Following the conventions of \hologo{plainTeX} and \hologo{LaTeX}
% the first ten registers are free to use. Even numbered registers
% are for local, odd numbered for global use.
%
% \begin{declcs}{ltx@(Loc,Glob)(Toks,Dimen,Skip)(A,B,C,D,E)}
% \end{declcs}
% The name consists of the prefix \cs{ltx@}, then
% \texttt{Loc} or \texttt{Glob} for local or global usage follows.
% The register type is given by \texttt{Toks} for token register,
% \texttt{Dimen} for dimen register and \texttt{Skip} for skip
% register. As last part the registers are numbered from \texttt{A}
% to \texttt{E}. Example: \cs{ltx@LocToksA}.
%
% Since 2011/04/14 v1.19.
%
% \subsection{Argument killers}
%
% \begin{declcs}{ltx@gobble} \M{1} \TO\\
%   \cs{ltx@gobbletwo} \M{1} \M{2} \TO\\
%   \cs{ltx@gobblethree} \M{1} \M{2} \M{3} \TO\\
%   \cs{ltx@gobblefour} \M{1} \M{2} \M{3} \M{4} \TO
% \end{declcs}
%
% \begin{declcs}{ltx@GobbleNum} \M{num} \M{1} \M{2} \dots\space
%   \M{\meta{num}} \TO
% \end{declcs}
% The first argument \meta{num} of macro \cs{ltx@GobbleNum} specifies,
% how many following arguments are eaten. Macro \cs{ltx@GobbleNum}
% is expandable in exact two expansion steps.
%
% \subsection{Argument grabbers}
%
% \begin{declcs}{ltx@firstofone} \M{1} \TO \meta{1}\\[1ex]
%  \cs{ltx@firstoftwo} \M{1} \M{2} \TO \meta{1}\\
%  \cs{ltx@secondoftwo} \M{1} \M{2} \TO \meta{2}\\[1ex]
%  \cs{ltx@firstofthree} \M{1} \M{2} \M{3} \TO \meta{1}\\
%  \cs{ltx@secondofthree} \M{1} \M{2} \M{3} \TO \meta{2}\\
%  \cs{ltx@thirdofthree} \M{1} \M{2} \M{3} \TO \meta{3}\\
%  \cs{ltx@firstoffour} \M{1} \M{2} \M{3} \M{4} \TO \meta{1}\\
%  \cs{ltx@secondoffour} \M{1} \M{2} \M{3} \M{4} \TO \meta{2}\\
%  \cs{ltx@thirdoffour} \M{1} \M{2} \M{3} \M{4} \TO \meta{3}\\
%  \cs{ltx@fourthoffour} \M{1} \M{2} \M{3} \M{4} \TO \meta{4}
% \end{declcs}
% Macros \cs{ltx@firstofthree}, \cs{ltx@secondofthree}
% and \cs{ltx@thirdofthree} were added in version
% 2010/11/12 v1.11.
% Macros \cs{ltx@firstoffour}, \dots, \cs{ltx@fourthoffour}
% were added in version 2011/02/04 v1.16.
%
% \subsection{List helpers}
%
% \begin{declcs}{ltx@carzero} \dots\ \cs{@nil} \TO\\
%   \cs{ltx@cdrzero} \dots\ \cs{@nil} \TO \dots
% \end{declcs}
%
% \begin{declcs}{ltx@car} \M{1} \dots\ \cs{@nil} \TO \meta{1}\\
%   \cs{ltx@cdr} \M{1} \dots\ \cs{@nil} \TO \dots
% \end{declcs}
%
% \begin{declcs}{ltx@cartwo} \M{1} \M{2} \dots\ \cs{@nil}
%   \TO \meta{1}\meta{2}\\
%   \cs{ltx@carsecond} \M{1} \M{2} \dots\ \cs{@nil} \TO \meta{2}\\
%   \cs{ltx@cdrtwo} \M{1} \M{2} \dots\ \cs{@nil} \TO \dots
% \end{declcs}
%
% \begin{declcs}{ltx@carthree} \M{1} \M{2} \M{3} \dots\ \cs{@nil}
%       \TO \meta{1}\meta{2}\meta{3}\\
%   \cs{ltx@carthird} \M{1} \M{2} \M{3} \dots\ \cs{@nil}
%       \TO \meta{3}\\
%   \cs{ltx@cdrthree} \M{1} \M{2} \M{3} \dots\ \cs{@nil} \TO \dots
% \end{declcs}
%
% \begin{declcs}{ltx@carfour} \M{1} \M{2} \M{3} \M{4} \dots\ \cs{@nil}
%       \TO \meta{1}\meta{2}\meta{3}\meta{4}\\
%   \cs{ltx@carfourth} \M{1} \M{2} \M{3} \M{4} \dots\ \cs{@nil}
%       \TO \meta{4}\\
%   \cs{ltx@cdrfour} \M{1} \M{2} \M{3} \M{4} \dots\ \cs{@nil} \TO \dots
% \end{declcs}
%
% \begin{declcs}{ltx@CarNum} \M{num} \M{1} \dots\ \M{\meta{num}}
%   \M{\meta{num}+1} \dots\ \cs{@nil}\\
%   \TO \M{1} \dots\ \M{\meta{num}} \dots\\
%   \cs{ltx@CarNumth} \M{num} \M{1} \dots\ \M{\meta{num}}
%   \M{\meta{num}+1} \dots\ \cs{@nil}\\
%   \TO \M{\meta{num}} \dots\\
%   \cs{ltx@CdrNum} \M{num} \M{1} \dots\ \M{\meta{num}}
%   \M{\meta{num}+1} \dots\ \cs{@nil}\\
%   \TO \M{\meta{num}+1} \dots
% \end{declcs}
%
% Macros with uppercase letters are expandable in two expansion steps.
% Changes in version 2019/12/15 v1.24:
% \begin{itemize}
% \item
%   Macros \cs{ltx@carsecond}, \cs{ltx@carthird}, \cs{ltx@carfourth},
%   \cs{ltx@CarNumth} added.
% \item
%   Macros \cs{ltx@cdr}, \cs{ltx@cdrtwo}, \cs{ltx@cdrthree}, \cs{ltx@cdrfour},
%   \cs{ltx@CdrNum} are expandable in two expansion steps and retain
%   spaces and braces after the first gobbled arguments.
% \end{itemize}
%
% \subsection{Tail recursion}
%
% \begin{declcs}{ltx@ReturnAfterFi} \M{1} \cs{fi} \TO \cs{fi} \meta{1}\\
%   \cs{ltx@ReturnAfterElseFi} \M{1} \cs{else} \M{2} \cs{fi}
%     \TO \cs{fi} \meta{1}
% \end{declcs}
%
% \subsection{Empty macro}
%
% \begin{declcs}{ltx@empty} \TO
% \end{declcs}
%
% \subsection{Characters}
%
% \begin{declcs}{ltx@space} \TO \verb*| |\\
%   \cs{ltx@percentchar} \TO \texttt{\%}\\
%   \cs{ltx@backslashchar} \TO
%      \makeatletter\texttt{\@backslashchar}\\
%   \cs{ltx@hashchar} \TO \texttt{\#}\qquad(since v1.7)\\
%   \cs{ltx@leftbracechar} \TO \texttt{\textbraceleft}\qquad(since v1.8)\\
%   \cs{ltx@rightbracechar} \TO \texttt{\textbraceright}\qquad(since v1.8)
% \end{declcs}
%
% \subsection{Boolean switch}
%
% \begin{declcs}{ltx@newif} \M{cmd}
% \end{declcs}
% \cs{ltx@newif} defines a new boolean switch \meta{cmd} like \cs{newif}.
% Unlike \plainTeX's \cs{newif}, \cs{ltx@newif} is not \cs{outer}.
% The command \meta{cmd} must start with the two characters |if|.
%
% \begin{declcs}{ltx@newglobalif} \M{cmd}
% \end{declcs}
% \cs{ltx@newglobalif} defines a new boolean switch \meta{cmd}
% like \cs{ltx@newif}. However the switch setting commands,
% \meta{cmd} without the prefix |if| and followed by |true| or |false|
% are acting globally.
%
% \subsection{Command definitions}
%
% \begin{declcs}{ltx@ifundefined} \M{cmd} \M{yes} \M{no}
% \end{declcs}
% If \eTeX\ is available, \cs{ifcsname} is used that does
% not have the side effect of defining undefined commands
% with meaning of \cs{relax}. This command is always expandable.
% Change in version 1.1: Also the meaning \cs{relax} is
% always considered ``undefined''.
%
% \begin{declcs}{ltx@IfUndefined} \M{cmd} \M{yes} \M{no}
% \end{declcs}
% If \eTeX\ is available, \cs{ifcsname} is used that does
% not have the side effect of defining undefined commands
% with meaning of \cs{relax}. Also it always checks for
% the meaning of \cs{relax} and considers this as undefined.
% This macro is not expandable without \eTeX.
%
% \begin{declcs}{ltx@LocalExpandAfter}
% \end{declcs}
% It expands the token after the next token but in a local
% context. That is the difference to \cs{expandafter}.
% The local context discards the side effect of \cs{csname}
% and let the command undefined after the expansion step.
%
% \subsection{Stripping}
%
% \begin{declcs}{ltx@RemovePrefix}\\
%   \cs{ltx@StripPrefix}
% \end{declcs}
% All tokens up to and including the next available character
% `\textgreater' are thrown away. Usually it is used to strip
% the first part of the output of the commands \cs{meaning}
% or \cs{pdflastmatch}. Macro \cs{ltx@RemovePrefix}
% has the same meaning as \LaTeX's \cs{strip@prefix}, whereas
% macro \cs{ltx@StripPrefix} expands the next token
% once before stripping the prefix.
%
% \begin{declcs}{ltx@onelevel@sanitize} \M{macro}
% \end{declcs}
% Macro \cs{ltx@onelevel@sanitize} provides \hologo{LaTeX}'s
% \cs{@onelevel@sanitize}. The macro is expanded once and
% the contents is converted to characters with catcode 12 (other)
% and space tokens with catcode 10 (space). Then then sanitized
% contents is stored into the macro again. Since version 1.12.
%
% \subsection{File management}
%
% All macros in this section are expandable like
% the counterparts of the \LaTeX\ kernel. Also they
% can be used after the preamble.
%
% \subsubsection{File extensions}
%
% \begin{declcs}{ltx@clsextension}\\
%   \cs{ltx@pkgextension}
% \end{declcs}
% Macros \cs{ltx@clsextension} and \cs{ltx@styextension}
% stores the strings |cls| and |sty|. In opposite to \hologo{LaTeX}'s
% \cs{@clsextension} and \cs{@styextension} they can also be
% used after |\begin{document}|.
%
% \subsubsection{Load check}
%
% \begin{declcs}{ltx@ifclassloaded} \M{class} \M{yes} \M{no}\\
%   \cs{ltx@ifpackageloaded} \M{package} \M{yes} \M{no}
% \end{declcs}
% Macros \cs{ltx@ifclassloaded}\slash\cs{ltx@ifpackageloaded}
% execute \meta{yes}, if the \meta{class} or \meta{package} is loaded,
% otherwise \meta{no} is called.
% Both \meta{class} and \meta{package} are specified without
% extension. The macros can also be used after |\begin{document}|.
%
% \begin{declcs}{ltx@iffileloaded} \M{file} \M{yes} \M{no}
% \end{declcs}
% If \LaTeX's \cs{ProvidesFile} macro was called before
% using \meta{file} as argument, then \cs{ltx@iffileloaded}
% calls \meta{yes}, otherwise \meta{no}.
% Therefore it is possible
% that the \meta{file} is loaded, but \meta{no} is executed because
% of a missing \cs{ProvidesFile}. The \LaTeX\ kernel does not have
% a counterpart of \cs{ltx@iffileloaded}.
%
% Note that the file name used in \cs{ProvidesFile} and
% \cs{ltx@iffileloaded} must match. For example, if \TeX's
% default extension \xext{tex} was given in the first command,
% then it must also specified in the latter command and vice versa.
%
% \subsubsection{Version date check}
%
% \begin{declcs}{ltx@ifclasslater} \M{class} \M{date} \M{yes} \M{no}\\
%   \cs{ltx@ifpackagelater} \M{package} \M{date} \M{yes} \M{no}\\
%   \cs{ltx@iffilelater} \M{file} \M{date} \M{yes} \M{no}
% \end{declcs}
% If a \cs{ProvidesClass}\slash\cs{ProvidesPackage}\slash
% \cs{ProvidesFile} command with exact the same class\slash package\slash
% file was executed before with an optional
% argument that starts with a \LaTeX\ version date, then
% this version date is compared with the argument \meta{date}.
% If they are equal or if the version date is the later date,
% then \meta{yes} is called. In all other cases \meta{no} is executed.
%
% A \LaTeX\ date has the format |YYYY/MM/DD| with |YYYY| as year
% with four digits, |MM| as month with two digits and |DD| as day
% with two digits. If \pdfTeX's \cs{pdfmatch} is available, then
% it is used to detect the version date, to reject invalid
% date formats and to reject some invalid dates.
% Dates before 1994/01/01 are always invalid, because version
% dates are introduced with \LaTeXe\ in 1994.
%
% \subsection{Macro additions}
%
% \begin{declcs}{ltx@GlobalAppendToMacro} \M{cmd} \M{addition}\\
%   \cs{ltx@LocalAppendToMacro} \M{cmd} \M{addition}
% \end{declcs}
% The \meta{addition} is appended to the parameterless macro \meta{cmd}.
% If \meta{cmd} is undefined or has the meaning \cs{relax}, then
% it will be initialized as empty macro beforehand. Due to a bug \meta{addition}
% must not contain \cs{par} before version 2010/10/25 v1.9.
%
% \begin{declcs}{ltx@GlobalPrependToMacro} \M{cmd} \M{addition}\\
%   \cs{ltx@LocalPrependToMacro} \M{cmd} \M{addition}
% \end{declcs}
% The \meta{addition} is prepended to the parameterless macro \meta{cmd}.
% If \meta{cmd} is undefined or has the meaning \cs{relax}, then
% it will be initialized as empty macro beforehand. The macros
% were added in version 2011/08/22 v1.21.
%
% \subsection{Next character detection}
%
% \begin{declcs}{ltx@ifnextchar} \M{char} \M{yes} \M{no}
% \end{declcs}
% If next character is \meta{char} then \meta{yes} is called,
% otherwise \meta{no}. The character is not removed.
% Spaces are silently removed when looking for \meta{char}
% as \hologo{LaTeX}'s version \cs{kernel@ifnextchar} does.
% But there are also small differences:
% \begin{itemize}
% \item The space can be used as \meta{char}. In this case
%   optional spaces before \meta{char} are not supported of course.
% \item If the optional space is a command that is a character
%   (defined by \cs{let} or \cs{futurelet}), then
%   \cs{kernel@ifnextchar} breaks with an \hologo{TeX} error.
%   \cs{ltx@ifnextchar} silently removes this token as
%   optional space.
% \end{itemize}
% Since 2010/03/01 v1.3.
%
% \begin{declcs}{ltx@ifnextchar@nospace} \M{char} \M{yes} \M{no}
% \end{declcs}
% Macro \cs{ltx@ifnextchar@nospace} behaves like
% macro \cs{ltx@ifnextchar} with the exception that
% optional spaces are not supported before \meta{char}.
% Since 2011/04/14 v1.19.
%
% \subsection{\cs{ltx@leavevmode}, \cs{ltx@mbox}}
%
% \begin{declcs}{ltx@leavevmode}
% \end{declcs}
% Macro \cs{ltx@leavevmode} calls \hologo{pdfTeX}'s \cs{quitvmode}.
% Otherwise \cs{leavevmode} is used and defined if it is necessary.
%
% \begin{declcs}{ltx@mbox}
% \end{declcs}
% Macro \cs{ltx@mbox} reimplements \cs{mbox} with two changes.
% Instead of \cs{leavevmode} it uses \cs{ltx@leavevmode} and
% stops right after \cs{hbox}. Especially it does not grab
% the argument and allows the extended syntax of \cs{hbox}.
%
% \subsection{Expandable test for emptiness}
%
% \begin{declcs}{ltx@ifempty} \M{stuff} \M{yes} \M{no}
% \end{declcs}
% Macro \cs{ltx@ifempty} checks in exact two expansion steps
% whether \meta{stuff}
% is empty or contains token. Depending on the result
% \meta{yes} or \meta{no} is executed. The token in \meta{stuff}
% may contain \cs{par} and unmatched conditionals (\cs{if}, \cs{else},
% \cs{fi}, \dots). Since version 2010/11/12 v1.11.
%
% \begin{declcs}{ltx@ifblank} \M{stuff} \M{yes} \M{no}
% \end{declcs}
% Macro \cs{ltx@ifblank} tests in exact two expansion steps
% if \meta{stuff} is empty or contain only blank spaces.
% In this case argument \meta{yes} is called. If \meta{stuff}
% contains other tokens than spaces then \meta{no} is executed.
% Since version 2010/12/04 v1.13.
%
% \subsection{Stripping spaces}
%
% \begin{declcs}{ltx@zapspace} \M{stuff}
% \end{declcs}
% Macro \cs{ltx@zapspace} strips spaces from \meta{stuff}
% that are not hidden inside curly braces. Like \hologo{LaTeX}'s
% \cs{zap@space} it is expandable. Differences:
% \begin{itemize}
% \item
% Syntax: \cs{zap@space} also expects a space token
% and \cs{@empty} after \meta{stuff}.
% \item
% Macro \cs{ltx@zapspace} is expandable in exact two
% expansion steps.
% \item
% Macro \cs{ltx@zapspace} always retains curly braces.
% \item
% Macro \cs{zap@space} has a bug. It stops stripping spaces
% after a token group in curly braces if the first two tokens
% inside the group are equal.
% \item
% Macro \cs{ltx@zapspace} also works with \cs{par}
% and conditionals (\cs{if}, \cs{else}, \cs{fi}, \dots).
% \end{itemize}
% Macro \cs{ltx@zapspace} is available since version
% 2010/12/07 v1.14.
%
% \subsection{Check for emptiness of boxes}
%
% \begin{declcs}{ltx@IfBoxEmpty} \M{box register number}
%   \M{yes} \M{no}
% \end{declcs}
% Macro \cs{ltx@IfBoxEmpty} calls \meta{yes} if
% the box exists (\cs{ifvoid} returns false) and the box
% does not contain any content. Otherwise if the box
% is void or contains something, then \meta{no} is executed.
% Thus being empty means that the box exists and is either
% an \cs{hbox} or a \cs{vbox} and may even have dimensions other than
% 0.0\,pt, but the box does not contain anything.
% Macro \cs{ltx@IfBoxEmpty} is available since 2010/02/04 v1.16.
%
% \begin{declcs}{ltx@IfBoxVoidOrEmpty} \M{box register number}
%   \M{yes} \M{no}
% \end{declcs}
% Macro \cs{ltx@IfBoxVoidOrEmpty} calls \meta{yes} if
% the box is either void or does not contain any content.
% Otherwise \meta{no} is executed.
% Macro \cs{ltx@IfBoxVoidOrEmpty} is available since 2010/02/04 v1.16.
%
% \StopEventually{
% }
%
% \section{Implementation}
%
% \subsection{Identification}
%
%    \begin{macrocode}
%<*package>
%    \end{macrocode}
%    Reload check, especially if the package is not used with \LaTeX.
%    \begin{macrocode}
\begingroup\catcode61\catcode48\catcode32=10\relax%
  \catcode13=5 % ^^M
  \endlinechar=13 %
  \catcode35=6 % #
  \catcode39=12 % '
  \catcode44=12 % ,
  \catcode45=12 % -
  \catcode46=12 % .
  \catcode58=12 % :
  \catcode64=11 % @
  \catcode123=1 % {
  \catcode125=2 % }
  \expandafter\let\expandafter\x\csname ver@ltxcmds.sty\endcsname
  \ifx\x\relax % plain-TeX, first loading
  \else
    \def\empty{}%
    \ifx\x\empty % LaTeX, first loading,
      % variable is initialized, but \ProvidesPackage not yet seen
    \else
      \expandafter\ifx\csname PackageInfo\endcsname\relax
        \def\x#1#2{%
          \immediate\write-1{Package #1 Info: #2.}%
        }%
      \else
        \def\x#1#2{\PackageInfo{#1}{#2, stopped}}%
      \fi
      \x{ltxcmds}{The package is already loaded}%
      \aftergroup\endinput
    \fi
  \fi
\endgroup%
%    \end{macrocode}
%    Package identification:
%    \begin{macrocode}
\begingroup\catcode61\catcode48\catcode32=10\relax%
  \catcode13=5 % ^^M
  \endlinechar=13 %
  \catcode35=6 % #
  \catcode39=12 % '
  \catcode40=12 % (
  \catcode41=12 % )
  \catcode44=12 % ,
  \catcode45=12 % -
  \catcode46=12 % .
  \catcode47=12 % /
  \catcode58=12 % :
  \catcode64=11 % @
  \catcode91=12 % [
  \catcode93=12 % ]
  \catcode123=1 % {
  \catcode125=2 % }
  \expandafter\ifx\csname ProvidesPackage\endcsname\relax
    \def\x#1#2#3[#4]{\endgroup
      \immediate\write-1{Package: #3 #4}%
      \xdef#1{#4}%
    }%
  \else
    \def\x#1#2[#3]{\endgroup
      #2[{#3}]%
      \ifx#1\@undefined
        \xdef#1{#3}%
      \fi
      \ifx#1\relax
        \xdef#1{#3}%
      \fi
    }%
  \fi
\expandafter\x\csname ver@ltxcmds.sty\endcsname
\ProvidesPackage{ltxcmds}%
  [2019/12/15 v1.24 LaTeX kernel commands for general use (HO)]%
%    \end{macrocode}
%
%    \begin{macrocode}
\begingroup\catcode61\catcode48\catcode32=10\relax%
  \catcode13=5 % ^^M
  \endlinechar=13 %
  \catcode123=1 % {
  \catcode125=2 % }
  \catcode64=11 % @
  \def\x{\endgroup
    \expandafter\edef\csname LTXcmds@AtEnd\endcsname{%
      \endlinechar=\the\endlinechar\relax
      \catcode13=\the\catcode13\relax
      \catcode32=\the\catcode32\relax
      \catcode35=\the\catcode35\relax
      \catcode61=\the\catcode61\relax
      \catcode64=\the\catcode64\relax
      \catcode123=\the\catcode123\relax
      \catcode125=\the\catcode125\relax
    }%
  }%
\x\catcode61\catcode48\catcode32=10\relax%
\catcode13=5 % ^^M
\endlinechar=13 %
\catcode35=6 % #
\catcode64=11 % @
\catcode123=1 % {
\catcode125=2 % }
\def\TMP@EnsureCode#1#2{%
  \edef\LTXcmds@AtEnd{%
    \LTXcmds@AtEnd
    \catcode#1=\the\catcode#1\relax
  }%
  \catcode#1=#2\relax
}
\TMP@EnsureCode{36}{3}% $
\TMP@EnsureCode{38}{4}% &
\TMP@EnsureCode{40}{12}% (
\TMP@EnsureCode{41}{12}% )
\TMP@EnsureCode{45}{12}% -
\TMP@EnsureCode{46}{12}% .
\TMP@EnsureCode{47}{12}% /
\TMP@EnsureCode{60}{12}% <
\TMP@EnsureCode{62}{12}% >
\TMP@EnsureCode{91}{12}% [
\TMP@EnsureCode{96}{12}% `
\TMP@EnsureCode{93}{12}% ]
\TMP@EnsureCode{94}{12}% ^ (superscript) (!)
\TMP@EnsureCode{124}{12}% |
\edef\LTXcmds@AtEnd{\LTXcmds@AtEnd\noexpand\endinput}
%    \end{macrocode}
%
% \subsection{Numbers}
%
%    \begin{macro}{\ltx@zero}
%    \begin{macrocode}
\chardef\ltx@zero=0 %
%    \end{macrocode}
%    \end{macro}
%    \begin{macro}{\ltx@one}
%    \begin{macrocode}
\chardef\ltx@one=1 %
%    \end{macrocode}
%    \end{macro}
%    \begin{macro}{\ltx@two}
%    \begin{macrocode}
\chardef\ltx@two=2 %
%    \end{macrocode}
%    \end{macro}
%    \begin{macro}{\ltx@active}
%    \begin{macrocode}
\chardef\ltx@active=13 %
%    \end{macrocode}
%    \end{macro}
%    \begin{macro}{\ltx@cclv}
%    \begin{macrocode}
\chardef\ltx@cclv=255 %
%    \end{macrocode}
%    \end{macro}
%
%    \begin{macro}{\ltx@minusone}
%    \begin{macrocode}
\def\ltx@minusone{%
  -\ltx@one
}
%    \end{macrocode}
%    \end{macro}
%
% \subsection{Scratch registers}
%
%    \begin{macro}{\ltx@LocToksA}
%    \begin{macrocode}
\toksdef\ltx@LocToksA=0 %
%    \end{macrocode}
%    \end{macro}
%    \begin{macro}{\ltx@LocToksB}
%    \begin{macrocode}
\toksdef\ltx@LocToksB=2 %
%    \end{macrocode}
%    \end{macro}
%    \begin{macro}{\ltx@LocToksC}
%    \begin{macrocode}
\toksdef\ltx@LocToksC=4 %
%    \end{macrocode}
%    \end{macro}
%    \begin{macro}{\ltx@LocToksD}
%    \begin{macrocode}
\toksdef\ltx@LocToksD=6 %
%    \end{macrocode}
%    \end{macro}
%    \begin{macro}{\ltx@LocToksE}
%    \begin{macrocode}
\toksdef\ltx@LocToksE=8 %
%    \end{macrocode}
%    \end{macro}
%
%    \begin{macro}{\ltx@GlobToksA}
%    \begin{macrocode}
\toksdef\ltx@GlobToksA=1 %
%    \end{macrocode}
%    \end{macro}
%    \begin{macro}{\ltx@GlobToksB}
%    \begin{macrocode}
\toksdef\ltx@GlobToksB=3 %
%    \end{macrocode}
%    \end{macro}
%    \begin{macro}{\ltx@GlobToksC}
%    \begin{macrocode}
\toksdef\ltx@GlobToksC=5 %
%    \end{macrocode}
%    \end{macro}
%    \begin{macro}{\ltx@GlobToksD}
%    \begin{macrocode}
\toksdef\ltx@GlobToksD=7 %
%    \end{macrocode}
%    \end{macro}
%    \begin{macro}{\ltx@GlobToksE}
%    \begin{macrocode}
\toksdef\ltx@GlobToksE=9 %
%    \end{macrocode}
%    \end{macro}
%
%    \begin{macro}{\ltx@LocDimenA}
%    \begin{macrocode}
\dimendef\ltx@LocDimenA=0 %
%    \end{macrocode}
%    \end{macro}
%    \begin{macro}{\ltx@LocDimenB}
%    \begin{macrocode}
\dimendef\ltx@LocDimenB=2 %
%    \end{macrocode}
%    \end{macro}
%    \begin{macro}{\ltx@LocDimenC}
%    \begin{macrocode}
\dimendef\ltx@LocDimenC=4 %
%    \end{macrocode}
%    \end{macro}
%    \begin{macro}{\ltx@LocDimenD}
%    \begin{macrocode}
\dimendef\ltx@LocDimenD=6 %
%    \end{macrocode}
%    \end{macro}
%    \begin{macro}{\ltx@LocDimenE}
%    \begin{macrocode}
\dimendef\ltx@LocDimenE=8 %
%    \end{macrocode}
%    \end{macro}
%
%    \begin{macro}{\ltx@GlobDimenA}
%    \begin{macrocode}
\dimendef\ltx@GlobDimenA=1 %
%    \end{macrocode}
%    \end{macro}
%    \begin{macro}{\ltx@GlobDimenB}
%    \begin{macrocode}
\dimendef\ltx@GlobDimenB=3 %
%    \end{macrocode}
%    \end{macro}
%    \begin{macro}{\ltx@GlobDimenC}
%    \begin{macrocode}
\dimendef\ltx@GlobDimenC=5 %
%    \end{macrocode}
%    \end{macro}
%    \begin{macro}{\ltx@GlobDimenD}
%    \begin{macrocode}
\dimendef\ltx@GlobDimenD=7 %
%    \end{macrocode}
%    \end{macro}
%    \begin{macro}{\ltx@GlobDimenE}
%    \begin{macrocode}
\dimendef\ltx@GlobDimenE=9 %
%    \end{macrocode}
%    \end{macro}
%
%    \begin{macro}{\ltx@LocSkipA}
%    \begin{macrocode}
\skipdef\ltx@LocSkipA=0 %
%    \end{macrocode}
%    \end{macro}
%    \begin{macro}{\ltx@LocSkipB}
%    \begin{macrocode}
\skipdef\ltx@LocSkipB=2 %
%    \end{macrocode}
%    \end{macro}
%    \begin{macro}{\ltx@LocSkipC}
%    \begin{macrocode}
\skipdef\ltx@LocSkipC=4 %
%    \end{macrocode}
%    \end{macro}
%    \begin{macro}{\ltx@LocSkipD}
%    \begin{macrocode}
\skipdef\ltx@LocSkipD=6 %
%    \end{macrocode}
%    \end{macro}
%    \begin{macro}{\ltx@LocSkipE}
%    \begin{macrocode}
\skipdef\ltx@LocSkipE=8 %
%    \end{macrocode}
%    \end{macro}
%
%    \begin{macro}{\ltx@GlobSkipA}
%    \begin{macrocode}
\skipdef\ltx@GlobSkipA=1 %
%    \end{macrocode}
%    \end{macro}
%    \begin{macro}{\ltx@GlobSkipB}
%    \begin{macrocode}
\skipdef\ltx@GlobSkipB=3 %
%    \end{macrocode}
%    \end{macro}
%    \begin{macro}{\ltx@GlobSkipC}
%    \begin{macrocode}
\skipdef\ltx@GlobSkipC=5 %
%    \end{macrocode}
%    \end{macro}
%    \begin{macro}{\ltx@GlobSkipD}
%    \begin{macrocode}
\skipdef\ltx@GlobSkipD=7 %
%    \end{macrocode}
%    \end{macro}
%    \begin{macro}{\ltx@GlobSkipE}
%    \begin{macrocode}
\skipdef\ltx@GlobSkipE=9 %
%    \end{macrocode}
%    \end{macro}
%
%
% \subsection{Argument killers}
%
%    \begin{macro}{\ltx@gobble}
%    \begin{macrocode}
\long\def\ltx@gobble#1{}
%    \end{macrocode}
%    \end{macro}
%    \begin{macro}{\ltx@gobbletwo}
%    \begin{macrocode}
\long\def\ltx@gobbletwo#1#2{}
%    \end{macrocode}
%    \end{macro}
%    \begin{macro}{\ltx@gobblethree}
%    \begin{macrocode}
\long\def\ltx@gobblethree#1#2#3{}
%    \end{macrocode}
%    \end{macro}
%    \begin{macro}{\ltx@gobblefour}
%    \begin{macrocode}
\long\def\ltx@gobblefour#1#2#3#4{}
%    \end{macrocode}
%    \end{macro}
%
%    \begin{macro}{\ltx@GobbleNum}
%    \begin{macrocode}
\def\ltx@GobbleNum#1{%
  \romannumeral
  \csname ltx@zero%
  \expandafter\LTXcmds@GobbleNum
  \romannumeral\LTXcmds@num{#1}000{m\endcsname}%
}
%    \end{macrocode}
%    \end{macro}
%    \begin{macro}{\LTXcmds@GobbleNum}
%    \begin{macrocode}
\def\LTXcmds@GobbleNum#1{%
  \csname LTXcmds@G#1\LTXcmds@GobbleNum
}
%    \end{macrocode}
%    \end{macro}
%    \begin{macro}{\LTXcmds@Gm}
%    \begin{macrocode}
\long\def\LTXcmds@Gm#1{%
  \endcsname
}
%    \end{macrocode}
%    \end{macro}
%
% \subsection{Argument grabbers}
%
%    \begin{macro}{\ltx@firstofone}
%    \begin{macrocode}
\long\def\ltx@firstofone#1{#1}
%    \end{macrocode}
%    \end{macro}
%    \begin{macro}{\ltx@firstoftwo}
%    \begin{macrocode}
\long\def\ltx@firstoftwo#1#2{#1}
%    \end{macrocode}
%    \end{macro}
%    \begin{macro}{\ltx@secondoftwo}
%    \begin{macrocode}
\long\def\ltx@secondoftwo#1#2{#2}
%    \end{macrocode}
%    \end{macro}
%
%    \begin{macro}{\ltx@firstofthree}
%    \begin{macrocode}
\long\def\ltx@firstofthree#1#2#3{#1}
%    \end{macrocode}
%    \end{macro}
%    \begin{macro}{\ltx@secondofthree}
%    \begin{macrocode}
\long\def\ltx@secondofthree#1#2#3{#2}
%    \end{macrocode}
%    \end{macro}
%    \begin{macro}{\ltx@thirdofthree}
%    \begin{macrocode}
\long\def\ltx@thirdofthree#1#2#3{#3}%
%    \end{macrocode}
%    \end{macro}
%
%    \begin{macro}{\ltx@firstoffour}
%    \begin{macrocode}
\long\def\ltx@firstoffour#1#2#3#4{#1}
%    \end{macrocode}
%    \end{macro}
%    \begin{macro}{\ltx@secondoffour}
%    \begin{macrocode}
\long\def\ltx@secondoffour#1#2#3#4{#2}
%    \end{macrocode}
%    \end{macro}
%    \begin{macro}{\ltx@thirdoffour}
%    \begin{macrocode}
\long\def\ltx@thirdoffour#1#2#3#4{#3}%
%    \end{macrocode}
%    \end{macro}
%    \begin{macro}{\ltx@fourthoffour}
%    \begin{macrocode}
\long\def\ltx@fourthoffour#1#2#3#4{#4}%
%    \end{macrocode}
%    \end{macro}
%
% \subsection{List helpers}
%
%    \begin{macro}{\ltx@carzero}
%    \begin{macrocode}
\long\def\ltx@carzero#1\@nil{}%
%    \end{macrocode}
%    \end{macro}
%    \begin{macro}{\LTXcmds@cdrzero}
%    \begin{macrocode}
\long\def\LTXcmds@cdrzero#1\@nil{#1}
%    \end{macrocode}
%    \end{macro}
%    \begin{macro}{\ltx@cdrzero}
%    \begin{macrocode}
\def\ltx@cdrzero{%
  \romannumeral\LTXcmds@cdrzero\ltx@zero
}
%    \end{macrocode}
%    \end{macro}
%
%    \begin{macro}{\ltx@car}
%    \begin{macrocode}
\long\def\ltx@car#1#2\@nil{#1}
%    \end{macrocode}
%    \end{macro}
%    \begin{macro}{\ltx@cdr}
%    \begin{macrocode}
\long\def\ltx@cdr#1{%
  \romannumeral\LTXcmds@cdrzero\ltx@zero
}
%    \end{macrocode}
%    \end{macro}
%
%    \begin{macro}{\ltx@cartwo}
%    \begin{macrocode}
\long\def\ltx@cartwo#1#2#3\@nil{#1#2}
%    \end{macrocode}
%    \end{macro}
%    \begin{macro}{\ltx@carsecond}
%    \begin{macrocode}
\long\def\ltx@carsecond#1#2#3\@nil{#2}
%    \end{macrocode}
%    \end{macro}
%    \begin{macro}{\ltx@cdrtwo}
%    \begin{macrocode}
\long\def\ltx@cdrtwo#1#2{%
  \romannumeral\LTXcmds@cdrzero\ltx@zero
}
%    \end{macrocode}
%    \end{macro}
%
%    \begin{macro}{\ltx@carthree}
%    \begin{macrocode}
\long\def\ltx@carthree#1#2#3#4\@nil{#1#2#3}
%    \end{macrocode}
%    \end{macro}
%    \begin{macro}{\ltx@carthird}
%    \begin{macrocode}
\long\def\ltx@carthird#1#2#3#4\@nil{#3}
%    \end{macrocode}
%    \end{macro}
%    \begin{macro}{\ltx@cdrthree}
%    \begin{macrocode}
\long\def\ltx@cdrthree#1#2#3{%
  \romannumeral\LTXcmds@cdrzero\ltx@zero
}
%    \end{macrocode}
%    \end{macro}
%
%    \begin{macro}{\ltx@carfour}
%    \begin{macrocode}
\long\def\ltx@carfour#1#2#3#4#5\@nil{#1#2#3#4}
%    \end{macrocode}
%    \end{macro}
%    \begin{macro}{\ltx@carfourth}
%    \begin{macrocode}
\long\def\ltx@carfourth#1#2#3#4#5\@nil{#4}
%    \end{macrocode}
%    \end{macro}
%    \begin{macro}{\ltx@cdrfour}
%    \begin{macrocode}
\long\def\ltx@cdrfour#1#2#3#4{%
  \romannumeral\LTXcmds@cdrzero\ltx@zero
}
%    \end{macrocode}
%    \end{macro}
%
%    \begin{macro}{\ltx@CarNum}
%    \begin{macrocode}
\def\ltx@CarNum#1{%
  \romannumeral
  \csname LTXcmds@CarNumFinish%
  \expandafter\LTXcmds@CarNum
  \romannumeral\LTXcmds@num{#1}000{x\endcsname}%
}
%    \end{macrocode}
%    \end{macro}
%    \begin{macro}{\LTXcmds@CarNum}
%    \begin{macrocode}
\def\LTXcmds@CarNum#1{%
  \csname LTXcmds@C#1\LTXcmds@CarNum
}
%    \end{macrocode}
%    \end{macro}
%    \begin{macro}{\LTXcmds@Cm}
%    \begin{macrocode}
\long\def\LTXcmds@Cm#1#2{%
  \endcsname{#1#2}%
}
%    \end{macrocode}
%    \end{macro}
%    \begin{macro}{\LTXcmds@Cx}
%    \begin{macrocode}
\def\LTXcmds@Cx#1{%
  \endcsname{}%
}
%    \end{macrocode}
%    \end{macro}
%    \begin{macro}{\LTXcmds@CarNumFinish}
%    \begin{macrocode}
\long\def\LTXcmds@CarNumFinish#1#2\@nil{%
  \ltx@zero
  #1%
}
%    \end{macrocode}
%    \end{macro}
%
%    \begin{macro}{\ltx@CarNumth}
%    \begin{macrocode}
\def\ltx@CarNumth#1{%
  \romannumeral
  \expandafter\expandafter\expandafter
  \LTXcmds@CarNumth
  \ltx@GobbleNum{#1}{}%
}
%    \end{macrocode}
%    \end{macro}
%    \begin{macro}{\LTXcmds@CarNumth}
%    \begin{macrocode}
\long\def\LTXcmds@CarNumth#1#2\@nil{%
  \ltx@zero
  #1%
}
%    \end{macrocode}
%    \end{macro}
%
%    \begin{macro}{\ltx@CdrNum}
%    \begin{macrocode}
\def\ltx@CdrNum#1{%
  \romannumeral%
  \expandafter\expandafter\expandafter\ltx@cdrzero
  \expandafter\expandafter\expandafter\ltx@zero
  \ltx@GobbleNum{#1}%
}
%    \end{macrocode}
%    \end{macro}
%
% \subsection{Tail recursion}
%
%    \begin{macro}{\ltx@ReturnAfterFi}
%    \begin{macrocode}
\long\def\ltx@ReturnAfterFi#1\fi{\fi#1}
%    \end{macrocode}
%    \end{macro}
%    \begin{macro}{\ltx@ReturnAfterElseFi}
%    \begin{macrocode}
\long\def\ltx@ReturnAfterElseFi#1\else#2\fi{\fi#1}
%    \end{macrocode}
%    \end{macro}
%
% \subsection{Empty macro}
%
%    \begin{macro}{\ltx@empty}
%    \begin{macrocode}
\def\ltx@empty{}
%    \end{macrocode}
%    \end{macro}
%
% \subsection{Characters}
%
%    \begin{macro}{\ltx@space}
%    \begin{macrocode}
\def\ltx@space{ }
%    \end{macrocode}
%    \end{macro}
%    \begin{macro}{\ltx@percentchar}
%    \begin{macrocode}
\begingroup
  \lccode`0=`\%\relax
\lowercase{\endgroup
  \def\ltx@percentchar{0}%
}
%    \end{macrocode}
%    \end{macro}
%    \begin{macro}{\ltx@backslashchar}
%    \begin{macrocode}
\begingroup
  \lccode`0=`\\\relax
\lowercase{\endgroup
  \def\ltx@backslashchar{0}%
}
%    \end{macrocode}
%    \end{macro}
%    \begin{macro}{\ltx@hashchar}
%    \begin{macrocode}
\begingroup
  \lccode`0=`\#\relax
\lowercase{\endgroup
  \def\ltx@hashchar{0}%
}
%    \end{macrocode}
%    \end{macro}
%    \begin{macro}{\ltx@leftbracechar}
%    \begin{macrocode}
\begingroup
  \lccode`0=`\{\relax
\lowercase{\endgroup
  \def\ltx@leftbracechar{0}%
}
%    \end{macrocode}
%    \end{macro}
%    \begin{macro}{\ltx@rightbracechar}
%    \begin{macrocode}
\begingroup
  \lccode`0=`\}\relax
\lowercase{\endgroup
  \def\ltx@rightbracechar{0}%
}
%    \end{macrocode}
%    \end{macro}
%
% \subsection{Boolean switch}
%
%    \begin{macro}{\ltx@newif}
%    \begin{macrocode}
\def\ltx@newif#1{%
  \begingroup
    \escapechar=-1 %
  \expandafter\endgroup
  \expandafter\LTXcmds@newif\string#1\@nil
}
%    \end{macrocode}
%    \end{macro}
%    \begin{macro}{\LTXcmds@newif}
%    \begin{macrocode}
\begingroup
  \escapechar=-1 %
\expandafter\endgroup
\expandafter\def\expandafter\LTXcmds@newif\string\if#1\@nil{%
  \expandafter\edef\csname#1true\endcsname{%
    \let
    \expandafter\noexpand\csname if#1\endcsname
    \noexpand\iftrue
  }%
  \expandafter\edef\csname#1false\endcsname{%
    \let
    \expandafter\noexpand\csname if#1\endcsname
    \noexpand\iffalse
  }%
  \csname#1false\endcsname
}
%    \end{macrocode}
%    \end{macro}
%
%    \begin{macro}{\ltx@newglobalif}
%    \begin{macrocode}
\def\ltx@newglobalif#1{%
  \begingroup
    \escapechar=-1 %
  \expandafter\endgroup
  \expandafter\LTXcmds@newglobalif\string#1\@nil
}
%    \end{macrocode}
%    \end{macro}
%    \begin{macro}{\LTXcmds@newglobalif}
%    \begin{macrocode}
\begingroup
  \escapechar=-1 %
\expandafter\endgroup
\expandafter
\def\expandafter\LTXcmds@newglobalif\string\if#1\@nil{%
  \expandafter\edef\csname#1true\endcsname{%
    \global\let
    \expandafter\noexpand\csname if#1\endcsname
    \noexpand\iftrue
  }%
  \expandafter\edef\csname#1false\endcsname{%
    \global\let
    \expandafter\noexpand\csname if#1\endcsname
    \noexpand\iffalse
  }%
  \csname#1false\endcsname
}
%    \end{macrocode}
%    \end{macro}
%
% \subsection{Command definitions}
%
%    \begin{macro}{\ltx@LocalExpandAfter}
%    \begin{macrocode}
\def\ltx@LocalExpandAfter{%
  \begingroup
    \expandafter\expandafter\expandafter
  \endgroup
  \expandafter
}
%    \end{macrocode}
%    \end{macro}
%
%    \begin{macrocode}
\ltx@LocalExpandAfter
\ifx\csname ifcsname\endcsname\relax
%    \end{macrocode}
%    \begin{macro}{\ltx@ifundefined}
%    \begin{macrocode}
  \def\ltx@ifundefined#1{%
    \expandafter\ifx\csname #1\endcsname\relax
      \expandafter\ltx@firstoftwo
    \else
      \expandafter\ltx@secondoftwo
    \fi
  }%
%    \end{macrocode}
%    \end{macro}
%    \begin{macro}{\ltx@IfUndefined}
%    \begin{macrocode}
  \def\ltx@IfUndefined#1{%
    \begingroup\expandafter\expandafter\expandafter\endgroup
    \expandafter\ifx\csname #1\endcsname\relax
      \expandafter\ltx@firstoftwo
    \else
      \expandafter\ltx@secondoftwo
    \fi
  }%
%    \end{macrocode}
%    \end{macro}
%    \begin{macrocode}
  \expandafter\ltx@gobble
\else
  \expandafter\ltx@firstofone
\fi
{%
%    \end{macrocode}
%    \begin{macro}{\ltx@ifundefined}
%    \begin{macrocode}
  \def\ltx@ifundefined#1{%
    \ifcsname #1\endcsname
      \expandafter\ifx\csname #1\endcsname\relax
        \expandafter\expandafter\expandafter\ltx@firstoftwo
      \else
        \expandafter\expandafter\expandafter\ltx@secondoftwo
      \fi
    \else
      \expandafter\ltx@firstoftwo
    \fi
  }%
%    \end{macrocode}
%    \end{macro}
%    \begin{macro}{\ltx@IfUndefined}
%    \begin{macrocode}
  \let\ltx@IfUndefined\ltx@ifundefined
%    \end{macrocode}
%    \end{macro}
%    \begin{macrocode}
}
%    \end{macrocode}
%
% \subsection{Stripping}
%
%    \begin{macro}{\ltx@RemovePrefix}
%    \begin{macrocode}
\def\ltx@RemovePrefix#1>{}
%    \end{macrocode}
%    \end{macro}
%
%    \begin{macro}{\ltx@StripPrefix}
%    \begin{macrocode}
\def\ltx@StripPrefix{%
  \expandafter\ltx@RemovePrefix
}
%    \end{macrocode}
%    \end{macro}
%
%    \begin{macro}{\ltx@onelevel@sanitize}
%    \begin{macrocode}
\def\ltx@onelevel@sanitize#1{%
  \edef#1{%
    \expandafter
    \ltx@RemovePrefix\meaning#1%
  }%
}
%    \end{macrocode}
%    \end{macro}
%
% \subsection{File management}
%
% \subsubsection{File extensions}
%
%    \begin{macro}{\ltx@clsextension}
%    \begin{macrocode}
\def\ltx@clsextension{cls}
%    \end{macrocode}
%    \end{macro}
%    \begin{macro}{\ltx@pkgextension}
%    \begin{macrocode}
\def\ltx@pkgextension{sty}
%    \end{macrocode}
%    \end{macro}
%
% \subsubsection{Load check}
%
%    \begin{macro}{\ltx@iffileloaded}
%    \begin{macrocode}
\def\ltx@iffileloaded#1{%
  \ltx@ifundefined{ver@#1}\ltx@secondoftwo\ltx@firstoftwo
}
%    \end{macrocode}
%    \end{macro}
%    \begin{macro}{\ltx@ifclassloaded}
%    \begin{macrocode}
\def\ltx@ifclassloaded#1{%
  \ltx@iffileloaded{#1.\ltx@clsextension}%
}
%    \end{macrocode}
%    \end{macro}
%    \begin{macro}{\ltx@ifpackageloaded}
%    \begin{macrocode}
\def\ltx@ifpackageloaded#1{%
  \ltx@iffileloaded{#1.\ltx@pkgextension}%
}
%    \end{macrocode}
%    \end{macro}
%
% \subsubsection{Version date check}
%
%    \begin{macro}{\ltx@iffilelater}
%    \begin{macrocode}
\def\ltx@iffilelater#1#2{%
  \ltx@iffileloaded{#1}{%
    \expandafter\LTXcmds@IfLater\expandafter{%
      \number
      \expandafter\expandafter\expandafter\LTXcmds@ParseVersion
      \expandafter\expandafter\expandafter{%
        \csname ver@#1\endcsname
      }%
    \expandafter}\expandafter{%
      \number
      \expandafter\LTXcmds@ParseVersion\expandafter{#2}%
    }%
  }\ltx@secondoftwo
}
%    \end{macrocode}
%    \end{macro}
%    \begin{macro}{\LTXcmds@IfLater}
%    \begin{macrocode}
\def\LTXcmds@IfLater#1#2{%
  \ifcase 0%
      \ifnum#1<19940101 %
      \else
        \ifnum#2<19940101 %
        \else
          \ifnum#2>#1 %
          \else
            1%
          \fi
        \fi
      \fi
      \ltx@space
    \expandafter\ltx@secondoftwo
  \else
    \expandafter\ltx@firstoftwo
  \fi
}
%    \end{macrocode}
%    \end{macro}
%    \begin{macro}{\ltx@ifclasslater}
%    \begin{macrocode}
\def\ltx@ifclasslater#1{%
  \ltx@iffilelater{#1.\ltx@clsextension}%
}
%    \end{macrocode}
%    \end{macro}
%    \begin{macro}{\ltx@ifpackagelater}
%    \begin{macrocode}
\def\ltx@ifpackagelater#1{%
  \ltx@iffilelater{#1.\ltx@pkgextension}%
}
%    \end{macrocode}
%    \end{macro}
%
%    \begin{macrocode}
\ltx@IfUndefined{pdfmatch}{%
%    \end{macrocode}
%    \begin{macro}{\LTXcmds@ParseVersion}
%    \begin{macrocode}
  \def\LTXcmds@ParseVersion#1{%
    \LTXcmds@@ParseVersion#10000/00/00\@nil
  }%
%    \end{macrocode}
%    \end{macro}
%    \begin{macro}{\LTXcmds@@ParseVersion}
%    \begin{macrocode}
  \def\LTXcmds@@ParseVersion#1#2#3#4/#5#6/#7#8#9\@nil{%
    #1#2#3#4#5#6#7#8%
  }%
%    \end{macrocode}
%    \end{macro}
%    \begin{macrocode}
}{%
%    \end{macrocode}
%    \begin{macro}{\LTXcmds@ParseVersion}
%    \begin{macrocode}
  \def\LTXcmds@ParseVersion#1{%
    \ifnum\pdfmatch{%
      ^%
      (199[4-9]|[2-9][0-9][0-9][0-9])/%
      (0[1-9]|1[0-2])/%
      (0[1-9]|[1-2][0-9]|3[0-1])%
    }{#1}=1 %
      \ltx@StripPrefix\pdflastmatch1 %
      \ltx@StripPrefix\pdflastmatch2 %
      \ltx@StripPrefix\pdflastmatch3 %
    \else
      0%
    \fi
  }%
%    \end{macrocode}
%    \end{macro}
%    \begin{macrocode}
}
%    \end{macrocode}
%
% \subsection{Macro additions}
%
%    \begin{macro}{\ltx@GlobalAppendToMacro}
%    \begin{macrocode}
\long\def\ltx@GlobalAppendToMacro#1#2{%
  \ifx\ltx@undefined#1%
    \let#1\ltx@empty
  \else
    \ifx\relax#1%
      \let#1\ltx@empty
    \fi
  \fi
  \begingroup
    \ltx@LocToksA\expandafter{#1#2}%
    \xdef#1{\the\ltx@LocToksA}%
  \endgroup
}
%    \end{macrocode}
%    \end{macro}
%    \begin{macro}{\ltx@LocalAppendToMacro}
%    \begin{macrocode}
\long\def\ltx@LocalAppendToMacro#1#2{%
  \global\let\LTXcmds@gtemp#1%
  \ifx\ltx@undefined\LTXcmds@gtemp
    \global\let\LTXcmds@gtemp\ltx@empty
  \else
    \ifx\relax\LTXcmds@gtemp
      \global\letLTXcmds@gtemp\ltx@empty
    \fi
  \fi
  \begingroup
    \ltx@LocToksA\expandafter{\LTXcmds@gtemp#2}%
    \xdef\LTXcmds@gtemp{\the\ltx@LocToksA}%
  \endgroup
  \let#1\LTXcmds@gtemp
}
%    \end{macrocode}
%    \end{macro}
%
%    \begin{macro}{\ltx@GlobalPrependToMacro}
%    \begin{macrocode}
\long\def\ltx@GlobalPrependToMacro#1#2{%
  \ifx\ltx@undefined#1%
    \let#1\ltx@empty
  \else
    \ifx\relax#1%
      \let#1\ltx@empty
    \fi
  \fi
  \begingroup
    \ltx@LocToksA{#2}%
    \ltx@LocToksB\expandafter{#1}%
    \xdef#1{\the\ltx@LocToksA\the\ltx@LocToksB}%
  \endgroup
}
%    \end{macrocode}
%    \end{macro}
%    \begin{macro}{\ltx@LocalPrependToMacro}
%    \begin{macrocode}
\long\def\ltx@LocalPrependToMacro#1#2{%
  \global\let\LTXcmds@gtemp#1%
  \ifx\ltx@undefined\LTXcmds@gtemp
    \global\let\LTXcmds@gtemp\ltx@empty
  \else
    \ifx\relax\LTXcmds@gtemp
      \global\letLTXcmds@gtemp\ltx@empty
    \fi
  \fi
  \begingroup
    \ltx@LocToksA{#2}%
    \ltx@LocToksB\expandafter{\LTXcmds@gtemp}%
    \xdef\LTXcmds@gtemp{\the\ltx@LocToksA\the\ltx@LocToksB}%
  \endgroup
  \let#1\LTXcmds@gtemp
}
%    \end{macrocode}
%    \end{macro}
%
% \subsection{Next character detection}
%
%    \begin{macro}{\ltx@ifnextchar}
%    \begin{macrocode}
\long\def\ltx@ifnextchar#1#2#3{%
  \begingroup
  \let\LTXcmds@CharToken= #1\relax
  \ltx@LocToksA{\endgroup#2}%
  \ltx@LocToksB{\endgroup#3}%
  \futurelet\LTXcmds@LetToken\LTXcmds@ifnextchar
}
%    \end{macrocode}
%    \end{macro}
%    \begin{macro}{\LTXcmds@ifnextchar}
%    \begin{macrocode}
\def\LTXcmds@ifnextchar{%
  \ifx\LTXcmds@LetToken\LTXcmds@CharToken
    \the\expandafter\ltx@LocToksA
  \else
    \expandafter
      \ifx\csname LTXcmds@LetToken\endcsname\LTXcmds@SpaceToken
      \expandafter\expandafter\expandafter\LTXcmds@@ifnextchar
    \else
      \the\expandafter\expandafter\expandafter\ltx@LocToksB
    \fi
  \fi
}
%    \end{macrocode}
%    \end{macro}
%    \begin{macro}{\LTXcmds@@ifnextchar}
%    \cs{futurelet} does not distinguish between a
%    character and a command that is a character
%    (defined by using \cs{let} or \cs{futurelet}).
%    Therefore the space is catched by \cs{romannumeral}
%    with negative character constant that gobbles one
%    optional space.
%    \begin{macrocode}
\def\LTXcmds@@ifnextchar{%
  \expandafter\futurelet
  \expandafter\LTXcmds@LetToken
  \expandafter\LTXcmds@ifnextchar
  \romannumeral-`\.%
}
%    \end{macrocode}
%    \end{macro}
%    \begin{macro}{\LTXcmds@SpaceToken}
%    \begin{macrocode}
\ltx@firstofone{\let\LTXcmds@SpaceToken= } %
%    \end{macrocode}
%    \end{macro}
%
%    \begin{macro}{\ltx@ifnextchar@nospace}
%    \begin{macrocode}
\long\def\ltx@ifnextchar@nospace#1#2#3{%
  \begingroup
  \let\LTXcmds@CharToken= #1\relax
  \ltx@LocToksA{\endgroup#2}%
  \ltx@LocToksB{\endgroup#3}%
  \futurelet\LTXcmds@LetToken\LTXcmds@ifnextchar@nospace
}
%    \end{macrocode}
%    \end{macro}
%    \begin{macro}{\LTXcmds@ifnextchar@nospace}
%    \begin{macrocode}
\def\LTXcmds@ifnextchar@nospace{%
  \the
  \ifx\LTXcmds@LetToken\LTXcmds@CharToken
    \expandafter\ltx@LocToksA
  \else
    \expandafter\ltx@LocToksB
  \fi
}
%    \end{macrocode}
%    \end{macro}
% \subsection{\cs{ltx@leavevmode}, \cs{ltx@mbox}}
%
%    \begin{macro}{\ltx@leavevmode}
%    \begin{macrocode}
\ltx@IfUndefined{quitvmode}{%
  \ltx@IfUndefined{leavevmode}{%
    \ltx@IfUndefined{voidb@x}{%
      \ltx@IfUndefined{newbox}{%
        \def\ltx@leavevmode{%
          \begingroup
            \setbox\ltx@zero=\hbox{}%
            \begingroup
              \setbox\ltx@zero=\hbox{\box\ltx@zero}%
            \endgroup
            \unhbox\ltx@zero
          \endgroup
        }%
      }{%
        \csname newbox\endcsname\LTXcmds@VoidBox
        \ifvoid\LTXcmds@VoidBox
        \else
          \setbox\LTXcmds@VoidBox=\hbox{}%
          \begingroup
            \setbox\LTXcmds@VoidBox=\hbox{\box\LTXcmds@VoidBox}%
          \endgroup
        \fi
        \def\ltx@leavevmode{\unhbox\LTXcmds@VoidBox}%
      }%
    }{%
      \def\ltx@leavevmode{\unhbox\voidb@x}%
    }%
  }{%
    \let\ltx@leavevmode\leavevmode
  }%
}{%
  \let\ltx@leavevmode\quitvmode
}
%    \end{macrocode}
%    \end{macro}
%
%    \begin{macro}{\ltx@mbox}
%    \begin{macrocode}
\def\ltx@mbox{%
  \ltx@leavevmode
  \hbox
}
%    \end{macrocode}
%    \end{macro}
%
% \subsection{Help macros}
%
%    \begin{macro}{\LTXcmds@num}
%    \begin{macrocode}
\ltx@IfUndefined{numexpr}{%
  \def\LTXcmds@num#1{%
    \expandafter\ltx@firstofone\expandafter{%
      \number#1%
    }%
  }%
}{%
  \def\LTXcmds@num#1{%
    \expandafter\ltx@firstofone\expandafter{%
      \the\numexpr#1%
    }%
  }%
}
%    \end{macrocode}
%    \end{macro}
%
% \subsection{Expandable test for emptiness}
%
%    \begin{macrocode}
\ltx@IfUndefined{detokenize}{%
%    \end{macrocode}
%
% \subsubsection{Vanilla \hologo{TeX}}
%
%    \begin{macro}{\ltx@ifempty}
%    The macro is based on \cs{@ifempty} of Robert R. Schneck
%    \cite{ifempty:schneck}
%    and \cs{@ifnull} of Ulrich Diez \cite{ifnull:diez}.
%    There are three cases to consider:
%    \begin{compactenum}
%    \item |#1| is empty,
%    \item |#1| is not empty and the first token
%      is not a begingroup character,
%    \item |#1| starts with a begingroup character (catcode 1).
%    \end{compactenum}
%    \begin{macrocode}
  \def\LTXcmds@temp#1{%
    \long\def\ltx@ifempty##1{%
      \romannumeral0%
      \iffalse{\fi
        \expandafter\ltx@gobble\expandafter{%
          \expandafter{\string##1}%
          \expandafter\ltx@gobble\string
        }%
        \expandafter\ltx@firstofthree\expandafter
        {\iffalse}\fi
        \expandafter#1\ltx@secondoftwo
      }%
      \expandafter#1\ltx@firstoftwo
    }%
%    \end{macrocode}
%    \end{macro}
%    \begin{macro}{\ltx@ifblank}
%    \begin{macrocode}
    \long\def\ltx@ifblank##1{%
      \romannumeral0%
      \iffalse{\fi
        \expandafter\expandafter\expandafter\ltx@gobble
        \expandafter\expandafter\expandafter{%
          \expandafter\expandafter\expandafter{%
            \expandafter\string\ltx@gobble##1.%
          }%
          \expandafter\ltx@gobble\string
        }%
        \expandafter\ltx@firstofthree\expandafter
        {\iffalse}\fi
        \expandafter#1\ltx@secondoftwo
      }%
      \expandafter#1\ltx@firstoftwo
    }%
  }%
  \LTXcmds@temp{ }%
%    \end{macrocode}
%    \end{macro}
%    \begin{macrocode}
}{%
%    \end{macrocode}
%
% \subsubsection{With \cs{detokenize}}
%
%    Ahmed Musa provided \cs{ifstrempty} using \cs{detokenize}
%    and \cs{pdfstrcmp} \cite{ifstrempty:musa}. Ulrich Diez, GL,
%    Heiko Oberdiek
%    improved it further by removing \cs{pdfstrcmp} and
%    taking three arguments
%    \cite{ifstrempty:diez1,ifstrempty:diez2,ifstrempty:diez3,^^A
%      ifstrempty:gl,ifstrempty:ho}.
%
%    \begin{macro}{\ltx@ifempty}
%    \begin{macrocode}
  \long\def\ltx@ifempty#1{%
    \romannumeral%
    \csname
      LTXcmds@ifempty%
      \ifcat$\detokenize{#1}$%
        @%
      \fi
    \endcsname
  }%
%    \end{macrocode}
%    \end{macro}
%    \begin{macro}{\LTXcmds@ifempty@}
%    \begin{macrocode}
  \long\def\LTXcmds@ifempty@#1#2{0 #1}%
%    \end{macrocode}
%    \end{macro}
%    \begin{macro}{\LTXcmds@ifempty}
%    \begin{macrocode}
  \long\def\LTXcmds@ifempty#1#2{0 #2}%
%    \end{macrocode}
%    \end{macro}
%
% \subsubsection{\cs{ltx@ifblank}}
%
%    \begin{macro}{\ltx@ifblank}
%    \begin{macrocode}
  \long\def\ltx@ifblank#1{%
    \romannumeral%
    \csname
      LTXcmds@ifempty%
      \ifcat$\detokenize\expandafter{\ltx@gobble#1.}$%
        @%
      \fi
    \endcsname
  }%
%    \end{macrocode}
%    \end{macro}
%
%    \begin{macrocode}
}
%    \end{macrocode}
%
% \subsection{\cs{ltx@zapspace}}
%
%    \begin{macro}{\ltx@zapspace}
%    \begin{macrocode}
\long\def\ltx@zapspace#1{%
  \romannumeral
  \LTXcmds@zapspace\ltx@zero#1 \@nil
}
%    \end{macrocode}
%    \end{macro}
%    \begin{macro}{\LTXcmds@zapspace}
%    \begin{macrocode}
\long\def\LTXcmds@zapspace#1 #2\@nil{%
  \ltx@ifempty{#2}{%
    #1%
  }{%
    \LTXcmds@zapspace#1#2\@nil
  }%
}
%    \end{macrocode}
%    \end{macro}
%
% \subsection{\cs{ltx@IfBoxEmpty}}
%
%    In case of \hologo{eTeX} the test for an empty box
%    is done via \cs{lastnodetype} as suggested
%    by David Kastrup \cite{ifboxempty:dak}.
%    \begin{macrocode}
\ltx@IfUndefined{lastnodetype}{%
  \catcode`\$=9 %
  \catcode`\&=14 %
}{%
  \catcode`\$=14 %
  \catcode`\&=9 %
}
%    \end{macrocode}
%    \begin{macro}{\ltx@IfBoxEmpty}
%    \begin{macrocode}
\def\ltx@IfBoxEmpty#1{%
  \ifvoid#1\relax
    \expandafter\ltx@secondoftwo
  \else
%    \end{macrocode}
%    Implementation using \hologo{eTeX}'s \cs{lastnodetype}.
%    \begin{macrocode}
&   \begingroup
&     \setbox\ltx@zero=\ifhbox#1\hbox\else\vbox\fi{%
&       \ifhmode\unhcopy\else\unvcopy\fi#1\relax
&       \expandafter
&     }%
&   \expandafter\endgroup
&   \ifnum\lastnodetype<\ltx@zero
&     \expandafter\expandafter\expandafter\ltx@firstoftwo
&   \else
&     \expandafter\expandafter\expandafter\ltx@secondoftwo
&   \fi
%    \end{macrocode}
%    Implementation without \hologo{eTeX} using a signature
%    at the beginning of the test box.
%    \begin{macrocode}
$   \begingroup
$     \setbox\ltx@zero=\ifhbox#1\hbox\else\vbox\fi{%
$       \penalty\ltx@one
$       \ifhmode\unhcopy\else\unvcopy\fi#1\relax
$       \expandafter
$     }%
$     \ifnum\lastpenalty=\ltx@one
%    \end{macrocode}
%    Box 0 has been changed and is restored by
%    closing the group.
%    \begin{macrocode}
$       \endgroup
$       \begingroup
$       \setbox\ltx@zero=\ifhbox#1\hbox\else\vbox\fi{%
$         \penalty\ltx@two
$         \ifhmode\unhcopy\else\unvcopy\fi#1\relax
$         \expandafter
$       }%
$       \ifnum\lastpenalty=\ltx@two
$         \def\next{\endgroup\expandafter\ltx@firstoftwo}%
$       \else
$         \def\next{\endgroup\expandafter\ltx@secondoftwo}%
$       \fi
$     \else
$       \def\next{\endgroup\expandafter\ltx@secondoftwo}%
$     \fi
$   \next
  \fi
}
%    \end{macrocode}
%    \end{macro}
%
%    \begin{macro}{\ltx@IfBoxVoidOrEmpty}
%    \begin{macrocode}
\def\ltx@IfBoxVoidOrEmpty#1{%
  \ifvoid#1\relax
    \expandafter\ltx@thirdoffour
  \fi
  \ltx@IfBoxEmpty{#1}%
}
%    \end{macrocode}
%    \end{macro}
%
%    \begin{macrocode}
\LTXcmds@AtEnd%
%</package>
%    \end{macrocode}
%% \section{Installation}
%
% \subsection{Download}
%
% \paragraph{Package.} This package is available on
% CTAN\footnote{\CTANpkg{ltxcmds}}:
% \begin{description}
% \item[\CTAN{macros/latex/contrib/ltxcmds/ltxcmds.dtx}] The source file.
% \item[\CTAN{macros/latex/contrib/ltxcmds/ltxcmds.pdf}] Documentation.
% \end{description}
%
%
% \paragraph{Bundle.} All the packages of the bundle `ltxcmds'
% are also available in a TDS compliant ZIP archive. There
% the packages are already unpacked and the documentation files
% are generated. The files and directories obey the TDS standard.
% \begin{description}
% \item[\CTANinstall{install/macros/latex/contrib/ltxcmds.tds.zip}]
% \end{description}
% \emph{TDS} refers to the standard ``A Directory Structure
% for \TeX\ Files'' (\CTANpkg{tds}). Directories
% with \xfile{texmf} in their name are usually organized this way.
%
% \subsection{Bundle installation}
%
% \paragraph{Unpacking.} Unpack the \xfile{ltxcmds.tds.zip} in the
% TDS tree (also known as \xfile{texmf} tree) of your choice.
% Example (linux):
% \begin{quote}
%   |unzip ltxcmds.tds.zip -d ~/texmf|
% \end{quote}
%
% \subsection{Package installation}
%
% \paragraph{Unpacking.} The \xfile{.dtx} file is a self-extracting
% \docstrip\ archive. The files are extracted by running the
% \xfile{.dtx} through \plainTeX:
% \begin{quote}
%   \verb|tex ltxcmds.dtx|
% \end{quote}
%
% \paragraph{TDS.} Now the different files must be moved into
% the different directories in your installation TDS tree
% (also known as \xfile{texmf} tree):
% \begin{quote}
% \def\t{^^A
% \begin{tabular}{@{}>{\ttfamily}l@{ $\rightarrow$ }>{\ttfamily}l@{}}
%   ltxcmds.sty & tex/generic/ltxcmds/ltxcmds.sty\\
%   ltxcmds.pdf & doc/latex/ltxcmds/ltxcmds.pdf\\
%   ltxcmds.dtx & source/latex/ltxcmds/ltxcmds.dtx\\
% \end{tabular}^^A
% }^^A
% \sbox0{\t}^^A
% \ifdim\wd0>\linewidth
%   \begingroup
%     \advance\linewidth by\leftmargin
%     \advance\linewidth by\rightmargin
%   \edef\x{\endgroup
%     \def\noexpand\lw{\the\linewidth}^^A
%   }\x
%   \def\lwbox{^^A
%     \leavevmode
%     \hbox to \linewidth{^^A
%       \kern-\leftmargin\relax
%       \hss
%       \usebox0
%       \hss
%       \kern-\rightmargin\relax
%     }^^A
%   }^^A
%   \ifdim\wd0>\lw
%     \sbox0{\small\t}^^A
%     \ifdim\wd0>\linewidth
%       \ifdim\wd0>\lw
%         \sbox0{\footnotesize\t}^^A
%         \ifdim\wd0>\linewidth
%           \ifdim\wd0>\lw
%             \sbox0{\scriptsize\t}^^A
%             \ifdim\wd0>\linewidth
%               \ifdim\wd0>\lw
%                 \sbox0{\tiny\t}^^A
%                 \ifdim\wd0>\linewidth
%                   \lwbox
%                 \else
%                   \usebox0
%                 \fi
%               \else
%                 \lwbox
%               \fi
%             \else
%               \usebox0
%             \fi
%           \else
%             \lwbox
%           \fi
%         \else
%           \usebox0
%         \fi
%       \else
%         \lwbox
%       \fi
%     \else
%       \usebox0
%     \fi
%   \else
%     \lwbox
%   \fi
% \else
%   \usebox0
% \fi
% \end{quote}
% If you have a \xfile{docstrip.cfg} that configures and enables \docstrip's
% TDS installing feature, then some files can already be in the right
% place, see the documentation of \docstrip.
%
% \subsection{Refresh file name databases}
%
% If your \TeX~distribution
% (\TeX\,Live, \mikTeX, \dots) relies on file name databases, you must refresh
% these. For example, \TeX\,Live\ users run \verb|texhash| or
% \verb|mktexlsr|.
%
% \subsection{Some details for the interested}
%
% \paragraph{Unpacking with \LaTeX.}
% The \xfile{.dtx} chooses its action depending on the format:
% \begin{description}
% \item[\plainTeX:] Run \docstrip\ and extract the files.
% \item[\LaTeX:] Generate the documentation.
% \end{description}
% If you insist on using \LaTeX\ for \docstrip\ (really,
% \docstrip\ does not need \LaTeX), then inform the autodetect routine
% about your intention:
% \begin{quote}
%   \verb|latex \let\install=y% \iffalse meta-comment
%
% File: ltxcmds.dtx
% Version: 2019/12/15 v1.24
% Info: LaTeX kernel commands for general use
%
% Copyright (C)
%    2009-2011 Heiko Oberdiek
%    2016-2019 Oberdiek Package Support Group
%    https://github.com/ho-tex/ltxcmds/issues
%
% This work may be distributed and/or modified under the
% conditions of the LaTeX Project Public License, either
% version 1.3c of this license or (at your option) any later
% version. This version of this license is in
%    https://www.latex-project.org/lppl/lppl-1-3c.txt
% and the latest version of this license is in
%    https://www.latex-project.org/lppl.txt
% and version 1.3 or later is part of all distributions of
% LaTeX version 2005/12/01 or later.
%
% This work has the LPPL maintenance status "maintained".
%
% The Current Maintainers of this work are
% Heiko Oberdiek and the Oberdiek Package Support Group
% https://github.com/ho-tex/ltxcmds/issues
%
% The Base Interpreter refers to any `TeX-Format',
% because some files are installed in TDS:tex/generic//.
%
% This work consists of the main source file ltxcmds.dtx
% and the derived files
%    ltxcmds.sty, ltxcmds.pdf, ltxcmds.ins, ltxcmds.drv,
%    ltxcmds-test1.tex, ltxcmds-test-gobble.tex,
%    ltxcmds-test-ifempty.tex, ltxcmds-test-zapspace.tex,
%    ltxcmds-test-ifboxempty.tex, ltxcmds-test-nextchar.tex,
%    ltxcmds-test-carcdr.tex.
%
% Distribution:
%    CTAN:macros/latex/contrib/ltxcmds/ltxcmds.dtx
%    CTAN:macros/latex/contrib/ltxcmds/ltxcmds.pdf
%
% Unpacking:
%    (a) If ltxcmds.ins is present:
%           tex ltxcmds.ins
%    (b) Without ltxcmds.ins:
%           tex ltxcmds.dtx
%    (c) If you insist on using LaTeX
%           latex \let\install=y% \iffalse meta-comment
%
% File: ltxcmds.dtx
% Version: 2019/12/15 v1.24
% Info: LaTeX kernel commands for general use
%
% Copyright (C)
%    2009-2011 Heiko Oberdiek
%    2016-2019 Oberdiek Package Support Group
%    https://github.com/ho-tex/ltxcmds/issues
%
% This work may be distributed and/or modified under the
% conditions of the LaTeX Project Public License, either
% version 1.3c of this license or (at your option) any later
% version. This version of this license is in
%    https://www.latex-project.org/lppl/lppl-1-3c.txt
% and the latest version of this license is in
%    https://www.latex-project.org/lppl.txt
% and version 1.3 or later is part of all distributions of
% LaTeX version 2005/12/01 or later.
%
% This work has the LPPL maintenance status "maintained".
%
% The Current Maintainers of this work are
% Heiko Oberdiek and the Oberdiek Package Support Group
% https://github.com/ho-tex/ltxcmds/issues
%
% The Base Interpreter refers to any `TeX-Format',
% because some files are installed in TDS:tex/generic//.
%
% This work consists of the main source file ltxcmds.dtx
% and the derived files
%    ltxcmds.sty, ltxcmds.pdf, ltxcmds.ins, ltxcmds.drv,
%    ltxcmds-test1.tex, ltxcmds-test-gobble.tex,
%    ltxcmds-test-ifempty.tex, ltxcmds-test-zapspace.tex,
%    ltxcmds-test-ifboxempty.tex, ltxcmds-test-nextchar.tex,
%    ltxcmds-test-carcdr.tex.
%
% Distribution:
%    CTAN:macros/latex/contrib/ltxcmds/ltxcmds.dtx
%    CTAN:macros/latex/contrib/ltxcmds/ltxcmds.pdf
%
% Unpacking:
%    (a) If ltxcmds.ins is present:
%           tex ltxcmds.ins
%    (b) Without ltxcmds.ins:
%           tex ltxcmds.dtx
%    (c) If you insist on using LaTeX
%           latex \let\install=y\input{ltxcmds.dtx}
%        (quote the arguments according to the demands of your shell)
%
% Documentation:
%    (a) If ltxcmds.drv is present:
%           latex ltxcmds.drv
%    (b) Without ltxcmds.drv:
%           latex ltxcmds.dtx; ...
%    The class ltxdoc loads the configuration file ltxdoc.cfg
%    if available. Here you can specify further options, e.g.
%    use A4 as paper format:
%       \PassOptionsToClass{a4paper}{article}
%
%    Programm calls to get the documentation (example):
%       pdflatex ltxcmds.dtx
%       makeindex -s gind.ist ltxcmds.idx
%       pdflatex ltxcmds.dtx
%       makeindex -s gind.ist ltxcmds.idx
%       pdflatex ltxcmds.dtx
%
% Installation:
%    TDS:tex/generic/ltxcmds/ltxcmds.sty
%    TDS:doc/latex/ltxcmds/ltxcmds.pdf
%    TDS:source/latex/ltxcmds/ltxcmds.dtx
%
%<*ignore>
\begingroup
  \catcode123=1 %
  \catcode125=2 %
  \def\x{LaTeX2e}%
\expandafter\endgroup
\ifcase 0\ifx\install y1\fi\expandafter
         \ifx\csname processbatchFile\endcsname\relax\else1\fi
         \ifx\fmtname\x\else 1\fi\relax
\else\csname fi\endcsname
%</ignore>
%<*install>
\input docstrip.tex
\Msg{************************************************************************}
\Msg{* Installation}
\Msg{* Package: ltxcmds 2019/12/15 v1.24 LaTeX kernel commands for general use (HO)}
\Msg{************************************************************************}

\keepsilent
\askforoverwritefalse

\let\MetaPrefix\relax
\preamble

This is a generated file.

Project: ltxcmds
Version: 2019/12/15 v1.24

Copyright (C)
   2009-2011 Heiko Oberdiek
   2016-2019 Oberdiek Package Support Group

This work may be distributed and/or modified under the
conditions of the LaTeX Project Public License, either
version 1.3c of this license or (at your option) any later
version. This version of this license is in
   https://www.latex-project.org/lppl/lppl-1-3c.txt
and the latest version of this license is in
   https://www.latex-project.org/lppl.txt
and version 1.3 or later is part of all distributions of
LaTeX version 2005/12/01 or later.

This work has the LPPL maintenance status "maintained".

The Current Maintainers of this work are
Heiko Oberdiek and the Oberdiek Package Support Group
https://github.com/ho-tex/ltxcmds/issues


The Base Interpreter refers to any `TeX-Format',
because some files are installed in TDS:tex/generic//.

This work consists of the main source file ltxcmds.dtx
and the derived files
   ltxcmds.sty, ltxcmds.pdf, ltxcmds.ins, ltxcmds.drv,
   ltxcmds-test1.tex, ltxcmds-test-gobble.tex,
   ltxcmds-test-ifempty.tex, ltxcmds-test-zapspace.tex,
   ltxcmds-test-ifboxempty.tex, ltxcmds-test-nextchar.tex,
   ltxcmds-test-carcdr.tex.

\endpreamble
\let\MetaPrefix\DoubleperCent

\generate{%
  \file{ltxcmds.ins}{\from{ltxcmds.dtx}{install}}%
  \file{ltxcmds.drv}{\from{ltxcmds.dtx}{driver}}%
  \usedir{tex/generic/ltxcmds}%
  \file{ltxcmds.sty}{\from{ltxcmds.dtx}{package}}%
}

\catcode32=13\relax% active space
\let =\space%
\Msg{************************************************************************}
\Msg{*}
\Msg{* To finish the installation you have to move the following}
\Msg{* file into a directory searched by TeX:}
\Msg{*}
\Msg{*     ltxcmds.sty}
\Msg{*}
\Msg{* To produce the documentation run the file `ltxcmds.drv'}
\Msg{* through LaTeX.}
\Msg{*}
\Msg{* Happy TeXing!}
\Msg{*}
\Msg{************************************************************************}

\endbatchfile
%</install>
%<*ignore>
\fi
%</ignore>
%<*driver>
\NeedsTeXFormat{LaTeX2e}
\ProvidesFile{ltxcmds.drv}%
  [2019/12/15 v1.24 LaTeX kernel commands for general use (HO)]%
\documentclass{ltxdoc}
\usepackage{holtxdoc}[2011/11/22]
\usepackage{zref-savepos}
\usepackage{paralist}
\hfuzz=1pt
\begin{document}
  \DocInput{ltxcmds.dtx}%
\end{document}
%</driver>
% \fi
%
%
%
% \GetFileInfo{ltxcmds.drv}
%
% \title{The \xpackage{ltxcmds} package}
% \date{2019/12/15 v1.24}
% \author{Heiko Oberdiek\thanks
% {Please report any issues at \url{https://github.com/ho-tex/ltxcmds/issues}}}
%
% \maketitle
%
% \begin{abstract}
% The package \xpackage{ltxcmds} exports some utility macros
% from the \LaTeX\ kernel into a separate namespace and
% also provides them for other formats such as plain-\TeX.
% \end{abstract}
%
% \tableofcontents
%
% \newcounter{to}
% \renewcommand*{\theto}{to\arabic{to}}
% \newlength\tolength
% \tolength=\dimexpr .5\linewidth+\oddsidemargin+1in+15mm\relax
% \newcommand*{\TO}{^^A
%   \quad
%   \stepcounter{to}^^A
%   \zsavepos\theto
%   \ifdim\zposx\theto sp>0pt
%     \ifdim\zposx\theto sp<\tolength
%       \kern\dimexpr\tolength-\zposx\theto sp\relax
%     \fi
%   \fi
%   $\rightarrow$\quad
%   \ignorespaces
% }
%
% \section{Documentation}
%
% \subsection{Introduction}
%
% Many of my packages also support other formats such as plain-\TeX.
% Because I am rather familiar with the utility macros from
% \LaTeX's kernel (e.g. \cs{@gobble}, \cs{@firstoftwo}), I found
% myself rewriting them again and again, because they are lacking
% in plain-\TeX.
%
% Therefore this package provides often used macros and similar
% ones with the name prefix \cs{ltx@}. This avoids also faulty
% redefinitions. I remember an example where a package redefined
% \cs{@firstoftwo} with forgetting \cs{long}.
%
% \subsection{Numbers}
%
% \begin{declcs}{ltx@zero} \TO 0\\
%   \cs{ltx@one} \TO 1\\
%   \cs{ltx@two} \TO 2\\
%   \cs{ltx@cclv} \TO 255\\
%   \cs{ltx@minusone} \TO -1
% \end{declcs}
% These commands are numbers 0, 1, 2, 255 and -1. They are not digits
% and a space is not gobbled afterwards. Macro \cs{ltx@minusone}
% is available since version 2010/12/12 v1.15.
%
% \subsection{Scratch registers}
%
% Following the conventions of \hologo{plainTeX} and \hologo{LaTeX}
% the first ten registers are free to use. Even numbered registers
% are for local, odd numbered for global use.
%
% \begin{declcs}{ltx@(Loc,Glob)(Toks,Dimen,Skip)(A,B,C,D,E)}
% \end{declcs}
% The name consists of the prefix \cs{ltx@}, then
% \texttt{Loc} or \texttt{Glob} for local or global usage follows.
% The register type is given by \texttt{Toks} for token register,
% \texttt{Dimen} for dimen register and \texttt{Skip} for skip
% register. As last part the registers are numbered from \texttt{A}
% to \texttt{E}. Example: \cs{ltx@LocToksA}.
%
% Since 2011/04/14 v1.19.
%
% \subsection{Argument killers}
%
% \begin{declcs}{ltx@gobble} \M{1} \TO\\
%   \cs{ltx@gobbletwo} \M{1} \M{2} \TO\\
%   \cs{ltx@gobblethree} \M{1} \M{2} \M{3} \TO\\
%   \cs{ltx@gobblefour} \M{1} \M{2} \M{3} \M{4} \TO
% \end{declcs}
%
% \begin{declcs}{ltx@GobbleNum} \M{num} \M{1} \M{2} \dots\space
%   \M{\meta{num}} \TO
% \end{declcs}
% The first argument \meta{num} of macro \cs{ltx@GobbleNum} specifies,
% how many following arguments are eaten. Macro \cs{ltx@GobbleNum}
% is expandable in exact two expansion steps.
%
% \subsection{Argument grabbers}
%
% \begin{declcs}{ltx@firstofone} \M{1} \TO \meta{1}\\[1ex]
%  \cs{ltx@firstoftwo} \M{1} \M{2} \TO \meta{1}\\
%  \cs{ltx@secondoftwo} \M{1} \M{2} \TO \meta{2}\\[1ex]
%  \cs{ltx@firstofthree} \M{1} \M{2} \M{3} \TO \meta{1}\\
%  \cs{ltx@secondofthree} \M{1} \M{2} \M{3} \TO \meta{2}\\
%  \cs{ltx@thirdofthree} \M{1} \M{2} \M{3} \TO \meta{3}\\
%  \cs{ltx@firstoffour} \M{1} \M{2} \M{3} \M{4} \TO \meta{1}\\
%  \cs{ltx@secondoffour} \M{1} \M{2} \M{3} \M{4} \TO \meta{2}\\
%  \cs{ltx@thirdoffour} \M{1} \M{2} \M{3} \M{4} \TO \meta{3}\\
%  \cs{ltx@fourthoffour} \M{1} \M{2} \M{3} \M{4} \TO \meta{4}
% \end{declcs}
% Macros \cs{ltx@firstofthree}, \cs{ltx@secondofthree}
% and \cs{ltx@thirdofthree} were added in version
% 2010/11/12 v1.11.
% Macros \cs{ltx@firstoffour}, \dots, \cs{ltx@fourthoffour}
% were added in version 2011/02/04 v1.16.
%
% \subsection{List helpers}
%
% \begin{declcs}{ltx@carzero} \dots\ \cs{@nil} \TO\\
%   \cs{ltx@cdrzero} \dots\ \cs{@nil} \TO \dots
% \end{declcs}
%
% \begin{declcs}{ltx@car} \M{1} \dots\ \cs{@nil} \TO \meta{1}\\
%   \cs{ltx@cdr} \M{1} \dots\ \cs{@nil} \TO \dots
% \end{declcs}
%
% \begin{declcs}{ltx@cartwo} \M{1} \M{2} \dots\ \cs{@nil}
%   \TO \meta{1}\meta{2}\\
%   \cs{ltx@carsecond} \M{1} \M{2} \dots\ \cs{@nil} \TO \meta{2}\\
%   \cs{ltx@cdrtwo} \M{1} \M{2} \dots\ \cs{@nil} \TO \dots
% \end{declcs}
%
% \begin{declcs}{ltx@carthree} \M{1} \M{2} \M{3} \dots\ \cs{@nil}
%       \TO \meta{1}\meta{2}\meta{3}\\
%   \cs{ltx@carthird} \M{1} \M{2} \M{3} \dots\ \cs{@nil}
%       \TO \meta{3}\\
%   \cs{ltx@cdrthree} \M{1} \M{2} \M{3} \dots\ \cs{@nil} \TO \dots
% \end{declcs}
%
% \begin{declcs}{ltx@carfour} \M{1} \M{2} \M{3} \M{4} \dots\ \cs{@nil}
%       \TO \meta{1}\meta{2}\meta{3}\meta{4}\\
%   \cs{ltx@carfourth} \M{1} \M{2} \M{3} \M{4} \dots\ \cs{@nil}
%       \TO \meta{4}\\
%   \cs{ltx@cdrfour} \M{1} \M{2} \M{3} \M{4} \dots\ \cs{@nil} \TO \dots
% \end{declcs}
%
% \begin{declcs}{ltx@CarNum} \M{num} \M{1} \dots\ \M{\meta{num}}
%   \M{\meta{num}+1} \dots\ \cs{@nil}\\
%   \TO \M{1} \dots\ \M{\meta{num}} \dots\\
%   \cs{ltx@CarNumth} \M{num} \M{1} \dots\ \M{\meta{num}}
%   \M{\meta{num}+1} \dots\ \cs{@nil}\\
%   \TO \M{\meta{num}} \dots\\
%   \cs{ltx@CdrNum} \M{num} \M{1} \dots\ \M{\meta{num}}
%   \M{\meta{num}+1} \dots\ \cs{@nil}\\
%   \TO \M{\meta{num}+1} \dots
% \end{declcs}
%
% Macros with uppercase letters are expandable in two expansion steps.
% Changes in version 2019/12/15 v1.24:
% \begin{itemize}
% \item
%   Macros \cs{ltx@carsecond}, \cs{ltx@carthird}, \cs{ltx@carfourth},
%   \cs{ltx@CarNumth} added.
% \item
%   Macros \cs{ltx@cdr}, \cs{ltx@cdrtwo}, \cs{ltx@cdrthree}, \cs{ltx@cdrfour},
%   \cs{ltx@CdrNum} are expandable in two expansion steps and retain
%   spaces and braces after the first gobbled arguments.
% \end{itemize}
%
% \subsection{Tail recursion}
%
% \begin{declcs}{ltx@ReturnAfterFi} \M{1} \cs{fi} \TO \cs{fi} \meta{1}\\
%   \cs{ltx@ReturnAfterElseFi} \M{1} \cs{else} \M{2} \cs{fi}
%     \TO \cs{fi} \meta{1}
% \end{declcs}
%
% \subsection{Empty macro}
%
% \begin{declcs}{ltx@empty} \TO
% \end{declcs}
%
% \subsection{Characters}
%
% \begin{declcs}{ltx@space} \TO \verb*| |\\
%   \cs{ltx@percentchar} \TO \texttt{\%}\\
%   \cs{ltx@backslashchar} \TO
%      \makeatletter\texttt{\@backslashchar}\\
%   \cs{ltx@hashchar} \TO \texttt{\#}\qquad(since v1.7)\\
%   \cs{ltx@leftbracechar} \TO \texttt{\textbraceleft}\qquad(since v1.8)\\
%   \cs{ltx@rightbracechar} \TO \texttt{\textbraceright}\qquad(since v1.8)
% \end{declcs}
%
% \subsection{Boolean switch}
%
% \begin{declcs}{ltx@newif} \M{cmd}
% \end{declcs}
% \cs{ltx@newif} defines a new boolean switch \meta{cmd} like \cs{newif}.
% Unlike \plainTeX's \cs{newif}, \cs{ltx@newif} is not \cs{outer}.
% The command \meta{cmd} must start with the two characters |if|.
%
% \begin{declcs}{ltx@newglobalif} \M{cmd}
% \end{declcs}
% \cs{ltx@newglobalif} defines a new boolean switch \meta{cmd}
% like \cs{ltx@newif}. However the switch setting commands,
% \meta{cmd} without the prefix |if| and followed by |true| or |false|
% are acting globally.
%
% \subsection{Command definitions}
%
% \begin{declcs}{ltx@ifundefined} \M{cmd} \M{yes} \M{no}
% \end{declcs}
% If \eTeX\ is available, \cs{ifcsname} is used that does
% not have the side effect of defining undefined commands
% with meaning of \cs{relax}. This command is always expandable.
% Change in version 1.1: Also the meaning \cs{relax} is
% always considered ``undefined''.
%
% \begin{declcs}{ltx@IfUndefined} \M{cmd} \M{yes} \M{no}
% \end{declcs}
% If \eTeX\ is available, \cs{ifcsname} is used that does
% not have the side effect of defining undefined commands
% with meaning of \cs{relax}. Also it always checks for
% the meaning of \cs{relax} and considers this as undefined.
% This macro is not expandable without \eTeX.
%
% \begin{declcs}{ltx@LocalExpandAfter}
% \end{declcs}
% It expands the token after the next token but in a local
% context. That is the difference to \cs{expandafter}.
% The local context discards the side effect of \cs{csname}
% and let the command undefined after the expansion step.
%
% \subsection{Stripping}
%
% \begin{declcs}{ltx@RemovePrefix}\\
%   \cs{ltx@StripPrefix}
% \end{declcs}
% All tokens up to and including the next available character
% `\textgreater' are thrown away. Usually it is used to strip
% the first part of the output of the commands \cs{meaning}
% or \cs{pdflastmatch}. Macro \cs{ltx@RemovePrefix}
% has the same meaning as \LaTeX's \cs{strip@prefix}, whereas
% macro \cs{ltx@StripPrefix} expands the next token
% once before stripping the prefix.
%
% \begin{declcs}{ltx@onelevel@sanitize} \M{macro}
% \end{declcs}
% Macro \cs{ltx@onelevel@sanitize} provides \hologo{LaTeX}'s
% \cs{@onelevel@sanitize}. The macro is expanded once and
% the contents is converted to characters with catcode 12 (other)
% and space tokens with catcode 10 (space). Then then sanitized
% contents is stored into the macro again. Since version 1.12.
%
% \subsection{File management}
%
% All macros in this section are expandable like
% the counterparts of the \LaTeX\ kernel. Also they
% can be used after the preamble.
%
% \subsubsection{File extensions}
%
% \begin{declcs}{ltx@clsextension}\\
%   \cs{ltx@pkgextension}
% \end{declcs}
% Macros \cs{ltx@clsextension} and \cs{ltx@styextension}
% stores the strings |cls| and |sty|. In opposite to \hologo{LaTeX}'s
% \cs{@clsextension} and \cs{@styextension} they can also be
% used after |\begin{document}|.
%
% \subsubsection{Load check}
%
% \begin{declcs}{ltx@ifclassloaded} \M{class} \M{yes} \M{no}\\
%   \cs{ltx@ifpackageloaded} \M{package} \M{yes} \M{no}
% \end{declcs}
% Macros \cs{ltx@ifclassloaded}\slash\cs{ltx@ifpackageloaded}
% execute \meta{yes}, if the \meta{class} or \meta{package} is loaded,
% otherwise \meta{no} is called.
% Both \meta{class} and \meta{package} are specified without
% extension. The macros can also be used after |\begin{document}|.
%
% \begin{declcs}{ltx@iffileloaded} \M{file} \M{yes} \M{no}
% \end{declcs}
% If \LaTeX's \cs{ProvidesFile} macro was called before
% using \meta{file} as argument, then \cs{ltx@iffileloaded}
% calls \meta{yes}, otherwise \meta{no}.
% Therefore it is possible
% that the \meta{file} is loaded, but \meta{no} is executed because
% of a missing \cs{ProvidesFile}. The \LaTeX\ kernel does not have
% a counterpart of \cs{ltx@iffileloaded}.
%
% Note that the file name used in \cs{ProvidesFile} and
% \cs{ltx@iffileloaded} must match. For example, if \TeX's
% default extension \xext{tex} was given in the first command,
% then it must also specified in the latter command and vice versa.
%
% \subsubsection{Version date check}
%
% \begin{declcs}{ltx@ifclasslater} \M{class} \M{date} \M{yes} \M{no}\\
%   \cs{ltx@ifpackagelater} \M{package} \M{date} \M{yes} \M{no}\\
%   \cs{ltx@iffilelater} \M{file} \M{date} \M{yes} \M{no}
% \end{declcs}
% If a \cs{ProvidesClass}\slash\cs{ProvidesPackage}\slash
% \cs{ProvidesFile} command with exact the same class\slash package\slash
% file was executed before with an optional
% argument that starts with a \LaTeX\ version date, then
% this version date is compared with the argument \meta{date}.
% If they are equal or if the version date is the later date,
% then \meta{yes} is called. In all other cases \meta{no} is executed.
%
% A \LaTeX\ date has the format |YYYY/MM/DD| with |YYYY| as year
% with four digits, |MM| as month with two digits and |DD| as day
% with two digits. If \pdfTeX's \cs{pdfmatch} is available, then
% it is used to detect the version date, to reject invalid
% date formats and to reject some invalid dates.
% Dates before 1994/01/01 are always invalid, because version
% dates are introduced with \LaTeXe\ in 1994.
%
% \subsection{Macro additions}
%
% \begin{declcs}{ltx@GlobalAppendToMacro} \M{cmd} \M{addition}\\
%   \cs{ltx@LocalAppendToMacro} \M{cmd} \M{addition}
% \end{declcs}
% The \meta{addition} is appended to the parameterless macro \meta{cmd}.
% If \meta{cmd} is undefined or has the meaning \cs{relax}, then
% it will be initialized as empty macro beforehand. Due to a bug \meta{addition}
% must not contain \cs{par} before version 2010/10/25 v1.9.
%
% \begin{declcs}{ltx@GlobalPrependToMacro} \M{cmd} \M{addition}\\
%   \cs{ltx@LocalPrependToMacro} \M{cmd} \M{addition}
% \end{declcs}
% The \meta{addition} is prepended to the parameterless macro \meta{cmd}.
% If \meta{cmd} is undefined or has the meaning \cs{relax}, then
% it will be initialized as empty macro beforehand. The macros
% were added in version 2011/08/22 v1.21.
%
% \subsection{Next character detection}
%
% \begin{declcs}{ltx@ifnextchar} \M{char} \M{yes} \M{no}
% \end{declcs}
% If next character is \meta{char} then \meta{yes} is called,
% otherwise \meta{no}. The character is not removed.
% Spaces are silently removed when looking for \meta{char}
% as \hologo{LaTeX}'s version \cs{kernel@ifnextchar} does.
% But there are also small differences:
% \begin{itemize}
% \item The space can be used as \meta{char}. In this case
%   optional spaces before \meta{char} are not supported of course.
% \item If the optional space is a command that is a character
%   (defined by \cs{let} or \cs{futurelet}), then
%   \cs{kernel@ifnextchar} breaks with an \hologo{TeX} error.
%   \cs{ltx@ifnextchar} silently removes this token as
%   optional space.
% \end{itemize}
% Since 2010/03/01 v1.3.
%
% \begin{declcs}{ltx@ifnextchar@nospace} \M{char} \M{yes} \M{no}
% \end{declcs}
% Macro \cs{ltx@ifnextchar@nospace} behaves like
% macro \cs{ltx@ifnextchar} with the exception that
% optional spaces are not supported before \meta{char}.
% Since 2011/04/14 v1.19.
%
% \subsection{\cs{ltx@leavevmode}, \cs{ltx@mbox}}
%
% \begin{declcs}{ltx@leavevmode}
% \end{declcs}
% Macro \cs{ltx@leavevmode} calls \hologo{pdfTeX}'s \cs{quitvmode}.
% Otherwise \cs{leavevmode} is used and defined if it is necessary.
%
% \begin{declcs}{ltx@mbox}
% \end{declcs}
% Macro \cs{ltx@mbox} reimplements \cs{mbox} with two changes.
% Instead of \cs{leavevmode} it uses \cs{ltx@leavevmode} and
% stops right after \cs{hbox}. Especially it does not grab
% the argument and allows the extended syntax of \cs{hbox}.
%
% \subsection{Expandable test for emptiness}
%
% \begin{declcs}{ltx@ifempty} \M{stuff} \M{yes} \M{no}
% \end{declcs}
% Macro \cs{ltx@ifempty} checks in exact two expansion steps
% whether \meta{stuff}
% is empty or contains token. Depending on the result
% \meta{yes} or \meta{no} is executed. The token in \meta{stuff}
% may contain \cs{par} and unmatched conditionals (\cs{if}, \cs{else},
% \cs{fi}, \dots). Since version 2010/11/12 v1.11.
%
% \begin{declcs}{ltx@ifblank} \M{stuff} \M{yes} \M{no}
% \end{declcs}
% Macro \cs{ltx@ifblank} tests in exact two expansion steps
% if \meta{stuff} is empty or contain only blank spaces.
% In this case argument \meta{yes} is called. If \meta{stuff}
% contains other tokens than spaces then \meta{no} is executed.
% Since version 2010/12/04 v1.13.
%
% \subsection{Stripping spaces}
%
% \begin{declcs}{ltx@zapspace} \M{stuff}
% \end{declcs}
% Macro \cs{ltx@zapspace} strips spaces from \meta{stuff}
% that are not hidden inside curly braces. Like \hologo{LaTeX}'s
% \cs{zap@space} it is expandable. Differences:
% \begin{itemize}
% \item
% Syntax: \cs{zap@space} also expects a space token
% and \cs{@empty} after \meta{stuff}.
% \item
% Macro \cs{ltx@zapspace} is expandable in exact two
% expansion steps.
% \item
% Macro \cs{ltx@zapspace} always retains curly braces.
% \item
% Macro \cs{zap@space} has a bug. It stops stripping spaces
% after a token group in curly braces if the first two tokens
% inside the group are equal.
% \item
% Macro \cs{ltx@zapspace} also works with \cs{par}
% and conditionals (\cs{if}, \cs{else}, \cs{fi}, \dots).
% \end{itemize}
% Macro \cs{ltx@zapspace} is available since version
% 2010/12/07 v1.14.
%
% \subsection{Check for emptiness of boxes}
%
% \begin{declcs}{ltx@IfBoxEmpty} \M{box register number}
%   \M{yes} \M{no}
% \end{declcs}
% Macro \cs{ltx@IfBoxEmpty} calls \meta{yes} if
% the box exists (\cs{ifvoid} returns false) and the box
% does not contain any content. Otherwise if the box
% is void or contains something, then \meta{no} is executed.
% Thus being empty means that the box exists and is either
% an \cs{hbox} or a \cs{vbox} and may even have dimensions other than
% 0.0\,pt, but the box does not contain anything.
% Macro \cs{ltx@IfBoxEmpty} is available since 2010/02/04 v1.16.
%
% \begin{declcs}{ltx@IfBoxVoidOrEmpty} \M{box register number}
%   \M{yes} \M{no}
% \end{declcs}
% Macro \cs{ltx@IfBoxVoidOrEmpty} calls \meta{yes} if
% the box is either void or does not contain any content.
% Otherwise \meta{no} is executed.
% Macro \cs{ltx@IfBoxVoidOrEmpty} is available since 2010/02/04 v1.16.
%
% \StopEventually{
% }
%
% \section{Implementation}
%
% \subsection{Identification}
%
%    \begin{macrocode}
%<*package>
%    \end{macrocode}
%    Reload check, especially if the package is not used with \LaTeX.
%    \begin{macrocode}
\begingroup\catcode61\catcode48\catcode32=10\relax%
  \catcode13=5 % ^^M
  \endlinechar=13 %
  \catcode35=6 % #
  \catcode39=12 % '
  \catcode44=12 % ,
  \catcode45=12 % -
  \catcode46=12 % .
  \catcode58=12 % :
  \catcode64=11 % @
  \catcode123=1 % {
  \catcode125=2 % }
  \expandafter\let\expandafter\x\csname ver@ltxcmds.sty\endcsname
  \ifx\x\relax % plain-TeX, first loading
  \else
    \def\empty{}%
    \ifx\x\empty % LaTeX, first loading,
      % variable is initialized, but \ProvidesPackage not yet seen
    \else
      \expandafter\ifx\csname PackageInfo\endcsname\relax
        \def\x#1#2{%
          \immediate\write-1{Package #1 Info: #2.}%
        }%
      \else
        \def\x#1#2{\PackageInfo{#1}{#2, stopped}}%
      \fi
      \x{ltxcmds}{The package is already loaded}%
      \aftergroup\endinput
    \fi
  \fi
\endgroup%
%    \end{macrocode}
%    Package identification:
%    \begin{macrocode}
\begingroup\catcode61\catcode48\catcode32=10\relax%
  \catcode13=5 % ^^M
  \endlinechar=13 %
  \catcode35=6 % #
  \catcode39=12 % '
  \catcode40=12 % (
  \catcode41=12 % )
  \catcode44=12 % ,
  \catcode45=12 % -
  \catcode46=12 % .
  \catcode47=12 % /
  \catcode58=12 % :
  \catcode64=11 % @
  \catcode91=12 % [
  \catcode93=12 % ]
  \catcode123=1 % {
  \catcode125=2 % }
  \expandafter\ifx\csname ProvidesPackage\endcsname\relax
    \def\x#1#2#3[#4]{\endgroup
      \immediate\write-1{Package: #3 #4}%
      \xdef#1{#4}%
    }%
  \else
    \def\x#1#2[#3]{\endgroup
      #2[{#3}]%
      \ifx#1\@undefined
        \xdef#1{#3}%
      \fi
      \ifx#1\relax
        \xdef#1{#3}%
      \fi
    }%
  \fi
\expandafter\x\csname ver@ltxcmds.sty\endcsname
\ProvidesPackage{ltxcmds}%
  [2019/12/15 v1.24 LaTeX kernel commands for general use (HO)]%
%    \end{macrocode}
%
%    \begin{macrocode}
\begingroup\catcode61\catcode48\catcode32=10\relax%
  \catcode13=5 % ^^M
  \endlinechar=13 %
  \catcode123=1 % {
  \catcode125=2 % }
  \catcode64=11 % @
  \def\x{\endgroup
    \expandafter\edef\csname LTXcmds@AtEnd\endcsname{%
      \endlinechar=\the\endlinechar\relax
      \catcode13=\the\catcode13\relax
      \catcode32=\the\catcode32\relax
      \catcode35=\the\catcode35\relax
      \catcode61=\the\catcode61\relax
      \catcode64=\the\catcode64\relax
      \catcode123=\the\catcode123\relax
      \catcode125=\the\catcode125\relax
    }%
  }%
\x\catcode61\catcode48\catcode32=10\relax%
\catcode13=5 % ^^M
\endlinechar=13 %
\catcode35=6 % #
\catcode64=11 % @
\catcode123=1 % {
\catcode125=2 % }
\def\TMP@EnsureCode#1#2{%
  \edef\LTXcmds@AtEnd{%
    \LTXcmds@AtEnd
    \catcode#1=\the\catcode#1\relax
  }%
  \catcode#1=#2\relax
}
\TMP@EnsureCode{36}{3}% $
\TMP@EnsureCode{38}{4}% &
\TMP@EnsureCode{40}{12}% (
\TMP@EnsureCode{41}{12}% )
\TMP@EnsureCode{45}{12}% -
\TMP@EnsureCode{46}{12}% .
\TMP@EnsureCode{47}{12}% /
\TMP@EnsureCode{60}{12}% <
\TMP@EnsureCode{62}{12}% >
\TMP@EnsureCode{91}{12}% [
\TMP@EnsureCode{96}{12}% `
\TMP@EnsureCode{93}{12}% ]
\TMP@EnsureCode{94}{12}% ^ (superscript) (!)
\TMP@EnsureCode{124}{12}% |
\edef\LTXcmds@AtEnd{\LTXcmds@AtEnd\noexpand\endinput}
%    \end{macrocode}
%
% \subsection{Numbers}
%
%    \begin{macro}{\ltx@zero}
%    \begin{macrocode}
\chardef\ltx@zero=0 %
%    \end{macrocode}
%    \end{macro}
%    \begin{macro}{\ltx@one}
%    \begin{macrocode}
\chardef\ltx@one=1 %
%    \end{macrocode}
%    \end{macro}
%    \begin{macro}{\ltx@two}
%    \begin{macrocode}
\chardef\ltx@two=2 %
%    \end{macrocode}
%    \end{macro}
%    \begin{macro}{\ltx@active}
%    \begin{macrocode}
\chardef\ltx@active=13 %
%    \end{macrocode}
%    \end{macro}
%    \begin{macro}{\ltx@cclv}
%    \begin{macrocode}
\chardef\ltx@cclv=255 %
%    \end{macrocode}
%    \end{macro}
%
%    \begin{macro}{\ltx@minusone}
%    \begin{macrocode}
\def\ltx@minusone{%
  -\ltx@one
}
%    \end{macrocode}
%    \end{macro}
%
% \subsection{Scratch registers}
%
%    \begin{macro}{\ltx@LocToksA}
%    \begin{macrocode}
\toksdef\ltx@LocToksA=0 %
%    \end{macrocode}
%    \end{macro}
%    \begin{macro}{\ltx@LocToksB}
%    \begin{macrocode}
\toksdef\ltx@LocToksB=2 %
%    \end{macrocode}
%    \end{macro}
%    \begin{macro}{\ltx@LocToksC}
%    \begin{macrocode}
\toksdef\ltx@LocToksC=4 %
%    \end{macrocode}
%    \end{macro}
%    \begin{macro}{\ltx@LocToksD}
%    \begin{macrocode}
\toksdef\ltx@LocToksD=6 %
%    \end{macrocode}
%    \end{macro}
%    \begin{macro}{\ltx@LocToksE}
%    \begin{macrocode}
\toksdef\ltx@LocToksE=8 %
%    \end{macrocode}
%    \end{macro}
%
%    \begin{macro}{\ltx@GlobToksA}
%    \begin{macrocode}
\toksdef\ltx@GlobToksA=1 %
%    \end{macrocode}
%    \end{macro}
%    \begin{macro}{\ltx@GlobToksB}
%    \begin{macrocode}
\toksdef\ltx@GlobToksB=3 %
%    \end{macrocode}
%    \end{macro}
%    \begin{macro}{\ltx@GlobToksC}
%    \begin{macrocode}
\toksdef\ltx@GlobToksC=5 %
%    \end{macrocode}
%    \end{macro}
%    \begin{macro}{\ltx@GlobToksD}
%    \begin{macrocode}
\toksdef\ltx@GlobToksD=7 %
%    \end{macrocode}
%    \end{macro}
%    \begin{macro}{\ltx@GlobToksE}
%    \begin{macrocode}
\toksdef\ltx@GlobToksE=9 %
%    \end{macrocode}
%    \end{macro}
%
%    \begin{macro}{\ltx@LocDimenA}
%    \begin{macrocode}
\dimendef\ltx@LocDimenA=0 %
%    \end{macrocode}
%    \end{macro}
%    \begin{macro}{\ltx@LocDimenB}
%    \begin{macrocode}
\dimendef\ltx@LocDimenB=2 %
%    \end{macrocode}
%    \end{macro}
%    \begin{macro}{\ltx@LocDimenC}
%    \begin{macrocode}
\dimendef\ltx@LocDimenC=4 %
%    \end{macrocode}
%    \end{macro}
%    \begin{macro}{\ltx@LocDimenD}
%    \begin{macrocode}
\dimendef\ltx@LocDimenD=6 %
%    \end{macrocode}
%    \end{macro}
%    \begin{macro}{\ltx@LocDimenE}
%    \begin{macrocode}
\dimendef\ltx@LocDimenE=8 %
%    \end{macrocode}
%    \end{macro}
%
%    \begin{macro}{\ltx@GlobDimenA}
%    \begin{macrocode}
\dimendef\ltx@GlobDimenA=1 %
%    \end{macrocode}
%    \end{macro}
%    \begin{macro}{\ltx@GlobDimenB}
%    \begin{macrocode}
\dimendef\ltx@GlobDimenB=3 %
%    \end{macrocode}
%    \end{macro}
%    \begin{macro}{\ltx@GlobDimenC}
%    \begin{macrocode}
\dimendef\ltx@GlobDimenC=5 %
%    \end{macrocode}
%    \end{macro}
%    \begin{macro}{\ltx@GlobDimenD}
%    \begin{macrocode}
\dimendef\ltx@GlobDimenD=7 %
%    \end{macrocode}
%    \end{macro}
%    \begin{macro}{\ltx@GlobDimenE}
%    \begin{macrocode}
\dimendef\ltx@GlobDimenE=9 %
%    \end{macrocode}
%    \end{macro}
%
%    \begin{macro}{\ltx@LocSkipA}
%    \begin{macrocode}
\skipdef\ltx@LocSkipA=0 %
%    \end{macrocode}
%    \end{macro}
%    \begin{macro}{\ltx@LocSkipB}
%    \begin{macrocode}
\skipdef\ltx@LocSkipB=2 %
%    \end{macrocode}
%    \end{macro}
%    \begin{macro}{\ltx@LocSkipC}
%    \begin{macrocode}
\skipdef\ltx@LocSkipC=4 %
%    \end{macrocode}
%    \end{macro}
%    \begin{macro}{\ltx@LocSkipD}
%    \begin{macrocode}
\skipdef\ltx@LocSkipD=6 %
%    \end{macrocode}
%    \end{macro}
%    \begin{macro}{\ltx@LocSkipE}
%    \begin{macrocode}
\skipdef\ltx@LocSkipE=8 %
%    \end{macrocode}
%    \end{macro}
%
%    \begin{macro}{\ltx@GlobSkipA}
%    \begin{macrocode}
\skipdef\ltx@GlobSkipA=1 %
%    \end{macrocode}
%    \end{macro}
%    \begin{macro}{\ltx@GlobSkipB}
%    \begin{macrocode}
\skipdef\ltx@GlobSkipB=3 %
%    \end{macrocode}
%    \end{macro}
%    \begin{macro}{\ltx@GlobSkipC}
%    \begin{macrocode}
\skipdef\ltx@GlobSkipC=5 %
%    \end{macrocode}
%    \end{macro}
%    \begin{macro}{\ltx@GlobSkipD}
%    \begin{macrocode}
\skipdef\ltx@GlobSkipD=7 %
%    \end{macrocode}
%    \end{macro}
%    \begin{macro}{\ltx@GlobSkipE}
%    \begin{macrocode}
\skipdef\ltx@GlobSkipE=9 %
%    \end{macrocode}
%    \end{macro}
%
%
% \subsection{Argument killers}
%
%    \begin{macro}{\ltx@gobble}
%    \begin{macrocode}
\long\def\ltx@gobble#1{}
%    \end{macrocode}
%    \end{macro}
%    \begin{macro}{\ltx@gobbletwo}
%    \begin{macrocode}
\long\def\ltx@gobbletwo#1#2{}
%    \end{macrocode}
%    \end{macro}
%    \begin{macro}{\ltx@gobblethree}
%    \begin{macrocode}
\long\def\ltx@gobblethree#1#2#3{}
%    \end{macrocode}
%    \end{macro}
%    \begin{macro}{\ltx@gobblefour}
%    \begin{macrocode}
\long\def\ltx@gobblefour#1#2#3#4{}
%    \end{macrocode}
%    \end{macro}
%
%    \begin{macro}{\ltx@GobbleNum}
%    \begin{macrocode}
\def\ltx@GobbleNum#1{%
  \romannumeral
  \csname ltx@zero%
  \expandafter\LTXcmds@GobbleNum
  \romannumeral\LTXcmds@num{#1}000{m\endcsname}%
}
%    \end{macrocode}
%    \end{macro}
%    \begin{macro}{\LTXcmds@GobbleNum}
%    \begin{macrocode}
\def\LTXcmds@GobbleNum#1{%
  \csname LTXcmds@G#1\LTXcmds@GobbleNum
}
%    \end{macrocode}
%    \end{macro}
%    \begin{macro}{\LTXcmds@Gm}
%    \begin{macrocode}
\long\def\LTXcmds@Gm#1{%
  \endcsname
}
%    \end{macrocode}
%    \end{macro}
%
% \subsection{Argument grabbers}
%
%    \begin{macro}{\ltx@firstofone}
%    \begin{macrocode}
\long\def\ltx@firstofone#1{#1}
%    \end{macrocode}
%    \end{macro}
%    \begin{macro}{\ltx@firstoftwo}
%    \begin{macrocode}
\long\def\ltx@firstoftwo#1#2{#1}
%    \end{macrocode}
%    \end{macro}
%    \begin{macro}{\ltx@secondoftwo}
%    \begin{macrocode}
\long\def\ltx@secondoftwo#1#2{#2}
%    \end{macrocode}
%    \end{macro}
%
%    \begin{macro}{\ltx@firstofthree}
%    \begin{macrocode}
\long\def\ltx@firstofthree#1#2#3{#1}
%    \end{macrocode}
%    \end{macro}
%    \begin{macro}{\ltx@secondofthree}
%    \begin{macrocode}
\long\def\ltx@secondofthree#1#2#3{#2}
%    \end{macrocode}
%    \end{macro}
%    \begin{macro}{\ltx@thirdofthree}
%    \begin{macrocode}
\long\def\ltx@thirdofthree#1#2#3{#3}%
%    \end{macrocode}
%    \end{macro}
%
%    \begin{macro}{\ltx@firstoffour}
%    \begin{macrocode}
\long\def\ltx@firstoffour#1#2#3#4{#1}
%    \end{macrocode}
%    \end{macro}
%    \begin{macro}{\ltx@secondoffour}
%    \begin{macrocode}
\long\def\ltx@secondoffour#1#2#3#4{#2}
%    \end{macrocode}
%    \end{macro}
%    \begin{macro}{\ltx@thirdoffour}
%    \begin{macrocode}
\long\def\ltx@thirdoffour#1#2#3#4{#3}%
%    \end{macrocode}
%    \end{macro}
%    \begin{macro}{\ltx@fourthoffour}
%    \begin{macrocode}
\long\def\ltx@fourthoffour#1#2#3#4{#4}%
%    \end{macrocode}
%    \end{macro}
%
% \subsection{List helpers}
%
%    \begin{macro}{\ltx@carzero}
%    \begin{macrocode}
\long\def\ltx@carzero#1\@nil{}%
%    \end{macrocode}
%    \end{macro}
%    \begin{macro}{\LTXcmds@cdrzero}
%    \begin{macrocode}
\long\def\LTXcmds@cdrzero#1\@nil{#1}
%    \end{macrocode}
%    \end{macro}
%    \begin{macro}{\ltx@cdrzero}
%    \begin{macrocode}
\def\ltx@cdrzero{%
  \romannumeral\LTXcmds@cdrzero\ltx@zero
}
%    \end{macrocode}
%    \end{macro}
%
%    \begin{macro}{\ltx@car}
%    \begin{macrocode}
\long\def\ltx@car#1#2\@nil{#1}
%    \end{macrocode}
%    \end{macro}
%    \begin{macro}{\ltx@cdr}
%    \begin{macrocode}
\long\def\ltx@cdr#1{%
  \romannumeral\LTXcmds@cdrzero\ltx@zero
}
%    \end{macrocode}
%    \end{macro}
%
%    \begin{macro}{\ltx@cartwo}
%    \begin{macrocode}
\long\def\ltx@cartwo#1#2#3\@nil{#1#2}
%    \end{macrocode}
%    \end{macro}
%    \begin{macro}{\ltx@carsecond}
%    \begin{macrocode}
\long\def\ltx@carsecond#1#2#3\@nil{#2}
%    \end{macrocode}
%    \end{macro}
%    \begin{macro}{\ltx@cdrtwo}
%    \begin{macrocode}
\long\def\ltx@cdrtwo#1#2{%
  \romannumeral\LTXcmds@cdrzero\ltx@zero
}
%    \end{macrocode}
%    \end{macro}
%
%    \begin{macro}{\ltx@carthree}
%    \begin{macrocode}
\long\def\ltx@carthree#1#2#3#4\@nil{#1#2#3}
%    \end{macrocode}
%    \end{macro}
%    \begin{macro}{\ltx@carthird}
%    \begin{macrocode}
\long\def\ltx@carthird#1#2#3#4\@nil{#3}
%    \end{macrocode}
%    \end{macro}
%    \begin{macro}{\ltx@cdrthree}
%    \begin{macrocode}
\long\def\ltx@cdrthree#1#2#3{%
  \romannumeral\LTXcmds@cdrzero\ltx@zero
}
%    \end{macrocode}
%    \end{macro}
%
%    \begin{macro}{\ltx@carfour}
%    \begin{macrocode}
\long\def\ltx@carfour#1#2#3#4#5\@nil{#1#2#3#4}
%    \end{macrocode}
%    \end{macro}
%    \begin{macro}{\ltx@carfourth}
%    \begin{macrocode}
\long\def\ltx@carfourth#1#2#3#4#5\@nil{#4}
%    \end{macrocode}
%    \end{macro}
%    \begin{macro}{\ltx@cdrfour}
%    \begin{macrocode}
\long\def\ltx@cdrfour#1#2#3#4{%
  \romannumeral\LTXcmds@cdrzero\ltx@zero
}
%    \end{macrocode}
%    \end{macro}
%
%    \begin{macro}{\ltx@CarNum}
%    \begin{macrocode}
\def\ltx@CarNum#1{%
  \romannumeral
  \csname LTXcmds@CarNumFinish%
  \expandafter\LTXcmds@CarNum
  \romannumeral\LTXcmds@num{#1}000{x\endcsname}%
}
%    \end{macrocode}
%    \end{macro}
%    \begin{macro}{\LTXcmds@CarNum}
%    \begin{macrocode}
\def\LTXcmds@CarNum#1{%
  \csname LTXcmds@C#1\LTXcmds@CarNum
}
%    \end{macrocode}
%    \end{macro}
%    \begin{macro}{\LTXcmds@Cm}
%    \begin{macrocode}
\long\def\LTXcmds@Cm#1#2{%
  \endcsname{#1#2}%
}
%    \end{macrocode}
%    \end{macro}
%    \begin{macro}{\LTXcmds@Cx}
%    \begin{macrocode}
\def\LTXcmds@Cx#1{%
  \endcsname{}%
}
%    \end{macrocode}
%    \end{macro}
%    \begin{macro}{\LTXcmds@CarNumFinish}
%    \begin{macrocode}
\long\def\LTXcmds@CarNumFinish#1#2\@nil{%
  \ltx@zero
  #1%
}
%    \end{macrocode}
%    \end{macro}
%
%    \begin{macro}{\ltx@CarNumth}
%    \begin{macrocode}
\def\ltx@CarNumth#1{%
  \romannumeral
  \expandafter\expandafter\expandafter
  \LTXcmds@CarNumth
  \ltx@GobbleNum{#1}{}%
}
%    \end{macrocode}
%    \end{macro}
%    \begin{macro}{\LTXcmds@CarNumth}
%    \begin{macrocode}
\long\def\LTXcmds@CarNumth#1#2\@nil{%
  \ltx@zero
  #1%
}
%    \end{macrocode}
%    \end{macro}
%
%    \begin{macro}{\ltx@CdrNum}
%    \begin{macrocode}
\def\ltx@CdrNum#1{%
  \romannumeral%
  \expandafter\expandafter\expandafter\ltx@cdrzero
  \expandafter\expandafter\expandafter\ltx@zero
  \ltx@GobbleNum{#1}%
}
%    \end{macrocode}
%    \end{macro}
%
% \subsection{Tail recursion}
%
%    \begin{macro}{\ltx@ReturnAfterFi}
%    \begin{macrocode}
\long\def\ltx@ReturnAfterFi#1\fi{\fi#1}
%    \end{macrocode}
%    \end{macro}
%    \begin{macro}{\ltx@ReturnAfterElseFi}
%    \begin{macrocode}
\long\def\ltx@ReturnAfterElseFi#1\else#2\fi{\fi#1}
%    \end{macrocode}
%    \end{macro}
%
% \subsection{Empty macro}
%
%    \begin{macro}{\ltx@empty}
%    \begin{macrocode}
\def\ltx@empty{}
%    \end{macrocode}
%    \end{macro}
%
% \subsection{Characters}
%
%    \begin{macro}{\ltx@space}
%    \begin{macrocode}
\def\ltx@space{ }
%    \end{macrocode}
%    \end{macro}
%    \begin{macro}{\ltx@percentchar}
%    \begin{macrocode}
\begingroup
  \lccode`0=`\%\relax
\lowercase{\endgroup
  \def\ltx@percentchar{0}%
}
%    \end{macrocode}
%    \end{macro}
%    \begin{macro}{\ltx@backslashchar}
%    \begin{macrocode}
\begingroup
  \lccode`0=`\\\relax
\lowercase{\endgroup
  \def\ltx@backslashchar{0}%
}
%    \end{macrocode}
%    \end{macro}
%    \begin{macro}{\ltx@hashchar}
%    \begin{macrocode}
\begingroup
  \lccode`0=`\#\relax
\lowercase{\endgroup
  \def\ltx@hashchar{0}%
}
%    \end{macrocode}
%    \end{macro}
%    \begin{macro}{\ltx@leftbracechar}
%    \begin{macrocode}
\begingroup
  \lccode`0=`\{\relax
\lowercase{\endgroup
  \def\ltx@leftbracechar{0}%
}
%    \end{macrocode}
%    \end{macro}
%    \begin{macro}{\ltx@rightbracechar}
%    \begin{macrocode}
\begingroup
  \lccode`0=`\}\relax
\lowercase{\endgroup
  \def\ltx@rightbracechar{0}%
}
%    \end{macrocode}
%    \end{macro}
%
% \subsection{Boolean switch}
%
%    \begin{macro}{\ltx@newif}
%    \begin{macrocode}
\def\ltx@newif#1{%
  \begingroup
    \escapechar=-1 %
  \expandafter\endgroup
  \expandafter\LTXcmds@newif\string#1\@nil
}
%    \end{macrocode}
%    \end{macro}
%    \begin{macro}{\LTXcmds@newif}
%    \begin{macrocode}
\begingroup
  \escapechar=-1 %
\expandafter\endgroup
\expandafter\def\expandafter\LTXcmds@newif\string\if#1\@nil{%
  \expandafter\edef\csname#1true\endcsname{%
    \let
    \expandafter\noexpand\csname if#1\endcsname
    \noexpand\iftrue
  }%
  \expandafter\edef\csname#1false\endcsname{%
    \let
    \expandafter\noexpand\csname if#1\endcsname
    \noexpand\iffalse
  }%
  \csname#1false\endcsname
}
%    \end{macrocode}
%    \end{macro}
%
%    \begin{macro}{\ltx@newglobalif}
%    \begin{macrocode}
\def\ltx@newglobalif#1{%
  \begingroup
    \escapechar=-1 %
  \expandafter\endgroup
  \expandafter\LTXcmds@newglobalif\string#1\@nil
}
%    \end{macrocode}
%    \end{macro}
%    \begin{macro}{\LTXcmds@newglobalif}
%    \begin{macrocode}
\begingroup
  \escapechar=-1 %
\expandafter\endgroup
\expandafter
\def\expandafter\LTXcmds@newglobalif\string\if#1\@nil{%
  \expandafter\edef\csname#1true\endcsname{%
    \global\let
    \expandafter\noexpand\csname if#1\endcsname
    \noexpand\iftrue
  }%
  \expandafter\edef\csname#1false\endcsname{%
    \global\let
    \expandafter\noexpand\csname if#1\endcsname
    \noexpand\iffalse
  }%
  \csname#1false\endcsname
}
%    \end{macrocode}
%    \end{macro}
%
% \subsection{Command definitions}
%
%    \begin{macro}{\ltx@LocalExpandAfter}
%    \begin{macrocode}
\def\ltx@LocalExpandAfter{%
  \begingroup
    \expandafter\expandafter\expandafter
  \endgroup
  \expandafter
}
%    \end{macrocode}
%    \end{macro}
%
%    \begin{macrocode}
\ltx@LocalExpandAfter
\ifx\csname ifcsname\endcsname\relax
%    \end{macrocode}
%    \begin{macro}{\ltx@ifundefined}
%    \begin{macrocode}
  \def\ltx@ifundefined#1{%
    \expandafter\ifx\csname #1\endcsname\relax
      \expandafter\ltx@firstoftwo
    \else
      \expandafter\ltx@secondoftwo
    \fi
  }%
%    \end{macrocode}
%    \end{macro}
%    \begin{macro}{\ltx@IfUndefined}
%    \begin{macrocode}
  \def\ltx@IfUndefined#1{%
    \begingroup\expandafter\expandafter\expandafter\endgroup
    \expandafter\ifx\csname #1\endcsname\relax
      \expandafter\ltx@firstoftwo
    \else
      \expandafter\ltx@secondoftwo
    \fi
  }%
%    \end{macrocode}
%    \end{macro}
%    \begin{macrocode}
  \expandafter\ltx@gobble
\else
  \expandafter\ltx@firstofone
\fi
{%
%    \end{macrocode}
%    \begin{macro}{\ltx@ifundefined}
%    \begin{macrocode}
  \def\ltx@ifundefined#1{%
    \ifcsname #1\endcsname
      \expandafter\ifx\csname #1\endcsname\relax
        \expandafter\expandafter\expandafter\ltx@firstoftwo
      \else
        \expandafter\expandafter\expandafter\ltx@secondoftwo
      \fi
    \else
      \expandafter\ltx@firstoftwo
    \fi
  }%
%    \end{macrocode}
%    \end{macro}
%    \begin{macro}{\ltx@IfUndefined}
%    \begin{macrocode}
  \let\ltx@IfUndefined\ltx@ifundefined
%    \end{macrocode}
%    \end{macro}
%    \begin{macrocode}
}
%    \end{macrocode}
%
% \subsection{Stripping}
%
%    \begin{macro}{\ltx@RemovePrefix}
%    \begin{macrocode}
\def\ltx@RemovePrefix#1>{}
%    \end{macrocode}
%    \end{macro}
%
%    \begin{macro}{\ltx@StripPrefix}
%    \begin{macrocode}
\def\ltx@StripPrefix{%
  \expandafter\ltx@RemovePrefix
}
%    \end{macrocode}
%    \end{macro}
%
%    \begin{macro}{\ltx@onelevel@sanitize}
%    \begin{macrocode}
\def\ltx@onelevel@sanitize#1{%
  \edef#1{%
    \expandafter
    \ltx@RemovePrefix\meaning#1%
  }%
}
%    \end{macrocode}
%    \end{macro}
%
% \subsection{File management}
%
% \subsubsection{File extensions}
%
%    \begin{macro}{\ltx@clsextension}
%    \begin{macrocode}
\def\ltx@clsextension{cls}
%    \end{macrocode}
%    \end{macro}
%    \begin{macro}{\ltx@pkgextension}
%    \begin{macrocode}
\def\ltx@pkgextension{sty}
%    \end{macrocode}
%    \end{macro}
%
% \subsubsection{Load check}
%
%    \begin{macro}{\ltx@iffileloaded}
%    \begin{macrocode}
\def\ltx@iffileloaded#1{%
  \ltx@ifundefined{ver@#1}\ltx@secondoftwo\ltx@firstoftwo
}
%    \end{macrocode}
%    \end{macro}
%    \begin{macro}{\ltx@ifclassloaded}
%    \begin{macrocode}
\def\ltx@ifclassloaded#1{%
  \ltx@iffileloaded{#1.\ltx@clsextension}%
}
%    \end{macrocode}
%    \end{macro}
%    \begin{macro}{\ltx@ifpackageloaded}
%    \begin{macrocode}
\def\ltx@ifpackageloaded#1{%
  \ltx@iffileloaded{#1.\ltx@pkgextension}%
}
%    \end{macrocode}
%    \end{macro}
%
% \subsubsection{Version date check}
%
%    \begin{macro}{\ltx@iffilelater}
%    \begin{macrocode}
\def\ltx@iffilelater#1#2{%
  \ltx@iffileloaded{#1}{%
    \expandafter\LTXcmds@IfLater\expandafter{%
      \number
      \expandafter\expandafter\expandafter\LTXcmds@ParseVersion
      \expandafter\expandafter\expandafter{%
        \csname ver@#1\endcsname
      }%
    \expandafter}\expandafter{%
      \number
      \expandafter\LTXcmds@ParseVersion\expandafter{#2}%
    }%
  }\ltx@secondoftwo
}
%    \end{macrocode}
%    \end{macro}
%    \begin{macro}{\LTXcmds@IfLater}
%    \begin{macrocode}
\def\LTXcmds@IfLater#1#2{%
  \ifcase 0%
      \ifnum#1<19940101 %
      \else
        \ifnum#2<19940101 %
        \else
          \ifnum#2>#1 %
          \else
            1%
          \fi
        \fi
      \fi
      \ltx@space
    \expandafter\ltx@secondoftwo
  \else
    \expandafter\ltx@firstoftwo
  \fi
}
%    \end{macrocode}
%    \end{macro}
%    \begin{macro}{\ltx@ifclasslater}
%    \begin{macrocode}
\def\ltx@ifclasslater#1{%
  \ltx@iffilelater{#1.\ltx@clsextension}%
}
%    \end{macrocode}
%    \end{macro}
%    \begin{macro}{\ltx@ifpackagelater}
%    \begin{macrocode}
\def\ltx@ifpackagelater#1{%
  \ltx@iffilelater{#1.\ltx@pkgextension}%
}
%    \end{macrocode}
%    \end{macro}
%
%    \begin{macrocode}
\ltx@IfUndefined{pdfmatch}{%
%    \end{macrocode}
%    \begin{macro}{\LTXcmds@ParseVersion}
%    \begin{macrocode}
  \def\LTXcmds@ParseVersion#1{%
    \LTXcmds@@ParseVersion#10000/00/00\@nil
  }%
%    \end{macrocode}
%    \end{macro}
%    \begin{macro}{\LTXcmds@@ParseVersion}
%    \begin{macrocode}
  \def\LTXcmds@@ParseVersion#1#2#3#4/#5#6/#7#8#9\@nil{%
    #1#2#3#4#5#6#7#8%
  }%
%    \end{macrocode}
%    \end{macro}
%    \begin{macrocode}
}{%
%    \end{macrocode}
%    \begin{macro}{\LTXcmds@ParseVersion}
%    \begin{macrocode}
  \def\LTXcmds@ParseVersion#1{%
    \ifnum\pdfmatch{%
      ^%
      (199[4-9]|[2-9][0-9][0-9][0-9])/%
      (0[1-9]|1[0-2])/%
      (0[1-9]|[1-2][0-9]|3[0-1])%
    }{#1}=1 %
      \ltx@StripPrefix\pdflastmatch1 %
      \ltx@StripPrefix\pdflastmatch2 %
      \ltx@StripPrefix\pdflastmatch3 %
    \else
      0%
    \fi
  }%
%    \end{macrocode}
%    \end{macro}
%    \begin{macrocode}
}
%    \end{macrocode}
%
% \subsection{Macro additions}
%
%    \begin{macro}{\ltx@GlobalAppendToMacro}
%    \begin{macrocode}
\long\def\ltx@GlobalAppendToMacro#1#2{%
  \ifx\ltx@undefined#1%
    \let#1\ltx@empty
  \else
    \ifx\relax#1%
      \let#1\ltx@empty
    \fi
  \fi
  \begingroup
    \ltx@LocToksA\expandafter{#1#2}%
    \xdef#1{\the\ltx@LocToksA}%
  \endgroup
}
%    \end{macrocode}
%    \end{macro}
%    \begin{macro}{\ltx@LocalAppendToMacro}
%    \begin{macrocode}
\long\def\ltx@LocalAppendToMacro#1#2{%
  \global\let\LTXcmds@gtemp#1%
  \ifx\ltx@undefined\LTXcmds@gtemp
    \global\let\LTXcmds@gtemp\ltx@empty
  \else
    \ifx\relax\LTXcmds@gtemp
      \global\letLTXcmds@gtemp\ltx@empty
    \fi
  \fi
  \begingroup
    \ltx@LocToksA\expandafter{\LTXcmds@gtemp#2}%
    \xdef\LTXcmds@gtemp{\the\ltx@LocToksA}%
  \endgroup
  \let#1\LTXcmds@gtemp
}
%    \end{macrocode}
%    \end{macro}
%
%    \begin{macro}{\ltx@GlobalPrependToMacro}
%    \begin{macrocode}
\long\def\ltx@GlobalPrependToMacro#1#2{%
  \ifx\ltx@undefined#1%
    \let#1\ltx@empty
  \else
    \ifx\relax#1%
      \let#1\ltx@empty
    \fi
  \fi
  \begingroup
    \ltx@LocToksA{#2}%
    \ltx@LocToksB\expandafter{#1}%
    \xdef#1{\the\ltx@LocToksA\the\ltx@LocToksB}%
  \endgroup
}
%    \end{macrocode}
%    \end{macro}
%    \begin{macro}{\ltx@LocalPrependToMacro}
%    \begin{macrocode}
\long\def\ltx@LocalPrependToMacro#1#2{%
  \global\let\LTXcmds@gtemp#1%
  \ifx\ltx@undefined\LTXcmds@gtemp
    \global\let\LTXcmds@gtemp\ltx@empty
  \else
    \ifx\relax\LTXcmds@gtemp
      \global\letLTXcmds@gtemp\ltx@empty
    \fi
  \fi
  \begingroup
    \ltx@LocToksA{#2}%
    \ltx@LocToksB\expandafter{\LTXcmds@gtemp}%
    \xdef\LTXcmds@gtemp{\the\ltx@LocToksA\the\ltx@LocToksB}%
  \endgroup
  \let#1\LTXcmds@gtemp
}
%    \end{macrocode}
%    \end{macro}
%
% \subsection{Next character detection}
%
%    \begin{macro}{\ltx@ifnextchar}
%    \begin{macrocode}
\long\def\ltx@ifnextchar#1#2#3{%
  \begingroup
  \let\LTXcmds@CharToken= #1\relax
  \ltx@LocToksA{\endgroup#2}%
  \ltx@LocToksB{\endgroup#3}%
  \futurelet\LTXcmds@LetToken\LTXcmds@ifnextchar
}
%    \end{macrocode}
%    \end{macro}
%    \begin{macro}{\LTXcmds@ifnextchar}
%    \begin{macrocode}
\def\LTXcmds@ifnextchar{%
  \ifx\LTXcmds@LetToken\LTXcmds@CharToken
    \the\expandafter\ltx@LocToksA
  \else
    \expandafter
      \ifx\csname LTXcmds@LetToken\endcsname\LTXcmds@SpaceToken
      \expandafter\expandafter\expandafter\LTXcmds@@ifnextchar
    \else
      \the\expandafter\expandafter\expandafter\ltx@LocToksB
    \fi
  \fi
}
%    \end{macrocode}
%    \end{macro}
%    \begin{macro}{\LTXcmds@@ifnextchar}
%    \cs{futurelet} does not distinguish between a
%    character and a command that is a character
%    (defined by using \cs{let} or \cs{futurelet}).
%    Therefore the space is catched by \cs{romannumeral}
%    with negative character constant that gobbles one
%    optional space.
%    \begin{macrocode}
\def\LTXcmds@@ifnextchar{%
  \expandafter\futurelet
  \expandafter\LTXcmds@LetToken
  \expandafter\LTXcmds@ifnextchar
  \romannumeral-`\.%
}
%    \end{macrocode}
%    \end{macro}
%    \begin{macro}{\LTXcmds@SpaceToken}
%    \begin{macrocode}
\ltx@firstofone{\let\LTXcmds@SpaceToken= } %
%    \end{macrocode}
%    \end{macro}
%
%    \begin{macro}{\ltx@ifnextchar@nospace}
%    \begin{macrocode}
\long\def\ltx@ifnextchar@nospace#1#2#3{%
  \begingroup
  \let\LTXcmds@CharToken= #1\relax
  \ltx@LocToksA{\endgroup#2}%
  \ltx@LocToksB{\endgroup#3}%
  \futurelet\LTXcmds@LetToken\LTXcmds@ifnextchar@nospace
}
%    \end{macrocode}
%    \end{macro}
%    \begin{macro}{\LTXcmds@ifnextchar@nospace}
%    \begin{macrocode}
\def\LTXcmds@ifnextchar@nospace{%
  \the
  \ifx\LTXcmds@LetToken\LTXcmds@CharToken
    \expandafter\ltx@LocToksA
  \else
    \expandafter\ltx@LocToksB
  \fi
}
%    \end{macrocode}
%    \end{macro}
% \subsection{\cs{ltx@leavevmode}, \cs{ltx@mbox}}
%
%    \begin{macro}{\ltx@leavevmode}
%    \begin{macrocode}
\ltx@IfUndefined{quitvmode}{%
  \ltx@IfUndefined{leavevmode}{%
    \ltx@IfUndefined{voidb@x}{%
      \ltx@IfUndefined{newbox}{%
        \def\ltx@leavevmode{%
          \begingroup
            \setbox\ltx@zero=\hbox{}%
            \begingroup
              \setbox\ltx@zero=\hbox{\box\ltx@zero}%
            \endgroup
            \unhbox\ltx@zero
          \endgroup
        }%
      }{%
        \csname newbox\endcsname\LTXcmds@VoidBox
        \ifvoid\LTXcmds@VoidBox
        \else
          \setbox\LTXcmds@VoidBox=\hbox{}%
          \begingroup
            \setbox\LTXcmds@VoidBox=\hbox{\box\LTXcmds@VoidBox}%
          \endgroup
        \fi
        \def\ltx@leavevmode{\unhbox\LTXcmds@VoidBox}%
      }%
    }{%
      \def\ltx@leavevmode{\unhbox\voidb@x}%
    }%
  }{%
    \let\ltx@leavevmode\leavevmode
  }%
}{%
  \let\ltx@leavevmode\quitvmode
}
%    \end{macrocode}
%    \end{macro}
%
%    \begin{macro}{\ltx@mbox}
%    \begin{macrocode}
\def\ltx@mbox{%
  \ltx@leavevmode
  \hbox
}
%    \end{macrocode}
%    \end{macro}
%
% \subsection{Help macros}
%
%    \begin{macro}{\LTXcmds@num}
%    \begin{macrocode}
\ltx@IfUndefined{numexpr}{%
  \def\LTXcmds@num#1{%
    \expandafter\ltx@firstofone\expandafter{%
      \number#1%
    }%
  }%
}{%
  \def\LTXcmds@num#1{%
    \expandafter\ltx@firstofone\expandafter{%
      \the\numexpr#1%
    }%
  }%
}
%    \end{macrocode}
%    \end{macro}
%
% \subsection{Expandable test for emptiness}
%
%    \begin{macrocode}
\ltx@IfUndefined{detokenize}{%
%    \end{macrocode}
%
% \subsubsection{Vanilla \hologo{TeX}}
%
%    \begin{macro}{\ltx@ifempty}
%    The macro is based on \cs{@ifempty} of Robert R. Schneck
%    \cite{ifempty:schneck}
%    and \cs{@ifnull} of Ulrich Diez \cite{ifnull:diez}.
%    There are three cases to consider:
%    \begin{compactenum}
%    \item |#1| is empty,
%    \item |#1| is not empty and the first token
%      is not a begingroup character,
%    \item |#1| starts with a begingroup character (catcode 1).
%    \end{compactenum}
%    \begin{macrocode}
  \def\LTXcmds@temp#1{%
    \long\def\ltx@ifempty##1{%
      \romannumeral0%
      \iffalse{\fi
        \expandafter\ltx@gobble\expandafter{%
          \expandafter{\string##1}%
          \expandafter\ltx@gobble\string
        }%
        \expandafter\ltx@firstofthree\expandafter
        {\iffalse}\fi
        \expandafter#1\ltx@secondoftwo
      }%
      \expandafter#1\ltx@firstoftwo
    }%
%    \end{macrocode}
%    \end{macro}
%    \begin{macro}{\ltx@ifblank}
%    \begin{macrocode}
    \long\def\ltx@ifblank##1{%
      \romannumeral0%
      \iffalse{\fi
        \expandafter\expandafter\expandafter\ltx@gobble
        \expandafter\expandafter\expandafter{%
          \expandafter\expandafter\expandafter{%
            \expandafter\string\ltx@gobble##1.%
          }%
          \expandafter\ltx@gobble\string
        }%
        \expandafter\ltx@firstofthree\expandafter
        {\iffalse}\fi
        \expandafter#1\ltx@secondoftwo
      }%
      \expandafter#1\ltx@firstoftwo
    }%
  }%
  \LTXcmds@temp{ }%
%    \end{macrocode}
%    \end{macro}
%    \begin{macrocode}
}{%
%    \end{macrocode}
%
% \subsubsection{With \cs{detokenize}}
%
%    Ahmed Musa provided \cs{ifstrempty} using \cs{detokenize}
%    and \cs{pdfstrcmp} \cite{ifstrempty:musa}. Ulrich Diez, GL,
%    Heiko Oberdiek
%    improved it further by removing \cs{pdfstrcmp} and
%    taking three arguments
%    \cite{ifstrempty:diez1,ifstrempty:diez2,ifstrempty:diez3,^^A
%      ifstrempty:gl,ifstrempty:ho}.
%
%    \begin{macro}{\ltx@ifempty}
%    \begin{macrocode}
  \long\def\ltx@ifempty#1{%
    \romannumeral%
    \csname
      LTXcmds@ifempty%
      \ifcat$\detokenize{#1}$%
        @%
      \fi
    \endcsname
  }%
%    \end{macrocode}
%    \end{macro}
%    \begin{macro}{\LTXcmds@ifempty@}
%    \begin{macrocode}
  \long\def\LTXcmds@ifempty@#1#2{0 #1}%
%    \end{macrocode}
%    \end{macro}
%    \begin{macro}{\LTXcmds@ifempty}
%    \begin{macrocode}
  \long\def\LTXcmds@ifempty#1#2{0 #2}%
%    \end{macrocode}
%    \end{macro}
%
% \subsubsection{\cs{ltx@ifblank}}
%
%    \begin{macro}{\ltx@ifblank}
%    \begin{macrocode}
  \long\def\ltx@ifblank#1{%
    \romannumeral%
    \csname
      LTXcmds@ifempty%
      \ifcat$\detokenize\expandafter{\ltx@gobble#1.}$%
        @%
      \fi
    \endcsname
  }%
%    \end{macrocode}
%    \end{macro}
%
%    \begin{macrocode}
}
%    \end{macrocode}
%
% \subsection{\cs{ltx@zapspace}}
%
%    \begin{macro}{\ltx@zapspace}
%    \begin{macrocode}
\long\def\ltx@zapspace#1{%
  \romannumeral
  \LTXcmds@zapspace\ltx@zero#1 \@nil
}
%    \end{macrocode}
%    \end{macro}
%    \begin{macro}{\LTXcmds@zapspace}
%    \begin{macrocode}
\long\def\LTXcmds@zapspace#1 #2\@nil{%
  \ltx@ifempty{#2}{%
    #1%
  }{%
    \LTXcmds@zapspace#1#2\@nil
  }%
}
%    \end{macrocode}
%    \end{macro}
%
% \subsection{\cs{ltx@IfBoxEmpty}}
%
%    In case of \hologo{eTeX} the test for an empty box
%    is done via \cs{lastnodetype} as suggested
%    by David Kastrup \cite{ifboxempty:dak}.
%    \begin{macrocode}
\ltx@IfUndefined{lastnodetype}{%
  \catcode`\$=9 %
  \catcode`\&=14 %
}{%
  \catcode`\$=14 %
  \catcode`\&=9 %
}
%    \end{macrocode}
%    \begin{macro}{\ltx@IfBoxEmpty}
%    \begin{macrocode}
\def\ltx@IfBoxEmpty#1{%
  \ifvoid#1\relax
    \expandafter\ltx@secondoftwo
  \else
%    \end{macrocode}
%    Implementation using \hologo{eTeX}'s \cs{lastnodetype}.
%    \begin{macrocode}
&   \begingroup
&     \setbox\ltx@zero=\ifhbox#1\hbox\else\vbox\fi{%
&       \ifhmode\unhcopy\else\unvcopy\fi#1\relax
&       \expandafter
&     }%
&   \expandafter\endgroup
&   \ifnum\lastnodetype<\ltx@zero
&     \expandafter\expandafter\expandafter\ltx@firstoftwo
&   \else
&     \expandafter\expandafter\expandafter\ltx@secondoftwo
&   \fi
%    \end{macrocode}
%    Implementation without \hologo{eTeX} using a signature
%    at the beginning of the test box.
%    \begin{macrocode}
$   \begingroup
$     \setbox\ltx@zero=\ifhbox#1\hbox\else\vbox\fi{%
$       \penalty\ltx@one
$       \ifhmode\unhcopy\else\unvcopy\fi#1\relax
$       \expandafter
$     }%
$     \ifnum\lastpenalty=\ltx@one
%    \end{macrocode}
%    Box 0 has been changed and is restored by
%    closing the group.
%    \begin{macrocode}
$       \endgroup
$       \begingroup
$       \setbox\ltx@zero=\ifhbox#1\hbox\else\vbox\fi{%
$         \penalty\ltx@two
$         \ifhmode\unhcopy\else\unvcopy\fi#1\relax
$         \expandafter
$       }%
$       \ifnum\lastpenalty=\ltx@two
$         \def\next{\endgroup\expandafter\ltx@firstoftwo}%
$       \else
$         \def\next{\endgroup\expandafter\ltx@secondoftwo}%
$       \fi
$     \else
$       \def\next{\endgroup\expandafter\ltx@secondoftwo}%
$     \fi
$   \next
  \fi
}
%    \end{macrocode}
%    \end{macro}
%
%    \begin{macro}{\ltx@IfBoxVoidOrEmpty}
%    \begin{macrocode}
\def\ltx@IfBoxVoidOrEmpty#1{%
  \ifvoid#1\relax
    \expandafter\ltx@thirdoffour
  \fi
  \ltx@IfBoxEmpty{#1}%
}
%    \end{macrocode}
%    \end{macro}
%
%    \begin{macrocode}
\LTXcmds@AtEnd%
%</package>
%    \end{macrocode}
%% \section{Installation}
%
% \subsection{Download}
%
% \paragraph{Package.} This package is available on
% CTAN\footnote{\CTANpkg{ltxcmds}}:
% \begin{description}
% \item[\CTAN{macros/latex/contrib/ltxcmds/ltxcmds.dtx}] The source file.
% \item[\CTAN{macros/latex/contrib/ltxcmds/ltxcmds.pdf}] Documentation.
% \end{description}
%
%
% \paragraph{Bundle.} All the packages of the bundle `ltxcmds'
% are also available in a TDS compliant ZIP archive. There
% the packages are already unpacked and the documentation files
% are generated. The files and directories obey the TDS standard.
% \begin{description}
% \item[\CTANinstall{install/macros/latex/contrib/ltxcmds.tds.zip}]
% \end{description}
% \emph{TDS} refers to the standard ``A Directory Structure
% for \TeX\ Files'' (\CTANpkg{tds}). Directories
% with \xfile{texmf} in their name are usually organized this way.
%
% \subsection{Bundle installation}
%
% \paragraph{Unpacking.} Unpack the \xfile{ltxcmds.tds.zip} in the
% TDS tree (also known as \xfile{texmf} tree) of your choice.
% Example (linux):
% \begin{quote}
%   |unzip ltxcmds.tds.zip -d ~/texmf|
% \end{quote}
%
% \subsection{Package installation}
%
% \paragraph{Unpacking.} The \xfile{.dtx} file is a self-extracting
% \docstrip\ archive. The files are extracted by running the
% \xfile{.dtx} through \plainTeX:
% \begin{quote}
%   \verb|tex ltxcmds.dtx|
% \end{quote}
%
% \paragraph{TDS.} Now the different files must be moved into
% the different directories in your installation TDS tree
% (also known as \xfile{texmf} tree):
% \begin{quote}
% \def\t{^^A
% \begin{tabular}{@{}>{\ttfamily}l@{ $\rightarrow$ }>{\ttfamily}l@{}}
%   ltxcmds.sty & tex/generic/ltxcmds/ltxcmds.sty\\
%   ltxcmds.pdf & doc/latex/ltxcmds/ltxcmds.pdf\\
%   ltxcmds.dtx & source/latex/ltxcmds/ltxcmds.dtx\\
% \end{tabular}^^A
% }^^A
% \sbox0{\t}^^A
% \ifdim\wd0>\linewidth
%   \begingroup
%     \advance\linewidth by\leftmargin
%     \advance\linewidth by\rightmargin
%   \edef\x{\endgroup
%     \def\noexpand\lw{\the\linewidth}^^A
%   }\x
%   \def\lwbox{^^A
%     \leavevmode
%     \hbox to \linewidth{^^A
%       \kern-\leftmargin\relax
%       \hss
%       \usebox0
%       \hss
%       \kern-\rightmargin\relax
%     }^^A
%   }^^A
%   \ifdim\wd0>\lw
%     \sbox0{\small\t}^^A
%     \ifdim\wd0>\linewidth
%       \ifdim\wd0>\lw
%         \sbox0{\footnotesize\t}^^A
%         \ifdim\wd0>\linewidth
%           \ifdim\wd0>\lw
%             \sbox0{\scriptsize\t}^^A
%             \ifdim\wd0>\linewidth
%               \ifdim\wd0>\lw
%                 \sbox0{\tiny\t}^^A
%                 \ifdim\wd0>\linewidth
%                   \lwbox
%                 \else
%                   \usebox0
%                 \fi
%               \else
%                 \lwbox
%               \fi
%             \else
%               \usebox0
%             \fi
%           \else
%             \lwbox
%           \fi
%         \else
%           \usebox0
%         \fi
%       \else
%         \lwbox
%       \fi
%     \else
%       \usebox0
%     \fi
%   \else
%     \lwbox
%   \fi
% \else
%   \usebox0
% \fi
% \end{quote}
% If you have a \xfile{docstrip.cfg} that configures and enables \docstrip's
% TDS installing feature, then some files can already be in the right
% place, see the documentation of \docstrip.
%
% \subsection{Refresh file name databases}
%
% If your \TeX~distribution
% (\TeX\,Live, \mikTeX, \dots) relies on file name databases, you must refresh
% these. For example, \TeX\,Live\ users run \verb|texhash| or
% \verb|mktexlsr|.
%
% \subsection{Some details for the interested}
%
% \paragraph{Unpacking with \LaTeX.}
% The \xfile{.dtx} chooses its action depending on the format:
% \begin{description}
% \item[\plainTeX:] Run \docstrip\ and extract the files.
% \item[\LaTeX:] Generate the documentation.
% \end{description}
% If you insist on using \LaTeX\ for \docstrip\ (really,
% \docstrip\ does not need \LaTeX), then inform the autodetect routine
% about your intention:
% \begin{quote}
%   \verb|latex \let\install=y\input{ltxcmds.dtx}|
% \end{quote}
% Do not forget to quote the argument according to the demands
% of your shell.
%
% \paragraph{Generating the documentation.}
% You can use both the \xfile{.dtx} or the \xfile{.drv} to generate
% the documentation. The process can be configured by the
% configuration file \xfile{ltxdoc.cfg}. For instance, put this
% line into this file, if you want to have A4 as paper format:
% \begin{quote}
%   \verb|\PassOptionsToClass{a4paper}{article}|
% \end{quote}
% An example follows how to generate the
% documentation with pdf\LaTeX:
% \begin{quote}
%\begin{verbatim}
%pdflatex ltxcmds.dtx
%makeindex -s gind.ist ltxcmds.idx
%pdflatex ltxcmds.dtx
%makeindex -s gind.ist ltxcmds.idx
%pdflatex ltxcmds.dtx
%\end{verbatim}
% \end{quote}
%
% \begin{thebibliography}{9}
% \raggedright
%
% \bibitem{ifempty:schneck}
% Robert R. Schneck:
% \textit{Re: \cs{ifempty} solution %
%   (was Macro puzzle: maximally general \cs{ifempty})};
% newsgroup \xnewsgroup{comp.text.tex},
% \url{news:3eef1ada_6@corp.newsgroups.com},
% 2003-06-17.\\
% {\small
% \url{https://groups.google.com/group/comp.text.tex/msg/be03a159ec374895}}
%
% \bibitem{ifnull:diez}
% Ulrich Diez:
% \textit{Re: TeX refuses to strip outer braces in argument};
% newsgroup \xnewsgroup{comp.text.tex},
% \url{news:ibk3t8$ee7$1@news.albasani.net},
% 2010-11-12.\\
% {\small
% \url{https://groups.google.com/group/comp.text.tex/msg/803bd57221a04996}}
%
% \bibitem{ifstrempty:musa}
% Ahmed Musa:
% \textit{Re: TeX refuses to strip outer braces in argument};
% newsgroup \xnewsgroup{comp.text.tex},
% \url{news:f5496afe-40ed-42bd-b629-a2419ecf7c0d@o14g2000prn.googlegroups.com},
% 2010-12-03.\\
% {\small
% \url{https://groups.google.com/group/comp.text.tex/msg/fbf7d61a0c3a807d}}
%
% \bibitem{ifstrempty:diez1}
% Ulrich Diez:
% \textit{Re: TeX refuses to strip outer braces in argument};
% newsgroup \xnewsgroup{comp.text.tex},
% \url{news:idbo94$uka$1@four.albasani.net},
% 2010-12-03.\\
% {\small
% \url{https://groups.google.com/group/comp.text.tex/msg/0c230ee479487962}}
%
% \bibitem{ifstrempty:diez2}
% Ulrich Diez:
% \textit{Re: TeX refuses to strip outer braces in argument};
% newsgroup \xnewsgroup{comp.text.tex},
% \url{news:idbpu4$cg1$1@news.albasani.net},
% 2010-12-03.\\
% {\small
% \url{https://groups.google.com/group/comp.text.tex/msg/bbef4263390d647b}}
%
% \bibitem{ifstrempty:diez3}
% Ulrich Diez:
% \textit{Re: TeX refuses to strip outer braces in argument};
% newsgroup \xnewsgroup{comp.text.tex},
% \url{news:idd4ga$r83$1@four.albasani.net},
% 2010-12-04.\\
% {\small
% \url{https://groups.google.com/group/comp.text.tex/msg/00dfd1ec103cd272}}
%
% \bibitem{ifstrempty:gl}
% GL:
% \textit{Re: TeX refuses to strip outer braces in argument};
% newsgroup \xnewsgroup{comp.text.tex},
% \url{news:4cfa2e27$0$7389$426a74cc@news.free.fr},
% 2010-12-04.\\
% {\small
% \url{https://groups.google.com/group/comp.text.tex/msg/d3a75995c1cf267e}}
%
% \bibitem{ifstrempty:ho}
% Heiko Oberdiek:
% \textit{Re: TeX refuses to strip outer braces in argument};
% newsgroup \xnewsgroup{comp.text.tex},
% \url{news:iddhq1$3kj$1@news.eternal-september.org},
% 2010-12-04.\\
% {\small
% \url{https://groups.google.com/group/comp.text.tex/msg/5f7a23e3ab70e347}}
%
% \bibitem{ifboxempty:dak}
% David Kastrup:
% \textit{How to detect if \cs{vbox} is empty};
% newsgroup \xnewsgroup{comp.text.tex},
% 2011-02-04.\\
% {\small
% \url{https://groups.google.com/group/comp.text.tex/msg/8d3cb89496a4d86d}}
%
% \end{thebibliography}
%
% \begin{History}
%   \begin{Version}{2009/08/05 v1.0}
%   \item
%     First version.
%   \end{Version}
%   \begin{Version}{2009/12/12 v1.1}
%   \item
%     Short title shortened.
%   \item
%     \cs{ltx@IfUndefined} added.
%   \end{Version}
%   \begin{Version}{2010/01/28 v1.2}
%   \item
%     \cs{ltx@RemovePrefix} and \cs{ltx@StripPrefix} added.
%   \item
%     \cs{ltx@ifclassloaded}, \cs{ltx@ifpackageloaded},
%     \cs{ltx@iffileloaded},
%     \cs{ltx@ifclasslater}, \cs{ltx@ifpackagelater},
%     \cs{ltx@iffilelater},
%     \cs{ltx@clsextension}, \cs{ltx@pkgextension} added.
%   \item
%     \cs{ltx@GlobalAppendToMacro}, \cs{ltx@LocalAppendToMacro} added.
%   \end{Version}
%   \begin{Version}{2010/03/01 v1.3}
%   \item
%     \cs{ltx@newif} added.
%   \item
%     \cs{ltx@ifnextchar} added.
%   \item
%     Numbers \cs{ltx@zero}, \cs{ltx@one}, \cs{ltx@two}, \cs{ltx@cclv} added.
%   \end{Version}
%   \begin{Version}{2010/03/09 v1.4}
%   \item
%     \cs{ltx@pkgextension} and \cs{ltx@clsextension} are hardcoded to
%     avoid trouble with \cs{@onlypreamble}.
%   \end{Version}
%   \begin{Version}{2010/04/08 v1.5}
%   \item
%     \cs{ltx@cartwo}, \cs{ltx@cdrtwo},
%     \cs{ltx@carthree}, \cs{ltx@cdrthree},
%     \cs{ltx@carfour}, \cs{ltx@cdrfour} added.
%   \item
%     \cs{ltx@ReturnAfterFi} and \cs{ltx@ReturnAfterElseFi} fixed.
%   \end{Version}
%   \begin{Version}{2010/04/16 v1.6}
%   \item
%     \cs{ltx@leavevmode}, \cs{ltx@mbox} added.
%   \end{Version}
%   \begin{Version}{2010/04/26 v1.7}
%   \item
%     \cs{ltx@GobbleNum}, \cs{ltx@CdrNum}, \cs{ltx@CarNum} added.
%   \item
%     \cs{ltx@carzero}, \cs{ltx@cdrzero} added.
%   \item
%     \cs{ltx@hashchar} added.
%   \end{Version}
%   \begin{Version}{2010/09/11 v1.8}
%   \item
%     \cs{ltx@leftbracechar}, \cs{ltx@rightbracechar} added.
%   \end{Version}
%   \begin{Version}{2010/10/25 v1.9}
%   \item
%     \cs{ltx@LocalAppendToMacro} and \cs{ltx@GlobalAppendToMacro} are
%     now \cs{long}.
%   \end{Version}
%   \begin{Version}{2010/10/31 v1.10}
%   \item
%     \cs{ltx@newglobalif} added.
%   \end{Version}
%   \begin{Version}{2010/11/12 v1.11}
%   \item
%     \cs{ltx@ifempty} added.
%   \item
%     \cs{ltx@firstofthree}, \cs{ltx@secondofthree}, \cs{ltx@thirdofthree}
%     added.
%   \end{Version}
%   \begin{Version}{2010/12/02 v1.12}
%   \item
%     \cs{ltx@onelevel@sanitize} added.
%   \item
%     \cs{LTXcmds@num} fixed for the case with \cs{numexpr}
%     (bug found by GL).
%   \end{Version}
%   \begin{Version}{2010/12/04 v1.13}
%   \item
%     \cs{ltx@ifblank} added.
%   \item
%     Optimization for \cs{ltx@ifempty}.
%   \end{Version}
%   \begin{Version}{2010/12/07 v1.14}
%   \item
%     \cs{ltx@zapspace} added.
%   \end{Version}
%   \begin{Version}{2010/12/12 v1.15}
%   \item
%     \cs{ltx@minusone} added.
%   \end{Version}
%   \begin{Version}{2011/02/04 v1.16}
%   \item
%     \cs{ltx@IfBoxEmpty} and \cs{ltx@IfBoxVoidOrEmpty} added.
%   \item
%     \cs{ltx@firstoffour}, \dots, \cs{ltx@fourthoffour} added.
%   \end{Version}
%   \begin{Version}{2011/02/05 v1.17}
%   \item
%     \cs{ltx@IfBoxEmpty}: an empty box may have non-zero dimensions.
%   \end{Version}
%   \begin{Version}{2011/03/16 v1.18}
%   \item
%     \cs{ltx@ifclasslater} fixed.
%   \end{Version}
%   \begin{Version}{2011/04/14 v1.19}
%   \item
%     \cs{ltx@ifnextchar}: detection of optional spaces modified.
%   \item
%     \cs{ltx(Loc,Glob)(Toks,Dimen,Skip)(A,B,C,D,E)} added.
%   \end{Version}
%   \begin{Version}{2011/04/18 v1.20}
%   \item
%     \cs{ltx@ifnextchar} with conditional support (thanks GL for
%     bug report).
%   \end{Version}
%   \begin{Version}{2011/08/22 v1.21}
%   \item
%     \cs{ltx@GlobalPrependToMacro}, \cs{ltx@LocalPrependToMacro} added
%     (feature request of Martin M\"unch).
%   \end{Version}
%   \begin{Version}{2011/11/09 v1.22}
%   \item
%     \cs{ltx@carsecond}, \cs{ltx@carthird}, \cs{ltx@carfourth},
%     \cs{ltx@CarNumth} added.
%   \item
%     \cs{ltx@cdrzero}, \cs{ltx@cdr}, \cs{ltx@cdrtwo}, cs{ltx@cdrthree},
%     \cs{ltx@cdrfour}, \cs{ltx@CdrNum} modified to retain braces and spaces.
%     They are expandable in two expansion steps.
%   \end{Version}
%   \begin{Version}{2016/05/16 v1.23}
%   \item
%     Documentation updates.
%   \end{Version}
%   \begin{Version}{2019/12/15 v1.24}
%   \item
%     Documentation updates.
%   \end{Version}
% \end{History}
%
% \PrintIndex
%
% \Finale
\endinput

%        (quote the arguments according to the demands of your shell)
%
% Documentation:
%    (a) If ltxcmds.drv is present:
%           latex ltxcmds.drv
%    (b) Without ltxcmds.drv:
%           latex ltxcmds.dtx; ...
%    The class ltxdoc loads the configuration file ltxdoc.cfg
%    if available. Here you can specify further options, e.g.
%    use A4 as paper format:
%       \PassOptionsToClass{a4paper}{article}
%
%    Programm calls to get the documentation (example):
%       pdflatex ltxcmds.dtx
%       makeindex -s gind.ist ltxcmds.idx
%       pdflatex ltxcmds.dtx
%       makeindex -s gind.ist ltxcmds.idx
%       pdflatex ltxcmds.dtx
%
% Installation:
%    TDS:tex/generic/ltxcmds/ltxcmds.sty
%    TDS:doc/latex/ltxcmds/ltxcmds.pdf
%    TDS:source/latex/ltxcmds/ltxcmds.dtx
%
%<*ignore>
\begingroup
  \catcode123=1 %
  \catcode125=2 %
  \def\x{LaTeX2e}%
\expandafter\endgroup
\ifcase 0\ifx\install y1\fi\expandafter
         \ifx\csname processbatchFile\endcsname\relax\else1\fi
         \ifx\fmtname\x\else 1\fi\relax
\else\csname fi\endcsname
%</ignore>
%<*install>
\input docstrip.tex
\Msg{************************************************************************}
\Msg{* Installation}
\Msg{* Package: ltxcmds 2019/12/15 v1.24 LaTeX kernel commands for general use (HO)}
\Msg{************************************************************************}

\keepsilent
\askforoverwritefalse

\let\MetaPrefix\relax
\preamble

This is a generated file.

Project: ltxcmds
Version: 2019/12/15 v1.24

Copyright (C)
   2009-2011 Heiko Oberdiek
   2016-2019 Oberdiek Package Support Group

This work may be distributed and/or modified under the
conditions of the LaTeX Project Public License, either
version 1.3c of this license or (at your option) any later
version. This version of this license is in
   https://www.latex-project.org/lppl/lppl-1-3c.txt
and the latest version of this license is in
   https://www.latex-project.org/lppl.txt
and version 1.3 or later is part of all distributions of
LaTeX version 2005/12/01 or later.

This work has the LPPL maintenance status "maintained".

The Current Maintainers of this work are
Heiko Oberdiek and the Oberdiek Package Support Group
https://github.com/ho-tex/ltxcmds/issues


The Base Interpreter refers to any `TeX-Format',
because some files are installed in TDS:tex/generic//.

This work consists of the main source file ltxcmds.dtx
and the derived files
   ltxcmds.sty, ltxcmds.pdf, ltxcmds.ins, ltxcmds.drv,
   ltxcmds-test1.tex, ltxcmds-test-gobble.tex,
   ltxcmds-test-ifempty.tex, ltxcmds-test-zapspace.tex,
   ltxcmds-test-ifboxempty.tex, ltxcmds-test-nextchar.tex,
   ltxcmds-test-carcdr.tex.

\endpreamble
\let\MetaPrefix\DoubleperCent

\generate{%
  \file{ltxcmds.ins}{\from{ltxcmds.dtx}{install}}%
  \file{ltxcmds.drv}{\from{ltxcmds.dtx}{driver}}%
  \usedir{tex/generic/ltxcmds}%
  \file{ltxcmds.sty}{\from{ltxcmds.dtx}{package}}%
}

\catcode32=13\relax% active space
\let =\space%
\Msg{************************************************************************}
\Msg{*}
\Msg{* To finish the installation you have to move the following}
\Msg{* file into a directory searched by TeX:}
\Msg{*}
\Msg{*     ltxcmds.sty}
\Msg{*}
\Msg{* To produce the documentation run the file `ltxcmds.drv'}
\Msg{* through LaTeX.}
\Msg{*}
\Msg{* Happy TeXing!}
\Msg{*}
\Msg{************************************************************************}

\endbatchfile
%</install>
%<*ignore>
\fi
%</ignore>
%<*driver>
\NeedsTeXFormat{LaTeX2e}
\ProvidesFile{ltxcmds.drv}%
  [2019/12/15 v1.24 LaTeX kernel commands for general use (HO)]%
\documentclass{ltxdoc}
\usepackage{holtxdoc}[2011/11/22]
\usepackage{zref-savepos}
\usepackage{paralist}
\hfuzz=1pt
\begin{document}
  \DocInput{ltxcmds.dtx}%
\end{document}
%</driver>
% \fi
%
%
%
% \GetFileInfo{ltxcmds.drv}
%
% \title{The \xpackage{ltxcmds} package}
% \date{2019/12/15 v1.24}
% \author{Heiko Oberdiek\thanks
% {Please report any issues at \url{https://github.com/ho-tex/ltxcmds/issues}}}
%
% \maketitle
%
% \begin{abstract}
% The package \xpackage{ltxcmds} exports some utility macros
% from the \LaTeX\ kernel into a separate namespace and
% also provides them for other formats such as plain-\TeX.
% \end{abstract}
%
% \tableofcontents
%
% \newcounter{to}
% \renewcommand*{\theto}{to\arabic{to}}
% \newlength\tolength
% \tolength=\dimexpr .5\linewidth+\oddsidemargin+1in+15mm\relax
% \newcommand*{\TO}{^^A
%   \quad
%   \stepcounter{to}^^A
%   \zsavepos\theto
%   \ifdim\zposx\theto sp>0pt
%     \ifdim\zposx\theto sp<\tolength
%       \kern\dimexpr\tolength-\zposx\theto sp\relax
%     \fi
%   \fi
%   $\rightarrow$\quad
%   \ignorespaces
% }
%
% \section{Documentation}
%
% \subsection{Introduction}
%
% Many of my packages also support other formats such as plain-\TeX.
% Because I am rather familiar with the utility macros from
% \LaTeX's kernel (e.g. \cs{@gobble}, \cs{@firstoftwo}), I found
% myself rewriting them again and again, because they are lacking
% in plain-\TeX.
%
% Therefore this package provides often used macros and similar
% ones with the name prefix \cs{ltx@}. This avoids also faulty
% redefinitions. I remember an example where a package redefined
% \cs{@firstoftwo} with forgetting \cs{long}.
%
% \subsection{Numbers}
%
% \begin{declcs}{ltx@zero} \TO 0\\
%   \cs{ltx@one} \TO 1\\
%   \cs{ltx@two} \TO 2\\
%   \cs{ltx@cclv} \TO 255\\
%   \cs{ltx@minusone} \TO -1
% \end{declcs}
% These commands are numbers 0, 1, 2, 255 and -1. They are not digits
% and a space is not gobbled afterwards. Macro \cs{ltx@minusone}
% is available since version 2010/12/12 v1.15.
%
% \subsection{Scratch registers}
%
% Following the conventions of \hologo{plainTeX} and \hologo{LaTeX}
% the first ten registers are free to use. Even numbered registers
% are for local, odd numbered for global use.
%
% \begin{declcs}{ltx@(Loc,Glob)(Toks,Dimen,Skip)(A,B,C,D,E)}
% \end{declcs}
% The name consists of the prefix \cs{ltx@}, then
% \texttt{Loc} or \texttt{Glob} for local or global usage follows.
% The register type is given by \texttt{Toks} for token register,
% \texttt{Dimen} for dimen register and \texttt{Skip} for skip
% register. As last part the registers are numbered from \texttt{A}
% to \texttt{E}. Example: \cs{ltx@LocToksA}.
%
% Since 2011/04/14 v1.19.
%
% \subsection{Argument killers}
%
% \begin{declcs}{ltx@gobble} \M{1} \TO\\
%   \cs{ltx@gobbletwo} \M{1} \M{2} \TO\\
%   \cs{ltx@gobblethree} \M{1} \M{2} \M{3} \TO\\
%   \cs{ltx@gobblefour} \M{1} \M{2} \M{3} \M{4} \TO
% \end{declcs}
%
% \begin{declcs}{ltx@GobbleNum} \M{num} \M{1} \M{2} \dots\space
%   \M{\meta{num}} \TO
% \end{declcs}
% The first argument \meta{num} of macro \cs{ltx@GobbleNum} specifies,
% how many following arguments are eaten. Macro \cs{ltx@GobbleNum}
% is expandable in exact two expansion steps.
%
% \subsection{Argument grabbers}
%
% \begin{declcs}{ltx@firstofone} \M{1} \TO \meta{1}\\[1ex]
%  \cs{ltx@firstoftwo} \M{1} \M{2} \TO \meta{1}\\
%  \cs{ltx@secondoftwo} \M{1} \M{2} \TO \meta{2}\\[1ex]
%  \cs{ltx@firstofthree} \M{1} \M{2} \M{3} \TO \meta{1}\\
%  \cs{ltx@secondofthree} \M{1} \M{2} \M{3} \TO \meta{2}\\
%  \cs{ltx@thirdofthree} \M{1} \M{2} \M{3} \TO \meta{3}\\
%  \cs{ltx@firstoffour} \M{1} \M{2} \M{3} \M{4} \TO \meta{1}\\
%  \cs{ltx@secondoffour} \M{1} \M{2} \M{3} \M{4} \TO \meta{2}\\
%  \cs{ltx@thirdoffour} \M{1} \M{2} \M{3} \M{4} \TO \meta{3}\\
%  \cs{ltx@fourthoffour} \M{1} \M{2} \M{3} \M{4} \TO \meta{4}
% \end{declcs}
% Macros \cs{ltx@firstofthree}, \cs{ltx@secondofthree}
% and \cs{ltx@thirdofthree} were added in version
% 2010/11/12 v1.11.
% Macros \cs{ltx@firstoffour}, \dots, \cs{ltx@fourthoffour}
% were added in version 2011/02/04 v1.16.
%
% \subsection{List helpers}
%
% \begin{declcs}{ltx@carzero} \dots\ \cs{@nil} \TO\\
%   \cs{ltx@cdrzero} \dots\ \cs{@nil} \TO \dots
% \end{declcs}
%
% \begin{declcs}{ltx@car} \M{1} \dots\ \cs{@nil} \TO \meta{1}\\
%   \cs{ltx@cdr} \M{1} \dots\ \cs{@nil} \TO \dots
% \end{declcs}
%
% \begin{declcs}{ltx@cartwo} \M{1} \M{2} \dots\ \cs{@nil}
%   \TO \meta{1}\meta{2}\\
%   \cs{ltx@carsecond} \M{1} \M{2} \dots\ \cs{@nil} \TO \meta{2}\\
%   \cs{ltx@cdrtwo} \M{1} \M{2} \dots\ \cs{@nil} \TO \dots
% \end{declcs}
%
% \begin{declcs}{ltx@carthree} \M{1} \M{2} \M{3} \dots\ \cs{@nil}
%       \TO \meta{1}\meta{2}\meta{3}\\
%   \cs{ltx@carthird} \M{1} \M{2} \M{3} \dots\ \cs{@nil}
%       \TO \meta{3}\\
%   \cs{ltx@cdrthree} \M{1} \M{2} \M{3} \dots\ \cs{@nil} \TO \dots
% \end{declcs}
%
% \begin{declcs}{ltx@carfour} \M{1} \M{2} \M{3} \M{4} \dots\ \cs{@nil}
%       \TO \meta{1}\meta{2}\meta{3}\meta{4}\\
%   \cs{ltx@carfourth} \M{1} \M{2} \M{3} \M{4} \dots\ \cs{@nil}
%       \TO \meta{4}\\
%   \cs{ltx@cdrfour} \M{1} \M{2} \M{3} \M{4} \dots\ \cs{@nil} \TO \dots
% \end{declcs}
%
% \begin{declcs}{ltx@CarNum} \M{num} \M{1} \dots\ \M{\meta{num}}
%   \M{\meta{num}+1} \dots\ \cs{@nil}\\
%   \TO \M{1} \dots\ \M{\meta{num}} \dots\\
%   \cs{ltx@CarNumth} \M{num} \M{1} \dots\ \M{\meta{num}}
%   \M{\meta{num}+1} \dots\ \cs{@nil}\\
%   \TO \M{\meta{num}} \dots\\
%   \cs{ltx@CdrNum} \M{num} \M{1} \dots\ \M{\meta{num}}
%   \M{\meta{num}+1} \dots\ \cs{@nil}\\
%   \TO \M{\meta{num}+1} \dots
% \end{declcs}
%
% Macros with uppercase letters are expandable in two expansion steps.
% Changes in version 2019/12/15 v1.24:
% \begin{itemize}
% \item
%   Macros \cs{ltx@carsecond}, \cs{ltx@carthird}, \cs{ltx@carfourth},
%   \cs{ltx@CarNumth} added.
% \item
%   Macros \cs{ltx@cdr}, \cs{ltx@cdrtwo}, \cs{ltx@cdrthree}, \cs{ltx@cdrfour},
%   \cs{ltx@CdrNum} are expandable in two expansion steps and retain
%   spaces and braces after the first gobbled arguments.
% \end{itemize}
%
% \subsection{Tail recursion}
%
% \begin{declcs}{ltx@ReturnAfterFi} \M{1} \cs{fi} \TO \cs{fi} \meta{1}\\
%   \cs{ltx@ReturnAfterElseFi} \M{1} \cs{else} \M{2} \cs{fi}
%     \TO \cs{fi} \meta{1}
% \end{declcs}
%
% \subsection{Empty macro}
%
% \begin{declcs}{ltx@empty} \TO
% \end{declcs}
%
% \subsection{Characters}
%
% \begin{declcs}{ltx@space} \TO \verb*| |\\
%   \cs{ltx@percentchar} \TO \texttt{\%}\\
%   \cs{ltx@backslashchar} \TO
%      \makeatletter\texttt{\@backslashchar}\\
%   \cs{ltx@hashchar} \TO \texttt{\#}\qquad(since v1.7)\\
%   \cs{ltx@leftbracechar} \TO \texttt{\textbraceleft}\qquad(since v1.8)\\
%   \cs{ltx@rightbracechar} \TO \texttt{\textbraceright}\qquad(since v1.8)
% \end{declcs}
%
% \subsection{Boolean switch}
%
% \begin{declcs}{ltx@newif} \M{cmd}
% \end{declcs}
% \cs{ltx@newif} defines a new boolean switch \meta{cmd} like \cs{newif}.
% Unlike \plainTeX's \cs{newif}, \cs{ltx@newif} is not \cs{outer}.
% The command \meta{cmd} must start with the two characters |if|.
%
% \begin{declcs}{ltx@newglobalif} \M{cmd}
% \end{declcs}
% \cs{ltx@newglobalif} defines a new boolean switch \meta{cmd}
% like \cs{ltx@newif}. However the switch setting commands,
% \meta{cmd} without the prefix |if| and followed by |true| or |false|
% are acting globally.
%
% \subsection{Command definitions}
%
% \begin{declcs}{ltx@ifundefined} \M{cmd} \M{yes} \M{no}
% \end{declcs}
% If \eTeX\ is available, \cs{ifcsname} is used that does
% not have the side effect of defining undefined commands
% with meaning of \cs{relax}. This command is always expandable.
% Change in version 1.1: Also the meaning \cs{relax} is
% always considered ``undefined''.
%
% \begin{declcs}{ltx@IfUndefined} \M{cmd} \M{yes} \M{no}
% \end{declcs}
% If \eTeX\ is available, \cs{ifcsname} is used that does
% not have the side effect of defining undefined commands
% with meaning of \cs{relax}. Also it always checks for
% the meaning of \cs{relax} and considers this as undefined.
% This macro is not expandable without \eTeX.
%
% \begin{declcs}{ltx@LocalExpandAfter}
% \end{declcs}
% It expands the token after the next token but in a local
% context. That is the difference to \cs{expandafter}.
% The local context discards the side effect of \cs{csname}
% and let the command undefined after the expansion step.
%
% \subsection{Stripping}
%
% \begin{declcs}{ltx@RemovePrefix}\\
%   \cs{ltx@StripPrefix}
% \end{declcs}
% All tokens up to and including the next available character
% `\textgreater' are thrown away. Usually it is used to strip
% the first part of the output of the commands \cs{meaning}
% or \cs{pdflastmatch}. Macro \cs{ltx@RemovePrefix}
% has the same meaning as \LaTeX's \cs{strip@prefix}, whereas
% macro \cs{ltx@StripPrefix} expands the next token
% once before stripping the prefix.
%
% \begin{declcs}{ltx@onelevel@sanitize} \M{macro}
% \end{declcs}
% Macro \cs{ltx@onelevel@sanitize} provides \hologo{LaTeX}'s
% \cs{@onelevel@sanitize}. The macro is expanded once and
% the contents is converted to characters with catcode 12 (other)
% and space tokens with catcode 10 (space). Then then sanitized
% contents is stored into the macro again. Since version 1.12.
%
% \subsection{File management}
%
% All macros in this section are expandable like
% the counterparts of the \LaTeX\ kernel. Also they
% can be used after the preamble.
%
% \subsubsection{File extensions}
%
% \begin{declcs}{ltx@clsextension}\\
%   \cs{ltx@pkgextension}
% \end{declcs}
% Macros \cs{ltx@clsextension} and \cs{ltx@styextension}
% stores the strings |cls| and |sty|. In opposite to \hologo{LaTeX}'s
% \cs{@clsextension} and \cs{@styextension} they can also be
% used after |\begin{document}|.
%
% \subsubsection{Load check}
%
% \begin{declcs}{ltx@ifclassloaded} \M{class} \M{yes} \M{no}\\
%   \cs{ltx@ifpackageloaded} \M{package} \M{yes} \M{no}
% \end{declcs}
% Macros \cs{ltx@ifclassloaded}\slash\cs{ltx@ifpackageloaded}
% execute \meta{yes}, if the \meta{class} or \meta{package} is loaded,
% otherwise \meta{no} is called.
% Both \meta{class} and \meta{package} are specified without
% extension. The macros can also be used after |\begin{document}|.
%
% \begin{declcs}{ltx@iffileloaded} \M{file} \M{yes} \M{no}
% \end{declcs}
% If \LaTeX's \cs{ProvidesFile} macro was called before
% using \meta{file} as argument, then \cs{ltx@iffileloaded}
% calls \meta{yes}, otherwise \meta{no}.
% Therefore it is possible
% that the \meta{file} is loaded, but \meta{no} is executed because
% of a missing \cs{ProvidesFile}. The \LaTeX\ kernel does not have
% a counterpart of \cs{ltx@iffileloaded}.
%
% Note that the file name used in \cs{ProvidesFile} and
% \cs{ltx@iffileloaded} must match. For example, if \TeX's
% default extension \xext{tex} was given in the first command,
% then it must also specified in the latter command and vice versa.
%
% \subsubsection{Version date check}
%
% \begin{declcs}{ltx@ifclasslater} \M{class} \M{date} \M{yes} \M{no}\\
%   \cs{ltx@ifpackagelater} \M{package} \M{date} \M{yes} \M{no}\\
%   \cs{ltx@iffilelater} \M{file} \M{date} \M{yes} \M{no}
% \end{declcs}
% If a \cs{ProvidesClass}\slash\cs{ProvidesPackage}\slash
% \cs{ProvidesFile} command with exact the same class\slash package\slash
% file was executed before with an optional
% argument that starts with a \LaTeX\ version date, then
% this version date is compared with the argument \meta{date}.
% If they are equal or if the version date is the later date,
% then \meta{yes} is called. In all other cases \meta{no} is executed.
%
% A \LaTeX\ date has the format |YYYY/MM/DD| with |YYYY| as year
% with four digits, |MM| as month with two digits and |DD| as day
% with two digits. If \pdfTeX's \cs{pdfmatch} is available, then
% it is used to detect the version date, to reject invalid
% date formats and to reject some invalid dates.
% Dates before 1994/01/01 are always invalid, because version
% dates are introduced with \LaTeXe\ in 1994.
%
% \subsection{Macro additions}
%
% \begin{declcs}{ltx@GlobalAppendToMacro} \M{cmd} \M{addition}\\
%   \cs{ltx@LocalAppendToMacro} \M{cmd} \M{addition}
% \end{declcs}
% The \meta{addition} is appended to the parameterless macro \meta{cmd}.
% If \meta{cmd} is undefined or has the meaning \cs{relax}, then
% it will be initialized as empty macro beforehand. Due to a bug \meta{addition}
% must not contain \cs{par} before version 2010/10/25 v1.9.
%
% \begin{declcs}{ltx@GlobalPrependToMacro} \M{cmd} \M{addition}\\
%   \cs{ltx@LocalPrependToMacro} \M{cmd} \M{addition}
% \end{declcs}
% The \meta{addition} is prepended to the parameterless macro \meta{cmd}.
% If \meta{cmd} is undefined or has the meaning \cs{relax}, then
% it will be initialized as empty macro beforehand. The macros
% were added in version 2011/08/22 v1.21.
%
% \subsection{Next character detection}
%
% \begin{declcs}{ltx@ifnextchar} \M{char} \M{yes} \M{no}
% \end{declcs}
% If next character is \meta{char} then \meta{yes} is called,
% otherwise \meta{no}. The character is not removed.
% Spaces are silently removed when looking for \meta{char}
% as \hologo{LaTeX}'s version \cs{kernel@ifnextchar} does.
% But there are also small differences:
% \begin{itemize}
% \item The space can be used as \meta{char}. In this case
%   optional spaces before \meta{char} are not supported of course.
% \item If the optional space is a command that is a character
%   (defined by \cs{let} or \cs{futurelet}), then
%   \cs{kernel@ifnextchar} breaks with an \hologo{TeX} error.
%   \cs{ltx@ifnextchar} silently removes this token as
%   optional space.
% \end{itemize}
% Since 2010/03/01 v1.3.
%
% \begin{declcs}{ltx@ifnextchar@nospace} \M{char} \M{yes} \M{no}
% \end{declcs}
% Macro \cs{ltx@ifnextchar@nospace} behaves like
% macro \cs{ltx@ifnextchar} with the exception that
% optional spaces are not supported before \meta{char}.
% Since 2011/04/14 v1.19.
%
% \subsection{\cs{ltx@leavevmode}, \cs{ltx@mbox}}
%
% \begin{declcs}{ltx@leavevmode}
% \end{declcs}
% Macro \cs{ltx@leavevmode} calls \hologo{pdfTeX}'s \cs{quitvmode}.
% Otherwise \cs{leavevmode} is used and defined if it is necessary.
%
% \begin{declcs}{ltx@mbox}
% \end{declcs}
% Macro \cs{ltx@mbox} reimplements \cs{mbox} with two changes.
% Instead of \cs{leavevmode} it uses \cs{ltx@leavevmode} and
% stops right after \cs{hbox}. Especially it does not grab
% the argument and allows the extended syntax of \cs{hbox}.
%
% \subsection{Expandable test for emptiness}
%
% \begin{declcs}{ltx@ifempty} \M{stuff} \M{yes} \M{no}
% \end{declcs}
% Macro \cs{ltx@ifempty} checks in exact two expansion steps
% whether \meta{stuff}
% is empty or contains token. Depending on the result
% \meta{yes} or \meta{no} is executed. The token in \meta{stuff}
% may contain \cs{par} and unmatched conditionals (\cs{if}, \cs{else},
% \cs{fi}, \dots). Since version 2010/11/12 v1.11.
%
% \begin{declcs}{ltx@ifblank} \M{stuff} \M{yes} \M{no}
% \end{declcs}
% Macro \cs{ltx@ifblank} tests in exact two expansion steps
% if \meta{stuff} is empty or contain only blank spaces.
% In this case argument \meta{yes} is called. If \meta{stuff}
% contains other tokens than spaces then \meta{no} is executed.
% Since version 2010/12/04 v1.13.
%
% \subsection{Stripping spaces}
%
% \begin{declcs}{ltx@zapspace} \M{stuff}
% \end{declcs}
% Macro \cs{ltx@zapspace} strips spaces from \meta{stuff}
% that are not hidden inside curly braces. Like \hologo{LaTeX}'s
% \cs{zap@space} it is expandable. Differences:
% \begin{itemize}
% \item
% Syntax: \cs{zap@space} also expects a space token
% and \cs{@empty} after \meta{stuff}.
% \item
% Macro \cs{ltx@zapspace} is expandable in exact two
% expansion steps.
% \item
% Macro \cs{ltx@zapspace} always retains curly braces.
% \item
% Macro \cs{zap@space} has a bug. It stops stripping spaces
% after a token group in curly braces if the first two tokens
% inside the group are equal.
% \item
% Macro \cs{ltx@zapspace} also works with \cs{par}
% and conditionals (\cs{if}, \cs{else}, \cs{fi}, \dots).
% \end{itemize}
% Macro \cs{ltx@zapspace} is available since version
% 2010/12/07 v1.14.
%
% \subsection{Check for emptiness of boxes}
%
% \begin{declcs}{ltx@IfBoxEmpty} \M{box register number}
%   \M{yes} \M{no}
% \end{declcs}
% Macro \cs{ltx@IfBoxEmpty} calls \meta{yes} if
% the box exists (\cs{ifvoid} returns false) and the box
% does not contain any content. Otherwise if the box
% is void or contains something, then \meta{no} is executed.
% Thus being empty means that the box exists and is either
% an \cs{hbox} or a \cs{vbox} and may even have dimensions other than
% 0.0\,pt, but the box does not contain anything.
% Macro \cs{ltx@IfBoxEmpty} is available since 2010/02/04 v1.16.
%
% \begin{declcs}{ltx@IfBoxVoidOrEmpty} \M{box register number}
%   \M{yes} \M{no}
% \end{declcs}
% Macro \cs{ltx@IfBoxVoidOrEmpty} calls \meta{yes} if
% the box is either void or does not contain any content.
% Otherwise \meta{no} is executed.
% Macro \cs{ltx@IfBoxVoidOrEmpty} is available since 2010/02/04 v1.16.
%
% \StopEventually{
% }
%
% \section{Implementation}
%
% \subsection{Identification}
%
%    \begin{macrocode}
%<*package>
%    \end{macrocode}
%    Reload check, especially if the package is not used with \LaTeX.
%    \begin{macrocode}
\begingroup\catcode61\catcode48\catcode32=10\relax%
  \catcode13=5 % ^^M
  \endlinechar=13 %
  \catcode35=6 % #
  \catcode39=12 % '
  \catcode44=12 % ,
  \catcode45=12 % -
  \catcode46=12 % .
  \catcode58=12 % :
  \catcode64=11 % @
  \catcode123=1 % {
  \catcode125=2 % }
  \expandafter\let\expandafter\x\csname ver@ltxcmds.sty\endcsname
  \ifx\x\relax % plain-TeX, first loading
  \else
    \def\empty{}%
    \ifx\x\empty % LaTeX, first loading,
      % variable is initialized, but \ProvidesPackage not yet seen
    \else
      \expandafter\ifx\csname PackageInfo\endcsname\relax
        \def\x#1#2{%
          \immediate\write-1{Package #1 Info: #2.}%
        }%
      \else
        \def\x#1#2{\PackageInfo{#1}{#2, stopped}}%
      \fi
      \x{ltxcmds}{The package is already loaded}%
      \aftergroup\endinput
    \fi
  \fi
\endgroup%
%    \end{macrocode}
%    Package identification:
%    \begin{macrocode}
\begingroup\catcode61\catcode48\catcode32=10\relax%
  \catcode13=5 % ^^M
  \endlinechar=13 %
  \catcode35=6 % #
  \catcode39=12 % '
  \catcode40=12 % (
  \catcode41=12 % )
  \catcode44=12 % ,
  \catcode45=12 % -
  \catcode46=12 % .
  \catcode47=12 % /
  \catcode58=12 % :
  \catcode64=11 % @
  \catcode91=12 % [
  \catcode93=12 % ]
  \catcode123=1 % {
  \catcode125=2 % }
  \expandafter\ifx\csname ProvidesPackage\endcsname\relax
    \def\x#1#2#3[#4]{\endgroup
      \immediate\write-1{Package: #3 #4}%
      \xdef#1{#4}%
    }%
  \else
    \def\x#1#2[#3]{\endgroup
      #2[{#3}]%
      \ifx#1\@undefined
        \xdef#1{#3}%
      \fi
      \ifx#1\relax
        \xdef#1{#3}%
      \fi
    }%
  \fi
\expandafter\x\csname ver@ltxcmds.sty\endcsname
\ProvidesPackage{ltxcmds}%
  [2019/12/15 v1.24 LaTeX kernel commands for general use (HO)]%
%    \end{macrocode}
%
%    \begin{macrocode}
\begingroup\catcode61\catcode48\catcode32=10\relax%
  \catcode13=5 % ^^M
  \endlinechar=13 %
  \catcode123=1 % {
  \catcode125=2 % }
  \catcode64=11 % @
  \def\x{\endgroup
    \expandafter\edef\csname LTXcmds@AtEnd\endcsname{%
      \endlinechar=\the\endlinechar\relax
      \catcode13=\the\catcode13\relax
      \catcode32=\the\catcode32\relax
      \catcode35=\the\catcode35\relax
      \catcode61=\the\catcode61\relax
      \catcode64=\the\catcode64\relax
      \catcode123=\the\catcode123\relax
      \catcode125=\the\catcode125\relax
    }%
  }%
\x\catcode61\catcode48\catcode32=10\relax%
\catcode13=5 % ^^M
\endlinechar=13 %
\catcode35=6 % #
\catcode64=11 % @
\catcode123=1 % {
\catcode125=2 % }
\def\TMP@EnsureCode#1#2{%
  \edef\LTXcmds@AtEnd{%
    \LTXcmds@AtEnd
    \catcode#1=\the\catcode#1\relax
  }%
  \catcode#1=#2\relax
}
\TMP@EnsureCode{36}{3}% $
\TMP@EnsureCode{38}{4}% &
\TMP@EnsureCode{40}{12}% (
\TMP@EnsureCode{41}{12}% )
\TMP@EnsureCode{45}{12}% -
\TMP@EnsureCode{46}{12}% .
\TMP@EnsureCode{47}{12}% /
\TMP@EnsureCode{60}{12}% <
\TMP@EnsureCode{62}{12}% >
\TMP@EnsureCode{91}{12}% [
\TMP@EnsureCode{96}{12}% `
\TMP@EnsureCode{93}{12}% ]
\TMP@EnsureCode{94}{12}% ^ (superscript) (!)
\TMP@EnsureCode{124}{12}% |
\edef\LTXcmds@AtEnd{\LTXcmds@AtEnd\noexpand\endinput}
%    \end{macrocode}
%
% \subsection{Numbers}
%
%    \begin{macro}{\ltx@zero}
%    \begin{macrocode}
\chardef\ltx@zero=0 %
%    \end{macrocode}
%    \end{macro}
%    \begin{macro}{\ltx@one}
%    \begin{macrocode}
\chardef\ltx@one=1 %
%    \end{macrocode}
%    \end{macro}
%    \begin{macro}{\ltx@two}
%    \begin{macrocode}
\chardef\ltx@two=2 %
%    \end{macrocode}
%    \end{macro}
%    \begin{macro}{\ltx@active}
%    \begin{macrocode}
\chardef\ltx@active=13 %
%    \end{macrocode}
%    \end{macro}
%    \begin{macro}{\ltx@cclv}
%    \begin{macrocode}
\chardef\ltx@cclv=255 %
%    \end{macrocode}
%    \end{macro}
%
%    \begin{macro}{\ltx@minusone}
%    \begin{macrocode}
\def\ltx@minusone{%
  -\ltx@one
}
%    \end{macrocode}
%    \end{macro}
%
% \subsection{Scratch registers}
%
%    \begin{macro}{\ltx@LocToksA}
%    \begin{macrocode}
\toksdef\ltx@LocToksA=0 %
%    \end{macrocode}
%    \end{macro}
%    \begin{macro}{\ltx@LocToksB}
%    \begin{macrocode}
\toksdef\ltx@LocToksB=2 %
%    \end{macrocode}
%    \end{macro}
%    \begin{macro}{\ltx@LocToksC}
%    \begin{macrocode}
\toksdef\ltx@LocToksC=4 %
%    \end{macrocode}
%    \end{macro}
%    \begin{macro}{\ltx@LocToksD}
%    \begin{macrocode}
\toksdef\ltx@LocToksD=6 %
%    \end{macrocode}
%    \end{macro}
%    \begin{macro}{\ltx@LocToksE}
%    \begin{macrocode}
\toksdef\ltx@LocToksE=8 %
%    \end{macrocode}
%    \end{macro}
%
%    \begin{macro}{\ltx@GlobToksA}
%    \begin{macrocode}
\toksdef\ltx@GlobToksA=1 %
%    \end{macrocode}
%    \end{macro}
%    \begin{macro}{\ltx@GlobToksB}
%    \begin{macrocode}
\toksdef\ltx@GlobToksB=3 %
%    \end{macrocode}
%    \end{macro}
%    \begin{macro}{\ltx@GlobToksC}
%    \begin{macrocode}
\toksdef\ltx@GlobToksC=5 %
%    \end{macrocode}
%    \end{macro}
%    \begin{macro}{\ltx@GlobToksD}
%    \begin{macrocode}
\toksdef\ltx@GlobToksD=7 %
%    \end{macrocode}
%    \end{macro}
%    \begin{macro}{\ltx@GlobToksE}
%    \begin{macrocode}
\toksdef\ltx@GlobToksE=9 %
%    \end{macrocode}
%    \end{macro}
%
%    \begin{macro}{\ltx@LocDimenA}
%    \begin{macrocode}
\dimendef\ltx@LocDimenA=0 %
%    \end{macrocode}
%    \end{macro}
%    \begin{macro}{\ltx@LocDimenB}
%    \begin{macrocode}
\dimendef\ltx@LocDimenB=2 %
%    \end{macrocode}
%    \end{macro}
%    \begin{macro}{\ltx@LocDimenC}
%    \begin{macrocode}
\dimendef\ltx@LocDimenC=4 %
%    \end{macrocode}
%    \end{macro}
%    \begin{macro}{\ltx@LocDimenD}
%    \begin{macrocode}
\dimendef\ltx@LocDimenD=6 %
%    \end{macrocode}
%    \end{macro}
%    \begin{macro}{\ltx@LocDimenE}
%    \begin{macrocode}
\dimendef\ltx@LocDimenE=8 %
%    \end{macrocode}
%    \end{macro}
%
%    \begin{macro}{\ltx@GlobDimenA}
%    \begin{macrocode}
\dimendef\ltx@GlobDimenA=1 %
%    \end{macrocode}
%    \end{macro}
%    \begin{macro}{\ltx@GlobDimenB}
%    \begin{macrocode}
\dimendef\ltx@GlobDimenB=3 %
%    \end{macrocode}
%    \end{macro}
%    \begin{macro}{\ltx@GlobDimenC}
%    \begin{macrocode}
\dimendef\ltx@GlobDimenC=5 %
%    \end{macrocode}
%    \end{macro}
%    \begin{macro}{\ltx@GlobDimenD}
%    \begin{macrocode}
\dimendef\ltx@GlobDimenD=7 %
%    \end{macrocode}
%    \end{macro}
%    \begin{macro}{\ltx@GlobDimenE}
%    \begin{macrocode}
\dimendef\ltx@GlobDimenE=9 %
%    \end{macrocode}
%    \end{macro}
%
%    \begin{macro}{\ltx@LocSkipA}
%    \begin{macrocode}
\skipdef\ltx@LocSkipA=0 %
%    \end{macrocode}
%    \end{macro}
%    \begin{macro}{\ltx@LocSkipB}
%    \begin{macrocode}
\skipdef\ltx@LocSkipB=2 %
%    \end{macrocode}
%    \end{macro}
%    \begin{macro}{\ltx@LocSkipC}
%    \begin{macrocode}
\skipdef\ltx@LocSkipC=4 %
%    \end{macrocode}
%    \end{macro}
%    \begin{macro}{\ltx@LocSkipD}
%    \begin{macrocode}
\skipdef\ltx@LocSkipD=6 %
%    \end{macrocode}
%    \end{macro}
%    \begin{macro}{\ltx@LocSkipE}
%    \begin{macrocode}
\skipdef\ltx@LocSkipE=8 %
%    \end{macrocode}
%    \end{macro}
%
%    \begin{macro}{\ltx@GlobSkipA}
%    \begin{macrocode}
\skipdef\ltx@GlobSkipA=1 %
%    \end{macrocode}
%    \end{macro}
%    \begin{macro}{\ltx@GlobSkipB}
%    \begin{macrocode}
\skipdef\ltx@GlobSkipB=3 %
%    \end{macrocode}
%    \end{macro}
%    \begin{macro}{\ltx@GlobSkipC}
%    \begin{macrocode}
\skipdef\ltx@GlobSkipC=5 %
%    \end{macrocode}
%    \end{macro}
%    \begin{macro}{\ltx@GlobSkipD}
%    \begin{macrocode}
\skipdef\ltx@GlobSkipD=7 %
%    \end{macrocode}
%    \end{macro}
%    \begin{macro}{\ltx@GlobSkipE}
%    \begin{macrocode}
\skipdef\ltx@GlobSkipE=9 %
%    \end{macrocode}
%    \end{macro}
%
%
% \subsection{Argument killers}
%
%    \begin{macro}{\ltx@gobble}
%    \begin{macrocode}
\long\def\ltx@gobble#1{}
%    \end{macrocode}
%    \end{macro}
%    \begin{macro}{\ltx@gobbletwo}
%    \begin{macrocode}
\long\def\ltx@gobbletwo#1#2{}
%    \end{macrocode}
%    \end{macro}
%    \begin{macro}{\ltx@gobblethree}
%    \begin{macrocode}
\long\def\ltx@gobblethree#1#2#3{}
%    \end{macrocode}
%    \end{macro}
%    \begin{macro}{\ltx@gobblefour}
%    \begin{macrocode}
\long\def\ltx@gobblefour#1#2#3#4{}
%    \end{macrocode}
%    \end{macro}
%
%    \begin{macro}{\ltx@GobbleNum}
%    \begin{macrocode}
\def\ltx@GobbleNum#1{%
  \romannumeral
  \csname ltx@zero%
  \expandafter\LTXcmds@GobbleNum
  \romannumeral\LTXcmds@num{#1}000{m\endcsname}%
}
%    \end{macrocode}
%    \end{macro}
%    \begin{macro}{\LTXcmds@GobbleNum}
%    \begin{macrocode}
\def\LTXcmds@GobbleNum#1{%
  \csname LTXcmds@G#1\LTXcmds@GobbleNum
}
%    \end{macrocode}
%    \end{macro}
%    \begin{macro}{\LTXcmds@Gm}
%    \begin{macrocode}
\long\def\LTXcmds@Gm#1{%
  \endcsname
}
%    \end{macrocode}
%    \end{macro}
%
% \subsection{Argument grabbers}
%
%    \begin{macro}{\ltx@firstofone}
%    \begin{macrocode}
\long\def\ltx@firstofone#1{#1}
%    \end{macrocode}
%    \end{macro}
%    \begin{macro}{\ltx@firstoftwo}
%    \begin{macrocode}
\long\def\ltx@firstoftwo#1#2{#1}
%    \end{macrocode}
%    \end{macro}
%    \begin{macro}{\ltx@secondoftwo}
%    \begin{macrocode}
\long\def\ltx@secondoftwo#1#2{#2}
%    \end{macrocode}
%    \end{macro}
%
%    \begin{macro}{\ltx@firstofthree}
%    \begin{macrocode}
\long\def\ltx@firstofthree#1#2#3{#1}
%    \end{macrocode}
%    \end{macro}
%    \begin{macro}{\ltx@secondofthree}
%    \begin{macrocode}
\long\def\ltx@secondofthree#1#2#3{#2}
%    \end{macrocode}
%    \end{macro}
%    \begin{macro}{\ltx@thirdofthree}
%    \begin{macrocode}
\long\def\ltx@thirdofthree#1#2#3{#3}%
%    \end{macrocode}
%    \end{macro}
%
%    \begin{macro}{\ltx@firstoffour}
%    \begin{macrocode}
\long\def\ltx@firstoffour#1#2#3#4{#1}
%    \end{macrocode}
%    \end{macro}
%    \begin{macro}{\ltx@secondoffour}
%    \begin{macrocode}
\long\def\ltx@secondoffour#1#2#3#4{#2}
%    \end{macrocode}
%    \end{macro}
%    \begin{macro}{\ltx@thirdoffour}
%    \begin{macrocode}
\long\def\ltx@thirdoffour#1#2#3#4{#3}%
%    \end{macrocode}
%    \end{macro}
%    \begin{macro}{\ltx@fourthoffour}
%    \begin{macrocode}
\long\def\ltx@fourthoffour#1#2#3#4{#4}%
%    \end{macrocode}
%    \end{macro}
%
% \subsection{List helpers}
%
%    \begin{macro}{\ltx@carzero}
%    \begin{macrocode}
\long\def\ltx@carzero#1\@nil{}%
%    \end{macrocode}
%    \end{macro}
%    \begin{macro}{\LTXcmds@cdrzero}
%    \begin{macrocode}
\long\def\LTXcmds@cdrzero#1\@nil{#1}
%    \end{macrocode}
%    \end{macro}
%    \begin{macro}{\ltx@cdrzero}
%    \begin{macrocode}
\def\ltx@cdrzero{%
  \romannumeral\LTXcmds@cdrzero\ltx@zero
}
%    \end{macrocode}
%    \end{macro}
%
%    \begin{macro}{\ltx@car}
%    \begin{macrocode}
\long\def\ltx@car#1#2\@nil{#1}
%    \end{macrocode}
%    \end{macro}
%    \begin{macro}{\ltx@cdr}
%    \begin{macrocode}
\long\def\ltx@cdr#1{%
  \romannumeral\LTXcmds@cdrzero\ltx@zero
}
%    \end{macrocode}
%    \end{macro}
%
%    \begin{macro}{\ltx@cartwo}
%    \begin{macrocode}
\long\def\ltx@cartwo#1#2#3\@nil{#1#2}
%    \end{macrocode}
%    \end{macro}
%    \begin{macro}{\ltx@carsecond}
%    \begin{macrocode}
\long\def\ltx@carsecond#1#2#3\@nil{#2}
%    \end{macrocode}
%    \end{macro}
%    \begin{macro}{\ltx@cdrtwo}
%    \begin{macrocode}
\long\def\ltx@cdrtwo#1#2{%
  \romannumeral\LTXcmds@cdrzero\ltx@zero
}
%    \end{macrocode}
%    \end{macro}
%
%    \begin{macro}{\ltx@carthree}
%    \begin{macrocode}
\long\def\ltx@carthree#1#2#3#4\@nil{#1#2#3}
%    \end{macrocode}
%    \end{macro}
%    \begin{macro}{\ltx@carthird}
%    \begin{macrocode}
\long\def\ltx@carthird#1#2#3#4\@nil{#3}
%    \end{macrocode}
%    \end{macro}
%    \begin{macro}{\ltx@cdrthree}
%    \begin{macrocode}
\long\def\ltx@cdrthree#1#2#3{%
  \romannumeral\LTXcmds@cdrzero\ltx@zero
}
%    \end{macrocode}
%    \end{macro}
%
%    \begin{macro}{\ltx@carfour}
%    \begin{macrocode}
\long\def\ltx@carfour#1#2#3#4#5\@nil{#1#2#3#4}
%    \end{macrocode}
%    \end{macro}
%    \begin{macro}{\ltx@carfourth}
%    \begin{macrocode}
\long\def\ltx@carfourth#1#2#3#4#5\@nil{#4}
%    \end{macrocode}
%    \end{macro}
%    \begin{macro}{\ltx@cdrfour}
%    \begin{macrocode}
\long\def\ltx@cdrfour#1#2#3#4{%
  \romannumeral\LTXcmds@cdrzero\ltx@zero
}
%    \end{macrocode}
%    \end{macro}
%
%    \begin{macro}{\ltx@CarNum}
%    \begin{macrocode}
\def\ltx@CarNum#1{%
  \romannumeral
  \csname LTXcmds@CarNumFinish%
  \expandafter\LTXcmds@CarNum
  \romannumeral\LTXcmds@num{#1}000{x\endcsname}%
}
%    \end{macrocode}
%    \end{macro}
%    \begin{macro}{\LTXcmds@CarNum}
%    \begin{macrocode}
\def\LTXcmds@CarNum#1{%
  \csname LTXcmds@C#1\LTXcmds@CarNum
}
%    \end{macrocode}
%    \end{macro}
%    \begin{macro}{\LTXcmds@Cm}
%    \begin{macrocode}
\long\def\LTXcmds@Cm#1#2{%
  \endcsname{#1#2}%
}
%    \end{macrocode}
%    \end{macro}
%    \begin{macro}{\LTXcmds@Cx}
%    \begin{macrocode}
\def\LTXcmds@Cx#1{%
  \endcsname{}%
}
%    \end{macrocode}
%    \end{macro}
%    \begin{macro}{\LTXcmds@CarNumFinish}
%    \begin{macrocode}
\long\def\LTXcmds@CarNumFinish#1#2\@nil{%
  \ltx@zero
  #1%
}
%    \end{macrocode}
%    \end{macro}
%
%    \begin{macro}{\ltx@CarNumth}
%    \begin{macrocode}
\def\ltx@CarNumth#1{%
  \romannumeral
  \expandafter\expandafter\expandafter
  \LTXcmds@CarNumth
  \ltx@GobbleNum{#1}{}%
}
%    \end{macrocode}
%    \end{macro}
%    \begin{macro}{\LTXcmds@CarNumth}
%    \begin{macrocode}
\long\def\LTXcmds@CarNumth#1#2\@nil{%
  \ltx@zero
  #1%
}
%    \end{macrocode}
%    \end{macro}
%
%    \begin{macro}{\ltx@CdrNum}
%    \begin{macrocode}
\def\ltx@CdrNum#1{%
  \romannumeral%
  \expandafter\expandafter\expandafter\ltx@cdrzero
  \expandafter\expandafter\expandafter\ltx@zero
  \ltx@GobbleNum{#1}%
}
%    \end{macrocode}
%    \end{macro}
%
% \subsection{Tail recursion}
%
%    \begin{macro}{\ltx@ReturnAfterFi}
%    \begin{macrocode}
\long\def\ltx@ReturnAfterFi#1\fi{\fi#1}
%    \end{macrocode}
%    \end{macro}
%    \begin{macro}{\ltx@ReturnAfterElseFi}
%    \begin{macrocode}
\long\def\ltx@ReturnAfterElseFi#1\else#2\fi{\fi#1}
%    \end{macrocode}
%    \end{macro}
%
% \subsection{Empty macro}
%
%    \begin{macro}{\ltx@empty}
%    \begin{macrocode}
\def\ltx@empty{}
%    \end{macrocode}
%    \end{macro}
%
% \subsection{Characters}
%
%    \begin{macro}{\ltx@space}
%    \begin{macrocode}
\def\ltx@space{ }
%    \end{macrocode}
%    \end{macro}
%    \begin{macro}{\ltx@percentchar}
%    \begin{macrocode}
\begingroup
  \lccode`0=`\%\relax
\lowercase{\endgroup
  \def\ltx@percentchar{0}%
}
%    \end{macrocode}
%    \end{macro}
%    \begin{macro}{\ltx@backslashchar}
%    \begin{macrocode}
\begingroup
  \lccode`0=`\\\relax
\lowercase{\endgroup
  \def\ltx@backslashchar{0}%
}
%    \end{macrocode}
%    \end{macro}
%    \begin{macro}{\ltx@hashchar}
%    \begin{macrocode}
\begingroup
  \lccode`0=`\#\relax
\lowercase{\endgroup
  \def\ltx@hashchar{0}%
}
%    \end{macrocode}
%    \end{macro}
%    \begin{macro}{\ltx@leftbracechar}
%    \begin{macrocode}
\begingroup
  \lccode`0=`\{\relax
\lowercase{\endgroup
  \def\ltx@leftbracechar{0}%
}
%    \end{macrocode}
%    \end{macro}
%    \begin{macro}{\ltx@rightbracechar}
%    \begin{macrocode}
\begingroup
  \lccode`0=`\}\relax
\lowercase{\endgroup
  \def\ltx@rightbracechar{0}%
}
%    \end{macrocode}
%    \end{macro}
%
% \subsection{Boolean switch}
%
%    \begin{macro}{\ltx@newif}
%    \begin{macrocode}
\def\ltx@newif#1{%
  \begingroup
    \escapechar=-1 %
  \expandafter\endgroup
  \expandafter\LTXcmds@newif\string#1\@nil
}
%    \end{macrocode}
%    \end{macro}
%    \begin{macro}{\LTXcmds@newif}
%    \begin{macrocode}
\begingroup
  \escapechar=-1 %
\expandafter\endgroup
\expandafter\def\expandafter\LTXcmds@newif\string\if#1\@nil{%
  \expandafter\edef\csname#1true\endcsname{%
    \let
    \expandafter\noexpand\csname if#1\endcsname
    \noexpand\iftrue
  }%
  \expandafter\edef\csname#1false\endcsname{%
    \let
    \expandafter\noexpand\csname if#1\endcsname
    \noexpand\iffalse
  }%
  \csname#1false\endcsname
}
%    \end{macrocode}
%    \end{macro}
%
%    \begin{macro}{\ltx@newglobalif}
%    \begin{macrocode}
\def\ltx@newglobalif#1{%
  \begingroup
    \escapechar=-1 %
  \expandafter\endgroup
  \expandafter\LTXcmds@newglobalif\string#1\@nil
}
%    \end{macrocode}
%    \end{macro}
%    \begin{macro}{\LTXcmds@newglobalif}
%    \begin{macrocode}
\begingroup
  \escapechar=-1 %
\expandafter\endgroup
\expandafter
\def\expandafter\LTXcmds@newglobalif\string\if#1\@nil{%
  \expandafter\edef\csname#1true\endcsname{%
    \global\let
    \expandafter\noexpand\csname if#1\endcsname
    \noexpand\iftrue
  }%
  \expandafter\edef\csname#1false\endcsname{%
    \global\let
    \expandafter\noexpand\csname if#1\endcsname
    \noexpand\iffalse
  }%
  \csname#1false\endcsname
}
%    \end{macrocode}
%    \end{macro}
%
% \subsection{Command definitions}
%
%    \begin{macro}{\ltx@LocalExpandAfter}
%    \begin{macrocode}
\def\ltx@LocalExpandAfter{%
  \begingroup
    \expandafter\expandafter\expandafter
  \endgroup
  \expandafter
}
%    \end{macrocode}
%    \end{macro}
%
%    \begin{macrocode}
\ltx@LocalExpandAfter
\ifx\csname ifcsname\endcsname\relax
%    \end{macrocode}
%    \begin{macro}{\ltx@ifundefined}
%    \begin{macrocode}
  \def\ltx@ifundefined#1{%
    \expandafter\ifx\csname #1\endcsname\relax
      \expandafter\ltx@firstoftwo
    \else
      \expandafter\ltx@secondoftwo
    \fi
  }%
%    \end{macrocode}
%    \end{macro}
%    \begin{macro}{\ltx@IfUndefined}
%    \begin{macrocode}
  \def\ltx@IfUndefined#1{%
    \begingroup\expandafter\expandafter\expandafter\endgroup
    \expandafter\ifx\csname #1\endcsname\relax
      \expandafter\ltx@firstoftwo
    \else
      \expandafter\ltx@secondoftwo
    \fi
  }%
%    \end{macrocode}
%    \end{macro}
%    \begin{macrocode}
  \expandafter\ltx@gobble
\else
  \expandafter\ltx@firstofone
\fi
{%
%    \end{macrocode}
%    \begin{macro}{\ltx@ifundefined}
%    \begin{macrocode}
  \def\ltx@ifundefined#1{%
    \ifcsname #1\endcsname
      \expandafter\ifx\csname #1\endcsname\relax
        \expandafter\expandafter\expandafter\ltx@firstoftwo
      \else
        \expandafter\expandafter\expandafter\ltx@secondoftwo
      \fi
    \else
      \expandafter\ltx@firstoftwo
    \fi
  }%
%    \end{macrocode}
%    \end{macro}
%    \begin{macro}{\ltx@IfUndefined}
%    \begin{macrocode}
  \let\ltx@IfUndefined\ltx@ifundefined
%    \end{macrocode}
%    \end{macro}
%    \begin{macrocode}
}
%    \end{macrocode}
%
% \subsection{Stripping}
%
%    \begin{macro}{\ltx@RemovePrefix}
%    \begin{macrocode}
\def\ltx@RemovePrefix#1>{}
%    \end{macrocode}
%    \end{macro}
%
%    \begin{macro}{\ltx@StripPrefix}
%    \begin{macrocode}
\def\ltx@StripPrefix{%
  \expandafter\ltx@RemovePrefix
}
%    \end{macrocode}
%    \end{macro}
%
%    \begin{macro}{\ltx@onelevel@sanitize}
%    \begin{macrocode}
\def\ltx@onelevel@sanitize#1{%
  \edef#1{%
    \expandafter
    \ltx@RemovePrefix\meaning#1%
  }%
}
%    \end{macrocode}
%    \end{macro}
%
% \subsection{File management}
%
% \subsubsection{File extensions}
%
%    \begin{macro}{\ltx@clsextension}
%    \begin{macrocode}
\def\ltx@clsextension{cls}
%    \end{macrocode}
%    \end{macro}
%    \begin{macro}{\ltx@pkgextension}
%    \begin{macrocode}
\def\ltx@pkgextension{sty}
%    \end{macrocode}
%    \end{macro}
%
% \subsubsection{Load check}
%
%    \begin{macro}{\ltx@iffileloaded}
%    \begin{macrocode}
\def\ltx@iffileloaded#1{%
  \ltx@ifundefined{ver@#1}\ltx@secondoftwo\ltx@firstoftwo
}
%    \end{macrocode}
%    \end{macro}
%    \begin{macro}{\ltx@ifclassloaded}
%    \begin{macrocode}
\def\ltx@ifclassloaded#1{%
  \ltx@iffileloaded{#1.\ltx@clsextension}%
}
%    \end{macrocode}
%    \end{macro}
%    \begin{macro}{\ltx@ifpackageloaded}
%    \begin{macrocode}
\def\ltx@ifpackageloaded#1{%
  \ltx@iffileloaded{#1.\ltx@pkgextension}%
}
%    \end{macrocode}
%    \end{macro}
%
% \subsubsection{Version date check}
%
%    \begin{macro}{\ltx@iffilelater}
%    \begin{macrocode}
\def\ltx@iffilelater#1#2{%
  \ltx@iffileloaded{#1}{%
    \expandafter\LTXcmds@IfLater\expandafter{%
      \number
      \expandafter\expandafter\expandafter\LTXcmds@ParseVersion
      \expandafter\expandafter\expandafter{%
        \csname ver@#1\endcsname
      }%
    \expandafter}\expandafter{%
      \number
      \expandafter\LTXcmds@ParseVersion\expandafter{#2}%
    }%
  }\ltx@secondoftwo
}
%    \end{macrocode}
%    \end{macro}
%    \begin{macro}{\LTXcmds@IfLater}
%    \begin{macrocode}
\def\LTXcmds@IfLater#1#2{%
  \ifcase 0%
      \ifnum#1<19940101 %
      \else
        \ifnum#2<19940101 %
        \else
          \ifnum#2>#1 %
          \else
            1%
          \fi
        \fi
      \fi
      \ltx@space
    \expandafter\ltx@secondoftwo
  \else
    \expandafter\ltx@firstoftwo
  \fi
}
%    \end{macrocode}
%    \end{macro}
%    \begin{macro}{\ltx@ifclasslater}
%    \begin{macrocode}
\def\ltx@ifclasslater#1{%
  \ltx@iffilelater{#1.\ltx@clsextension}%
}
%    \end{macrocode}
%    \end{macro}
%    \begin{macro}{\ltx@ifpackagelater}
%    \begin{macrocode}
\def\ltx@ifpackagelater#1{%
  \ltx@iffilelater{#1.\ltx@pkgextension}%
}
%    \end{macrocode}
%    \end{macro}
%
%    \begin{macrocode}
\ltx@IfUndefined{pdfmatch}{%
%    \end{macrocode}
%    \begin{macro}{\LTXcmds@ParseVersion}
%    \begin{macrocode}
  \def\LTXcmds@ParseVersion#1{%
    \LTXcmds@@ParseVersion#10000/00/00\@nil
  }%
%    \end{macrocode}
%    \end{macro}
%    \begin{macro}{\LTXcmds@@ParseVersion}
%    \begin{macrocode}
  \def\LTXcmds@@ParseVersion#1#2#3#4/#5#6/#7#8#9\@nil{%
    #1#2#3#4#5#6#7#8%
  }%
%    \end{macrocode}
%    \end{macro}
%    \begin{macrocode}
}{%
%    \end{macrocode}
%    \begin{macro}{\LTXcmds@ParseVersion}
%    \begin{macrocode}
  \def\LTXcmds@ParseVersion#1{%
    \ifnum\pdfmatch{%
      ^%
      (199[4-9]|[2-9][0-9][0-9][0-9])/%
      (0[1-9]|1[0-2])/%
      (0[1-9]|[1-2][0-9]|3[0-1])%
    }{#1}=1 %
      \ltx@StripPrefix\pdflastmatch1 %
      \ltx@StripPrefix\pdflastmatch2 %
      \ltx@StripPrefix\pdflastmatch3 %
    \else
      0%
    \fi
  }%
%    \end{macrocode}
%    \end{macro}
%    \begin{macrocode}
}
%    \end{macrocode}
%
% \subsection{Macro additions}
%
%    \begin{macro}{\ltx@GlobalAppendToMacro}
%    \begin{macrocode}
\long\def\ltx@GlobalAppendToMacro#1#2{%
  \ifx\ltx@undefined#1%
    \let#1\ltx@empty
  \else
    \ifx\relax#1%
      \let#1\ltx@empty
    \fi
  \fi
  \begingroup
    \ltx@LocToksA\expandafter{#1#2}%
    \xdef#1{\the\ltx@LocToksA}%
  \endgroup
}
%    \end{macrocode}
%    \end{macro}
%    \begin{macro}{\ltx@LocalAppendToMacro}
%    \begin{macrocode}
\long\def\ltx@LocalAppendToMacro#1#2{%
  \global\let\LTXcmds@gtemp#1%
  \ifx\ltx@undefined\LTXcmds@gtemp
    \global\let\LTXcmds@gtemp\ltx@empty
  \else
    \ifx\relax\LTXcmds@gtemp
      \global\letLTXcmds@gtemp\ltx@empty
    \fi
  \fi
  \begingroup
    \ltx@LocToksA\expandafter{\LTXcmds@gtemp#2}%
    \xdef\LTXcmds@gtemp{\the\ltx@LocToksA}%
  \endgroup
  \let#1\LTXcmds@gtemp
}
%    \end{macrocode}
%    \end{macro}
%
%    \begin{macro}{\ltx@GlobalPrependToMacro}
%    \begin{macrocode}
\long\def\ltx@GlobalPrependToMacro#1#2{%
  \ifx\ltx@undefined#1%
    \let#1\ltx@empty
  \else
    \ifx\relax#1%
      \let#1\ltx@empty
    \fi
  \fi
  \begingroup
    \ltx@LocToksA{#2}%
    \ltx@LocToksB\expandafter{#1}%
    \xdef#1{\the\ltx@LocToksA\the\ltx@LocToksB}%
  \endgroup
}
%    \end{macrocode}
%    \end{macro}
%    \begin{macro}{\ltx@LocalPrependToMacro}
%    \begin{macrocode}
\long\def\ltx@LocalPrependToMacro#1#2{%
  \global\let\LTXcmds@gtemp#1%
  \ifx\ltx@undefined\LTXcmds@gtemp
    \global\let\LTXcmds@gtemp\ltx@empty
  \else
    \ifx\relax\LTXcmds@gtemp
      \global\letLTXcmds@gtemp\ltx@empty
    \fi
  \fi
  \begingroup
    \ltx@LocToksA{#2}%
    \ltx@LocToksB\expandafter{\LTXcmds@gtemp}%
    \xdef\LTXcmds@gtemp{\the\ltx@LocToksA\the\ltx@LocToksB}%
  \endgroup
  \let#1\LTXcmds@gtemp
}
%    \end{macrocode}
%    \end{macro}
%
% \subsection{Next character detection}
%
%    \begin{macro}{\ltx@ifnextchar}
%    \begin{macrocode}
\long\def\ltx@ifnextchar#1#2#3{%
  \begingroup
  \let\LTXcmds@CharToken= #1\relax
  \ltx@LocToksA{\endgroup#2}%
  \ltx@LocToksB{\endgroup#3}%
  \futurelet\LTXcmds@LetToken\LTXcmds@ifnextchar
}
%    \end{macrocode}
%    \end{macro}
%    \begin{macro}{\LTXcmds@ifnextchar}
%    \begin{macrocode}
\def\LTXcmds@ifnextchar{%
  \ifx\LTXcmds@LetToken\LTXcmds@CharToken
    \the\expandafter\ltx@LocToksA
  \else
    \expandafter
      \ifx\csname LTXcmds@LetToken\endcsname\LTXcmds@SpaceToken
      \expandafter\expandafter\expandafter\LTXcmds@@ifnextchar
    \else
      \the\expandafter\expandafter\expandafter\ltx@LocToksB
    \fi
  \fi
}
%    \end{macrocode}
%    \end{macro}
%    \begin{macro}{\LTXcmds@@ifnextchar}
%    \cs{futurelet} does not distinguish between a
%    character and a command that is a character
%    (defined by using \cs{let} or \cs{futurelet}).
%    Therefore the space is catched by \cs{romannumeral}
%    with negative character constant that gobbles one
%    optional space.
%    \begin{macrocode}
\def\LTXcmds@@ifnextchar{%
  \expandafter\futurelet
  \expandafter\LTXcmds@LetToken
  \expandafter\LTXcmds@ifnextchar
  \romannumeral-`\.%
}
%    \end{macrocode}
%    \end{macro}
%    \begin{macro}{\LTXcmds@SpaceToken}
%    \begin{macrocode}
\ltx@firstofone{\let\LTXcmds@SpaceToken= } %
%    \end{macrocode}
%    \end{macro}
%
%    \begin{macro}{\ltx@ifnextchar@nospace}
%    \begin{macrocode}
\long\def\ltx@ifnextchar@nospace#1#2#3{%
  \begingroup
  \let\LTXcmds@CharToken= #1\relax
  \ltx@LocToksA{\endgroup#2}%
  \ltx@LocToksB{\endgroup#3}%
  \futurelet\LTXcmds@LetToken\LTXcmds@ifnextchar@nospace
}
%    \end{macrocode}
%    \end{macro}
%    \begin{macro}{\LTXcmds@ifnextchar@nospace}
%    \begin{macrocode}
\def\LTXcmds@ifnextchar@nospace{%
  \the
  \ifx\LTXcmds@LetToken\LTXcmds@CharToken
    \expandafter\ltx@LocToksA
  \else
    \expandafter\ltx@LocToksB
  \fi
}
%    \end{macrocode}
%    \end{macro}
% \subsection{\cs{ltx@leavevmode}, \cs{ltx@mbox}}
%
%    \begin{macro}{\ltx@leavevmode}
%    \begin{macrocode}
\ltx@IfUndefined{quitvmode}{%
  \ltx@IfUndefined{leavevmode}{%
    \ltx@IfUndefined{voidb@x}{%
      \ltx@IfUndefined{newbox}{%
        \def\ltx@leavevmode{%
          \begingroup
            \setbox\ltx@zero=\hbox{}%
            \begingroup
              \setbox\ltx@zero=\hbox{\box\ltx@zero}%
            \endgroup
            \unhbox\ltx@zero
          \endgroup
        }%
      }{%
        \csname newbox\endcsname\LTXcmds@VoidBox
        \ifvoid\LTXcmds@VoidBox
        \else
          \setbox\LTXcmds@VoidBox=\hbox{}%
          \begingroup
            \setbox\LTXcmds@VoidBox=\hbox{\box\LTXcmds@VoidBox}%
          \endgroup
        \fi
        \def\ltx@leavevmode{\unhbox\LTXcmds@VoidBox}%
      }%
    }{%
      \def\ltx@leavevmode{\unhbox\voidb@x}%
    }%
  }{%
    \let\ltx@leavevmode\leavevmode
  }%
}{%
  \let\ltx@leavevmode\quitvmode
}
%    \end{macrocode}
%    \end{macro}
%
%    \begin{macro}{\ltx@mbox}
%    \begin{macrocode}
\def\ltx@mbox{%
  \ltx@leavevmode
  \hbox
}
%    \end{macrocode}
%    \end{macro}
%
% \subsection{Help macros}
%
%    \begin{macro}{\LTXcmds@num}
%    \begin{macrocode}
\ltx@IfUndefined{numexpr}{%
  \def\LTXcmds@num#1{%
    \expandafter\ltx@firstofone\expandafter{%
      \number#1%
    }%
  }%
}{%
  \def\LTXcmds@num#1{%
    \expandafter\ltx@firstofone\expandafter{%
      \the\numexpr#1%
    }%
  }%
}
%    \end{macrocode}
%    \end{macro}
%
% \subsection{Expandable test for emptiness}
%
%    \begin{macrocode}
\ltx@IfUndefined{detokenize}{%
%    \end{macrocode}
%
% \subsubsection{Vanilla \hologo{TeX}}
%
%    \begin{macro}{\ltx@ifempty}
%    The macro is based on \cs{@ifempty} of Robert R. Schneck
%    \cite{ifempty:schneck}
%    and \cs{@ifnull} of Ulrich Diez \cite{ifnull:diez}.
%    There are three cases to consider:
%    \begin{compactenum}
%    \item |#1| is empty,
%    \item |#1| is not empty and the first token
%      is not a begingroup character,
%    \item |#1| starts with a begingroup character (catcode 1).
%    \end{compactenum}
%    \begin{macrocode}
  \def\LTXcmds@temp#1{%
    \long\def\ltx@ifempty##1{%
      \romannumeral0%
      \iffalse{\fi
        \expandafter\ltx@gobble\expandafter{%
          \expandafter{\string##1}%
          \expandafter\ltx@gobble\string
        }%
        \expandafter\ltx@firstofthree\expandafter
        {\iffalse}\fi
        \expandafter#1\ltx@secondoftwo
      }%
      \expandafter#1\ltx@firstoftwo
    }%
%    \end{macrocode}
%    \end{macro}
%    \begin{macro}{\ltx@ifblank}
%    \begin{macrocode}
    \long\def\ltx@ifblank##1{%
      \romannumeral0%
      \iffalse{\fi
        \expandafter\expandafter\expandafter\ltx@gobble
        \expandafter\expandafter\expandafter{%
          \expandafter\expandafter\expandafter{%
            \expandafter\string\ltx@gobble##1.%
          }%
          \expandafter\ltx@gobble\string
        }%
        \expandafter\ltx@firstofthree\expandafter
        {\iffalse}\fi
        \expandafter#1\ltx@secondoftwo
      }%
      \expandafter#1\ltx@firstoftwo
    }%
  }%
  \LTXcmds@temp{ }%
%    \end{macrocode}
%    \end{macro}
%    \begin{macrocode}
}{%
%    \end{macrocode}
%
% \subsubsection{With \cs{detokenize}}
%
%    Ahmed Musa provided \cs{ifstrempty} using \cs{detokenize}
%    and \cs{pdfstrcmp} \cite{ifstrempty:musa}. Ulrich Diez, GL,
%    Heiko Oberdiek
%    improved it further by removing \cs{pdfstrcmp} and
%    taking three arguments
%    \cite{ifstrempty:diez1,ifstrempty:diez2,ifstrempty:diez3,^^A
%      ifstrempty:gl,ifstrempty:ho}.
%
%    \begin{macro}{\ltx@ifempty}
%    \begin{macrocode}
  \long\def\ltx@ifempty#1{%
    \romannumeral%
    \csname
      LTXcmds@ifempty%
      \ifcat$\detokenize{#1}$%
        @%
      \fi
    \endcsname
  }%
%    \end{macrocode}
%    \end{macro}
%    \begin{macro}{\LTXcmds@ifempty@}
%    \begin{macrocode}
  \long\def\LTXcmds@ifempty@#1#2{0 #1}%
%    \end{macrocode}
%    \end{macro}
%    \begin{macro}{\LTXcmds@ifempty}
%    \begin{macrocode}
  \long\def\LTXcmds@ifempty#1#2{0 #2}%
%    \end{macrocode}
%    \end{macro}
%
% \subsubsection{\cs{ltx@ifblank}}
%
%    \begin{macro}{\ltx@ifblank}
%    \begin{macrocode}
  \long\def\ltx@ifblank#1{%
    \romannumeral%
    \csname
      LTXcmds@ifempty%
      \ifcat$\detokenize\expandafter{\ltx@gobble#1.}$%
        @%
      \fi
    \endcsname
  }%
%    \end{macrocode}
%    \end{macro}
%
%    \begin{macrocode}
}
%    \end{macrocode}
%
% \subsection{\cs{ltx@zapspace}}
%
%    \begin{macro}{\ltx@zapspace}
%    \begin{macrocode}
\long\def\ltx@zapspace#1{%
  \romannumeral
  \LTXcmds@zapspace\ltx@zero#1 \@nil
}
%    \end{macrocode}
%    \end{macro}
%    \begin{macro}{\LTXcmds@zapspace}
%    \begin{macrocode}
\long\def\LTXcmds@zapspace#1 #2\@nil{%
  \ltx@ifempty{#2}{%
    #1%
  }{%
    \LTXcmds@zapspace#1#2\@nil
  }%
}
%    \end{macrocode}
%    \end{macro}
%
% \subsection{\cs{ltx@IfBoxEmpty}}
%
%    In case of \hologo{eTeX} the test for an empty box
%    is done via \cs{lastnodetype} as suggested
%    by David Kastrup \cite{ifboxempty:dak}.
%    \begin{macrocode}
\ltx@IfUndefined{lastnodetype}{%
  \catcode`\$=9 %
  \catcode`\&=14 %
}{%
  \catcode`\$=14 %
  \catcode`\&=9 %
}
%    \end{macrocode}
%    \begin{macro}{\ltx@IfBoxEmpty}
%    \begin{macrocode}
\def\ltx@IfBoxEmpty#1{%
  \ifvoid#1\relax
    \expandafter\ltx@secondoftwo
  \else
%    \end{macrocode}
%    Implementation using \hologo{eTeX}'s \cs{lastnodetype}.
%    \begin{macrocode}
&   \begingroup
&     \setbox\ltx@zero=\ifhbox#1\hbox\else\vbox\fi{%
&       \ifhmode\unhcopy\else\unvcopy\fi#1\relax
&       \expandafter
&     }%
&   \expandafter\endgroup
&   \ifnum\lastnodetype<\ltx@zero
&     \expandafter\expandafter\expandafter\ltx@firstoftwo
&   \else
&     \expandafter\expandafter\expandafter\ltx@secondoftwo
&   \fi
%    \end{macrocode}
%    Implementation without \hologo{eTeX} using a signature
%    at the beginning of the test box.
%    \begin{macrocode}
$   \begingroup
$     \setbox\ltx@zero=\ifhbox#1\hbox\else\vbox\fi{%
$       \penalty\ltx@one
$       \ifhmode\unhcopy\else\unvcopy\fi#1\relax
$       \expandafter
$     }%
$     \ifnum\lastpenalty=\ltx@one
%    \end{macrocode}
%    Box 0 has been changed and is restored by
%    closing the group.
%    \begin{macrocode}
$       \endgroup
$       \begingroup
$       \setbox\ltx@zero=\ifhbox#1\hbox\else\vbox\fi{%
$         \penalty\ltx@two
$         \ifhmode\unhcopy\else\unvcopy\fi#1\relax
$         \expandafter
$       }%
$       \ifnum\lastpenalty=\ltx@two
$         \def\next{\endgroup\expandafter\ltx@firstoftwo}%
$       \else
$         \def\next{\endgroup\expandafter\ltx@secondoftwo}%
$       \fi
$     \else
$       \def\next{\endgroup\expandafter\ltx@secondoftwo}%
$     \fi
$   \next
  \fi
}
%    \end{macrocode}
%    \end{macro}
%
%    \begin{macro}{\ltx@IfBoxVoidOrEmpty}
%    \begin{macrocode}
\def\ltx@IfBoxVoidOrEmpty#1{%
  \ifvoid#1\relax
    \expandafter\ltx@thirdoffour
  \fi
  \ltx@IfBoxEmpty{#1}%
}
%    \end{macrocode}
%    \end{macro}
%
%    \begin{macrocode}
\LTXcmds@AtEnd%
%</package>
%    \end{macrocode}
%% \section{Installation}
%
% \subsection{Download}
%
% \paragraph{Package.} This package is available on
% CTAN\footnote{\CTANpkg{ltxcmds}}:
% \begin{description}
% \item[\CTAN{macros/latex/contrib/ltxcmds/ltxcmds.dtx}] The source file.
% \item[\CTAN{macros/latex/contrib/ltxcmds/ltxcmds.pdf}] Documentation.
% \end{description}
%
%
% \paragraph{Bundle.} All the packages of the bundle `ltxcmds'
% are also available in a TDS compliant ZIP archive. There
% the packages are already unpacked and the documentation files
% are generated. The files and directories obey the TDS standard.
% \begin{description}
% \item[\CTANinstall{install/macros/latex/contrib/ltxcmds.tds.zip}]
% \end{description}
% \emph{TDS} refers to the standard ``A Directory Structure
% for \TeX\ Files'' (\CTANpkg{tds}). Directories
% with \xfile{texmf} in their name are usually organized this way.
%
% \subsection{Bundle installation}
%
% \paragraph{Unpacking.} Unpack the \xfile{ltxcmds.tds.zip} in the
% TDS tree (also known as \xfile{texmf} tree) of your choice.
% Example (linux):
% \begin{quote}
%   |unzip ltxcmds.tds.zip -d ~/texmf|
% \end{quote}
%
% \subsection{Package installation}
%
% \paragraph{Unpacking.} The \xfile{.dtx} file is a self-extracting
% \docstrip\ archive. The files are extracted by running the
% \xfile{.dtx} through \plainTeX:
% \begin{quote}
%   \verb|tex ltxcmds.dtx|
% \end{quote}
%
% \paragraph{TDS.} Now the different files must be moved into
% the different directories in your installation TDS tree
% (also known as \xfile{texmf} tree):
% \begin{quote}
% \def\t{^^A
% \begin{tabular}{@{}>{\ttfamily}l@{ $\rightarrow$ }>{\ttfamily}l@{}}
%   ltxcmds.sty & tex/generic/ltxcmds/ltxcmds.sty\\
%   ltxcmds.pdf & doc/latex/ltxcmds/ltxcmds.pdf\\
%   ltxcmds.dtx & source/latex/ltxcmds/ltxcmds.dtx\\
% \end{tabular}^^A
% }^^A
% \sbox0{\t}^^A
% \ifdim\wd0>\linewidth
%   \begingroup
%     \advance\linewidth by\leftmargin
%     \advance\linewidth by\rightmargin
%   \edef\x{\endgroup
%     \def\noexpand\lw{\the\linewidth}^^A
%   }\x
%   \def\lwbox{^^A
%     \leavevmode
%     \hbox to \linewidth{^^A
%       \kern-\leftmargin\relax
%       \hss
%       \usebox0
%       \hss
%       \kern-\rightmargin\relax
%     }^^A
%   }^^A
%   \ifdim\wd0>\lw
%     \sbox0{\small\t}^^A
%     \ifdim\wd0>\linewidth
%       \ifdim\wd0>\lw
%         \sbox0{\footnotesize\t}^^A
%         \ifdim\wd0>\linewidth
%           \ifdim\wd0>\lw
%             \sbox0{\scriptsize\t}^^A
%             \ifdim\wd0>\linewidth
%               \ifdim\wd0>\lw
%                 \sbox0{\tiny\t}^^A
%                 \ifdim\wd0>\linewidth
%                   \lwbox
%                 \else
%                   \usebox0
%                 \fi
%               \else
%                 \lwbox
%               \fi
%             \else
%               \usebox0
%             \fi
%           \else
%             \lwbox
%           \fi
%         \else
%           \usebox0
%         \fi
%       \else
%         \lwbox
%       \fi
%     \else
%       \usebox0
%     \fi
%   \else
%     \lwbox
%   \fi
% \else
%   \usebox0
% \fi
% \end{quote}
% If you have a \xfile{docstrip.cfg} that configures and enables \docstrip's
% TDS installing feature, then some files can already be in the right
% place, see the documentation of \docstrip.
%
% \subsection{Refresh file name databases}
%
% If your \TeX~distribution
% (\TeX\,Live, \mikTeX, \dots) relies on file name databases, you must refresh
% these. For example, \TeX\,Live\ users run \verb|texhash| or
% \verb|mktexlsr|.
%
% \subsection{Some details for the interested}
%
% \paragraph{Unpacking with \LaTeX.}
% The \xfile{.dtx} chooses its action depending on the format:
% \begin{description}
% \item[\plainTeX:] Run \docstrip\ and extract the files.
% \item[\LaTeX:] Generate the documentation.
% \end{description}
% If you insist on using \LaTeX\ for \docstrip\ (really,
% \docstrip\ does not need \LaTeX), then inform the autodetect routine
% about your intention:
% \begin{quote}
%   \verb|latex \let\install=y% \iffalse meta-comment
%
% File: ltxcmds.dtx
% Version: 2019/12/15 v1.24
% Info: LaTeX kernel commands for general use
%
% Copyright (C)
%    2009-2011 Heiko Oberdiek
%    2016-2019 Oberdiek Package Support Group
%    https://github.com/ho-tex/ltxcmds/issues
%
% This work may be distributed and/or modified under the
% conditions of the LaTeX Project Public License, either
% version 1.3c of this license or (at your option) any later
% version. This version of this license is in
%    https://www.latex-project.org/lppl/lppl-1-3c.txt
% and the latest version of this license is in
%    https://www.latex-project.org/lppl.txt
% and version 1.3 or later is part of all distributions of
% LaTeX version 2005/12/01 or later.
%
% This work has the LPPL maintenance status "maintained".
%
% The Current Maintainers of this work are
% Heiko Oberdiek and the Oberdiek Package Support Group
% https://github.com/ho-tex/ltxcmds/issues
%
% The Base Interpreter refers to any `TeX-Format',
% because some files are installed in TDS:tex/generic//.
%
% This work consists of the main source file ltxcmds.dtx
% and the derived files
%    ltxcmds.sty, ltxcmds.pdf, ltxcmds.ins, ltxcmds.drv,
%    ltxcmds-test1.tex, ltxcmds-test-gobble.tex,
%    ltxcmds-test-ifempty.tex, ltxcmds-test-zapspace.tex,
%    ltxcmds-test-ifboxempty.tex, ltxcmds-test-nextchar.tex,
%    ltxcmds-test-carcdr.tex.
%
% Distribution:
%    CTAN:macros/latex/contrib/ltxcmds/ltxcmds.dtx
%    CTAN:macros/latex/contrib/ltxcmds/ltxcmds.pdf
%
% Unpacking:
%    (a) If ltxcmds.ins is present:
%           tex ltxcmds.ins
%    (b) Without ltxcmds.ins:
%           tex ltxcmds.dtx
%    (c) If you insist on using LaTeX
%           latex \let\install=y\input{ltxcmds.dtx}
%        (quote the arguments according to the demands of your shell)
%
% Documentation:
%    (a) If ltxcmds.drv is present:
%           latex ltxcmds.drv
%    (b) Without ltxcmds.drv:
%           latex ltxcmds.dtx; ...
%    The class ltxdoc loads the configuration file ltxdoc.cfg
%    if available. Here you can specify further options, e.g.
%    use A4 as paper format:
%       \PassOptionsToClass{a4paper}{article}
%
%    Programm calls to get the documentation (example):
%       pdflatex ltxcmds.dtx
%       makeindex -s gind.ist ltxcmds.idx
%       pdflatex ltxcmds.dtx
%       makeindex -s gind.ist ltxcmds.idx
%       pdflatex ltxcmds.dtx
%
% Installation:
%    TDS:tex/generic/ltxcmds/ltxcmds.sty
%    TDS:doc/latex/ltxcmds/ltxcmds.pdf
%    TDS:source/latex/ltxcmds/ltxcmds.dtx
%
%<*ignore>
\begingroup
  \catcode123=1 %
  \catcode125=2 %
  \def\x{LaTeX2e}%
\expandafter\endgroup
\ifcase 0\ifx\install y1\fi\expandafter
         \ifx\csname processbatchFile\endcsname\relax\else1\fi
         \ifx\fmtname\x\else 1\fi\relax
\else\csname fi\endcsname
%</ignore>
%<*install>
\input docstrip.tex
\Msg{************************************************************************}
\Msg{* Installation}
\Msg{* Package: ltxcmds 2019/12/15 v1.24 LaTeX kernel commands for general use (HO)}
\Msg{************************************************************************}

\keepsilent
\askforoverwritefalse

\let\MetaPrefix\relax
\preamble

This is a generated file.

Project: ltxcmds
Version: 2019/12/15 v1.24

Copyright (C)
   2009-2011 Heiko Oberdiek
   2016-2019 Oberdiek Package Support Group

This work may be distributed and/or modified under the
conditions of the LaTeX Project Public License, either
version 1.3c of this license or (at your option) any later
version. This version of this license is in
   https://www.latex-project.org/lppl/lppl-1-3c.txt
and the latest version of this license is in
   https://www.latex-project.org/lppl.txt
and version 1.3 or later is part of all distributions of
LaTeX version 2005/12/01 or later.

This work has the LPPL maintenance status "maintained".

The Current Maintainers of this work are
Heiko Oberdiek and the Oberdiek Package Support Group
https://github.com/ho-tex/ltxcmds/issues


The Base Interpreter refers to any `TeX-Format',
because some files are installed in TDS:tex/generic//.

This work consists of the main source file ltxcmds.dtx
and the derived files
   ltxcmds.sty, ltxcmds.pdf, ltxcmds.ins, ltxcmds.drv,
   ltxcmds-test1.tex, ltxcmds-test-gobble.tex,
   ltxcmds-test-ifempty.tex, ltxcmds-test-zapspace.tex,
   ltxcmds-test-ifboxempty.tex, ltxcmds-test-nextchar.tex,
   ltxcmds-test-carcdr.tex.

\endpreamble
\let\MetaPrefix\DoubleperCent

\generate{%
  \file{ltxcmds.ins}{\from{ltxcmds.dtx}{install}}%
  \file{ltxcmds.drv}{\from{ltxcmds.dtx}{driver}}%
  \usedir{tex/generic/ltxcmds}%
  \file{ltxcmds.sty}{\from{ltxcmds.dtx}{package}}%
}

\catcode32=13\relax% active space
\let =\space%
\Msg{************************************************************************}
\Msg{*}
\Msg{* To finish the installation you have to move the following}
\Msg{* file into a directory searched by TeX:}
\Msg{*}
\Msg{*     ltxcmds.sty}
\Msg{*}
\Msg{* To produce the documentation run the file `ltxcmds.drv'}
\Msg{* through LaTeX.}
\Msg{*}
\Msg{* Happy TeXing!}
\Msg{*}
\Msg{************************************************************************}

\endbatchfile
%</install>
%<*ignore>
\fi
%</ignore>
%<*driver>
\NeedsTeXFormat{LaTeX2e}
\ProvidesFile{ltxcmds.drv}%
  [2019/12/15 v1.24 LaTeX kernel commands for general use (HO)]%
\documentclass{ltxdoc}
\usepackage{holtxdoc}[2011/11/22]
\usepackage{zref-savepos}
\usepackage{paralist}
\hfuzz=1pt
\begin{document}
  \DocInput{ltxcmds.dtx}%
\end{document}
%</driver>
% \fi
%
%
%
% \GetFileInfo{ltxcmds.drv}
%
% \title{The \xpackage{ltxcmds} package}
% \date{2019/12/15 v1.24}
% \author{Heiko Oberdiek\thanks
% {Please report any issues at \url{https://github.com/ho-tex/ltxcmds/issues}}}
%
% \maketitle
%
% \begin{abstract}
% The package \xpackage{ltxcmds} exports some utility macros
% from the \LaTeX\ kernel into a separate namespace and
% also provides them for other formats such as plain-\TeX.
% \end{abstract}
%
% \tableofcontents
%
% \newcounter{to}
% \renewcommand*{\theto}{to\arabic{to}}
% \newlength\tolength
% \tolength=\dimexpr .5\linewidth+\oddsidemargin+1in+15mm\relax
% \newcommand*{\TO}{^^A
%   \quad
%   \stepcounter{to}^^A
%   \zsavepos\theto
%   \ifdim\zposx\theto sp>0pt
%     \ifdim\zposx\theto sp<\tolength
%       \kern\dimexpr\tolength-\zposx\theto sp\relax
%     \fi
%   \fi
%   $\rightarrow$\quad
%   \ignorespaces
% }
%
% \section{Documentation}
%
% \subsection{Introduction}
%
% Many of my packages also support other formats such as plain-\TeX.
% Because I am rather familiar with the utility macros from
% \LaTeX's kernel (e.g. \cs{@gobble}, \cs{@firstoftwo}), I found
% myself rewriting them again and again, because they are lacking
% in plain-\TeX.
%
% Therefore this package provides often used macros and similar
% ones with the name prefix \cs{ltx@}. This avoids also faulty
% redefinitions. I remember an example where a package redefined
% \cs{@firstoftwo} with forgetting \cs{long}.
%
% \subsection{Numbers}
%
% \begin{declcs}{ltx@zero} \TO 0\\
%   \cs{ltx@one} \TO 1\\
%   \cs{ltx@two} \TO 2\\
%   \cs{ltx@cclv} \TO 255\\
%   \cs{ltx@minusone} \TO -1
% \end{declcs}
% These commands are numbers 0, 1, 2, 255 and -1. They are not digits
% and a space is not gobbled afterwards. Macro \cs{ltx@minusone}
% is available since version 2010/12/12 v1.15.
%
% \subsection{Scratch registers}
%
% Following the conventions of \hologo{plainTeX} and \hologo{LaTeX}
% the first ten registers are free to use. Even numbered registers
% are for local, odd numbered for global use.
%
% \begin{declcs}{ltx@(Loc,Glob)(Toks,Dimen,Skip)(A,B,C,D,E)}
% \end{declcs}
% The name consists of the prefix \cs{ltx@}, then
% \texttt{Loc} or \texttt{Glob} for local or global usage follows.
% The register type is given by \texttt{Toks} for token register,
% \texttt{Dimen} for dimen register and \texttt{Skip} for skip
% register. As last part the registers are numbered from \texttt{A}
% to \texttt{E}. Example: \cs{ltx@LocToksA}.
%
% Since 2011/04/14 v1.19.
%
% \subsection{Argument killers}
%
% \begin{declcs}{ltx@gobble} \M{1} \TO\\
%   \cs{ltx@gobbletwo} \M{1} \M{2} \TO\\
%   \cs{ltx@gobblethree} \M{1} \M{2} \M{3} \TO\\
%   \cs{ltx@gobblefour} \M{1} \M{2} \M{3} \M{4} \TO
% \end{declcs}
%
% \begin{declcs}{ltx@GobbleNum} \M{num} \M{1} \M{2} \dots\space
%   \M{\meta{num}} \TO
% \end{declcs}
% The first argument \meta{num} of macro \cs{ltx@GobbleNum} specifies,
% how many following arguments are eaten. Macro \cs{ltx@GobbleNum}
% is expandable in exact two expansion steps.
%
% \subsection{Argument grabbers}
%
% \begin{declcs}{ltx@firstofone} \M{1} \TO \meta{1}\\[1ex]
%  \cs{ltx@firstoftwo} \M{1} \M{2} \TO \meta{1}\\
%  \cs{ltx@secondoftwo} \M{1} \M{2} \TO \meta{2}\\[1ex]
%  \cs{ltx@firstofthree} \M{1} \M{2} \M{3} \TO \meta{1}\\
%  \cs{ltx@secondofthree} \M{1} \M{2} \M{3} \TO \meta{2}\\
%  \cs{ltx@thirdofthree} \M{1} \M{2} \M{3} \TO \meta{3}\\
%  \cs{ltx@firstoffour} \M{1} \M{2} \M{3} \M{4} \TO \meta{1}\\
%  \cs{ltx@secondoffour} \M{1} \M{2} \M{3} \M{4} \TO \meta{2}\\
%  \cs{ltx@thirdoffour} \M{1} \M{2} \M{3} \M{4} \TO \meta{3}\\
%  \cs{ltx@fourthoffour} \M{1} \M{2} \M{3} \M{4} \TO \meta{4}
% \end{declcs}
% Macros \cs{ltx@firstofthree}, \cs{ltx@secondofthree}
% and \cs{ltx@thirdofthree} were added in version
% 2010/11/12 v1.11.
% Macros \cs{ltx@firstoffour}, \dots, \cs{ltx@fourthoffour}
% were added in version 2011/02/04 v1.16.
%
% \subsection{List helpers}
%
% \begin{declcs}{ltx@carzero} \dots\ \cs{@nil} \TO\\
%   \cs{ltx@cdrzero} \dots\ \cs{@nil} \TO \dots
% \end{declcs}
%
% \begin{declcs}{ltx@car} \M{1} \dots\ \cs{@nil} \TO \meta{1}\\
%   \cs{ltx@cdr} \M{1} \dots\ \cs{@nil} \TO \dots
% \end{declcs}
%
% \begin{declcs}{ltx@cartwo} \M{1} \M{2} \dots\ \cs{@nil}
%   \TO \meta{1}\meta{2}\\
%   \cs{ltx@carsecond} \M{1} \M{2} \dots\ \cs{@nil} \TO \meta{2}\\
%   \cs{ltx@cdrtwo} \M{1} \M{2} \dots\ \cs{@nil} \TO \dots
% \end{declcs}
%
% \begin{declcs}{ltx@carthree} \M{1} \M{2} \M{3} \dots\ \cs{@nil}
%       \TO \meta{1}\meta{2}\meta{3}\\
%   \cs{ltx@carthird} \M{1} \M{2} \M{3} \dots\ \cs{@nil}
%       \TO \meta{3}\\
%   \cs{ltx@cdrthree} \M{1} \M{2} \M{3} \dots\ \cs{@nil} \TO \dots
% \end{declcs}
%
% \begin{declcs}{ltx@carfour} \M{1} \M{2} \M{3} \M{4} \dots\ \cs{@nil}
%       \TO \meta{1}\meta{2}\meta{3}\meta{4}\\
%   \cs{ltx@carfourth} \M{1} \M{2} \M{3} \M{4} \dots\ \cs{@nil}
%       \TO \meta{4}\\
%   \cs{ltx@cdrfour} \M{1} \M{2} \M{3} \M{4} \dots\ \cs{@nil} \TO \dots
% \end{declcs}
%
% \begin{declcs}{ltx@CarNum} \M{num} \M{1} \dots\ \M{\meta{num}}
%   \M{\meta{num}+1} \dots\ \cs{@nil}\\
%   \TO \M{1} \dots\ \M{\meta{num}} \dots\\
%   \cs{ltx@CarNumth} \M{num} \M{1} \dots\ \M{\meta{num}}
%   \M{\meta{num}+1} \dots\ \cs{@nil}\\
%   \TO \M{\meta{num}} \dots\\
%   \cs{ltx@CdrNum} \M{num} \M{1} \dots\ \M{\meta{num}}
%   \M{\meta{num}+1} \dots\ \cs{@nil}\\
%   \TO \M{\meta{num}+1} \dots
% \end{declcs}
%
% Macros with uppercase letters are expandable in two expansion steps.
% Changes in version 2019/12/15 v1.24:
% \begin{itemize}
% \item
%   Macros \cs{ltx@carsecond}, \cs{ltx@carthird}, \cs{ltx@carfourth},
%   \cs{ltx@CarNumth} added.
% \item
%   Macros \cs{ltx@cdr}, \cs{ltx@cdrtwo}, \cs{ltx@cdrthree}, \cs{ltx@cdrfour},
%   \cs{ltx@CdrNum} are expandable in two expansion steps and retain
%   spaces and braces after the first gobbled arguments.
% \end{itemize}
%
% \subsection{Tail recursion}
%
% \begin{declcs}{ltx@ReturnAfterFi} \M{1} \cs{fi} \TO \cs{fi} \meta{1}\\
%   \cs{ltx@ReturnAfterElseFi} \M{1} \cs{else} \M{2} \cs{fi}
%     \TO \cs{fi} \meta{1}
% \end{declcs}
%
% \subsection{Empty macro}
%
% \begin{declcs}{ltx@empty} \TO
% \end{declcs}
%
% \subsection{Characters}
%
% \begin{declcs}{ltx@space} \TO \verb*| |\\
%   \cs{ltx@percentchar} \TO \texttt{\%}\\
%   \cs{ltx@backslashchar} \TO
%      \makeatletter\texttt{\@backslashchar}\\
%   \cs{ltx@hashchar} \TO \texttt{\#}\qquad(since v1.7)\\
%   \cs{ltx@leftbracechar} \TO \texttt{\textbraceleft}\qquad(since v1.8)\\
%   \cs{ltx@rightbracechar} \TO \texttt{\textbraceright}\qquad(since v1.8)
% \end{declcs}
%
% \subsection{Boolean switch}
%
% \begin{declcs}{ltx@newif} \M{cmd}
% \end{declcs}
% \cs{ltx@newif} defines a new boolean switch \meta{cmd} like \cs{newif}.
% Unlike \plainTeX's \cs{newif}, \cs{ltx@newif} is not \cs{outer}.
% The command \meta{cmd} must start with the two characters |if|.
%
% \begin{declcs}{ltx@newglobalif} \M{cmd}
% \end{declcs}
% \cs{ltx@newglobalif} defines a new boolean switch \meta{cmd}
% like \cs{ltx@newif}. However the switch setting commands,
% \meta{cmd} without the prefix |if| and followed by |true| or |false|
% are acting globally.
%
% \subsection{Command definitions}
%
% \begin{declcs}{ltx@ifundefined} \M{cmd} \M{yes} \M{no}
% \end{declcs}
% If \eTeX\ is available, \cs{ifcsname} is used that does
% not have the side effect of defining undefined commands
% with meaning of \cs{relax}. This command is always expandable.
% Change in version 1.1: Also the meaning \cs{relax} is
% always considered ``undefined''.
%
% \begin{declcs}{ltx@IfUndefined} \M{cmd} \M{yes} \M{no}
% \end{declcs}
% If \eTeX\ is available, \cs{ifcsname} is used that does
% not have the side effect of defining undefined commands
% with meaning of \cs{relax}. Also it always checks for
% the meaning of \cs{relax} and considers this as undefined.
% This macro is not expandable without \eTeX.
%
% \begin{declcs}{ltx@LocalExpandAfter}
% \end{declcs}
% It expands the token after the next token but in a local
% context. That is the difference to \cs{expandafter}.
% The local context discards the side effect of \cs{csname}
% and let the command undefined after the expansion step.
%
% \subsection{Stripping}
%
% \begin{declcs}{ltx@RemovePrefix}\\
%   \cs{ltx@StripPrefix}
% \end{declcs}
% All tokens up to and including the next available character
% `\textgreater' are thrown away. Usually it is used to strip
% the first part of the output of the commands \cs{meaning}
% or \cs{pdflastmatch}. Macro \cs{ltx@RemovePrefix}
% has the same meaning as \LaTeX's \cs{strip@prefix}, whereas
% macro \cs{ltx@StripPrefix} expands the next token
% once before stripping the prefix.
%
% \begin{declcs}{ltx@onelevel@sanitize} \M{macro}
% \end{declcs}
% Macro \cs{ltx@onelevel@sanitize} provides \hologo{LaTeX}'s
% \cs{@onelevel@sanitize}. The macro is expanded once and
% the contents is converted to characters with catcode 12 (other)
% and space tokens with catcode 10 (space). Then then sanitized
% contents is stored into the macro again. Since version 1.12.
%
% \subsection{File management}
%
% All macros in this section are expandable like
% the counterparts of the \LaTeX\ kernel. Also they
% can be used after the preamble.
%
% \subsubsection{File extensions}
%
% \begin{declcs}{ltx@clsextension}\\
%   \cs{ltx@pkgextension}
% \end{declcs}
% Macros \cs{ltx@clsextension} and \cs{ltx@styextension}
% stores the strings |cls| and |sty|. In opposite to \hologo{LaTeX}'s
% \cs{@clsextension} and \cs{@styextension} they can also be
% used after |\begin{document}|.
%
% \subsubsection{Load check}
%
% \begin{declcs}{ltx@ifclassloaded} \M{class} \M{yes} \M{no}\\
%   \cs{ltx@ifpackageloaded} \M{package} \M{yes} \M{no}
% \end{declcs}
% Macros \cs{ltx@ifclassloaded}\slash\cs{ltx@ifpackageloaded}
% execute \meta{yes}, if the \meta{class} or \meta{package} is loaded,
% otherwise \meta{no} is called.
% Both \meta{class} and \meta{package} are specified without
% extension. The macros can also be used after |\begin{document}|.
%
% \begin{declcs}{ltx@iffileloaded} \M{file} \M{yes} \M{no}
% \end{declcs}
% If \LaTeX's \cs{ProvidesFile} macro was called before
% using \meta{file} as argument, then \cs{ltx@iffileloaded}
% calls \meta{yes}, otherwise \meta{no}.
% Therefore it is possible
% that the \meta{file} is loaded, but \meta{no} is executed because
% of a missing \cs{ProvidesFile}. The \LaTeX\ kernel does not have
% a counterpart of \cs{ltx@iffileloaded}.
%
% Note that the file name used in \cs{ProvidesFile} and
% \cs{ltx@iffileloaded} must match. For example, if \TeX's
% default extension \xext{tex} was given in the first command,
% then it must also specified in the latter command and vice versa.
%
% \subsubsection{Version date check}
%
% \begin{declcs}{ltx@ifclasslater} \M{class} \M{date} \M{yes} \M{no}\\
%   \cs{ltx@ifpackagelater} \M{package} \M{date} \M{yes} \M{no}\\
%   \cs{ltx@iffilelater} \M{file} \M{date} \M{yes} \M{no}
% \end{declcs}
% If a \cs{ProvidesClass}\slash\cs{ProvidesPackage}\slash
% \cs{ProvidesFile} command with exact the same class\slash package\slash
% file was executed before with an optional
% argument that starts with a \LaTeX\ version date, then
% this version date is compared with the argument \meta{date}.
% If they are equal or if the version date is the later date,
% then \meta{yes} is called. In all other cases \meta{no} is executed.
%
% A \LaTeX\ date has the format |YYYY/MM/DD| with |YYYY| as year
% with four digits, |MM| as month with two digits and |DD| as day
% with two digits. If \pdfTeX's \cs{pdfmatch} is available, then
% it is used to detect the version date, to reject invalid
% date formats and to reject some invalid dates.
% Dates before 1994/01/01 are always invalid, because version
% dates are introduced with \LaTeXe\ in 1994.
%
% \subsection{Macro additions}
%
% \begin{declcs}{ltx@GlobalAppendToMacro} \M{cmd} \M{addition}\\
%   \cs{ltx@LocalAppendToMacro} \M{cmd} \M{addition}
% \end{declcs}
% The \meta{addition} is appended to the parameterless macro \meta{cmd}.
% If \meta{cmd} is undefined or has the meaning \cs{relax}, then
% it will be initialized as empty macro beforehand. Due to a bug \meta{addition}
% must not contain \cs{par} before version 2010/10/25 v1.9.
%
% \begin{declcs}{ltx@GlobalPrependToMacro} \M{cmd} \M{addition}\\
%   \cs{ltx@LocalPrependToMacro} \M{cmd} \M{addition}
% \end{declcs}
% The \meta{addition} is prepended to the parameterless macro \meta{cmd}.
% If \meta{cmd} is undefined or has the meaning \cs{relax}, then
% it will be initialized as empty macro beforehand. The macros
% were added in version 2011/08/22 v1.21.
%
% \subsection{Next character detection}
%
% \begin{declcs}{ltx@ifnextchar} \M{char} \M{yes} \M{no}
% \end{declcs}
% If next character is \meta{char} then \meta{yes} is called,
% otherwise \meta{no}. The character is not removed.
% Spaces are silently removed when looking for \meta{char}
% as \hologo{LaTeX}'s version \cs{kernel@ifnextchar} does.
% But there are also small differences:
% \begin{itemize}
% \item The space can be used as \meta{char}. In this case
%   optional spaces before \meta{char} are not supported of course.
% \item If the optional space is a command that is a character
%   (defined by \cs{let} or \cs{futurelet}), then
%   \cs{kernel@ifnextchar} breaks with an \hologo{TeX} error.
%   \cs{ltx@ifnextchar} silently removes this token as
%   optional space.
% \end{itemize}
% Since 2010/03/01 v1.3.
%
% \begin{declcs}{ltx@ifnextchar@nospace} \M{char} \M{yes} \M{no}
% \end{declcs}
% Macro \cs{ltx@ifnextchar@nospace} behaves like
% macro \cs{ltx@ifnextchar} with the exception that
% optional spaces are not supported before \meta{char}.
% Since 2011/04/14 v1.19.
%
% \subsection{\cs{ltx@leavevmode}, \cs{ltx@mbox}}
%
% \begin{declcs}{ltx@leavevmode}
% \end{declcs}
% Macro \cs{ltx@leavevmode} calls \hologo{pdfTeX}'s \cs{quitvmode}.
% Otherwise \cs{leavevmode} is used and defined if it is necessary.
%
% \begin{declcs}{ltx@mbox}
% \end{declcs}
% Macro \cs{ltx@mbox} reimplements \cs{mbox} with two changes.
% Instead of \cs{leavevmode} it uses \cs{ltx@leavevmode} and
% stops right after \cs{hbox}. Especially it does not grab
% the argument and allows the extended syntax of \cs{hbox}.
%
% \subsection{Expandable test for emptiness}
%
% \begin{declcs}{ltx@ifempty} \M{stuff} \M{yes} \M{no}
% \end{declcs}
% Macro \cs{ltx@ifempty} checks in exact two expansion steps
% whether \meta{stuff}
% is empty or contains token. Depending on the result
% \meta{yes} or \meta{no} is executed. The token in \meta{stuff}
% may contain \cs{par} and unmatched conditionals (\cs{if}, \cs{else},
% \cs{fi}, \dots). Since version 2010/11/12 v1.11.
%
% \begin{declcs}{ltx@ifblank} \M{stuff} \M{yes} \M{no}
% \end{declcs}
% Macro \cs{ltx@ifblank} tests in exact two expansion steps
% if \meta{stuff} is empty or contain only blank spaces.
% In this case argument \meta{yes} is called. If \meta{stuff}
% contains other tokens than spaces then \meta{no} is executed.
% Since version 2010/12/04 v1.13.
%
% \subsection{Stripping spaces}
%
% \begin{declcs}{ltx@zapspace} \M{stuff}
% \end{declcs}
% Macro \cs{ltx@zapspace} strips spaces from \meta{stuff}
% that are not hidden inside curly braces. Like \hologo{LaTeX}'s
% \cs{zap@space} it is expandable. Differences:
% \begin{itemize}
% \item
% Syntax: \cs{zap@space} also expects a space token
% and \cs{@empty} after \meta{stuff}.
% \item
% Macro \cs{ltx@zapspace} is expandable in exact two
% expansion steps.
% \item
% Macro \cs{ltx@zapspace} always retains curly braces.
% \item
% Macro \cs{zap@space} has a bug. It stops stripping spaces
% after a token group in curly braces if the first two tokens
% inside the group are equal.
% \item
% Macro \cs{ltx@zapspace} also works with \cs{par}
% and conditionals (\cs{if}, \cs{else}, \cs{fi}, \dots).
% \end{itemize}
% Macro \cs{ltx@zapspace} is available since version
% 2010/12/07 v1.14.
%
% \subsection{Check for emptiness of boxes}
%
% \begin{declcs}{ltx@IfBoxEmpty} \M{box register number}
%   \M{yes} \M{no}
% \end{declcs}
% Macro \cs{ltx@IfBoxEmpty} calls \meta{yes} if
% the box exists (\cs{ifvoid} returns false) and the box
% does not contain any content. Otherwise if the box
% is void or contains something, then \meta{no} is executed.
% Thus being empty means that the box exists and is either
% an \cs{hbox} or a \cs{vbox} and may even have dimensions other than
% 0.0\,pt, but the box does not contain anything.
% Macro \cs{ltx@IfBoxEmpty} is available since 2010/02/04 v1.16.
%
% \begin{declcs}{ltx@IfBoxVoidOrEmpty} \M{box register number}
%   \M{yes} \M{no}
% \end{declcs}
% Macro \cs{ltx@IfBoxVoidOrEmpty} calls \meta{yes} if
% the box is either void or does not contain any content.
% Otherwise \meta{no} is executed.
% Macro \cs{ltx@IfBoxVoidOrEmpty} is available since 2010/02/04 v1.16.
%
% \StopEventually{
% }
%
% \section{Implementation}
%
% \subsection{Identification}
%
%    \begin{macrocode}
%<*package>
%    \end{macrocode}
%    Reload check, especially if the package is not used with \LaTeX.
%    \begin{macrocode}
\begingroup\catcode61\catcode48\catcode32=10\relax%
  \catcode13=5 % ^^M
  \endlinechar=13 %
  \catcode35=6 % #
  \catcode39=12 % '
  \catcode44=12 % ,
  \catcode45=12 % -
  \catcode46=12 % .
  \catcode58=12 % :
  \catcode64=11 % @
  \catcode123=1 % {
  \catcode125=2 % }
  \expandafter\let\expandafter\x\csname ver@ltxcmds.sty\endcsname
  \ifx\x\relax % plain-TeX, first loading
  \else
    \def\empty{}%
    \ifx\x\empty % LaTeX, first loading,
      % variable is initialized, but \ProvidesPackage not yet seen
    \else
      \expandafter\ifx\csname PackageInfo\endcsname\relax
        \def\x#1#2{%
          \immediate\write-1{Package #1 Info: #2.}%
        }%
      \else
        \def\x#1#2{\PackageInfo{#1}{#2, stopped}}%
      \fi
      \x{ltxcmds}{The package is already loaded}%
      \aftergroup\endinput
    \fi
  \fi
\endgroup%
%    \end{macrocode}
%    Package identification:
%    \begin{macrocode}
\begingroup\catcode61\catcode48\catcode32=10\relax%
  \catcode13=5 % ^^M
  \endlinechar=13 %
  \catcode35=6 % #
  \catcode39=12 % '
  \catcode40=12 % (
  \catcode41=12 % )
  \catcode44=12 % ,
  \catcode45=12 % -
  \catcode46=12 % .
  \catcode47=12 % /
  \catcode58=12 % :
  \catcode64=11 % @
  \catcode91=12 % [
  \catcode93=12 % ]
  \catcode123=1 % {
  \catcode125=2 % }
  \expandafter\ifx\csname ProvidesPackage\endcsname\relax
    \def\x#1#2#3[#4]{\endgroup
      \immediate\write-1{Package: #3 #4}%
      \xdef#1{#4}%
    }%
  \else
    \def\x#1#2[#3]{\endgroup
      #2[{#3}]%
      \ifx#1\@undefined
        \xdef#1{#3}%
      \fi
      \ifx#1\relax
        \xdef#1{#3}%
      \fi
    }%
  \fi
\expandafter\x\csname ver@ltxcmds.sty\endcsname
\ProvidesPackage{ltxcmds}%
  [2019/12/15 v1.24 LaTeX kernel commands for general use (HO)]%
%    \end{macrocode}
%
%    \begin{macrocode}
\begingroup\catcode61\catcode48\catcode32=10\relax%
  \catcode13=5 % ^^M
  \endlinechar=13 %
  \catcode123=1 % {
  \catcode125=2 % }
  \catcode64=11 % @
  \def\x{\endgroup
    \expandafter\edef\csname LTXcmds@AtEnd\endcsname{%
      \endlinechar=\the\endlinechar\relax
      \catcode13=\the\catcode13\relax
      \catcode32=\the\catcode32\relax
      \catcode35=\the\catcode35\relax
      \catcode61=\the\catcode61\relax
      \catcode64=\the\catcode64\relax
      \catcode123=\the\catcode123\relax
      \catcode125=\the\catcode125\relax
    }%
  }%
\x\catcode61\catcode48\catcode32=10\relax%
\catcode13=5 % ^^M
\endlinechar=13 %
\catcode35=6 % #
\catcode64=11 % @
\catcode123=1 % {
\catcode125=2 % }
\def\TMP@EnsureCode#1#2{%
  \edef\LTXcmds@AtEnd{%
    \LTXcmds@AtEnd
    \catcode#1=\the\catcode#1\relax
  }%
  \catcode#1=#2\relax
}
\TMP@EnsureCode{36}{3}% $
\TMP@EnsureCode{38}{4}% &
\TMP@EnsureCode{40}{12}% (
\TMP@EnsureCode{41}{12}% )
\TMP@EnsureCode{45}{12}% -
\TMP@EnsureCode{46}{12}% .
\TMP@EnsureCode{47}{12}% /
\TMP@EnsureCode{60}{12}% <
\TMP@EnsureCode{62}{12}% >
\TMP@EnsureCode{91}{12}% [
\TMP@EnsureCode{96}{12}% `
\TMP@EnsureCode{93}{12}% ]
\TMP@EnsureCode{94}{12}% ^ (superscript) (!)
\TMP@EnsureCode{124}{12}% |
\edef\LTXcmds@AtEnd{\LTXcmds@AtEnd\noexpand\endinput}
%    \end{macrocode}
%
% \subsection{Numbers}
%
%    \begin{macro}{\ltx@zero}
%    \begin{macrocode}
\chardef\ltx@zero=0 %
%    \end{macrocode}
%    \end{macro}
%    \begin{macro}{\ltx@one}
%    \begin{macrocode}
\chardef\ltx@one=1 %
%    \end{macrocode}
%    \end{macro}
%    \begin{macro}{\ltx@two}
%    \begin{macrocode}
\chardef\ltx@two=2 %
%    \end{macrocode}
%    \end{macro}
%    \begin{macro}{\ltx@active}
%    \begin{macrocode}
\chardef\ltx@active=13 %
%    \end{macrocode}
%    \end{macro}
%    \begin{macro}{\ltx@cclv}
%    \begin{macrocode}
\chardef\ltx@cclv=255 %
%    \end{macrocode}
%    \end{macro}
%
%    \begin{macro}{\ltx@minusone}
%    \begin{macrocode}
\def\ltx@minusone{%
  -\ltx@one
}
%    \end{macrocode}
%    \end{macro}
%
% \subsection{Scratch registers}
%
%    \begin{macro}{\ltx@LocToksA}
%    \begin{macrocode}
\toksdef\ltx@LocToksA=0 %
%    \end{macrocode}
%    \end{macro}
%    \begin{macro}{\ltx@LocToksB}
%    \begin{macrocode}
\toksdef\ltx@LocToksB=2 %
%    \end{macrocode}
%    \end{macro}
%    \begin{macro}{\ltx@LocToksC}
%    \begin{macrocode}
\toksdef\ltx@LocToksC=4 %
%    \end{macrocode}
%    \end{macro}
%    \begin{macro}{\ltx@LocToksD}
%    \begin{macrocode}
\toksdef\ltx@LocToksD=6 %
%    \end{macrocode}
%    \end{macro}
%    \begin{macro}{\ltx@LocToksE}
%    \begin{macrocode}
\toksdef\ltx@LocToksE=8 %
%    \end{macrocode}
%    \end{macro}
%
%    \begin{macro}{\ltx@GlobToksA}
%    \begin{macrocode}
\toksdef\ltx@GlobToksA=1 %
%    \end{macrocode}
%    \end{macro}
%    \begin{macro}{\ltx@GlobToksB}
%    \begin{macrocode}
\toksdef\ltx@GlobToksB=3 %
%    \end{macrocode}
%    \end{macro}
%    \begin{macro}{\ltx@GlobToksC}
%    \begin{macrocode}
\toksdef\ltx@GlobToksC=5 %
%    \end{macrocode}
%    \end{macro}
%    \begin{macro}{\ltx@GlobToksD}
%    \begin{macrocode}
\toksdef\ltx@GlobToksD=7 %
%    \end{macrocode}
%    \end{macro}
%    \begin{macro}{\ltx@GlobToksE}
%    \begin{macrocode}
\toksdef\ltx@GlobToksE=9 %
%    \end{macrocode}
%    \end{macro}
%
%    \begin{macro}{\ltx@LocDimenA}
%    \begin{macrocode}
\dimendef\ltx@LocDimenA=0 %
%    \end{macrocode}
%    \end{macro}
%    \begin{macro}{\ltx@LocDimenB}
%    \begin{macrocode}
\dimendef\ltx@LocDimenB=2 %
%    \end{macrocode}
%    \end{macro}
%    \begin{macro}{\ltx@LocDimenC}
%    \begin{macrocode}
\dimendef\ltx@LocDimenC=4 %
%    \end{macrocode}
%    \end{macro}
%    \begin{macro}{\ltx@LocDimenD}
%    \begin{macrocode}
\dimendef\ltx@LocDimenD=6 %
%    \end{macrocode}
%    \end{macro}
%    \begin{macro}{\ltx@LocDimenE}
%    \begin{macrocode}
\dimendef\ltx@LocDimenE=8 %
%    \end{macrocode}
%    \end{macro}
%
%    \begin{macro}{\ltx@GlobDimenA}
%    \begin{macrocode}
\dimendef\ltx@GlobDimenA=1 %
%    \end{macrocode}
%    \end{macro}
%    \begin{macro}{\ltx@GlobDimenB}
%    \begin{macrocode}
\dimendef\ltx@GlobDimenB=3 %
%    \end{macrocode}
%    \end{macro}
%    \begin{macro}{\ltx@GlobDimenC}
%    \begin{macrocode}
\dimendef\ltx@GlobDimenC=5 %
%    \end{macrocode}
%    \end{macro}
%    \begin{macro}{\ltx@GlobDimenD}
%    \begin{macrocode}
\dimendef\ltx@GlobDimenD=7 %
%    \end{macrocode}
%    \end{macro}
%    \begin{macro}{\ltx@GlobDimenE}
%    \begin{macrocode}
\dimendef\ltx@GlobDimenE=9 %
%    \end{macrocode}
%    \end{macro}
%
%    \begin{macro}{\ltx@LocSkipA}
%    \begin{macrocode}
\skipdef\ltx@LocSkipA=0 %
%    \end{macrocode}
%    \end{macro}
%    \begin{macro}{\ltx@LocSkipB}
%    \begin{macrocode}
\skipdef\ltx@LocSkipB=2 %
%    \end{macrocode}
%    \end{macro}
%    \begin{macro}{\ltx@LocSkipC}
%    \begin{macrocode}
\skipdef\ltx@LocSkipC=4 %
%    \end{macrocode}
%    \end{macro}
%    \begin{macro}{\ltx@LocSkipD}
%    \begin{macrocode}
\skipdef\ltx@LocSkipD=6 %
%    \end{macrocode}
%    \end{macro}
%    \begin{macro}{\ltx@LocSkipE}
%    \begin{macrocode}
\skipdef\ltx@LocSkipE=8 %
%    \end{macrocode}
%    \end{macro}
%
%    \begin{macro}{\ltx@GlobSkipA}
%    \begin{macrocode}
\skipdef\ltx@GlobSkipA=1 %
%    \end{macrocode}
%    \end{macro}
%    \begin{macro}{\ltx@GlobSkipB}
%    \begin{macrocode}
\skipdef\ltx@GlobSkipB=3 %
%    \end{macrocode}
%    \end{macro}
%    \begin{macro}{\ltx@GlobSkipC}
%    \begin{macrocode}
\skipdef\ltx@GlobSkipC=5 %
%    \end{macrocode}
%    \end{macro}
%    \begin{macro}{\ltx@GlobSkipD}
%    \begin{macrocode}
\skipdef\ltx@GlobSkipD=7 %
%    \end{macrocode}
%    \end{macro}
%    \begin{macro}{\ltx@GlobSkipE}
%    \begin{macrocode}
\skipdef\ltx@GlobSkipE=9 %
%    \end{macrocode}
%    \end{macro}
%
%
% \subsection{Argument killers}
%
%    \begin{macro}{\ltx@gobble}
%    \begin{macrocode}
\long\def\ltx@gobble#1{}
%    \end{macrocode}
%    \end{macro}
%    \begin{macro}{\ltx@gobbletwo}
%    \begin{macrocode}
\long\def\ltx@gobbletwo#1#2{}
%    \end{macrocode}
%    \end{macro}
%    \begin{macro}{\ltx@gobblethree}
%    \begin{macrocode}
\long\def\ltx@gobblethree#1#2#3{}
%    \end{macrocode}
%    \end{macro}
%    \begin{macro}{\ltx@gobblefour}
%    \begin{macrocode}
\long\def\ltx@gobblefour#1#2#3#4{}
%    \end{macrocode}
%    \end{macro}
%
%    \begin{macro}{\ltx@GobbleNum}
%    \begin{macrocode}
\def\ltx@GobbleNum#1{%
  \romannumeral
  \csname ltx@zero%
  \expandafter\LTXcmds@GobbleNum
  \romannumeral\LTXcmds@num{#1}000{m\endcsname}%
}
%    \end{macrocode}
%    \end{macro}
%    \begin{macro}{\LTXcmds@GobbleNum}
%    \begin{macrocode}
\def\LTXcmds@GobbleNum#1{%
  \csname LTXcmds@G#1\LTXcmds@GobbleNum
}
%    \end{macrocode}
%    \end{macro}
%    \begin{macro}{\LTXcmds@Gm}
%    \begin{macrocode}
\long\def\LTXcmds@Gm#1{%
  \endcsname
}
%    \end{macrocode}
%    \end{macro}
%
% \subsection{Argument grabbers}
%
%    \begin{macro}{\ltx@firstofone}
%    \begin{macrocode}
\long\def\ltx@firstofone#1{#1}
%    \end{macrocode}
%    \end{macro}
%    \begin{macro}{\ltx@firstoftwo}
%    \begin{macrocode}
\long\def\ltx@firstoftwo#1#2{#1}
%    \end{macrocode}
%    \end{macro}
%    \begin{macro}{\ltx@secondoftwo}
%    \begin{macrocode}
\long\def\ltx@secondoftwo#1#2{#2}
%    \end{macrocode}
%    \end{macro}
%
%    \begin{macro}{\ltx@firstofthree}
%    \begin{macrocode}
\long\def\ltx@firstofthree#1#2#3{#1}
%    \end{macrocode}
%    \end{macro}
%    \begin{macro}{\ltx@secondofthree}
%    \begin{macrocode}
\long\def\ltx@secondofthree#1#2#3{#2}
%    \end{macrocode}
%    \end{macro}
%    \begin{macro}{\ltx@thirdofthree}
%    \begin{macrocode}
\long\def\ltx@thirdofthree#1#2#3{#3}%
%    \end{macrocode}
%    \end{macro}
%
%    \begin{macro}{\ltx@firstoffour}
%    \begin{macrocode}
\long\def\ltx@firstoffour#1#2#3#4{#1}
%    \end{macrocode}
%    \end{macro}
%    \begin{macro}{\ltx@secondoffour}
%    \begin{macrocode}
\long\def\ltx@secondoffour#1#2#3#4{#2}
%    \end{macrocode}
%    \end{macro}
%    \begin{macro}{\ltx@thirdoffour}
%    \begin{macrocode}
\long\def\ltx@thirdoffour#1#2#3#4{#3}%
%    \end{macrocode}
%    \end{macro}
%    \begin{macro}{\ltx@fourthoffour}
%    \begin{macrocode}
\long\def\ltx@fourthoffour#1#2#3#4{#4}%
%    \end{macrocode}
%    \end{macro}
%
% \subsection{List helpers}
%
%    \begin{macro}{\ltx@carzero}
%    \begin{macrocode}
\long\def\ltx@carzero#1\@nil{}%
%    \end{macrocode}
%    \end{macro}
%    \begin{macro}{\LTXcmds@cdrzero}
%    \begin{macrocode}
\long\def\LTXcmds@cdrzero#1\@nil{#1}
%    \end{macrocode}
%    \end{macro}
%    \begin{macro}{\ltx@cdrzero}
%    \begin{macrocode}
\def\ltx@cdrzero{%
  \romannumeral\LTXcmds@cdrzero\ltx@zero
}
%    \end{macrocode}
%    \end{macro}
%
%    \begin{macro}{\ltx@car}
%    \begin{macrocode}
\long\def\ltx@car#1#2\@nil{#1}
%    \end{macrocode}
%    \end{macro}
%    \begin{macro}{\ltx@cdr}
%    \begin{macrocode}
\long\def\ltx@cdr#1{%
  \romannumeral\LTXcmds@cdrzero\ltx@zero
}
%    \end{macrocode}
%    \end{macro}
%
%    \begin{macro}{\ltx@cartwo}
%    \begin{macrocode}
\long\def\ltx@cartwo#1#2#3\@nil{#1#2}
%    \end{macrocode}
%    \end{macro}
%    \begin{macro}{\ltx@carsecond}
%    \begin{macrocode}
\long\def\ltx@carsecond#1#2#3\@nil{#2}
%    \end{macrocode}
%    \end{macro}
%    \begin{macro}{\ltx@cdrtwo}
%    \begin{macrocode}
\long\def\ltx@cdrtwo#1#2{%
  \romannumeral\LTXcmds@cdrzero\ltx@zero
}
%    \end{macrocode}
%    \end{macro}
%
%    \begin{macro}{\ltx@carthree}
%    \begin{macrocode}
\long\def\ltx@carthree#1#2#3#4\@nil{#1#2#3}
%    \end{macrocode}
%    \end{macro}
%    \begin{macro}{\ltx@carthird}
%    \begin{macrocode}
\long\def\ltx@carthird#1#2#3#4\@nil{#3}
%    \end{macrocode}
%    \end{macro}
%    \begin{macro}{\ltx@cdrthree}
%    \begin{macrocode}
\long\def\ltx@cdrthree#1#2#3{%
  \romannumeral\LTXcmds@cdrzero\ltx@zero
}
%    \end{macrocode}
%    \end{macro}
%
%    \begin{macro}{\ltx@carfour}
%    \begin{macrocode}
\long\def\ltx@carfour#1#2#3#4#5\@nil{#1#2#3#4}
%    \end{macrocode}
%    \end{macro}
%    \begin{macro}{\ltx@carfourth}
%    \begin{macrocode}
\long\def\ltx@carfourth#1#2#3#4#5\@nil{#4}
%    \end{macrocode}
%    \end{macro}
%    \begin{macro}{\ltx@cdrfour}
%    \begin{macrocode}
\long\def\ltx@cdrfour#1#2#3#4{%
  \romannumeral\LTXcmds@cdrzero\ltx@zero
}
%    \end{macrocode}
%    \end{macro}
%
%    \begin{macro}{\ltx@CarNum}
%    \begin{macrocode}
\def\ltx@CarNum#1{%
  \romannumeral
  \csname LTXcmds@CarNumFinish%
  \expandafter\LTXcmds@CarNum
  \romannumeral\LTXcmds@num{#1}000{x\endcsname}%
}
%    \end{macrocode}
%    \end{macro}
%    \begin{macro}{\LTXcmds@CarNum}
%    \begin{macrocode}
\def\LTXcmds@CarNum#1{%
  \csname LTXcmds@C#1\LTXcmds@CarNum
}
%    \end{macrocode}
%    \end{macro}
%    \begin{macro}{\LTXcmds@Cm}
%    \begin{macrocode}
\long\def\LTXcmds@Cm#1#2{%
  \endcsname{#1#2}%
}
%    \end{macrocode}
%    \end{macro}
%    \begin{macro}{\LTXcmds@Cx}
%    \begin{macrocode}
\def\LTXcmds@Cx#1{%
  \endcsname{}%
}
%    \end{macrocode}
%    \end{macro}
%    \begin{macro}{\LTXcmds@CarNumFinish}
%    \begin{macrocode}
\long\def\LTXcmds@CarNumFinish#1#2\@nil{%
  \ltx@zero
  #1%
}
%    \end{macrocode}
%    \end{macro}
%
%    \begin{macro}{\ltx@CarNumth}
%    \begin{macrocode}
\def\ltx@CarNumth#1{%
  \romannumeral
  \expandafter\expandafter\expandafter
  \LTXcmds@CarNumth
  \ltx@GobbleNum{#1}{}%
}
%    \end{macrocode}
%    \end{macro}
%    \begin{macro}{\LTXcmds@CarNumth}
%    \begin{macrocode}
\long\def\LTXcmds@CarNumth#1#2\@nil{%
  \ltx@zero
  #1%
}
%    \end{macrocode}
%    \end{macro}
%
%    \begin{macro}{\ltx@CdrNum}
%    \begin{macrocode}
\def\ltx@CdrNum#1{%
  \romannumeral%
  \expandafter\expandafter\expandafter\ltx@cdrzero
  \expandafter\expandafter\expandafter\ltx@zero
  \ltx@GobbleNum{#1}%
}
%    \end{macrocode}
%    \end{macro}
%
% \subsection{Tail recursion}
%
%    \begin{macro}{\ltx@ReturnAfterFi}
%    \begin{macrocode}
\long\def\ltx@ReturnAfterFi#1\fi{\fi#1}
%    \end{macrocode}
%    \end{macro}
%    \begin{macro}{\ltx@ReturnAfterElseFi}
%    \begin{macrocode}
\long\def\ltx@ReturnAfterElseFi#1\else#2\fi{\fi#1}
%    \end{macrocode}
%    \end{macro}
%
% \subsection{Empty macro}
%
%    \begin{macro}{\ltx@empty}
%    \begin{macrocode}
\def\ltx@empty{}
%    \end{macrocode}
%    \end{macro}
%
% \subsection{Characters}
%
%    \begin{macro}{\ltx@space}
%    \begin{macrocode}
\def\ltx@space{ }
%    \end{macrocode}
%    \end{macro}
%    \begin{macro}{\ltx@percentchar}
%    \begin{macrocode}
\begingroup
  \lccode`0=`\%\relax
\lowercase{\endgroup
  \def\ltx@percentchar{0}%
}
%    \end{macrocode}
%    \end{macro}
%    \begin{macro}{\ltx@backslashchar}
%    \begin{macrocode}
\begingroup
  \lccode`0=`\\\relax
\lowercase{\endgroup
  \def\ltx@backslashchar{0}%
}
%    \end{macrocode}
%    \end{macro}
%    \begin{macro}{\ltx@hashchar}
%    \begin{macrocode}
\begingroup
  \lccode`0=`\#\relax
\lowercase{\endgroup
  \def\ltx@hashchar{0}%
}
%    \end{macrocode}
%    \end{macro}
%    \begin{macro}{\ltx@leftbracechar}
%    \begin{macrocode}
\begingroup
  \lccode`0=`\{\relax
\lowercase{\endgroup
  \def\ltx@leftbracechar{0}%
}
%    \end{macrocode}
%    \end{macro}
%    \begin{macro}{\ltx@rightbracechar}
%    \begin{macrocode}
\begingroup
  \lccode`0=`\}\relax
\lowercase{\endgroup
  \def\ltx@rightbracechar{0}%
}
%    \end{macrocode}
%    \end{macro}
%
% \subsection{Boolean switch}
%
%    \begin{macro}{\ltx@newif}
%    \begin{macrocode}
\def\ltx@newif#1{%
  \begingroup
    \escapechar=-1 %
  \expandafter\endgroup
  \expandafter\LTXcmds@newif\string#1\@nil
}
%    \end{macrocode}
%    \end{macro}
%    \begin{macro}{\LTXcmds@newif}
%    \begin{macrocode}
\begingroup
  \escapechar=-1 %
\expandafter\endgroup
\expandafter\def\expandafter\LTXcmds@newif\string\if#1\@nil{%
  \expandafter\edef\csname#1true\endcsname{%
    \let
    \expandafter\noexpand\csname if#1\endcsname
    \noexpand\iftrue
  }%
  \expandafter\edef\csname#1false\endcsname{%
    \let
    \expandafter\noexpand\csname if#1\endcsname
    \noexpand\iffalse
  }%
  \csname#1false\endcsname
}
%    \end{macrocode}
%    \end{macro}
%
%    \begin{macro}{\ltx@newglobalif}
%    \begin{macrocode}
\def\ltx@newglobalif#1{%
  \begingroup
    \escapechar=-1 %
  \expandafter\endgroup
  \expandafter\LTXcmds@newglobalif\string#1\@nil
}
%    \end{macrocode}
%    \end{macro}
%    \begin{macro}{\LTXcmds@newglobalif}
%    \begin{macrocode}
\begingroup
  \escapechar=-1 %
\expandafter\endgroup
\expandafter
\def\expandafter\LTXcmds@newglobalif\string\if#1\@nil{%
  \expandafter\edef\csname#1true\endcsname{%
    \global\let
    \expandafter\noexpand\csname if#1\endcsname
    \noexpand\iftrue
  }%
  \expandafter\edef\csname#1false\endcsname{%
    \global\let
    \expandafter\noexpand\csname if#1\endcsname
    \noexpand\iffalse
  }%
  \csname#1false\endcsname
}
%    \end{macrocode}
%    \end{macro}
%
% \subsection{Command definitions}
%
%    \begin{macro}{\ltx@LocalExpandAfter}
%    \begin{macrocode}
\def\ltx@LocalExpandAfter{%
  \begingroup
    \expandafter\expandafter\expandafter
  \endgroup
  \expandafter
}
%    \end{macrocode}
%    \end{macro}
%
%    \begin{macrocode}
\ltx@LocalExpandAfter
\ifx\csname ifcsname\endcsname\relax
%    \end{macrocode}
%    \begin{macro}{\ltx@ifundefined}
%    \begin{macrocode}
  \def\ltx@ifundefined#1{%
    \expandafter\ifx\csname #1\endcsname\relax
      \expandafter\ltx@firstoftwo
    \else
      \expandafter\ltx@secondoftwo
    \fi
  }%
%    \end{macrocode}
%    \end{macro}
%    \begin{macro}{\ltx@IfUndefined}
%    \begin{macrocode}
  \def\ltx@IfUndefined#1{%
    \begingroup\expandafter\expandafter\expandafter\endgroup
    \expandafter\ifx\csname #1\endcsname\relax
      \expandafter\ltx@firstoftwo
    \else
      \expandafter\ltx@secondoftwo
    \fi
  }%
%    \end{macrocode}
%    \end{macro}
%    \begin{macrocode}
  \expandafter\ltx@gobble
\else
  \expandafter\ltx@firstofone
\fi
{%
%    \end{macrocode}
%    \begin{macro}{\ltx@ifundefined}
%    \begin{macrocode}
  \def\ltx@ifundefined#1{%
    \ifcsname #1\endcsname
      \expandafter\ifx\csname #1\endcsname\relax
        \expandafter\expandafter\expandafter\ltx@firstoftwo
      \else
        \expandafter\expandafter\expandafter\ltx@secondoftwo
      \fi
    \else
      \expandafter\ltx@firstoftwo
    \fi
  }%
%    \end{macrocode}
%    \end{macro}
%    \begin{macro}{\ltx@IfUndefined}
%    \begin{macrocode}
  \let\ltx@IfUndefined\ltx@ifundefined
%    \end{macrocode}
%    \end{macro}
%    \begin{macrocode}
}
%    \end{macrocode}
%
% \subsection{Stripping}
%
%    \begin{macro}{\ltx@RemovePrefix}
%    \begin{macrocode}
\def\ltx@RemovePrefix#1>{}
%    \end{macrocode}
%    \end{macro}
%
%    \begin{macro}{\ltx@StripPrefix}
%    \begin{macrocode}
\def\ltx@StripPrefix{%
  \expandafter\ltx@RemovePrefix
}
%    \end{macrocode}
%    \end{macro}
%
%    \begin{macro}{\ltx@onelevel@sanitize}
%    \begin{macrocode}
\def\ltx@onelevel@sanitize#1{%
  \edef#1{%
    \expandafter
    \ltx@RemovePrefix\meaning#1%
  }%
}
%    \end{macrocode}
%    \end{macro}
%
% \subsection{File management}
%
% \subsubsection{File extensions}
%
%    \begin{macro}{\ltx@clsextension}
%    \begin{macrocode}
\def\ltx@clsextension{cls}
%    \end{macrocode}
%    \end{macro}
%    \begin{macro}{\ltx@pkgextension}
%    \begin{macrocode}
\def\ltx@pkgextension{sty}
%    \end{macrocode}
%    \end{macro}
%
% \subsubsection{Load check}
%
%    \begin{macro}{\ltx@iffileloaded}
%    \begin{macrocode}
\def\ltx@iffileloaded#1{%
  \ltx@ifundefined{ver@#1}\ltx@secondoftwo\ltx@firstoftwo
}
%    \end{macrocode}
%    \end{macro}
%    \begin{macro}{\ltx@ifclassloaded}
%    \begin{macrocode}
\def\ltx@ifclassloaded#1{%
  \ltx@iffileloaded{#1.\ltx@clsextension}%
}
%    \end{macrocode}
%    \end{macro}
%    \begin{macro}{\ltx@ifpackageloaded}
%    \begin{macrocode}
\def\ltx@ifpackageloaded#1{%
  \ltx@iffileloaded{#1.\ltx@pkgextension}%
}
%    \end{macrocode}
%    \end{macro}
%
% \subsubsection{Version date check}
%
%    \begin{macro}{\ltx@iffilelater}
%    \begin{macrocode}
\def\ltx@iffilelater#1#2{%
  \ltx@iffileloaded{#1}{%
    \expandafter\LTXcmds@IfLater\expandafter{%
      \number
      \expandafter\expandafter\expandafter\LTXcmds@ParseVersion
      \expandafter\expandafter\expandafter{%
        \csname ver@#1\endcsname
      }%
    \expandafter}\expandafter{%
      \number
      \expandafter\LTXcmds@ParseVersion\expandafter{#2}%
    }%
  }\ltx@secondoftwo
}
%    \end{macrocode}
%    \end{macro}
%    \begin{macro}{\LTXcmds@IfLater}
%    \begin{macrocode}
\def\LTXcmds@IfLater#1#2{%
  \ifcase 0%
      \ifnum#1<19940101 %
      \else
        \ifnum#2<19940101 %
        \else
          \ifnum#2>#1 %
          \else
            1%
          \fi
        \fi
      \fi
      \ltx@space
    \expandafter\ltx@secondoftwo
  \else
    \expandafter\ltx@firstoftwo
  \fi
}
%    \end{macrocode}
%    \end{macro}
%    \begin{macro}{\ltx@ifclasslater}
%    \begin{macrocode}
\def\ltx@ifclasslater#1{%
  \ltx@iffilelater{#1.\ltx@clsextension}%
}
%    \end{macrocode}
%    \end{macro}
%    \begin{macro}{\ltx@ifpackagelater}
%    \begin{macrocode}
\def\ltx@ifpackagelater#1{%
  \ltx@iffilelater{#1.\ltx@pkgextension}%
}
%    \end{macrocode}
%    \end{macro}
%
%    \begin{macrocode}
\ltx@IfUndefined{pdfmatch}{%
%    \end{macrocode}
%    \begin{macro}{\LTXcmds@ParseVersion}
%    \begin{macrocode}
  \def\LTXcmds@ParseVersion#1{%
    \LTXcmds@@ParseVersion#10000/00/00\@nil
  }%
%    \end{macrocode}
%    \end{macro}
%    \begin{macro}{\LTXcmds@@ParseVersion}
%    \begin{macrocode}
  \def\LTXcmds@@ParseVersion#1#2#3#4/#5#6/#7#8#9\@nil{%
    #1#2#3#4#5#6#7#8%
  }%
%    \end{macrocode}
%    \end{macro}
%    \begin{macrocode}
}{%
%    \end{macrocode}
%    \begin{macro}{\LTXcmds@ParseVersion}
%    \begin{macrocode}
  \def\LTXcmds@ParseVersion#1{%
    \ifnum\pdfmatch{%
      ^%
      (199[4-9]|[2-9][0-9][0-9][0-9])/%
      (0[1-9]|1[0-2])/%
      (0[1-9]|[1-2][0-9]|3[0-1])%
    }{#1}=1 %
      \ltx@StripPrefix\pdflastmatch1 %
      \ltx@StripPrefix\pdflastmatch2 %
      \ltx@StripPrefix\pdflastmatch3 %
    \else
      0%
    \fi
  }%
%    \end{macrocode}
%    \end{macro}
%    \begin{macrocode}
}
%    \end{macrocode}
%
% \subsection{Macro additions}
%
%    \begin{macro}{\ltx@GlobalAppendToMacro}
%    \begin{macrocode}
\long\def\ltx@GlobalAppendToMacro#1#2{%
  \ifx\ltx@undefined#1%
    \let#1\ltx@empty
  \else
    \ifx\relax#1%
      \let#1\ltx@empty
    \fi
  \fi
  \begingroup
    \ltx@LocToksA\expandafter{#1#2}%
    \xdef#1{\the\ltx@LocToksA}%
  \endgroup
}
%    \end{macrocode}
%    \end{macro}
%    \begin{macro}{\ltx@LocalAppendToMacro}
%    \begin{macrocode}
\long\def\ltx@LocalAppendToMacro#1#2{%
  \global\let\LTXcmds@gtemp#1%
  \ifx\ltx@undefined\LTXcmds@gtemp
    \global\let\LTXcmds@gtemp\ltx@empty
  \else
    \ifx\relax\LTXcmds@gtemp
      \global\letLTXcmds@gtemp\ltx@empty
    \fi
  \fi
  \begingroup
    \ltx@LocToksA\expandafter{\LTXcmds@gtemp#2}%
    \xdef\LTXcmds@gtemp{\the\ltx@LocToksA}%
  \endgroup
  \let#1\LTXcmds@gtemp
}
%    \end{macrocode}
%    \end{macro}
%
%    \begin{macro}{\ltx@GlobalPrependToMacro}
%    \begin{macrocode}
\long\def\ltx@GlobalPrependToMacro#1#2{%
  \ifx\ltx@undefined#1%
    \let#1\ltx@empty
  \else
    \ifx\relax#1%
      \let#1\ltx@empty
    \fi
  \fi
  \begingroup
    \ltx@LocToksA{#2}%
    \ltx@LocToksB\expandafter{#1}%
    \xdef#1{\the\ltx@LocToksA\the\ltx@LocToksB}%
  \endgroup
}
%    \end{macrocode}
%    \end{macro}
%    \begin{macro}{\ltx@LocalPrependToMacro}
%    \begin{macrocode}
\long\def\ltx@LocalPrependToMacro#1#2{%
  \global\let\LTXcmds@gtemp#1%
  \ifx\ltx@undefined\LTXcmds@gtemp
    \global\let\LTXcmds@gtemp\ltx@empty
  \else
    \ifx\relax\LTXcmds@gtemp
      \global\letLTXcmds@gtemp\ltx@empty
    \fi
  \fi
  \begingroup
    \ltx@LocToksA{#2}%
    \ltx@LocToksB\expandafter{\LTXcmds@gtemp}%
    \xdef\LTXcmds@gtemp{\the\ltx@LocToksA\the\ltx@LocToksB}%
  \endgroup
  \let#1\LTXcmds@gtemp
}
%    \end{macrocode}
%    \end{macro}
%
% \subsection{Next character detection}
%
%    \begin{macro}{\ltx@ifnextchar}
%    \begin{macrocode}
\long\def\ltx@ifnextchar#1#2#3{%
  \begingroup
  \let\LTXcmds@CharToken= #1\relax
  \ltx@LocToksA{\endgroup#2}%
  \ltx@LocToksB{\endgroup#3}%
  \futurelet\LTXcmds@LetToken\LTXcmds@ifnextchar
}
%    \end{macrocode}
%    \end{macro}
%    \begin{macro}{\LTXcmds@ifnextchar}
%    \begin{macrocode}
\def\LTXcmds@ifnextchar{%
  \ifx\LTXcmds@LetToken\LTXcmds@CharToken
    \the\expandafter\ltx@LocToksA
  \else
    \expandafter
      \ifx\csname LTXcmds@LetToken\endcsname\LTXcmds@SpaceToken
      \expandafter\expandafter\expandafter\LTXcmds@@ifnextchar
    \else
      \the\expandafter\expandafter\expandafter\ltx@LocToksB
    \fi
  \fi
}
%    \end{macrocode}
%    \end{macro}
%    \begin{macro}{\LTXcmds@@ifnextchar}
%    \cs{futurelet} does not distinguish between a
%    character and a command that is a character
%    (defined by using \cs{let} or \cs{futurelet}).
%    Therefore the space is catched by \cs{romannumeral}
%    with negative character constant that gobbles one
%    optional space.
%    \begin{macrocode}
\def\LTXcmds@@ifnextchar{%
  \expandafter\futurelet
  \expandafter\LTXcmds@LetToken
  \expandafter\LTXcmds@ifnextchar
  \romannumeral-`\.%
}
%    \end{macrocode}
%    \end{macro}
%    \begin{macro}{\LTXcmds@SpaceToken}
%    \begin{macrocode}
\ltx@firstofone{\let\LTXcmds@SpaceToken= } %
%    \end{macrocode}
%    \end{macro}
%
%    \begin{macro}{\ltx@ifnextchar@nospace}
%    \begin{macrocode}
\long\def\ltx@ifnextchar@nospace#1#2#3{%
  \begingroup
  \let\LTXcmds@CharToken= #1\relax
  \ltx@LocToksA{\endgroup#2}%
  \ltx@LocToksB{\endgroup#3}%
  \futurelet\LTXcmds@LetToken\LTXcmds@ifnextchar@nospace
}
%    \end{macrocode}
%    \end{macro}
%    \begin{macro}{\LTXcmds@ifnextchar@nospace}
%    \begin{macrocode}
\def\LTXcmds@ifnextchar@nospace{%
  \the
  \ifx\LTXcmds@LetToken\LTXcmds@CharToken
    \expandafter\ltx@LocToksA
  \else
    \expandafter\ltx@LocToksB
  \fi
}
%    \end{macrocode}
%    \end{macro}
% \subsection{\cs{ltx@leavevmode}, \cs{ltx@mbox}}
%
%    \begin{macro}{\ltx@leavevmode}
%    \begin{macrocode}
\ltx@IfUndefined{quitvmode}{%
  \ltx@IfUndefined{leavevmode}{%
    \ltx@IfUndefined{voidb@x}{%
      \ltx@IfUndefined{newbox}{%
        \def\ltx@leavevmode{%
          \begingroup
            \setbox\ltx@zero=\hbox{}%
            \begingroup
              \setbox\ltx@zero=\hbox{\box\ltx@zero}%
            \endgroup
            \unhbox\ltx@zero
          \endgroup
        }%
      }{%
        \csname newbox\endcsname\LTXcmds@VoidBox
        \ifvoid\LTXcmds@VoidBox
        \else
          \setbox\LTXcmds@VoidBox=\hbox{}%
          \begingroup
            \setbox\LTXcmds@VoidBox=\hbox{\box\LTXcmds@VoidBox}%
          \endgroup
        \fi
        \def\ltx@leavevmode{\unhbox\LTXcmds@VoidBox}%
      }%
    }{%
      \def\ltx@leavevmode{\unhbox\voidb@x}%
    }%
  }{%
    \let\ltx@leavevmode\leavevmode
  }%
}{%
  \let\ltx@leavevmode\quitvmode
}
%    \end{macrocode}
%    \end{macro}
%
%    \begin{macro}{\ltx@mbox}
%    \begin{macrocode}
\def\ltx@mbox{%
  \ltx@leavevmode
  \hbox
}
%    \end{macrocode}
%    \end{macro}
%
% \subsection{Help macros}
%
%    \begin{macro}{\LTXcmds@num}
%    \begin{macrocode}
\ltx@IfUndefined{numexpr}{%
  \def\LTXcmds@num#1{%
    \expandafter\ltx@firstofone\expandafter{%
      \number#1%
    }%
  }%
}{%
  \def\LTXcmds@num#1{%
    \expandafter\ltx@firstofone\expandafter{%
      \the\numexpr#1%
    }%
  }%
}
%    \end{macrocode}
%    \end{macro}
%
% \subsection{Expandable test for emptiness}
%
%    \begin{macrocode}
\ltx@IfUndefined{detokenize}{%
%    \end{macrocode}
%
% \subsubsection{Vanilla \hologo{TeX}}
%
%    \begin{macro}{\ltx@ifempty}
%    The macro is based on \cs{@ifempty} of Robert R. Schneck
%    \cite{ifempty:schneck}
%    and \cs{@ifnull} of Ulrich Diez \cite{ifnull:diez}.
%    There are three cases to consider:
%    \begin{compactenum}
%    \item |#1| is empty,
%    \item |#1| is not empty and the first token
%      is not a begingroup character,
%    \item |#1| starts with a begingroup character (catcode 1).
%    \end{compactenum}
%    \begin{macrocode}
  \def\LTXcmds@temp#1{%
    \long\def\ltx@ifempty##1{%
      \romannumeral0%
      \iffalse{\fi
        \expandafter\ltx@gobble\expandafter{%
          \expandafter{\string##1}%
          \expandafter\ltx@gobble\string
        }%
        \expandafter\ltx@firstofthree\expandafter
        {\iffalse}\fi
        \expandafter#1\ltx@secondoftwo
      }%
      \expandafter#1\ltx@firstoftwo
    }%
%    \end{macrocode}
%    \end{macro}
%    \begin{macro}{\ltx@ifblank}
%    \begin{macrocode}
    \long\def\ltx@ifblank##1{%
      \romannumeral0%
      \iffalse{\fi
        \expandafter\expandafter\expandafter\ltx@gobble
        \expandafter\expandafter\expandafter{%
          \expandafter\expandafter\expandafter{%
            \expandafter\string\ltx@gobble##1.%
          }%
          \expandafter\ltx@gobble\string
        }%
        \expandafter\ltx@firstofthree\expandafter
        {\iffalse}\fi
        \expandafter#1\ltx@secondoftwo
      }%
      \expandafter#1\ltx@firstoftwo
    }%
  }%
  \LTXcmds@temp{ }%
%    \end{macrocode}
%    \end{macro}
%    \begin{macrocode}
}{%
%    \end{macrocode}
%
% \subsubsection{With \cs{detokenize}}
%
%    Ahmed Musa provided \cs{ifstrempty} using \cs{detokenize}
%    and \cs{pdfstrcmp} \cite{ifstrempty:musa}. Ulrich Diez, GL,
%    Heiko Oberdiek
%    improved it further by removing \cs{pdfstrcmp} and
%    taking three arguments
%    \cite{ifstrempty:diez1,ifstrempty:diez2,ifstrempty:diez3,^^A
%      ifstrempty:gl,ifstrempty:ho}.
%
%    \begin{macro}{\ltx@ifempty}
%    \begin{macrocode}
  \long\def\ltx@ifempty#1{%
    \romannumeral%
    \csname
      LTXcmds@ifempty%
      \ifcat$\detokenize{#1}$%
        @%
      \fi
    \endcsname
  }%
%    \end{macrocode}
%    \end{macro}
%    \begin{macro}{\LTXcmds@ifempty@}
%    \begin{macrocode}
  \long\def\LTXcmds@ifempty@#1#2{0 #1}%
%    \end{macrocode}
%    \end{macro}
%    \begin{macro}{\LTXcmds@ifempty}
%    \begin{macrocode}
  \long\def\LTXcmds@ifempty#1#2{0 #2}%
%    \end{macrocode}
%    \end{macro}
%
% \subsubsection{\cs{ltx@ifblank}}
%
%    \begin{macro}{\ltx@ifblank}
%    \begin{macrocode}
  \long\def\ltx@ifblank#1{%
    \romannumeral%
    \csname
      LTXcmds@ifempty%
      \ifcat$\detokenize\expandafter{\ltx@gobble#1.}$%
        @%
      \fi
    \endcsname
  }%
%    \end{macrocode}
%    \end{macro}
%
%    \begin{macrocode}
}
%    \end{macrocode}
%
% \subsection{\cs{ltx@zapspace}}
%
%    \begin{macro}{\ltx@zapspace}
%    \begin{macrocode}
\long\def\ltx@zapspace#1{%
  \romannumeral
  \LTXcmds@zapspace\ltx@zero#1 \@nil
}
%    \end{macrocode}
%    \end{macro}
%    \begin{macro}{\LTXcmds@zapspace}
%    \begin{macrocode}
\long\def\LTXcmds@zapspace#1 #2\@nil{%
  \ltx@ifempty{#2}{%
    #1%
  }{%
    \LTXcmds@zapspace#1#2\@nil
  }%
}
%    \end{macrocode}
%    \end{macro}
%
% \subsection{\cs{ltx@IfBoxEmpty}}
%
%    In case of \hologo{eTeX} the test for an empty box
%    is done via \cs{lastnodetype} as suggested
%    by David Kastrup \cite{ifboxempty:dak}.
%    \begin{macrocode}
\ltx@IfUndefined{lastnodetype}{%
  \catcode`\$=9 %
  \catcode`\&=14 %
}{%
  \catcode`\$=14 %
  \catcode`\&=9 %
}
%    \end{macrocode}
%    \begin{macro}{\ltx@IfBoxEmpty}
%    \begin{macrocode}
\def\ltx@IfBoxEmpty#1{%
  \ifvoid#1\relax
    \expandafter\ltx@secondoftwo
  \else
%    \end{macrocode}
%    Implementation using \hologo{eTeX}'s \cs{lastnodetype}.
%    \begin{macrocode}
&   \begingroup
&     \setbox\ltx@zero=\ifhbox#1\hbox\else\vbox\fi{%
&       \ifhmode\unhcopy\else\unvcopy\fi#1\relax
&       \expandafter
&     }%
&   \expandafter\endgroup
&   \ifnum\lastnodetype<\ltx@zero
&     \expandafter\expandafter\expandafter\ltx@firstoftwo
&   \else
&     \expandafter\expandafter\expandafter\ltx@secondoftwo
&   \fi
%    \end{macrocode}
%    Implementation without \hologo{eTeX} using a signature
%    at the beginning of the test box.
%    \begin{macrocode}
$   \begingroup
$     \setbox\ltx@zero=\ifhbox#1\hbox\else\vbox\fi{%
$       \penalty\ltx@one
$       \ifhmode\unhcopy\else\unvcopy\fi#1\relax
$       \expandafter
$     }%
$     \ifnum\lastpenalty=\ltx@one
%    \end{macrocode}
%    Box 0 has been changed and is restored by
%    closing the group.
%    \begin{macrocode}
$       \endgroup
$       \begingroup
$       \setbox\ltx@zero=\ifhbox#1\hbox\else\vbox\fi{%
$         \penalty\ltx@two
$         \ifhmode\unhcopy\else\unvcopy\fi#1\relax
$         \expandafter
$       }%
$       \ifnum\lastpenalty=\ltx@two
$         \def\next{\endgroup\expandafter\ltx@firstoftwo}%
$       \else
$         \def\next{\endgroup\expandafter\ltx@secondoftwo}%
$       \fi
$     \else
$       \def\next{\endgroup\expandafter\ltx@secondoftwo}%
$     \fi
$   \next
  \fi
}
%    \end{macrocode}
%    \end{macro}
%
%    \begin{macro}{\ltx@IfBoxVoidOrEmpty}
%    \begin{macrocode}
\def\ltx@IfBoxVoidOrEmpty#1{%
  \ifvoid#1\relax
    \expandafter\ltx@thirdoffour
  \fi
  \ltx@IfBoxEmpty{#1}%
}
%    \end{macrocode}
%    \end{macro}
%
%    \begin{macrocode}
\LTXcmds@AtEnd%
%</package>
%    \end{macrocode}
%% \section{Installation}
%
% \subsection{Download}
%
% \paragraph{Package.} This package is available on
% CTAN\footnote{\CTANpkg{ltxcmds}}:
% \begin{description}
% \item[\CTAN{macros/latex/contrib/ltxcmds/ltxcmds.dtx}] The source file.
% \item[\CTAN{macros/latex/contrib/ltxcmds/ltxcmds.pdf}] Documentation.
% \end{description}
%
%
% \paragraph{Bundle.} All the packages of the bundle `ltxcmds'
% are also available in a TDS compliant ZIP archive. There
% the packages are already unpacked and the documentation files
% are generated. The files and directories obey the TDS standard.
% \begin{description}
% \item[\CTANinstall{install/macros/latex/contrib/ltxcmds.tds.zip}]
% \end{description}
% \emph{TDS} refers to the standard ``A Directory Structure
% for \TeX\ Files'' (\CTANpkg{tds}). Directories
% with \xfile{texmf} in their name are usually organized this way.
%
% \subsection{Bundle installation}
%
% \paragraph{Unpacking.} Unpack the \xfile{ltxcmds.tds.zip} in the
% TDS tree (also known as \xfile{texmf} tree) of your choice.
% Example (linux):
% \begin{quote}
%   |unzip ltxcmds.tds.zip -d ~/texmf|
% \end{quote}
%
% \subsection{Package installation}
%
% \paragraph{Unpacking.} The \xfile{.dtx} file is a self-extracting
% \docstrip\ archive. The files are extracted by running the
% \xfile{.dtx} through \plainTeX:
% \begin{quote}
%   \verb|tex ltxcmds.dtx|
% \end{quote}
%
% \paragraph{TDS.} Now the different files must be moved into
% the different directories in your installation TDS tree
% (also known as \xfile{texmf} tree):
% \begin{quote}
% \def\t{^^A
% \begin{tabular}{@{}>{\ttfamily}l@{ $\rightarrow$ }>{\ttfamily}l@{}}
%   ltxcmds.sty & tex/generic/ltxcmds/ltxcmds.sty\\
%   ltxcmds.pdf & doc/latex/ltxcmds/ltxcmds.pdf\\
%   ltxcmds.dtx & source/latex/ltxcmds/ltxcmds.dtx\\
% \end{tabular}^^A
% }^^A
% \sbox0{\t}^^A
% \ifdim\wd0>\linewidth
%   \begingroup
%     \advance\linewidth by\leftmargin
%     \advance\linewidth by\rightmargin
%   \edef\x{\endgroup
%     \def\noexpand\lw{\the\linewidth}^^A
%   }\x
%   \def\lwbox{^^A
%     \leavevmode
%     \hbox to \linewidth{^^A
%       \kern-\leftmargin\relax
%       \hss
%       \usebox0
%       \hss
%       \kern-\rightmargin\relax
%     }^^A
%   }^^A
%   \ifdim\wd0>\lw
%     \sbox0{\small\t}^^A
%     \ifdim\wd0>\linewidth
%       \ifdim\wd0>\lw
%         \sbox0{\footnotesize\t}^^A
%         \ifdim\wd0>\linewidth
%           \ifdim\wd0>\lw
%             \sbox0{\scriptsize\t}^^A
%             \ifdim\wd0>\linewidth
%               \ifdim\wd0>\lw
%                 \sbox0{\tiny\t}^^A
%                 \ifdim\wd0>\linewidth
%                   \lwbox
%                 \else
%                   \usebox0
%                 \fi
%               \else
%                 \lwbox
%               \fi
%             \else
%               \usebox0
%             \fi
%           \else
%             \lwbox
%           \fi
%         \else
%           \usebox0
%         \fi
%       \else
%         \lwbox
%       \fi
%     \else
%       \usebox0
%     \fi
%   \else
%     \lwbox
%   \fi
% \else
%   \usebox0
% \fi
% \end{quote}
% If you have a \xfile{docstrip.cfg} that configures and enables \docstrip's
% TDS installing feature, then some files can already be in the right
% place, see the documentation of \docstrip.
%
% \subsection{Refresh file name databases}
%
% If your \TeX~distribution
% (\TeX\,Live, \mikTeX, \dots) relies on file name databases, you must refresh
% these. For example, \TeX\,Live\ users run \verb|texhash| or
% \verb|mktexlsr|.
%
% \subsection{Some details for the interested}
%
% \paragraph{Unpacking with \LaTeX.}
% The \xfile{.dtx} chooses its action depending on the format:
% \begin{description}
% \item[\plainTeX:] Run \docstrip\ and extract the files.
% \item[\LaTeX:] Generate the documentation.
% \end{description}
% If you insist on using \LaTeX\ for \docstrip\ (really,
% \docstrip\ does not need \LaTeX), then inform the autodetect routine
% about your intention:
% \begin{quote}
%   \verb|latex \let\install=y\input{ltxcmds.dtx}|
% \end{quote}
% Do not forget to quote the argument according to the demands
% of your shell.
%
% \paragraph{Generating the documentation.}
% You can use both the \xfile{.dtx} or the \xfile{.drv} to generate
% the documentation. The process can be configured by the
% configuration file \xfile{ltxdoc.cfg}. For instance, put this
% line into this file, if you want to have A4 as paper format:
% \begin{quote}
%   \verb|\PassOptionsToClass{a4paper}{article}|
% \end{quote}
% An example follows how to generate the
% documentation with pdf\LaTeX:
% \begin{quote}
%\begin{verbatim}
%pdflatex ltxcmds.dtx
%makeindex -s gind.ist ltxcmds.idx
%pdflatex ltxcmds.dtx
%makeindex -s gind.ist ltxcmds.idx
%pdflatex ltxcmds.dtx
%\end{verbatim}
% \end{quote}
%
% \begin{thebibliography}{9}
% \raggedright
%
% \bibitem{ifempty:schneck}
% Robert R. Schneck:
% \textit{Re: \cs{ifempty} solution %
%   (was Macro puzzle: maximally general \cs{ifempty})};
% newsgroup \xnewsgroup{comp.text.tex},
% \url{news:3eef1ada_6@corp.newsgroups.com},
% 2003-06-17.\\
% {\small
% \url{https://groups.google.com/group/comp.text.tex/msg/be03a159ec374895}}
%
% \bibitem{ifnull:diez}
% Ulrich Diez:
% \textit{Re: TeX refuses to strip outer braces in argument};
% newsgroup \xnewsgroup{comp.text.tex},
% \url{news:ibk3t8$ee7$1@news.albasani.net},
% 2010-11-12.\\
% {\small
% \url{https://groups.google.com/group/comp.text.tex/msg/803bd57221a04996}}
%
% \bibitem{ifstrempty:musa}
% Ahmed Musa:
% \textit{Re: TeX refuses to strip outer braces in argument};
% newsgroup \xnewsgroup{comp.text.tex},
% \url{news:f5496afe-40ed-42bd-b629-a2419ecf7c0d@o14g2000prn.googlegroups.com},
% 2010-12-03.\\
% {\small
% \url{https://groups.google.com/group/comp.text.tex/msg/fbf7d61a0c3a807d}}
%
% \bibitem{ifstrempty:diez1}
% Ulrich Diez:
% \textit{Re: TeX refuses to strip outer braces in argument};
% newsgroup \xnewsgroup{comp.text.tex},
% \url{news:idbo94$uka$1@four.albasani.net},
% 2010-12-03.\\
% {\small
% \url{https://groups.google.com/group/comp.text.tex/msg/0c230ee479487962}}
%
% \bibitem{ifstrempty:diez2}
% Ulrich Diez:
% \textit{Re: TeX refuses to strip outer braces in argument};
% newsgroup \xnewsgroup{comp.text.tex},
% \url{news:idbpu4$cg1$1@news.albasani.net},
% 2010-12-03.\\
% {\small
% \url{https://groups.google.com/group/comp.text.tex/msg/bbef4263390d647b}}
%
% \bibitem{ifstrempty:diez3}
% Ulrich Diez:
% \textit{Re: TeX refuses to strip outer braces in argument};
% newsgroup \xnewsgroup{comp.text.tex},
% \url{news:idd4ga$r83$1@four.albasani.net},
% 2010-12-04.\\
% {\small
% \url{https://groups.google.com/group/comp.text.tex/msg/00dfd1ec103cd272}}
%
% \bibitem{ifstrempty:gl}
% GL:
% \textit{Re: TeX refuses to strip outer braces in argument};
% newsgroup \xnewsgroup{comp.text.tex},
% \url{news:4cfa2e27$0$7389$426a74cc@news.free.fr},
% 2010-12-04.\\
% {\small
% \url{https://groups.google.com/group/comp.text.tex/msg/d3a75995c1cf267e}}
%
% \bibitem{ifstrempty:ho}
% Heiko Oberdiek:
% \textit{Re: TeX refuses to strip outer braces in argument};
% newsgroup \xnewsgroup{comp.text.tex},
% \url{news:iddhq1$3kj$1@news.eternal-september.org},
% 2010-12-04.\\
% {\small
% \url{https://groups.google.com/group/comp.text.tex/msg/5f7a23e3ab70e347}}
%
% \bibitem{ifboxempty:dak}
% David Kastrup:
% \textit{How to detect if \cs{vbox} is empty};
% newsgroup \xnewsgroup{comp.text.tex},
% 2011-02-04.\\
% {\small
% \url{https://groups.google.com/group/comp.text.tex/msg/8d3cb89496a4d86d}}
%
% \end{thebibliography}
%
% \begin{History}
%   \begin{Version}{2009/08/05 v1.0}
%   \item
%     First version.
%   \end{Version}
%   \begin{Version}{2009/12/12 v1.1}
%   \item
%     Short title shortened.
%   \item
%     \cs{ltx@IfUndefined} added.
%   \end{Version}
%   \begin{Version}{2010/01/28 v1.2}
%   \item
%     \cs{ltx@RemovePrefix} and \cs{ltx@StripPrefix} added.
%   \item
%     \cs{ltx@ifclassloaded}, \cs{ltx@ifpackageloaded},
%     \cs{ltx@iffileloaded},
%     \cs{ltx@ifclasslater}, \cs{ltx@ifpackagelater},
%     \cs{ltx@iffilelater},
%     \cs{ltx@clsextension}, \cs{ltx@pkgextension} added.
%   \item
%     \cs{ltx@GlobalAppendToMacro}, \cs{ltx@LocalAppendToMacro} added.
%   \end{Version}
%   \begin{Version}{2010/03/01 v1.3}
%   \item
%     \cs{ltx@newif} added.
%   \item
%     \cs{ltx@ifnextchar} added.
%   \item
%     Numbers \cs{ltx@zero}, \cs{ltx@one}, \cs{ltx@two}, \cs{ltx@cclv} added.
%   \end{Version}
%   \begin{Version}{2010/03/09 v1.4}
%   \item
%     \cs{ltx@pkgextension} and \cs{ltx@clsextension} are hardcoded to
%     avoid trouble with \cs{@onlypreamble}.
%   \end{Version}
%   \begin{Version}{2010/04/08 v1.5}
%   \item
%     \cs{ltx@cartwo}, \cs{ltx@cdrtwo},
%     \cs{ltx@carthree}, \cs{ltx@cdrthree},
%     \cs{ltx@carfour}, \cs{ltx@cdrfour} added.
%   \item
%     \cs{ltx@ReturnAfterFi} and \cs{ltx@ReturnAfterElseFi} fixed.
%   \end{Version}
%   \begin{Version}{2010/04/16 v1.6}
%   \item
%     \cs{ltx@leavevmode}, \cs{ltx@mbox} added.
%   \end{Version}
%   \begin{Version}{2010/04/26 v1.7}
%   \item
%     \cs{ltx@GobbleNum}, \cs{ltx@CdrNum}, \cs{ltx@CarNum} added.
%   \item
%     \cs{ltx@carzero}, \cs{ltx@cdrzero} added.
%   \item
%     \cs{ltx@hashchar} added.
%   \end{Version}
%   \begin{Version}{2010/09/11 v1.8}
%   \item
%     \cs{ltx@leftbracechar}, \cs{ltx@rightbracechar} added.
%   \end{Version}
%   \begin{Version}{2010/10/25 v1.9}
%   \item
%     \cs{ltx@LocalAppendToMacro} and \cs{ltx@GlobalAppendToMacro} are
%     now \cs{long}.
%   \end{Version}
%   \begin{Version}{2010/10/31 v1.10}
%   \item
%     \cs{ltx@newglobalif} added.
%   \end{Version}
%   \begin{Version}{2010/11/12 v1.11}
%   \item
%     \cs{ltx@ifempty} added.
%   \item
%     \cs{ltx@firstofthree}, \cs{ltx@secondofthree}, \cs{ltx@thirdofthree}
%     added.
%   \end{Version}
%   \begin{Version}{2010/12/02 v1.12}
%   \item
%     \cs{ltx@onelevel@sanitize} added.
%   \item
%     \cs{LTXcmds@num} fixed for the case with \cs{numexpr}
%     (bug found by GL).
%   \end{Version}
%   \begin{Version}{2010/12/04 v1.13}
%   \item
%     \cs{ltx@ifblank} added.
%   \item
%     Optimization for \cs{ltx@ifempty}.
%   \end{Version}
%   \begin{Version}{2010/12/07 v1.14}
%   \item
%     \cs{ltx@zapspace} added.
%   \end{Version}
%   \begin{Version}{2010/12/12 v1.15}
%   \item
%     \cs{ltx@minusone} added.
%   \end{Version}
%   \begin{Version}{2011/02/04 v1.16}
%   \item
%     \cs{ltx@IfBoxEmpty} and \cs{ltx@IfBoxVoidOrEmpty} added.
%   \item
%     \cs{ltx@firstoffour}, \dots, \cs{ltx@fourthoffour} added.
%   \end{Version}
%   \begin{Version}{2011/02/05 v1.17}
%   \item
%     \cs{ltx@IfBoxEmpty}: an empty box may have non-zero dimensions.
%   \end{Version}
%   \begin{Version}{2011/03/16 v1.18}
%   \item
%     \cs{ltx@ifclasslater} fixed.
%   \end{Version}
%   \begin{Version}{2011/04/14 v1.19}
%   \item
%     \cs{ltx@ifnextchar}: detection of optional spaces modified.
%   \item
%     \cs{ltx(Loc,Glob)(Toks,Dimen,Skip)(A,B,C,D,E)} added.
%   \end{Version}
%   \begin{Version}{2011/04/18 v1.20}
%   \item
%     \cs{ltx@ifnextchar} with conditional support (thanks GL for
%     bug report).
%   \end{Version}
%   \begin{Version}{2011/08/22 v1.21}
%   \item
%     \cs{ltx@GlobalPrependToMacro}, \cs{ltx@LocalPrependToMacro} added
%     (feature request of Martin M\"unch).
%   \end{Version}
%   \begin{Version}{2011/11/09 v1.22}
%   \item
%     \cs{ltx@carsecond}, \cs{ltx@carthird}, \cs{ltx@carfourth},
%     \cs{ltx@CarNumth} added.
%   \item
%     \cs{ltx@cdrzero}, \cs{ltx@cdr}, \cs{ltx@cdrtwo}, cs{ltx@cdrthree},
%     \cs{ltx@cdrfour}, \cs{ltx@CdrNum} modified to retain braces and spaces.
%     They are expandable in two expansion steps.
%   \end{Version}
%   \begin{Version}{2016/05/16 v1.23}
%   \item
%     Documentation updates.
%   \end{Version}
%   \begin{Version}{2019/12/15 v1.24}
%   \item
%     Documentation updates.
%   \end{Version}
% \end{History}
%
% \PrintIndex
%
% \Finale
\endinput
|
% \end{quote}
% Do not forget to quote the argument according to the demands
% of your shell.
%
% \paragraph{Generating the documentation.}
% You can use both the \xfile{.dtx} or the \xfile{.drv} to generate
% the documentation. The process can be configured by the
% configuration file \xfile{ltxdoc.cfg}. For instance, put this
% line into this file, if you want to have A4 as paper format:
% \begin{quote}
%   \verb|\PassOptionsToClass{a4paper}{article}|
% \end{quote}
% An example follows how to generate the
% documentation with pdf\LaTeX:
% \begin{quote}
%\begin{verbatim}
%pdflatex ltxcmds.dtx
%makeindex -s gind.ist ltxcmds.idx
%pdflatex ltxcmds.dtx
%makeindex -s gind.ist ltxcmds.idx
%pdflatex ltxcmds.dtx
%\end{verbatim}
% \end{quote}
%
% \begin{thebibliography}{9}
% \raggedright
%
% \bibitem{ifempty:schneck}
% Robert R. Schneck:
% \textit{Re: \cs{ifempty} solution %
%   (was Macro puzzle: maximally general \cs{ifempty})};
% newsgroup \xnewsgroup{comp.text.tex},
% \url{news:3eef1ada_6@corp.newsgroups.com},
% 2003-06-17.\\
% {\small
% \url{https://groups.google.com/group/comp.text.tex/msg/be03a159ec374895}}
%
% \bibitem{ifnull:diez}
% Ulrich Diez:
% \textit{Re: TeX refuses to strip outer braces in argument};
% newsgroup \xnewsgroup{comp.text.tex},
% \url{news:ibk3t8$ee7$1@news.albasani.net},
% 2010-11-12.\\
% {\small
% \url{https://groups.google.com/group/comp.text.tex/msg/803bd57221a04996}}
%
% \bibitem{ifstrempty:musa}
% Ahmed Musa:
% \textit{Re: TeX refuses to strip outer braces in argument};
% newsgroup \xnewsgroup{comp.text.tex},
% \url{news:f5496afe-40ed-42bd-b629-a2419ecf7c0d@o14g2000prn.googlegroups.com},
% 2010-12-03.\\
% {\small
% \url{https://groups.google.com/group/comp.text.tex/msg/fbf7d61a0c3a807d}}
%
% \bibitem{ifstrempty:diez1}
% Ulrich Diez:
% \textit{Re: TeX refuses to strip outer braces in argument};
% newsgroup \xnewsgroup{comp.text.tex},
% \url{news:idbo94$uka$1@four.albasani.net},
% 2010-12-03.\\
% {\small
% \url{https://groups.google.com/group/comp.text.tex/msg/0c230ee479487962}}
%
% \bibitem{ifstrempty:diez2}
% Ulrich Diez:
% \textit{Re: TeX refuses to strip outer braces in argument};
% newsgroup \xnewsgroup{comp.text.tex},
% \url{news:idbpu4$cg1$1@news.albasani.net},
% 2010-12-03.\\
% {\small
% \url{https://groups.google.com/group/comp.text.tex/msg/bbef4263390d647b}}
%
% \bibitem{ifstrempty:diez3}
% Ulrich Diez:
% \textit{Re: TeX refuses to strip outer braces in argument};
% newsgroup \xnewsgroup{comp.text.tex},
% \url{news:idd4ga$r83$1@four.albasani.net},
% 2010-12-04.\\
% {\small
% \url{https://groups.google.com/group/comp.text.tex/msg/00dfd1ec103cd272}}
%
% \bibitem{ifstrempty:gl}
% GL:
% \textit{Re: TeX refuses to strip outer braces in argument};
% newsgroup \xnewsgroup{comp.text.tex},
% \url{news:4cfa2e27$0$7389$426a74cc@news.free.fr},
% 2010-12-04.\\
% {\small
% \url{https://groups.google.com/group/comp.text.tex/msg/d3a75995c1cf267e}}
%
% \bibitem{ifstrempty:ho}
% Heiko Oberdiek:
% \textit{Re: TeX refuses to strip outer braces in argument};
% newsgroup \xnewsgroup{comp.text.tex},
% \url{news:iddhq1$3kj$1@news.eternal-september.org},
% 2010-12-04.\\
% {\small
% \url{https://groups.google.com/group/comp.text.tex/msg/5f7a23e3ab70e347}}
%
% \bibitem{ifboxempty:dak}
% David Kastrup:
% \textit{How to detect if \cs{vbox} is empty};
% newsgroup \xnewsgroup{comp.text.tex},
% 2011-02-04.\\
% {\small
% \url{https://groups.google.com/group/comp.text.tex/msg/8d3cb89496a4d86d}}
%
% \end{thebibliography}
%
% \begin{History}
%   \begin{Version}{2009/08/05 v1.0}
%   \item
%     First version.
%   \end{Version}
%   \begin{Version}{2009/12/12 v1.1}
%   \item
%     Short title shortened.
%   \item
%     \cs{ltx@IfUndefined} added.
%   \end{Version}
%   \begin{Version}{2010/01/28 v1.2}
%   \item
%     \cs{ltx@RemovePrefix} and \cs{ltx@StripPrefix} added.
%   \item
%     \cs{ltx@ifclassloaded}, \cs{ltx@ifpackageloaded},
%     \cs{ltx@iffileloaded},
%     \cs{ltx@ifclasslater}, \cs{ltx@ifpackagelater},
%     \cs{ltx@iffilelater},
%     \cs{ltx@clsextension}, \cs{ltx@pkgextension} added.
%   \item
%     \cs{ltx@GlobalAppendToMacro}, \cs{ltx@LocalAppendToMacro} added.
%   \end{Version}
%   \begin{Version}{2010/03/01 v1.3}
%   \item
%     \cs{ltx@newif} added.
%   \item
%     \cs{ltx@ifnextchar} added.
%   \item
%     Numbers \cs{ltx@zero}, \cs{ltx@one}, \cs{ltx@two}, \cs{ltx@cclv} added.
%   \end{Version}
%   \begin{Version}{2010/03/09 v1.4}
%   \item
%     \cs{ltx@pkgextension} and \cs{ltx@clsextension} are hardcoded to
%     avoid trouble with \cs{@onlypreamble}.
%   \end{Version}
%   \begin{Version}{2010/04/08 v1.5}
%   \item
%     \cs{ltx@cartwo}, \cs{ltx@cdrtwo},
%     \cs{ltx@carthree}, \cs{ltx@cdrthree},
%     \cs{ltx@carfour}, \cs{ltx@cdrfour} added.
%   \item
%     \cs{ltx@ReturnAfterFi} and \cs{ltx@ReturnAfterElseFi} fixed.
%   \end{Version}
%   \begin{Version}{2010/04/16 v1.6}
%   \item
%     \cs{ltx@leavevmode}, \cs{ltx@mbox} added.
%   \end{Version}
%   \begin{Version}{2010/04/26 v1.7}
%   \item
%     \cs{ltx@GobbleNum}, \cs{ltx@CdrNum}, \cs{ltx@CarNum} added.
%   \item
%     \cs{ltx@carzero}, \cs{ltx@cdrzero} added.
%   \item
%     \cs{ltx@hashchar} added.
%   \end{Version}
%   \begin{Version}{2010/09/11 v1.8}
%   \item
%     \cs{ltx@leftbracechar}, \cs{ltx@rightbracechar} added.
%   \end{Version}
%   \begin{Version}{2010/10/25 v1.9}
%   \item
%     \cs{ltx@LocalAppendToMacro} and \cs{ltx@GlobalAppendToMacro} are
%     now \cs{long}.
%   \end{Version}
%   \begin{Version}{2010/10/31 v1.10}
%   \item
%     \cs{ltx@newglobalif} added.
%   \end{Version}
%   \begin{Version}{2010/11/12 v1.11}
%   \item
%     \cs{ltx@ifempty} added.
%   \item
%     \cs{ltx@firstofthree}, \cs{ltx@secondofthree}, \cs{ltx@thirdofthree}
%     added.
%   \end{Version}
%   \begin{Version}{2010/12/02 v1.12}
%   \item
%     \cs{ltx@onelevel@sanitize} added.
%   \item
%     \cs{LTXcmds@num} fixed for the case with \cs{numexpr}
%     (bug found by GL).
%   \end{Version}
%   \begin{Version}{2010/12/04 v1.13}
%   \item
%     \cs{ltx@ifblank} added.
%   \item
%     Optimization for \cs{ltx@ifempty}.
%   \end{Version}
%   \begin{Version}{2010/12/07 v1.14}
%   \item
%     \cs{ltx@zapspace} added.
%   \end{Version}
%   \begin{Version}{2010/12/12 v1.15}
%   \item
%     \cs{ltx@minusone} added.
%   \end{Version}
%   \begin{Version}{2011/02/04 v1.16}
%   \item
%     \cs{ltx@IfBoxEmpty} and \cs{ltx@IfBoxVoidOrEmpty} added.
%   \item
%     \cs{ltx@firstoffour}, \dots, \cs{ltx@fourthoffour} added.
%   \end{Version}
%   \begin{Version}{2011/02/05 v1.17}
%   \item
%     \cs{ltx@IfBoxEmpty}: an empty box may have non-zero dimensions.
%   \end{Version}
%   \begin{Version}{2011/03/16 v1.18}
%   \item
%     \cs{ltx@ifclasslater} fixed.
%   \end{Version}
%   \begin{Version}{2011/04/14 v1.19}
%   \item
%     \cs{ltx@ifnextchar}: detection of optional spaces modified.
%   \item
%     \cs{ltx(Loc,Glob)(Toks,Dimen,Skip)(A,B,C,D,E)} added.
%   \end{Version}
%   \begin{Version}{2011/04/18 v1.20}
%   \item
%     \cs{ltx@ifnextchar} with conditional support (thanks GL for
%     bug report).
%   \end{Version}
%   \begin{Version}{2011/08/22 v1.21}
%   \item
%     \cs{ltx@GlobalPrependToMacro}, \cs{ltx@LocalPrependToMacro} added
%     (feature request of Martin M\"unch).
%   \end{Version}
%   \begin{Version}{2011/11/09 v1.22}
%   \item
%     \cs{ltx@carsecond}, \cs{ltx@carthird}, \cs{ltx@carfourth},
%     \cs{ltx@CarNumth} added.
%   \item
%     \cs{ltx@cdrzero}, \cs{ltx@cdr}, \cs{ltx@cdrtwo}, cs{ltx@cdrthree},
%     \cs{ltx@cdrfour}, \cs{ltx@CdrNum} modified to retain braces and spaces.
%     They are expandable in two expansion steps.
%   \end{Version}
%   \begin{Version}{2016/05/16 v1.23}
%   \item
%     Documentation updates.
%   \end{Version}
%   \begin{Version}{2019/12/15 v1.24}
%   \item
%     Documentation updates.
%   \end{Version}
% \end{History}
%
% \PrintIndex
%
% \Finale
\endinput
|
% \end{quote}
% Do not forget to quote the argument according to the demands
% of your shell.
%
% \paragraph{Generating the documentation.}
% You can use both the \xfile{.dtx} or the \xfile{.drv} to generate
% the documentation. The process can be configured by the
% configuration file \xfile{ltxdoc.cfg}. For instance, put this
% line into this file, if you want to have A4 as paper format:
% \begin{quote}
%   \verb|\PassOptionsToClass{a4paper}{article}|
% \end{quote}
% An example follows how to generate the
% documentation with pdf\LaTeX:
% \begin{quote}
%\begin{verbatim}
%pdflatex ltxcmds.dtx
%makeindex -s gind.ist ltxcmds.idx
%pdflatex ltxcmds.dtx
%makeindex -s gind.ist ltxcmds.idx
%pdflatex ltxcmds.dtx
%\end{verbatim}
% \end{quote}
%
% \begin{thebibliography}{9}
% \raggedright
%
% \bibitem{ifempty:schneck}
% Robert R. Schneck:
% \textit{Re: \cs{ifempty} solution %
%   (was Macro puzzle: maximally general \cs{ifempty})};
% newsgroup \xnewsgroup{comp.text.tex},
% \url{news:3eef1ada_6@corp.newsgroups.com},
% 2003-06-17.\\
% {\small
% \url{https://groups.google.com/group/comp.text.tex/msg/be03a159ec374895}}
%
% \bibitem{ifnull:diez}
% Ulrich Diez:
% \textit{Re: TeX refuses to strip outer braces in argument};
% newsgroup \xnewsgroup{comp.text.tex},
% \url{news:ibk3t8$ee7$1@news.albasani.net},
% 2010-11-12.\\
% {\small
% \url{https://groups.google.com/group/comp.text.tex/msg/803bd57221a04996}}
%
% \bibitem{ifstrempty:musa}
% Ahmed Musa:
% \textit{Re: TeX refuses to strip outer braces in argument};
% newsgroup \xnewsgroup{comp.text.tex},
% \url{news:f5496afe-40ed-42bd-b629-a2419ecf7c0d@o14g2000prn.googlegroups.com},
% 2010-12-03.\\
% {\small
% \url{https://groups.google.com/group/comp.text.tex/msg/fbf7d61a0c3a807d}}
%
% \bibitem{ifstrempty:diez1}
% Ulrich Diez:
% \textit{Re: TeX refuses to strip outer braces in argument};
% newsgroup \xnewsgroup{comp.text.tex},
% \url{news:idbo94$uka$1@four.albasani.net},
% 2010-12-03.\\
% {\small
% \url{https://groups.google.com/group/comp.text.tex/msg/0c230ee479487962}}
%
% \bibitem{ifstrempty:diez2}
% Ulrich Diez:
% \textit{Re: TeX refuses to strip outer braces in argument};
% newsgroup \xnewsgroup{comp.text.tex},
% \url{news:idbpu4$cg1$1@news.albasani.net},
% 2010-12-03.\\
% {\small
% \url{https://groups.google.com/group/comp.text.tex/msg/bbef4263390d647b}}
%
% \bibitem{ifstrempty:diez3}
% Ulrich Diez:
% \textit{Re: TeX refuses to strip outer braces in argument};
% newsgroup \xnewsgroup{comp.text.tex},
% \url{news:idd4ga$r83$1@four.albasani.net},
% 2010-12-04.\\
% {\small
% \url{https://groups.google.com/group/comp.text.tex/msg/00dfd1ec103cd272}}
%
% \bibitem{ifstrempty:gl}
% GL:
% \textit{Re: TeX refuses to strip outer braces in argument};
% newsgroup \xnewsgroup{comp.text.tex},
% \url{news:4cfa2e27$0$7389$426a74cc@news.free.fr},
% 2010-12-04.\\
% {\small
% \url{https://groups.google.com/group/comp.text.tex/msg/d3a75995c1cf267e}}
%
% \bibitem{ifstrempty:ho}
% Heiko Oberdiek:
% \textit{Re: TeX refuses to strip outer braces in argument};
% newsgroup \xnewsgroup{comp.text.tex},
% \url{news:iddhq1$3kj$1@news.eternal-september.org},
% 2010-12-04.\\
% {\small
% \url{https://groups.google.com/group/comp.text.tex/msg/5f7a23e3ab70e347}}
%
% \bibitem{ifboxempty:dak}
% David Kastrup:
% \textit{How to detect if \cs{vbox} is empty};
% newsgroup \xnewsgroup{comp.text.tex},
% 2011-02-04.\\
% {\small
% \url{https://groups.google.com/group/comp.text.tex/msg/8d3cb89496a4d86d}}
%
% \end{thebibliography}
%
% \begin{History}
%   \begin{Version}{2009/08/05 v1.0}
%   \item
%     First version.
%   \end{Version}
%   \begin{Version}{2009/12/12 v1.1}
%   \item
%     Short title shortened.
%   \item
%     \cs{ltx@IfUndefined} added.
%   \end{Version}
%   \begin{Version}{2010/01/28 v1.2}
%   \item
%     \cs{ltx@RemovePrefix} and \cs{ltx@StripPrefix} added.
%   \item
%     \cs{ltx@ifclassloaded}, \cs{ltx@ifpackageloaded},
%     \cs{ltx@iffileloaded},
%     \cs{ltx@ifclasslater}, \cs{ltx@ifpackagelater},
%     \cs{ltx@iffilelater},
%     \cs{ltx@clsextension}, \cs{ltx@pkgextension} added.
%   \item
%     \cs{ltx@GlobalAppendToMacro}, \cs{ltx@LocalAppendToMacro} added.
%   \end{Version}
%   \begin{Version}{2010/03/01 v1.3}
%   \item
%     \cs{ltx@newif} added.
%   \item
%     \cs{ltx@ifnextchar} added.
%   \item
%     Numbers \cs{ltx@zero}, \cs{ltx@one}, \cs{ltx@two}, \cs{ltx@cclv} added.
%   \end{Version}
%   \begin{Version}{2010/03/09 v1.4}
%   \item
%     \cs{ltx@pkgextension} and \cs{ltx@clsextension} are hardcoded to
%     avoid trouble with \cs{@onlypreamble}.
%   \end{Version}
%   \begin{Version}{2010/04/08 v1.5}
%   \item
%     \cs{ltx@cartwo}, \cs{ltx@cdrtwo},
%     \cs{ltx@carthree}, \cs{ltx@cdrthree},
%     \cs{ltx@carfour}, \cs{ltx@cdrfour} added.
%   \item
%     \cs{ltx@ReturnAfterFi} and \cs{ltx@ReturnAfterElseFi} fixed.
%   \end{Version}
%   \begin{Version}{2010/04/16 v1.6}
%   \item
%     \cs{ltx@leavevmode}, \cs{ltx@mbox} added.
%   \end{Version}
%   \begin{Version}{2010/04/26 v1.7}
%   \item
%     \cs{ltx@GobbleNum}, \cs{ltx@CdrNum}, \cs{ltx@CarNum} added.
%   \item
%     \cs{ltx@carzero}, \cs{ltx@cdrzero} added.
%   \item
%     \cs{ltx@hashchar} added.
%   \end{Version}
%   \begin{Version}{2010/09/11 v1.8}
%   \item
%     \cs{ltx@leftbracechar}, \cs{ltx@rightbracechar} added.
%   \end{Version}
%   \begin{Version}{2010/10/25 v1.9}
%   \item
%     \cs{ltx@LocalAppendToMacro} and \cs{ltx@GlobalAppendToMacro} are
%     now \cs{long}.
%   \end{Version}
%   \begin{Version}{2010/10/31 v1.10}
%   \item
%     \cs{ltx@newglobalif} added.
%   \end{Version}
%   \begin{Version}{2010/11/12 v1.11}
%   \item
%     \cs{ltx@ifempty} added.
%   \item
%     \cs{ltx@firstofthree}, \cs{ltx@secondofthree}, \cs{ltx@thirdofthree}
%     added.
%   \end{Version}
%   \begin{Version}{2010/12/02 v1.12}
%   \item
%     \cs{ltx@onelevel@sanitize} added.
%   \item
%     \cs{LTXcmds@num} fixed for the case with \cs{numexpr}
%     (bug found by GL).
%   \end{Version}
%   \begin{Version}{2010/12/04 v1.13}
%   \item
%     \cs{ltx@ifblank} added.
%   \item
%     Optimization for \cs{ltx@ifempty}.
%   \end{Version}
%   \begin{Version}{2010/12/07 v1.14}
%   \item
%     \cs{ltx@zapspace} added.
%   \end{Version}
%   \begin{Version}{2010/12/12 v1.15}
%   \item
%     \cs{ltx@minusone} added.
%   \end{Version}
%   \begin{Version}{2011/02/04 v1.16}
%   \item
%     \cs{ltx@IfBoxEmpty} and \cs{ltx@IfBoxVoidOrEmpty} added.
%   \item
%     \cs{ltx@firstoffour}, \dots, \cs{ltx@fourthoffour} added.
%   \end{Version}
%   \begin{Version}{2011/02/05 v1.17}
%   \item
%     \cs{ltx@IfBoxEmpty}: an empty box may have non-zero dimensions.
%   \end{Version}
%   \begin{Version}{2011/03/16 v1.18}
%   \item
%     \cs{ltx@ifclasslater} fixed.
%   \end{Version}
%   \begin{Version}{2011/04/14 v1.19}
%   \item
%     \cs{ltx@ifnextchar}: detection of optional spaces modified.
%   \item
%     \cs{ltx(Loc,Glob)(Toks,Dimen,Skip)(A,B,C,D,E)} added.
%   \end{Version}
%   \begin{Version}{2011/04/18 v1.20}
%   \item
%     \cs{ltx@ifnextchar} with conditional support (thanks GL for
%     bug report).
%   \end{Version}
%   \begin{Version}{2011/08/22 v1.21}
%   \item
%     \cs{ltx@GlobalPrependToMacro}, \cs{ltx@LocalPrependToMacro} added
%     (feature request of Martin M\"unch).
%   \end{Version}
%   \begin{Version}{2011/11/09 v1.22}
%   \item
%     \cs{ltx@carsecond}, \cs{ltx@carthird}, \cs{ltx@carfourth},
%     \cs{ltx@CarNumth} added.
%   \item
%     \cs{ltx@cdrzero}, \cs{ltx@cdr}, \cs{ltx@cdrtwo}, cs{ltx@cdrthree},
%     \cs{ltx@cdrfour}, \cs{ltx@CdrNum} modified to retain braces and spaces.
%     They are expandable in two expansion steps.
%   \end{Version}
%   \begin{Version}{2016/05/16 v1.23}
%   \item
%     Documentation updates.
%   \end{Version}
%   \begin{Version}{2019/12/15 v1.24}
%   \item
%     Documentation updates.
%   \end{Version}
% \end{History}
%
% \PrintIndex
%
% \Finale
\endinput
|
% \end{quote}
% Do not forget to quote the argument according to the demands
% of your shell.
%
% \paragraph{Generating the documentation.}
% You can use both the \xfile{.dtx} or the \xfile{.drv} to generate
% the documentation. The process can be configured by the
% configuration file \xfile{ltxdoc.cfg}. For instance, put this
% line into this file, if you want to have A4 as paper format:
% \begin{quote}
%   \verb|\PassOptionsToClass{a4paper}{article}|
% \end{quote}
% An example follows how to generate the
% documentation with pdf\LaTeX:
% \begin{quote}
%\begin{verbatim}
%pdflatex ltxcmds.dtx
%makeindex -s gind.ist ltxcmds.idx
%pdflatex ltxcmds.dtx
%makeindex -s gind.ist ltxcmds.idx
%pdflatex ltxcmds.dtx
%\end{verbatim}
% \end{quote}
%
% \begin{thebibliography}{9}
% \raggedright
%
% \bibitem{ifempty:schneck}
% Robert R. Schneck:
% \textit{Re: \cs{ifempty} solution %
%   (was Macro puzzle: maximally general \cs{ifempty})};
% newsgroup \xnewsgroup{comp.text.tex},
% \url{news:3eef1ada_6@corp.newsgroups.com},
% 2003-06-17.\\
% {\small
% \url{https://groups.google.com/group/comp.text.tex/msg/be03a159ec374895}}
%
% \bibitem{ifnull:diez}
% Ulrich Diez:
% \textit{Re: TeX refuses to strip outer braces in argument};
% newsgroup \xnewsgroup{comp.text.tex},
% \url{news:ibk3t8$ee7$1@news.albasani.net},
% 2010-11-12.\\
% {\small
% \url{https://groups.google.com/group/comp.text.tex/msg/803bd57221a04996}}
%
% \bibitem{ifstrempty:musa}
% Ahmed Musa:
% \textit{Re: TeX refuses to strip outer braces in argument};
% newsgroup \xnewsgroup{comp.text.tex},
% \url{news:f5496afe-40ed-42bd-b629-a2419ecf7c0d@o14g2000prn.googlegroups.com},
% 2010-12-03.\\
% {\small
% \url{https://groups.google.com/group/comp.text.tex/msg/fbf7d61a0c3a807d}}
%
% \bibitem{ifstrempty:diez1}
% Ulrich Diez:
% \textit{Re: TeX refuses to strip outer braces in argument};
% newsgroup \xnewsgroup{comp.text.tex},
% \url{news:idbo94$uka$1@four.albasani.net},
% 2010-12-03.\\
% {\small
% \url{https://groups.google.com/group/comp.text.tex/msg/0c230ee479487962}}
%
% \bibitem{ifstrempty:diez2}
% Ulrich Diez:
% \textit{Re: TeX refuses to strip outer braces in argument};
% newsgroup \xnewsgroup{comp.text.tex},
% \url{news:idbpu4$cg1$1@news.albasani.net},
% 2010-12-03.\\
% {\small
% \url{https://groups.google.com/group/comp.text.tex/msg/bbef4263390d647b}}
%
% \bibitem{ifstrempty:diez3}
% Ulrich Diez:
% \textit{Re: TeX refuses to strip outer braces in argument};
% newsgroup \xnewsgroup{comp.text.tex},
% \url{news:idd4ga$r83$1@four.albasani.net},
% 2010-12-04.\\
% {\small
% \url{https://groups.google.com/group/comp.text.tex/msg/00dfd1ec103cd272}}
%
% \bibitem{ifstrempty:gl}
% GL:
% \textit{Re: TeX refuses to strip outer braces in argument};
% newsgroup \xnewsgroup{comp.text.tex},
% \url{news:4cfa2e27$0$7389$426a74cc@news.free.fr},
% 2010-12-04.\\
% {\small
% \url{https://groups.google.com/group/comp.text.tex/msg/d3a75995c1cf267e}}
%
% \bibitem{ifstrempty:ho}
% Heiko Oberdiek:
% \textit{Re: TeX refuses to strip outer braces in argument};
% newsgroup \xnewsgroup{comp.text.tex},
% \url{news:iddhq1$3kj$1@news.eternal-september.org},
% 2010-12-04.\\
% {\small
% \url{https://groups.google.com/group/comp.text.tex/msg/5f7a23e3ab70e347}}
%
% \bibitem{ifboxempty:dak}
% David Kastrup:
% \textit{How to detect if \cs{vbox} is empty};
% newsgroup \xnewsgroup{comp.text.tex},
% 2011-02-04.\\
% {\small
% \url{https://groups.google.com/group/comp.text.tex/msg/8d3cb89496a4d86d}}
%
% \end{thebibliography}
%
% \begin{History}
%   \begin{Version}{2009/08/05 v1.0}
%   \item
%     First version.
%   \end{Version}
%   \begin{Version}{2009/12/12 v1.1}
%   \item
%     Short title shortened.
%   \item
%     \cs{ltx@IfUndefined} added.
%   \end{Version}
%   \begin{Version}{2010/01/28 v1.2}
%   \item
%     \cs{ltx@RemovePrefix} and \cs{ltx@StripPrefix} added.
%   \item
%     \cs{ltx@ifclassloaded}, \cs{ltx@ifpackageloaded},
%     \cs{ltx@iffileloaded},
%     \cs{ltx@ifclasslater}, \cs{ltx@ifpackagelater},
%     \cs{ltx@iffilelater},
%     \cs{ltx@clsextension}, \cs{ltx@pkgextension} added.
%   \item
%     \cs{ltx@GlobalAppendToMacro}, \cs{ltx@LocalAppendToMacro} added.
%   \end{Version}
%   \begin{Version}{2010/03/01 v1.3}
%   \item
%     \cs{ltx@newif} added.
%   \item
%     \cs{ltx@ifnextchar} added.
%   \item
%     Numbers \cs{ltx@zero}, \cs{ltx@one}, \cs{ltx@two}, \cs{ltx@cclv} added.
%   \end{Version}
%   \begin{Version}{2010/03/09 v1.4}
%   \item
%     \cs{ltx@pkgextension} and \cs{ltx@clsextension} are hardcoded to
%     avoid trouble with \cs{@onlypreamble}.
%   \end{Version}
%   \begin{Version}{2010/04/08 v1.5}
%   \item
%     \cs{ltx@cartwo}, \cs{ltx@cdrtwo},
%     \cs{ltx@carthree}, \cs{ltx@cdrthree},
%     \cs{ltx@carfour}, \cs{ltx@cdrfour} added.
%   \item
%     \cs{ltx@ReturnAfterFi} and \cs{ltx@ReturnAfterElseFi} fixed.
%   \end{Version}
%   \begin{Version}{2010/04/16 v1.6}
%   \item
%     \cs{ltx@leavevmode}, \cs{ltx@mbox} added.
%   \end{Version}
%   \begin{Version}{2010/04/26 v1.7}
%   \item
%     \cs{ltx@GobbleNum}, \cs{ltx@CdrNum}, \cs{ltx@CarNum} added.
%   \item
%     \cs{ltx@carzero}, \cs{ltx@cdrzero} added.
%   \item
%     \cs{ltx@hashchar} added.
%   \end{Version}
%   \begin{Version}{2010/09/11 v1.8}
%   \item
%     \cs{ltx@leftbracechar}, \cs{ltx@rightbracechar} added.
%   \end{Version}
%   \begin{Version}{2010/10/25 v1.9}
%   \item
%     \cs{ltx@LocalAppendToMacro} and \cs{ltx@GlobalAppendToMacro} are
%     now \cs{long}.
%   \end{Version}
%   \begin{Version}{2010/10/31 v1.10}
%   \item
%     \cs{ltx@newglobalif} added.
%   \end{Version}
%   \begin{Version}{2010/11/12 v1.11}
%   \item
%     \cs{ltx@ifempty} added.
%   \item
%     \cs{ltx@firstofthree}, \cs{ltx@secondofthree}, \cs{ltx@thirdofthree}
%     added.
%   \end{Version}
%   \begin{Version}{2010/12/02 v1.12}
%   \item
%     \cs{ltx@onelevel@sanitize} added.
%   \item
%     \cs{LTXcmds@num} fixed for the case with \cs{numexpr}
%     (bug found by GL).
%   \end{Version}
%   \begin{Version}{2010/12/04 v1.13}
%   \item
%     \cs{ltx@ifblank} added.
%   \item
%     Optimization for \cs{ltx@ifempty}.
%   \end{Version}
%   \begin{Version}{2010/12/07 v1.14}
%   \item
%     \cs{ltx@zapspace} added.
%   \end{Version}
%   \begin{Version}{2010/12/12 v1.15}
%   \item
%     \cs{ltx@minusone} added.
%   \end{Version}
%   \begin{Version}{2011/02/04 v1.16}
%   \item
%     \cs{ltx@IfBoxEmpty} and \cs{ltx@IfBoxVoidOrEmpty} added.
%   \item
%     \cs{ltx@firstoffour}, \dots, \cs{ltx@fourthoffour} added.
%   \end{Version}
%   \begin{Version}{2011/02/05 v1.17}
%   \item
%     \cs{ltx@IfBoxEmpty}: an empty box may have non-zero dimensions.
%   \end{Version}
%   \begin{Version}{2011/03/16 v1.18}
%   \item
%     \cs{ltx@ifclasslater} fixed.
%   \end{Version}
%   \begin{Version}{2011/04/14 v1.19}
%   \item
%     \cs{ltx@ifnextchar}: detection of optional spaces modified.
%   \item
%     \cs{ltx(Loc,Glob)(Toks,Dimen,Skip)(A,B,C,D,E)} added.
%   \end{Version}
%   \begin{Version}{2011/04/18 v1.20}
%   \item
%     \cs{ltx@ifnextchar} with conditional support (thanks GL for
%     bug report).
%   \end{Version}
%   \begin{Version}{2011/08/22 v1.21}
%   \item
%     \cs{ltx@GlobalPrependToMacro}, \cs{ltx@LocalPrependToMacro} added
%     (feature request of Martin M\"unch).
%   \end{Version}
%   \begin{Version}{2011/11/09 v1.22}
%   \item
%     \cs{ltx@carsecond}, \cs{ltx@carthird}, \cs{ltx@carfourth},
%     \cs{ltx@CarNumth} added.
%   \item
%     \cs{ltx@cdrzero}, \cs{ltx@cdr}, \cs{ltx@cdrtwo}, cs{ltx@cdrthree},
%     \cs{ltx@cdrfour}, \cs{ltx@CdrNum} modified to retain braces and spaces.
%     They are expandable in two expansion steps.
%   \end{Version}
%   \begin{Version}{2016/05/16 v1.23}
%   \item
%     Documentation updates.
%   \end{Version}
%   \begin{Version}{2019/12/15 v1.24}
%   \item
%     Documentation updates.
%   \end{Version}
% \end{History}
%
% \PrintIndex
%
% \Finale
\endinput
