\ProvidesFile{catcodes.tex}[2015/11/14 documenting catcodes files]
\title{\pkg{catcodes}\\---\\``Generic" Switching of Category Codes}
% \listfiles
{ \RequirePackage{makedoc} \ProcessLineMessage{}
  \renewcommand\mdSectionLevelOne  {\string\subsection}
  \renewcommand\mdSectionLevelTwo  {\string\subsubsection}
  \MainDocParser{\SectionLevelTwoParseInput}
  \HeaderLines{18}              %% 18 2012/08/26
  \MakeSingleDoc{stacklet.sty}
  \MakeSingleDoc{actcodes.sty}  %% 2012/08/26
  \MakeSingleDoc{catchdq.sty}   %% 2012/09/16
}
% \RequirePackage[ir]{inputtrc}                 %% 2015/11/14
\documentclass[fleqn]{article}%% TODO paper dimensions!?
\ProvidesFile{makedoc.cfg}[{2013/03/25 documentation settings}] 
%%
\author{Uwe L\"uck\thanks{%
        \url{http://contact-ednotes.sty.de.vu}}}
%%
%% 'hyperref':
\RequirePackage{ifpdf}
\usepackage[%
  \ifpdf
%     bookmarks=false,                  %% 2010/12/22
%     bookmarksnumbered,
    bookmarksopen,                      %% 2011/01/24!?
    bookmarksopenlevel=2,               %% 2011/01/23
%     pdfpagemode=UseNone,
%     pdfstartpage=10,
    pdfstartview=FitH,                  %% 2012/11/26 again
%     pdfstartview=0 0 100,             %% 2011/08/22
%     pdfstartview={XYZ null null 1},   %% 2011/08/25
%     pdfstartview={XYZ null null null},%% 2011/08/25
%     pdfstartview={XYZ null null .5},    %% 2011/08/26
%     pdffitwindow=true,          %% 2011/08/22
    citebordercolor={ .6 1    .6},
    filebordercolor={1    .6 1},
    linkbordercolor={1    .9  .7},
     urlbordercolor={ .7 1   1},   %% playing 2011/01/24
  \else
    draft
  \fi
]{hyperref}
\hypersetup{% 
    pdfauthor={Uwe L\374ck}% 
}
%% metadata, |\MDkeywords{<text>}|, |\MDkeywordsstring|:
%% %% 2011/08/22:
\makeatletter
  \newcommand*{\MDkeywords}[1]{%
    \gdef\MDkeywordsstring{#1}%
    \hypersetup{pdfkeywords=\MDkeywordsstring}%% TODO!?
  }
  \@onlypreamble\MDkeywords
%% |\MDaddtoabstract{<par-head>}|, `:' added:
  \newcommand*{\MDaddtoabstract}[1]{%           %% 2012/05/10
    \par\smallskip\noindent
    \strong{#1:}\quad\ignorespaces}
%% \pagebreak[2]
%% |\printMDkeywords|:
  \newcommand*{\printMDkeywords}{%
    \MDaddtoabstract{Keywords}%
    \MDkeywordsstring 
%     \global\let\MDkeywordsstring\relax    %% `%' 2012/11/12
  }
%% The previous definitions mainly are useful with a variant 
%% |\begin{MDabstract}| of \LaTeX's `{abstract}' environment:
  \newenvironment{MDabstract}
                 {\abstract\noindent
                  \hspace{1sp}%% for niceverb
                  \ignorespaces}
                 {\@ifundefined{MDkeywordsstring}%
                               {}%
                               {\printMDkeywords}%
                  \global\let\MDabstract\relax    %% 2012/11/12
                  \global\let\endMDabstract\relax %% 2012/11/12
                  \endabstract}
%% |\[MD]docnewline| 2012/11/12 from `readprov.tex':
  \newcommand*{\MDdocnewline}{\leavevmode\@normalcr[\topsep]}
%% <- `\leavevmode' for use with `\paragraph'.
%%    Sometimes needs to be preceded by a space.
%% 
%% |\MDfinaldatechecks[<tex-script>]| with \ctanpkgref{filedate}:
  \newcommand*{\MDfinaldatechecks}[1][fdatechk]{%
    \AtEndDocument{%
%       \clearpage %% 2013/03/25 no avail -- with `filedate'!
      \def\@pkgextension{sty}%
      \def\NeedsTeXFormat##1[##2]{}%
      \noNiceVerb                       %% 2013/03/22
      \input{#1}%
    }}
  \@onlypreamble\MDfinaldatechecks
\makeatother
%% Use other packages:
\RequirePackage{niceverb}[2011/01/24] 
\RequirePackage{readprov}               %% 2010/12/08
\RequirePackage{hypertoc}               %% 2011/01/23
\RequirePackage{texlinks}               %% 2011/01/24
\RequirePackage{relsize}                %% 2011/06/27
\RequirePackage{color}                  %% 2011/08/06
\RequirePackage{lmodern}                %% 2012/10/29
\RequirePackage{filedate}               %% 2012/11/12
\RequirePackage{filesdo}                %% 2013/03/22 
%% \pagebreak[3]
%% Logical markup:\qquad  |\strong{<chars>}|, |\meta{<chars>}|, 
%% |\acro{<chars>}|, |\pkg{<chars>}|, 
%% |\code{<chars>}|, |\file{<chars>}|:{\sloppy\par}
\makeatletter
  \def\do#1#2{\@ifdefinable#1{\let#1#2}}%% 2012/07/13
  \do\strong\textbf \do\file\texttt \do\acro\textsmaller 
  %% <- wrong tests before 2012/07/13
  \do\meta\textit   \do \pkg\textsf \do\code\texttt
  \ifpdf
    \pdfstringdefDisableCommands{%
        \let\acro\textrm 
        \let\file\textrm                            %% 2011/11/09
        \let\code\textrm                            %% 2011/11/20
        \let\pkg \textrm                            %% 2012/03/23
    }
  \fi
  %% TODO 2011/07/22 -> `htlogml.sty'
\makeatother
%% |\qtdcode{<text>}|: 2012/10/24:
    \newcommand*{\qtdcode}[1]{`\code{#1}'} 
%% |\pkgtitle{<package-name>}{<caption>}| 
\newcommand*{\pkgtitle}[2]{%            %% 2012/07/13
    \global\let\pkgtitle\relax
    \pkg{\huge #1}\\---\\#2\thanks{This 
       document describes version 
       \textcolor{blue}{\UseVersionOf{\jobname.sty}} 
       of \textsf{\jobname.sty} as of \UseDateOf{\jobname.sty}.}}
%% TODO: %% |\TODO| bad with `mdoccorr.cfg'
\newcommand*{\TODO}{\textcolor{blue}{\acro{TODO}}}  %% 2012/11/06
%% `\MDsampleinput[{<file>}' was added 2012/11/06. 
%% Problems with `myfilist.tex' were due to 'parskip.sty'
%% there. On 2012/11/12, we change the former simple macro to a 
%% much more complex
%% |\MDsamplecodeinput[<add-hfuss>]{<file>}| 
\newcommand*{\MDsamplecodeinput}[2][]{%
    \begingroup
        \vskip\bigskipamount \hrule
        \nobreak\vskip-\parskip 
%         \nobreak\vskip\medskipamount
%% Previous mistake (same below) due to manual change 
%% of `\topsep' in the file `myfilist.tex' (2012/11/30).
        \ifx\\#1\\\else
            \hfuzz=\textwidth \advance\hfuzz#1\relax
        \fi
        \noNiceVerb \verbatiminput{#2}%
%         \nobreak\vskip\medskipamount 
        \hrule \vskip-\parskip 
        \bigskip %%% \bigbreak
%% `\bigbreak' made much larger space in `myfilist.tex'.
    \endgroup
}
%% |\ctanpkgdref{<pkg-id>}| adds the printed link to 
%% `ctan.org/pkg' as a footnote. There is a little space 
%% for coloured link borders:
\newcommand*{\ctanpkgdref}[1]{%
    \ctanpkgref{#1}\,\urlfoot{CtanPkgRef}{#1}}
\errorcontextlines=4
\pagestyle{headings}

\endinput 
 %% shared formatting settings
% \ReadPackageInfos{stacklet}
\MDkeywords{Macro programming, category codes, private letters, 
            active characters, double quotes}
\newcommand*{\headersec}{%
    \subsection{Package File Header---\pkg{plainpkg} and Legalese}}
\usepackage{ngerman} \originalTeX
\usepackage{catchdq}    %% try it more often! TODO 2012/11/07
\usepackage{langcode}                           %% 2012/11/07
\usepackage{filedate}                           %% 2012/11/06
\sloppy
% \flushbottom                                    %% 2012/11/07
\begin{document}
\maketitle
\begin{MDabstract}
The 'catcodes' bundle provides small packages for switching
category codes, usable both with \LaTeX\ and without. \ 
(i)\enspace \pkg{stacklet.sty} maintains stacks for ``private letters," 
needed for 'plainpkg.tex''s minimal framework for ``generic" 
packages. \ 
(ii)\enspace \pkg{actcodes.sty} deals with ``active characters," 
switching their category codes and assigning meanings to 
``active-character tokens." \ 
(iii)\enspace
%% 2012/09/16:
\pkg{catchdq.sty} uses the 
``\acro{ASCII}                          %% 2012/11/06
double quote" as an active character 
for simplified access to typographical double quotes.---These 
packages are ``generic" in the sense that they should 
be usable at least both with \LaTeX\ and Plain \TeX, 
based on 'plainpkg.tex'.

\MDaddtoabstract{Required Packages}
\ctanpkgref{plainpkg}, 'stacklet'
\MDaddtoabstract{Related Packages} 
\ctanpkgref{catoptions}, 'pcatcode' from \ctanpkgref{amsrefs}, 
\ctanpkgref{texapi}, 
\ctanpkgref{csquotes}.  %% 2012/09/16
\end{MDabstract}
\tableofcontents
\section{Overview}
Sorry, \dots, the abstract and the table of contents 
must suffice for today (2012-11-07) \TODO

\section{Shared Features of Usage}
%%% rm. 2012/09/17:
% \section{General Background---Usage, Required}
% The packages of this bundle are part of a certain family of 
% ``generic" packages: They should work \emph{with} \LaTeX\ 
% as well as \emph{without} \LaTeX, aiming at 
% ``independency of format"---actully too much now~...
% most importantly, they require \ctanpkgref{plainpkg}. 
%%% rm 2012/09/16:
% , 
% and the latter's documentation should tell you about 
% how to use the present packages.
%
All the packages of the bundle are ``\pkg{plainpkg} packages" 
in the sense of the 
\ctanpkgref{plainpkg}\foothttpurlref{ctan.org/pkg/plainpkg}
documentation that exhibits details of what is summarized here. 
Therefore:
\pagebreak[2]                               %% 2012/11/07
\begin{itemize}
  \item All of them require that \TeX\ finds `plainpkg.tex' 
        as well as `stackrel.sty'.
  \item In order to load `<catcodes>.sty'
        (where <catcodes> is `stacklet', `actcodes', or `catchdq'), 
        type \ |\usepackage{<catcodes>}| \ within a \LaTeX\ document 
        preamble, \ |\RequirePackage{<catcodes>}| \ in a 
        ``\pkg{plainpkg} package", or \ |\input <catcodes>.sty| \
        $\dots$ \ or perhaps `\input{<catcodes>.sty}'? 
\end{itemize}

\section{'actcodes.sty'---Active Characters}
%% intro 2012/11/07:
\subsection{Introduction}
Active characters can simplify syntax often, i.e., the code 
may be very pleasant to type and read. But sometimes something 
may fail. See Section~\ref{sec:actcodes-cmds} for how to cope
with possibilities and difficulties.
\headersec
                                                 \input plainpkg
\ProvidesPackage{actcodes}[2012/11/07 v0.2a active characters (UL)]
%%
%% Copyright (C) 2012 Uwe Lueck,
%% http://www.contact-ednotes.sty.de.vu
%% -- author-maintained in the sense of LPPL below --
%%
%% This file can be redistributed and/or modified under
%% the terms of the LaTeX Project Public License; either
%% version 1.3c of the License, or any later version.
%% The latest version of this license is in
%%     http://www.latex-project.org/lppl.txt
%% There is NO WARRANTY (actually somewhat experimental).
%%
%% Please report bugs, problems, and suggestions via
%%
%%   http://www.contact-ednotes.sty.de.vu
%%
%%  == Purpose and Usage            ==
%% The package derives from switching category codes 
%% in the 'nicetext' and 'morehype' bundles and should 
%% improve them. 
%% 
%% === Installing and Calling       ===
%% The file `actcodes.sty' is provided ready, 
%% installation only requires
%% putting it somewhere where \TeX\ finds it
%% (which may need updating the filename data
%%  base).\urlfoot{ukfaqref}{inst-wlcf} 
%% However, the files `plainpkg.tex' and `stacklet.sty' %% stacklet 2012/09/19
%% must be installed likewise.
%% 
%% \emph{With} \LaTeX, the file should be loaded by `\RequirePackage{actcodes}'
%% or `\usepackage{actcodes}'. 
%% 
%% \emph{Without} \LaTeX, load it by `\input actcodes.sty'.
%%
%% As explained in `plainpgk-doc.pdf', however, ``generic" 
%% packages based on 'plainpkg' should load 'actcodes' by 
%% `\RequirePackage{actcodes}'.
%%
%% === Commands and Syntax          ===
%% \label{sec:actcodes-cmds}
%% 'actcodes.sty' provides \ |\MakeActive|, |\MakeActiveAss|, 
%% |\MakeActiveDef|, |\MakeActiveLet|, 
%% |\MakeOther|, |\MakeActiveOther| \ with rather obvious 
%% syntax---you find more detailed descriptions 
%% mixed with implementation below ... TODO
%% ---Without \LaTeX, the latter's internal |\@gobble<arg>|
%% is provided as well.
%% 
%%  == The Code                     ==
%% === Our Private Letters          ===
\PushCatMakeLetterAt
%%
%% === The Core                     ===
%% |\MakeActiveAss<ass-fun>\<char><ass-args>| ``activates" <char> 
%% and then applies assignment function <ass-fun> with arguments 
%% <ass-args> to it. The code derives from \LaTeX's
%% `\@sverb' and `\do@noligs' and was also discussed 
%% on the \acro{LATEX-L} mailing list September 2010
%% (Will Robertson; Heiko Oberdiek). 
%% The present definition generalizes 
%% `\MakeActiveDef' and `\MakeActiveLet' of my earlier packages.
\gdef\MakeActiveAss#1#2{%
  \MakeActive#2%
  \begingroup \lccode`\~`#2\relax \lowercase{\endgroup #1~}} 
%% I was reluctant to provide |\MakeActive\<char>|, 
%% but with 'catchdq.sty', it would be better ...   %% 2012/09/16
\gdef\MakeActive#1{\catcode`#1\active}
%% ... it just ``revives" the meaning that the corresponding 
%% active-character token last time ...
%%
%% === \cs{def} and \cs{let}        ===
%% |\MakeActiveDef\<char><parameters>{<replace>}| 
%% has been employed in 'fifinddo' and 'blog' 
%% (which is based on 'fifinddo') so far. 
\gdef\MakeActiveDef{\MakeActiveAss\def}
%% W.r.t.\ the definition of this `\MakeActiveDef', 
%% Heiko Oberdiek remarked that it allows \emph{macro parameters}, 
%% as opposed to my earlier definition in 'fifinddo'. 
%% Without parameters, this kind of macro has been used for 
%% conversion of text encodings ('atari.fdf', 
%% and I thought this was the idea of \ctanpkgref{stringenc} ...).
%% 
%% |\MakeActiveLet\<char><cmd>| has been provided in 'niceverb' so far. 
%% The present package has been made in order to have 
%% `\MakeActiveLet' with 'blog.sty' as well, it was too annoying 
%% to use `\MakeActiveDef' there so often.
\gdef\MakeActiveLet{\MakeActiveAss\let}
%% 
%% === Switching Back \textellipsis ===
%% Sometimes, the ``active" behaviour of <char> is too difficult, 
%% and you may want to switch bach to its ``simple" way ... 
%% This may work by |\MakeOther\<char>| ... 
%% with \LaTeX, `\MakeOther' just is `\@makeother' ...
\ifltx \global\let\MakeOther\@makeother
\else  \gdef\MakeOther#1{\catcode`#112\relax}
\fi
%% But within a macro (or other) argument, you can't change the `\catcode'. 
%% (I~lost some time by not realizing that it was within a large argument 
%%  where I tried to switch the `\catcode'.)
%% Anyway or in certain cases, it may be better to keep a character 
%% ``active" throughout a document and just to change the \emph{expansion}
%% of the ``active-character token." This can be done with 
%% `\MakeActiveLet' and `\MakeActiveDef' in certain cases already. 
%% E.g., when the \emph{``blank space"} has been ``activated" by 
%% `\obeylines', `\MakeActiveLet\ \space' ``undoes" this half-way, 
%% while it does not restore ``argument skipping" and 
%% ``compressing blank spaces."
%% 
%% When character <char> should be ``active" for some time, 
%% but for certain moments you prefer that it behaves like an 
%% ``other character", you can switch to its ``other" expansion by 
%% |\MakeActiveOther\<char>|:
\gdef\MakeActiveOther#1{%
    \MakeActiveAss\edef#1{\expandafter\@gobble\string#1}}
%% `\MakeActiveOther' uses \LaTeX's |\@gobble<arg>|, 
%% \emph{without} \LaTeX, 'actcodes' provides it:
\ifltx\else \long\gdef\@gobble#1{} \fi
% \show_ \MakeActiveOther\_ \show_ \expandafter\show_
%% I am \emph{not} providing a version \emph{without}
%% the `\catcode' change, although the latter is superfluous here
%% TODO ...
%% 
%% `niceverb' also provides |\MakeNormal\<char>|, it may migrate %% |...| 2012/11/07
%% to here in the future, and there may be |\MakeActiveNormal\<char>|
%% extending the above `\MakeActiveOther' TODO ...
%% 
%% Also, a \emph{stack} might be used as in 'stacklet', 
%% even to switch \emph{meanings} of active-character tokens ... 
%% not sure TODO ...
%% 
%% \ctanpkgref{babel} does similar things, but I never have ... TODO
%%
%% === Leaving and Version History  ===
\PopLetterCatAt
\endinput
%%
%% VERSION HISTORY

v0.1   2012/08/26   started, almost completed 
       2012/08/27   completed; realizing \Push...At ..., bug fixes
v0.2   2012/08/28   \global\let, \def -> \gdef
       2012/09/16   \MakeActive
       2012/09/19   doc.: stacklet
v0.2a  2012/11/07   doc.: |...| on \MakeNormal


\section{'catchdq.sty'---Typographical Double Quotes}
%% <- TODO cf. morgan.sty 2012/11/06 ->
% \section{'catchdq.sty'---Automatic Typographical Double Quotes by Toggling}
% \section{'catchdq.sty'---Proper Double Quotes by Toggling}
%% <- morgan.sty 2012/11/06 ->
\subsection{Introduction}
\catchdqs
The 'catchdq' package allows getting typographical double quotes 
by just using the "\acro{ASCII} double quote" |"|. A more precise overview:
\begin{enumerate}
  \item Typically, "typographical" quotation marks mean distinguishing 
        between "opening" and "closing" quotation marks. 
        Usually, you must enter different characters or commands for 
        the distinction, such as |``| for "opening" and |''| for 
        \emph{closing}---in \emph{English} with \emph{\TeX}. 
        For \emph{English} with \emph{Plain~\TeX}, even |"| suffices for "closing."
  \item There are much different conventions especially for \emph{German} 
        and \emph{French}. They require different characters or \TeX~commands
        than for \emph{English}. The packages \ctanpkgref{german}, 
        \ctanpkgref{ngerman}, and \ctanpkgref{babel} have dealt with 
        such conventions. 
  \item Understanding the ideas mentioned before has been difficult 
        for a long time, probably because typewriter and computer \emph{keyboards} 
        never have offered the appropriate keys. Rather, they only offered 
        the "\acro{ASCII} double quote" that produced an approximation 
        ("neutral quotation marks")
        \emph{not} making the difference. Many users and readers 
        have not realized the difference, they have not realized 
        how their screen or printer output differed from double quotes 
        in books and newspapers. Cf.~the 
        \wikienref{Quotation mark}{Wikipedia article}\footnote{%
            \urlhttpref{en.wikipedia.org/wiki/Quotation mark}} 
  \item The idea of the 'catchdq' package is that the user indeed should 
        not worry about that difference and just type 
        "\acro{ASCII} double quotes", and they should be "converted" 
        into the appropriate typographical quotation marks \emph{automatically}. 
        This should work by "toggling," i.e., the first 
        "\acro{ASCII} double quote" is interpreted as "opening," 
        the second as "closing," the next one as "opening"~\dots
        ---Word processors have provided this feature (as an option) 
        as well.
  \item Language-dependency of the feature currently is managed 
        through the \ctanpkgref{langcode} package.
  \item The feature may cause problems sometimes. 
        Therefore, explicit switching the feature "on" 
        and "off" is required.
  \item The \ctanpkgref{csquotes} package addresses the issue 
        in a more comprehensive and perhaps more stable way.
\end{enumerate}
See Section~\ref{sec:catchdq-cmds} for 
% the commands provided.
% Note that the \ctanpkgref{csquotes} 
% provides more comprehensive functionality.
additional details.                         %% 2012/11/06
\MakeOther\"                                %% TODO 2012/11/06

\headersec
                                                 \input plainpkg
\ProvidesPackage{catchdq}[2015/05/22 v0.21 typographic dqs (UL)]
%% %% rm. "simple" -- too long -- 2015/05/22
%%
%% Copyright (C) 2012 2015 Uwe Lueck,
%% http://www.contact-ednotes.sty.de.vu
%% -- author-maintained in the sense of LPPL below --
%%
%% This file can be redistributed and/or modified under
%% the terms of the LaTeX Project Public License; either
%% version 1.3c of the License, or any later version.
%% The latest version of this license is in
%%     http://www.latex-project.org/lppl.txt
%% There is NO WARRANTY (actually somewhat experimental).
%%
%% Please report bugs, problems, and suggestions via
%%
%%   http://www.contact-ednotes.sty.de.vu
%%
%%  == Purpose and Usage            ==
%% 
%% === Installing and Calling       ===
%% The file `catchdq.sty' is provided ready, 
%% installation only requires
%% putting it somewhere where \TeX\ finds it
%% (which may need updating the filename data
%%  base).\urlfoot{ukfaqref}{inst-wlcf} 
%% However, the files `plainpkg.tex' and `stacklet.sty' %% stacklet 2012/09/19
%% must be installed likewise.
%% 
%% \emph{With} \LaTeX, the file should be loaded by `\RequirePackage{catchdq}'
%% or `\usepackage{catchdq}'. 
%% 
%% \emph{Without} \LaTeX, load it by `\input catchdq.sty'.
%%
%% As explained in `plainpgk-doc.pdf', however, ``generic" 
%% packages based on 'plainpkg' should load 'catchdq' by 
%% `\RequirePackage{catchdq}'.
%%
%% === Commands and Syntax          ===
%% \label{sec:catchdq-cmds}
%% 'catchdq.sty' (indirectly) allows using |"<no-dqs>"|
%% for surrounding <no-dqs> with typographical quotation marks, 
%% using that double quote |"| as an active character. 
%% As rendering that `"' active during defining macros can 
%% corrupt the latter, the user (or package writer) must 
%% activate that `"' explicitly by |\catchdqs|. 
%%
%% Further difficulties may arise after `\catchdqs', 
%% various ways to get around them are described in the remaing sections.
%% 
%%  == The Code                     ==
%% === Required ===
%% The package is an application (of ideas of) 'actcodes.sty':
\RequirePackage{actcodes}
%%
%% === The Core: \cs{catchdq} ===
%% % \label{sec:catchdq-core}
%% |\catchdq<no-dqs>"| will expand to |\dqtd{<no-dqs>}|, 
%% provided the \acro{ASCII} double quote is an active character: 
{\MakeActive\"\gdef\catchdq#1"{\dqtd{#1}}}
%% 
%% === What Double Quotes Actually Are ===
%% |\dqtd| in turn is a kind of ``variable." 'blog.sty' offered 
%% |\endqtd| for English typographical double quotes, 
%% |\dedqtd| for German ones, and |\asciidqtd| for ``non-typographical" 
%% double quotes (as needed for \acro{XML} attributes).
%% |\asciidq| accesses a single \acro{ASCII} double quote, 
%% |\enldq| a single English typographical left one, 
%% |\enrdq| a single English typographical right one. 
%% (It may be useful to access them indepentently of each other, 
%%  in certain complex situations ...)
%% 'blog.sty', dealing with \acro{HTML}, of course has different ideas 
%% about them TODO.
\gdef\asciidq{"}
\gdef\asciidqtd#1{"#1"}
%% We allow loading 'catchdq' \emph{after} another package 
%% (such as 'blog.sty') has chosen meanings for `\endqtd' and the 
%% like. Before v0.21, definedness was tested by
%% `\ifx'...`\undefined', which two times fell
%% prey to some earlier `\@ifundefined'. So now:
\begingroup \escapechar=-1
\def\provass#1#2#3{%
    \expandafter\ifx\csname \string#2\endcsname\relax #1#2#3\fi}
\provass \gdef         \enldq  {{``}}
\provass {\global\let} \enrdq  \asciidq
\provass \gdef         \endqtd {#1{\enldq#1\enrdq}}
%% Typographical alternatives to `\endqtd' may be obtained 
%% from \ctanpkgstyref{ngerman} or so, if you are smart ...
%% %% <- 2012/09/20 ->
%% (see Section~\ref{sec:sw} for how it works):
\provass \gdef         \dedqtd {#1{\glqq#1\grqq}}
%% 'blog.sty', dealing with \acro{HTML}, had a different idea 
%% about `\endqtd' of course. It has also used the mechanism of 
%% the \ctanpkgref{langcode} package that allows using `\dqtd' and other 
%% language-depended constructs with an ``implicit" choice according 
%% to the ``current language code," which should appear soon.
%% 
%% === Switching ===
%% \label{sec:sw}
%% 'blog.sty' usually does a single switch which gets a new name now:
%% |\catchdqs|.
\gdef\catchdqs{\MakeActiveLet\"\catchdq}
%% After this, |"<no-dqs>"| will expand to `\dqtd{#1}'.
%% The default expansion for |\dqtd| will be |\endqtd|:
\provass {\global\let} \dqtd \endqtd
%% Might be done by |\endqs|---when there are alternatives, 
%% but 'blog.sty' and 'langcode.sty' do this in a different way ... TODO 
% \gdef\endqs{\let\dqtd\endqtd}
% \ifx\dqtd\undefined \global\endqs \fi
\endgroup
%% \catchdqs Actually, here is a little "Tessst"~...
%% \let\dqtd\dedqtd \ and here with "doytshe doppleta unf..." $\dots$ \ 
%% %% anf -> unf 2015/05/22
%% The latter has been achieved by 
%% \[`\usepackage{ngerman} \originalTeX'\]
%% \MakeOther\"
%% 
%% |\MakeOther\"| may switch off catching mode (---done just before, 
%%  as \ctanpkgref{niceverb} at present doesn't render it verbatim). 
%%                                  'actcodes' suggests 
%% a different way to return from the `\catchdqs' state: 
%% Let the character active and change its meaning only, let it 
%% \emph{expand} to its ``other" version---by |\activeasciidqs|?
%% |\MakeActiveOther\"| and \catchdqs 
%% \let"\asciidq 
%% |\let"\asciidq| (it works!) or |\MakeActiveLet\"\asciidq|
%% (abbreviate as `\activeasciidqs'?) ... In 'blog.sty', there never 
%% was a need for switching back. We must rework interaction with 
%% 'niceverb' and can perhaps simplify the latter, ... TODO
%%
%% === Leaving and Version History  ===
\endinput
%%
%% VERSION HISTORY
 
v0.1   2010/11/13   in texblog.fdf
v0.2   2012/09/17   own file, new ideas ... 
       2012/09/19   doc: stacklet
       2012/09/20   \dedqtd conditionally; reworked doc., 
                    tested ngerman.sty
v0.21  2015/05/22   better test for undefinedness


\pagebreak
\section{'stacklet.sty'---Private Letters}
\subsection{Introduction}
\catchdqs           %% 2012/11/07
\uselangcode{en}    %% 2012/11/07, \dedqtd above
"Private letters" \emph{here} are meant to be 
characters that belong to the "letter" 
category only within packages. A package typically provides 
user commands as well as internal commands, and the latter 
are characterized by containing funny letters in commands such 
as `\@gobble.' This is to avoid conflicts.
See Section~\ref{sec:stacklet-cmds} for the commands provided.
\MakeOther\"        %% 2012/11/07
\headersec
                                                 \input plainpkg
\ProvidesPackage{stacklet}[2012/11/07 v0.3a private letters (UL)]
%%
%% Copyright (C) 2012 Uwe Lueck,
%% http://www.contact-ednotes.sty.de.vu
%% -- author-maintained in the sense of LPPL below --
%%
%% This file can be redistributed and/or modified under
%% the terms of the LaTeX Project Public License; either
%% version 1.3c of the License, or any later version.
%% The latest version of this license is in
%%     http://www.latex-project.org/lppl.txt
%% There is NO WARRANTY (actually somewhat experimental).
%%
%% Please report bugs, problems, and suggestions via
%%
%%   http://www.contact-ednotes.sty.de.vu
%%
%%  == Usage                    ==
%% === Installing and Calling   ===
%% The file `stacklet.sty' is provided ready, 
%% installation only requires
%% putting it somewhere where \TeX\ finds it
%% (which may need updating the filename data
%%  base).\urlfoot{ukfaqref}{inst-wlcf} 
%% However, the file `plainpkg.tex' must be installed likewise.
%% 
%% \emph{With} \LaTeX, the file should be loaded by `\RequirePackage{stacklet}'
%% or `\usepackage{stacklet}'. 
%% 
%% \emph{Without} \LaTeX, both `\input stacklet.sty' and 
%% `\input plainpkg' load `stacklet.sty'.
%%
%% \filbreak                                        %% 2012/11/07
%% === Commands and Syntax      ===
%% \label{sec:stacklet-cmds}
%% 'stacklet.sty' provides 
%% \[|\PushCatMakeLetter\<char>|
%%   \quad \textit{and}\quad 
%%   |\PopLetterCat\<char>|\]
%% for getting ``private letters" and giving them back
%% their previous category code
%% in package files with or without \LaTeX.
%% As \LaTeX\ has its own stack for `@', there are also 
%% \[|\PushCatMakeLetterAt|
%%   \quad \textit{and}\quad 
%%   |\PopLetterCatAt|\]
%% that care for `@''s category code \emph{without} \LaTeX\ only.
%% 
%%  == The Code                 ==
%% === Name Space               ===
%% Each ``private letter" <char> gets its own stack, 
%% in some name space, determined by |\cat_stack|
%% (`\withcsname' is from 'plainpkg.tex'):
\withcsname\xdef cat_stack\endcsname{%
    \noexpand\string \withcsname\noexpand cat_stack\endcsname
        \noexpand\string}
%% I.e., ?`cat_stack' will expand to 
%% \[?`string'\,?`cat_stack'\,?`string'\]
%% in the notation of the \ctanpkgref{dowith} package
%% documentation.                                       %% 2012/11/06
% \withcsname\show cat_stack\endcsname
%% 
%% === Pushing                  ===
%% |\PushCatMakeLetter\<char>| ...
\xdef\PushCatMakeLetter#1{%
  \noexpand\withcsname
    \withcsname\noexpand pushcat_makeletter\endcsname
      \withcsname\noexpand cat_stack\endcsname#1\endcsname#1}
% \show\PushCatMakeLetter
\withcsname\gdef pushcat_makeletter\endcsname#1#2{%
%% #1 is the stack token, #2 is the ``quoted" character. Pushing~...
    \xdef#1{\the\catcode`#2\relax%
%% ... the new entry. `\relax' separates entries, 
%% braces instead tend to get lost in popping ...
%% If the stack has existed before, 
%% its previous content is appended:
            \ifx#1\relax \else #1\fi}%
%% I thought of storing `\catcode's hexadecimally (without braces) 
%% using \LaTeX's `\hexnumber', but the latter has so many tokens~...
%% Finally rendering <char> a ``letter":
    \catcode`#211\relax}
%% Now we can use a ``private letter stack" for our own package:
\PushCatMakeLetter\_
%%
%% === Popping                  ===
%% |\PopLetterCat\<char>| passes `\<char>', the corresponding 
%% stack token, and the latter's expansion to `\popcat_'. 
%% `\end' serves as argument delimiter for the end of the stack only:
\gdef\PopLetterCat#1{%
  \expandafter\expandafter\expandafter
    \popcat_\csname\cat_stack#1\expandafter\endcsname
      \expandafter \end \csname\cat_stack#1\endcsname#1}
%% `\popcat_' parses the expansion, assigns the old 
%% category code and and stores the reduced stack:
\gdef\popcat_#1\relax#2\end#3#4{\catcode`#4#1\gdef#3{#2}}
%% ... check existence? TODO
%%
%% === No `@' Stack with \LaTeX ===
%% |\PushCatMakeLetterAt| is like `\PushCatMakeLetter\@'
%% except that it has no effect under \LaTeX:
\gdef\PushCatMakeLetterAt{\ifltx\else\PushCatMakeLetter\@\fi}
%% |\PopLetterCatAt| by analogy ...
\gdef\PopLetterCatAt{\ifltx\else\PopLetterCat\@\fi}
%% 
%% === Leaving the Package File ===
%% ... in our new way:
\PopLetterCat\_
\endinput
%%
%% === VERSION HISTORY ===

v0.1   2012/08/24   started
       2012/08/25   completed
       2012/08/26   extending doc.; \def\withcsname removed
v0.2   2012/08/26   \with_catstack containing \endcsname and with
                    three popping macros replaced by \csname 
                    content \cat_stack, cf. memory.tex; 
                    restructured
       2012/08/27   \PushCatMakeLetterAt fixed
v0.3   2012/08/27   def.s global
v0.3a  2012/11/06   doc.: "documentation"
       2012/11/07   \filbreak


%% 'filedate' check 2012/11/06:
\noNiceVerb 
\ProvidesFile{fdatechk.tex}[2015/11/14 dowith filedate checks]
%% load earlier:
%\RequirePackage{filedate,filesdo}
%\UseReferenceDate{\thepdfmoddate}
\ModDates 
\DatesDiffErrors
\FileDateAutoChecks            %% 2012/12/20
\DoWithBasesExts\ReadFileInfos{
    dowith,domore}{sty,tex}
\ReadFileInfos{dowith.RLS,fdatechk.tex}
\DatesDiffWarnings             %% 2015/11/14 
\CheckDateOfToday{dowith.RLS}  %% corr. 2015/05/22

\end{document}

VERSION HISTORY

2012/08/24   for stacklet v0.1  very first
2012/08/26   for stacklet v0.2   increased \HeaderLines, abstract; 
             for actcodes v0.1  added ...
2012/09/16f. for catchdq  v0.2  added ...
2012/09/19                      "plainpkg packages", \pagebreaks, 
                                Roman numbers in abstract, \headersec
2012/09/19                      trying ngerman.sty

             UPLOADED with `catcodes' r0.1 2012-09-20

2012/11/06                      supplementing missing word; 
                                TODO on catchdq, `filedate' checks, 
                                typeset with recent `niceverb';
                                more accurate about `catchdq', 
                                long introduction
2012/11/07                      final formatting, some more info on 
                                `stacklet' and `actcodes', TODOs, 
                                some \catchdqs, `Overview',
                                2 more \subsec...Intro... 

             UPLOADED with `catcodes' r0.1a 2012-11-07

2015/11/14   for catchdq v0.21  debugging
