\ProvidesFile{langcode.tex}[2012/09/20 documenting langcode.sty]
\title{\pkgtitle{langcode.sty}{%
       Simple Language-Dependent Settings\\
       Based on Language Codes}} 
{ \RequirePackage{makedoc} \ProcessLineMessage{}
  \MakeJobDoc{17}%% 2011/11/23
  {\SectionLevelThreeParseInput}  }             %% 2012/09/17
\documentclass[fleqn]{article}
% \usepackage{inputtrc} \dotracinginputs
\ProvidesFile{makedoc.cfg}[{2013/03/25 documentation settings}] 
%%
\author{Uwe L\"uck\thanks{%
        \url{http://contact-ednotes.sty.de.vu}}}
%%
%% 'hyperref':
\RequirePackage{ifpdf}
\usepackage[%
  \ifpdf
%     bookmarks=false,                  %% 2010/12/22
%     bookmarksnumbered,
    bookmarksopen,                      %% 2011/01/24!?
    bookmarksopenlevel=2,               %% 2011/01/23
%     pdfpagemode=UseNone,
%     pdfstartpage=10,
    pdfstartview=FitH,                  %% 2012/11/26 again
%     pdfstartview=0 0 100,             %% 2011/08/22
%     pdfstartview={XYZ null null 1},   %% 2011/08/25
%     pdfstartview={XYZ null null null},%% 2011/08/25
%     pdfstartview={XYZ null null .5},    %% 2011/08/26
%     pdffitwindow=true,          %% 2011/08/22
    citebordercolor={ .6 1    .6},
    filebordercolor={1    .6 1},
    linkbordercolor={1    .9  .7},
     urlbordercolor={ .7 1   1},   %% playing 2011/01/24
  \else
    draft
  \fi
]{hyperref}
\hypersetup{% 
    pdfauthor={Uwe L\374ck}% 
}
%% metadata, |\MDkeywords{<text>}|, |\MDkeywordsstring|:
%% %% 2011/08/22:
\makeatletter
  \newcommand*{\MDkeywords}[1]{%
    \gdef\MDkeywordsstring{#1}%
    \hypersetup{pdfkeywords=\MDkeywordsstring}%% TODO!?
  }
  \@onlypreamble\MDkeywords
%% |\MDaddtoabstract{<par-head>}|, `:' added:
  \newcommand*{\MDaddtoabstract}[1]{%           %% 2012/05/10
    \par\smallskip\noindent
    \strong{#1:}\quad\ignorespaces}
%% \pagebreak[2]
%% |\printMDkeywords|:
  \newcommand*{\printMDkeywords}{%
    \MDaddtoabstract{Keywords}%
    \MDkeywordsstring 
%     \global\let\MDkeywordsstring\relax    %% `%' 2012/11/12
  }
%% The previous definitions mainly are useful with a variant 
%% |\begin{MDabstract}| of \LaTeX's `{abstract}' environment:
  \newenvironment{MDabstract}
                 {\abstract\noindent
                  \hspace{1sp}%% for niceverb
                  \ignorespaces}
                 {\@ifundefined{MDkeywordsstring}%
                               {}%
                               {\printMDkeywords}%
                  \global\let\MDabstract\relax    %% 2012/11/12
                  \global\let\endMDabstract\relax %% 2012/11/12
                  \endabstract}
%% |\[MD]docnewline| 2012/11/12 from `readprov.tex':
  \newcommand*{\MDdocnewline}{\leavevmode\@normalcr[\topsep]}
%% <- `\leavevmode' for use with `\paragraph'.
%%    Sometimes needs to be preceded by a space.
%% 
%% |\MDfinaldatechecks[<tex-script>]| with \ctanpkgref{filedate}:
  \newcommand*{\MDfinaldatechecks}[1][fdatechk]{%
    \AtEndDocument{%
%       \clearpage %% 2013/03/25 no avail -- with `filedate'!
      \def\@pkgextension{sty}%
      \def\NeedsTeXFormat##1[##2]{}%
      \noNiceVerb                       %% 2013/03/22
      \input{#1}%
    }}
  \@onlypreamble\MDfinaldatechecks
\makeatother
%% Use other packages:
\RequirePackage{niceverb}[2011/01/24] 
\RequirePackage{readprov}               %% 2010/12/08
\RequirePackage{hypertoc}               %% 2011/01/23
\RequirePackage{texlinks}               %% 2011/01/24
\RequirePackage{relsize}                %% 2011/06/27
\RequirePackage{color}                  %% 2011/08/06
\RequirePackage{lmodern}                %% 2012/10/29
\RequirePackage{filedate}               %% 2012/11/12
\RequirePackage{filesdo}                %% 2013/03/22 
%% \pagebreak[3]
%% Logical markup:\qquad  |\strong{<chars>}|, |\meta{<chars>}|, 
%% |\acro{<chars>}|, |\pkg{<chars>}|, 
%% |\code{<chars>}|, |\file{<chars>}|:{\sloppy\par}
\makeatletter
  \def\do#1#2{\@ifdefinable#1{\let#1#2}}%% 2012/07/13
  \do\strong\textbf \do\file\texttt \do\acro\textsmaller 
  %% <- wrong tests before 2012/07/13
  \do\meta\textit   \do \pkg\textsf \do\code\texttt
  \ifpdf
    \pdfstringdefDisableCommands{%
        \let\acro\textrm 
        \let\file\textrm                            %% 2011/11/09
        \let\code\textrm                            %% 2011/11/20
        \let\pkg \textrm                            %% 2012/03/23
    }
  \fi
  %% TODO 2011/07/22 -> `htlogml.sty'
\makeatother
%% |\qtdcode{<text>}|: 2012/10/24:
    \newcommand*{\qtdcode}[1]{`\code{#1}'} 
%% |\pkgtitle{<package-name>}{<caption>}| 
\newcommand*{\pkgtitle}[2]{%            %% 2012/07/13
    \global\let\pkgtitle\relax
    \pkg{\huge #1}\\---\\#2\thanks{This 
       document describes version 
       \textcolor{blue}{\UseVersionOf{\jobname.sty}} 
       of \textsf{\jobname.sty} as of \UseDateOf{\jobname.sty}.}}
%% TODO: %% |\TODO| bad with `mdoccorr.cfg'
\newcommand*{\TODO}{\textcolor{blue}{\acro{TODO}}}  %% 2012/11/06
%% `\MDsampleinput[{<file>}' was added 2012/11/06. 
%% Problems with `myfilist.tex' were due to 'parskip.sty'
%% there. On 2012/11/12, we change the former simple macro to a 
%% much more complex
%% |\MDsamplecodeinput[<add-hfuss>]{<file>}| 
\newcommand*{\MDsamplecodeinput}[2][]{%
    \begingroup
        \vskip\bigskipamount \hrule
        \nobreak\vskip-\parskip 
%         \nobreak\vskip\medskipamount
%% Previous mistake (same below) due to manual change 
%% of `\topsep' in the file `myfilist.tex' (2012/11/30).
        \ifx\\#1\\\else
            \hfuzz=\textwidth \advance\hfuzz#1\relax
        \fi
        \noNiceVerb \verbatiminput{#2}%
%         \nobreak\vskip\medskipamount 
        \hrule \vskip-\parskip 
        \bigskip %%% \bigbreak
%% `\bigbreak' made much larger space in `myfilist.tex'.
    \endgroup
}
%% |\ctanpkgdref{<pkg-id>}| adds the printed link to 
%% `ctan.org/pkg' as a footnote. There is a little space 
%% for coloured link borders:
\newcommand*{\ctanpkgdref}[1]{%
    \ctanpkgref{#1}\,\urlfoot{CtanPkgRef}{#1}}
\errorcontextlines=4
\pagestyle{headings}

\endinput 
 %% shared formatting settings
% \ReadPackageInfos{langcode}
% \usepackage{langcode,catchdq,ngerman} \originalTeX
\usepackage{catchdq,langcode,ngerman} \originalTeX
%  \show\endqtd
\MDkeywords{languages other than English; German, macro programming 
            (programming structures), hypertext}
\sloppy
% \listfiles
\begin{document}
\maketitle
\begin{MDabstract}
'langcode.sty' in the first instance provides a command 
$$|\uselangcode{<chars>}|$$ to adjust language-dependent settings, 
such as key words, typographical conventions, and language codes 
(\acro{\Wikiref{ISO-639-1}}).
% it is intended to be a kind of ``leight-weight" \ctanpkgref{babel}. 
% It uses \ctanpkgref{dowith} for adjustments and 
% \ctanpkgref{plainpkg} for use with both \LaTeX\ and Plain \TeX.
An author frequently writing documents in two or more languages 
can use the same commands independently of the language, 
provided they are gathered in a list macro to be 
used by the \ctanpkgref{dowith} package.
If `\<cmd>' is in the list, it is set to work like 
`\<chars><cmd>', and a macro `\langcode' will expand to 
<chars> (the respective tokens), usable in \acro{URL}s.---The 
package is ``generic," based on \ctanpkgref{plainpkg}.
The code has been used with \ctanpkgref{morehype} and 
'catchdq' (\ctanpkgref{catcodes}), but may be useful more generally.
\MDaddtoabstract{Related packages} \ctanpkgref{babel}, \ctanpkgref{polyglossia}
% \ctanpkgref{morehype}, 'catchdq' (\ctanpkgref{catcodes}), 
% \ctanpkgref{dowith} 
\end{MDabstract}
\newpage
\tableofcontents
% \newpage
% \section{Features and Usage}
\section{Installing and Calling}
The file 'langcode.sty' is provided ready, installation only requires
putting it somewhere where \TeX\ finds it
(which may need updating the filename data
 base).\urlfoot{ukfaqref}{inst-wlcf}           %% corr. 2011/02/08
The packages \ctanpkgref{dowith}, \ctanpkgref{plainpkg}, 
and 'stacklet' (\ctanpkgref{catcodes}) must be installed as well.

As to calling (loading): 'langcode' is a ``\pkg{plainpkg} package" 
in the sense of the 
\ctanpkgref{plainpkg}\,\foothttpurlref{ctan.org/pkg/plainpkg} 
documentation that you may consult for details.
So roughly, 
\begin{itemize}
  \item load it by \ |\usepackage{langcode}| \ if you can, 
  \item otherwise by \ |\RequirePackage{langcode}| \\
        (perhaps from within another ``\pkg{plainpkg} package"), 
  \item or by \ |\input langcode.sty| 
  \item or even by \ |                                                     \input plainpkg
\ProvidesPackage{langcode}[2012/09/20 v0.2 language adjustment (UL)]

%% Copyright (C) 2012 Uwe Lueck,
%% http://www.contact-ednotes.sty.de.vu
%% -- author-maintained in the sense of LPPL below --
%%
%% This file can be redistributed and/or modified under
%% the terms of the LaTeX Project Public License; either
%% version 1.3c of the License, or any later version.
%% The latest version of this license is in
%%     http://www.latex-project.org/lppl.txt
%% We did our best to help you, but there is NO WARRANTY.
%%
%% Please report bugs, problems, and suggestions via
%%
%%   http://www.contact-ednotes.sty.de.vu
%%
%% == Required                 ==
%% 'langcode' is based on \ctanpkgref{dowith}:
\RequirePackage{dowith}
%% 
%% == General Commands         ==
%% |\uselangcode{<lcode>}| \ sets `\langcode' to <lcode>, runs 
%% `\langcodeadjust' on the items stored in `\langcodedependent', 
%% and finally executes what is stored in `\langcodeextras':
\def\uselangcode#1{%
    \def\langcode{#1}%                                  %% 2012/01/07:
    \DoWithAllIn\langcodeadjust\langcodedependent
    \langcodeextras} 
%% |\langcodeadjust\<letters>| \ defines `\<letters>' to expand to 
%% `\<lcode><letters>':
\def\langcodeadjust#1{%
%     \edef#1{\expandafter\noexpand
%                 \csname\langcode
    %% <- 2012/09/17 ->
    \edef#1{\withcsname\noexpand \langcode
                    \expandafter\@gobble\string#1\endcsname}}
%% 
%% == Sample Settings          == 
%% Such settings once should be in some \file{.cfg} file TODO
%% === What Must Be Varied    ===
%% |\langcodedependent| \ is a list of commands that 
%% must be adjusted:                                    %% ready 2012/09/20
\def\langcodedependent{%                                %% 2012/01/07
    \langcodeextras\dqtd\qtd\pardash\lastrev\totopofpage
    \monthname}                                         %% 2012/01/17
%%
%% === English vs. German     ===
%% ==== Months ====
%% |\enmonthname{<num>}| yields the English name of the <num>th month 
%% of the year:
\def\enmonthname#1{%
    \ifcase #1\or
        January\or February\or March\or    April\or
            May\or June\or     July\or     August\or
      September\or October\or  November\or December%
    \fi}
%% |\demonthname{<num>}| yields the \emph{German} name of the <num>th month 
%% month of the year:
\def\demonthname#1{%
    \ifcase #1\or
         Januar\or Februar\or  M\"arz\or   April\or
            Mai\or Juni\or     Juli\or     August\or
      September\or Oktober\or  November\or Dezember%
    \fi}
%% ==== Quotes ====
%% For |\dqtd| settings, see (load)                     %% load 2012/09/20
%% 'catchdq' (\ctanpkgref{catcodes} bundle).\footnote{% %%      2012/09/20
%%   The test below required loading 'catchdq' \emph{earlier} 
%%   than 'langcode'.} 
%% % while |\qtd| settings are in 'blog.sty' (\ctanpkgref{morehype} bundle).
%% |\enqtd{<en-text>}| and |\deqtd{<de-text>}| 
%% (overridable, e.g., with 'blog.sty'):
\ifx\deqtd\undefined \def\deqtd#1{\glq#1\grq} \fi
\ifx\enqtd\undefined \def\enqtd#1{`#1'}       \fi
%% You may get `\glq' and `\grq' from \CtanPkgRef{ngerman}{ngerman.sty}
%% \[(`\usepackage{ngerman}\originalTeX'\]
%% if needed. Here is a little 
%% \dqtd{tessst \qtd{inner}}, 
%% due to the default |\uselangcode{en}|, we now issue |\uselangcode{de}|
%% \let\delangcodeextras\empty
%% \uselangcode{de}to get
%% \dqtd{tessst \qtd{inner}}. 
%% ==== Dashes ====
%% I introduced |\pardash| thinking of 
%% (German \dqtd{Gedankenstrich} and)
%% long dashes as a weak version of a paragraph break. 
%% \uselangcode{en}
%% A paragraph break somehow means moving from one thought 
%% to another\pardash almost the same what a long dash may 
%% mean.\pardash Here I have switched to \qtd{`en'} 
%% again, in order to get |\enpardash| by |\pardash|:
\ifx\enpardash\undefined \let\enpardash\textemdash \fi
%% \uselangcode{de}
%% The \dqtd{Gedankenstrich} |\depardash| is not as long 
%% \uselangcode{en}
%% as the \dqtd{thought dash}\uselangcode{de}\pardash er ist nur 
%% halb so lang\uselangcode{en}\pardash but it is surrounded 
%% by regular spaces:
\ifx\depardash\undefined \def\depardash{ \textendash\space} \fi
%% Some people prefer a so-called \dqtd{\Wikiref{hair space}}
%% surrounding the long dash. 
%% ==== Wikipedia ====
%% The previous Wikipedia link was obtained by |\Wikiref{hair space}|, 
%% working like |\Wikienref{hair space}|. This is an example of how 
%% \ctanpkgref{texlinks} makes use of |\uselangcode{<two-chars>}|.
%% ==== 'blog.sty' ====
%% In the present sample of \ |\langcodedependent|, \ 
%% \[|\lastrev|\quad\mbox{and}\quad|\totopofpage|\] remain. 
%% I use them for \acro{HTML} with 'blog.sty'\pardash sorry, 
%% at present I cannot afford separating 
%% settings for a wider audience from my own ones.
%% 
%% === Other Settings         ===
%% I haven't used English extras so far:
\let\enlangcodeextras\empty                      %% \empty 2012/01/17
%% With German, I have used the \ctanpkgref{dhua} package 
%% for certain abbrevations:
\def\delangcodeextras{\RequirePackage{dhua}}
%% However, this setting disables `\uselangcode{de}' after 
%% `\begin{document}'---which has not a problem with 'blog.sty', 
%% where I use it daily. For the tests above, 
%% I \emph{emptied} `\delangcodeextras'.
%% I had not thought of changing the language \emph{within} 
%% one document before.
%% 
%% == Default Language Code ==
%% The default `\langcode' is \qtd{`en'} for English:
\uselangcode{en}
%% == Leaving the Package File ==
\endinput
%%
%% == VERSION HISTORY ==

v0.1   2012/01/07   in `texblog.fdf'
       2012/01/17
v0.2   2012/09/17   own plainpkg package (\newcommand -> \def, ...) 
       2012/09/20   doc. much expanded; 
                    more blog-independent settings; 
                    \Provides: v0.2, caption shortened (tld -> lc ->)
|~\dots
\end{itemize}

%   \pagebreak
% \section{The Package File}
\section{Header---\pkg{plainpkg} and Legalese}
On the right hand side, that `plainpkg.tex' is loaded, 
before the package version is declared, for ``generic" function:
\ProvidesFile{langcode.tex}[2012/09/20 documenting langcode.sty]
\title{\pkgtitle{langcode.sty}{%
       Simple Language-Dependent Settings\\
       Based on Language Codes}} 
{ \RequirePackage{makedoc} \ProcessLineMessage{}
  \MakeJobDoc{17}%% 2011/11/23
  {\SectionLevelThreeParseInput}  }             %% 2012/09/17
\documentclass[fleqn]{article}
% \usepackage{inputtrc} \dotracinginputs
\ProvidesFile{makedoc.cfg}[{2013/03/25 documentation settings}] 
%%
\author{Uwe L\"uck\thanks{%
        \url{http://contact-ednotes.sty.de.vu}}}
%%
%% 'hyperref':
\RequirePackage{ifpdf}
\usepackage[%
  \ifpdf
%     bookmarks=false,                  %% 2010/12/22
%     bookmarksnumbered,
    bookmarksopen,                      %% 2011/01/24!?
    bookmarksopenlevel=2,               %% 2011/01/23
%     pdfpagemode=UseNone,
%     pdfstartpage=10,
    pdfstartview=FitH,                  %% 2012/11/26 again
%     pdfstartview=0 0 100,             %% 2011/08/22
%     pdfstartview={XYZ null null 1},   %% 2011/08/25
%     pdfstartview={XYZ null null null},%% 2011/08/25
%     pdfstartview={XYZ null null .5},    %% 2011/08/26
%     pdffitwindow=true,          %% 2011/08/22
    citebordercolor={ .6 1    .6},
    filebordercolor={1    .6 1},
    linkbordercolor={1    .9  .7},
     urlbordercolor={ .7 1   1},   %% playing 2011/01/24
  \else
    draft
  \fi
]{hyperref}
\hypersetup{% 
    pdfauthor={Uwe L\374ck}% 
}
%% metadata, |\MDkeywords{<text>}|, |\MDkeywordsstring|:
%% %% 2011/08/22:
\makeatletter
  \newcommand*{\MDkeywords}[1]{%
    \gdef\MDkeywordsstring{#1}%
    \hypersetup{pdfkeywords=\MDkeywordsstring}%% TODO!?
  }
  \@onlypreamble\MDkeywords
%% |\MDaddtoabstract{<par-head>}|, `:' added:
  \newcommand*{\MDaddtoabstract}[1]{%           %% 2012/05/10
    \par\smallskip\noindent
    \strong{#1:}\quad\ignorespaces}
%% \pagebreak[2]
%% |\printMDkeywords|:
  \newcommand*{\printMDkeywords}{%
    \MDaddtoabstract{Keywords}%
    \MDkeywordsstring 
%     \global\let\MDkeywordsstring\relax    %% `%' 2012/11/12
  }
%% The previous definitions mainly are useful with a variant 
%% |\begin{MDabstract}| of \LaTeX's `{abstract}' environment:
  \newenvironment{MDabstract}
                 {\abstract\noindent
                  \hspace{1sp}%% for niceverb
                  \ignorespaces}
                 {\@ifundefined{MDkeywordsstring}%
                               {}%
                               {\printMDkeywords}%
                  \global\let\MDabstract\relax    %% 2012/11/12
                  \global\let\endMDabstract\relax %% 2012/11/12
                  \endabstract}
%% |\[MD]docnewline| 2012/11/12 from `readprov.tex':
  \newcommand*{\MDdocnewline}{\leavevmode\@normalcr[\topsep]}
%% <- `\leavevmode' for use with `\paragraph'.
%%    Sometimes needs to be preceded by a space.
%% 
%% |\MDfinaldatechecks[<tex-script>]| with \ctanpkgref{filedate}:
  \newcommand*{\MDfinaldatechecks}[1][fdatechk]{%
    \AtEndDocument{%
%       \clearpage %% 2013/03/25 no avail -- with `filedate'!
      \def\@pkgextension{sty}%
      \def\NeedsTeXFormat##1[##2]{}%
      \noNiceVerb                       %% 2013/03/22
      \input{#1}%
    }}
  \@onlypreamble\MDfinaldatechecks
\makeatother
%% Use other packages:
\RequirePackage{niceverb}[2011/01/24] 
\RequirePackage{readprov}               %% 2010/12/08
\RequirePackage{hypertoc}               %% 2011/01/23
\RequirePackage{texlinks}               %% 2011/01/24
\RequirePackage{relsize}                %% 2011/06/27
\RequirePackage{color}                  %% 2011/08/06
\RequirePackage{lmodern}                %% 2012/10/29
\RequirePackage{filedate}               %% 2012/11/12
\RequirePackage{filesdo}                %% 2013/03/22 
%% \pagebreak[3]
%% Logical markup:\qquad  |\strong{<chars>}|, |\meta{<chars>}|, 
%% |\acro{<chars>}|, |\pkg{<chars>}|, 
%% |\code{<chars>}|, |\file{<chars>}|:{\sloppy\par}
\makeatletter
  \def\do#1#2{\@ifdefinable#1{\let#1#2}}%% 2012/07/13
  \do\strong\textbf \do\file\texttt \do\acro\textsmaller 
  %% <- wrong tests before 2012/07/13
  \do\meta\textit   \do \pkg\textsf \do\code\texttt
  \ifpdf
    \pdfstringdefDisableCommands{%
        \let\acro\textrm 
        \let\file\textrm                            %% 2011/11/09
        \let\code\textrm                            %% 2011/11/20
        \let\pkg \textrm                            %% 2012/03/23
    }
  \fi
  %% TODO 2011/07/22 -> `htlogml.sty'
\makeatother
%% |\qtdcode{<text>}|: 2012/10/24:
    \newcommand*{\qtdcode}[1]{`\code{#1}'} 
%% |\pkgtitle{<package-name>}{<caption>}| 
\newcommand*{\pkgtitle}[2]{%            %% 2012/07/13
    \global\let\pkgtitle\relax
    \pkg{\huge #1}\\---\\#2\thanks{This 
       document describes version 
       \textcolor{blue}{\UseVersionOf{\jobname.sty}} 
       of \textsf{\jobname.sty} as of \UseDateOf{\jobname.sty}.}}
%% TODO: %% |\TODO| bad with `mdoccorr.cfg'
\newcommand*{\TODO}{\textcolor{blue}{\acro{TODO}}}  %% 2012/11/06
%% `\MDsampleinput[{<file>}' was added 2012/11/06. 
%% Problems with `myfilist.tex' were due to 'parskip.sty'
%% there. On 2012/11/12, we change the former simple macro to a 
%% much more complex
%% |\MDsamplecodeinput[<add-hfuss>]{<file>}| 
\newcommand*{\MDsamplecodeinput}[2][]{%
    \begingroup
        \vskip\bigskipamount \hrule
        \nobreak\vskip-\parskip 
%         \nobreak\vskip\medskipamount
%% Previous mistake (same below) due to manual change 
%% of `\topsep' in the file `myfilist.tex' (2012/11/30).
        \ifx\\#1\\\else
            \hfuzz=\textwidth \advance\hfuzz#1\relax
        \fi
        \noNiceVerb \verbatiminput{#2}%
%         \nobreak\vskip\medskipamount 
        \hrule \vskip-\parskip 
        \bigskip %%% \bigbreak
%% `\bigbreak' made much larger space in `myfilist.tex'.
    \endgroup
}
%% |\ctanpkgdref{<pkg-id>}| adds the printed link to 
%% `ctan.org/pkg' as a footnote. There is a little space 
%% for coloured link borders:
\newcommand*{\ctanpkgdref}[1]{%
    \ctanpkgref{#1}\,\urlfoot{CtanPkgRef}{#1}}
\errorcontextlines=4
\pagestyle{headings}

\endinput 
 %% shared formatting settings
% \ReadPackageInfos{langcode}
% \usepackage{langcode,catchdq,ngerman} \originalTeX
\usepackage{catchdq,langcode,ngerman} \originalTeX
%  \show\endqtd
\MDkeywords{languages other than English; German, macro programming 
            (programming structures), hypertext}
\sloppy
% \listfiles
\begin{document}
\maketitle
\begin{MDabstract}
'langcode.sty' in the first instance provides a command 
$$|\uselangcode{<chars>}|$$ to adjust language-dependent settings, 
such as key words, typographical conventions, and language codes 
(\acro{\Wikiref{ISO-639-1}}).
% it is intended to be a kind of ``leight-weight" \ctanpkgref{babel}. 
% It uses \ctanpkgref{dowith} for adjustments and 
% \ctanpkgref{plainpkg} for use with both \LaTeX\ and Plain \TeX.
An author frequently writing documents in two or more languages 
can use the same commands independently of the language, 
provided they are gathered in a list macro to be 
used by the \ctanpkgref{dowith} package.
If `\<cmd>' is in the list, it is set to work like 
`\<chars><cmd>', and a macro `\langcode' will expand to 
<chars> (the respective tokens), usable in \acro{URL}s.---The 
package is ``generic," based on \ctanpkgref{plainpkg}.
The code has been used with \ctanpkgref{morehype} and 
'catchdq' (\ctanpkgref{catcodes}), but may be useful more generally.
\MDaddtoabstract{Related packages} \ctanpkgref{babel}, \ctanpkgref{polyglossia}
% \ctanpkgref{morehype}, 'catchdq' (\ctanpkgref{catcodes}), 
% \ctanpkgref{dowith} 
\end{MDabstract}
\newpage
\tableofcontents
% \newpage
% \section{Features and Usage}
\section{Installing and Calling}
The file 'langcode.sty' is provided ready, installation only requires
putting it somewhere where \TeX\ finds it
(which may need updating the filename data
 base).\urlfoot{ukfaqref}{inst-wlcf}           %% corr. 2011/02/08
The packages \ctanpkgref{dowith}, \ctanpkgref{plainpkg}, 
and 'stacklet' (\ctanpkgref{catcodes}) must be installed as well.

As to calling (loading): 'langcode' is a ``\pkg{plainpkg} package" 
in the sense of the 
\ctanpkgref{plainpkg}\,\foothttpurlref{ctan.org/pkg/plainpkg} 
documentation that you may consult for details.
So roughly, 
\begin{itemize}
  \item load it by \ |\usepackage{langcode}| \ if you can, 
  \item otherwise by \ |\RequirePackage{langcode}| \\
        (perhaps from within another ``\pkg{plainpkg} package"), 
  \item or by \ |\input langcode.sty| 
  \item or even by \ |                                                     \input plainpkg
\ProvidesPackage{langcode}[2012/09/20 v0.2 language adjustment (UL)]

%% Copyright (C) 2012 Uwe Lueck,
%% http://www.contact-ednotes.sty.de.vu
%% -- author-maintained in the sense of LPPL below --
%%
%% This file can be redistributed and/or modified under
%% the terms of the LaTeX Project Public License; either
%% version 1.3c of the License, or any later version.
%% The latest version of this license is in
%%     http://www.latex-project.org/lppl.txt
%% We did our best to help you, but there is NO WARRANTY.
%%
%% Please report bugs, problems, and suggestions via
%%
%%   http://www.contact-ednotes.sty.de.vu
%%
%% == Required                 ==
%% 'langcode' is based on \ctanpkgref{dowith}:
\RequirePackage{dowith}
%% 
%% == General Commands         ==
%% |\uselangcode{<lcode>}| \ sets `\langcode' to <lcode>, runs 
%% `\langcodeadjust' on the items stored in `\langcodedependent', 
%% and finally executes what is stored in `\langcodeextras':
\def\uselangcode#1{%
    \def\langcode{#1}%                                  %% 2012/01/07:
    \DoWithAllIn\langcodeadjust\langcodedependent
    \langcodeextras} 
%% |\langcodeadjust\<letters>| \ defines `\<letters>' to expand to 
%% `\<lcode><letters>':
\def\langcodeadjust#1{%
%     \edef#1{\expandafter\noexpand
%                 \csname\langcode
    %% <- 2012/09/17 ->
    \edef#1{\withcsname\noexpand \langcode
                    \expandafter\@gobble\string#1\endcsname}}
%% 
%% == Sample Settings          == 
%% Such settings once should be in some \file{.cfg} file TODO
%% === What Must Be Varied    ===
%% |\langcodedependent| \ is a list of commands that 
%% must be adjusted:                                    %% ready 2012/09/20
\def\langcodedependent{%                                %% 2012/01/07
    \langcodeextras\dqtd\qtd\pardash\lastrev\totopofpage
    \monthname}                                         %% 2012/01/17
%%
%% === English vs. German     ===
%% ==== Months ====
%% |\enmonthname{<num>}| yields the English name of the <num>th month 
%% of the year:
\def\enmonthname#1{%
    \ifcase #1\or
        January\or February\or March\or    April\or
            May\or June\or     July\or     August\or
      September\or October\or  November\or December%
    \fi}
%% |\demonthname{<num>}| yields the \emph{German} name of the <num>th month 
%% month of the year:
\def\demonthname#1{%
    \ifcase #1\or
         Januar\or Februar\or  M\"arz\or   April\or
            Mai\or Juni\or     Juli\or     August\or
      September\or Oktober\or  November\or Dezember%
    \fi}
%% ==== Quotes ====
%% For |\dqtd| settings, see (load)                     %% load 2012/09/20
%% 'catchdq' (\ctanpkgref{catcodes} bundle).\footnote{% %%      2012/09/20
%%   The test below required loading 'catchdq' \emph{earlier} 
%%   than 'langcode'.} 
%% % while |\qtd| settings are in 'blog.sty' (\ctanpkgref{morehype} bundle).
%% |\enqtd{<en-text>}| and |\deqtd{<de-text>}| 
%% (overridable, e.g., with 'blog.sty'):
\ifx\deqtd\undefined \def\deqtd#1{\glq#1\grq} \fi
\ifx\enqtd\undefined \def\enqtd#1{`#1'}       \fi
%% You may get `\glq' and `\grq' from \CtanPkgRef{ngerman}{ngerman.sty}
%% \[(`\usepackage{ngerman}\originalTeX'\]
%% if needed. Here is a little 
%% \dqtd{tessst \qtd{inner}}, 
%% due to the default |\uselangcode{en}|, we now issue |\uselangcode{de}|
%% \let\delangcodeextras\empty
%% \uselangcode{de}to get
%% \dqtd{tessst \qtd{inner}}. 
%% ==== Dashes ====
%% I introduced |\pardash| thinking of 
%% (German \dqtd{Gedankenstrich} and)
%% long dashes as a weak version of a paragraph break. 
%% \uselangcode{en}
%% A paragraph break somehow means moving from one thought 
%% to another\pardash almost the same what a long dash may 
%% mean.\pardash Here I have switched to \qtd{`en'} 
%% again, in order to get |\enpardash| by |\pardash|:
\ifx\enpardash\undefined \let\enpardash\textemdash \fi
%% \uselangcode{de}
%% The \dqtd{Gedankenstrich} |\depardash| is not as long 
%% \uselangcode{en}
%% as the \dqtd{thought dash}\uselangcode{de}\pardash er ist nur 
%% halb so lang\uselangcode{en}\pardash but it is surrounded 
%% by regular spaces:
\ifx\depardash\undefined \def\depardash{ \textendash\space} \fi
%% Some people prefer a so-called \dqtd{\Wikiref{hair space}}
%% surrounding the long dash. 
%% ==== Wikipedia ====
%% The previous Wikipedia link was obtained by |\Wikiref{hair space}|, 
%% working like |\Wikienref{hair space}|. This is an example of how 
%% \ctanpkgref{texlinks} makes use of |\uselangcode{<two-chars>}|.
%% ==== 'blog.sty' ====
%% In the present sample of \ |\langcodedependent|, \ 
%% \[|\lastrev|\quad\mbox{and}\quad|\totopofpage|\] remain. 
%% I use them for \acro{HTML} with 'blog.sty'\pardash sorry, 
%% at present I cannot afford separating 
%% settings for a wider audience from my own ones.
%% 
%% === Other Settings         ===
%% I haven't used English extras so far:
\let\enlangcodeextras\empty                      %% \empty 2012/01/17
%% With German, I have used the \ctanpkgref{dhua} package 
%% for certain abbrevations:
\def\delangcodeextras{\RequirePackage{dhua}}
%% However, this setting disables `\uselangcode{de}' after 
%% `\begin{document}'---which has not a problem with 'blog.sty', 
%% where I use it daily. For the tests above, 
%% I \emph{emptied} `\delangcodeextras'.
%% I had not thought of changing the language \emph{within} 
%% one document before.
%% 
%% == Default Language Code ==
%% The default `\langcode' is \qtd{`en'} for English:
\uselangcode{en}
%% == Leaving the Package File ==
\endinput
%%
%% == VERSION HISTORY ==

v0.1   2012/01/07   in `texblog.fdf'
       2012/01/17
v0.2   2012/09/17   own plainpkg package (\newcommand -> \def, ...) 
       2012/09/20   doc. much expanded; 
                    more blog-independent settings; 
                    \Provides: v0.2, caption shortened (tld -> lc ->)
|~\dots
\end{itemize}

%   \pagebreak
% \section{The Package File}
\section{Header---\pkg{plainpkg} and Legalese}
On the right hand side, that `plainpkg.tex' is loaded, 
before the package version is declared, for ``generic" function:
\ProvidesFile{langcode.tex}[2012/09/20 documenting langcode.sty]
\title{\pkgtitle{langcode.sty}{%
       Simple Language-Dependent Settings\\
       Based on Language Codes}} 
{ \RequirePackage{makedoc} \ProcessLineMessage{}
  \MakeJobDoc{17}%% 2011/11/23
  {\SectionLevelThreeParseInput}  }             %% 2012/09/17
\documentclass[fleqn]{article}
% \usepackage{inputtrc} \dotracinginputs
\ProvidesFile{makedoc.cfg}[{2013/03/25 documentation settings}] 
%%
\author{Uwe L\"uck\thanks{%
        \url{http://contact-ednotes.sty.de.vu}}}
%%
%% 'hyperref':
\RequirePackage{ifpdf}
\usepackage[%
  \ifpdf
%     bookmarks=false,                  %% 2010/12/22
%     bookmarksnumbered,
    bookmarksopen,                      %% 2011/01/24!?
    bookmarksopenlevel=2,               %% 2011/01/23
%     pdfpagemode=UseNone,
%     pdfstartpage=10,
    pdfstartview=FitH,                  %% 2012/11/26 again
%     pdfstartview=0 0 100,             %% 2011/08/22
%     pdfstartview={XYZ null null 1},   %% 2011/08/25
%     pdfstartview={XYZ null null null},%% 2011/08/25
%     pdfstartview={XYZ null null .5},    %% 2011/08/26
%     pdffitwindow=true,          %% 2011/08/22
    citebordercolor={ .6 1    .6},
    filebordercolor={1    .6 1},
    linkbordercolor={1    .9  .7},
     urlbordercolor={ .7 1   1},   %% playing 2011/01/24
  \else
    draft
  \fi
]{hyperref}
\hypersetup{% 
    pdfauthor={Uwe L\374ck}% 
}
%% metadata, |\MDkeywords{<text>}|, |\MDkeywordsstring|:
%% %% 2011/08/22:
\makeatletter
  \newcommand*{\MDkeywords}[1]{%
    \gdef\MDkeywordsstring{#1}%
    \hypersetup{pdfkeywords=\MDkeywordsstring}%% TODO!?
  }
  \@onlypreamble\MDkeywords
%% |\MDaddtoabstract{<par-head>}|, `:' added:
  \newcommand*{\MDaddtoabstract}[1]{%           %% 2012/05/10
    \par\smallskip\noindent
    \strong{#1:}\quad\ignorespaces}
%% \pagebreak[2]
%% |\printMDkeywords|:
  \newcommand*{\printMDkeywords}{%
    \MDaddtoabstract{Keywords}%
    \MDkeywordsstring 
%     \global\let\MDkeywordsstring\relax    %% `%' 2012/11/12
  }
%% The previous definitions mainly are useful with a variant 
%% |\begin{MDabstract}| of \LaTeX's `{abstract}' environment:
  \newenvironment{MDabstract}
                 {\abstract\noindent
                  \hspace{1sp}%% for niceverb
                  \ignorespaces}
                 {\@ifundefined{MDkeywordsstring}%
                               {}%
                               {\printMDkeywords}%
                  \global\let\MDabstract\relax    %% 2012/11/12
                  \global\let\endMDabstract\relax %% 2012/11/12
                  \endabstract}
%% |\[MD]docnewline| 2012/11/12 from `readprov.tex':
  \newcommand*{\MDdocnewline}{\leavevmode\@normalcr[\topsep]}
%% <- `\leavevmode' for use with `\paragraph'.
%%    Sometimes needs to be preceded by a space.
%% 
%% |\MDfinaldatechecks[<tex-script>]| with \ctanpkgref{filedate}:
  \newcommand*{\MDfinaldatechecks}[1][fdatechk]{%
    \AtEndDocument{%
%       \clearpage %% 2013/03/25 no avail -- with `filedate'!
      \def\@pkgextension{sty}%
      \def\NeedsTeXFormat##1[##2]{}%
      \noNiceVerb                       %% 2013/03/22
      \input{#1}%
    }}
  \@onlypreamble\MDfinaldatechecks
\makeatother
%% Use other packages:
\RequirePackage{niceverb}[2011/01/24] 
\RequirePackage{readprov}               %% 2010/12/08
\RequirePackage{hypertoc}               %% 2011/01/23
\RequirePackage{texlinks}               %% 2011/01/24
\RequirePackage{relsize}                %% 2011/06/27
\RequirePackage{color}                  %% 2011/08/06
\RequirePackage{lmodern}                %% 2012/10/29
\RequirePackage{filedate}               %% 2012/11/12
\RequirePackage{filesdo}                %% 2013/03/22 
%% \pagebreak[3]
%% Logical markup:\qquad  |\strong{<chars>}|, |\meta{<chars>}|, 
%% |\acro{<chars>}|, |\pkg{<chars>}|, 
%% |\code{<chars>}|, |\file{<chars>}|:{\sloppy\par}
\makeatletter
  \def\do#1#2{\@ifdefinable#1{\let#1#2}}%% 2012/07/13
  \do\strong\textbf \do\file\texttt \do\acro\textsmaller 
  %% <- wrong tests before 2012/07/13
  \do\meta\textit   \do \pkg\textsf \do\code\texttt
  \ifpdf
    \pdfstringdefDisableCommands{%
        \let\acro\textrm 
        \let\file\textrm                            %% 2011/11/09
        \let\code\textrm                            %% 2011/11/20
        \let\pkg \textrm                            %% 2012/03/23
    }
  \fi
  %% TODO 2011/07/22 -> `htlogml.sty'
\makeatother
%% |\qtdcode{<text>}|: 2012/10/24:
    \newcommand*{\qtdcode}[1]{`\code{#1}'} 
%% |\pkgtitle{<package-name>}{<caption>}| 
\newcommand*{\pkgtitle}[2]{%            %% 2012/07/13
    \global\let\pkgtitle\relax
    \pkg{\huge #1}\\---\\#2\thanks{This 
       document describes version 
       \textcolor{blue}{\UseVersionOf{\jobname.sty}} 
       of \textsf{\jobname.sty} as of \UseDateOf{\jobname.sty}.}}
%% TODO: %% |\TODO| bad with `mdoccorr.cfg'
\newcommand*{\TODO}{\textcolor{blue}{\acro{TODO}}}  %% 2012/11/06
%% `\MDsampleinput[{<file>}' was added 2012/11/06. 
%% Problems with `myfilist.tex' were due to 'parskip.sty'
%% there. On 2012/11/12, we change the former simple macro to a 
%% much more complex
%% |\MDsamplecodeinput[<add-hfuss>]{<file>}| 
\newcommand*{\MDsamplecodeinput}[2][]{%
    \begingroup
        \vskip\bigskipamount \hrule
        \nobreak\vskip-\parskip 
%         \nobreak\vskip\medskipamount
%% Previous mistake (same below) due to manual change 
%% of `\topsep' in the file `myfilist.tex' (2012/11/30).
        \ifx\\#1\\\else
            \hfuzz=\textwidth \advance\hfuzz#1\relax
        \fi
        \noNiceVerb \verbatiminput{#2}%
%         \nobreak\vskip\medskipamount 
        \hrule \vskip-\parskip 
        \bigskip %%% \bigbreak
%% `\bigbreak' made much larger space in `myfilist.tex'.
    \endgroup
}
%% |\ctanpkgdref{<pkg-id>}| adds the printed link to 
%% `ctan.org/pkg' as a footnote. There is a little space 
%% for coloured link borders:
\newcommand*{\ctanpkgdref}[1]{%
    \ctanpkgref{#1}\,\urlfoot{CtanPkgRef}{#1}}
\errorcontextlines=4
\pagestyle{headings}

\endinput 
 %% shared formatting settings
% \ReadPackageInfos{langcode}
% \usepackage{langcode,catchdq,ngerman} \originalTeX
\usepackage{catchdq,langcode,ngerman} \originalTeX
%  \show\endqtd
\MDkeywords{languages other than English; German, macro programming 
            (programming structures), hypertext}
\sloppy
% \listfiles
\begin{document}
\maketitle
\begin{MDabstract}
'langcode.sty' in the first instance provides a command 
$$|\uselangcode{<chars>}|$$ to adjust language-dependent settings, 
such as key words, typographical conventions, and language codes 
(\acro{\Wikiref{ISO-639-1}}).
% it is intended to be a kind of ``leight-weight" \ctanpkgref{babel}. 
% It uses \ctanpkgref{dowith} for adjustments and 
% \ctanpkgref{plainpkg} for use with both \LaTeX\ and Plain \TeX.
An author frequently writing documents in two or more languages 
can use the same commands independently of the language, 
provided they are gathered in a list macro to be 
used by the \ctanpkgref{dowith} package.
If `\<cmd>' is in the list, it is set to work like 
`\<chars><cmd>', and a macro `\langcode' will expand to 
<chars> (the respective tokens), usable in \acro{URL}s.---The 
package is ``generic," based on \ctanpkgref{plainpkg}.
The code has been used with \ctanpkgref{morehype} and 
'catchdq' (\ctanpkgref{catcodes}), but may be useful more generally.
\MDaddtoabstract{Related packages} \ctanpkgref{babel}, \ctanpkgref{polyglossia}
% \ctanpkgref{morehype}, 'catchdq' (\ctanpkgref{catcodes}), 
% \ctanpkgref{dowith} 
\end{MDabstract}
\newpage
\tableofcontents
% \newpage
% \section{Features and Usage}
\section{Installing and Calling}
The file 'langcode.sty' is provided ready, installation only requires
putting it somewhere where \TeX\ finds it
(which may need updating the filename data
 base).\urlfoot{ukfaqref}{inst-wlcf}           %% corr. 2011/02/08
The packages \ctanpkgref{dowith}, \ctanpkgref{plainpkg}, 
and 'stacklet' (\ctanpkgref{catcodes}) must be installed as well.

As to calling (loading): 'langcode' is a ``\pkg{plainpkg} package" 
in the sense of the 
\ctanpkgref{plainpkg}\,\foothttpurlref{ctan.org/pkg/plainpkg} 
documentation that you may consult for details.
So roughly, 
\begin{itemize}
  \item load it by \ |\usepackage{langcode}| \ if you can, 
  \item otherwise by \ |\RequirePackage{langcode}| \\
        (perhaps from within another ``\pkg{plainpkg} package"), 
  \item or by \ |\input langcode.sty| 
  \item or even by \ |                                                     \input plainpkg
\ProvidesPackage{langcode}[2012/09/20 v0.2 language adjustment (UL)]

%% Copyright (C) 2012 Uwe Lueck,
%% http://www.contact-ednotes.sty.de.vu
%% -- author-maintained in the sense of LPPL below --
%%
%% This file can be redistributed and/or modified under
%% the terms of the LaTeX Project Public License; either
%% version 1.3c of the License, or any later version.
%% The latest version of this license is in
%%     http://www.latex-project.org/lppl.txt
%% We did our best to help you, but there is NO WARRANTY.
%%
%% Please report bugs, problems, and suggestions via
%%
%%   http://www.contact-ednotes.sty.de.vu
%%
%% == Required                 ==
%% 'langcode' is based on \ctanpkgref{dowith}:
\RequirePackage{dowith}
%% 
%% == General Commands         ==
%% |\uselangcode{<lcode>}| \ sets `\langcode' to <lcode>, runs 
%% `\langcodeadjust' on the items stored in `\langcodedependent', 
%% and finally executes what is stored in `\langcodeextras':
\def\uselangcode#1{%
    \def\langcode{#1}%                                  %% 2012/01/07:
    \DoWithAllIn\langcodeadjust\langcodedependent
    \langcodeextras} 
%% |\langcodeadjust\<letters>| \ defines `\<letters>' to expand to 
%% `\<lcode><letters>':
\def\langcodeadjust#1{%
%     \edef#1{\expandafter\noexpand
%                 \csname\langcode
    %% <- 2012/09/17 ->
    \edef#1{\withcsname\noexpand \langcode
                    \expandafter\@gobble\string#1\endcsname}}
%% 
%% == Sample Settings          == 
%% Such settings once should be in some \file{.cfg} file TODO
%% === What Must Be Varied    ===
%% |\langcodedependent| \ is a list of commands that 
%% must be adjusted:                                    %% ready 2012/09/20
\def\langcodedependent{%                                %% 2012/01/07
    \langcodeextras\dqtd\qtd\pardash\lastrev\totopofpage
    \monthname}                                         %% 2012/01/17
%%
%% === English vs. German     ===
%% ==== Months ====
%% |\enmonthname{<num>}| yields the English name of the <num>th month 
%% of the year:
\def\enmonthname#1{%
    \ifcase #1\or
        January\or February\or March\or    April\or
            May\or June\or     July\or     August\or
      September\or October\or  November\or December%
    \fi}
%% |\demonthname{<num>}| yields the \emph{German} name of the <num>th month 
%% month of the year:
\def\demonthname#1{%
    \ifcase #1\or
         Januar\or Februar\or  M\"arz\or   April\or
            Mai\or Juni\or     Juli\or     August\or
      September\or Oktober\or  November\or Dezember%
    \fi}
%% ==== Quotes ====
%% For |\dqtd| settings, see (load)                     %% load 2012/09/20
%% 'catchdq' (\ctanpkgref{catcodes} bundle).\footnote{% %%      2012/09/20
%%   The test below required loading 'catchdq' \emph{earlier} 
%%   than 'langcode'.} 
%% % while |\qtd| settings are in 'blog.sty' (\ctanpkgref{morehype} bundle).
%% |\enqtd{<en-text>}| and |\deqtd{<de-text>}| 
%% (overridable, e.g., with 'blog.sty'):
\ifx\deqtd\undefined \def\deqtd#1{\glq#1\grq} \fi
\ifx\enqtd\undefined \def\enqtd#1{`#1'}       \fi
%% You may get `\glq' and `\grq' from \CtanPkgRef{ngerman}{ngerman.sty}
%% \[(`\usepackage{ngerman}\originalTeX'\]
%% if needed. Here is a little 
%% \dqtd{tessst \qtd{inner}}, 
%% due to the default |\uselangcode{en}|, we now issue |\uselangcode{de}|
%% \let\delangcodeextras\empty
%% \uselangcode{de}to get
%% \dqtd{tessst \qtd{inner}}. 
%% ==== Dashes ====
%% I introduced |\pardash| thinking of 
%% (German \dqtd{Gedankenstrich} and)
%% long dashes as a weak version of a paragraph break. 
%% \uselangcode{en}
%% A paragraph break somehow means moving from one thought 
%% to another\pardash almost the same what a long dash may 
%% mean.\pardash Here I have switched to \qtd{`en'} 
%% again, in order to get |\enpardash| by |\pardash|:
\ifx\enpardash\undefined \let\enpardash\textemdash \fi
%% \uselangcode{de}
%% The \dqtd{Gedankenstrich} |\depardash| is not as long 
%% \uselangcode{en}
%% as the \dqtd{thought dash}\uselangcode{de}\pardash er ist nur 
%% halb so lang\uselangcode{en}\pardash but it is surrounded 
%% by regular spaces:
\ifx\depardash\undefined \def\depardash{ \textendash\space} \fi
%% Some people prefer a so-called \dqtd{\Wikiref{hair space}}
%% surrounding the long dash. 
%% ==== Wikipedia ====
%% The previous Wikipedia link was obtained by |\Wikiref{hair space}|, 
%% working like |\Wikienref{hair space}|. This is an example of how 
%% \ctanpkgref{texlinks} makes use of |\uselangcode{<two-chars>}|.
%% ==== 'blog.sty' ====
%% In the present sample of \ |\langcodedependent|, \ 
%% \[|\lastrev|\quad\mbox{and}\quad|\totopofpage|\] remain. 
%% I use them for \acro{HTML} with 'blog.sty'\pardash sorry, 
%% at present I cannot afford separating 
%% settings for a wider audience from my own ones.
%% 
%% === Other Settings         ===
%% I haven't used English extras so far:
\let\enlangcodeextras\empty                      %% \empty 2012/01/17
%% With German, I have used the \ctanpkgref{dhua} package 
%% for certain abbrevations:
\def\delangcodeextras{\RequirePackage{dhua}}
%% However, this setting disables `\uselangcode{de}' after 
%% `\begin{document}'---which has not a problem with 'blog.sty', 
%% where I use it daily. For the tests above, 
%% I \emph{emptied} `\delangcodeextras'.
%% I had not thought of changing the language \emph{within} 
%% one document before.
%% 
%% == Default Language Code ==
%% The default `\langcode' is \qtd{`en'} for English:
\uselangcode{en}
%% == Leaving the Package File ==
\endinput
%%
%% == VERSION HISTORY ==

v0.1   2012/01/07   in `texblog.fdf'
       2012/01/17
v0.2   2012/09/17   own plainpkg package (\newcommand -> \def, ...) 
       2012/09/20   doc. much expanded; 
                    more blog-independent settings; 
                    \Provides: v0.2, caption shortened (tld -> lc ->)
|~\dots
\end{itemize}

%   \pagebreak
% \section{The Package File}
\section{Header---\pkg{plainpkg} and Legalese}
On the right hand side, that `plainpkg.tex' is loaded, 
before the package version is declared, for ``generic" function:
\ProvidesFile{langcode.tex}[2012/09/20 documenting langcode.sty]
\title{\pkgtitle{langcode.sty}{%
       Simple Language-Dependent Settings\\
       Based on Language Codes}} 
{ \RequirePackage{makedoc} \ProcessLineMessage{}
  \MakeJobDoc{17}%% 2011/11/23
  {\SectionLevelThreeParseInput}  }             %% 2012/09/17
\documentclass[fleqn]{article}
% \usepackage{inputtrc} \dotracinginputs
\input{makedoc.cfg} %% shared formatting settings
% \ReadPackageInfos{langcode}
% \usepackage{langcode,catchdq,ngerman} \originalTeX
\usepackage{catchdq,langcode,ngerman} \originalTeX
%  \show\endqtd
\MDkeywords{languages other than English; German, macro programming 
            (programming structures), hypertext}
\sloppy
% \listfiles
\begin{document}
\maketitle
\begin{MDabstract}
'langcode.sty' in the first instance provides a command 
$$|\uselangcode{<chars>}|$$ to adjust language-dependent settings, 
such as key words, typographical conventions, and language codes 
(\acro{\Wikiref{ISO-639-1}}).
% it is intended to be a kind of ``leight-weight" \ctanpkgref{babel}. 
% It uses \ctanpkgref{dowith} for adjustments and 
% \ctanpkgref{plainpkg} for use with both \LaTeX\ and Plain \TeX.
An author frequently writing documents in two or more languages 
can use the same commands independently of the language, 
provided they are gathered in a list macro to be 
used by the \ctanpkgref{dowith} package.
If `\<cmd>' is in the list, it is set to work like 
`\<chars><cmd>', and a macro `\langcode' will expand to 
<chars> (the respective tokens), usable in \acro{URL}s.---The 
package is ``generic," based on \ctanpkgref{plainpkg}.
The code has been used with \ctanpkgref{morehype} and 
'catchdq' (\ctanpkgref{catcodes}), but may be useful more generally.
\MDaddtoabstract{Related packages} \ctanpkgref{babel}, \ctanpkgref{polyglossia}
% \ctanpkgref{morehype}, 'catchdq' (\ctanpkgref{catcodes}), 
% \ctanpkgref{dowith} 
\end{MDabstract}
\newpage
\tableofcontents
% \newpage
% \section{Features and Usage}
\section{Installing and Calling}
The file 'langcode.sty' is provided ready, installation only requires
putting it somewhere where \TeX\ finds it
(which may need updating the filename data
 base).\urlfoot{ukfaqref}{inst-wlcf}           %% corr. 2011/02/08
The packages \ctanpkgref{dowith}, \ctanpkgref{plainpkg}, 
and 'stacklet' (\ctanpkgref{catcodes}) must be installed as well.

As to calling (loading): 'langcode' is a ``\pkg{plainpkg} package" 
in the sense of the 
\ctanpkgref{plainpkg}\,\foothttpurlref{ctan.org/pkg/plainpkg} 
documentation that you may consult for details.
So roughly, 
\begin{itemize}
  \item load it by \ |\usepackage{langcode}| \ if you can, 
  \item otherwise by \ |\RequirePackage{langcode}| \\
        (perhaps from within another ``\pkg{plainpkg} package"), 
  \item or by \ |\input langcode.sty| 
  \item or even by \ |\input{langcode.sty}|~\dots
\end{itemize}

%   \pagebreak
% \section{The Package File}
\section{Header---\pkg{plainpkg} and Legalese}
On the right hand side, that `plainpkg.tex' is loaded, 
before the package version is declared, for ``generic" function:
\input{langcode.doc}

\end{document}

VERSION HISTORY

2012/09/17  for v0.1    very first 
2012/09/20              extended ...


\end{document}

VERSION HISTORY

2012/09/17  for v0.1    very first 
2012/09/20              extended ...


\end{document}

VERSION HISTORY

2012/09/17  for v0.1    very first 
2012/09/20              extended ...


\end{document}

VERSION HISTORY

2012/09/17  for v0.1    very first 
2012/09/20              extended ...
